\label{sect:compare}

Hoare/Incorrectness Logic are concerned with the study of the effect of some given code. They answer the question whether some specific code (\prg{cmd}) might lead have a certain effect (going from $A_1$ to $A_2$). Thus they under/under approx the post condition ($A_2$) given a precondition  (here $A_1$) and some code (here \prg{cmd}). 

In contrast, NL is concerned with security properties; in security,  the code being executed might come from any, untrusted, third party,
and therefore is unknown. Therefore, NL is not concerned with a \emph{particular} code; instead, it is concerned with the conditions under which some effect  (going from $A_1$ to $A_2$) , which take place. NL can be used to prove that certain effect will never take place.  
\vspace{.2in}

$\begin{array}{|l|l|}
\hline
 & \\
\mbox{Hoare Logic}  &  
\\
 & \\
\ \  \{\, A_1 \, \}\ \prg{cmd} \  \{\, A_2 \, \} \ \ \ &\  \forall \sigma, \sigma'. [ \ \ \sigma \models A_1 \ \wedge\ \sigma, \prg{cmd} \leadsto^* \sigma' \ \longrightarrow \ \sigma' \models A_2\ \  ]  \\
 & \ \ \ \ \ \  A_2 \mbox{ over--approximates the outcome of executing \prg{cmd} in } A_1  \ \ \ \ \ \ \ \ \\
& \\
\hline 
\hline
 & \\
 \mbox{Incorrectness Logic}  &  \\
 & \\
\ \  [\, A_1 \, ]\ \prg{cmd} \  [\, A_2 \, ] \ \ \ &\  \forall \sigma'. [ \ \ \sigma' \models A_2 \ \ \longrightarrow \ \exists \sigma.( \ \sigma \models A_1 \ \wedge\ \sigma, \prg{cmd} \leadsto^* \sigma' \ ) \ \  ]  \\
 &  \ \ \ \ \ \   A_2 \mbox{ under-approximates the outcome of executing  \prg{cmd} in } A_1  \ \ \ \ \ \ \ \ \\
& \\
\hline 
\hline
 & \\
 \mbox{Necessity Logic} & \\
 & \\
\ \ \onlyIf{A_1}{A_2}{A_3}\ \ \ &\  \forall \sigma, \sigma',\prg{cmd}.\ [ \ \sigma \models A_1 \ \wedge\ \sigma, \prg{cmd} \leadsto^* \sigma' \  \wedge\  \sigma' \models A_2 \ \  \longrightarrow \ \sigma \models A_3\ \  ]  \\\

 &  \ \ \ \ \ \   A_3 \mbox{ over--approximates   preconditions that lead from  $A_1$ to $A_2$}  \\
 & \\
\hline
\end{array}
$

\vspace{.1in} 

In Hoare Logic as well as in Incorrectness Logic, $A_1$ is a  \emph{sufficient} condition for  the 
\emph{particular} command \prg{cmd} to reach $A_2$. 
But in Necessity Logic, $A_3$ is a \emph{necessary} precondition for \emph{any} command starting 
at $A_1$ to reach  $A_2$.

\vspace{.2in}

 

In particular, if we have proven that a module satisfies $\onlyIf{A_1}{A_2}{false}$, then we know that a transition from $A_1$ to $A_2$ will never happen. 
Note that we could also have Incorrectness-Necessity Logic, where we underapproxiate the necessary condition, defined below. Now if we can prove that $ \onlyIfIncorrect{A_1}{A_2}{A_3}$ and $A_3$ is not $false$, then we would have proven that a a transition from $A_1$ to $A_2$  can happen.\footnote{SOPHIA: not clear this is so; would we need the logic to be complete? TO-THINK.}

\noindent

$\begin{array}{|l|l|}
\hline
 & \\
 \mbox{I-Necessity Logic} & \\
 \ \ \ \ \mbox{(future work ?)}  &  
\\
 & \\
\ \ \onlyIfIncorrect{A_1}{A_2}{A_3}\ \ \ &\  \forall \sigma, [ \ \ \sigma \models A_3 \ \wedge\ \sigma \models A_1 \ \Longrightarrow \exists \prg{cmd}, \sigma' (\ \sigma, \prg{cmd} \leadsto^* \sigma' \ \wedge\ \sigma' \models A_2\ \  ]  \\\

 &  \ \ \ \ \ \   A_3 \mbox{ under--approximates   preconditions that lead from  $A_1$ to $A_2$}  \\
 & \\
\hline

\end{array}
$
