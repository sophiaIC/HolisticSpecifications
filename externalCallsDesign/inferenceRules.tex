Here we will write the inference rules we cam up with.

Sophia thinks that we need no more than a way to argue that a sequence of internal code and codes to external sartsfies a Hoare triple. And I now think we do not need to "go beyond" Hoare triple notation. :-)



\begin{mathpar}
\infer
	{
	\Mod{},\,  \SpecB{} \ \vdash\ \ \hoare{A_1}{stms_1} {A_2} \ \
	\ \ \ \ \ \ \ \ \ \ \ \ \ \ \ \ \ \ \ \ \\
	stmts_2 \   \mbox{ consists exclusively of external calls}\ \ 
 \\
	\overline{z} \mbox{ are all the internal variables passed as arguments in }  stmts_2
	\\
	\Mod{}, \, \SpecB{} \ \vdash\   { \Agree {A_2}  {\SpecB{}}  {\outside{\overline{z}}}  {A_3}}
	}
	{
	\Mod{}, \, \SpecB{} \ \vdash\  \hoare
		{A_1}
		{stmts_1; \ stmts_2}
		{A_3}
	}
\end{mathpar} 
\footnote{HELP: why bold?}


The challenge is how to define the $\Agree {A_1}   {\SpecB{}}   {A_3}   {A_4} $. The intuition is that we havoc the assertions in $A_1$ but only if allowed by $\SpecB{}$, while also making $A_3$ hold; and this "agrees" with $A_4$.
