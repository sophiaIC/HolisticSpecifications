% !TEX TS-program = pdflatex
% !TEX encoding = UTF-8 Unicode

% This is a simple template for a LaTeX document using the "article" class.
% See "book", "report", "letter" for other types of document.

\documentclass[11pt]{article} % use larger type; default would be 10pt

\usepackage[utf8]{inputenc} % set input encoding (not needed with XeLaTeX)
\usepackage{relsize}
\usepackage{mathpartir}
\usepackage{amsmath}
\usepackage{amsthm}
\usepackage{listings}
\usepackage{xspace}
\usepackage{definitions}
\usepackage{multirow,bigdelim}
\usepackage{pbox}
\usepackage{courier}
\usepackage{amssymb}
\usepackage{textcomp}

\theoremstyle{definition}
\newtheorem{definition}{Definition}[section]

%%% Examples of Article customizations
% These packages are optional, depending whether you want the features they provide.
% See the LaTeX Companion or other references for full information.

%%% PAGE DIMENSIONS
\usepackage{geometry} % to change the page dimensions
\geometry{a4paper} % or letterpaper (US) or a5paper or....
% \geometry{margin=2in} % for example, change the margins to 2 inches all round
% \geometry{landscape} % set up the page for landscape
%   read geometry.pdf for detailed page layout information

\usepackage{graphicx} % support the \includegraphics command and options

% \usepackage[parfill]{parskip} % Activate to begin paragraphs with an empty line rather than an indent

%%% PACKAGES
\usepackage{booktabs} % for much better looking tables
\usepackage{array} % for better arrays (eg matrices) in maths
\usepackage{paralist} % very flexible & customisable lists (eg. enumerate/itemize, etc.)
\usepackage{verbatim} % adds environment for commenting out blocks of text & for better verbatim
\usepackage{subfig} % make it possible to include more than one captioned figure/table in a single float
% These packages are all incorporated in the memoir class to one degree or another...

\usepackage{mathtools} % for cases
%%% HEADERS & FOOTERS
\usepackage{fancyhdr} % This should be set AFTER setting up the page geometry
\pagestyle{fancy} % options: empty , plain , fancy
\renewcommand{\headrulewidth}{0pt} % customise the layout...
\lhead{}\chead{}\rhead{}
\lfoot{}\cfoot{\thepage}\rfoot{}

%%% SECTION TITLE APPEARANCE
\usepackage{sectsty}
\allsectionsfont{\sffamily\mdseries\upshape} % (See the fntguide.pdf for font help)
% (This matches ConTeXt defaults)

%%% ToC (table of contents) APPEARANCE
\usepackage[nottoc,notlof,notlot]{tocbibind} % Put the bibliography in the ToC
\usepackage[titles,subfigure]{tocloft} % Alter the style of the Table of Contents
\renewcommand{\cftsecfont}{\rmfamily\mdseries\upshape}
\renewcommand{\cftsecpagefont}{\rmfamily\mdseries\upshape} % No bold!
\newcommand{\onCall}[1]{\blacktriangleright #1}
\newcommand{\onReturn}[1]{#1 \blacktriangleleft}

%%% END Article customizations

%%% The "real" document content comes below...

\title{External Calls}
%\date{} % Activate to display a given date or no date (if empty),
         % otherwise the current date is printed 
\author{Sophia}
 
\begin{document}
\maketitle

\section{Questions we discussed during   the Feb/March and then April meetings in London}

\begin{enumerate}
\item
Things we can do next
\begin{enumerate}
\item
better notation
\item
logic rules for deducing "inside"

\item
an operator, say  $\oplus$,  that combines two holistic specs. That is, assuming that
$\Mod{_a} \models \Spec{a}$, and $\Mod{_b} \models \Spec{b}$, we should obtain
$\Mod{_a}\circ \Mod{_b} \models  \Spec{a} \oplus \Spec{b}$.

\item
complete Rule-1, and Rule-2

\item
Things we can do in the future
\item 
Is there use for an inside predicate which specifies the module it refers to?

\item
Make "inside" "object"-specific, rather than "module"-specific as is now

\item
Is our logic complete? And do we need all of the logic -- compare with ProVerif!
\end{enumerate}

 

\item
How are Necessity Logic and Incorrectness Logic related?

We discussed this, and have a crisp statement. Sophia has written it up in \ref{sect:compare}.

\item
Can we find a different running example?

??? 
\item
Do we need/want to change  the specification language? In particular, use ideas from adversarial logic?

At least we will change the notation, cf. sect \ref{sect:notation}. Also some new concepts possibly, eg $Path(...)$.
\SP: Some new ideas, with which we would   need only binary operators for next-only-if and only-if (do not think only-through could be binary); cf. sect \ref{sect:notation}. 

\item
How can we reason about methods which allow external calls? 
How to extend our Logic, so that we can make the argument form sect 2.3.1, from the OOPSLA paper


this document, and in particular

\begin{itemize}
\item
Can we link  modules, and how much of spec can we inherit  from small to large? ANSWER: yes, we can, and should work on this now!
\item 
???
\end{itemize}

\item
Do we want "deep" or "shallow" meaning of "next observable" state in the semantics?

We (provisionally) settled on "shallow". \SP: We had thought that "shallow" cannot express the DAO vulnerability, but now Sophia think it does, c.f. Sect\ref{s:DAO}.

\item
Title for next paper: "Local reasoning about non-local properties" SD thinks it is a great title, but fears it is not true.

\item
When we consider whether an assertion is satisfied in a state, do we a) \textbf{From\_Current} only consider the objects which are transitively accessible from the currently top frame, or b)  \textbf{From\_All} all the objects which are transitively accessible from the all the frames? Sophia prefers a), but Julian developed the \prg{Blackadder} example, which, he convinced us all on Friday, would create problems for  \textbf{From\_Current} . \SP: But now Sophia thinks it does not!.
\item
Sophia's report about the meeting with Jules, 
\begin{enumerate}
\item 
He asked about tool support for NL, and whether NL is modular, and how NL treats the universal quantifiers 
\item
He pointed us to Unrealizability  Logic, from POPL 2023
\item
Discussed relation with Incorrectness Logic
\item
Asked about relation with temporal logic, and suggested (very humbly) that we use more succinct an "mathematical" notation
\item
Sophia thinks that we can adopt "adversarial" notation and so facilitate the explanation of the "external steps|" semantics.
\end{enumerate}
\end{enumerate}



\section{Comparison with Hoare Logic and Incorrectness Logic}
\label{sect:compare}

Hoare/Incorrectness Logic are concerned with the study of the effect of some given code. They answer the question whether some specific code (\prg{cmd}) might lead have a certain effect (going from $A_1$ to $A_2$). Thus they under/under approx the post condition ($A_2$) given a precondition  (here $A_1$) and some code (here \prg{cmd}). 

In contrast, NL is concerned with security properties; in security,  the code being executed might come from any, untrusted, third party,
and therefore is unknown. Therefore, NL is not concerned with a \emph{particular} code; instead, it is concerned with the conditions under which some effect  (going from $A_1$ to $A_2$) , which take place. NL can be used to prove that certain effect will never take place.  
\vspace{.2in}

$\begin{array}{|l|l|}
\hline
 & \\
\mbox{Hoare Logic}  &  
\\
 & \\
\ \  \{\, A_1 \, \}\ \prg{cmd} \  \{\, A_2 \, \} \ \ \ &\  \forall \sigma, \sigma'. [ \ \ \sigma \models A_1 \ \wedge\ \sigma, \prg{cmd} \leadsto^* \sigma' \ \longrightarrow \ \sigma' \models A_2\ \  ]  \\
 & \ \ \ \ \ \  A_2 \mbox{ over--approximates the outcome of executing \prg{cmd} in } A_1  \ \ \ \ \ \ \ \ \\
& \\
\hline 
\hline
 & \\
 \mbox{Incorrectness Logic}  &  \\
 & \\
\ \  [\, A_1 \, ]\ \prg{cmd} \  [\, A_2 \, ] \ \ \ &\  \forall \sigma'. [ \ \ \sigma' \models A_2 \ \ \longrightarrow \ \exists \sigma.( \ \sigma \models A_1 \ \wedge\ \sigma, \prg{cmd} \leadsto^* \sigma' \ ) \ \  ]  \\
 &  \ \ \ \ \ \   A_2 \mbox{ under-approximates the outcome of executing  \prg{cmd} in } A_1  \ \ \ \ \ \ \ \ \\
& \\
\hline 
\hline
 & \\
 \mbox{Necessity Logic} & \\
 & \\
\ \ \onlyIf{A_1}{A_2}{A_3}\ \ \ &\  \forall \sigma, \sigma',\prg{cmd}.\ [ \ \sigma \models A_1 \ \wedge\ \sigma, \prg{cmd} \leadsto^* \sigma' \  \wedge\  \sigma' \models A_2 \ \  \longrightarrow \ \sigma \models A_3\ \  ]  \\\

 &  \ \ \ \ \ \   A_3 \mbox{ over--approximates   preconditions that lead from  $A_1$ to $A_2$}  \\
 & \\
\hline
\end{array}
$

\vspace{.1in} 

In Hoare Logic as well as in Incorrectness Logic, $A_1$ is a  \emph{sufficient} condition for  the 
\emph{particular} command \prg{cmd} to reach $A_2$. 
But in Necessity Logic, $A_3$ is a \emph{necessary} precondition for \emph{any} command starting 
at $A_1$ to reach  $A_2$.

\vspace{.2in}

 

In particular, if we have proven that a module satisfies $\onlyIf{A_1}{A_2}{false}$, then we know that a transition from $A_1$ to $A_2$ will never happen. 
Note that we could also have Incorrectness-Necessity Logic, where we underapproxiate the necessary condition, defined below. Now if we can prove that $ \onlyIfIncorrect{A_1}{A_2}{A_3}$ and $A_3$ is not $false$, then we would have proven that a a transition from $A_1$ to $A_2$  can happen.\footnote{SOPHIA: not clear this is so; would we need the logic to be complete? TO-THINK.}

\noindent

$\begin{array}{|l|l|}
\hline
 & \\
 \mbox{I-Necessity Logic} & \\
 \ \ \ \ \mbox{(future work ?)}  &  
\\
 & \\
\ \ \onlyIfIncorrect{A_1}{A_2}{A_3}\ \ \ &\  \forall \sigma, [ \ \ \sigma \models A_3 \ \wedge\ \sigma \models A_1 \ \Longrightarrow \exists \prg{cmd}, \sigma' (\ \sigma, \prg{cmd} \leadsto^* \sigma' \ \wedge\ \sigma' \models A_2\ \  ]  \\\

 &  \ \ \ \ \ \   A_3 \mbox{ under--approximates   preconditions that lead from  $A_1$ to $A_2$}  \\
 & \\
\hline

\end{array}
$


\section{Notation}
\label{sect:notation}



\subsection{Specifications} Proposed new notation for specifications 

\noindent 
For $from\ A1\ to\  A2\ onlyIf A3$, is: \ \ \ \ \ \ \  \  \ \ \ $\onlyIf{A1} {A2} {A3}$. \\
\ \ \ \ For $from\ A1\ next\  A2\ onlyIf A3$, it is: \ \ \ \ \  $\onlyIfSingle{A1} {A2} {A3}$.
\\
\ \ \ \  For  $from\ A1\ to\  A2\ onlyThrough A3$, is:  \ \ \ $\onlyThrough{A1} {A2} {A3}$.

\vspace{.3in}
\noindent
\textbf{Proposal} Should we  perhaps only write necessity specs which will never happen, eg rather than 
$\onlyIf{A_1}{A_2}{A_3}$ have something like 
$Never({(A_1 \wedge \neg A_3) \diamond A_2})$. In terms of temporal logic, this would have the meaning: {$\square( \neg ( {(A_1 \wedge \neg A_3)  \wedge  \diamond A_2}))$}.  
Or, we could go one step further, and specify in the terms of $\square( ( A_1 \wedge \neg A_3) \diamond \neg A_2)$. This would turn our specs into binary operators. 

\vspace{.1in}
\noindent
We could have something like \\
$(*)\ \  \strut \ \ \ \ \ \ \ \ \ \ \ \ \ \ \ \   \TwoState{A} {A'}$  \\
which would be like a shorthand for  the temporal  $\square ( A \rightarrow \diamond A')$, which means 
\\
$(**)\ \  \strut \ \ \ \ \ \ \ \ \ \ \ \ \ \ \ \ \forall \sigma, \sigma',\prg{cmd}.\ [ \ \sigma \models A  \ \wedge\ \sigma, \prg{cmd} \leadsto^* \sigma' \    \   \longrightarrow \ \ \sigma' \models A'\ ].$

\vspace{.1in}
\noindent
Then, our Bank spec could have the form 

$\TwoState {\prg{a:Account}\wedge \inside{\prg{a.passwd}} \wedge \prg{a.balance}=b}  {\prg{a.balance}\geq b}$ 

\vspace{.1in}
\noindent
\textbf{Discussion}  Having only binary operators would be good, but would open two problems a) how to describe "only-through", b) the "narrative" would be different. We would not have "necessary preconditions" any more. Our spec would be more like 2-state invariants.

\paragraph{Specification Implication}
\label{lab:def:imply}

\begin{definition}
A specification $\SpecB{'}$ is \emph{stronger} than a specification $\SpecB{}$:

$\begin{array}{lll}
 \SpecB{'}\prec \SpecB{} & \triangleq &  
 \forall  \Mod . [\ \  \Mod \models \SpecB{'} \ \ \longrightarrow\ \   \Mod  \models \SpecB{}\ \ ]
  \end{array}
$

\end{definition}\footnote{ We should explore some $\prec$ relations, in the future. And relate with similar in temporal logic. But perhaps all the $\prec$ are already covered in the inference rules that Julian had created? Check whether Julian's rules are complete?}


\subsection{Assertions}

Proposed new notation for access: $\access{y} {x}$, and for external: $\external{p}$. Therefore, we will have nothing like "words" or "keywords" in the assertions, or the specifications/

Based on the above, we define "outside", as something that is accessible from something external:

\begin{definition}

We define the predicates $\outside{\_}$, and $\inside{\_}$ as follows:

$\begin{array}{lll}
   \Mod, \sigma \models \outside{x} & \triangleq & \ \Mod, \sigma \models \exists y.[ \ \access{y} {x} \wedge \external {y}  \ ] 
  ~ \\
   \\
      \Mod, \sigma \models \inside{x} & \triangleq & \ \neg \ (\  \Mod, \sigma \models \outside{x}  \ )
  \end{array}
$

\end{definition}



\paragraph{Implication}

Implication in assertions must be $\longrightarrow$. But what would it be in the metalanguage, eg in Def. \ref{lab:def:imply}.

\section{Code Examples}
Here we will write examples of the codes


\subsection{Something}

These  method s have no "story"; they serve  as warnings about things we should not take for grated.

\begin{lstlisting}[language=chainmail, mathescape=true, frame=lines]
class Account

     void warning( ) {
          p := new Password
          p' := new Password
          a := new Accoount.
          a.setPassword(null,p);
          outside(a)
          this.password := p  /*  and in variant\_1 */     this.password := p' 
          return p    
     }
     
     void dare_you(){
          this.status := frozen    //  to transfer we need the account to be unfrozen 
          outside(this.password)
          this.password:=new Password
          this.frozen := false
     }   // SD: hmhhh,  essentially we de-activated the account! We need 2.3.1!!!
     
...
\end{lstlisting}


\subsection*{Deactivate and leak old}

This method writes a new \prg{Passowrd} into the account, and thus essentially disables it. And it returns the old password.

\begin{lstlisting}[language=chainmail, mathescape=true, frame=lines]
class Account

     Password deacrtivate_and_leak_old( ) {
          p := this.password
          this.password := new Password
          return p    
     }
     
...
\end{lstlisting}

\subsection{Transfer with external}

This method  does half the job of  \prg{transfer}, then calls "outside", and then does the rest

\begin{lstlisting}[language=chainmail, mathescape=true, frame=lines]
class Account

     void transfer_2 (Password p, Account to, int amt) {
          if (p==this.password){
                  this.balance -= 2
                  to.balance += 2
                  outside();
                  this.balance -= amt -2
                  to.balance += amt-2
                     
     }
     
...
\end{lstlisting}

\subsection{Bank may leak unused Passwords}

Bank keeps all unused passwords in a list. 

\begin{lstlisting}[language=chainmail, mathescape=true, frame=lines]

class Bank

     Password makePassword( ) {
          p := new Password
          this.listUnusedPwds.enter(p) 
          return p;                    
     }
     
     Password getMeAPassword(){
          p := this.listUnusedPwds.popAndTop();
          return p;
     }
     
...
\end{lstlisting}


\section{Specification Examples}
Here we will write examples of the specs

\section{Reflections}
Here we will say which function satisfy which specs, under which other specs.

\section{Extensions to the Inference System of Necessity Logic}
We discussed a lot of variations of inference rules. But in the end, 
Sophia thinks that we need no more than a way to argue that a sequence of "known" code and calls to   unknown code  satisfies a Hoare triple. And I now think we do not need to "go beyond" Hoare triple notation. :-)

Note that here, the code about which we are reasoning (ie creating Hoare triples) does not need to be coming from the "safe" module. Instead, we may be reasoning about  code which is a client of the "safe" module. That is, we have three "views": the code being checked, i.e., $stmts_1;\ stmts_2$ below, the "unknown" code (here the calls in $stmts_2$), and the "safe" module, $\ModB{}$, which satisfies the specification \SpecB{}. it is possible that in $stmts_1$ we have calls to function from $\ModB{}$, and we could reason about them using their pre- and post-conditions. \footnote{The above should address the question that was raised by Peter Mueller when he read section 2.3.1. I fear it is not that crisp... HELP.}

\begin{mathpar}
\infer
	{
	\ \ \ \ \ \ \ \ \ \ \ \Mod{},\,  \SpecB{} \ \vdash\  \hoare{A_1}{stms_1} {A_2} \ \
	 \ \ \ \ \ \ \ \ \ \\
	stmts_2 \   \mbox{ consists exclusively of calls to "unknown" objects}\ \ 
 \\
	\overline{z} \mbox{ are all the variables passed as arguments in }  stmts_2
	\\
	\Out{A_2}{\overline{z}}=A_3\ \ \ \ \ \ \ \ \ {\Preserve{A_3} {\SpecB{}} {\Mod{}}  = {A_4}}
	}
	{
	\Mod{}, \, \SpecB{} \ \vdash\  \hoare
		{A_1}
		{stmts_1; \ stmts_2}
		{A_4}
	}
\end{mathpar} 


In the above, the term $\Out{A}{\overline{z}}$ returns an assertion $A'$ which is essentially $A$, with the difference that all elements from ${\overline{z}}$ are considered as $\outside{\_}$. 
Moreover,   ${\Preserve{A} {\SpecB{}} {\Mod{}} }$, contains the parts of $A$ that are preserved through  $\SpecB{}$.\footnote{needs better explanation.}

\begin{definition}
We define the  function \Out{\_}{\_}  below:

$\begin{array}{llll}
\Out{A}{\epsilon} & \triangleq & A
\\
\Out{A}{(z,\overline{z})} & \triangleq & \Out{\Out{A}{z}} {\overline{z}} 
\\
\Out{true}{z} & \triangleq & true \wedge \outside{z}
\\
\Out{false}{z} & \triangleq & true \wedge \outside{z}
\\
\Out{\inside{u}}{z} & \triangleq & (u\neq z \wedge  {\inside{u}} \vee u=z) \wedge  \outside{z}
\\
\Out{\outside{u}}{z} & \triangleq &  {\outside{u}}   \wedge  \outside{z}
\\
\Out{p=p'}{z} & \triangleq &  {p=p'}   \wedge  \outside{z}& \mbox{where }p, p' \mbox{ are paths}
\\
\Out{A \wedge A'}{z} & \triangleq &  \Out{A}{z}   \wedge \Out{A'}{z}
\\
\Out{A \vee A'}{z} & \triangleq &  \Out{A}{z}   \vee \Out{A'}{z}
\\
\Out{\neg A}{z} & \triangleq &???
\\
\Out{\forall u.A}{z} & \triangleq &\forall u. \Out{A}{z}
\\
\Out{\exists u.A}{z} & \triangleq &\exists u. \Out{A}{z}\footnote{here we also need a $\ModB{}$}
\\
\Out{Q(u)}{z} & \triangleq & \Out{Qbody[x/u]}{z} & \mbox{ where } Q \mbox{ is inductively defined }
\end{array}
$
\end{definition}

 \begin{lemma}
 If $\Out {A} {z}$=$A'$, then $A \rightarrow A'$, and $A' \rightarrow \outside{z}$. \footnote{here we also need a $\ModB{}$}
 \end{lemma}
 
 We will now define the preservation function:
 
 \begin{definition}
We define ${\Preserve{A} {\SpecB{}} {\Mod{}}}$ by cases over $\SpecB{}$ and its relation 
to $A$  below:

 
\[
\Preserve{A} {\onlyIf{A_1}{A_2}{A_3}} {\Mod{}}   \  \triangleq \   \begin{dcases*}
\neg A_3 
   & if  $\Mod {} \vdash  A\ \rightarrow \ A_1 \wedge \neg A_3$\,, \\[1ex]
true 
   & otherwise\,.
\end{dcases*} 
\]
 
$\begin{array}{lll}
 \Preserve{A} {\onlyIf{A_1}{A_2}{A_3};\SpecB{}} {\Mod{}}   & \triangleq &   \Preserve{A} {\onlyIf{A_1}{A_2}{A_3}} {\Mod{}}  \ \wedge \  \\
 & &  \Preserve{A} {\SpecB{}} {\Mod{}}
  \\
 \Preserve{A} {\onlyThrough{A_1}{A_2}{A_3};\SpecB{}} {\Mod{}}   & \triangleq &   \Preserve{A} {\SpecB{}} {\Mod{}}
 \\
  \Preserve{A} {\onlyIfSingle{A_1}{A_2}{A_3};\SpecB{}} {\Mod{}}   & \triangleq &   \Preserve{A} {\SpecB{}} {\Mod{}}
 \\

\end{array}
$
\end{definition}

 


\section{What we can / cannot prove}
Here we discuss which examples we can and which we cannot prove now

\subsection{Adding external calls to the "safe module's" code}

The method from below can be checked with $Rule_2$. \SP But notice that the spec is too strong! as it assumes that no external entity has access to the password. In that case, nobody can call the method \prg{transfer\_3}

\begin{lstlisting}[language=chainmail, mathescape=true, frame=lines]
class Account

    fld logger: Logger // external class, untrusted

     void transfer_3a (Password p, Account toAcc, int amt) 
     POST:  $\inside{\prg{this.passwd}}$ $\longrightarrow$ [ p==this.password) $\rightarrow$ ...
                                    $\wedge$
                                    $\forall$ a:Account.( a$\neq$fromAcc,toAcc $\rightarrow$ ...    )  ]
     {
          if (p==this.password) 
                  this.balance -= amt
                  toAcc.balance += amt
                  logger.log(this)
     }
 ...
\end{lstlisting}

\SP Should we consider "relative" inside/outside predicates, e.g. $\outsideTo{x} {y}$ holds if we only consider  objects transitively accessible from \prg{y}; that is  $\outsideTo{x} {y}$ holds if none of the external objects transitively accessible from \prg{y} had access to \prg{x}?

Here with the new spec

\begin{lstlisting}[language=chainmail, mathescape=true, frame=lines]
class Account

     field passwd : Password
     field logger: Logger // external class, untrusted

     void transfer_3b (Password p, Account toAcc, int amt) 
     POST: [ $\forall$ a:Account. $\outsideTo{\prg{a.passwd}}{\prg{logger}}$ ] $\longrightarrow$   
                                [ p == this.password) $\rightarrow$ ...
                                  $\wedge$
                                  $\forall$ a:Account.( a$\neq$fromAcc,toAcc $\rightarrow$ ...    )  ]
     {  ... $\mbox{as in }$ transfer_3a  ...  }
 ...
\end{lstlisting}




\subsection{Adding external calls in "our" code --  Me Bean}

We now consider functions which belong  to a class that is not part of $\Mod{_{BA}}$, which use $\Mod{_{BA}}$ as well as some untrusted third party code.  Here we have the module  \prg{MrBean} (ours), which relies on $\Mod{_{BA}}$ in order to pass its account to a third party while preserving some guarantees.

Consider method  \prg{transfer\_4}, which  is similar to  \prg{transfer\_3}. \SP I changed the spec.
\footnote{Here some earlier thoughts on the matter in the new spec: I suppose that we were assuming that we have something like  $\neg \Outside{\prg{a.passwd}}$? Or that ${\prg{a.passwd}}$ is not outside the module 
$\prg{MrBean}\circ\prg{BankAccount}$? Perhaps $\Outside{\_}$ should mention what it is outside of? Perhaps the module \prg{MrBean} would have as invariant that all "its" Accounts keeps their Passwords "inside"?}


\begin{lstlisting}[language=chainmail, mathescape=true, frame=lines]
module MrBean

   class MrAtkinson

      field logger: Logger // external class, untrusted

     void transfer_4 (Account fromAcc, Password p, Account toAcc, int amt) 
     POST:  [  $\forall$ a:Account. $\outsideTo{\prg{a.passwd}}{\prg{logger}}$ ] $\longrightarrow$ 
                                [ p == fromAcc.password) $\rightarrow$ ...
                                  $\wedge$
                                  $\forall$ a:Account.( a$\neq$tfromAcc,toAcc $\rightarrow$ ...    )  ]

     {
          fromAcc.transfer(p,toAcc,amt);
          logger.log(fromAcc)
      }
 ...
\end{lstlisting}


In \prg{transfer\_4} from above, we are passing  an object from
the "safe" module (here \prg{fromAcc}) to the untrusted object. 
What if we were  to pass not only a "safe" object, but also an object from "our" module?
Consider the function \prg{transfer\_5} from below.


\begin{lstlisting}[language=chainmail, mathescape=true, frame=lines]
module MrBean

   class MrAtkinson

    field logger: Logger // external class, untrusted

     // CHALLENGE 
     void transfer_5 (Account fromAcc, Password p, Account toAcc, int amt) 
     POST:  [  $\forall$ a:Account. $\outsideTo{\prg{a.passwd}}{\prg{logger}}$ ] $\longrightarrow$ 
                                [ p == fromAcc.password) $\rightarrow$ ...
                                  $\wedge$
                                  $\forall$ a:Account.  a$\neq$this,toAcc $\rightarrow$ ...     ]

     {   
         fromAcc.transfer(p,toAcc,amt);
         logger.log(fromAcc,this)
      }
 ...
\end{lstlisting}

In \prg{transfer\_5} above, we are passing \prg{this}  to the \prg{logger}. The function \prg{transfer\_5} is
only satisfied if \prg{MrAtkinson} does not leak the password, and does nor make unauthorized payments. So, 
it seems as if proving \prg{transfer\_5} requires a holistic spec for \prg{MrArkinson} too.

\subsection{What can be proven and how}

We can prove  \prg{transfer\_3a} using $Rule_1$, but as noted above, its spec is far too strong.

How could we prove the spec of \prg{transfer\_3a}, and \prg{transfer\_4a} and  \prg{transfer\_4b}.There are the following solutions

\SP The below needs revisiting:

\begin{enumerate}
\item
Include the \prg{Logger}'s code (or its spec) into the proof basis of \prg{Bank}, and make a holistic spec for both of them together. This is a non-solution, because then the logger would not be "external", and "untrusted".
\item
Include the \prg{Logger}'s code (or its spec) into the proof basis of \prg{MrBean}. This is also a non-solution, because then the logger would not be "external", and "untrusted".
\item
Include  \prg{MrBean}'s code  into the proof basis of \prg{Bank}, and make a holistic spec for both of them together. 
\begin{enumerate}
\item
We just prove holistically the "whole" new module. We then only need 
\item
We were hoping that we would be able to "inherit" some of the holistic spec of the \prg{Bank}. As we know, she meaning of $\Outside{\_}$ is delicate as it depends on modules, and appears in both positive and negative positions. But it might be possible.
\end{enumerate}
%\item
%Overwrite the password temporarily before giving it out. Code as in Sect. \ref{s:safe:extern}. 
\item
Introduce the concept of path-accessibility, which might be $Path( {o}, {o'}, {A(o''))}$, and would mean that any path from $o$ to $o'$  must go through an object $o''$ which satisfies  property $A(o'')$
\item
 Adopt $Rule_2$

\end{enumerate}

\noindent,
\textbf{Wrong Conjecture}  Sophia thought that if all the calls in  the body of a function in a client module are either calls to untrusted, or to the safe module (ie no calls to the module at hand), then we should be able to verify it, without holistic spec of the module itself. But \prg{transfer\_5} proves her wrong.

\vspace{.1in}

\textbf{Still Open} How to prove \prg{transfer\_5}  in a modular way?



\end{document}
