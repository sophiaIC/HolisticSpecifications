\section{Proving Necessity}
\label{s:inference}

In this Section we provide a proof system for constructing 
proofs of the \Nec specifications defined in \S \ref{s:holistic-guarantees}.
As discussed in \S \ref{s:approach},
\sdNr[]{such proofs consist} of 
\jm[]{four parts}: % proof of a \Nec specification:
\begin{description} 
\item[(Part 1)]
Proving Assertion Encapsulation (\S \ref{s:encaps-proof})
\item[(Part 2)]
Proving Per-Method \Nec specifications for a single internal method from the \funcSpec of that method (\S \ref{s:classical-proof})
\item[(Part 3)]
Proving Per-Step \Nec specifications by combining \sdNr[replaced per-call by per-method, as the prev sentence talks of per-method]{per-method} \Nec specifications (\S \ref{s:module-proof})
\item[(Part 4)]
Raising necessary conditions to construct proofs of \sdN{properties of} emergent behaviour (\S \ref{s:emergent-proof})
\end{description}

\sdr[replace concern by part, and reword]{Part 1 is, to a certain extent, orthogonal to the main aims of our work;
in this paper we propose a simple approach based on the type system, while also acknowledging that 
better solutions are possible.
For Parts 2-4, we  came up with the key ideas outlined in  \S \ref{s:approach}, which we
develop in more detail in \S \ref{s:classical-proof}-\S \ref{s:emergent-proof}.}


\subsection {Assertion Encapsulation}
\label{s:encaps-proof}

{
%A key component of constructing 
\sdNr[reword]{\Nec proofs 
often leverage the fact that some assertions cannot be invalidated unless some 
} internal (and thus known)
computation took place. 
{We refer to this property as \emph{Assertion Encapsulation}.}
}
\sdNr[reword, and banned model]{In this work, we define the property $M\ \vDash A'\ \Rightarrow\ \encaps{A}$, which states that
under the conditions described by assertion $A'$, the assertion $A$ is encapsulated by module $M$.
We  do not mandate how this property should be derived -- instead, we rely on a judgment 
$M\ \vdash A'\ \Rightarrow\ \encaps{A}$ provided by some external system. 
Thus, \Nec is parametric over the derivation of the encapsulation
     judgment; in fact, several ways to do that are possible \cite{TAME2003,ownEncaps,objInvars}. In 
    \jm[]{the appendices of the full paper \cite{necessityFull} we present a }
	%Appendix~\ref{s:encap-proof} and
    %Figure~\ref{f:asrt-encap}  we present a 
    rudimentary system that is sufficient to support our example
    proof. } 


\subsubsection{Assertion Encapsulation Semantics}

%%% WHY THE FUCK IS A' the subject and A the aux assertion???
\sdNr[somebody wrote "WHY THE FUCK IS A' the subject and A the aux assertion", and they were right, so I swapped A and A']{}
\sdNr[also, no "models the notion"]{}

\sdNr[]{As we said earlier,  an assertion $A$  is  encapsulated by a module $M$ under condition $A'$,
if in all possible states which arise from execution of module $M$ with any other external module $M'$, and which satisfy $A'$, 
the validity of $A$} 
\sdr[shortened, but all is here]{ can only be changed via computations internal to that module} -- \emph{i.e.},  via a call to
% In \Loo, that means by
%calls to objects defined in $M$ but \jm[removed: that are]{} accessible from the
%outside.
%%
%We provide a definition: $A'$ is encapsulated if whenever
%we go from state $\sigma$ to $\sigma'$, and when $A'$
%\jm[]{is invalidated} (i.e.\ to $\neg A'$) then we must have called a method on one
%of $M$'s internal objects.
%%
%In fact we rely on a slightly more subtle \jm[removed: underlying]{} definition,
%``conditional'' encapsulation where 
%$M\ \vDash A\ \Rightarrow\ \encaps{A'}$ expresses that in states which satisfy $A$, the assertion 
%$A'$ cannot be invalidated, unless 
a method from $M$.
In \Loo, that means by
calls to objects defined in $M$ but \jm[removed: that are]{} accessible from the
outside.
\jm[sophia asked if caller should be external, and if that is in coq - it is not in coq, but is implicit given the operational semantics of \Loo and the definition of Arising]{}

\begin{definition}[Assertion Encapsulation]
\label{def:encapsulation}
An assertion $A$ is \emph{encapsulated} by module $M$ and assertion $A'$, written as\ \  $M\ \vDash A'\ \Rightarrow\ \encaps{A}$, \ \ if and only if
for all external modules $M'$, and all states $\sigma$, $\sigma'$
such that $\arising{M'}{M}{\sigma}$:

\begin{tabular}{lr}
$\;\;\;\;$- $\reduction{M'}{M}{\sigma}{\sigma'}$  & \rdelim\}{3}{4mm}[$\;\;\;\Rightarrow\;\;\;$  $\exists x,\ m,\ \overline{z}. (\ \satisfiesA{M}{\sigma}{\calls{\_}{x}{m}{\overline{z}} \wedge\ \internal{x}}\ )$] \\
$\;\;\;\;$- $\satisfiesA{M}{\sigma}{A \wedge  A'}$ \\
$\;\;\;\;$- $\satisfiesA{M}{\sigma' \triangleleft \sigma}{\neg A}$ \\
\end{tabular} 
\end{definition}


\noindent
\sdN{Note that this definition   uses adaptation, 
${\sigma' \triangleleft \sigma}$. The application of the adaptation operator is necessary
because we  interpret the assertion $A$ in the current state, $\sigma$, while we interpret the assertion $\neg A$ in 
the future state, $\sigma' \triangleleft \sigma$.
}

Revisiting the examples from \S~\ref{s:outline}, % we can see
both \ModB and \ModC encapsulate   the \jm[]{equality of the \prg{balance} of an account to some value \prg{bal}: }
\sdr[]{This equality can only be invalidated} through calling  methods on internal objects.
%
\\
\strut \hspace{1cm}
$\ModB\ \vDash \prg{a}:\prg{Account}\ \Rightarrow\ \encaps{\prg{a.balance}=\prg{bal}}$
\\
\strut \hspace{1cm}
$\ModC\ \vDash \prg{a}:\prg{Account}\ \Rightarrow\ \encaps{\prg{a.balance}=\prg{bal}}$

%\susan[Does this help the story move on? I would omit it]{
%Note that encapsulation of an assertion does not imply encapsulation of its negation; 
%for example $\wrapped{o}$ is encapsulated, but $\neg  \wrapped{o}$ is not.}

\noindent 
%The key consequence of soundness is that -- SD dropped; it is   not a consequence of soundness!
\sdN{Moreover, the property that an object is only accessible from module-internal objects is encapsulated, that is, for all \prg{o}, and all modules $M$:}
\\
\strut \hspace{1cm}
$M\ \vDash \prg{o}:\prg{Object}\ \Rightarrow\ \encaps{\wrapped{\prg{o}}}$

\noindent
\sdN{This is so because any object which is only internally accessible} can become
  \jm[]{externally accessible} only via an internal call.

\sdN{In general},  code that does not contain 
calls to a \jm[]{given} module is guaranteed not to invalidate any assertions encapsulated by that module.
 Assertion encapsulation has been used in proof systems to {address}   the  {frame} problem
 \cite{objInvars,encaps}. 

\subsubsection{\sdN{Deriving} Assertion Encapsulation}

%As we have already stated at the beginning of this section,
%encapsulation is a deep topic that is well studied in the literature, 
%and is not the focus of this paper. For now, we simply assume the existence 
%of a proof system for encapsulation as it is secondary to the central topic 
%of this paper. We need only assert that such an algorithmic proof system 
%must be sound (Definition \ref{lem:encap-soundness}).
%% \susan[I commented out what was there as I thought it was repetious]
%% {We are assuming the existence of a proof system for encapsulation and only need to assert that such an algorithmic proof system nust be sound.}
%% The construction of the algorithmic system is not central to our work,
%% because, as we shall see in later sections, our logic 
%% does not rely on the specifics of an encapsulation algorithm, only its soundness.

Our logic does not \sd{deal with, nor} rely on, the specifics of  how   encapsulation
\sdN{is derived}.
  % model, 
\sdN{Instead, it relies} on an encapsulation judgment and expects it to be sound:

\begin{definition}[Encapsulation Soundness]
\label{lem:encap-soundness}
A judgement of the form $\proves{M}{\givenA{A'}{\encaps{A}}}$  is\  \emph{sound}, \ if 
for all modules $M$, and assertions $A$ and $A'$, if \\

$\strut \hspace{1.5cm} \proves{M}{\givenA{A'}{\encaps{A}}}\ \ \ \ $ implies $\ \ \ \ \satisfies{M}{\givenA{A'}{\encaps{A}}}$.
\end{definition}


  \jm[I'm not sure this paragraph should exist. It's key, but does it belong here or add anything?
  \sdN{I modified the paragraph on types}]{}



\paragraph{Types for Assertion Encapsulation}
\label{types}
\sdNr[I have unified the two separate descriptions of the types system.]{}
%To allow for an easy way to judge encapsulation of
\sdN{Even though the derivation of assertion encapsulation  is not the focus of this paper, 
for illustrative purposes, we will outline %here how it can be derived with the use of 
now a  very simple type system which supports such derivations:}
We assume that 
%assertions, we assume a very simple type system, where 
field declarations, method arguments
and method results are annotated with class names, and that classes may  
be annotated as \enclosed. A  \enclosed object  
\sdN{is not} accessed by external objects; that is, it is always \inside. 

The type system then checks 
that field assignments, method calls, and method returns adhere to these expectations,
and in particular, that objects of \enclosed type
are never returned from method bodies 
\sdN{-- this is a simplified version of the type system described in \cite{confined}.}
Because the type system is so simple, we do not include its formalization in the paper.
Note however, that the type system has one further implication: modules are typed 
in isolation, thereby implicitly prohibiting
method calls from internal objects to external objects. 

Based on this type system, we define a predicate $\intrnl{e}$, in \jm[]{the appendices of the full paper \cite{necessityFull},} %Appendix \ref{s:encap-proof},
which asserts that any \sdN{objects read} during the evaluation of $e$ are internal.
Thus, any assertion that only involves $\intrnl{\_}$ expressions is encapsulated -- more can be found in \jm[]{the appendices of the full paper \cite{necessityFull}.} %in Appendix \ref{s:encap-proof}.

\subsection{Per-Method \Nec Specifications}
\label{s:classical-proof}
In this section we detail how we use \funcSpecs
%\sdN{of the form ${\hoare{P}{\prg{s}}{Q}}$}
to \jm[]{prove} per-method \Nec specifications  
% That is, for some method $m$ in class $C$, we \jm[]{prove} a specification 
of the form 
$$\onlyIfSingle{A_1\ \wedge\ x : C\ \wedge\ \calls{\_}{x}{m}{\overline{z}}}{A_2}{A}$$
where $C$ is a class, and $m$ a method in $C$.
\sdNr[removed "Thus, $A$ is a necessary precondition for reaching $A_2$ from $A_1$ via a method call $m$ to an object of class $C$."
we should know that by now]{}


The first key idea in \S \ref{s:approach}
is that if a precondition and a certain statement is \emph{sufficient}
to achieve a particular result, % \sdN{\emph{eg}  $\hoare{P}{s}{Q}$},
then the negation of that precondition
is \emph{necessary} to achieve the negation of the result after executing that statement.
Specifically, 
 \sdN{$\hoare{P}{s}{Q}$ implies} that $\neg P$ is a \emph{necessary precondition} for $\neg Q$ to 
hold following the execution of $\prg{s}$.

For the use in \funcSpecs, we define \emph{Classical assertions}, a subset of \SpecO, comprising only those 
assertions that are commonly present in other specification languages.
% We provide this subset in Definition \ref{f:classical-syntax}. That is, classical assertions
\jm[]{They} are restricted to expressions, class assertions, the usual connectives, negation, 
implication, and the usual quantifiers.


\begin{definition}
%\begin{figure}[tbp]
% \footnotesize
Classical assertions, $P$, $Q$, are defined as follows 

$
\begin{syntax}
\syntaxElement{P, Q} {} % {Classical Assertion}
		{
		\syntaxline
				{e}
				{e : C}
				{P\ \wedge\ P}
				{P\ \vee\ P}
				{P\ \longrightarrow\ P}
				{\neg P}
				{\forall x.[P]}
				{\exists x.[P]}
		\endsyntaxline
		}
\endSyntaxElement\\
\end{syntax}
$
% \caption{Classical Assertion Syntax}
\label{f:classical-syntax}
% \end{figure}
\end{definition}


We assume that there exists some
proof system  that \sdN{derives} % allows us to prove 
\sdN{functional} specifications of the form  $\proves{M}{\hoare{P}{\prg{s}}{Q}}$.
This implies that we can also have guarantees of  
$$M\ \vdash\ \hoare{P}{\prg{res} = x.m(\overline{z})}{Q}$$
That is,   % if we assume we are able to prove for some method $m$
% defined in module $M$,
 the execution of $x.m(\overline{z})$ 
with the precondition $P$ results in a program state that 
satisfies postcondition $Q$, where the returned value is represented
by \prg{res} in $Q$.
\jm[]{We further assume that such a proof system is sound, i.e. that 
if $\proves{M}{\hoare{P}{\prg{res = x.m($\overline{z}$)}}{Q}}$, then 
for every program state $\sigma$ that satisfies $P$, the execution of the method call \prg{x.m($\overline{z}$)}
results in a program state satisfying $Q$.}
As we have previously discussed (see \S \ref{s:approach}), we build \Nec specifications
on top of \funcSpecs using the fact that 
validity of $\hoare{P}{\prg{res} = x.m(\overline{z})}{Q}$ implies that
$\neg P$ is a necessary pre-condition 
to $\neg Q$ being true after execution of ${\prg{res} = x.m(\overline{z})}$.


Proof  rules for per-method specifications are given in % captured by the two rules in 
Figure \ref{f:classical->singlestep}. \julian{Note that the receiver $x$ in the rules
in \ref{f:classical->singlestep} is implicitly an internal object. This is because 
we only have access to internal code, and thus are only able to prove the validity 
of the associated Hoare triple.}
\sdNr[chopped: "\textsc{If1-Classical} and \textsc{If1-Inside}
raise \funcSpecs to \Nec specifications.
These are rules whose conclusion have the form Single-Step Only If." as it breaks the flow]{}

\begin{figure}[t]
\footnotesize
\begin{mathpar}
\infer
	{
	\proves{M}{\hoare
						{x : C \ \wedge\ P_1\ \wedge\ \neg P}
						{\prg{res} = x.m(\overline{z})}
						{\neg P_2}}
	}
	{
	\proves{M}{\onlyIfSingle
			{P_1\ \wedge\ x : C \wedge\ \calls{\_}{x}{m}{\overline{z}}}
			{P_2}
			{P}}
	}
	\quad(\textsc{If1-Classical})
	\and
\infer
	{
	\proves{M}{\hoare
						{x : C \ \wedge\ \neg P}
						{\prg{res} = x.m(\overline{z})}
						{\prg{res} \neq y}}
	}
	{
	\proves{M}{\onlyIfSingle{\wrapped{y}\ \wedge\ x : C \wedge\ \calls{\_}{x}{m}{\overline{z}}}{\neg \wrapped{y}}{P}}
	}
	\quad(\textsc{If1-Inside})
\end{mathpar}
\caption{Per-Method \Nec specifications}
\label{f:classical->singlestep}
\end{figure}

 

 \textsc{If1-Classical} states that  
if \sdNr[dropped "by some classical logic" because usually classical logics are first order logics]{}  the execution of $x.m(\overline{z})$, with precondition $P \wedge \neg P_1$,
%has a resulting 
\sdN{leads} to a state satisfying postcondition $\neg P_2$, then $P_1$ is a \emph{necessary} precondition to the 
resulting state satisfying $P_2$.

\jm[I removed the old bit that was here because I felt we had already explained this several times before.]{}
% any state which satisfies $P_1$ and $\neg P$ and executes  the method $m$ on an \jm[typo]{object} of class $C$, leads to a state that satisfies $\neg P_2$, then, any state which satisfies $P_1$ and calls $m$ on an object of class $C$ will lead to a state that satisfies $P_2$ only if the original state also satisfied $P$.
%We can explain this also as follows: If the triple $.. \vdash \{R_1 \wedge R2\}\ s\ \{Q\}$ holds, then any state that satisfies $R_1$ and which upon execution of \prg{s} leads to a state that satisfies $\neg Q$, cannot satisfy $R_2$ -- because  if it did, then the ensuing state would have to satisfy $Q$,
 


 
%\textsc{If1-Inside} states that a method which does not return an object $y$ 
%preserves the ``insideness'' of $y$. 
\jm[]{\textsc{If1-Inside} states} that if  \sdNr[simpliified] {the precondition $\neg P$} guarantees that the result of
the call $x.m(\overline{z})$  
 is not $y$, then $P$ is a necessary pre-condition to invalidate $\wrapped{y}$ by calling
$x.m(\overline{z})$.
This is sound, \sdNr[added details]{because the premise} of \textsc{If1-Inside} implies that  $P$ is
a necessary precondition for \sdN{the call $x.m(\overline{z})$ to return} an object $y$; this, in turn,
implies that    $P$ is a necessary precondition for the call $x.m(\overline{z}$) to 
result in an external object gaining access to $y$.
%At first glance this rule might seem unsound, however the 
\sdN{The latter implication is valid} because
\jm[]{the} \sdN{rule is applicable only to external states semantics, which means that}
the call  $x.m(\overline{z})$ is a call from an external object to
some internal object $x$. \sdN{Namely,} there are only four ways
an object $o$ might gain access to another object $o'$: 
\sdNr[replaced $x$ and $y$ by $o$ and $o'$, because $x$ and $y$ are taken from earlier.]{}
(1) $o'$ is created by $o$ as the result of a \prg{new} expression, 
(2) $o'$ is written to some field of $o$, 
(3) $o'$ is passed to $o$ as an argument to a method call on $o$,
or (4) $o'$ is returned to $o$ as the result of a method call from an object $o''$ that has access to $o'$.
%The rules in Fig. \ref{f:classical->singlestep} 
\sdN{The rule \textsc{If1-Inside}} is only concerned with 
effects on program state resulting from a method call to some internal object, and thus (1) and (2) need not be considered as 
neither object creation or field writes may result in an external object gaining access \sdNr[was from]{to}  
an
\sdNr[used to say "an internal object", but I think this is too weak]{object that is only internally accessible.}
Since we are only concerned with describing how internal objects grant access to external objects,
our restriction on external method calls within internal code prohibits (3) from occuring. Finally,
(4) is described by \textsc{If1-Inside}.
In further work we plan to weaken the restriction on external method calls, and will  
strengthen this rule.
Note that \textsc{If1-Inside}  is essentially  a specialized version of \textsc{If1-Classical}
for the $\wrapped{\_}$ predicate. Since $\wrapped{\_}$ is not a classical
assertion, we cannot use \funcSpecs to reason about necessary conditions
for invalidating $\wrapped{\_}$.
 
 
 

\subsection{Per-Step \Nec Specifications}
\label{s:module-proof}

\begin{figure}[thb]
\footnotesize
\begin{mathpar}
\infer
	{
	\left[\infer{\textit{for all}\ \ C \in dom(M)\ \ \textit{and}\ \  m \in M(C).\prg{mths}, \\\\
				[\proves{M}{\onlyIfSingle
								{A_1\ \wedge\ x : C\ \wedge\ \calls{\_}{x}{m}{\overline{z}}}
								{A_2}
								{A_3}}]}{}\right]\\
	\proves{M}{A_1\ \longrightarrow\ \neg A_2}\\
	\proves{M}{\givenA{A_1}{\encaps{A_2}}}
	}
	{
	M\ \vdash\ \onlyIfSingle{A_1}{A_2}{A_3}
	}
	\quad(\textsc{If1-Internal})
	\and
%\infer
%	{\proves{M}{\onlyIf{A_1}{A_2}{A}}}
%	{\proves{M}{\onlyIfSingle{A_1}{A_2}{A}}}
%	\quad(\textsc{If1-If})
%	\and
\infer
	{
	\proves{M}{A_1 \longrightarrow A_1'}\\
	\proves{M}{A_2 \longrightarrow A_2'}\\
	\proves{M}{A_3' \longrightarrow A_3}\\
	\proves{M}{\onlyIfSingle{A_1'}{A_2'}{A_3'}}
	}
	{\proves{M}{\onlyIfSingle{A_1}{A_2}{A_3}}}
	\quad(\textsc{If1-$\longrightarrow$})
	\and
%\infer
%	{
%	\proves{M}{\onlyIfSingle{A_1}{A_2}{A}} \\
%	\proves{M}{\onlyIfSingle{A_1'}{A_2}{A'}}
%	}
%	{\proves{M}{\onlyIfSingle{A_1\ \vee\ A_1'}{A_2}{A\ \vee\ A'}}}
%	\quad(\textsc{If1-$\vee$I$_1$})
%	\and
%\infer
%	{
%	\proves{M}{\onlyIfSingle{A_1}{A_2}{A}} \\
%	\proves{M}{\onlyIfSingle{A_1}{A_2'}{A'}}
%	}
%	{\proves{M}{\onlyIfSingle{A_1}{A_2\ \vee\ A_2'}{A\ \vee\ A'}}}
%	\quad(\textsc{If1-$\vee$I$_2$})
%	\and
\infer
	{
	\proves{M}{\onlyIfSingle{A_1}{A_2}{A\ \vee\ A'}} \\
	\proves{M}{\onlyThrough{A'}{A_2}{\prg{false}}}
	}
	{\proves{M}{\onlyIfSingle{A_1}{A_2}{A}}}
	\quad(\textsc{If1-$\vee$E})
	\and
%\infer
%	{
%	\proves{M}{\onlyIfSingle{A_1}{A_2}{A}} \\\\
%	\proves{M}{\onlyIfSingle{A_1}{A_2}{A'}}
%	}
%	{\proves{M}{\onlyIf{A_1}{A_2}{A\ \wedge\ A'}}}
%	\quad(\textsc{If1-$\wedge$I})
%	\and
\infer
	{
	\forall y,\; \proves{M}{\onlyIfSingle{([y / x]A_1)}{A_2}{A}}
	}
	{\proves{M}{\onlyIfSingle{\exists x. [A_1]}{A_2}{A}}}
	\quad(\textsc{If1-$\exists_1$})
%	\and
%\infer
%	{
%	\forall y,\; \proves{M}{\onlyIfSingle{A_1}{([y / x]A_2)}{A}}
%	}
%	{\proves{M}{\onlyIfSingle{A_1}{\exists x. [A_2]}{A}}}
%	\quad(\textsc{If1-$\exists_2$})
\end{mathpar}
\caption{Selected rules for Single-Step \emph{Only If}}
\label{f:only-if-single}
\end{figure}

{The second key idea in \S \ref{s:approach}} allows us to
\sdNr[rephrase]{leverage several per-method \Nec specifications 
to obtain one per-step \Nec specification:}
%We now  {use our second breakthrough observation (\S \ref{intro:this:work}):}
Namely, if an assertion is encapsulated, and all methods within the internal module
\sdN{require} the same condition to the \sdN{invalidation} of that assertion, then 
% we are able to generalize} 
this condition is a necessary, program-wide, single-step condition   to the invalidation of that assertion.
\sdr[replaced necessary pre-condition by necessary condition]{}

\sdr[] {In this section} we present a selection of the rules whose conclusion is of the form Single Step Only If in Fig. \ref{f:only-if-single}.
 The full rule set can be found in \jm[]{the full paper \cite{necessityFull}.} %Fig. \ref{f:app:only-if-single}.

\textsc{If1-Internal} 
 lifts a \sdN{set of} per-method \Nec \jm[]{specifications} to a per-step \Nec specification.
Any \Nec specification which is satisfied for \jm[]{all} method
calls sent to any object in a module, is satisfied for \emph{any step}, even
an external step, provided that the effect involved, \ie going from $A_1$ states to
$A_2$ states, is encapsulated.

 The remaining rules are more standard, and are reminiscent of the Hoare logic rule of consequence.
\jm[]{We present a few of the more interesting rules here}:
 
The  rule for implication (\textsc{If1-$\longrightarrow$}) may strengthen
 properties of either the starting or ending state, or 
weaken the necessary precondition. 
%
%There are two disjunction introduction rules: 
%(a) \textsc{If1-$\vee$I1} states that any execution
%starting from a state satisfying some disjunction that reaches some future state, 
%must pass through either a necessary 
%intermediate state for the first branch, or a necessary 
%intermediate state for the second branch.
%(b) \textsc{If1-$\vee$I2} states that any execution 
%starting from some state and ending in a state satisfying a disjunction
%must pass through either a necessary intermediate state for 
%the first branch, or a necessary intermediate state for the second branch.
%
The disjunction
elimination rule (\textsc{IF1-$\vee$E}) mirrors typical disjunction elimination
rules, with a variation stating that if it is not possible  to reach 
the end state from one branch of the disjunction, then we can eliminate 
that branch. 

Two rules support existential elimination on the left hand side. % \jm[]{We present one here.}
\textsc{If1-$\exists_1$} states that if any single step of execution starting
from a state satisfying $[y/x]A_1$ for all possible $y$, reaching some state satisfying
$A_2$ has $A$ as a necessary precondition, it follows that any single step execution
starting in a state where such a $y$ exists, and ending in a state satisfying $A_2$,
must have $A$ as a necessary precondition.  \jm[]{The other rules  can be found in the full paper \cite{necessityFull}.} %Fig. \ref{f:app:only-if-single}.}



\begin{figure}[t]
\footnotesize
\begin{mathpar}
\infer
	{\proves{M}{\onlyIfSingle{A}{\neg A}{A'}}}
	{
	\proves{M}{\onlyThrough{A}{\neg A}{A'}}
	}
	\quad(\textsc{Changes})
	\and
\infer
	{
	\proves{M}{\onlyThrough{A_1}{A_2}{A_3}} \\\\
	\proves{M}{\onlyThrough{A_1}{A_3}{A}}
	}
	{\proves{M}{\onlyThrough{A_1}{A_2}{A}}}
	\quad(\textsc{Trans$_1$})
	\and
\infer
	{
	\proves{M}{\onlyThrough{A_1}{A_2}{A_3}} \\\\
	\proves{M}{\onlyThrough{A_3}{A_2}{A}}
	}
	{\proves{M}{\onlyThrough{A_1}{A_2}{A}}}
	\quad(\textsc{Trans$_2$})
	\and
\infer
	{
	\proves{M}{\onlyIf{A_1}{A_2}{A}}
	}
	{\proves{M}{\onlyThrough{A_1}{A_2}{A}}}
	\quad(\textsc{If})
	\and
\infer
	{}
	{\proves{M}{\onlyThrough{A_1}{A_2}{A_2}}}
	\quad(\textsc{End})
\end{mathpar}
\caption{Selected rules for \emph{Only Through} -- the rest can be found in the full paper \cite{necessityFull}} % Figure \ref{app:f:only-through-full}}
\label{f:only-through}
\end{figure}
\begin{figure}[t]
\footnotesize
\begin{mathpar}
\infer
	{
	\proves{M}{\onlyThrough{A_1}{A_2}{A_3}} \\
	\proves{M}{\onlyIf{A_1}{A_3}{A}}
	}
	{\proves{M}{\onlyIf{A_1}{A_2}{A}}}
	\quad(\textsc{If-Trans)}
	\and
\infer
	{}
	{\proves{M}{\onlyIf{x\ :\ C}{\neg\ x\ :\ C}{\false}}}
	\quad(\textsc{If-Class})
	\and	
\infer
	{}
	{\proves{M}{\onlyIf{A_1}{A_2}{A_1}}}
	\quad(\textsc{If-Start})
\end{mathpar}
\caption{Selected rules for \emph{Only If} -- the rest can be found in the full paper \cite{necessityFull}} %Figure \ref{app:f:only-if-full}}
\label{f:only-if}
\end{figure}
\jm[referenced full paper in only if/through figs]{}




%\subsection{Single-Step Necessary Conditions}
%\label{s:singleStep-proof}
%
\subsection{Emergent \Nec Specifications}
\label{s:emergent-proof}

The third key idea in \S \ref{s:approach}  allows us to
\sdNr[rephrase]{leverage several per-step \Nec specifications to 
obtain  multiple-step \Nec specifications, and thus enables the description of the module's emergent behaviour.}
We
\sdN{combine}   per-step \Nec specifications into  
multiple-step \Nec specifications, as well as several  multiple step \Nec specifications into further multiple step \Nec specifications.

 Figure \ref{f:only-through}  {presents} some of the rules with conclusion \emph{Only Through}, while Figure \ref{f:only-if}
provides some of the rules with conclusion \emph{Only If}. 
The \jm[I couldn't read Susan's comment on this]{full rules} can be found \jm[]{in the appendices of the full paper \cite{necessityFull}} %Appendix \ref{a:necSpec}.

%The rules for both of these relations are fairly similar to each other, 
%and to those of the single step \Nec specification from 
%section \ref{s:module-proof}. 
%Both relations include 
%rules for implication along with disjunction introduction and elimination.
%%
%While Fig. \ref{f:only-if} includes a rule for conjunction introduction (\textsc{If-$\wedge$I}),
%such a rule is not possible for \emph{only through}, as unlike \emph{only if}, where
%the necessary condition must hold, specifically, in the starting state, 
%there is no such specific moment in time in which the necessary condition 
%for \emph{only through} must hold. 
%Both relations also include rules for existentials on the left hand side ($\exists_1$, $\exists_2$, \textsc{If-$\exists_1$}, and \textsc{If-$\exists_2$}) \jm[]{(Appendix \ref{a:necSpec})}.
%These rules follow the corresponding \textsc{If1-$\exists_1$} and \textsc{If1-$\exists_2$}.

\textsc{Changes}, in Figure \ref{f:only-through}, 
{states that 
if   \jm[re:susan's coment, is this a question of primes vs subscripts?]{$A'$} is a necessary condition for the satisfaction of $A$ to change  in \emph{one} step, then
it is also a  necessary condition for the satisfaction of $A$ to change  in \emph{any number of} steps.
This is sound, because if  the satisfaction of some assertion changes over time, then 
 there must be some specific intermediate state where that change occurred.}
\textsc{Changes} is an important % rule in the logic, and is an
 enabler for proofs of emergent properties:
\jm[]{Since \Nec specifications}  are concerned with \sdN{necessary conditions for} change,
\sdN{their} proofs typically hinge around such \sdN{necessary conditions} for certain properties  %of the program state 
to change. For example,
under what conditions \sdN{may} our account's balance decrease? 
%It is this rule that ultimately connects program 
%execution to encapsulated properties.

It \jm[]{might} seem natural \jm[]{that} \textsc{Changes} \jm[]{had} the more
general form:
$$\infer{\proves{M}{\onlyIfSingle{A_1}{A_2}{A_3}}}{\proves{M}{\onlyThrough{A_1}{A_2}{A_3}}}\quad(\textsc{ChangesUnsound})$$
\sdN{\textsc{(ChangesUnsound)} is not sound because} the conclusion  of the rule describes 
transitions from a state satisfying $A_1$ to one satisfying $A_2$  \sdN{which may occur} occur over several steps,
\sdN{while the premise describes a transition that takes place} over one single step.
\sdN{Such a concern does not apply to \textsc{(Changes)},}  because 
a change in satisfaction for a specific assertion (\ie $A$ to $\neg A$) can \emph{only} take place in a single step.

\textsc{Trans}$_1$ and \textsc{Trans}$_2$  {are \jm[]{rules about transitivity.}}
\jm[]{They} {state} that necessary conditions to reach intermediate states or 
proceed from intermediate states are themselves necessary intermediate states. 
\jm[moved]{Any \emph{Only If} specification entails the corresponding
 \emph{Only Through} specification (\textsc{If}).}
\jm[]{Finally, \textsc{End} states that the ending condition is 
a necessary intermediate condition.}
%

\emph{Only If} also includes a rule for transitivity (\textsc{If-Trans}), but 
since the necessary condition must be true in the beginning state,
there is only a single rule. \textsc{If-Class} expresses that
an object's class never changes.
Finally, any starting condition is
itself a necessary precondition (\textsc{If-Start}). 



\subsection{Soundness of the \Nec Logic}

\label{s:soundness}

\begin{theorem}[Soundness]
\label{thm:soundness}
Assuming a sound \SpecO proof system, $\proves{M}{A}$, and
a sound encapsulation inference system, $\proves{M}{\givenA{A}{\encaps{A'}}}$,
 and  that on top of these systems we built
 the \Nec logic according to the rules in Figures \ref{f:classical->singlestep},  \ref{f:only-if-single}, 
 \ref{f:only-through},  and \ref{f:only-if},   then, for    all modules $M$, and all \Nec specifications  $S$:
 
 $$\proves{M}{S}\ \ \ \ \ \ \ \mbox{implies}\ \ \ \ \ \  \ \ \ \satisfies{M}{S}$$
\end{theorem}

\begin{proof}
by induction on the derivation of $\proves{M}{S}$.
\end{proof}
%\jm[]{The proof of soundness (Theorem. \ref{thm:soundness}) proves
%that our proof system for \Nec adheres to the semantics of \Nec specifications.
%We make two assumptions for soundness: (1) a sound proof system for assertion encapsulation, 
%and (2) a sound proof system for \SpecO. It is notable that \Nec is parametric with both 
%of these judgments.}

Theorem. \ref{thm:soundness} demonstrates 
 that the   \Nec logic is sound with respect to the semantics of \Nec specifications.
 The \Nec logic parametric wrt to the algorithms for proving validity of assertions
 $\proves{M}{A}$, and 
 assertion encapsulation ($\proves{M}{\givenA{A}{\encaps{A'}}}$), and is sound
 provided that these two proof systems are sound.


% and the 
%The  two assumptions for soundness: (1) a sound proof system for assertion encapsulation, 
% and (2) a sound proof system for \SpecO. It is notable that \Nec is parametric with both 
%%of these judgments.

% We have mechanized
The mechanized  proof of Theorem \ref{thm:soundness} in Coq 
can be found in the associated artifact \cite{necessityCoq2022}. 
The   Coq formalism deviates slightly from the system as
presented here,  mostly in the formalization of the 
\SpecO language. The Coq version of \SpecO restricts variable usage to expressions, and allows only addresses to 
be used as part of non-expression syntax. 
For example, in the Coq formalism
we can write assertions like $x.f==\prg{this}$ and
$x==\alpha_y$ and  $\access{\alpha_x}{\alpha_y}$, but we cannot write assertions 
like $\access{x}{y}$, where $x$ and $y$ are variables, and $\alpha_x$ and $\alpha_y$ are
addresses.
The reason for this restriction in the Coq formalism is to avoid spending % sizable 
significant effort encoding variable
renaming and substitution, a well-known difficulty for languages such as Coq. 
% This is justifiable, as we are still 
This restriction does not affect the expressiveness of %assertions in the
our  Coq formalism: we are
able to express assertions such as $\access{x}{y}$, by using addresses and introducing equality expressions % as part of expressions 
to connect variables to address, \ie
 $\access{\alpha_x}{\alpha_y} \wedge \alpha_x == x \wedge \alpha_y == y$.
 \jm[]{The Coq formalism makes use of the \prg{CpdtTactics} \cite{chlipala}
 library of tactics to discharge some proofs.}
