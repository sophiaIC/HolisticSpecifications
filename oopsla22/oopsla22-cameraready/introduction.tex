\section{Introduction: Necessary Conditions and Robustness}
\label{s:intro}

%\subsection{Necessary conditions and Robustness}

%% Today's   software has been built 
%% over decades by combining modules and components of
%% different provenance and 
%% %different degrees of 
%% trustworthiness, and
%% is \emph{open}, interacting with other programs, devices, and people.

% according to IEEE standard,
% robust = The degree to which a system or component can function correctly in the presence 
% of invalid inputs or stressful environmental conditions.  

{Software needs} to be both {\emph{correct}} ({programs do what they
  are supposed to}) and {\emph{robust}} ({programs \emph{only} do what they are supposed to}). 
 {We use the term \emph{robust} as a generalisation of \emph{robust safety}~\cite{gordonJefferyRobustSafety, Bugliesi:resource-aware,ddd}  whereby a module or process or ADT is \emph{robustly safe} if its execution preserves some safety guarantees even when run together with unknown,   unverified, potentially malicious client code. 
 The particular safety guarantees vary across the literature. 
We are interested in \emph{program-specific} safety guarantees which describe  \emph{necessary conditions}
 for some effect to take place.
In this work we propose how to specify such necessary conditions, and how to prove that modules adhere to such specifications.}   
 
 We motivate the need for  necessary conditions for effects through an example:
%  {We use robustness to refer
%  to \emph{robust safety}, a term 
%  used in both the security~\cite{gordonJefferyRobustSafety, Bugliesi:resource-aware} and object capabilities literature~\cite{ddd} to mean that
%programs don't do what they aren't supposed to do, even in the
%presence of untrusted or malicious {clients}}.
% robust = The degree to which a system or component can function correctly in the presence 
% of invalid inputs or stressful environmental conditions.  
{Correctness is} {traditionally} specified
%formally 
through \citeasnoun{Hoare69} triples: a  precondition, a code snippet, and a
 postcondition. 
 For example,  {part of the \funcSpec
   of a \prg{transfer} method for a bank module is that the source account's balance decreases:}
 \begin{quote}
   \Scorrect\ \ $\triangleq$  
 % SD I could not make the below work ...
 % {\scriptsize \lstinline*{pwd=src.pwd  $\wedge$ \lstinline* src.bal=b} src.transfer(dst,pwd) {src.bal=b-100 * $\wedge$ \ldots } } }\\
 {\footnotesize{ $\{\,$\prg{pwd=src.pwd} $\,\wedge\,$ \prg{src.bal=b}$\,\}$ \prg{src.transfer(dst,pwd)} $\{$ \prg{src.bal=b-100}$\,\wedge\,\dots \}$ }} Calling \prg{transfer} on  {an account with the correct password} will transfer the money.
\end{quote}
Assuming termination, the precondition is a \emph{sufficient} condition for the {code snippet to behave correctly}: 
%assuming termination, 
the precondition (\eg providing the right 
password) guarantees that
the code (\eg call the \prg{transfer} function)
will always achieve the postcondition (the money is transferred).
 
    \vspace{.05in}
    % I think we need the vspace
 
%While 
\Scorrect  describes  {the \emph{correct use} of the  {particular function}, but is \emph{not} concerned with the {module's} \emph{robustness}.}
{For example, can I pass an account to foreign untrusted code, in the expectation of receiving a payment,
but without fear that a malicious client might use the account to steal my money \cite{ELang}?}
 A first  {attempt} to specify robustness could be:
 

\begin{quote}
\SrobustA\ \ $\triangleq$ \ \ An account's balance does not decrease unless \prg{transfer} was called 
with the correct password.
\end{quote}

Specification \SrobustA % gives   the guarantee 
{guarantees} that it is not possible to  take money out of the account 
% through some method other than \prg{transfer}
{without calling \prg{transfer} and} without providing the password.
  Calling \prg{transfer}   with the  correct password is 
a \emph{necessary condition} for {(the effect of)} reducing the account's  balance.

\SrobustA is  crucial, but  not   enough:
it does not take  account of the module's \emph{emergent behaviour},
{that is, does not cater for the potential interplay of several methods offered by the module.}
 What if the module provided further methods which leaked the password?  
%  {or allowed for it to be arbitrarily changed}?
{ While no single procedure call is capable of breaking the intent of \SrobustA, a sequence of calls might.}
{What} we really need is
 \begin{quote}
\SrobustB\ \ $\triangleq$ \ \ The balance of an account does not {\emph{ever}} decrease in the future unless some external 
object  {\emph{now}} has access to the account's current password.
\end{quote}
With \SrobustB, I can confidently pass my account to  {any, potentially untrusted context, where my password is not known;} 
%some untrusted client who does not have knowledge of the password; 
 {the payment I was expecting may or may not be made,}
%they may or may not make the payment I was expecting, 
but I know 
%they will not  {be able to} steal my money 
 {that my money will not be stolen} \cite{ooToSecurity}. 
 Note that \SrobustB  does not mention
 the names of any functions in the module, and 
 thus can be expressed without reference to any particular API ---
 indeed \SrobustB can constrain \emph{any} API with an account, an account
 balance, and a password.

 
 %{\sd{Many} 
%kinds of} guarantees have been proposed\sophiaPonder[dropped: ``proposed for  robustness'']{}, differing in the level 
%of granularity,   target  language or calculi, and intended use.  
{Earlier work addressing robustness} includes object capabilities  \cite{MillerPhD, dd, threoremsFreeSep}, 
information control flow \cite{Zdancewic:Myers:01,noninteferenceOS}, 
 correspondence assertions \cite{Maffeis:aiamb:thesis00},
 sandboxing  \cite{robustSafetyPatrignani,sandbox},
robust linear temporal logic   \cite{RLTL2022} -- to name a few. %are some of the proposals to ensure some level of robust safety. 
{Most of these  %works 
propose \emph{generic}  guarantees (\emph{e.g.} no dependencies from high values to low values),
\sdN{or preservation of module invariants},
while we work with  \emph{problem-specific}  guarantees \sdN{concerned with necessary conditions 
for specific effects}  (\emph{e.g.} no decrease in balance without access to password).}
{{{\sc{VerX}} \cite{VerX} and  \emph{Chainmail} \cite{FASE} also work on problem-specific guarantees.}
Both these approaches are able to express necessary conditions
  like \SrobustA using
  temporal logic operators and implication,
  and Chainmail is able to express \SrobustB,
  however neither have a proof logic
  \jm[removed:able]{} to prove adherence to such specifications.}
%






\subsection{\Nec}
\label{intro:this:work}
{In this paper we introduce \Nec,} the first approach that is able to  both express and prove
(through an inference system)
robustness specifications such as  \SrobustB.
{Developing a specification language with a proof logic that is able to prove properties such as \SrobustB % is complex
and must tread a fine line: the language must be rich enough to express complex specifications; temporal operators are needed along with object capability style operators 
that describe \emph{permission} and \emph{provenance}, while also being simple enough that proof rules might be devised.}


\vspace{.07in}
 {
The {first main} contribution  
{is} three novel operators that merge
temporal operators and implication 
and most importantly are both expressive enough to capture the 
examples we have found in the literature and provable through an inference system.
}
%
%%For the specification language we adopted  
%%\emph{Chainmail}'s    capability operators.
%%{For the 
%%  temporal operators, we observed that while their
%%   unrestricted combination with  other logical connectives allows us to talk about any
%%   number of points in time, the examples found in the literature talk about two or at most three such points. }
%
%  
%  {This led to the  crucial insight that we could merge  temporal operators and the implication 
% logical connective into our three}
%   \emph{necessity} operators. 
 One such necessity operator is \\
$ 
\strut \hspace{1.7in} \onlyIf {A_{curr}} {A_{fut}} {A_{nec}}
$  
% \begin{lstlisting}[mathescape=true, language=chainmail, frame=lines]
%                                from ${A_{curr}}$ to ${A_{fut}}$ onlyIf ${A_{nec}}$ 
%\end{lstlisting}
%  %      $A$          from ${A_{curr}}$ to ${A_{fut}}$ onlyIf ${A_{nec}}$          from ${A_{curr}}$ to ${A_{fut}}$ onlyThrough ${A_{nec}}$
\\
This form says that  
a  {transition} from a current state satisfying assertion $A_{curr}$ to a future
state satisfying $A_{fut}$  is possible only if  the   necessary 
condition
$A_{nec}$ holds in the \emph{current} state.
Using this operator, 
we can formulate  \SrobustB  
as
\begin{lstlisting}[language = Chainmail, mathescape=true, frame=lines]
   $\text{\SrobustB}$  $\triangleq$   from   a:Account $\wedge$ a.balance==bal    to   a.balance < bal
               onlyIf  $\exists$ o.[$\external{\texttt{o}}$ $\wedge$ $\access{\texttt{o}}{\texttt{a.pwd}}$]
\end{lstlisting}
Namely, a transition from a  {current} state where an account's balance is \prg{bal}, to a  {future} state where 
it has decreased, may \emph{only} occur if  {in the current state} some {\color{blue}{\prg{external}}}, unknown client object  
has access to that account's password.
More    in \S\ref{s:bankSpecEx}. 
 

  
Unlike  \emph{Chainmail}'s temporal operators, 
 the necessity operators %  $\onlyIf {\_} {\_} {\_}$  and $\onlyThrough {\_} {\_} {\_}$
 are  not first class, and may not appear in the assertions  {(\eg  ${A_{curr}}$)}. 
 This simplification enabled us to develop our proof logic. 
 Thus, we {have reached} a  sweet spot between expressiveness and 
 provability.
 
 
 \vspace{.07in}
 The second main contribution is  a logic that  enables us to prove that code 
 obeys \Nec specifications. {}
{Our insight was  that \Nec specifications are logically equivalent to the
intersection of an \emph{infinite} number of Hoare triples, \emph{i.e.,} 
$\onlyIf{A_1} {A_2} {A_3}$ is 
logically equivalent
% with 
{to}
 $\forall \prg{stmts}. \{ A_1 \wedge \neg A_3\} \prg{stmts} \{ \neg A_2 \}$.
\sdS{Note that in the above, the assertions $A_1$, $A_2$ and $A_3$ are fixed, while the code \prg(stmts) is universally quantified.
  This leaves the challenge that, usually, Hoare logics do not support such infinite quantification over the code.}}


{We addressed that  challenge} through three further insights: 
 {(1) \Nec specifications of emergent behaviour can be built up from \Nec specifications of
 single-step executions, which (2) can be built from encapsulation and \emph{finite} intersections
 of \Nec specifications of function calls, which  
 (3) in turn can be obtained from \emph{traditional} \funcSpecs.}
 %
%%The Eureka moment was the realisation that all the information we required was hiding  { in the individual methods' classical specifications:}
%The Eureka moment \jm[four? is 2 an observation?]{arose from four crucial observations:} %{ in the individual methods' classical specifications:}
%\jm[]{\begin{enumerate}
%\item 
%If an assertion is invalidated over an arbitrary number of execution steps, then there must have
%been some single execution step that invalidated it.
%\item
%Proofs of necessary conditions in the open world must be based on restricted permission
%\item
%If all methods within a module share the same necessary precondition to invalidate an ``\emph{encapsulated}'' assertion,
%then that precondition is generally necessary to invalidate that assertion.
%\item
%Necessary preconditions to functions can be encoded using functional specifications.
%\end{enumerate}}
%%\jm[]{These four observations are the basis of \Nec.}
 % \vspace{.02in}
 
 
 
  
\subsection{Contributions and Paper Organization}

%%In the next section, (\S\ref{s:outline}),  we outline our approach using a
%% bank account as  a motivating example.

%\jm[should this be ``a bank account''? This is the first time we mention it]{}
%
The contributions of this {work} are:
\begin{enumerate}
\item   A language to express \Nec specifications (\S\ref{s:semantics}), including three novel \Nec operators (\S\ref{s:holistic-guarantees}) that combine implication and temporal operators.  
% \item
%A language to
%express \Nec specifications (\S\ref{s:semantics}).

 \item
A logic for proving a module's adherence to \susan[]{its}
 \Nec specifications (\S\ref{s:inference}), and a proof of soundness of the logic, (\S\ref{s:soundness}),
both mechanised in Coq \jm[]{\cite{necessityCoq2022}}. 
 \item
A proof in our logic % the bank account 
  that our bank module {obeys} % the \Nec specification 
  \SrobustB (\S\ref{s:examples}),     mechanised in Coq.
  And a proof that
  a richer bank module which uses ghostfields and confined classes  obeys  \SrobustB (\S\ref{app:BankAccount}),
  also mechanised in Coq.
\item \ {Examples taken from the literature  (\S\ref{s:expressiveness} and \S\ref{s:expressiveness:appendix}) specified in \Nec  .}

\end{enumerate}


%\jm[]{
%Our formalization of \Nec does have two limitations. 
%Firstly, it is specifically a logic for 
%necessity and robustness properties. \Nec 
%is parametric with respect assertion satisfaction, correctness,
%encapsulation, and the type system. Much of these are
%well-trod ground in the literature, and where needed we
%do introduce simplistic language mechanisms to deal with them (e.g. a simple type system or notion of encapsulation).
%Secondly, we do not allow for ``callbacks'' 
%to external objects. This is a common restriction in the literature
%as many either prohibit callbacks, or require
%``effectively callback free contracts'' \cite{VerX},
%or place significant restrictions on callbacks \cite{}
%}


%We  discuss these  issues % limitations 
%further as 
 We place \Nec into the context of 
related work (\S\ref{s:related}) and consider our overall conclusions
(\S\ref{s:conclusion}). 
%
The Coq proofs of 
(2) and (3) above appear in the
supplementary material, along with {a}ppendices containing expanded 
definitions and further examples.
In the next section, (\S\ref{s:outline}),  we outline our approach using a
  bank   as  a motivating example.

% SD moved this para to  section 1.2, rather than the end of 1/1, because I wamted 1.1 to finsih on
% the strengths of our work, while this paragraph is about limitations.
  {A strength of our} work % formalization of \Nec  
is {that it is
parametric} with respect to assertion
satisfaction \sdS{and functional specifications} -- these questions are well covered in the literature,
{and offer several off-the-shelf solutions.}
 % formalization of \Nec  
\sdS{The current work is based on 
  a simple, imperative, typed, object oriented
language with unforgeable addresses and private fields; nevertheless,
 we believe
 that   our approach
is applicable to several programming paradigms, and 
 that   unforgeability and privacy
 can be replaced 
 by lower level mechanisms such as capability machines \cite{vanproving,davis2019cheriabi}.}
\sdS{In line  with other work in the literature, we do not --yet-- support
``callbacks'' out from internal objects (whose code gas been checked) to  external objects (ie unknown objects
whose code has not been checked).}

