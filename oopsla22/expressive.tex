\subsection{More Examples expressed in \Nec}
% do not say \Nec Specifications
% because it is language that is expressive, not the specification
\label{s:expressiveness}

In this section we introduce some further specification examples, and use them to elucidate finer points
in the semantics of \Nec. % We also  discuss which modules    satisfy  which specifications.

 \subsubsection{More examples of the Bank}
Looking back at the examples from  \S\ref{s:bankSpecEx},   it holds that
%\\
%\strut $\hspace{.2in}$  \ModA$\vDash$ \SrobustA    $\hspace{.6in}$ \ModB$\vDash$ \SrobustA
%  $\hspace{.6in}$ \ModC$\vDash$ \SrobustA
%  \\
%\strut   $\hspace{.2in}$  \ModA$\vDash$ \SrobustB    $\hspace{.6in}$ \ModB$\nvDash$ \SrobustB
%  $\hspace{.6in}$ \ModC$\vDash$ \SrobustB
  \\
  $\begin{array}{llll}
  \ \  \ \ \ \ \ & \ModA \vDash  \SrobustA    \ \ \ \ \ \ & \ModB \vDash \SrobustA \ \ \ \ \ \
  &  \ModC \vDash \SrobustA
  \\
 &  \ModA \vDash  \SrobustB    \ \ \ \ \ & \ModB \nvDash \SrobustB \ \ \ \ \ 
  &  \ModC \vDash \SrobustB
  \end{array}$
 

 
% SD dropped the below -- not that useful
% For more specification examples, consider the
%bank account discussed in Section \ref{s:intro}. We have already shown
%how we can specify knowledge of an account's password using \SrobustB,
%but we are also able to write other useful properties about the bank account. 

 
Consider now another four \Nec specifications:
 
\begin{lstlisting}[language = Chainmail, mathescape=true, frame=lines]
     $\text{\SRobustNextAcc}$   $\triangleq$  from a:Account $\wedge$ a.balance==bal  next a.balance < bal
                        onlyIf $\exists$ o.[$\external{\texttt{o}}$ $\wedge$ $\access{\prg{o}}{\prg{a.pwd}}$]                                           

     $\text{\SRobustNextCall}$  $\triangleq$  from a:Account $\wedge$ a.balance==bal  next a.balance < bal
                        onlyIf $\exists$ o.[$\external{\texttt{o}}$ $\wedge$ $\calls{\prg{o}}{\prg{a}}{\prg{transfer}}{\prg{\_, \_, \_}}$]
                       
     $\text{\SRobustToCall}$   $\triangleq$  from a:Account $\wedge$ a.balance==bal to a.balance < bal
                        onlyIf $\exists$ o.[$\external{\texttt{o}}$ $\wedge$ $\calls{\prg{o}}{\prg{a}}{\prg{transfer}}{\prg{\_, \_, \_}}$]  
                                          
     $\text{\SRobustThroughCall}$  $\triangleq$  from a:Account $\wedge$ a.balance==bal to a.balance < bal
                       onlyThrough $\exists$ o.[$\external{\texttt{o}}$ $\wedge$ $\calls{\prg{o}}{\prg{a}}{\prg{transfer}}{\prg{\_, \_, \_}}$]

\end{lstlisting}


{The specification \SRobustNextAcc  states that
the balance of an account decreases \emph{in one step}, only if an external object has access to the 
password. It a weaker specification than \SrobustB, because it applies when the 
decrease   takes place in \emph{one} step, rather than in \emph{a number} of steps.}
Even though \ModB does not satisfy \SrobustB, it does satisfy \SRobustNextAcc:
 
  $\begin{array}{llll}
  \   & \ModA \vDash \SRobustNextAcc  \   & \ModB \vDash \SRobustNextAcc \  
  & \ModC \vDash \SRobustNextAcc \\
  
  \end{array}$

\vspace{.07in} % SD thinks some space is needed here

{The specifications \SRobustNextCall and   \SRobustToCall are similar:
they both say that a decrease of the balance can only happen if the current statement is a call to \prg{transfer}.  
The former considers   a \emph{single} step, while the latter allows for \emph{any number} of steps. 
 \SrobustB is slightly different, because it  says that such a decrease is only possible if some \emph{intermediate}
 step called \prg{transfer}.
 All three   modules satisfy  \SRobustNextCall.  
On the other hand, the code \prg{a1=new Account; a2.transfer}$(...)$ decrements the balance of \prg{a2} and
does call \prg{transfer} but not as a first step; therefore, none of the modules satisfy 
\SRobustToCall. That is:
} 

 $\begin{array}{llll}
  & \ModA \vDash \SRobustNextCall     & \ModB \vDash \SRobustNextCall   
  & \ModC \vDash \SRobustNextCall
  \\
  & \ModA \nvDash \SRobustToCall     & \ModB \nvDash \SRobustToCall  
  & \ModC \nvDash \SRobustToCall
  \end{array}$
  
  \vspace{.07in} % SD thinks some space is needed here

 Finally, \SRobustThroughCall is a weaker requirement than \SRobustToCall, because it only asks
  that the \prg{transfer} method is called in \emph{some intermediate} step. 
  All modules satisfy it:
 
 
   $\begin{array}{llll}
  & \ModA \vDash\SRobustThroughCall     & \ModB \vDash \SRobustThroughCall  
  & \ModC \vDash \SRobustThroughCall
   \end{array}$
 

\label{ss:DOM}

\subsubsection{The DOM}  %\sophiaPonder[renamed Wrapper to Proxy]{  
This is the motivating example in \cite{dd},
dealing with a tree of DOM nodes: Access to a DOM node
gives access to all its \prg{parent} and \prg{children} nodes, with the ability to
modify the node's \prg{property} -- where  \prg{parent}, \prg{children} and \prg{property}
are fields in class \prg{Node}. Since the top nodes of the tree
usually contain privileged information, while the lower nodes contain
less crucial third-party information, we must be able to limit 
 access given to third parties to only the lower part of the DOM tree. We do this through a \prg{Proxy} class, which has a field \prg{node} pointing to a \prg{Node}, and a field \prg{height}, which restricts the range of \prg{Node}s which may be modified through the use of the particular \prg{Proxy}. Namely, when you hold a \prg{Proxy}  you can modify the \prg{property} of all the descendants of the    \prg{height}-th ancestors of the \prg{node} of that particular \prg{Proxy}.  We say that
\prg{pr} has \emph{modification-capabilities} on \prg{nd}, where \prg{pr} is
a  \prg{Proxy} and \prg{nd} is a \prg{Node}, if the \prg{pr.height}-th  \prg{parent}
of the node at \prg{pr.node} is an ancestor of \prg{nd}.
%}


The specification \prg{DOMSpec} states that the \prg{property} of a node can only change if
some external object presently has 
access to a node of the DOM tree, or to some \prg{Proxy} with modification-capabilties
to the node that was modified.
\begin{lstlisting}[language = Chainmail, mathescape=true, frame=lines]
DOMSpec $\triangleq$ from nd : Node $\wedge$ nd.property = p  to nd.property != p
          onlyIf $\exists$ o.[ $\external {\prg{o}}$ $\wedge$ 
                       $( \  \exists$ nd':Node.[ $\access{\prg{o}}{\prg{nd'}}$ ]  $\vee$ 
                         $\exists$ pr:Proxy,k:$\mathbb{N}$.[$\, \access{\prg{o}}{\prg{pr}}$ $\wedge$ nd.parent$^{\prg{k}}$=pr.node.parent$^{\prg{pr.height}}$ ] $\,$ ) $\,$ ]
\end{lstlisting}

\subsubsection{Expressiveness}
%or {\sc{VerX}}. 
% Nevertheless, 
%we believe that
%it  is powerful enough for the purpose of straightforwardly
%expressing robustness requirements. 
In order to investigate \Nec's expressiveness,  
we used it for
examples provided in the literature. In this section we considered the DOM, %  example, 
proposed by  \citeasnoun{dd}. In Appendix \ref{s:expressiveness:appendix},
we compare with examples proposed by  \citeasnoun{FASE}, and \citeasnoun{VerX}.
 

