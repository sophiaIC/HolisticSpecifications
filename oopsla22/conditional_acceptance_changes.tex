\documentclass[11pt]{amsart}
\usepackage{geometry}                % See geometry.pdf to learn the layout options. There are lots.
\geometry{letterpaper}                   % ... or a4paper or a5paper or ... 
%\geometry{landscape}                % Activate for for rotated page geometry
%\usepackage[parfill]{parskip}    % Activate to begin paragraphs with an empty line rather than an indent
\usepackage{graphicx}
\usepackage{amssymb}
\usepackage{epstopdf}

\newcommand{\rev}[1]{\emph #1}
\newcommand{\us}[1]{\bf #1}
\newcommand{\prg}[1]{\texttt{#1}}
\DeclareGraphicsRule{.tif}{png}{.png}{`convert #1 `dirname #1`/`basename #1 .tif`.png}

\usepackage{filecontents}


\usepackage{natbib}

\title{OOPSLA 2022 Paper 308 Conditional Acceptance Changes}
%\author{The Author}
%\date{3 December 2021}                                           % Activate to display a given date or no date

\begin{document}
\maketitle

\section{Mandatory Changes}

\begin{itemize}
\item
\emph{Address Rev\_A’s question about evaluating the quality of a specification}:

\begin{itemize}
\item
We provided a proof sketch of our claim in Section 2.3.1, that ${\mathcal S}_{robust2}$ guarantees that from a scope which does not
have (transitive) external access to the password, the balance will not increase, even though the context calling that scope might have
(transitive) external access to the password.

\item
We have also improved the discussion in section 2.3.1. 
\end{itemize}

\item
\emph{Tone down all strong words used;  e.g. “insight” is sufficient, rather than “breakthrough insight”,}

Done
\item
\emph{Drop the statements that no Hoare logic supports infinite intersections.}

Done
\item
\emph{Improve comparisons with related work section.}

\begin{itemize}
\item
We are now in touch with  the authors of some of the related work. 

\item
We have improved the comparison with the work by Swasey et al according to the authors' feedback. 

\item
We are also happy to implement any further, specific instructions from the reviewers. 
\end{itemize}

\item
\emph{Address all technical questions raised by Rev\_A -- more later.}

\begin{itemize}
\item
``\emph{Line 53+: The work by Swasey et al that is being cited here also shows some problem-specific reasoning, like the "usetwo" example in the introduction.}''
\item
``\emph{Line 74: So the specifications in existing work (VerX, Chainmail) are not "provable"? I am not quite sure what exactly the claim is here.}''
\item
``\emph{Line 92 (and 280): The paper claims repeatedly that no Hoare logic supports infinite intersections. That claim is clearly wrong. Several Hoare logics support using regular universal quantification over arbitrary (including infinite) types, e.g. HoCAP (Svendsen et al 2013, "Modular Reasoning about Separation of Concurrent Data Structures") or Iris (Jung et al 2018, "Iris from the ground up"). Prior work has even proven specifications of the form "forall code, property(code) -> {P} code {Q}" (i.e., exactly the kind of Hoare triple the authors are talking about here); this is standard in the work on logical relations. In particular, such a proof is at the heart of Swasey et al (cited in the paper). A simple example can be found in Krebbers et al (2017, "Interactive Proofs in Higher-Order Concurrent Separation Logic"), Theorem 6.1. (Note that the $\vDash$ there is but a Hoare triple.)}''
\begin{itemize}
\item
We have dropped all claims indicating no Hoare logic supports infinite intersections.
\end{itemize}
\item
``\emph{Line 580: This looks like to the right of the arrow, it is re-quantifying over an execution from $\sigma_1$ to $\sigma_n$. Doesn't this go wrong if execution is non-deterministic, in the sense that it might pick a different execution than the one to the left of the arrow?}''
\begin{itemize}
\item
We have adopted the equivalent form we proposed in our earlier response.
\end{itemize}
\item
``\emph{Line 1055: Have these proofs about the bank been mechanized in Coq? If not, what would it take to be able to do that, on top of the formalization of the core logic itself?}''
\begin{itemize}
\item
We have updated the beginning of Section 5 and Appendix G to indicate that the proofs have been mechanised in Coq, and where the sources can be found.
\end{itemize}
\end{itemize}

\item
\emph{Improve the explanations of Def. 3.9}

We have re-worked the explanations of Def. 3.9  and 3.10, so as to first explain the intuitions of the necessity operators regardless of the adaptation operator, also using a new  diagram (Fig. 3), and examples (lines  605-615), and then defined the adaptation operator (lines 616-668).

\item
\emph{Follow any further suggestions by the reviewers}
\begin{itemize}
\item
\emph{``My biggest problem was with Definition 3.9 - to my understanding the central definition of this paper and it remains somewhat blurry to me. Its presentation has worsened in the sense that the $\triangleleft$- operator is now being used before being defined or explained.''}
\begin{itemize}
\item
We have expanded our explanation of Def. 3.9 and the $\triangleleft$ operator, and have included a diagram to help provide intuition for Def. 3.9
\end{itemize}
\end{itemize}
\end{itemize}



\bibliographystyle{plainnat}
\bibliography{Response1} 


\end{document}  
%
% For emergent behaviour we will include the reviewer's statement and also say that ``(S2) does not take account of the module's \emph{emergent behaviour}. That is, (S2) does not consider the behavior that emerges from the interaction between the 
%\texttt{transfer} method, and the other methods of the bank module. What if the module leaks the password?''
% 
% We will replace the current Bank Account proof with a simpler Coq proof that matches the straightforward introductory example. We will put the current example in an appendix so that we can 
%show reasoning about ghost fields and more complex data structures. 
%
%We will move the clarifying examples to Section 2.
%
%The largest piece of work is the proof and that shouldn't take more than a week so we believe that we can make substantial improvements in presentation before mid January.
%
%
%
% 
%
%
% 
% %We propose the following amended explanation to clarify both it's importance, and it's meaning:
%
% 
% A list of the changes that you plan to make in
%  response to the reviews and the timeline for those changes.
%  
% 
%  
%\section{Response} A reviewer-by-reviewer list of answers to questions
%  with context extracted from the reviews. Use markdown syntax.
%
%\bibliographystyle{plainnat}
%\bibliography{Response1} 
%
%
%\end{document}  