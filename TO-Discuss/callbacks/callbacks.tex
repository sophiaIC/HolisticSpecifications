% !TEX TS-program = pdflatex
% !TEX encoding = UTF-8 Unicode

% This is a simple template for a LaTeX document using the "article" class.
% See "book", "report", "letter" for other types of document.

\documentclass[11pt]{article} % use larger type; default would be 10pt

\usepackage[utf8]{inputenc} % set input encoding (not needed with XeLaTeX)
\usepackage{relsize}
\usepackage{mathpartir}
\usepackage{amsmath}
\usepackage{amsthm}
\usepackage{listings}
\usepackage{xspace}
\usepackage{definitions}
\usepackage{multirow,bigdelim}
\usepackage{pbox}
\usepackage{courier}
\usepackage{amssymb}

\theoremstyle{definition}
\newtheorem{definition}{Definition}[section]

%%% Examples of Article customizations
% These packages are optional, depending whether you want the features they provide.
% See the LaTeX Companion or other references for full information.

%%% PAGE DIMENSIONS
\usepackage{geometry} % to change the page dimensions
\geometry{a4paper} % or letterpaper (US) or a5paper or....
% \geometry{margin=2in} % for example, change the margins to 2 inches all round
% \geometry{landscape} % set up the page for landscape
%   read geometry.pdf for detailed page layout information

\usepackage{graphicx} % support the \includegraphics command and options

% \usepackage[parfill]{parskip} % Activate to begin paragraphs with an empty line rather than an indent

%%% PACKAGES
\usepackage{booktabs} % for much better looking tables
\usepackage{array} % for better arrays (eg matrices) in maths
\usepackage{paralist} % very flexible & customisable lists (eg. enumerate/itemize, etc.)
\usepackage{verbatim} % adds environment for commenting out blocks of text & for better verbatim
\usepackage{subfig} % make it possible to include more than one captioned figure/table in a single float
% These packages are all incorporated in the memoir class to one degree or another...

%%% HEADERS & FOOTERS
\usepackage{fancyhdr} % This should be set AFTER setting up the page geometry
\pagestyle{fancy} % options: empty , plain , fancy
\renewcommand{\headrulewidth}{0pt} % customise the layout...
\lhead{}\chead{}\rhead{}
\lfoot{}\cfoot{\thepage}\rfoot{}

%%% SECTION TITLE APPEARANCE
\usepackage{sectsty}
\allsectionsfont{\sffamily\mdseries\upshape} % (See the fntguide.pdf for font help)
% (This matches ConTeXt defaults)

%%% ToC (table of contents) APPEARANCE
\usepackage[nottoc,notlof,notlot]{tocbibind} % Put the bibliography in the ToC
\usepackage[titles,subfigure]{tocloft} % Alter the style of the Table of Contents
\renewcommand{\cftsecfont}{\rmfamily\mdseries\upshape}
\renewcommand{\cftsecpagefont}{\rmfamily\mdseries\upshape} % No bold!
\newcommand{\onCall}[1]{\blacktriangleright #1}
\newcommand{\onReturn}[1]{#1 \blacktriangleleft}

%%% END Article customizations

%%% The "real" document content comes below...

\title{Callbacks}
\date{} % Activate to display a given date or no date (if empty),
         % otherwise the current date is printed 

\begin{document}
\maketitle

\section{Unrestricted External Method Calls in \Nec}

In order to extend \Nec with logic for external method calls, we aim to generalize 
the \Nec of OOPSLA `22. Specifically, \Nec in OOPSLA `22 uses traditional Hoare triples to encode 
per-function necessity specifications. This took the form of the following rules:
\begin{mathpar}
\infer
	{\proves
		{M}
		{\hoare
			{x : C \wedge \neg A}
			{\prg{res}:=x.m(\overline{y})}
			{\neg P}}}
	{\proves
		{M}
		{\onlyIfSingle
			{x : C \wedge \calls{\_}{x}{m}{\overline{y}}}
			{P}
			{A}}}
	\quad(\textsc{If1-Classic})
	\and
\infer
	{\proves
		{M}
		{\hoare
			{x : C \wedge \neg A}
			{\prg{res}:=x.m(\overline{y})}
			{\prg{res} \neq z}}}
	{\proves
		{M}
		{\onlyIfSingle
			{x : C \wedge \calls{\_}{x}{m}{\overline{y}} \wedge \wrapped{z}}
			{\neg \wrapped{z}}
			{A}}}
	\quad(\textsc{If1-Inside})
	\and
\infer
	{
	\left[\infer{\textit{for all}\ \ C \in dom(M)\ \ \textit{and}\ \  m \in M(C).\prg{mths}, \\\\
				[\proves{M}{\onlyIfSingle
								{A_1\ \wedge\ x : C\ \wedge\ \calls{\_}{x}{m}{\overline{z}}}
								{A_2}
								{A_3}}]}{}\right]\\
	\proves{M}{A_1\ \longrightarrow\ \neg A_2}\\
	\proves{M}{\givenA{A_1}{\encaps{A_2}}}
	}
	{
	M\ \vdash\ \onlyIfSingle{A_1}{A_2}{A_3}
	}
	\quad(\textsc{If1-Internal})
\end{mathpar}
All interactions between external and internal code in OOPSLA `22 take the form:
\begin{lstlisting}[language = Chainmail, mathescape=true, frame=lines]
...            // some external code
i.m(...)       // where i is internal
...            // some external code
\end{lstlisting}
That is, every interaction between external and internal code consists of 
{
\newcommand{\callHL}{{\textbf{\color{red}{call}}}\xspace}
\newcommand{\returnHL}{{\textbf{\color{blue}{return}}}\xspace}
\begin{description}
\item[$\rightarrow$]
\callHL from external to internal code 
\item[$\rightarrow$]
internal computation
\item[$\rightarrow$]
\returnHL from internal to external code
\end{description}
When we allow for external calls, interactions between external and internal code 
can take one of the four following forms:
\begin{enumerate}
\item
	\begin{description}
	\item[$\rightarrow$]
	\callHL from external to internal code 
	\item[$\rightarrow$]
	internal computation
	\item[$\rightarrow$]
	\returnHL from internal to external code
	\end{description}
\item
	\begin{description}
	\item[$\rightarrow$]
	\callHL from external to internal code 
	\item[$\rightarrow$]
	internal computation
	\item[$\rightarrow$]
	\callHL from internal to external code
	\end{description}
\item
	\begin{description}
	\item[$\rightarrow$]
	\returnHL from external to internal code 
	\item[$\rightarrow$]
	internal computation
	\item[$\rightarrow$]
	\returnHL from internal to external code
	\end{description}
\item
	\begin{description}
	\item[$\rightarrow$]
	\returnHL from external to internal code 
	\item[$\rightarrow$]
	internal computation
	\item[$\rightarrow$]
	\callHL from internal to external code
	\end{description}
\end{enumerate}}
Therefore, to generalize the solution of OOPSLA `22, specifications 
on internal methods are not sufficient to encode necessary pre-conditions
because we are not interested necessarily in the effects of entire
methods, rather we are interested in the effects of the internal computation
that is book ended by external control. Instead, we rather specify 
those maximal chunks of internal code (which used to be methods, but are 
now portions of methods). Since we are not interested in specifying 
entire methods (unless they do not contain any external method calls),
then we are not interested in specifying the effects of external code,
only the necessary conditions required to induce internal code to 
achieve specific effects.

Consider the following simple example:
\begin{lstlisting}[language = Chainmail, mathescape=true, frame=lines]
module IntMod
	method internalMethod (ext : ...)
		someStmts
		ext.unknown(...)
		moreStmts
		return x
\end{lstlisting}
This example consists of 2 of the 4 interactions detailed above:
a \textbf{\red{call}}/\textbf{\red{call}} and a \textbf{\blue{return}}/\textbf{\blue{return}}
We wish to specify some necessary preconditions on the internal 
module \prg{IntMod}. In OOPSLA `22, we would start by proving 
necessary preconditions on the entire method \prg{internalMethod}
(assuming no external method calls). To address the external method
call \prg{ext.unknown(...)}, we need to instead prove necessary 
preconditions on the two maximally internal chunks: \prg{someStmts} and \prg{moreStmts}.
To do this, we might imagine annotating our code with some assertions/assumptions:
\begin{lstlisting}[language = Chainmail, mathescape=true, frame=lines]
module IntMod
	{pre: $\neg$ A$_1$}      // (1)
	method internalMethod (ext : ...)
		someStmts
		assert($\neg$ A$_2$) // (2)
		ext.unknown(...)
		assume($\neg$ A$_1$) // (3)
		moreStmts
		return x
	{post: $\neg$ A$_2$}     // (4)
\end{lstlisting}
That is, if we can prove that calling \prg{internalMethod} in a state
satisfying $\neg A_1$ (1), means that $\neg A_2$ is provable
at (2), and that if assuming $\neg A_1$ at (3) means $\neg A_2$ is provable
at (4), then it follows that for any individual interaction between
external and internal code, $A_1$ is a necessary precondition for achieving $A_2$.
I should mention here that this is true, but only with a slight modification. 
(2) is not actually the position where $\neg A_2$ must be proven true, it is actually
in the very first state following the dispatch of the call \prg{ext.unknown(...)}. This 
is similar to the fact that the post condition must hold in the program state 
directly following the method return, and not the program state directly preceding it.
(Julian: this probably is also true for the position of (3), i.e. it should rather hold
directly before the return from \prg{ext.unknown(...)}, but I'm not sure if it makes a
difference.)

A more practical example would be the following:
\begin{lstlisting}[language = Chainmail, frame = lines]
module Logging
  class Logger
  	field logFile : Out
  	method log(s : String)
  		logFile.append(s)
module Bank
	import Logging
	class Password
	class Account
		field balance : int
		field password : Password
		field logger : Logger
		method pay(pwd, to)
			if this.authenticate(pwd)
				balance -= amt
				to.balance += amt
				return null
			return null
		method setPassword(pwd, newPwd)
			if this.authenticate(pwd)
				this.password = pwd
		method authenticate(pwd)
			logger.log(``Authenticating'')
			return this.password == pwd
\end{lstlisting}
The above example is a variant of our running banking example
where the authenticate method includes an external call to 
\prg{logger.log}. As part of a larger proof of some \Nec spec on \prg{Bank}
we might want to prove that the maximally internal code chunks 
of \prg{authenticate} do not modify the balance. There are two 
maximally internal code chunks: (1) the empty statements preceding 
\prg{logger.log(``Authenticating'')}, and (2) the return statement
following the call. Following the earlier method, we annotate 
\prg{authenticate}:
\begin{lstlisting}[language = Chainmail, frame = lines, mathescape = true]
{PRE: a : Account $\wedge$ a.balance = bal}                // (1)
method authenticate(pwd)
	assert(a.balance = bal)                     // (2)
	logger.log(``Authenticating'')
	assume(a' : Account $\wedge$ a'.balance = bal') // (3)
	return this.password == pwd
{POST: a'.balance = bal'}                    // (4)
\end{lstlisting}
These are clearly provable as there are no heap modifications between
(1) and (2), and (3) and (4). Note: we use different variable names (\prg{a} 
and \prg{a'}) so there is no confusion between the two distinct 
specifications. We are not attempting to prove that \prg{authenticate}
as a whole does not modify the balance of an account, only that the chunks 
do not. In fact it would not be possible to prove that \prg{authenticate}
as a whole does not modify the balance without knowing what \prg{log} does.
For instance, \prg{log} may call \prg{pay}.


\section{Extending a Standard Hoare Logic with Control, Provenance, and Permision Operators for \Nec}

We need to to define Hoare logic rules for code that potentially contains calls or returns to external
code. Specifically, we need to define Hoare logic rules for internal code that ends in either a return
or call to external code, but is otherwise purely internal. Thus, the conclusion of our rules need to be of the following two forms:
$$\proves{M}{\hoare{A}{s; \prg{return } x}{A'}}$$ and
$$\proves{M}{\hoare{A}{s; \blacktriangleright x.m(\overline{y})\blacktriangleleft}{A'}}$$
The first is self explanatory, i.e. some statements (s) followed by a return. The second introduces new
syntax: $\blacktriangleright x.m(\overline{y}) \blacktriangleleft$. This means that the post-condition ($A'$)
holds directly after the method call is dispatched (i.e. in the new frame created by the method call $x.m(\overline{y})$), and not after the method call has completed.
We are not attempting to prove anything about the state of the program after some unknown 
external computation, we are only attempting to prove properties of the program state when entering 
external code from internal code. In fact, it is not clear that we can know much about the 
program state upon completion of $x.m(\overline{y})$ at this low level, for that we need much 
higher level \Nec specifications that consider the entire internal module.

\subsection{Extension to \Nec Assertions}
Before we introduce new rules for \Nec, we must introduce a new assertion form.
If unrestricted external method calls are allowed, we introduce counter-examples to proofs that certain methods 
must have been called. An example is the \prg{transfer} method below, a variant on the \prg{transfer} method 
from the bank account example in the OOPSLA 2022 paper.
\begin{lstlisting}[language = Chainmail, mathescape=true, frame=lines]
method transfer(pwd, acc, amt)
	if this.authenticate(pwd)
		this.logger.log(...)
		balance -= amt
		acc.balance += amt
	return null
\end{lstlisting}
Suppose we wish to prove:
\begin{lstlisting}[language = Chainmail, frame = lines, mathescape = true]
Strans $\triangleq$ a : Account $\wedge$ a.balance = bal
			to a.balance < bal
			onlyThrough _ calls a.transfer(a.password, acc, amt)
\end{lstlisting}
In the OOPSLA 2022 version there was no external call \prg{this.logger.log(...)}, and as a result we could prove that if the 
balance ever decreased then execution must have passed through a program state where \prg{transfer} was called.
This is not true in the \prg{transfer} method above as the initial program state referenced in \prg{Strans} might
refer to an external program state that arises from \prg{this.log.logger(...)}, i.e. midway through the call to \prg{transfer},
and not before the call to transfer. For this reason, we need a broader specification on method calls, an we introduce
the assertion form $\called{\_}{\prg{a}}{\prg{transfer}}{\prg{a.password}, \prg{acc}, \prg{amt}}$ with the following semantics:
\begin{itemize}
\item
$\satisfies{M; (\chi, \psi)}{\called{x}{y}{m}{\overline{z}}}$ iff there exists some $\phi \in \psi$ such that 
\begin{description}
\item[-]
$\phi.(\prg{contn}) = (w := y'.m(\overline{z'}); s)$ for some variables $w, y', \overline{z'}$, and some statement $s$, and
\item[-]
$\interpret{\sigma}{x} = \interpret{\sigma}{\prg{this}}$
\item[-]
$\interpret{\sigma}{y} = \interpret{\sigma}{y'}$
\item[-]
$\interpret{\sigma}{\overline{z}} = \interpret{\sigma}{\overline{z}'}$
\end{description}
\end{itemize}

\subsection{Extension to \Nec Proof System}
\label{s:extension}
Below are some proposed proof rules to extend \Nec so it is able to 
reason about unrestricted external method calls.

\begin{mathpar}
\infer
	{
	\proves{M}
		{\hoare
			{A_1}
			{s}
			{A_2}} \\
	\proves{M}
		{\hoare
			{A_2}
			{s'}
			{A_3}}
	}
	{\proves{M}
		{\hoare
			{A_1}
			{s;s'}
			{A_2}}}
	\quad(\textsc{Composition})
	\and
\infer
	{}
	{\proves{M}
		{\hoare
			{z.f = y}
			{x := z.f}
			{\access{x}{y}}}}
	\quad(\textsc{Read$_1$})
	\and
\infer
	{}
	{\proves{M}
		{\hoare
			{(z.f \neq y \vee x \neq w) \wedge \neg \access{w}{y}}
			{x := z.f}
			{\neg\access{w}{y}}}}
	\quad(\textsc{Read$_2$})
	\and
\infer
	{}
	{\proves{M}
		{\hoare
			{\true}
			{x.f := y}
			{\access{x}{y}}}}
	\quad(\textsc{Write$_1$})
	\and
\infer
	{}
	{\proves{M}
		{\hoare
			{(z \neq y \vee x \neq w) \wedge \neg \access{w}{z}}
			{x.f := y}
			{\neg\access{w}{z}}}}
	\quad(\textsc{Write$_2$})
	\and
\infer
	{}
	{\proves
		{M}
		{\hoare
			{\internal{x}}
			{s}
			{\internal{x}}
		}
	}
	\quad(\textsc{Internal})
	\and
\infer
	{}
	{\proves
		{M}
		{\hoare
			{\external{x}}
			{s}
			{\external{x}}
		}
	}
	\quad(\textsc{External})
	\and
\infer
	{}
	{\proves
		{M}
		{\hoare
			{y \neq z \wedge \neg \access{x}{y}}
			{\onReturn{\return{z}}}
			{\neg \access{x}{y}}
		}
	}
	\quad(\textsc{Ret-Access$_1$})
	\and
\infer
	{}
	{\proves
		{M}
		{\hoare
			{\prg{this} \neq x \wedge \access{x}{y}}
			{\onReturn{\return{z}}}
			{\access{x}{y}}
		}
	}
	\quad(\textsc{Ret-Access$_2$})
	\and
\infer
	{}
	{\proves{M}
		{\hoare
			{e}
			{\onReturn{\return{z}}}
			{e}}}
	\quad(\textsc{Ret-Exp})
	\and
\infer
	{}
	{\proves
		{M}
		{\hoare
			{x = this}
			{\onCall{y.m(\overline{z})}}
			{\called
				{x}
				{y}
				{m}
				{\overline{z}}}}}
	\quad(\textsc{Called$_I$})
	\and
\infer
	{}
	{\proves
		{M}
		{\hoare
			{\called
				{x}
				{y}
				{m}
				{\overline{z}}}
			{x' := y'.f}
			{\called
				{x}
				{y}
				{m}
				{\overline{z}}}}}
	\quad(\textsc{Called-Read})
	\and
\infer
	{}
	{\proves
		{M}
		{\hoare
			{\called
				{x}
				{y}
				{m}
				{\overline{z}}}
			{x'.f := y'}
			{\called
				{x}
				{y}
				{m}
				{\overline{z}}}}}
	\quad(\textsc{Called-Write})
	\and
\infer
	{}
	{\proves
		{M}
		{\hoare
			{\called
				{x}
				{y}
				{m}
				{\overline{z}}}
			{x' := y'.m'(\overline{z}')}
			{\called
				{x}
				{y}
				{m}
				{\overline{z}}}}}
	\quad(\textsc{Called-Call})
	\and
\infer
	{}
	{\proves
		{M}
		{\hoare
			{\called
				{x}
				{y}
				{m}
				{\overline{z}}}
			{\prg{if}(e)\ s1\ \prg{else}\ s2}
			{\called
				{x}
				{y}
				{m}
				{\overline{z}}}}}
	\quad(\textsc{Called-If})
	\and
\infer
	{
	\proves
		{M}
		{
		\hoare
			{A_1 \wedge e}
			{s1}
			{A_2}
		} \\
	\proves
		{M}
		{
		\hoare
			{A_1 \wedge \neg e}
			{s2}
			{A_2}
		}
	}
	{\proves
		{M}
		{\hoare
			{A_1}
			{\prg{if}(e)\ s1\ \prg{else}\ s2}
			{A_2}}}
	\quad(\textsc{If})
	\and
\infer
	{}
	{\proves{M}
		{\hoare
			{y \not\in \overline{z} \wedge \neg \access{x}{y}}
			{\onCall{z_0.m(\overline{z})}}
			{\neg \access{x}{y}}}}
	\quad(\textsc{Call-Access$_1$})
	\and
\infer
	{}
	{\proves{M}
		{\hoare
			{y \in \overline{z}}
			{\onCall{x.m(\overline{z})}}
			{\access{x}{y}}}}
	\quad(\textsc{Call-Access$_2$})
	\and
\infer
	{}
	{\proves{M}
		{\hoare
			{\access{x}{y}}
			{\onCall{z_0.m(\overline{z})}}
			{\access{x}{y}}}}
	\quad(\textsc{Call-Access$_3$})
	\and
\infer
	{}
	{\proves{M}
		{\hoare
			{e}
			{\onCall{z_0.m(\overline{z})}}
			{e}}}
	\quad(\textsc{Call-Exp})
%	\and
%\infer
%	{
%	\proves{M}
%		{\hoare
%			{A_1 \wedge \neg A}
%			{s}
%			{\neg A_2}}
%	}
%	{
%	\proves{M}
%		{\hoare
%			{A_1 \wedge \neg A}
%			{s}
%			{A_2}}
%	}
%	\quad(\textsc{Necessary})
\end{mathpar}
From \textsc{Ret-Access$_1$} and \textsc{Call-Access$_1$} we can prove \textsc{Ret-Inside}, \textsc{Call-Inside$_1$}, and \textsc{Call-Inside$_2$}:
\begin{mathpar}
\infer
	{}
	{\proves
		{M}
		{\hoare
			{x \neq y \wedge \wrapped{x}}
			{\return{y}}
			{\wrapped{x}}}}
	\quad(\textsc{Ret-Inside})
	\and
\infer
	{}
	{\proves
		{M}
		{\hoare
			{x \not\in \overline{z} \wedge \wrapped{x}}
			{\onCall{y.m(\overline{z})}}
			{\wrapped{x}}}}
	\quad(\textsc{Call-Inside$_1$})
	\and
\infer
	{}
	{\proves
		{M}
		{\hoare
			{\internal{y} \wedge \wrapped{x}}
			{\onCall{y.m(\overline{z})}}
			{\wrapped{x}}}}
	\quad(\textsc{Call-Inside$_2$})
%	\and
%\infer
%	{}
%	{\proves
%		{M}
%		{\hoare
%			{\wrapped{x}}
%			{\onReturn{y.m(\overline{z})}}
%			{\wrapped{x}}}}
%	\quad(\textsc{Call-Inside$_3$})
\end{mathpar}

\paragraph{Extended \Nec rules with Unrestricted External Method Calls.}
To help define an extension to \Nec, we introduce the following definition:
\begin{definition}[Maximally Internal Chunks]
For a module $M$, let $S_M$ be the the set of all maximally internal code chunks within $M$.
\end{definition}
We introduce the following rule to \Nec to allow for unrestricted external method calls from internal code.
\begin{mathpar}
\infer
	{
	[\forall s \in S_M, 
		\proves
			{M}
			{\hoare
				{A_1 \wedge \neg A}
				{s}
				{\neg A_2}}] \\
	\proves{M}{A_1 \longrightarrow A_2}\\
	\proves{M}{\givenA{A_1}{\encaps{A}}}
	}
	{\onlyIfSingle{A_1}{A_2}{A}}
	\quad(\textsc{If1-Encaps})
\end{mathpar}
We now remove \textsc{If1-Classical}, \textsc{If1-Inside}, and \textsc{If1-Internal} from the \Nec defined in the OOPSLA 2022 paper as
they are derivable from the rules defined above and are redundant.

\section{Sophia's Password Rewrite Example}
Sophia suggested the following counter example:
\begin{lstlisting}[language = Chainmail, mathescape=true, frame=lines]
module BadBank
	class Password
	class Account                            // bad
		field password : Password
		method hide(ext : Object)
			p := new Password
			ext.call(p)
			this.password := p
			return null
\end{lstlisting}
\begin{lstlisting}[language = Chainmail, mathescape=true, frame=lines]
module GoodBank
	class Password
	class Account                           // good
		field password : Password
		method hide(ext : Object)
			p := new Password
			p' := new Password
			ext.call(p)
			this.pwd := p'
			return null
\end{lstlisting}
Is it possible to write a necessity spec that rejects \prg{BadBank}, and accepts \prg{GoodBank}? The following spec would hopefully do that:
\begin{lstlisting}[language = Chainmail, frame = lines, mathescape = true]
SLeak $\triangleq$ a : Account $\wedge$ $\neg$ <x access a.password>
			next <x access a.password>
			onlyIf false
\end{lstlisting}
To prove this, we need to use \textsc{If1-Encaps}, and look at all of the internal code chunks. 
In both cases, there are 2 maximally internal code chunks:
\begin{description} 
\item[1] the one beginning with the call to \prg{hide}, and ending at the call to \prg{ext.call(p)},
\item[2] the one beginning with the return from \prg{ext.call(p)}, and ending with the return from \prg{hide}
\end{description}
In \textsc{If1-Encaps}, we need to prove that all maximally internal code chunks adhere to the same neccessary preconditions, specifically:
\begin{lstlisting}[language = Chainmail, frame = lines, mathescape = true]
SChunkLeak $\triangleq$ {a : Account $\wedge$ $\neg$ <x access a.password> $\wedge$ $\neg$ (false)}
				s
	    	  {$\neg$ <x access a.password>}
\end{lstlisting}
We are not able to use the Hoare logic extension in \S \ref{s:extension} to do this. The problem is
that we don't have anyway of reasoning about the access gained across \prg{ext.call(p)}. To 
help illustrate the problem, we would be able to prove that the following adhered to our spec:
\begin{lstlisting}[language = Chainmail, mathescape=true, frame=lines]
module GoodBank2
	class Password
	class Account                           // good
		field password : Password
		method hide(ext : Object)
			p := new Password
			ext.call(p)
			p' := new Password
			this.pwd := p'
			return null
\end{lstlisting}
Julian: I think there are 2 ways to approach this:
\begin{enumerate}
\item
design rules that take into account transitive accessibility to construct a restricted heap, and use those to reason about
what access can but more importantly, cannot be gained via an external call.
\item
write a more conservative rule that only reasons that objects that are only accessible in the current local variable map (and not as a field)
cannot be leaked in an external method call.
\end{enumerate}
The second of the two is probably easier to write:
\begin{mathpar}
\infer
	{
	\forall o\ f,\proves{M}{o.f \neq y} \\
	\forall z \in \overline{z}, \proves{M}{z \neq y} \\
	\proves{M}{\prg{ext} \neq y}
	}
	{\proves{M}
			{\hoare
				{\neg \access{x}{y}}
				{\prg{ext.call($\overline{z}$)}}
				{\neg \access{x}{y}}}}
\end{mathpar}
This does not however allow us to reason about more complex use cases, i.e. where we store the
new password as a field of an inaccessible object. Even the first option is perhaps too conservative, specifically, 
what if \prg{ext} has transitive access to the password (perhaps via the current account), 
but that object is secure (as we want to prove our account), then we are not able to prove that 
\prg{ext.call(p)} is still safe.




\section{Specifying Unrestricted External Method Calls}

Our proposed strategy for introducing unrestricted external method calls (and thus potential re-entrant code)
is to specify all ``\emph{maximally internal blocks}''. Consider the following variant of the \prg{Bank Account}
example:
\begin{lstlisting}[language = Chainmail, mathescape=true, frame=lines]
module Logging
  class Logger
  	field logFile : Out
  	method log(s : String)
  		logFile.append(s)
module Bank
	class Password
	class Account
		field logger : Logger
		field balance : int
		field password : Password
		method transfer(pwd, to, amt)
			if this.authenticate(pwd)
				this.balance -= amt
				to.balance += amt
			return null
		method setPassword(pwd, newPwd)
			if this.authenticate(pwd)
				this.password = pwd
			return null
		method authenticate(pwd)
			logger.log(``auth ...'')
			return this.password == pwd
\end{lstlisting}
Here an \prg{Account} object has \prg{transfer}, \prg{setPassword}, and \prg{authenticate} methods.
\prg{authenticate} includes an external call to a logger.
The \emph{maximally internal blocks} of \prg{authenticate} in the above example are:
\begin{lstlisting}[language = Chainmail, frame = lines, mathescape = true]
$\blacktriangleright$ authenticate(pwd)
	$\blacktriangleright$ logger.log(``auth ...'')
\end{lstlisting}
\begin{lstlisting}[language = Chainmail, frame = lines, mathescape]
return this.password == pwd $\blacktriangleleft$
\end{lstlisting}
\prg{transfer} and \prg{setPassword} contain no external method calls, and thus the method bodies are maximally internal.
Suppose we wish to prove that \prg{Bank} satisfies the following specification:
\begin{lstlisting}[language = Chainmail, frame = lines, mathescape = true]
Sbank $\triangleq$ a : Account $\wedge$ inside(a.password)
			next $\neg$inside(a.password)
			onlyIf false
\end{lstlisting}
As a first step to proving the above spec, we need to prove that no maximally internal code chunks leak the password.

\subsection{Proof that \prg{authenticate} adheres to \prg{SBank}}
Here we present a proof that the code chunks that make up \prg{authenticate}
do not leak the password (as the only method with an external method call).\\
\begin{math}
	\textbf{Code Chunk A:}\\
\small
  \begin{array}{llr}
  & \{\prg{a : Account} \wedge \wrapped{\prg{a.password}}\}\\
  \onCall{\prg{a'.authenticate(pwd)}} &   \\
  & \{\wrapped{\prg{a.password}}\} & \textsc{Call-Inside$_1$}\\
  \onCall{\prg{logger.log(``auth ...'')}} &   \\
  & \{\wrapped{\prg{a.password}}\} & \textsc{Call-Inside$_1$}\\
  \end{array}
\end{math}\\
\begin{math}
	\textbf{Code Chunk B:}\\
\small
  \begin{array}{llr}
  & \{\wrapped{\prg{a.password}}\}\\
  \onReturn{\return \prg{this.password == pwd}} &   \\
  & \{\wrapped{\prg{a.password}}\} & \textsc{Ret-Inside}\\
  \end{array}
\end{math}\\
For both proofs, we need to rely on the type system: to prove that the string \prg{``auth ...''} 
is not the password in the first chunk, and to prove that the boolean returned, is not the password 
in the second chunk.



%\subsection{Proof of Step \ref{A}}
%\label{ss:stepA}
%To prove step \ref{A} we need to prove that for all maximally internal chunks $s$ we have
%\begin{lstlisting}[language = Chainmail, frame = lines, mathescape = true]
%PayDecrBal $\triangleq$
%	{a : Account $\wedge$ a.balance = bal $\wedge$ $\neg$ _ calls a.pay(a.password, _, _)
%	$s$
%	{a.balance $\geq$ bal}
%\end{lstlisting}
%There are 5 maximally internal chunks to \prg{Bank}:
%\begin{enumerate}
%\item
%\label{chunk:pay:correctPwd1}
%The branch of \prg{pay} where the password is correct
%\begin{lstlisting}[language = Chainmail, frame  = lines, mathescape = true]
%$\blacktriangleright$a'.pay(pwd, acc, amt)$\blacktriangleleft$
%	if this.authenticate(pwd) // pwd == a'.password
%		balance -= amt
%		$\blacktriangleright$acc.send(amt)$\blacktriangleleft$
%\end{lstlisting}
%\item
%\label{chunk:pay:correctPwd2}
%The remaining statements after the return of \prg{send}  in the \prg{pay} method
%\begin{lstlisting}[language = Chainmail, frame  = lines, mathescape = true]
%return null // the remaining statements after the call to send
%\end{lstlisting}
%\item
%\label{chunk:pay:incorrectPwd}
%The branch of \prg{pay} where the password is not correct
%\begin{lstlisting}[language = Chainmail, frame  = lines, mathescape = true]
%$\blacktriangleright$a'.pay(pwd, acc, amt)$\blacktriangleleft$
%	if this.authenticate(pwd) // pwd != a'.password i.e. we skip the if statement
%		balance -= amt
%		acc.send(amt)
%	return null
%\end{lstlisting}
%\item
%\label{chunk:authenticate}
%The body of authenticate
%\begin{lstlisting}[language = Chainmail, frame  = lines, mathescape = true]
%$\blacktriangleright$a'.authenticate(pwd)$\blacktriangleleft$
%	return this.password == pwd
%\end{lstlisting}
%\item
%\label{chunk:setPassword}
%The body of \prg{setPassword}
%\begin{lstlisting}[language = Chainmail, frame  = lines, mathescape = true]
%$\blacktriangleright$a'.setPassword(pwd, newPwd)$\blacktriangleleft$
%	if this.authenticate(pwd)
%		this.pwd = newPwd
%	return null
%\end{lstlisting}
%\end{enumerate}
%We can prove \prg{PayDecrBal} for all these chunks. It is trivially satisfied by \ref{chunk:pay:correctPwd1}, \ref{chunk:pay:correctPwd2}, 
%and \ref{chunk:pay:incorrectPwd} because $\neg\called{\_}{\prg{a}}{\prg{called}}{\prg{a.password, \_, \_}}$ is not true, and thus 
%the precondition does not hold, and it is satisfied by \ref{chunk:authenticate} and \ref{chunk:setPassword} because they do not modify 
%\prg{a.balance}. To construct these proofs, we use the rules defined in \S \ref{s:extension}. We start by proving that the correct account
%and password are necessary to decrease the balance:
%\begin{lstlisting}[language = Chainmail, frame  = lines, mathescape = true]
%{a : Account $\wedge$ a.balance = bal $\wedge$ (a' $\neq$ a $\vee$ pwd $\neq$ a.password)}
%	$\blacktriangleright$a'.pay(pwd, acc, amt)$\blacktriangleleft$
%		if this.authenticate(pwd) // pwd == a'.password
%			balance -= amt
%			$\blacktriangleright$acc.send(amt)$\blacktriangleleft$
%{a.balance $\geq$ bal}
%\end{lstlisting}
%To construct this proof, we consider both branches of the proof independently. Thus
%by (\textsc{Call-Exp}) and consequence we get:
%\begin{lstlisting}[language = Chainmail, frame  = lines, mathescape = true]
%{a : Account $\wedge$ a.balance = bal $\wedge$ a' $\neq$ a}
%	$\blacktriangleright$a'.pay(pwd, acc, amt)$\blacktriangleleft$
%{a.balance $\geq$ bal $\wedge$ a' $\neq$ a}
%\end{lstlisting}
%by consequence and composition we get:
%\begin{lstlisting}[language = Chainmail, frame  = lines, mathescape = true]
%{a : Account $\wedge$ a.balance = bal $\wedge$ a' $\neq$ a}
%	$\blacktriangleright$a'.pay(pwd, acc, amt)$\blacktriangleleft$
%		if this.authenticate(pwd) // pwd == a'.password
%{a.balance $\geq$ bal $\wedge$ a' $\neq$ a}
%\end{lstlisting}
%by consequence and composition we get:
%\begin{lstlisting}[language = Chainmail, frame  = lines, mathescape = true]
%{a : Account $\wedge$ a.balance = bal $\wedge$ a' $\neq$ a}
%	$\blacktriangleright$a'.pay(pwd, acc, amt)$\blacktriangleleft$
%		if this.authenticate(pwd) // pwd == a'.password
%			balance -= amt
%{a.balance $\geq$ bal $\wedge$ a' $\neq$ a}
%\end{lstlisting}
%by consequence and composition we get:
%\begin{lstlisting}[language = Chainmail, frame  = lines, mathescape = true]
%{a : Account $\wedge$ a.balance = bal $\wedge$ a' $\neq$ a}
%	$\blacktriangleright$a'.pay(pwd, acc, amt)$\blacktriangleleft$
%		if this.authenticate(pwd) // pwd == a'.password
%			balance -= amt
%			$\blacktriangleright$acc.send(amt)$\blacktriangleleft$
%{a.balance $\geq$ bal}
%\end{lstlisting}
%Using the same approach we prove:
%\begin{lstlisting}[language = Chainmail, frame  = lines, mathescape = true]
%{a : Account $\wedge$ a.balance = bal $\wedge$ pwd $\neq$ a.password}
%	$\blacktriangleright$a'.pay(pwd, acc, amt)$\blacktriangleleft$
%		if this.authenticate(pwd) // pwd == a'.password
%			balance -= amt
%			$\blacktriangleright$acc.send(amt)$\blacktriangleleft$
%{a.balance $\geq$ bal}
%\end{lstlisting}
%
%\subsection{Proof of Step \ref{C}}
%The proof of step \ref{C} follows much the same structure as the proof of step \ref{A}. For all maximally internal chunks, we are interested 
%in proving the following specification holds:
%\subsection{Proof of Step \ref{A}}
%To prove step \ref{A} we need to prove that for all maximally internal chunks $s$ we have
%\begin{lstlisting}[language = Chainmail, frame = lines, mathescape = true]
%PwdLeak $\triangleq$
%	{a : Account $\wedge$ inside(a.password)}
%	$s$
%	{inside(a.password)}
%\end{lstlisting}
%Again, we refer to the maximally internal code chunks referred to in \S \ref{ss:stepA}.
%Specifically, none of the code chunks grant access to \prg{a.password}.
 
\section{External Method Call Examples}
\subsection{Call to External, Untrusted Code, Effectively Call-back Free}
\label{s:ex1}
In the example below, the \prg{pay} method makes an external method call \prg{send} to the unknown object \prg{acc}
after to decreasing the \prg{balance} by \prg{amt}. This is \emph{effectively callback free} as the external method 
call occurs at the end of the method, and thus \prg{acc.send(amt)} has no effect on the behaviour of \prg{pay}.
\begin{lstlisting}[language = Chainmail, mathescape=true, frame=lines]
module Bank
	class Password
	class Account
		field balance : int
		field password : Password
		method pay(pwd, acc, amt)
			if this.authenticate(pwd)
				balance -= amt
				acc.send(amt)
			return null
		method setPassword(pwd, newPwd)
			if this.authenticate(pwd)
				this.password = pwd
			return null
		method authenticate(pwd)
			return this.password == pwd
\end{lstlisting}

\subsection{Call to External, Untrusted Code, with a potential for Re-entrancy Attack}
The example below is a variant of the example in \S \ref{s:ex1}, where the call \prg{acc.send(amt)}
occurs before the \prg{balance} is decreased, and thus the account can be exploited by \prg{acc}
using re-entrant calls to \prg{pay} to send more money than is due.
\begin{lstlisting}[language = Chainmail, mathescape=true, frame=lines]
module Bank
	class Password
	class Account
		field balance : int
		field password : Password
		method pay(pwd, acc, amt)
			if this.authenticate(pwd)
				acc.send(amt)
				balance -= amt
			return null
		method setPassword(pwd, newPwd)
			if this.authenticate(pwd)
				this.password = pwd
			return null
		method authenticate(pwd)
			return this.password == pwd
\end{lstlisting}

\subsection{Call to External Code, Nested Within an Internal Call}
The example below includes a \prg{Logger} as a field to \prg{Account}. \prg{Logger}
is externally defined and untrusted. During payment, we pass to logger the amount being paid, 
and the receiver.
\begin{lstlisting}[language = Chainmail, frame = lines]
module Logging
  class Logger
  	field logFile : Out
  	method log(s : String)
  		logFile.append(s)
module Bank
	import Logging
	class Password
	class Account
		field balance : int
		field password : Password
		field logger : Logger
		method pay(pwd, acc, amt)
			if this.authenticate(pwd)
				balance -= amt
				logger.log(''${} paid to {}'', amt, acc)
				acc.send(amt)
				return null
			logger.log(''Authentication Failed.'')
			return null
		method setPassword(pwd, newPwd)
			if this.authenticate(pwd)
				this.password = pwd
		method authenticate(pwd)
			logger.log(``Authenticating'')
			return this.password == pwd
\end{lstlisting}

\subsection{Call to External, Trusted Code}

\begin{lstlisting}[language = Chainmail, frame = lines]
module Logging
  class Logger
  	field logFile : Out
  	method log(s : String)
  		logFile.append(s)
module Bank
	import Logging
	class Password
	class Account
		field balance : int
		field password : Password
		field logger : Logger
		method pay(pwd, acc, amt)
			if this.authenticate(pwd)
				balance -= amt
				logger.log(''${} paid to {}'', amt, acc)
				acc.send(amt)
				return null
			logger.log(''Authentication Failed.'')
			return null
		method setPassword(pwd, newPwd)
			if this.authenticate(pwd)
				this.password = pwd
		method authenticate(pwd)
			return this.password == pwd
\end{lstlisting}




\end{document}
