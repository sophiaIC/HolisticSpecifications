\documentclass[acmsmall,screen]{acmart}
\settopmatter{printfolios=true,printccs=false,printacmref=false}

% Journal information
% Supplied to authors by publisher for camera-ready submission;
% use defaults for review submission.
\acmJournal{PACMPL}
\acmVolume{--}
\acmNumber{--} % {CONF} % CONF = POPL or ICFP or OOPSLA
\acmArticle{} % appears to the left of page number
\acmYear{--}
\acmMonth{13}
\acmDOI{} % \acmDOI{10.1145/nnnnnnn.nnnnnnn}
\startPage{1}

\setcopyright{none}

\bibliographystyle{ACM-Reference-Format}
\citestyle{acmauthoryear}   %% For author/year citations

%%%%%%%%%%%%%%%%%%%%%%%%%%%%%%%%%%%%%%%%%%%%%%%%%%%%%%%%%%%%

%  \citestyle{acmauthoryear}
\usepackage{ amssymb }
\usepackage{stmaryrd}


\usepackage{subcaption}
\usepackage[T1]{fontenc}
\usepackage[utf8]{inputenc}
\usepackage[british]{babel}
\usepackage{xspace, listings, lstcustom, wrapfig, graphicx, enumerate}
\usepackage{paralist}
\usepackage{color,colortbl, relsize}
\usepackage{rotating}
\usepackage{pifont}
\usepackage{multirow}
\usepackage{soul}
\usepackage{tcolorbox}
\usepackage[scaled=.9, light]{zlmtt}
\usepackage{siunitx}
\usepackage{setspace}

 \newcommand{\ttt}{\prg{true}}
\newcommand{\ff}{\prg{false}}
\newcommand{\unkn}{\prg{b???}}
\newcommand{\bv}{\prg{bval}}


\newcommand{\prg}[1]{{\mbox{\tt{#1}}}}
 \newcommand{\prgCol}[1]{#1}

 \newcommand{\forget}[1]{}
\newcommand{\etc}{{\it etc.}}
\newcommand{\eg}{{\it e.g.\,}}
\newcommand{\ie}{{\it i.e.\,}}

\newcommand{\Future}[1]{\lozenge\, #1}% {\bullet #1}% {{{\mathcal F}}(#1)} % {{{\mathcal B}}(#1)}
\newcommand{\Using}[2]{#1\,\kw{in}\, #2} %{{{\mathcal U}}(#1,#2)}
\newcommand{\SigmaUsing}[2]{#1\@ #2} %{{{\mathcal U}}(#1,#2)}
\newcommand{\Past}[1] {\nabla #1} %{\lozenge\!\!\!\!\-\!\!-\,#1}
%{\lozenge\!\!\!\!\!\circ  #1} % {\lozenge\!\!\!\!\-\!\!- #1} %{\upupsilon #1}  %{\nabla #1} %{\circ #1}%  {{{\mathcal P}}(#1)}
\newcommand{\Initial}[1] {{{\mathcal I}\!nitial}(#1)}

\newcommand{\Pol}[1] {{\ensuremath{\prg{Pol}\_{\prg{#1}}}}}
%\newcommand{\MOne} {{\ensuremath{\prg{M}_{\prg{BA}}}}}
%\newcommand{\MTwo} {{\ensuremath{\prg{M}_{\prg{BA}'}}}}

\newcommand{\strongImplies}{\leqq} %{{ \,^\sqsubset\!\!\!_{\sim}\, }}
\newcommand{\weakImplies}{\lessapprox} %{{ \,^\sqsubset\!\!\!_{\sim}\, }}
\newcommand{\frames}{~\kw{frames}~}

\newcommand{\appref}[1]{see App.~\ref{#1}}

%\newcommand{\sE}{{\prg{e}}}

\newcommand{\LangOO} {\ensuremath{{\mathcal L}ang{_{\tt oo}}}}

% ------------------------------------------------------------------
%                                             positions, separations
\newcommand{\cf}{{\it c.f.~}}
%\newcommand{\HYPHENA}{{\em-- }}
%\newcommand{\HYPHENB}{{\em-- }}
%\newcommand{\SP}{{\hspace{.1in}}}
%\newcommand{\s}{{\hspace{.01in}}}

%\newcommand{\obeys}{\,\textbf{\textrm{obeys}}\,}
%\newcommand{\StrongDom}{\ensuremath{\mathcal{S}\textrm{\textit{trong}}{\mathcal{D}}\textrm{\textit{om}}}}
%\newcommand{\Dom}{\ensuremath{\mathcal{D}}\textrm{\textit{om}}}

\newcommand{\Changes}[1]{\ensuremath{\mathcal{C}\textrm{\textit{hanges}}(#1)}}
\newcommand{\VisibleLit}{\ensuremath{\mathcal{V}\textrm{\textit{isible}}}}

\newcommand{\Gives}{\ensuremath{\mathcal{G}\textrm{\textit{ives}}}}
\newcommand{\MayCall}{\ensuremath{\mathcal{M}\textrm{\textit{ay}}{\mathcal{C}}\textrm{\textit{all}}}}
%\newcommand{\Dom}{\ensuremath{\mathcal{D}\textrm{\textit{om}}}}
\newcommand{\MayRead}{\ensuremath{\mathcal{M}\textrm{\textit{ay}}{\mathcal{R}}\textrm{\textit{ead}}}}
\newcommand{\MayAccess}{\ensuremath{\mathcal{M}\textrm{\textit{ay}}{\mathcal{A}}\textrm{\textit{ccess}}}}
\newcommand{\CanAccess}[2]{\ensuremath{{\mathcal{A}}\textrm{\textit{ccess}}}(#1,#2)}
\newcommand{\Calls}[1]{\ensuremath{{\mathcal{C}}\textrm{\textit{alls}}}(\prg{#1})}
\newcommand{\Caller}{\ensuremath{{\mathcal{C}}\textrm{\textit{aller}}}}
%{\ensuremath{\mathcal{C}\textrm{\textit{an}}{\mathcal{A}}\textrm{\textit{ccess}}}(#1,#2)}
\newcommand{\WillAccessThrough}{\ensuremath{\mathcal{W}\textrm{\textit{ill}}{\mathcal{A}}\textrm{\textit{ccess}}{\mathcal{T}}\!\!\textrm{\textit{hrough}}}}
\newcommand{\modelsWithO}{\models\!\!\!\!{_{_{_{\tiny{\mathcal O}}}}}}
\newcommand{\A}{\ensuremath{A}}
\newcommand{\SA}{\ensuremath{{^{\small{\prg{s}}}\! A}}}
\newcommand{\SE}{\ensuremath{{^{\small{\prg{s}}}\!e}}}
\newcommand{\SEOne}{\ensuremath{{^{\small{\prg{s}}}\!e}}}
\newcommand{\SETwo}{\ensuremath{{^{\small{\prg{s}}}\!e'}}}
\newcommand{\B}{\ensuremath{B}}
\newcommand{\Arising}{{\mathcal{A}}\textrm{\textit{rising}}}

 %------------------------ syntax tables

\newcommand{\syntax}[1]{\prg{{\it #1}}}
\newcommand{\BBC}{$::=$} %in syntactic definitions
\newcommand{\SOR}{\ensuremath{\ \mid\ }} % BNF or
\newcommand{\MID}{{\SPsmall ~ \mid ~ \SPsmall }} % in sets


\newcommand{\pre}{\ensuremath{_{{pre}}}}   %kjx no \sc  in math mode
\newcommand{\post}{\ensuremath{_{{post}}}} %kjx no \sc  in math mode
\newcommand{\PRE}{\pre}
\newcommand{\POST}{\post}

%\newcommand{\eval}[2]{{\ensuremath{\langle{ {#1}}\rangle_{#2}}}}
\newcommand{\interp}[2]{{\ensuremath{\lfloor{ {#1}}\rfloor_{#2}}}}
%\newcommand{\interpBL}[1]{{\lceil   {#1}  \rfloor}}
%  { \langle \!\langle { {#1} \rangle \!\rangle}\! }
% { \langle   { {#1}  \rangle} }
% ------------------------------------------------------------------
%                                             English abbreviations

% ------------------------------------------------------------------
%                                              keywords, program text
\newcommand{\kw}[1]{{\bf{\sf {#1}}}}
%{\mbox{\prgCol{\rm{\bf {#1}}}}}
\newcommand{\lit}[1]{{\prg {#1}\xspace}}
\newcommand{\com}{\ensuremath{\prg{//}}}
%\newcommand{\cnf}{\ensuremath{\kappa}}

 \newcommand{\code}[1]{{\mbox{\tt{#1}}}}
\newcommand{\M}{\ensuremath{\prg{M}}}
%\newcommand{\C}{\ensuremath{\prg{C}}}

%\newcommand{\ext}{\mbox{\,\,{\kw {extends}}\,\,}}
%\newcommand{\extends}{\mbox{\,\,{\kw {extends}}\,\,}}
%\newcommand{\ass}{\mbox{{\kw {:=}}\,}}
\newcommand{\semi}{\mbox{{\kw {;}}\ }}
%\newcommand{\comma}{\mbox{{\kw {,}}\,}}
%\newcommand{\dotK}{\mbox{{\kw {.}}}}
%\newcommand{\class}{\mbox{{\kw {class}}\,\,}}
%\newcommand{\stat}{\mbox{{\kw {state}}}}
%\newcommand{\trans}{\mbox{{\kw {root}}}}
%            % {\mbox{{\kw {root state}}}}
%            % was {\mbox{{\kw {abs-state}}}}
\newcommand{\lb}{\prgCol{\mbox{\tt{\bf{\{ }}}}}
\newcommand{\rb}{\prgCol{\mbox{\tt{\bf{\} }}}}}
\newcommand{\lp}{\prgCol{\mbox{\tt{\bf{( }}}}}
\newcommand{\rp}{\prgCol{\mbox{\tt{\bf{) }}}}}
 





\newcommand{\assertTC}[2]{{\M{#1} \typCol{\vdash} {\prg{#2}}\s
    \typCol{\DDiamond_{r} }}}
\newcommand{\assertSC}[2]{{\M{#1} \typCol{\vdash} {\prg{#2}}\s
    \typCol{\DDiamond_{s}} }}




% \newcommand{\M}[1]  {{\ensuremath{\prg{M}{{\prg{#1}}}}}}
    % {\prg{P}}
%\newcommand{\Env}[1]{\envCol{\ensuremath{\Gamma{#1}}}}
% \newcommand{\state}[1]{\stCol{\ensuremath{\sigma???{#1}}}}
%\newcommand{\stackFrame}[1]{\stCol{\ensuremath{\phi{#1}}}}
%\newcommand{\heap}[1]{\stCol{\ensuremath{\chi{#1}}}}

%\newcommand{\expr}[1]{{\ensuremath{\prg{e{#1}}}}}
%\newcommand{\fld}[1]{{\ensuremath{\prg{f{#1}}}}}
%\newcommand{\param}{{\ensuremath{\prg{x}}}}
%
%\newcommand{\clss}[1]{\ensuremath{\prg{c}{\prg{#1}}}}
%\newcommand{\clssD}[1]{\ensuremath{\prg{d}{{#1}}}}
%\newcommand{\type}[1]{\ensuremath{\prg{t{#1}}}}
%
%\newcommand{\val}[1]{{\ensuremath{\prg{v}{\prgCol{#1}}}}}
%\newcommand{\res}[1]{{\ensuremath{\prg{dv}{#1}}}}
%\newcommand{\valOrDev}[1]{{\ensuremath{\prg{r}{#1}}}}


%\newcommand{\expandexp}[1]{}
%
%\newcommand{\oo}{object-oriented}
%\newcommand{\mExtS}{\ensuremath{\Downarrow}}
%
%% re-classification expression
%\newcommand{\cm}[1]{\this{\prgCol{\ensuremath{\mExtS}}}\prg{#1}}






% ------------------------------------------------------------------
%                                             identifiers in the examples
%                     ---------------------
%                                      Stack
 %                     ---------------------
%                                   Employee
\newcommand{\Empl}{\prg {Empl}}
\newcommand{\Pers}{\prg {Pers}}
\newcommand{\E}{{_\prg {e}}}
\newcommand{\Man}{\prg {Boss}}
\newcommand{\Stud}{\prg {Stdt}}
\newcommand{\Scholar}{\prg {Scholar}}
\newcommand{\sal}{\prg {sal}}
\newcommand{\bYear}{\prg {bYear}}
\newcommand{\frnd}{\prg {frnd}}
% \newcommand{\marks}{\prg {marks}}
\newcommand{\pay}{\prg {fee}}
\newcommand{\setP}{\prg {set}}
\newcommand{\String}{\prg {String}}
\newcommand{\hobby}{\prg {hobby}}
\newcommand{\intg}{\kw {int}}
\newcommand{\boolg}{\kw {bool}}
\newcommand{\ii}{\prg {i}}
\newcommand{\prom}{\prg {promote}}
\newcommand{\mkStud}{\prg {study}}
\newcommand{\dem}{\prg {demote}}
\newcommand{\grad}{\prg {employ}}
\newcommand{\assist}{\prg {assist}}
\newcommand{\amount}{\prg {amount}}

\newcommand{\Phil}{\prg {Phil}}
\newcommand{\Book}{\prg {Book}}
\newcommand{\Person}{\prg {Person}}
\newcommand{\fav}{\prg {favourite}}

\newcommand{\aMan}{\prg {billy}}
\newcommand{\anotherMan}{\prg {bob}}
\newcommand{\aMutMan}{\prg {bea}}
\newcommand{\employees}{employees}
\newcommand{\aStud}{\prg {steve}}
\newcommand{\aPers}{\prg {peter}}
\newcommand{\aStudEmpl}{\prg {mary}}
\newcommand{\anoStudEmpl}{\prg {judy}}
\newcommand{\yetAnoStudEmpl}{\prg {betty}}
\newcommand{\yetYetAnoStudEmpl}{\prg {joe}}

\newcommand{\exprog}{_{\prg{ei}}}



% structuring macros
\newcommand{\EndDefLemma}{\noindent $\bigtriangleup$}



\newcommand{\para}[1]{{\ensuremath{\prg{x}{#1}}}}


\newcommand{\methAndBodyNew}{\ensuremath{
\prg{\type{}~m \lp \type{_1}~\param{}\rp\   \ \lb
~\expr{}~\rb} }}
\newcommand{\methAndBodyNewOne}{\ensuremath{
\prg{\type{}~m \lp \type{_1}~\param{}\rp\   \ \lb
~\expr{'}~\rb} }}
\newcommand{\methAndBodyNewTwo}{\ensuremath{
\prg{\type{}~m \lp \type{_1}~\param{}\rp\   \ \lb
~\expr{''}~\rb} }}
\newcommand{\methAndBodyNewThree}{\ensuremath{
\prg{\type{''}~m \lp \type{_1''}~\param{}\rp\   \ \lb
~\expr{''}~\rb} }}
\newcommand{\methAndBodyNewPrime}{\ensuremath{
\prg{\type{}~m \lp  \type{_1}~\para{} \rp\   \
\lb~\expr{'}~\rb} }}
\newcommand{\methAndBodyNewPrimeAll}{\ensuremath{
\prg{\type{'}~m \lp  \type{_1'}~\para{} \rp\   \
\lb~\expr{'}~\rb} }}
\newcommand{\methAndBodyNewP}{\ensuremath{ % slight diff in Primes from above
\prg{\type{}~m \lp  \type{_1}~\para{} \rp\   \
\lb~\expr{'}~\rb} }}
\newcommand{\methAndBodyFromOneSig}{\ensuremath{
\prg{\type{}~m \lp  \type{_1}~\para{} \rp\ \eff{} \lb~ ... ~\rb}
}}

%-----------------

\newcommand{\Body}[2]{\ensuremath{\mathcal{B}ody(#1,\prg{#2})}}

\newcommand{\T}[1]{{\ensuremath{\type{}{#1}}}}

% find the declaration of an identifier
\newcommand{\LookUp}[2]{\ensuremath{ {#1}({\prg{#2}}) }}
\newcommand{\LookUpEnv}[2]{{ \Env{#1}({\prg{#2}}) }}
\newcommand{\LookUpState}[2]{{ \ensuremath{\sigma{#1}}({\prg{#2}}) }}
\newcommand{\LookUpHeap}[2]{{ \ensuremath{\heap{#1}}({\prg{#2}}) }}
\newcommand{\LookUpBody}[2]{{ {\M{#1}}({\prg{#2}}) }}
%\newcommand{\LookUpClass}[2]
%        { \ensuremath{{\mathcal C}%{\mathcal D}%{\it ef}
%        (}\M{#1},\prg{#2}\ensuremath{)}}
\newcommand{\LookUpField}[3]
        {\ensuremath{{\mathcal F}{\mathcal D}% {\it f}
                (}\M{#1},\prg{#2},\prg{#3}\ensuremath{)}}
\newcommand{\LookUpMethod}[3]
        {\ensuremath{{\mathcal M}{\mathcal D}% {\it f}
         (}\M{#1},\prg{#2},\prg{#3}\ensuremath{)}}
\newcommand{\Undef}{{\ensuremath{\mathcal U\!}{\it df}}}
\newcommand{\ClassOf}[2] { \ensuremath{{\mathcal C}{\mathit{lass}}(#1)_{#2}}}

%-------------------Part Three: Types ...
% Notation for types ( open and close types).


\newcommand{\subclass}{\typCol{\sqsubseteq}}
\newcommand{\sclass}{\subclass}
\newcommand{\widen}{\typCol{\ensuremath {\le}}}

%-------------------- implies, and, or, iff, etc -----------------
\newcommand{\inset}[3]{\prg{#1}\!\in\!\prg{#2},\ldots,\prg{#3}}
% do not change above, it is also used with \forall, \exists
%\newcommand{\IMPLIES}{{\SP \Longrightarrow \SP}}
 \newcommand{\AND}{{\SPsmall {\mbox{and}} \SPsmall}}
\newcommand{\WITH}{{\SPsmall {\mbox{with}} \SPsmall}}

%\newcommand{\IFF}{{\SPsmall {\mbox{iff}} \SPsmall}}
\newcommand{\IFF}{{\SP {\mbox{ iff }} \SP}}

\newcommand{\OR}{{\SPsmall {\mbox{or}} \SPsmall}}
\renewcommand{\implies}{{\ensuremath{\longrightarrow}}}
\newcommand{\upd}{{\mapsto}}

%----------------------- Conformance --------------------
\newcommand{\stateConf}[3]
  {\M{#1},\Env{#2}
  \rtjCol{\ensuremath{\vdash}}\,\ensuremath{\state{#3}}\,\rtjCol{\ensuremath{\DDiamond}}  }
\newcommand{\stackFrameHeapConf}[4]
  {\M{#1},\Env{#2}
  \rtjCol{\ensuremath{\vdash}}\,\ensuremath{\stackFrame{#3},\heap{#4}}\,\rtjCol{\ensuremath{\DDiamond}}  }
\newcommand{\stackFrameHeapConfS}[4]
  {\M{#1},\Env{#2}
  \rtjCol{\ensuremath{\vdash}}\,\ensuremath{ {#3},\heap{#4}}\,\rtjCol{\ensuremath{\DDiamond}}  }

\newcommand{\stackFrameHeapConfNOT}[4]
  {\M{#1},\Env{#2}
  \rtjCol{\ensuremath{\not\vdash}}\,\ensuremath{\stackFrame{#3},\heap{#4}}\,\rtjCol{\ensuremath{\DDiamond}}  }

\newcommand{\stackFrameHeapConfLong}[4]
  {\M{#1}, {#2}
  \rtjCol{\ensuremath{\vdash}}\,\ensuremath{\stackFrame{#3},\heap{#4}}\,\rtjCol{\ensuremath{\DDiamond}}  }
\newcommand{\stackFrameHeapConfLongNOT}[4]
  {\M{#1}, {#2}
  \rtjCol{\ensuremath{\not\vdash}}\,\ensuremath{\stackFrame{#3},\heap{#4}}\,\rtjCol{\ensuremath{\DDiamond}}  }


\newcommand{\stateConfEnv}[3]
  {\M{#1}, {#2}
  \rtjCol{\ensuremath{\vdash}}\,\ensuremath{\state{#3}}\,\rtjCol{\ensuremath{\DDiamond}}  }
  \newcommand{\stateConfLong}[3]
  { {#1}, {#2}
        \rtjCol{\ensuremath{\vdash}}\,\ensuremath{\state{#3}}\,\rtjCol{\ensuremath{\DDiamond} } }
\newcommand{\stateConfLongNot}[3]
  { {#1}, {#2}
        \rtjCol{\ensuremath{\vdash}\!\!\!\not}\ \ensuremath{\state{#3}}\,
         \rtjCol{\ensuremath{\DDiamond} } }
\newcommand{\conf}[4]
  {\M{#1},\ensuremath{\heap{#2}}
        \rtjCol{\ensuremath{\vdash}} \ensuremath{\prg{#3}} \rtjCol{\ensuremath{\lhd}}
        \ensuremath{\prg{#4}}}
\newcommand{\confW}[4]
  {\M{#1},\ensuremath{\heap{#2}}
        \rtjCol{\ensuremath{\vdash}} \ensuremath{\prg{#3}} \rtjCol{<:}
        \ensuremath{\prg{#4}}}
% states conforming to each other:


\newcommand{\confState}[5]
    % Prog, eff, thsiMut, state, state
  {\confStateL{\M{#1}}{\eff{#2}}{\clss{#3}}{\state{#4}}{\state{#5}}}
  \newcommand{\confStateP}[4]
    % Prog, eff, state, state
  {\ensuremath{\M{#1},{\eff{#2}}\vdash{\state{#3}}\lhd{\state{#4}}}}
\newcommand{\confStateL}[5]
  {\ensuremath{#1,{#2},{#3}\vdash{#4}\lhd{#5}}}
\newcommand{\confStateLP}[5]
  {\ensuremath{#1,{#2}\vdash{#3}\lhd{#4}}}




% ------------------------------------------- Type assertions ----------------
\newcommand{\assert}[3]{\prg{#1} \vdash \prg{#2}~:~\prg{#3}}
% \newcommand{\assertT}[5] {\M{#1}, \Env{#2} \   \ensuremath{\vdash} \ \prg{#4}~
% \ensuremath{:~\prg{#5}}}
\newcommand{\EFFSEP}{\typCol{[\!]}}

\newcommand{\assertTAll}[6]
    % program, env, expr, type, mut, eff
{\M{#1}, \Env{#2} \   \typCol{\ensuremath{\vdash}} \ \prg{#3}~
\ensuremath{\typCol{:~} \prg{#4} } }% ~\EFFSEP~ {#5}~\EFFSEP~{#6} }}
\newcommand{\assertTAllS}[6]
% with subsumption
    % program, env, expr, type, mut, eff
{\M{#1}, \Env{#2} \   \typCol{\ensuremath{\vdash_s}} \
\prg{#3}~
\ensuremath{\typCol{:~} \prg{#4} } }% ~\EFFSEP~ {#5}~\EFFSEP~{#6} }}
\newcommand{\assertTAllR}[6]
% with subsumption
    % program, env, expr, type, mut, eff
{\M{#1}, \Env{#2} \   \typCol{\ensuremath{\vdash_r}} \
\prg{#3}~
\ensuremath{\typCol{:~} \prg{#4} } }% ~\EFFSEP~ {#5}~\EFFSEP~{#6} }}
\newcommand{\assertTAllL}[6]
    % program, env, expr, type, mut, eff
{{#1}, {#2} \   \typCol{\ensuremath{\vdash}} \ \prg{#3}~
\ensuremath{\typCol{:~}} \prg{#4} }%~\EFFSEP~ {#5}~\EFFSEP~{#6} }}
\newcommand{\assertTFour}[4]
    % for the explanations, ignore P and \eff
{ {#1} \   \typCol{\ensuremath{\vdash}} \ \prg{#2}~
\ensuremath{\typCol{:~} \prg{#3} ~\EFFSEP~ {#4}  }}
% same as above, but in 2 lines!
% do not remove! it is difficult to program, so
% leave it in, even if temporarily not used!
\newcommand{\assertTAllTwoL}[6]
    % program, env, expr, type, mut, eff
{\M{#1}, \Env{#2} \   \ensuremath{\vdash} \ \prg{#3}~
\ensuremath{:~\\ \SP\SP\SP\SP\SP
   \prg{#4} }}
\newcommand{\assertTAllEnvS}[6]
    %as before, but the environment is given in full
{\M{#1}, {#2} \   \typCol{\ensuremath{\vdash}_s} \ \prg{#3}~
\ensuremath{\typCol{:~}}\prg{#4}  }
\newcommand{\assertTAllEnvR}[6]
    %as before, but the environment is given in full
{\M{#1}, {#2} \   \typCol{\ensuremath{\vdash}_r} \ \prg{#3}~
\ensuremath{\typCol{:~}}\prg{#4}  }
\newcommand{\assertTAllEnv}[6]
    %as before, but the environment is given in full
{\M{#1}, {#2} \   \typCol{\ensuremath{\vdash}} \ \prg{#3}~
\ensuremath{\typCol{:~}}\prg{#4}  }




 % Effects


% receiver mutations

\newcommand{\mutOr}[1] % takes as parameter the program
        {\ensuremath{\sqcup\!{_{{#1}}}}} % binary on muts

\newcommand{\assertc}[3]{\ensuremath{\M{#1}
 \typCol{\vdash}\, \prg{#2}\, \subclass\, \prg{#3}}}

 \newcommand{\assertw}[3]{\ensuremath{\M{#1} \typCol{\vdash}
\prg{#2}\widen\prg{#3}}}
\newcommand{\assertion}[3]{\ensuremath{#1 \typCol{\vdash} \prg{#2}~#3}}
\newcommand{\notStateConfEnv}[3]
  {\M{#1},\prg{#2}\,
  \rtjCol{\ensuremath{\not\vdash}}\,\ensuremath{\state{#3}}\,\rtjCol{\ensuremath{\DDiamond}}  }
 \newcommand{\notConf}[4]
  {\M{#1},\ensuremath{\heap{#2}}
        \rtjCol{\ensuremath{\not\vdash}} \ensuremath{\prg{#3}} \rtjCol{\ensuremath{\lhd}}
        \ensuremath{\prg{#4}}}







%Macros for inference rules
\newcommand{\inferencerule}[2]{
\begin{array}{l} #1 \\ \hline #2 \end{array}
}

\newcommand{\inferenceruleN}[3]
{
\begin{array}{l}
% \SP\SP\SP\SP\SP\SP\SP\SP
% \SP\SP\SP\SP\SP\SP\SP\SP
\SP\SP\SP\SP\SP\SP\SP\SP
\SP\SP\SP\SP\SP\SP  {\sf #1}
\\ #2  \\ \hline   #3
  \end{array}
}

\newcommand{\inferenceruleNN}[3]
{
\begin{array}{l}
\SP\SP\SP\SP\SP\SP\SP\SP
\SP\SP\SP\SP\SP\SP\SP\SP
\SP\SP\SP\SP\SP\SP\SP\SP
\SP\SP\SP\SP\SP\SP\SP\SP

   {\sf #1}
\\ #2  \\ \hline   #3
  \end{array}
}

%===========================================================================
%  Definition-Lemma-Theorem-Proof
%
% Adaptation of LaTeX's theorem environment; can be used as a command
% (eg just \Lemma not \begin{Lemma}) and no italicisation; also works
% with ptmac; result numbering is uniform within subsections and can be
% suppressed.
%
\newif\ifNumberResults\NumberResultstrue
\def\@@opargbegintheorem#1#2#3{\@@@@begintheorem{\bf\@@thmname{#1}{#2}(#3)}}
\def\@@begintheorem#1#2{\@@@@begintheorem{\bf\@@thmname{#1}{#2}}}
\def\@@@@begintheorem#1{\par\removelastskip\smallskip\noindent{#1}}
\def\@@thmname#1#2{#1\ \ifNumberResults#2\ \fi}

% similarly \Proof or \begin{Proof}...\end{Proof}
% prefer proofs with statements if possible - hence \penalty700
%\let\qedsymbol\S% make it \square or \blacksquare if you like for kb
\let\qedsymbol \Box
\def\qed{\hfill{$\qedsymbol$}}
\def\Proof{\par\removelastskip\smallskip\penalty700\noindent{\bf Proof}\enskip}
\def\endProof{\qed\penalty-700 \smallskip}
\let\endproof\endProof

%   The actual words

%\newtheorem{theo}{Theorem} this is necessay if we want eg \newtheorem{definition}[theo]{Definition} to work 
\newtheorem{definition}{Definition} %  \newtheorem{definition}[theo]{Definition}
\newtheorem{example}{Example} %  \newtheorem{example}[theo]{Example}
%\newtheorem{lemma}{Lemma} % \newtheorem{mylemma}[theo]{Lemma}
%\renewtheorem{mylemma}{Lemma} % \newtheorem{mylemma}[theo]{Lemma}
\newtheorem{conjecture}{Conjecture} %\newtheorem{conjecture}[theo]{Conjecture}
\newtheorem{theorem}{Theorem} %\newtheorem{theo}{Theorem}
\newtheorem{note}{Note} % \newtheorem{note}[theo]{Note}
\newtheorem{observation}{Observation} %  \newtheorem{observation}[theo]{Observation}


%--------------------------------- the ones that Susan introduced
\newcommand{\z}{{\prg z}}

\newcommand{\Fields}[3]{\ensuremath{{\mathcal F}(}\\Mg{#1},\prg{#2},
\prg{#3}\ensuremath{)} }
\newcommand{\FieldIds}[2]{\ensuremath{{\mathcal F}{\it {s}}(\M{#1},\prg{#2})}}
\newcommand{\Meths}[3]{\ensuremath{{\mathcal M}(}\M{#1},\prg{#2},
\prg{#3}\ensuremath{)} }







 


\newcommand{\rewriteLong}[1]
{\rtCol{\ensuremath{\ \leadsto\!\!\!\!\!\!_{_{#1}\,\ \ }}}}
\newcommand{\rewrite}[1]
{\rtCol{\ensuremath{\ \leadsto\!\!\!\!\!\!_{_{\M{#1}}\,\ \ }}}}
\newcommand{\rewriteP}
    {{\ensuremath{\ \ \leadsto\!\!_{_{\prg{P}}}\,\,\,}}}
 % {\rtCol{\ensuremath{\ \ \leadsto \!\!\!\!_{{\M{}}\,\ \ }}}}
% \newcommand{\Conf}[2]% configurations: < #1, #2 >
% inside the mathmode
% { \langle \prg{#1}, {#2} \rangle }
% {  \prg{#1}\ensuremath{,}{#2}  }
%\newcommand{\greenComma}{\rtCol{\ensuremath{,}}}
%\newcommand{\ConfL}[3]
%{  \prg{#1}\greenComma\stackFrame{#2}\greenComma\heap{#3}  }
%\newcommand{\ConfLLong}[3]
%{  \prg{#1}\greenComma {#2}\greenComma\heap{#3}  }
%\newcommand{\ConfR}[2]
%{  \prg{#1}\greenComma\heap{#2}  }
%

% ------------------------------------- Well formed, unique acyclic ---------------
 
%{\ensuremath{\M{#1} \vdash {\prg{#2}}\s \DDiamond_{\it {eff}}  }}

  \newcommand{\nullPEC}{\lit {nullPntrExc}}



%\newcommand{\Obj}{\prg{Object}}
%\newcommand{\void}{\kw{void}}
%\newcommand{\newK}{{\kw {new}}~} % {{\kw {new}}}% no ~ around it
%\newcommand{\newKW}{~{\kw {new}}~}
%




\newcommand{\back}{{$\!\!\!\!\!\!\!$}}


\newcommand{\WideFig}[3]
{
\begin{figure*}[t]
\begin{center}
\noindent
\fbox{
\begin{minipage}{4.7 in}
{#1} % the contents
\end{minipage}
}
\caption{#2}
\label{#3}
\end{center}
\end{figure*}
}


\newcommand{\WideFigWhere}[4] % you can specify where it should appear!
{
\begin{figure*}[{#4}]
\begin{center}
\noindent
\fbox{
\begin{minipage}{5. in}
{#1} % the contents
\end{minipage}
}
\caption{#2}
\label{#3}
\end{center}
\end{figure*}
}

\newcommand{\BigWideFigWhere}[4] % you can specify where it should appear!
{
\begin{figure*}[{#4}]
\begin{center}
\noindent
{\normalsize
\hrule
\begin{minipage}{5. in}
{#1} % the contents
\end{minipage}
\hrule
}
\caption{#2}
\label{#3}
\end{center}
\end{figure*}
}

\newcommand{\NotTooWideFigWhere}[4] % you can specify where it should appear!
{
\begin{figure*}[{#4}]
\begin{center}
\noindent
\fbox{
\begin{minipage}{4.3 in}
{#1} % the contents
\end{minipage}
}
\caption{#2}
\label{#3}
\end{center}
\end{figure*}
}


\newcommand{\opsemExprFig}
{\BigWideFigWhere {\opsemExpr} {Execution of expressions\MD}
{opsemTrad} {htbp} }



\newcommand{\mlc}{ }%{\heartsuit}
%\newcommand{\mcl}{ }%{\heartsuit}
\newcommand{\mc}{ }%{\heartsuit}

\newcommand{\BigNotTooWideFigWhere}[4] % you can specify where it should appear!
{
\begin{figure*}[{#4}]
\begin{center}
\noindent
{\normalsize
\hrule
\begin{minipage}{4.3 in}
{#1} % the contents
\end{minipage}
\hrule
}
\caption{#2}
\label{#3}
\end{center}
\end{figure*}
}

 

%]})
%}


\usepackage{times}
\usepackage{latexsym}
\usepackage{listings}
\definecolor{dkgreen}{rgb}{0,0.6,0}
\definecolor{gray}{rgb}{0.5,0.5,0.5}
\definecolor{mauve}{rgb}{0.58,0,0.82}

\lstset{ %
  language=Java,                % the language of the code
  mathescape=true,
  basicstyle=\footnotesize\tt,           % the size of the fonts that are used for the code
%  numbers=left,                   % where to put the line-numbers
%  numberstyle=\tiny\color{dkgreen},  % the style that is used for the line-numbers
%  stepnumber=1,                   % the step between two line-numbers. If it's 1, each line
                                  % will be numbered
%  numbersep=5pt,                  % how far the line-numbers are from the code
  backgroundcolor=\color{white},      % choose the background color. You must add \usepackage{color}
  showspaces=false,               % show spaces adding particular underscores
  showstringspaces=false,         % underline spaces within strings
  showtabs=false,                 % show tabs within strings adding particular underscores
%  frame=single,                   % adds a frame around the code
  rulecolor=\color{black},        % if not set, the frame-color may be changed on line-breaks within not-black text (e.g. commens (green here))
  tabsize=2,                      % sets default tabsize to 2 spaces
  captionpos=b,                   % sets the caption-position to bottom
  breaklines=true,                % sets automatic line breaking
  breakatwhitespace=false,        % sets if automatic breaks should only happen at whitespace
  title=\lstname,                   % show the filename of files included with \lstinputlisting;
                                  % also try caption instead of title
  keywordstyle=\color{blue},          % keyword style
  commentstyle=\color{gray},       % comment style
  stringstyle=\color{mauve},         % string literal style
 % escapeinside={\%*}{*)},            % if you want to add LaTeX within your code
  morekeywords={PRE,POST,result,assume,method,mthd,function,fresh,assert,private,then,elseif,public,final,this,throw,new,||,to,def,any,fun,fld,abstract,policy,specification,ghost,field,func}        }  
         % if you want to add more keywords to the set
 


\newcommand{\kjx}[1]{{\color{orange}{KJX: #1}}}
\newcommand{\scd}[1]{{\color{dkgreen}{SD: #1}}}
\newcommand{\sdcomment}[1]{{\ensuremath{\blacksquare}}\footnote{\color{dkgreen}{SD: #1}}}
\newcommand{\secomment}[1]{{\ensuremath{\blacksquare}}\footnote{\se{#1}}}
\newcommand{\jncomment}[1]{{\ensuremath{\blacksquare}}\footnote{\kjx{#1}}}

 \newcommand{\sd}[1]{#1} % {{\color{dkgreen}{#1}}} % {#1} %
\newcommand{\tobyM}[1]{#1} %[1]{{\color{purple}{Toby: #1}}}
\newcommand{\se}[1]{#1} %{{\color{blue}{#1}}}


\newcommand{\ponders}[3]{\marginpar{\tiny\itshape\raggedright\textcolor{#2}{\textbf{#1:} #3}}\ignorespaces}
\marginparwidth=1.6cm \marginparsep=0cm
\newcommand{\TODO}[1]{{\color{red}#1}}
\newcommand{\sophia}[1]{} % {\ponders{Sophia}{dkgreen}{#1}}
\newcommand{\toby}[1]{} % {\ponders{Toby}{purple}{#1}}
\newcommand{\susan}[1]{} %{\ponders{Susan}{blue}{#1}}
\newcommand{\james}[1]{} % {\ponders{James}{orange}{#1}}

\begin{document}

\author{Auhtor}\affiliation{Place}

%\author{Sophia Drossopoulou}\affiliation{Imperial College London}
%
%\author{James Noble}\affiliation{Victoria University of Wellington}
%
%\author{Susan Eisenbach}\affiliation{Imperial College London}


\title{Holistic Specifications for Robust Programs}
\subtitle{version of July 2019}


\begin{abstract}
Functional specifications of program components describe what
%components 
the code \emph{can} do --- the \emph{sufficient} conditions to
invoke the component's operations: 
 a client who supplies arguments
meeting that operation's preconditions can invoke it, % that \sd{behaviour}.
% \sophia{used to say  "invoke that operation" I think it is better to use one term.}
  and obtain the associated effect.
While specifications of sufficient conditions may be enough to reason about %the behaviour of
complete, unchanging  programs, they cannot support reasoning about
 components that interact with external components of possibly unknown provenance. 
In this  \emph{open world} setting, ensuring that your component is robust even when executing 
with buggy or malicious external code is critical.
 \emph{Holistic specifications}
--- as their name implies --- 
are concerned  with the \emph{overall} behaviour of a component,  in all possible 
interleavings of calls to the component's operations with those of the external code.
Thus, holistic specifications are concerned with \emph{sufficient} conditions, \ie
what is enough to \emph{cause} some effect, as well as with \emph{necessary} conditions, \ie
what are the conditions without which an effect will not happen. 
By adopting holistic specification techniques,
programmers can explicitly define what their components should not do,
making it easier to write
%  \sd{code}
% that support the construction of 
robust and reliable programs.

In this paper we argue for the need for such holistic specifications,
 propose a %specification 
 language \Chainmail for writing specifications, give a formal semantics, and discuss several
examples from the literature.
\end{abstract}


\maketitle

\section{Introduction}
 Software guards our secrets, our money, our intellectual property,
our reputation \cite{covern}.  We entrust personal and
corporate information to software which works in an \emph{open} world, 
where  it interacts with 
third party software of unknown provenance, possibly buggy and potentially malicious.

This means we need our software to be \emph{robust}.
We expect software to behave correctly even if  used 
by erroneous or malicious third parties.
 We expect that our bank will only make payments 
from our account if instructed by us, or by somebody we have authorised, 
that space on a web given to an advertiser will not be used
to obtain access to our bank details \cite{cwe}, or that an
airline will not sell more tickets than the number of seats.
The importance of robustness has led to the design of many programming
language mechanisms to help developers write robust programs:
constant fields, private methods, ownership \cite{ownalias}
as well as the object capability paradigm \cite{MillerPhD},
and its adoption in  web systems
\cite{CapJavaHayesAPLAS17,CapNetSocc17Eide,DOCaT14}, and programming languages such as Newspeak
\cite{newspeak17}, Dart \cite{dart15},
Grace \cite{grace,graceClasses}, and Wyvern \cite{wyverncapabilities}.

While such programming language mechanisms make it \textit{possible} to write robust
programs, they cannot \textit{ensure} that programs are robust.
Ensuring robustness is difficult because it means 
different things for different systems: perhaps
that critical operations should only be invoked with the requisite authority;
perhaps that sensitive personal information should not be leaked; 
or perhaps that resource belonging to one user should not be consumed by another.
%
To be able to ensure robustness, we need ways to specify what robustness means for a 
particular program, and ways to demonstrate that the particular program 
adheres to its specific robustness requirements.

 \begin{figure}[htb]
 \begin{tabular}{clclc}
\begin{minipage}{0.32\textwidth}
\begin{lstlisting}
class Wallet{
  fld balance 
  fld secret 
  mthd pay(who,amt,scr){
    if (secret==scr) then 
      balance-=amnt
      who.balance+=amnt   }
  $\mathit{... more\ methods ...}$  
 }
\end{lstlisting}
\end{minipage}
 & \ \ \ & 
\begin{minipage}{0.30\textwidth}
\begin{lstlisting}
class Wallet{
  fld balance  
  fld secret  
  mthd pay(...){
    $\mathit{... as\ version\ 1 ...}$
  }
  mthd sendSecret(who){
      who.take(secret) }
   $\mathit{... more\ methods ...}$  
}
\end{lstlisting}
\end{minipage} & \ \ \ &
\begin{minipage}{0.32\textwidth}
\begin{lstlisting}
class Wallet{
  fld balance  
  fld secret  
  fld owner  
  mthd pay(...){
    $\mathit{... as\ version\ 1 ...}$
  }
  mthd sendSecret( ){
    owner.take(secret)  }  
}
\end{lstlisting}
  \end{minipage}
\\
(a) version 1& & (b) version 2  & & (c)  version 3
 \end{tabular}
  \vspace*{-2.5mm}
  \caption{Three Versions of the class \prg{Wallet}}
 \label{fig:Example}
 \end{figure}

Consider the code snippets from Figure \ref{fig:Example}. It is about objects of
 class \prg{Wallet}, which hold a \prg{balance} and a \prg{secret}, and where only the holder of the secret can use the wallet to
 make payments
 -- for the sake of simplicity, we allow   balances to grow negative.
 We show the code in three versions; each of these versions has the same method \prg{pay}, and the two last versions
 have an additional method \prg{sendSecret}.
  We use a Java-like syntax, and assume an untyped language (as we are in the open world setting).
 Thus, the classical Hoare triple describing the behaviour of \prg{pay} would be:
 
% \begin{figure}[htbp] 
\begin{lstlisting}
   method pay(who,amt,scr)
   PRE:  this,who:Wallet $\wedge$  this$\neq$who $\wedge$ amt:$\mathbb{N}$  $\wedge$  scr=secret   
   POST: this.balance=this.balance$\pre$-amt $\wedge$ who.balance=who.balance$\pre$+amt 
 \end{lstlisting}
%^\end{figure} 
\vspace{-.2in}
The above specification shows that knowledge of the \prg{secret} is a \emph{sufficient} condition to make payments. 
But it does not show that it is a \emph{necessary} condition. To make the specification  
  more ``robust'' we can also describe the behaviour of  unction \prg{pay} when the pre-condition is not satisfied:

%\begin{figure}[htbp] 
\begin{lstlisting}
   method pay(who,amt,scr)
   PRE:  this:Wallet $\wedge$  $\neg$ (this$\neq$who $\wedge$ amt:$\mathbb{N}$  $\wedge$  this.scr=secret  )
   POST: $\forall$w: Wallet.$ \,$w.balance=w.balance$\pre$-amt 
 \end{lstlisting}
%\end{figure} 
\vspace{-.2in}
The specification from above mandates that the method \prg{pay} cannot make a payment unless the secret is
provided, but it cannot preclude that \prg{Wallet} contains more methods which might make the payment without =knowledge of the
secret. To preclude this, we introduce \emph{holistic specifications}, and require that:


(Spec1)\ \  $\triangleq$\ \ $\forall \prg{w}.[\ \ \prg{w}:\prg{Wallet} \wedge \Future{\Changes{\prg{w.balance}}} \ \    
    \longrightarrow \ \    \hfill$ \\
  $\strut \hspace{2.3cm} 
  \exists \prg{o}. [\  \External{\prg{o}} \wedge  \CanAccess{\prg{o}}{\prg{w.secret}}]  \  \ ] \ \ \ \ ] \hfill $



 
 
 
%
Holistic specifications must therefore address sufficient as well as necessary conditions.
In   part (c)    in Fig. \ref{fig:NecessaryAndSuff} we show the 
immediate consequence of putting together assertions from necessary and sufficient conditions: 
there are no transitions from or to yellow boxes.

When a component satisfies its holistic specification,
% used to say "has been specified holistically" but this is not sufficient
then the states  in the yellow boxes and the 
behaviours
represented  by the yellow arrows cannot occur, even when the
component interacts with other software of unknown provenance.
(In Section \ref{sect:discussion} we argue why necessary conditions are more than the complement of
sufficient conditions.

 
Necessary conditions are guarantees upheld throughout program execution.
Therefore, they are close to monitor or object
invariants \cite{Hoare74,Meyer97}. The difference between 
classical invariants and our holistic specifications is that classical invariants  reflect  on
the current program state (\ie the contents of the
stack frame and the heap for an individual program component) while
holistic specifications reflect on all aspects of a program's
execution, potentially across all the components making up that program.


In this paper we propose \Chainmail, a specification language to
express holistic specifications.
The design of \Chainmail was guided by the study of a sequence of
examples from the object-capability literature and the smart contracts world: the
membrane \cite{membranesJavascript}, the DOM \cite{dd,ddd}, the Mint/Purse \cite{MillerPhD}, the Escrow \cite{proxiesECOOP2013}, the DAO \cite{Dao,DaoBug} and
ERC20 \cite{ERC20}.  As we worked through the
examples, we found a small set of language constructs that let us
write holistic specifications across a range of different contexts.
%
While many individual features of \Chainmail can be found in other work, 
their power and novelty for specifying open systems lies in their careful combination.
In particular, \Chainmail extends 
traditional program specification languages \cite{Leavens-etal07,Meyer92} with features which talk about:
%
\begin{description}
\item[Permission: ] 
%\ \ \textbullet \ \emph{Permission}, \ie
Which objects may have access to which other objects; 
this is central since access to an object usually also grants access to the functions it provides.
%
\item[Control: ]
%\ \ \textbullet  \ \emph{Control}, \ie
Which objects called functions on other objects; this
 is useful in identifying the causes of certain effects - eg 
funds can only be reduced if the owner called a payment function.
%
%$\bullet$ \ \\emph{Authority}, \ie which objects' state or properties may change; this is useful in describing effects, such as reduction of funds.
%
\item[Time: ]
%\ \ \textbullet \ \emph{Time}:\ \ie
What holds some time in  the past, the future, and what changes with time,
\item[Space: ]
%\ \ \textbullet \ \emph{Space}:\ \ie
Which parts of the heap are considered when establishing some property, or when 
performing program execution; a concept
related to, but different from, memory footprints and separation logics,
\item[Viewpoint: ]
%\ \ \textbullet \ \emph{Viewpoints}:\ \ie
%a distinction between the objects internal to our component, and those external to it;
Which objects and which configurations are internal to our component, and which  are
external to it;
a concept related to the open world setting.
\end{description}

Holistic assertions usually employ several of these features. They often have the form  of a guarantee
that only if some property holds now will a certain effect occur in the future, or that
certain effects can only be caused if another property held earlier.
For example, if within a certain heap (space) some change is possible in the future (time), then this particular heap 
(space again) contains 
at least one object which has access (permission) to a specific other, privileged object.
%\james{moved around --- not sure we need this para}
%\susan{I think we don't so there is a paragraph I have commented out.}
\forget{Often, holistic assertions typically have the form of a guarantee
that if some property ever holds in the future, then some other property holds now.
For example, if within a certain heap some change is possible in the future, then this particular heap contains 
at least one object which has access to a specific other, privileged object.}
A module satisfies a holistic assertion if  
 the assertion is satisfied  in all runtime configurations reachable through execution of the combined code of our module and any other module.
  This reflects the open-world view.


The contributions of this paper are:
\begin{itemize}
\item the design of the holistic specification language \Chainmail,
\item the semantics of \Chainmail,
\item a validation of \Chainmail through its application to a sequence of examples.
%\item a further validation of \Chainmail through informal proofs of adherence of code to some of these specifications.
\end{itemize}  
  
  
The rest of the paper is organised as follows: Section ....
 
\section{Conclusion}
zzzzz
 
\bibliography{Case,more}

\end{document}
 
 
