%proposed replacement for discussion

%\kjx{could be two subsetions or one section}

\subsection{Examplars}

The design of \Chainmail was guided by the study of a sequence of
exemplars taken from the object-capability literature and the smart
contracts world.  
Space does not permit us to include more example specifications in any
detail in this paper, however 
%
Table~\ref{table} lists four exemplar specifications that we present as
appendices \cite{examples}. Our design was also driven by work on other
examples such as the membrane \cite{membranesJavascript},
the Mint/Purse \cite{MillerPhD}, the Escrow \cite{proxiesECOOP2013,swapsies}.

\begin{table}
  \begin{tabular}{|l|l|}
    \hline
\textbf{Bank} \cite{arnd18} & Bank and Account as in
Section~\ref{sect:motivate:Bank} with two different implementations.\\
\hline
\textbf{ERC20} \cite{ERC20} &   Ethereum-based token contract.\\
\hline
\textbf{DAO} \cite{Dao,DaoBug} & Ethereum contract for Decentralised Autonomous
Organisation.\\
\hline
\textbf{DOM} \cite{dd,ddd} & Restricting access to browser Domain Object Model\\
    \hline
  \end{tabular}
  \caption{Examplar specifications \cite{examples}}
  \label{table}
\end{table}





\subsection{Model}


We have constricted a Coq model of the core of the Chainmail
specification langauge. \ldots


\kjx{catch with this approach is: leaves the obvious question: haven't
  you proved your examples?}    We need enough content in this section
so that it's not the first thing people think to ask

More details of the formal foundations of \Chainmail, and the model,
are also in appendices \cite{examples}.

