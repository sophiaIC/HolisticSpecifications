	\title{Model Outline}	
	
	\documentclass[12pt]{article}
	
	\usepackage{mathpartir}
	\usepackage{amsmath}
	\usepackage{amsthm}
	\usepackage{amssymb}
	\usepackage{color,soul}
	\usepackage{oz}
	\usepackage{listings}
	\usepackage{color}
	\usepackage{csquotes}	
	\usepackage{relsize}

%constrained reduction
\newcommand{\constrained}{\mathrel{\leadsto\ \!\!\!\!{\raisebox{1pt}{$\mathsmaller{\mathsmaller{\mathsmaller{\mathsmaller\rvert}}}$}}}}

	
	\newcommand\trans[1]{\overset{#1}{\mathlarger{\mathlarger \leadsto}}}
	\newcommand\mut[3]{#1\ \texttt{mut}\ #2.#3}
	\newcommand\gives[3]{#1\ \texttt{gives}\ #2\ \texttt{to}\ #3}
	\newcommand\exposes[2]{#1\ \texttt{exposes}\ #2}
	\newcommand\univ{U}
	\newcommand\onlyIf[3]{\{#1\}\ \trans{}\ \{#2\}\ \texttt{only if} [#3]}
	\newcommand\onlyBy[3]{\{#1\}\ \trans{}\ \{#2\}\ \texttt{only by}\ [#3]}
	\newcommand\onlyThrough[3]{\{#1\}\ \trans{}\ \{#2\}\ \texttt{only through}\ [#3]}

	
	\newtheorem{definition}{Definition}
	\newtheorem{theorem}{Theorem}
	\newtheorem{lemma}{Lemma}
	
	\renewcommand\lstlistingname{Quelltext} % Change language of section name
	
	\lstset{ % General setup for the package
		language=Java,
		basicstyle=\scriptsize\sffamily,
		numbers=left,
	 	numberstyle=\tiny,
		frame=tb,
		tabsize=4,
		columns=fixed,
		showstringspaces=false,
		showtabs=false,
		keepspaces,
		morekeywords={field, method, module, calls, presumes, achieves, external, internal, access},
		commentstyle=\color{red},
		keywordstyle=\color{blue},
		deletekeywords={if}
	}
	
	\date{}
	
	\begin{document}
	\maketitle
	
	\section{Syntax}
	\begin{figure}[h]
	$$
	\begin{array}{llr}
	s & ::= &\\ 
	| & \texttt{skip} &\\
	| & \texttt{call}\ \alpha.\texttt{m}(\overline{\texttt{x} \mapsto v}) &\\ 
	| & \texttt{acc}\ \texttt{x}\ v &\\
	| & \texttt{drop}\ \texttt{x} &\\
	| & \texttt{mut}\ \alpha.\texttt{f} := v &\\
	| & \texttt{new}	\ \texttt{C}(\overline{\texttt{f} \mapsto v}) &\\
	| & \texttt{return}\ v
	\end{array}
	\begin{array}{llr}
	p & ::= &\\ 
	| & \texttt{true} &\\
	| & \texttt{false} &\\
	| & i & \textit{where}\ i \in \mathbb{Z}
	\end{array}
	\begin{array}{llr}
	v & ::= &\\ 
	| & p &\\
	| & \alpha\\
	| & \texttt{null}
	\end{array}
	\begin{array}{llr}
	b & ::= &\\ 
	| & \texttt{return}\ v &\\
	| & \texttt{x} := \bigcirc\ ; b&\\
	| & s\ ; b\\
	\end{array}
	$$
	\[
	\phi\ ::=\ (\texttt{this:=}\alpha\; ;\texttt{local:=}\overline{v}\; ; \texttt{contn:=}\{b\})\\
	\psi\ ::=\ \phi\ |\ \phi : \psi\\
	o\ ::=\ (\texttt{class:=C}\; ; \texttt{fields:=}\overline{\texttt{f}\mapsto v})\\
	\chi\ ::=\ \overline{\alpha \mapsto o}\\
	\sigma\ ::=\ (\chi, \psi)\\
	\mathcal{B} : \overline{\texttt{x} \mapsto v}\ \rightarrow\ b \\
	\texttt{CDef}_\texttt{E}\ ::=\ \texttt{class}\ \texttt{C} \{ 
		\overline{\texttt{f}}\; ; 
		\overline{\texttt{m} \mapsto \{ \mathcal{B}\}}
	\}\\
	\mathcal{M} : \sigma\ \rightarrow\ \sigma \\
	\mathcal{H} : \{P\}\ \_\ \{Q\}  \\
	\texttt{CDef}_\texttt{I}\ ::=\ \texttt{class}\ \texttt{C} \{ 
		\overline{\texttt{f}}\; ; 
		\overline{\texttt{m} \mapsto \overline{\mathcal{H}};\ \{\mathcal{M}\}}
	\}\\
	M_{\texttt{E}}\ ::=\ \overline{\texttt{C} \mapsto \texttt{CDef}_\texttt{E}}\\
	M_{\texttt{I}}\ ::=\ \overline{\texttt{C} \mapsto \texttt{CDef}_\texttt{I}}
	\]
	\caption{Syntax}
	\label{f:syntax}
	\end{figure}
	
	
	\section{Operational Semantics}
	
	\begin{figure}[h]
	\hfill \fbox{$M_\texttt{I}\ \fcmp\ M_\texttt{E}\ \bullet\ \sigma_1\ \leadsto\ \sigma_2$}
	\begin{mathpar}
	\infer
		{
		\alpha.\texttt{f},\ v\ \in\ visible((\chi, \psi), (\chi, \psi).\texttt{this}) \\
		(\chi, \psi).\texttt{contn:=}\ \{\texttt{mut}\ \alpha.\texttt{f}\ \texttt{:=}\ v\ ; \ b\}
		}
		{
		M_\texttt{I}\ \fcmp\ M_\texttt{E}\ \bullet\ 
			(\chi, \psi)\ 
			\leadsto\ 
			(\chi[\alpha.\texttt{f} \mapsto v], \psi[\texttt{contn:=\{b\}}])
		}
		\quad (\textsc{Mut})
		\and
	\infer
		{
		\overline{v}\ \in\ visible((\chi, \psi), (\chi, \psi).\texttt{this}) \\
		\phi.\texttt{contn:=}\ \{\texttt{x}\ :=\ \texttt{new}\ \texttt{C}(\overline{f\ \mapsto\ v})\ ; \ b\} \\
		\phi' = \phi[\texttt{local:=}[x \mapsto \alpha]\phi.\texttt{local}\ ;\ \texttt{contn:=}\{b\}] \\
		\alpha\ \textit{fresh in}\ \chi \\
		o = (\texttt{class:=C}\ ;\ \texttt{fields:=}\overline{\texttt{f}\ \mapsto\ v})
		}
		{
		M_1\ \fcmp\ M_2\ \bullet\ 
			(\chi, \phi:\psi)\ 
			\leadsto\ 
			((\chi)[\alpha \mapsto o], \phi\ :\ \psi)
		}
		\quad (\textsc{New})
		\and
	\infer
		{
		\alpha_\texttt{this} = (\chi, \psi).\texttt{this} \\
		\alpha, \overline{v}\ \in\ visible((\chi, \psi), \alpha_\texttt{this}) \\
		classOf((\chi, \psi), \alpha) = \texttt{C} \\
		M_2(C).m = \{b\}\\
		(\chi, \psi)\texttt{.contn} = \{\texttt{x}\ :=\ \texttt{call}\ \alpha.\texttt{m}(\overline{v})\}
		}
		{
		M_1\ \fcmp\ M_2\ \bullet\ 
			(\chi, \psi)\ 
			\leadsto\ 
			(\chi, (\texttt{this:=}\alpha_\texttt{this}\ ;\ \texttt{local:=}\overline{v}\ ;\ \texttt{contn:=}\{b\})\ :\ \psi)
		}
		\quad (\textsc{Call})
		\and
	\infer
		{ 
		\sigma_1 = (\chi, \phi_1\ :\ \phi_2 \ :\ \psi) \\
		v\ \in\ visible(\sigma_1, \sigma_1.\texttt{this}) \\
		\phi_1.\texttt{contn:=}\ \{\texttt{return}\ v\ ; \ b\} \\
		\phi = \phi_2[\texttt{local:=}\phi_2.\texttt{local}\cup\{v\}] 
		}
		{
		M_1\ \fcmp\ M_2\ \bullet\ 
			\sigma_1\ 
			\leadsto\ 
			(\chi, \phi\ :\ \psi)
		}
		\quad (\textsc{Rtrn$^1$})
		\and
	\infer
		{ 
		v\ \in\ visible((\chi, \phi_1\ :\ \phi_2 \ :\ \psi) , \phi_1.\texttt{this}) \\
		\phi_1.\texttt{contn:=}\ \{\texttt{return}\ v\} \\
		\phi = \phi_2[\texttt{local:=}[\texttt{x}\ \mapsto\ v]\phi_2.\texttt{local}] 
		}
		{
		M_1\ \fcmp\ M_2\ \bullet\ 
			(\chi, \phi_1\ :\ \phi_2 \ :\ \psi) \ 
			\leadsto\ 
			(\chi, \phi\ :\ \psi)
		}
		\quad (\textsc{Rtrn$^2$})
	\end{mathpar}
	\caption{Pair Reduction.}
	\label{f:pair_reduction}
	\end{figure}
	
	
	\section{External and Internal Steps}
	
	Pair reduction consists of a series of either external or internal steps. That is any pair reduction step of the form
	$$M_1\ \fcmp\ M_2\ \bullet\ \sigma_1\ \leadsto\ \sigma_2$$
	is either
	\begin{enumerate}
	\item
	An external step: i.e. a single reduction step purely between external frames. Only external objects can be modified in an external step.
	\item
	an internal step: a series of internal steps between two external frames that takes one of the following forms:
	\begin{enumerate}
	\item \texttt{call-call}: a method call on an internal object that in its execution makes an external method call. The resulting stack introduces a continuous block of internal frames  to the top of the stack.
	\item \texttt{call-return}: a method call on an internal object that returns without making any external method calls. 
	The resulting stack is more or less equal to to the initial stack (not including modifications to the local environment and continuation of the top frame).
	\item \texttt{return-call}: a return from a prior internal to external method call, followed by another method call. The resulting stack is  a single external frame on top of a continuous block of internal frames.
	\item \texttt{return-return}: a return from a prior internal to external method call, followed by a return from a prior external to internal method call. The resulting stack removes a continuous block of internal frames from the stack.
	\end{enumerate}
	\end{enumerate}
	
	The distinction between internal and external steps is important because it determines which objects in the heap might be modified during a particular pair reduction step, and what kind of internal reduction step is important because it is required to determine to which possible internal execution path might be 
	
	
	\section{Changes}
	
	In proving satisfaction of different specifications, it is useful to investigate when different portions of the heap is modified.
	Many specifications assert that certain portions of the heap can only be modified under certain conditions, and this is often 
	done using temporal operators. As a simple example, the bank account example from the paper:
	 
	  \vspace{.01in}
	(BankSpec)\ \  $\triangleq$\\ 
	$\forall \texttt{a}.[\ \ \texttt{a}:\texttt{Account} \wedge \texttt{changes}\langle{\texttt{a.balance}}\rangle  \ \    
	    \longrightarrow \ \    \hfill$ \\
	  $\strut \hspace{2.3cm} 
	% TODO explain:
	% we no longer need Past here, as we are ion visible states 
	  \exists \texttt{o}. [\    \texttt{o}\ \texttt{calls}\ {\texttt{a.}}{\texttt{deposit}}({\_,\_}) \vee\  \texttt{o}\ \texttt{calls}\ {\_.}{\texttt{deposit}}({\texttt{a},\_})\rangle\  \ ] \ \ \ \ ] \hfill $
	\vspace{.05in}
	
	Currently there is no \texttt{changes} assertion in  chainmail, however one might encode \texttt{changes} using \texttt{next}:
	$$\texttt{changes}\langle \texttt{x.f} \rangle = \exists \texttt{i}. [\texttt{x.f} = i\ \wedge\ \texttt{next}\langle \texttt{x.f}\neq \texttt{i} \rangle]$$
	The above encoding does not generally capture \texttt{changes} because \texttt{next} has no concept of the future if 
	the current configuration is a method return statement. This does not technically matter for \texttt{changes} because 
	method return does not modify fields. This arises more problematically if we need to assert that a reduction step 
	changes \texttt{access}, as this *can* be changed by method return (i.e. via the local enviroment).
	
	Of the 4 kinds of internal steps, a version of \texttt{changes} that is defined using \texttt{next} can only 
	capture internal steps of the form of (a) and (b), and not internal steps of the form of (c) and (d).
	
	\subsection{Hoare Logic}
	Define a Hoare triple as 
	$$M_1\ \fcmp\ M_	2, \sigma_0 \vDash \{A_1\}\ \sigma\ \{ A_2\}$$
	That is, for modules $M_1$ and $M_2$, initial configuration $\sigma_0$, current configuration $\sigma$, and assertions $A_1$ and $A_2$, 
	if $A_1$ is satisfied before reduction of $\sigma$ then $A_2$ will be satisfied afterward.
	
	Using such a hoare triple, we can define \texttt{changes} as such:
	$$M_1\ \fcmp\ M_	2, \sigma_0, \sigma \vDash \texttt{changes}\langle A \rangle\ \equiv\ M_1\ \fcmp\ M_2, \sigma_0 \vDash \{A\}\ \sigma\ \{ \neg A\}$$
	In most cases (I think, ..., probably, ..., not sure) we are concerned with changes that require an internal step, and not an external one. For this purpose, 
	it is helpful if we assume that the internal module $M_1$ is fully specified with classical specs. That is, for every possible pass through internal code 
	(of the forms listed (a), (b), (c), and (d) above) that internal step is fully specified with regards to modifications to the heap. In the bank account example
	this means that we are able to identify all possible internal methods that might modify the balance of an account, and thus determine that the only 
	possible step that might result in such a change is a call to \texttt{deposit}.
	
	\section{Examples}
	
	We classify pair reduction steps in the following way.
	\begin{definition}
	\begin{mathpar}
	\infer
		{(M_1\ \circ\ M_2)\ \bullet\ \sigma\ \leadsto\ \sigma_1\ \leadsto\ \ldots\ 
		\leadsto\ \sigma_n\ \leadsto\ \sigma'\\
		(\forall \sigma_i.\ M_1\ \fcmp\ M_2, \sigma_0, \sigma_i\ \vDash \texttt{internal}\langle \sigma_i.(\texttt{self}) \rangle)\\
		M_2, \sigma_0, \sigma\ \vDash \texttt{external}\langle \sigma.(\texttt{self}) \rangle\\
		M_2, \sigma_0, \sigma'\ \vDash \texttt{external}\langle \sigma'.(\texttt{self}) \rangle}
		{M_1\ \fcmp\ M_2\ \bullet\ \sigma\ \leadsto^\texttt{int}\ \sigma'}
		\quad(\textsc{Int})
		\and
	\infer
		{(M_1\ \circ\ M_2)\ \bullet\ \sigma\ \leadsto\ \sigma'\\
		M_2, \sigma_0, \sigma\ \vDash \texttt{external}\langle \sigma.(\texttt{self}) \rangle\\
		M_2, \sigma_0, \sigma'\ \vDash \texttt{external}\langle \sigma'.(\texttt{self}) \rangle}
		{M_1\ \fcmp\ M_2\ \bullet\ \sigma\ \leadsto^\texttt{ext}\ \sigma'}
		\quad(\textsc{Ext})
	\end{mathpar}
	\end{definition}
	\begin{definition}
	$$M_1\ \fcmp\ M_2\ \bullet\ \sigma\ \leadsto\ \sigma'$$
	if and only if
	\begin{itemize}
	\item
	$M_1\ \fcmp\ M_2\ \bullet\ \sigma\ \leadsto^\texttt{int}\ \sigma'$ or
	\item
	$M_1\ \fcmp\ M_2\ \bullet\ \sigma\ \leadsto^\texttt{ext}\ \sigma'$
	\end{itemize}
	\end{definition}
	
	Next we define some useful properties.
	
	$$\texttt{changes}\langle A \rangle\ \equiv\ A\ \wedge\ \texttt{next}\langle \neg A \rangle$$
	$$\texttt{changed}\langle A \rangle\ \equiv\ \neg A\ \wedge\ \texttt{prev}\langle A \rangle$$
	
	\begin{mathpar}
	\infer
		{
		\sigma\ \vDash\ \texttt{changes}\langle A \rangle
		}
		{
		\sigma\ \vDash\ \texttt{next}\langle\texttt{changed}\langle A \rangle \rangle
		}
		\and
	\infer
		{
		\sigma\ \vDash\ A \\
		\sigma\ \vDash\ \texttt{will}\langle \neg A \rangle
		}
		{
		\sigma\ \vDash\ \texttt{will}\langle\texttt{changed}\langle A \rangle\rangle
		}
		\and
	\infer
		{
		\sigma\ \vDash\ A_1\ \longrightarrow\ A_2 \\
		\sigma\ \vDash\ \texttt{will}\langle A_1 \rangle
		}
		{
		\sigma\ \vDash\ \texttt{will}\langle A_2 \rangle
		}
		\and
	\infer
		{
		\sigma\ \vDash\ \texttt{changed}\langle A \rangle
		}
		{
		\sigma\ \vDash\ \texttt{prev}\langle A \rangle
		}
		\and
	\infer
		{
		\sigma\ \vDash\ \texttt{changed}\langle A \rangle
		}
		{
		\sigma\ \vDash\ \neg A
		}
	\end{mathpar}
	
	
	\begin{definition}
	For an initial configuration $\sigma_0$, internal module $M_1$, and external module $M_2$, we define the following:
	\begin{itemize}
	\item
	$\Sigma(\sigma_0) = \{\sigma_0\} \cup\ \{\sigma | M_1\ \fcmp\ M_2\ \bullet\ \sigma_0 \leadsto^*\ \sigma\}$
	\item
	$\mathcal{L}(\sigma) = \{p | p\ \text{is primitive}\ \wedge\ \sigma\ \in\ \Sigma(\sigma_0)\ \wedge\ M_1\ \fcmp\ M_2\ \bullet\ \sigma_0, \sigma \vDash p\}$
	\item
	$\mathcal{H} = \{\ \{P\}\ \_\ \{Q\}\ |\ \sigma\ \sigma'\ \in\ \Sigma(\sigma_0)\ \wedge\ M_1\ \fcmp\ M_2\ \bullet\ \sigma\ \leadsto^\texttt{int}\ \sigma'\ \longrightarrow\ P(\sigma)\ \wedge\ Q(\sigma') \}$
	\end{itemize}
	\end{definition}
	
	\begin{definition}
	We define the set of internally changeable assertions as
	\begin{mathpar}
	\infer
		{}
		{v\ \in\ \mathcal{P}(\sigma)}
		\and
	\infer
		{e_1, e_2\ \in\ \mathcal{P}(\sigma)}
		{e_1\ =\ e_2\ \in\ \mathcal{P}(\sigma)}
		\and
	\infer
		{\sigma\ \vDash\texttt{internal}\langle \alpha \rangle}
		{\alpha.f \in\ \mathcal{P}(\sigma)}
		\and
	\infer
		{\sigma\ \vDash\texttt{internal}\langle \alpha \rangle}
		{\texttt{access}\langle \alpha, \_ \rangle \in\ \mathcal{P}(\sigma)}
		\and
	\infer
		{A\ \in\ \mathcal{P}(\sigma)}
		{\neg A \in\ \mathcal{P}(\sigma)}
		\and
	\infer
		{A_1, A_2\ \in\ \mathcal{P}(\sigma)}
		{A_1\ \wedge\ A_2 \in\ \mathcal{P}(\sigma)}
		\and
	\infer
		{A_1, A_2\ \in\ \mathcal{P}(\sigma)}
		{A_1\ \vee\ A_2 \in\ \mathcal{P}(\sigma)}
	\end{mathpar}
	\end{definition}
	Finally we state the following properties.
	\begin{mathpar}
	\infer
		{
		\sigma\ \vDash\ A \\ 
		\sigma\ \leadsto^\texttt{ex}\ \sigma' \\
		A\ \in\ \mathcal{P}(\sigma)
		}
		{
		\sigma'\ \vDash\ A
		}
		\and
	\infer
		{
		\sigma\ \vDash\ A \\
		\sigma'\ \vDash\ \neg A \\
		A\ \in\ \mathcal{P}(\sigma)
		}
		{
		\sigma\ \leadsto^\texttt{int}\ \sigma'
		}
		\and
	\infer
		{
		\sigma\ \leadsto^\texttt{int}\ \sigma'
		}
		{
		\exists\ P\ Q. [\{P\}\_\{Q\}\in\ \mathcal{H}\ \wedge\ P (\sigma)\ \wedge\ Q(\sigma')]
		}
	\end{mathpar}
	
	\subsection{Bank Account Example}
	
	\begin{lstlisting}[mathescape=true]
	class Account{
		field balance
		method deposit(from, amt){
			...
		}Pre:  $\exists$ b$_1$, this.balance = b$_1$
		       $\exists$ b$_2$, from.balance = b$_2$
		 Post: this.balance = b$_1$ + amt
		       from.balance = b$_2$ + amt
	}
	\end{lstlisting}
	 
	  \vspace{.01in}
	(BankSpec)\ \  $\triangleq$\\ 
	$\forall \texttt{a}.[\ \ \texttt{a}:\texttt{Account} \wedge \texttt{changes}\langle{\texttt{a.balance}}\rangle  \ \    
	    \longrightarrow \ \    \hfill$ \\
	  $\strut \hspace{2.3cm} 
	% TODO explain:
	% we no longer need Past here, as we are ion visible states 
	  \exists \texttt{o}. [\    \texttt{o}\ \texttt{calls}\ {\texttt{a.}}{\texttt{deposit}}({\_,\_}) \vee\  \texttt{o}\ \texttt{calls}\ {\_.}{\texttt{deposit}}({\texttt{a},\_})\rangle\  \ ] \ \ \ \ ] \hfill $
	\vspace{.05in}
	
	\noindent
	let the current configuration be $\sigma$. Thus we have
	$$(\texttt{a.balance}\ = \ \texttt{b})\ \in\ \mathcal{P}(\sigma)$$ and 
	$$\sigma \vDash \texttt{changes}\langle\texttt{a.balance}\ = \ \texttt{b})\rangle$$
	It follows that $\sigma$ must be an internal step, and that there must be a specified 
	internal path that modifies \texttt{a.balance}.
	It follows that the only possible path is a \texttt{deposit} call to either \texttt{a}
	or some other \texttt{Account} with \texttt{a} as the first argument.  
	
	%\subsection{Safe Example}
	%In the example below we assume that \texttt{secret} cannot be internal to another safe.
	% 
	%  \vspace{.01in}
	%(SafeSpec)\ \  $\triangleq$\\ 
	%$\forall \texttt{s}.[\ \ \texttt{s}:\texttt{Safe} \wedge\ \neg \texttt{s.treasure = null}\ \wedge\ \texttt{will}\langle \texttt{s.treasure = null} \rangle  \ \    
	%    \longrightarrow \ \    \hfill$ \\
	%  $\strut \hspace{2.3cm} 
	%  \exists \texttt{o}. [\    \texttt{access}\langle \texttt{o}, \texttt{s.secret} \rangle\  \ ] \ \ \ \ ] \hfill $
	%\vspace{.05in}
	%
	%$$\neg \texttt{s.treasure = null}$$
	%$$\texttt{will}\langle\neg \neg \texttt{s.treasure = null}\rangle$$
	%$$\texttt{will}\langle \texttt{changed}\langle \neg \texttt{s.treasure = null} \rangle \rangle$$
	%$$\texttt{will}\langle \texttt{changed}\langle \neg \texttt{s.treasure = null} \rangle \rangle$$
	
	
	
	\newpage

	
	\section{LTL Model of Chainmail}
	
	Following the section on LTL of Michael Huth and Mark Ryan, I am attempting to build
	a Chainmail model for checking satisfaction of Chainmail assertions. This diverges some what 
	from that book, especially in two ways:
	\begin{enumerate}
	\item
		The model is not generally finite as there may be an infinite number of program configurations. However, in proving satisfaction 
		we should only need to consider a finite number (is this true?).
	\item
		Assertions in Chainmail are not just about the current program configurations, but also about the transitions, i.e. method calls/returns etc.
		In fact I am trying to explicitly capture a way of modeling these transformations, and thus developing a standard way of proving satisfaction
		for code that includes external calls (i.e. something like the DAO).
	\end{enumerate}
	
	Firstly, state in the model is captured by program configurations, and transitions between configurations 
	is captured by pair reduction steps. That is, either a transition is a single reduction between external program
	configurations, or a series of steps through internal code.
	
	
	To model internal steps in the underlying language, we define transitions through internal code as a set of 
	functions mapping configurations to configurations:
	
	\begin{definition}[Module Transition functions]
	For a module $M$, all potential execution paths that consist of only module code
	can be represented by a finite set of transformations $\pi_M$, such that for any 
	external program configuration $\sigma$ of external module $M^\prime$ that either calls a method defined in $M$,
	or returns a value to a method in $M$, there exists $p\ \in\ \pi_M$ such that
	$$M\ \fcmp\ M^\prime\ \bullet\ \sigma\ \leadsto\ p(\sigma)$$
	\end{definition}
	
	It is in fact simple to check whether any particular program configuration makes a specific 
	module transition: i.e. if the configuration makes an internal method call, or if the 
	configuration is a method return, and the underlying frame is an internal frame.
	For this reason, we extend the above definition of a module transition path to include a predicate $R$
	that specifies how $p$ is called (or returned to), along with pre- and post- conditions.
	$$R\ :\ \{P\}\ p \{Q\}$$
	
	In the case of a method call, $R$ would resemble $x\ \texttt{calls}\ \texttt{y}.\texttt{m}(\ldots) $.
	In the case of a method return, $R$ would look something like $x\ \texttt{returns}\ v\ \texttt{to}\ y$.
	(should internal frames just have transition functions instead of a normal continuation?)
	
	\begin{definition}[Chainmail Model]
	A transition system $\mathcal{M} = (M_1, M_2, \sigma_0, \Sigma, L)$ 
	is a module pair $M_1$ and $M_2$,
	an initial program configuration $\sigma_0$,
	a set of program configurations $\Sigma$ such that
	$$\Sigma\ =\ \{ \sigma \vert\ M_1\ \fcmp\ M_2\ \bullet\ \sigma_0\ \leadsto^*\ \sigma \}\ \cup\ \{\sigma_0\} $$
	and a labelling function
	$L : \Sigma \mapsto P(Atoms)$.
	\end{definition}
	
	
	
	
	\begin{figure}[h]
	\[
	\begin{array}{lcr}
	\begin{array}{llr}
	a & ::= &\textit{Atoms}\\ 
	| & e & \\
	| & e\ :\ C & \\
	| & e\ \in\ S & \\
	| & \langle x\ \texttt{access}\ y \rangle &\\
	| & \texttt{internal}\langle x \rangle &\\
	| & \texttt{external}\langle y \rangle &\\
	| & \langle x\ \texttt{makes step}\ p \rangle & \\
	\end{array}&
	~~~~~~~~~~~~~~~ &
	\begin{array}{llr}
	A & ::= & \textit{Assertions}\\ 
	| & a & \\
	| & \neg A & \\
	| & A\ \wedge\ A & \\
	| & A\ \vee\ A & \\
	| & A\ \longrightarrow\ A & \\
	| & \forall\ x.\ A & \\
	| & \exists\ x.\ A & \\
	| & \forall\ S.\ A & \\
	| & \exists\ S.\ A & \\
	| & \texttt{next}\langle  A \rangle& \\
	| & \texttt{will}\langle  A \rangle& \\
	| & \texttt{prev}\langle  A \rangle& \\
	| & \texttt{was}\langle  A \rangle& 
	\end{array}
	\end{array}
	\]
	\caption{Assertions}
	\label{f:assertions}
	\end{figure}
	
	Fig. \ref{f:assertions} gives a syntax for Chainmail assertions.
	The main difference between the assertions of Fig. \ref{f:assertions}, and those of FASE is the 
	replacement of $\langle\texttt{x calls y.m(...)}\rangle$ with $\langle x\ \texttt{makes step}\ p\rangle$.
	
	\begin{definition}[Satisfaction of internal path step]
	$\sigma$ satisfies $\langle x\ \texttt{makes step}\ p \rangle$ if and only if there exists $R$, $P$, and $Q$ such that $R\ :\ \{P\}p\ \{Q\} \in\ \pi_M$ and
	$(R\ \sigma)$ holds.
	\end{definition}
	
	The second difference is the distinction between atomic assertions and general assertions.
	Atomic assertions are simple irreducible assertions.
	Satisfaction of atomic assertions can be determined in the absence of an initial configuration as they do not include temporal operators,
	thus we introduce an alternate form for satisfaction:
	$$M_1 \fcmp M_2,\ \sigma_1\ \vDash\ a$$
	Transitions between program configurations are defined using pair reduction:
	
	\begin{mathpar}
	\infer
		{
		(M_1 \circ M_2)\ \bullet\ \sigma_1\ \leadsto\ \sigma_2 \\
		classOf\ \sigma_1.(\texttt{this})\ \in M_2 \\
		classOf\ \sigma_2.(\texttt{this})\ \in M_2
		}
		{
		M_1 \fcmp M_2\ \bullet\ \sigma_1\ \leadsto\ \sigma_2
		}
		\and
	\infer
		{
		M_1 \fcmp M_2,\ \sigma_1\ \vDash\ \langle x\ \texttt{makes step}\ p\rangle
		}
		{
		M_1 \fcmp M_2\ \bullet\ \sigma\ \leadsto\ p(\sigma)
		}
	\end{mathpar}
	
	\subsection{Module Code without External Calls}
	
	In cases were modules do not include external calls, 
	module transition paths are relatively simple, and
	are just method calls. In this case, the syntax of Fig. \ref{f:assertions}
	becomes that of FASE.
	
	\subsection{Module Code with External Calls}

	
	\subsection{Changes Requiring Internal Reduction Steps}
	
	The main reason for broadening the chainmail expression of 
	an internal step to include not only method calls, but also  
	method returns is to be able to refer these boundaries inside of
	chainmail, and not just as part of some meta-reasoning. Here I
	connect the $\langle x \texttt{makes step}\ p\rangle$ to the 
	earlier reasoning about  \texttt{changed}.
	There are two categories of changes that require internal reduction steps:
	
	\begin{enumerate}
	\item Internal Mutation:
	\begin{mathpar}
	\infer
		{
		\exists \ x\ v. [\texttt{changed}\langle x.f\ =\ v \rangle\ \wedge\ 
		\langle x\ \texttt{internal} \rangle
		}
		{\exists\ y\ p, \langle y\ \texttt{makes step}\ p\rangle}
	\end{mathpar}
	\item Internal Encapsulation:
	\begin{mathpar}
	\infer
		{\exists \ x\ y. [\texttt{changed}\langle \neg \langle x\ \texttt{access}\ y\rangle \rangle\ \wedge\ \texttt{prev}\langle 
		\forall z. [\neg \langle z\ \texttt{access}\ y \rangle\ \vee\ \langle \texttt{internal}\ z \rangle] 
		\rangle ]}
		{\exists\ x'\ p, \langle x'\ \texttt{makes step}\ p\rangle}
	\end{mathpar}
	\end{enumerate}
	
	All other changes could occur as part of external client code.
	
	
	\subsubsection{Safe}
	
	As an example, consider a simplified version of the Safe specification:
	\begin{mathpar}
	\infer
		{\forall\ \texttt{s}, [\texttt{s}\ :\ \texttt{Safe}\ \wedge\ \texttt{will}\langle \texttt{change}\langle \texttt{s.treasure} \rangle\rangle]\ \longrightarrow\ 
		\texttt{will}\langle \exists o. [\langle o\ \texttt{access}\ \texttt{s.secret} \rangle] \rangle}
		{}
	\end{mathpar}
	\begin{mathpar}
	\infer
		{\forall\ \texttt{s}, [\texttt{s}\ :\ \texttt{Safe}\ \wedge\ \texttt{will}\langle \texttt{changed}\langle \texttt{s.treasure} \rangle\rangle]\ \longrightarrow\ 
		\texttt{will}\langle \exists o. [\langle o\ \texttt{access}\ \texttt{s.secret} \rangle] \rangle}
		{}
	\end{mathpar}
	\begin{mathpar}
	\infer
		{\forall\ \texttt{s}, [\texttt{s}\ :\ \texttt{Safe}\ \wedge\ \texttt{will}\langle \texttt{changed}\langle \texttt{s.treasure} \rangle\ \wedge\ \langle \texttt{s}\ \texttt{internal}\rangle\rangle]\ \longrightarrow}
		{}
	\end{mathpar}
	\begin{mathpar}
	\infer
		{\texttt{will}\langle \exists o. [\langle o\ \texttt{access}\ \texttt{s.secret} \rangle] \rangle}
		{}
	\end{mathpar}
	\begin{mathpar}
	\infer
		{\forall\ \texttt{s}, [\texttt{s}\ :\ \texttt{Safe}\ \wedge\ \texttt{will}\langle \texttt{changed}\langle \texttt{s.treasure} \rangle\ \wedge\ \langle \texttt{s}\ \texttt{internal}\rangle\rangle]\ \longrightarrow}
		{}
	\end{mathpar}
	\begin{mathpar}
	\infer
		{\texttt{will}\langle \exists o. [\langle o\ \texttt{access}\ \texttt{s.secret} \rangle] \rangle}
		{}
	\end{mathpar}
	
	


	
	\newpage
	
	\section{Hoare Logic and Reverse Hoare Logic}
	
	In order to simplify proof construction for satisfaction of Chainmail assertions we define a logic for reasoning about single
	internal steps for a module. To do this I make use of traditional Hoare triples, along with the reverse Hoare triples defined by 
	de Vries and Koutavas.
	
	In Chainmail we are not concerned with the effects that a statement or specific method has on the program state, but rather 
	what effects an entire module might have on the program state. As an intermediate step to constructing proofs about full Chainmail
	it is useful to be able to prove simpler specifications on a module, and the ways it might change the program state. That is, if we were able
	to prove limited specifications on how a single execution path through only module-internal code modifies 
	program state, it would assist in the proof of more general Chainmail specifications. Below I present a Hoare-like triple that 
	places a module in the middle rather that a single statement to be evaluated.
	$$\{A_1\}\ M\ \{A_2\}$$
	$A_1$ and $A_2$ are Chainmail assertions, and $M$ denotes some module. Thus, the above specification can be read as
	\begin{displayquote}
	\textit{for any purely module-internal execution path through $M$, if $A_1$ is true beforehand, $A_2$ will be true afterward}
	\end{displayquote}
	
	We are also not only concerned with constructing proofs about
	programs in a traditional forward style. We don't always need to reason about the program state after the execution
	of a program based on what is true before, but also the inverse, i.e. we often need to infer the program state before the 
	execution of a program based on the program state after evaluation. For this reason it is useful to include specifications
	of the form of de Vries and Koutavas' Reverse Hoare Logic.
	$$[A_1]\ M\ [A_2]$$
	Above I use the notation used by O'Hearn in his related paper on Incorrectness Logic, and the above specification can be read as
	\begin{displayquote}
	\textit{for any purely module-internal execution path through $M$, if $A_2$ is true afterward, $A_1$ will be true before}
	\end{displayquote}
	This is not exactly how de Vries and Koutavas phrase it. They might rather say that $A_1$ represents the a super set of the
	states from which $A_2$ is reachable. This is in contrast to saying that for a traditional Hoare triple $\{A_1\}\ M\ \{A_2\}$,
	$A_2$ represents a super set of the states that are reachable from $A_1$.
	
	It should be noted that these definitions can not be encoded in Chainmail currently. In fact the following equivalences do not generally hold:
	$$\{A_1\}\ M\ \{A_2\}\ \not\equiv\ M\ \vDash\ \texttt{prev}\langle A_1 \rangle\ \longrightarrow\ A_2$$
	$$[A_1]\ M\ [A_2]\ \not\equiv\ M\ \vDash\ A_2\ \longrightarrow\ \texttt{prev}\langle A_1 \rangle$$
	This is because the left hand sides of the above equivalences imply only internal steps, while satisfaction of the right hand sides 
	is determined for all possible steps, including external steps. While the above does not hold, the following do hold:
	$$M\ \vDash\ \texttt{prev}\langle A_1 \rangle\ \longrightarrow\ A_2\ \Rightarrow\ \{A_1\}\ M\ \{A_2\}$$
	$$M\ \vDash\ A_2\ \longrightarrow\ \texttt{prev}\langle A_1 \rangle\ \Rightarrow\ [A_1]\ M\ [A_2]$$
	The above hold because Chainmail satisfaction considers all potential pair reduction steps, including internal steps.
	
	Another note is that there does not seem to be anything stopping us from writing quite complex Chainmail specifications 
	in the above form. For this reason, if we actually intend these specifications to be a simpler, intermediate step then they 
	must be simpler. Thus, we restrict specifications to only those concerned with a single internal step. This is shown in Figure \ref{f:specs}.
	
	\begin{figure}[h]
	\footnotesize
	\[
	\begin{array}{lll}
	\begin{array}{llr}
	R & ::= &\textit{Simple Assertion}\\ 
	| & a & \\
	| & \neg R & \\
	| & R\ \wedge\ R & \\
	| & R\ \vee\ R & \\
	| & R\ \longrightarrow\ R & \\
	| & \forall\ x.\ R & \\
	| & \exists\ x.\ R & \\
	| & \forall\ S.\ R & \\
	| & \exists\ S.\ R & \\
	\end{array}&
	\begin{array}{llr}
	P & ::= &\textit{Pre}\\ 
	| & R & \\
	| & \texttt{next}\langle  R \rangle& \\
	| & \texttt{changes}\langle  R \rangle& \\
	| & \neg P & \\
	| & P\ \wedge\ P & \\
	| & P\ \vee\ P & \\
	| & P\ \longrightarrow\ P & \\
	| & \forall\ x.\ P & \\
	| & \exists\ x.\ P & \\
	| & \forall\ S.\ P & \\
	| & \exists\ S.\ P & \\
	\end{array}&
	\begin{array}{llr}
	Q & ::= & \textit{Post}\\ 
	| & R & \\
	| & \texttt{prev}\langle  R \rangle& \\
	| & \texttt{changed}\langle  R \rangle& \\
	| & \neg Q & \\
	| & Q\ \wedge\ Q & \\
	| & Q\ \vee\ Q & \\
	| & Q\ \longrightarrow\ Q & \\
	| & \forall\ x.\ Q & \\
	| & \exists\ x.\ Q & \\
	| & \forall\ S.\ Q & \\
	| & \exists\ S.\ Q & \\
	\end{array}
	\end{array}
	\]
	\normalsize
	\caption{Specification Syntax}
	\label{f:specs}
	\end{figure}
	
		I use $a$ to represent \emph{atomic} assertions as indicated in Figure \ref{f:assertions}, with the single modification that I replace $ \langle x\ \texttt{makes step}\ p \rangle$ 
		with  $\langle x\ \texttt{calls}\ y.m(x_1, ...) \rangle$, in line with the definition of chainmail specifications in the FASE 2020 paper. Further, the only temporal operators that may 
		be used are ones that refer to the other side of the triple, i.e. preconditions may use \texttt{next} or \texttt{changes} to refer to the state after evaluation, and postconditions 
		may use \texttt{prev} or \texttt{changed} to refer to the program state before evaluation.
		This syntax is chosen as satisfaction of assertions of this form can be checked by considering only those program states that abut an internal execution.
		
		We require specifications on modules take the following syntactic forms:
	$$\{P\}\ M\ \{Q\}$$
	$$[P]\ M\ [Q]$$

	\subsection{Examples}
	
	\subsubsection{Bank Account}

	
	In the following code examples I borrow O'Hearn's keywords \texttt{presumes} and \texttt{achieves} to indicate 
	the distinction between what is presumed to be true before execution, and what is achieved after execution.

	
	Below is the definition for the module \texttt{Bank}. \texttt{Bank} contains a single class, \texttt{Account}, which has a 
	\texttt{balance} field and a \texttt{deposit} method. \texttt{deposit} transfers a positive amount (\texttt{amt}) from some 
	account (\texttt{from}) 	to the reciever account. If \texttt{amt} is negative, or \texttt{from.balance} is less than or equal 
	to \texttt{0}, then neither balance is changed.
\begin{lstlisting}[mathescape=true]
module Bank {
	class Account {
		{presumes: $\exists$ x y acc.[ $\langle$ x calls this.deposit(acc, y) $\rangle$ $\vee$ 
							   $\langle$ x calls acc.deposit(this, y) $\rangle$ ]
		 achieves: $\exists$ b.[ changed$\langle$ this.balance = b $\rangle$ ]}
		 
		{presumes: x access this
		 achieves: x access this}
	
		field balance
		method deposit(from, amt) {
			...
		}
}
\end{lstlisting}
	As part of the definition of \texttt{Account}, I include some specifications in the form of \texttt{presumes} and \texttt{achieves}. The first can be
	read as: if an internal \texttt{Bank} execution step is made, and 
	$$\texttt{$\exists$ b.[ changed$\langle$ this.balance = b $\rangle$ ]} $$
	is achieved, then 
	$$\texttt{$\exists$ x y acc.[$\langle$ x calls this.deposit(acc, y) $\rangle$ $\vee$ $\langle$ x calls acc.deposit(this, y) $\rangle$]}$$
	was true beforehand. In other words, 
	\begin{displayquote}
	if the balance of an account changes then it was as a result of a call to \texttt{deposit} where the changed
	account was either the reciever or the \texttt{from} account
	\end{displayquote}
	I place the specifications within the \texttt{Account} class only so we might use the \texttt{this} keyword. A similar specification
	could be written outside of \texttt{Account} by replaceing \texttt{this} with \texttt{acc'} and including \texttt{acc' : Account} as part of the specification.
	
	The second states that all \texttt{Bank} operations do not grant access to \texttt{Account}s.
	
	The Chainmail specifications for the \texttt{Bank} as they are written in the TOSEM paper are as follows:
	
\begin{lstlisting}[mathescape=true]
(1) $\triangleq$ $\forall$ a. [ a : Account $\wedge$ changes$\langle$ a.balance $\rangle$ $\longrightarrow$
               $\exists$ o.[ $\langle$ o calls a.deposit(_, _) $\rangle$ $\vee$
                     $\langle$ o calls _.deposit(a, _) $\rangle$ ] ] 
                     
(2) $\triangleq$ $\forall$ a. $\forall$ S : Set. [ a : Account $\wedge$ $\langle$ will $\langle$ changes$\langle$ a.balance $\rangle\rangle$ in S $\rangle$ 
                           $\longrightarrow$
               $\exists$ o.[ o $\in$ S $\wedge$ external$\langle$ o $\rangle$ $\wedge$ $\langle$ o access a $\rangle$] ] 
                     
(3) $\triangleq$ $\forall$ a. $\forall$ S : Set. [ a : Account $\wedge$ $\langle$ will $\langle$ $\exists$o.[ external$\langle$ o $\rangle$ $\wedge$ $\langle$ o access a $\rangle$ ] $\rangle$ in S $\rangle$ 
                           $\longrightarrow$
               $\exists$ o'.[ o' $\in$ S $\wedge$ external$\langle$ o' $\rangle$ $\wedge$ $\langle$ o' access a $\rangle$] ] 
\end{lstlisting}

\texttt{(1)}
The first of the the three specifications follows simply from the presumption and result of the \texttt{Account} class.

\texttt{(2)}
The second specification uses the following rule for reverse Hoare logic:
\begin{mathpar}
\infer
	{
	P \Rightarrow P' \\
	[P]\ M\ [Q]
	}
	{
	[P']\ M\ [Q]
	}
\end{mathpar}
Since a change to the \texttt{balance} of \texttt{a} implies a call involving \texttt{a}, and 
a call involving \texttt{a} implies \texttt{access} to \texttt{a}, it follows that there must exist
some \texttt{o} with \texttt{access} to \texttt{a}.
Satisfaction then follows the argument that either there is some current \texttt{external} object that has \texttt{access} 
to the \texttt{a}, or \texttt{Bank} can give access to \texttt{a}. Following the second specification, we
know that the latter cannot be true, and we arrive at the result.

\texttt{(3)}
The proof of \texttt{(3)} follows a similar argument to that of \texttt{(2)}. 

\newpage

\section{Modelleing Temporal Assertions and their Implied Effects in Chainmail}

In order to develop a logic that can reason about the temporal 
aspects of Chainmail at a level above the details of execution paths 
of L$_{oo}$, I examine the possibility of using effects.

In Figure \ref{f:effects} I define the possible effects in L$_{oo}$ that might occur to the heap 
through the execution of a program.	
\begin{figure}[h]
\footnotesize
\[
\begin{array}{llr}
	\epsilon & ::= &\textit{Effects}\\ 
	| & \texttt{x mut y.f} & \\
	| & \texttt{x gives y to z} & \\
\end{array}
\]
\normalsize
\caption{Effects Syntax}
\label{f:effects}
\end{figure}
For any step of execution, an object might be mutated by another object, or access to an object might
be granted to another object by some third object. These effects are associated with some execution
step. i.e. either field access, method call, or method return are required for a \texttt{give} effect, while 
field assignment is required for a \texttt{mut} effect. Technically, the other effect  is the construction 
of new objects, but I won't deal with this yet.

For any program configuration with internal module $M_1$ and external module $M_2$ 
there is a set of possible effects. While it is not possible to define this set, is possible to 
approximate it if $M_1$ is known. Informally that is, the possible effects in the execution of any 
program configuration is a subset of 
$$E_{M_1}\ =\ E_\texttt{mut}\ \cup\ E_\texttt{gives}\ \cup\ E_{\texttt{Int}}$$
where $E_\texttt{mut}$ is the set of all possible mutations on external objects,
$E_\texttt{gives}$ is the set of all possible effects where one external object gives
 access to some object to another external object, and $E_{\texttt{Int}}$ is the set of all internal
 effects possible via $M_1$. To simplify matters, we can assume that all methods 
 defined by $M_1$ are free from external calls.

\begin{figure}[h]
\begin{mathpar}
\infer
	{}
	{
	M, \sigma \ \vdash\ \texttt{x.f := \_ }\ :\ \sigma.(\texttt{this})\ \texttt{mut}\ \texttt{x.f}
	}
	\and
\infer
	{
	[\texttt{x.f}\ \mapsto\ y] \in \sigma
	}
	{
	M, \sigma \ \vdash\ \texttt{\_ := x.f }\ :\ \sigma.(\texttt{this})\ \texttt{gives y to}\ \sigma.(\texttt{this})
	}
	\and
\infer
	{}
	{
	M, \sigma \ \vdash\ \texttt{\_ := x.m(..., y, ..)}\ :\ \sigma.(\texttt{this})\ \texttt{gives y to x}
	}
	\and
\infer
	{}
	{
	M, (\chi, \phi_1 : \phi_2 : \psi) \ \vdash\ \texttt{return x}\ :\ \phi_1.(\texttt{this})\ \texttt{gives x to}\ \phi_2.(\texttt{this})
	}
	\and
\infer
	{
	classOf\ \sigma\ \texttt{x}\ \texttt{C} \\
	\texttt{x := y.m(args)}\ :\ \texttt{y mut z.f}\ \in\ M.(\texttt{C})
	}
	{
	M, \sigma \ \vdash\ \texttt{x := y.m(args)}\ :\ \texttt{y mut z.f}
	}
	\and
\infer
	{
	classOf\ \sigma\ \texttt{x}\ \texttt{C} \\
	\texttt{x := y.m(args)}\ :\ \texttt{y gives z$_1$ to z$_2$}\ \in\ M.(\texttt{C})
	}
	{
	M, \sigma \ \vdash\ \texttt{x := y.m(args)}\ :\ \texttt{y gives z$_1$ to z$_2$}
	}
\end{mathpar} 
\caption{Statement Effects}
\label{f:effectAssign}
\end{figure}

Figure \ref{f:effectAssign} covers the assignment of effects to statements.
This is designed in the context of two module execution, and thus all
internal calls are treated as atomic, and effects are given per method.
All the effects are unsurprising, 
\begin{itemize}
\item
assignments mutate the heap, 
\item
field accesses, external method calls, and external method returns give access to external objects from other external objects
\item
effects of internal methods are defined by the internal module
\end{itemize}
Note, in Figure \ref{f:effectAssign}, only internal operations may expose or mutate internal data. 
Generally we are only interested in the last two rules for this reason.

We can also represent the relationships implied by Chainmail temporal assertions in the following manner:
$$A_1\ \trans{E}\ A_2$$
Where $A_1$ is some assertion that is true in the current moment, and $A_2$
is some assertion that is true in some moment in the future, and
$E_1$ is the set of effects on the heap necessary for those two assertions to be true.

For some internal module $M$, Chainmail proofs of satisfaction often take one of the following forms:
\begin{enumerate}
\item
From $A_1$ and $E_M$ we want to prove $A_2$ will necessarily hold:\\
eg. the secret to a safe will never change (assuming there is no internal method to change it)
\item
From $A_2$ and $E_M$ we want to prove $A_1$ does necessarily hold\\
eg. if an external object will have access to a bank account in the future, then there must be an external access with access to that account now.
\item
From $A_1$ and $A_2$ we want to prove that $E$ was necessary to go from $A_1$ to $A_2$\\
eg. if the balance of our account ever changes in the future, then there must have been an intervening execution step that caused 
that change
\end{enumerate}
It is (3) that the rest of this section is concerned with, i.e. identifying necessary effects that must occur for certain 
assertions to be true. Note, knowing the necessary effects is not enough to prove satisfaction of Chainmail 
assertions, generally we want to know which internal methods are implied by those effects. For 
example, we want to use the effects implied by a change in balance to prove that it was 
\texttt{deposit} that was called. 

\begin{figure}[t]
\small
\begin{mathpar}
\infer
	{}
	{
	\texttt{x.f}\ =\ \texttt{y}\ \trans{\{\texttt{\_ mut x.f}\}} \ \texttt{x.f}\ \neq\ \texttt{y}
	}
	\and
\infer
	{}
	{
	\neg \langle \texttt{x}\ \texttt{access}\ \texttt{y} \rangle\ \trans{ \{\texttt{\_ gives y to x}\}}\ \langle \texttt{x}\ \texttt{access}\ \texttt{y} \rangle
	}
	\and
\infer
	{
	A_1\ \trans{\{\texttt{\_ mut x.f}\}} \ A_2
	}
	{
	\texttt{internal}\langle x \rangle\ \wedge\ A_1\ \trans{\{M\ \texttt{mut x.f}\}} \ A_2
	}
	\and
\infer
	{
	A_1\ \trans{ \{\_\ \texttt{gives y to x}\}}\ A_2
	}
	{
	\forall \texttt{z}.[\texttt{internal} \langle \texttt{z} \rangle\ \vee\ \neg\ \langle\texttt{z}\ \texttt{access}\ \texttt{y} \rangle] \ \wedge\ A_1\ \trans{ \{M\ \texttt{gives y to x}\}}\ A_2
	}
	\and
\infer
	{
	A\ \trans{E}\ A_1 \\
	A\ \trans{E^\prime}\ A_2
	}
	{
	A\ \trans{E\ \cup\ E^\prime}\ A_1\ \wedge\ A_2
	}
	\and
\infer
	{
	A_1\ \trans{E}\ A \\
	A_2\ \trans{E^\prime}\ A
	}
	{
	A_1\ \wedge\ A_2\ \trans{E\ \cup\ E^\prime}\ A
	}
	\and
\infer
	{
	A\ \trans{E}\ A_1 \\
	A\ \trans{E^\prime}\ A_2
	}
	{
	A\ \trans{E\ \cap\ E^\prime}\ A_1\ \vee\ A_2
	}
	\and
\infer
	{
	A_1\ \trans{E}\ A \\
	A_2\ \trans{E^\prime}\ A
	}
	{
	A_1\ \vee\ A_2\ \trans{E\ \cap\ E^\prime}\ A
	}
	\and
\infer
	{
	A_1\ \longrightarrow\ A \\
	A\ \trans{E}\ A_2
	}
	{
	A_1\ \trans{E}\ A_2
	}
	\and
\infer
	{
	A_2\ \longrightarrow\ A \\
	A_1\ \trans{E}\ A
	}
	{
	A_1\ \trans{E}\  A_2
	}
	\and
%\infer
%	{
%	A_1\ \trans{E}\ A_2\\
%	A_1\ \trans{E^\prime}\ A_2
%	}
%	{
%	A_1\ \trans{E\ \cup\ E^\prime}\ A_2
%	}
\end{mathpar} 
\caption{Deriving Necessary Effects From Temporal Assertions}
\label{f:effectInference}
\end{figure}

Figure \ref{f:effectInference} defines rules for proving necessary effects. We read $A_1\ \trans{E}\ A_2$,
as \emph{``any path starting at a program state $A_1$ and ending at a program state $A_2$ must have at minimum the
resulting effects $E$''}. Thus $E$ is subset of the effects that might occur in the transition from $A_1$ to $A_2$. 
Note that we extend the effect syntax to include $M\ \texttt{mut x.f}$ and $M\ \texttt{gives x to y}$ to indicate 
that the effect must happen internal to the module $M$.

\paragraph{Bank Account Example:} We revisit the \texttt{Bank Account} example from the last section:
\begin{lstlisting}[mathescape=true]
module Bank {
	class Account {
	
		field balance
		method deposit(from, amt) : {this mut this.balance, 
		                             this mut from.balance} {
			...
		}
}
\end{lstlisting}
We address the first specification:
\begin{lstlisting}[mathescape=true]
(1) $\triangleq$ $\forall$ a. [ a : Account $\wedge$ changes$\langle$ a.balance $\rangle$ $\longrightarrow$
               $\exists$ o.[ $\langle$ o calls a.deposit(_, _) $\rangle$ $\vee$
                     $\langle$ o calls _.deposit(a, _) $\rangle$ ] ] 
\end{lstlisting}
$\texttt{changes}\langle \texttt{a.balance} \rangle$ can be rewritten as  such
\begin{lstlisting}[mathescape=true] 
changes $\langle$ a.balance $\rangle$ = $\exists$ x. [a.balance = x $\wedge$ next$\langle$ a.balance $\neq$ x $\rangle$]
\end{lstlisting}
Using the first rule from Figure \ref{f:effectInference} we get 
$$\texttt{a.balance}\ =\ \texttt{x}\ \trans{\{\texttt{\_ mut a.balance}\}} \ \texttt{a.balance}\ \neq\ \texttt{x}$$
Since \texttt{a : Account}, we are able to show that 
$$\texttt{internal}\langle \texttt{x} \rangle$$
and subsequently by the third rule in Figure \ref{f:effectInference} we get
$$\texttt{internal}\langle x \rangle\ \wedge\  \texttt{a.balance}\ =\ \texttt{x}\ \trans{\{\texttt{Bank mut a.balance}\}} \ \texttt{a.balance}\ \neq\ \texttt{x}$$
It then follows that there must be some call to \texttt{deposit}.

\paragraph{Safe Example:} The safe example features a \texttt{Safe} class defined in the following way:
\begin{lstlisting}[mathescape=true]
module SafeModule {
	class Safe {
	
		field secret
		field treasure
		
		method take(scr) : this mut this.treasure {
			...
		}
}
\end{lstlisting}
A safe contains a \texttt{secret}, a \texttt{treasure}, and a method \texttt{take} that allows for the removal 
of the \texttt{treasure} if the caller knows the \texttt{secret}. We specify \texttt{SafeModule} with the following holistic specification:
\begin{lstlisting}[mathescape=true]
(SafeSpec) $\triangleq$ $\forall$ s. [ s : Safe $\wedge$ s.treasure $\neq$ null $\wedge$ wil$\langle$ s.treasue = null $\rangle$ $\longrightarrow$
               $\exists$ o.[ external $\langle$ s, o $\rangle$ $\wedge$ $\langle$ o access s.secret $\rangle$ ] ] 
\end{lstlisting}
This specification does not hold in the case where \texttt{internal} and \texttt{external} are defined module wide, 
but rather when they are defined per data structure. We can prove satisfaction we modify the rules in Figure \ref{f:effectInference} with the following two
rules:
\begin{mathpar}
\infer
	{
	A_1\ \trans{\{\texttt{\_ mut y.f}\}} \ A_2
	}
	{
	\texttt{internal}\langle \texttt{x}, \texttt{y} \rangle\ \wedge\ A_1\ \trans{\{\texttt{x mut y.f}\}} \ A_2
	}
	\and
\infer
	{
	A_1\ \trans{ \{\_\ \texttt{gives y to x}\}}\ A_2
	}
	{
	\forall \texttt{o}.[\texttt{internal} \langle \texttt{z}, \texttt{o} \rangle\ \vee\ \neg\ \langle\texttt{o}\ \texttt{access}\ \texttt{y} \rangle] \ \wedge\ A_1\ \trans{ \{\texttt{z gives y to x}\}}\ A_2
	}
\end{mathpar}
These two rules above correspond to the datastructure notion of \texttt{internal} and external.
Now we can proceed by first noting that the first rule in Figure \ref{f:effectInference} gives us
$$\texttt{s.treasure}\ \neq\ \texttt{null}\ \trans{\{\texttt{\_ mut s.treasure}\}} \ \texttt{s.treasure}\ =\ \texttt{null}$$
then by noting that trivially \texttt{internal $\langle$s, s$\rangle$} and the first rule directly above
$$\texttt{internal}\langle \texttt{s}, \texttt{s} \rangle \wedge \texttt{s.treasure}\ \neq\ \texttt{null}\ \trans{\{\texttt{s mut s.treasure}\}} \ \texttt{s.treasure}\ =\ \texttt{null}$$
for \texttt{s mut s.treasure} to be a necessary effect we must have 
$$\texttt{internal}\langle \texttt{s}, \texttt{s} \rangle \wedge \texttt{s.treasure}\ \neq\ \texttt{null}\ \trans{\{\}} \ \texttt{o calls s.take(s.secret)}$$
Further, we can show that 
$$\texttt{internal}\langle \texttt{s}, \texttt{s} \rangle \wedge \texttt{s.treasure}\ \neq\ \texttt{null}\ \trans{\{\}}\ \ \langle\texttt{o access s.secret}\rangle$$
From here we need rules that haven't been written yet, i.e. we need to infer that given $\langle\texttt{o access s.secret}\rangle$, and the set of 
\emph{possible} effects, there must be some external access to \texttt{s.secret} in the present.

\newpage

\section{Proposed Triples}

Consider two program states that are connected by some program execution:
$$A_1\ \trans{}\ A_2$$
That is, a program starts at state $A_1$, and after some execution arrives at
$A_2$. This \textbf{hopefully} allows for the representation of temporal properties without the 
need for referring to explicit program states.
For this purpose we define a restricted subset of Chainmail assertions that
do not include temporal operators in Figure \ref{f:assertions_triple}.
	
	\begin{figure}[h]
	\[
	\begin{array}{llr}
	A & ::= & \textit{Assertions}\\  
	| & e & \\
	| & e\ :\ C & \\
	| & e\ \in\ S & \\
	| & \langle x\ \texttt{access}\ y \rangle &\\
	| & \langle \texttt{internal}x \rangle &\\
	| & \langle \texttt{external}y \rangle &\\
	| & \langle x\ \texttt{calls}\ y.m(args)\rangle & \\
	| & \neg A & \\
	| & A\ \wedge\ A & \\
	| & A\ \vee\ A & \\
	| & A\ \longrightarrow\ A & \\
	| & \forall\ x.\ A & \\
	| & \exists\ x.\ A & \\
	| & \forall\ S.\ A & \\
	| & \exists\ S.\ A &
	\end{array}
	\]
	\caption{Assertions}
	\label{f:assertions_triple}
	\end{figure}
	
In order to help reason about temporal properties in a way that does not explicitly require details of the program 
configuration, we propose three triples: 

\begin{enumerate}
\item
\emph{Only If} [$\onlyIf{A_1}{A_2}{A}$]: If program execution starts at some $A_1$ state, and reaches some $A_2$ state, 
then the original program state must have also satisfied $A$.\\
e.g. if the balance of a bank account changes over time, then there must be some external object in the current 
program state that has access to the account.
\item
\emph{Only By} [$\onlyBy{A_1}{A_2}{\epsilon}$]:  If a program execution starts at some $A_1$ state, and reaches some $A_2$ state, then the execution path that connects the two states must have had the effect $\epsilon$. Note, $\epsilon$ is an under-approximation of 
both the total effects and the necessary effects of a path connecting $A_1$ to $A_2$.\\
e.g. if the balance of the bank account changes over time, then the paths connecting the two states can be generally described as 
mutating the account object.
\item
\emph{Only Through} [$\onlyThrough{A_1}{A_2}{A}$]: If a program execution starts at some $A_1$ state, and reaches some $A_2$ state, then program execution must have passed through some $A$ state.
As in (2), $A$ is an under-approximation of 
both the states that must and will be passed through by a path connecting $A_1$ to $A_2$.\\
e.g.if the balance of the bank account changes over time, then the paths connecting the two states must contain 
an intermediate program state where the balance is changed.
\end{enumerate}

I will first discuss (2) and (3).

\subsection{Only By}
\begin{figure}[t]
\footnotesize
\[
\begin{array}{lcr}
\begin{array}{llr}
	\epsilon & ::= &\textit{Effects}\\ 
	| & \mut{s}{x}{f} & \\
	| & \gives{s}{x}{y} & \\
	| & \exposes{s}{x} & \\
	| & \epsilon\ \cap\ \epsilon & \\
	| & \epsilon\ \cup\ \epsilon & \\
	| & \univ & \\
	| & \emptyset & \\
\end{array} &
~~~~~~~~&
\begin{array}{llr}
s & ::= &\textit{Source}\\ 
	| & \_ & \\
	| & M & \\
\end{array}
\end{array}
\]
\normalsize
\caption{Effects Syntax}
\label{f:effect_syntax_triple}
\end{figure}
\begin{figure}[p]
\footnotesize
\begin{mathpar}
\infer
	{}
	{
	M\ \vdash\ \onlyBy{A_1}{A_2}{\univ}
	}
	\quad(\textsc{Universal})
	\and
\infer
	{}
	{
	M\ \vdash\ \onlyBy{x.f\ =\ y}{x.f\ \neq\ y}{\mut{\_}{x}{f}}
	}
	\quad(\textsc{Mut})
	\and
\infer
	{
	M\ \vdash\ \onlyBy{A_1}{A_2}{\texttt{\_ mut x.f}}
	}
	{
	M\ \vdash\ \onlyBy{\texttt{internal}\langle x \rangle\ \wedge\ A_1}
		{A_2}
		{M\ \texttt{mut x.f}}
	}
	\quad(\textsc{Mut-Int})
	\and
\infer
	{}
	{
	M\ \vdash\ \onlyBy{\neg \langle x\ \texttt{access}\ y \rangle}
		{\langle x\ \texttt{access}\ y \rangle}
		{\gives \_ y x}
	}
	\quad(\textsc{Give})
%	\and
%\infer
%	{
%	M\ \vdash\ \onlyBy{A_1}{A_2}{\gives \_ x y}
%	}
%	{
%	M\ \vdash\ \onlyBy{\forall \texttt{z}.[\langle \texttt{internal}\ \texttt{z} \rangle\ \vee\ \neg\ \langle\texttt{z}\ \texttt{access}\ \texttt{x} \rangle] \ \wedge\ A_1}
%	  {\langle \texttt{external}\ y \rangle\ \wedge\ A_2}
%	  {\exposes{M}{x}}
%	}
%	\quad(\textsc{Expose})
	\and
\infer
	{
	M\ \vdash\ \onlyBy{A_1}{A_2}{\gives \_ x y}
	}
	{
	M\ \vdash\ \onlyBy
		{\langle \texttt{encapsulated}\ x\rangle \ \wedge\ A_1}
	  	{\langle \texttt{external}\ y \rangle\ \wedge\ A_2}
	  	{\exposes{M}{x}}
	}
	\quad(\textsc{Expose})
	\and
\infer
	{
	A_1'\ \supseteq\ A_1 \\
	A_2'\ \supseteq\ A_2 \\
	M\ \vdash\ \onlyBy{A_1}{A_2}{\epsilon}
	}
	{
	M\ \vdash\ \onlyBy{A_1'}{A_2'}{\epsilon}
	}
	\quad(\textsc{Consequence})
%	\and
%\infer
%	{
%	A_2\ \longrightarrow\ A \\
%	\onlyBy{A_1}{\epsilon}{A}
%	}
%	{
%	\onlyBy{A_1}{\epsilon}{A_2}
%	}
%	\quad(\textsc{Consequence-2})
	\and
\infer
	{
	M\ \vdash\ \onlyBy{A_1}{A}{\epsilon_1} \\
	M\ \vdash\ \onlyBy{A_2}{A}{\epsilon_2}
	}
	{
	M\ \vdash\ \onlyBy{A_1\ \wedge\ A_2}{A}{\epsilon_1\ \cap\ \epsilon_2}
	}
	\quad(\textsc{Conjunction-1})
	\and
\infer
	{
	M\ \vdash\ \onlyBy{A}{A_1}{\epsilon_1} \\
	M\ \vdash\ \onlyBy{A}{A_1}{\epsilon_2}
	}
	{
	M\ \vdash\ \onlyBy{A}{A_1\ \wedge\ A_2}{\epsilon_1\ \cap\ \epsilon_2}
	}
	\quad(\textsc{Conjunction-2})
	\and
\infer
	{
	M\ \vdash\ \onlyBy{A_1}{A}{\epsilon_1} \\
	M\ \vdash\ \onlyBy{A_2}{A}{\epsilon_2}
	}
	{
	M\ \vdash\ \onlyBy{A_1\ \vee\ A_2}{A}{\epsilon_1\ \cup\ \epsilon_2}
	}
	\quad(\textsc{Disjunction-1})
	\and
\infer
	{
	M\ \vdash\ \onlyBy{A}{A_2}{\epsilon_1} \\
	M\ \vdash\ \onlyBy{A}{A_1}{\epsilon_2}
	}
	{
	M\ \vdash\ \onlyBy{A}{A_1\ \vee\ A_2}{\epsilon_1\ \cup\ \epsilon_2}
	}
	\quad(\textsc{Disjunction-2})
\end{mathpar} 
\caption{Only By}
\label{f:effect_infer_triple}
\end{figure}

The intent of (2) is to be able to generally describe the program that must exist connecting two states. 
In Hoare Logic we take a precondition and a program, and try to infer what the resulting program state must 
satisfy. In Chainmail we often start with both the precondition and the postcondition, and try to infer
what the intervening program must look like for those two conditions to hold.

There are often many different programs that can connect two program states, but we want to be able to 
identify their commonalities. i.e. there are many different programs that might change the balance of an 
account from \$50 to \$100, but they must all have the effect of assigning to \verb|a.balance|.

The design of ``Only By'' is much like that of Section 8. i.e. the effect syntax is given in Figure \ref{f:effect_syntax_triple}
and the rules for deriving effects are in Figure \ref{f:effect_infer_triple} (note, while not explicitly 
referenced, for now it is assumed that there is some internal module $M$).

 An effect indicates 
the set of possible paths connecting two program states. An effect may be either a field assignment (\texttt{mut}) or
the gaining of access (\texttt{gives}). Further, effects may also indicate the intersection of two effects ($\epsilon_1\ \cap\ \epsilon_2$)
indicating the set of paths that have both the effect $\epsilon_1$ and $\epsilon_2$. Effects may also
represent the union of paths ($\epsilon_1\ \cup\ \epsilon_2$), that is all paths that either have the effect 
$\epsilon_1$ or $\epsilon_2$. For now \texttt{x} and \texttt{y} refer to objects, but it is conceivable that they
could also refer to field accesses.
Every effect has a source (\texttt{X}) that is either unspecified (\_) or 
the internal module ($M$).

The difference between Figure \ref{f:effect_infer_triple} and that of Figure \ref{f:effectInference}
is that the later uses a set of necessary effects, while the former only references a specific effect
and includes intersections and unions as part of the effect syntax.
The rules for deriving effects is given in Figure \ref{f:effect_infer_triple},
and differs in similar ways to that of Section 8. 
Note that we define (see \textsc{Expose}) the following Chainmail predicate.
$$\langle \texttt{encapsulated}\ x\rangle\ \triangleq\ 
\forall y.[\neg\langle y\ \texttt{access}\ x \rangle\ \vee\ \langle \texttt{internal}\ y \rangle]$$

The problem with (2) is that we are not actually interested showing what effects must occur to 
connect two different program states, but rather which steps must be taken to give those effects.
i.e. if the balance of an account changes from \$50 to \$100, it must come as a result of a call to 
\verb|a.deposit|. For this reason, (3) more directly describes the specifications we intend.

\subsection{Only Through}

(2) and (3) are connected in intent. I believe that ``only through'' is really what we would like to be able to write, 
but would at least need something corresponding to ``only by'' to connect the dots. That is, often before we can say that a program 
state must occur (i.e. $A_3$ in ``only through'') we need to be able say what $A_3$ is implied by some set of effects 
that $A_3$ has on the program state. Thus, to use ``only by'' to express ``only through'', we need a judgment 
that connects the set of paths defined by an effect $\epsilon$ to an assertion. We express this below
$$M, \epsilon \vDash\ A$$
Informally, this is read as ``the program state $A$ sastisfies the path defined by $\epsilon$''.
This does not mean that $A$ is the only possible program state that satisfies $\epsilon$, 
and so this is an under-approximation of the necessary program states implied by $\epsilon$.

In order to define effect/path satisfaction, we give the following definitions.
$$M\ ::=\ \overline{C}$$
$$C\ ::=\ \overline{m\ :\overline{\{A\}\ \_\ \{\epsilon\})}}$$
We define a module as a set of class definitions, and class definitions as a set of method names
coupled with a set of precondition/effect pairs. We then define the following triple:
$$\{A\}\ \texttt{y.m(args)}\ \{\epsilon\}$$
This triple is interpreted as the method call \texttt{y.m(args)}
will have effect $\epsilon$ if the precondition $A$ is satisfied.
Note: both $A$ and $\epsilon$ may be dependent on \texttt{y} and \texttt{args}.
We can use this triple to now define effect/path satisfaction.

\begin{mathpar}
\infer
	{
	\textit{sourceOf}\ \epsilon\ \textit{is}\ M\\
	C\ \in\ M\\
	\{A\}\ \texttt{y.m(args)}\ \{\epsilon\}\ \in\ C
	}
	{
	M, \epsilon \vdash\ A\ \wedge\ \exists \texttt{o}.[\langle \texttt{o external} \rangle\ \wedge \langle \texttt{o calls y(args)} \rangle]
	\wedge (\texttt{y}\ :\ C)
	}
	\and
\infer
	{
	M, \epsilon_1 \vdash\ A_1\\
	M, \epsilon_2 \vdash\ A_2\\
	}
	{M, \epsilon_1\ \cap\ \epsilon_2 \vdash\ A_1\ \wedge\ A_2}
	\and
\infer
	{
	M, \epsilon_1 \vdash\ A_1\\
	M, \epsilon_2 \vdash\ A_2\\
	}
	{M, \epsilon_1\ \cup\ \epsilon_2 \vdash\ A_1\ \vee\ A_2}
\end{mathpar}
While the above judgment defines possible program states
that satisfy an effect, we are actually interested in 
defining all possible program states that satisfy an 
effect. For this we define $A_\epsilon$
$$A_\epsilon\ \triangleq\ \texttt{false}\vee(\underset{M, \epsilon\ \vDash\ A}{\bigvee}A)$$
Note, that if there is no possible program state that can 
satsify an effect $\epsilon$ then $A_\epsilon = \texttt{false}$, an impossible state.
This could potentially be usefull for proofs by contradiction.

We can now use the above definitions to arrive at rules for defining ``only through'' in Figure \ref{f:only_through}.
\begin{figure}[p]
\footnotesize
\begin{mathpar}
%\infer
%	{
%	M\ \vdash\ \onlyBy{A_1}{A_2}{\epsilon}
%	}
%	{M\ \vdash\ \onlyThrough{A_1}{A_2}{A_\epsilon}}
%	\quad(\textsc{OT-Effect})
%	\and
\infer
	{}
	{
	M\ \vdash\ \onlyThrough{x.f\ =\ y}{x.f\ \neq\ y}{\exists z.[\mut{z}{x}{f}]}
	}
	\quad(\textsc{OT-Mut})
	\and
\infer
	{
	M\ \vdash\ \onlyThrough{A_1}{A_2}{\exists z.[\mut{z}{x}{f}]}
	}
	{
	M\ \vdash\ \onlyThrough
					{A_1\ \wedge\ \langle \texttt{internal}\ x \rangle}
					{A_2}
					{\exists z.[\mut{z}{x}{f}\ \wedge\ \langle \texttt{internal}\ z \rangle]}
	}
	\quad(\textsc{OT-Mut-Int})
	\and
\infer
	{}
	{M\ \vdash\ \onlyThrough{\neg \langle x\ \texttt{access}\ y \rangle}{\langle x\ \texttt{access}\ y \rangle}{\exists z.[\gives{z}{x}{y}]}}
	\quad(\textsc{OT-Give})
	\and
\infer
	{
	M\ \vdash\ \onlyThrough{A_1}{A_2}{\exists z.[\gives{z}{x}{y}]}
	}
	{
	M\ \vdash\ \onlyThrough
		{A_1\ \wedge\ \langle \texttt{encapsulated}\ x \rangle }
		{A_2}
		{\exists z\ y'.[\gives{z}{x}{y'}\ \wedge\ \langle \texttt{internal}\ z\rangle]}
	}
	\quad(\textsc{OT-Give-Int1})
	\and
\infer
	{M\ \vdash\ \onlyThrough{A_1}{A_2}{\exists z\ y'.[\gives{z}{x}{y'}\ \wedge\ \langle \texttt{internal}\ z\rangle]}}
	{
	M\ \vdash\ \onlyThrough
		{A_1}
		{A_2\ \wedge\ \langle \texttt{external}\ y \rangle}
		{\exists z\ y'.[\gives{z}{x}{y'}\ \wedge\ \langle \texttt{internal}\ z\rangle\
		\wedge\ \langle \texttt{external}\ y' \rangle]}
	}
	\quad(\textsc{OT-Give-Int2})
	\and
\infer
	{
	M\ \vdash\ \onlyThrough{A_1}{A_2}{A_3} \\
	M\ \vdash\ \onlyThrough{A_1}{A_3}{A}
	}
	{M\ \vdash\ \onlyThrough{A_1}{A_2}{A}}
	\quad(\textsc{OT-Trans1})
	\and
\infer
	{
	M\ \vdash\ \onlyThrough{A_1}{A_2}{A_3} \\
	M\ \vdash\ \onlyThrough{A_3}{A_2}{A}
	}
	{M\ \vdash\ \onlyThrough{A_1}{A_2}{A}}
	\quad(\textsc{OT-Trans2})
	\and
\infer
	{
	M\ \vdash\ A \longrightarrow A'\\
	M\ \vdash\ \onlyThrough{A'}{A_1}{A_2}
	}
	{M\ \vdash\ \onlyThrough{A}{A_1}{A_2}}
	\quad(\textsc{OT-Consequence1})
	\and
\infer
	{
	M\ \vdash\ A \longrightarrow A'\\
	M\ \vdash\ \onlyThrough{A_1}{A'}{A_2}
	}
	{M\ \vdash\ \onlyThrough{A_1}{A}{A_2}}
	\quad(\textsc{OT-Consequence2})
	\and
\infer
	{
	M\ \vdash\ A \longrightarrow A'\\
	M\ \vdash\ \onlyThrough{A_1}{A_2}{A}
	}
	{M\ \vdash\ \onlyThrough{A_1}{A_2}{A'}}
	\quad(\textsc{OT-Consequence3})
	\and
\infer
	{
	M\ \vdash\ \onlyThrough{A_1}{A_2}{A} \\
	M\ \vdash\ \onlyThrough{A_1'}{A_2}{A'}
	}
	{M\ \vdash\ \onlyThrough{A_1\ \vee\ A_	1'}{A_2}{A\ \vee\ A'}}
	\quad(\textsc{OT-Disj1})
	\and
\infer
	{
	M\ \vdash\ \onlyThrough{A_1}{A_2}{A} \\
	M\ \vdash\ \onlyThrough{A_1}{A_2'}{A'}
	}
	{M\ \vdash\ \onlyThrough{A_1}{A_2\ \vee\ A_2'}{A\ \vee\ A'}}
	\quad(\textsc{OT-Disj2})
	\and
\infer
	{
	M\ \vdash\ \onlyThrough{A_1}{A_2}{A} \\
	M\ \vdash\ \onlyThrough{A_1'}{A_2}{A'}
	}
	{M\ \vdash\ \onlyThrough{A_1\ \wedge\ A_	1'}{A_2}{A\ \wedge\ A'}}
	\quad(\textsc{OT-Conj1})
	\and
\infer
	{
	M\ \vdash\ \onlyThrough{A_1}{A_2}{A} \\
	M\ \vdash\ \onlyThrough{A_1}{A_2'}{A'}
	}
	{M\ \vdash\ \onlyThrough{A_1}{A_2\ \wedge\ A_2'}{A\ \wedge\ A'}}
	\quad(\textsc{OT-Conj2})
\end{mathpar}
\caption{Only Through}
\label{f:only_through}
\end{figure}
Given a starting point $A_1$ and an end point $A_2$, along with an effect $\epsilon$, we can derive a necessary intermediate program state
that all paths must pass through.

\subsection{Soundness and Completeness of the Proposed Proof Rules}
For us to derive a proof of soundness or completeness for our proof rules, 
we need to define exactly what we are attempting to prove.

The effect $\epsilon$ is not in fact an effect, rather it is shorthand for a set of execution paths.
For example, $\mut{\_}{x}{f}$ denotes the set of execution paths that 
at least mutate the field $f$ in the object referenced by $x$, but may 
have other effects too. For this reason I will also refer to $\epsilon$ 
as an ``execution set''.
Thus, $\epsilon$ is an upper-bound on the set of execution paths
connecting two program states.
In other words: if $\onlyBy{A}{A'}{\epsilon}$, then for all
executions paths:
$$p\ =\ \sigma_1\ \leadsto\ \ldots\ \leadsto\ \sigma_n$$
where $M, \sigma_1\ \vDash\ A$ and $M,\ \sigma_n\ \vDash\ A'$
$$p\ \in\ \epsilon$$
That is $\epsilon$ is an over-approximation of the set of execution
paths that begin at a program state $A$ and end at a program state
$A'$. We state this judgment as $M,\ p\ \vDash\ \epsilon$.

Before I give the formal definition for the above property we 
first define what it means for a single reduction step to 
belong to an execution set. 
We do this in Figure \ref{f:effect_step}.

\begin{figure}[h]
\footnotesize
\begin{mathpar}
\infer
	{}
	{M\ \vDash \sigma_1\ \leadsto\ \sigma_2\ \in\ U}
	\quad(\textsc{Universal})
	\and
%\infer
%	{
%	(\forall\ x\ f\ y, M,\ \sigma_1\ \vDash\ x.f\ =\ y\ \Rightarrow\ M,\ \sigma_2\ \vDash\ x.f\ =\ y)\\
%	(\forall\ x\ y, M,\ \sigma_2\ \vDash\ \langle x\ \texttt{access}\ y\rangle\ \Rightarrow\ M,\ \sigma_1\ \vDash\ \langle x\ \texttt{access}\ y\rangle)
%	}
%	{M,\ \sigma_1\ \leadsto\ \sigma_2\ \vDash\ \emptyset}
%	\quad(\textsc{Empty})
%	\and
\infer
	{
	M,\ \sigma_1\ \vDash\ x.f\ =\ y\\
	M,\ \sigma_2\ \vDash\ x.f\ \neq\ y
	}
	{M\ \vDash\ \sigma_1\ \leadsto\ \sigma_2\ \in\ \mut{\_}{x}{f}}
	\quad(\textsc{Mut})
	\and
\infer
	{
	M\ \vDash\ \sigma_1\ \leadsto\ \sigma_2\ \in\ \mut{\_}{x}{f}\\
	M,\ \sigma_1\ \vDash\ \exists\ y\ m,[\langle \sigma_1.(\texttt{this})\ \texttt{calls}\ y.m(\_) \rangle\ \wedge\ \langle \texttt{internal}\ y\rangle]
	}
	{M\ \vDash\ \sigma_1\ \leadsto\ \sigma_2\ \in\ \mut{M}{x}{f}}
	\quad(\textsc{Mut-Int})
	\and
\infer
	{
	M,\ \sigma_1\ \vDash\ \neg\langle x\ \texttt{access}\ y \rangle\\
	M,\ \sigma_2\ \vDash\ \langle x\ \texttt{access}\ y \rangle
	}
	{M\ \vDash\ \sigma_1\ \leadsto\ \sigma_2\ \in\ \gives{\_}{x}{y}}
	\quad(\textsc{Gives})
	\and
\infer
	{
	M\ \vDash\ \sigma_1\ \leadsto\ \sigma_2\ \in\ \gives{\_}{x}{y}\\
	M,\ \sigma_1\ \vDash\ \exists\ z\ m,[\langle \sigma_1.(\texttt{this})\ \texttt{calls}\ z.m(\_) \rangle\ \wedge\ \langle \texttt{internal}\ z\rangle]
	}
	{M\ \vDash\ \sigma_1\ \leadsto\ \sigma_2\ \in\ \gives{M}{x}{y}}
	\quad(\textsc{Gives-Int})
	\and
\infer
	{
	\alpha\ \in\ \sigma_1.(\chi) \\
	M,\ \sigma_1\ \vDash\ [\alpha/x]A\\
	M\ \vDash\ \sigma_1\ \leadsto\ \sigma_2\ \in\ [\alpha/x]\epsilon
	}
	{M\ \vDash\ \sigma_1\ \leadsto\ \sigma_2\ \in\ \exists (x \vDash\ A).\epsilon}
	\quad(\textsc{Exists})
	\and
\infer
	{
	M\ \vDash\ \sigma_1\ \leadsto\ \sigma_2\ \in\ \epsilon \\
	M\ \vDash\ \sigma_1\ \leadsto\ \sigma_2\ \in\ \epsilon'
	}
	{M\ \vDash\ \sigma_1\ \leadsto\ \sigma_2\ \in\ \epsilon\ \cap\ \epsilon'}
	\quad(\textsc{Intersection})
	\and
\infer
	{
	M\ \vDash\ \sigma_1\ \leadsto\ \sigma_2\ \in\ \epsilon
	}
	{M\ \vDash\ \sigma_1\ \leadsto\ \sigma_2\ \in\ \epsilon\ \cup\ \epsilon'}
	\quad(\textsc{Union-1})
	\and
\infer
	{
	M\ \vDash\ \sigma_1\ \leadsto\ \sigma_2\ \in\ \epsilon'
	}
	{M\ \vDash\ \sigma_1\ \leadsto\ \sigma_2\ \in\ \epsilon\ \cup\ \epsilon'}
	\quad(\textsc{Union-2})
\end{mathpar}
\caption{Execution Sets of a Single Execution Step}
\label{f:effect_step}
\end{figure}

\begin{figure}[h]
\footnotesize
\begin{mathpar}
\infer
	{}
	{M\ \vDash p\ \in\ U}
	\quad(\textsc{Universal})
	\and
%\infer
%	{
%	(\forall\ x\ f\ y, M,\ \sigma_1\ \vDash\ x.f\ =\ y\ \Rightarrow\ M,\ \sigma_2\ \vDash\ x.f\ =\ y)\\
%	(\forall\ x\ y, M,\ \sigma_2\ \vDash\ \langle x\ \texttt{access}\ y\rangle\ \Rightarrow\ M,\ \sigma_1\ \vDash\ \langle x\ \texttt{access}\ y\rangle)
%	}
%	{M,\ \sigma_1\ \leadsto\ \sigma_2\ \vDash\ \emptyset}
%	\quad(\textsc{Empty})
%	\and
\infer
	{
	\sigma_1\ \leadsto\ \sigma_2\ \in\ p\\
	M\ \vDash\ \sigma_1\ \leadsto\ \sigma_2\ \in\ \mut{s}{x}{f}
	}
	{
	M\ \vDash\ p\ \in\ \mut{s}{x}{f}
	}
	\quad(\textsc{Mut})
	\and
\infer
	{
	\sigma_1\ \leadsto\ \sigma_2\ \in\ p\\
	M\ \vDash\ \sigma_1\ \leadsto\ \sigma_2\ \in\ \gives{s}{x}{y}
	}
	{M\ \vDash\ p\ \in\ \gives{s}{x}{y}}
	\quad(\textsc{Gives})
	\and
\infer
	{
	\sigma_1\ \leadsto\ \sigma_2\ \in\ p\\
	M\ \vDash\ \sigma_1\ \leadsto\ \sigma_2\ \in\ \exists (x \vDash\ A).\epsilon
	}
	{M\ \vDash\ \sigma_1\ \leadsto\ \sigma_2\ \in\ \exists (x \vDash\ A).\epsilon}
	\quad(\textsc{Exists})
	\and
\infer
	{
	M\ \vDash\ p\ \in\ \epsilon \\
	M\ \vDash\ p\ \in\ \epsilon'
	}
	{M\ \vDash\ p\ \in\ \epsilon\ \cap\ \epsilon'}
	\quad(\textsc{Intersection})
	\and
\infer
	{
	M\ \vDash\ p\ \in\ \epsilon
	}
	{M\ \vDash\ p\ \in\ \epsilon\ \cup\ \epsilon'}
	\quad(\textsc{Union-1})
	\and
\infer
	{
	M\ \vDash\ p\ \in\ \epsilon'
	}
	{M\ \vDash\ p\ \in\ \epsilon\ \cup\ \epsilon'}
	\quad(\textsc{Union-2})
\end{mathpar}
\caption{Execution Sets of Execution Paths}
\label{f:effect_path}
\end{figure}
Figure \ref{f:effect_path} defines what it means for a path to belong to an execution set.
We now define $\onlyBy{A_1}{A_2}{\epsilon}$
\begin{definition}[\textsc{Only By}]
$M\ \vDash\ \onlyBy{A_1}{A_2}{\epsilon}$ 
iff 
$\forall\ M'\ \sigma_0\ \sigma_n$, 
such that
$M\ \fcmp\ M'\ \bullet\ \sigma_0\ \leadsto\ \ldots\ \leadsto\ \sigma_n$
it follows that 
$M\ \vDash\ \sigma_0\ \leadsto\ \ldots\ \leadsto\ \sigma_n\ \in\ \epsilon$.
\end{definition}
We are now able to state both soundness and completeness.
\begin{theorem}[Soundness]
If $M\ \vdash\ \onlyBy{A_1}{A_2}{\epsilon}$ then 
$M\ \vDash\ \onlyBy{A_1}{A_2}{\epsilon}$.
\end{theorem}
\begin{proof}
The proof proceeds by induction on the derivation of $M\ \vdash\ \onlyBy{A_1}{A_2}{\epsilon}$.
\end{proof}\hfill\qed
\begin{theorem}[Completeness]
If $M\ \vDash\ \onlyBy{A_1}{A_2}{\epsilon}$ then 
$M\ \vdash\ \onlyBy{A_1}{A_2}{\epsilon}$.
\end{theorem}
\begin{proof}
({\color{red}Julian: The proof of completeness needs to probably be based on a specific form of
the proof designed for single step execution. I'm not 100\% sure that the proof rules are complete, but they are probably sound.})
\end{proof}\hfill\qed


%\footnotesize
%\begin{itemize}
%\item
%$M,\ \sigma_1\ \leadsto\ \sigma_2\ \vDash\ \mut{\_}{x}{f}\ iff$
%\begin{itemize}
%\item
%$\exists\ z,\ M,\ \sigma_1\ \vDash\ x.f\ =\ z$, $M,\sigma_2\ \vDash\ x.f\ \neq\ z$
%\end{itemize}
%
%\item
%$M,\ \sigma_1\ \leadsto\ \sigma_2\ \vDash\ \mut{M}{x}{f}\ iff$
%\begin{itemize}
%\item
%$\exists\ z,\ M,\ \sigma_1\ \vDash\ x.f\ =\ z$, $M,\sigma_2\ \vDash\ x.f\ \neq\ z$
%\item
%$M,\ \sigma_1\ \vDash\ \exists\ y\ m.[\langle \sigma_1.(\texttt{this})\ \texttt{calls}\ y.m(\_) \rangle\ \wedge\ \langle \texttt{internal}\ y\rangle]$
%\end{itemize}
%
%\item
%$M,\ \sigma_1\ \leadsto\ \sigma_2\ \vDash\ \gives{\_}{x}{y}\ iff$
%\begin{itemize}
%\item
%$M,\ \sigma_1\ \vDash\ $
%\end{itemize}
%\end{itemize}
%\normalsize

%\begin{figure}[h]
%\footnotesize
%\begin{mathpar}
%\infer
%	{}
%	{M,\ \sigma_1\ \leadsto\ \sigma_2\ \vDash\ u}
%	\quad(\textsc{Universal})
%	\and
%\infer
%	{\sigma_1.(\texttt{contn}) = x.f\ \texttt{:=}\ y  \\
%	M,\ \sigma_1\ \vDash\ x.f\ \neq\ y}
%	{M,\ \sigma_1\ \leadsto\ \sigma_2\ \vDash\ \mut{\_}{x}{f}}
%	\quad(\textsc{Mut-Ext})
%	\and
%\infer
%	{
%	\sigma_1.(\texttt{contn}) = \_\ \texttt{:=}\ x.f \\
%	M,\ \sigma_1\ \vDash\ \neg \langle \sigma_1.(\texttt{this})\ \texttt{access}\ x.f \rangle
%	}
%	{M,\ \sigma_1\ \leadsto\ \sigma_2\ \vDash\ \gives{\_}{x.f}{\sigma_1.(\texttt{this})}}
%	\quad(\textsc{Give-FAccess})
%	\and
%\infer
%	{\sigma_1.(\texttt{contn}) = x.f\ \texttt{:=}\ y}
%	{M,\ \sigma_1\ \leadsto\ \sigma_2\ \vDash\ \gives{\_}{y}{x}}
%	\quad(\textsc{Give-FAssign})
%	\and
%\infer
%	{
%	\sigma_1.(\texttt{contn}) = \texttt{return}\ x
%	}
%	{M,\ \sigma_1\ \leadsto\ \sigma_2\ \vDash\ \gives{\_}{x}{\sigma_2.(\texttt{this})}}
%	\quad(\textsc{Give-Return})
%	\and	
%\infer
%	{
%	\sigma_1.(\texttt{contn}) = \_\ =\ \texttt{x.m(args)} \\
%	M,\ \sigma_1\ \vDash\ \langle \texttt{internal}\ x \rangle \\
%	M\ \vDash\ \sigma_1.(\texttt{this})\ \texttt{calls x.m(args)}\ :\ \epsilon
%	}
%	{M,\ \sigma_1\ \leadsto\ \sigma_2\ \vDash\ \epsilon}
%	\quad(\textsc{Internal})
%	\and
%\infer
%	{
%	\alpha\ \in\ \sigma_1.(\chi) \\
%	M,\ \sigma_1\ \vDash\ [\alpha/x]A\\
%	M,\ \sigma_1\ \leadsto\ \sigma_2\ \vDash\ [\alpha/x]\epsilon
%	}
%	{M,\ \sigma_1\ \leadsto\ \sigma_2\ \vDash\ \exists (x \vDash\ A).\epsilon}
%	\quad(\textsc{Exists})
%	\and
%\infer
%	{
%	M,\ \sigma_1\ \leadsto\ \sigma_2\ \vDash\ \epsilon \\
%	M,\ \sigma_1\ \leadsto\ \sigma_2\ \vDash\ \epsilon'
%	}
%	{M,\ \sigma_1\ \leadsto\ \sigma_2\ \vDash\ \epsilon\ \cap\ \epsilon'}
%	\quad(\textsc{Intersection})
%	\and
%\infer
%	{
%	M,\ \sigma_1\ \leadsto\ \sigma_2\ \vDash\ \epsilon
%	}
%	{M,\ \sigma_1\ \leadsto\ \sigma_2\ \vDash\ \epsilon\ \cup\ \epsilon'}
%	\quad(\textsc{Union-1})
%	\and
%\infer
%	{
%	M,\ \sigma_1\ \leadsto\ \sigma_2\ \vDash\ \epsilon'
%	}
%	{M,\ \sigma_1\ \leadsto\ \sigma_2\ \vDash\ \epsilon\ \cup\ \epsilon'}
%	\quad(\textsc{Union-2})
%\end{mathpar}
%\caption{Effect of a Single Execution Step ({\color{red}Julian: I haven't figured out the last premise in \textsc{Internal} yet})}
%\label{f:effect_step}
%\end{figure}  
\newpage
\section{``Only Through'' and ``Only If'' Defined without Effects}

In this Section I try to define ``Only Through'' and ``Only If'' without using the intermediate of ``Only By''.
To do this I extend the assertion syntax to include two assertion forms that describe program behaviour
and side effects:
\begin{itemize}
\item
$\langle \mut{x}{y}{f} \rangle$: 
This can be interpreted as essentially the same as $\texttt{changes}\langle y.f \rangle$, but with including the fact that either $x$ was the object that changed $y.f$, or $y.f$ 
changed as a result of a call from client code to the internal object $x$.
\item
$\langle \gives{x}{y}{z} \rangle$: object $z$ gains access to $y$, where $x$ is either the object that performed the operation that gave that access, or $z$ gained
access to $y$ as a result of a call from client code tothe internal object $x$.
\end{itemize}
Thus, we give the new assertion syntax in Figure \ref{f:assertions_triple2}.
	
	\begin{figure}[h]
	\[
	\begin{array}{llr}
	A & ::= & \textit{Assertions}\\  
	| & e & \\
	| & e\ :\ C & \\
	| & e\ \in\ S & \\
	| & \langle x\ \texttt{access}\ y \rangle &\\
	| & \langle \texttt{internal}\ x \rangle &\\
	| & \langle \texttt{external}\ y \rangle &\\
	| & \langle \mut x y f \rangle &\\
	| & \langle \gives x y z \rangle &\\
	| & \langle x\ \texttt{calls}\ y.m(args)\rangle & \\
	| & \neg A & \\
	| & A\ \wedge\ A & \\
	| & A\ \vee\ A & \\
	| & A\ \longrightarrow\ A & \\
	| & \forall\ x.\ A & \\
	| & \exists\ x.\ A & \\
	| & \forall\ S.\ A & \\
	| & \exists\ S.\ A &
	\end{array}
	\]
	\caption{Assertions}
	\label{f:assertions_triple2}
	\end{figure}
	
Formally, the semantics of these two new assertions are as follows:
\begin{itemize}
\item
$M,\ \sigma \vDash\ \mut{x}{y}{f}$ iff $\forall\ v\ \sigma',$ such that $M\ \sigma\ \vDash\ y.f\ =\ v$ and $M\, \sigma\ \vDash\ \sigma\ \constrained\ \sigma'$ then 
\begin{itemize}
\item
$M\ \sigma'\ \vDash\ y.f\ \neq\ v$ and
\item
either
\begin{itemize}
\item
$\lfloor this \rfloor_\sigma\ = x$ and $\sigma.(\texttt{contn})$  is not an internal method call or
\item
$M,\ \sigma\ \vDash\ \exists x'\ m\ args,[\langle x'\ \texttt{calls}\ x.m(args)\rangle\ \wedge\ \langle \texttt{internal}\ x\rangle]$
\end{itemize}
\end{itemize}
\item
$M,\ \sigma\ \vDash\ \gives{x}{y}{z}$ iff $\forall\ \sigma',$ such that $M\, \sigma\ \vDash\ \sigma\ \constrained\ \sigma'$ then 
\begin{itemize}
\item
$M,\ \sigma\ \vDash\ \neg\langle z\ \texttt{access}\ y  \rangle$ and
\item
$M,\ \sigma'\ \vDash\ \langle z\ \texttt{access}\ y  \rangle$ and 
\item
either
\begin{itemize}
\item
$\lfloor this \rfloor_\sigma\ = x$ and $\sigma.(\texttt{contn})$  is not an internal method call or
\item
$M,\ \sigma\ \vDash\ \exists x'\ m\ args,[\langle x'\ \texttt{calls}\ x.m(args)\rangle\ \wedge\ \langle \texttt{internal}\ x\rangle]$
\end{itemize}
\end{itemize}
\end{itemize}
Now to simplify the following definitions, I introduce a chainmail predicate.
$$\langle\texttt{encapsulated}\ x\rangle\ \equiv\ \forall y.[\neg\langle y\ \texttt{access}\ x \rangle\ \vee\ \langle\texttt{internal}\ y\rangle]$$
Now I introduce proof rules for ``Only Through'' in Figures \ref{f:only_through_base} and \ref{f:only_through_connective}. 

\begin{figure}[p]
\footnotesize
\begin{mathpar}
%\infer
%	{
%	M\ \vdash\ \onlyBy{A_1}{A_2}{\epsilon}
%	}
%	{M\ \vdash\ \onlyThrough{A_1}{A_2}{A_\epsilon}}
%	\quad(\textsc{OT-Effect})
%	\and
\infer
	{}
	{
	M\ \vdash\ \onlyThrough{x.f\ =\ y}{x.f\ \neq\ y}{\exists z.[\mut{z}{x}{f}]}
	}
	\quad(\textsc{OT-Mut})
	\and
\infer
	{
	M\ \vdash\ \onlyThrough{A_1}{A_2}{\exists z.[\mut{z}{x}{f}]}
	}
	{
	M\ \vdash\ \onlyThrough
					{A_1\ \wedge\ \langle \texttt{internal}\ x \rangle}
					{A_2}
					{\exists z.[\mut{z}{x}{f}\ \wedge\ \langle \texttt{internal}\ z \rangle]}
	}
	\quad(\textsc{OT-Mut-Int})
	\and
\infer
	{}
	{M\ \vdash\ \onlyThrough{\neg \langle x\ \texttt{access}\ y \rangle}{\langle x\ \texttt{access}\ y \rangle}{\exists z.[\gives{z}{x}{y}]}}
	\quad(\textsc{OT-Give})
	\and
\infer
	{
	M\ \vdash\ \onlyThrough{A_1}{A_2}{\exists z.[\gives{z}{x}{y}]}
	}
	{
	M\ \vdash\ \onlyThrough
		{A_1\ \wedge\ \langle \texttt{encapsulated}\ x \rangle }
		{A_2}
		{\exists z\ y'.[\gives{z}{x}{y'}\ \wedge\ \langle \texttt{internal}\ z\rangle]}
	}
	\quad(\textsc{OT-Give-Int1})
	\and
\infer
	{M\ \vdash\ \onlyThrough{A_1}{A_2}{\exists z\ y'.[\gives{z}{x}{y'}\ \wedge\ \langle \texttt{internal}\ z\rangle]}}
	{
	M\ \vdash\ \onlyThrough
		{A_1}
		{A_2\ \wedge\ \langle \texttt{external}\ y \rangle}
		{\exists z\ y'.[\gives{z}{x}{y'}\ \wedge\ \langle \texttt{internal}\ z\rangle\
		\wedge\ \langle \texttt{external}\ y' \rangle]}
	}
	\quad(\textsc{OT-Give-Int2})
\end{mathpar}
\caption{Only Through Base Rules}
\label{f:only_through_base}
\end{figure}
\begin{figure}[p]
\footnotesize
\begin{mathpar}
\infer
	{
	M\ \vdash\ \onlyThrough{A_1}{A_2}{A_3} \\
	M\ \vdash\ \onlyThrough{A_1}{A_3}{A}
	}
	{M\ \vdash\ \onlyThrough{A_1}{A_2}{A}}
	\quad(\textsc{OT-Trans1})
	\and
\infer
	{
	M\ \vdash\ \onlyThrough{A_1}{A_2}{A_3} \\
	M\ \vdash\ \onlyThrough{A_3}{A_2}{A}
	}
	{M\ \vdash\ \onlyThrough{A_1}{A_2}{A}}
	\quad(\textsc{OT-Trans2})
	\and
\infer
	{
	M\ \vdash\ A \longrightarrow A'\\
	M\ \vdash\ \onlyThrough{A'}{A_1}{A_2}
	}
	{M\ \vdash\ \onlyThrough{A}{A_1}{A_2}}
	\quad(\textsc{OT-Consequence1})
	\and
\infer
	{
	M\ \vdash\ A \longrightarrow A'\\
	M\ \vdash\ \onlyThrough{A_1}{A'}{A_2}
	}
	{M\ \vdash\ \onlyThrough{A_1}{A}{A_2}}
	\quad(\textsc{OT-Consequence2})
	\and
\infer
	{
	M\ \vdash\ A \longrightarrow A'\\
	M\ \vdash\ \onlyThrough{A_1}{A_2}{A}
	}
	{M\ \vdash\ \onlyThrough{A_1}{A_2}{A'}}
	\quad(\textsc{OT-Consequence3})
	\and
\infer
	{
	M\ \vdash\ \onlyThrough{A_1}{A_2}{A} \\
	M\ \vdash\ \onlyThrough{A_1'}{A_2}{A'}
	}
	{M\ \vdash\ \onlyThrough{A_1\ \vee\ A_	1'}{A_2}{A\ \vee\ A'}}
	\quad(\textsc{OT-Disj1})
	\and
\infer
	{
	M\ \vdash\ \onlyThrough{A_1}{A_2}{A} \\
	M\ \vdash\ \onlyThrough{A_1}{A_2'}{A'}
	}
	{M\ \vdash\ \onlyThrough{A_1}{A_2\ \vee\ A_2'}{A\ \vee\ A'}}
	\quad(\textsc{OT-Disj2})
	\and
\infer
	{
	M\ \vdash\ \onlyThrough{A_1}{A_2}{A} \\
	M\ \vdash\ \onlyThrough{A_1'}{A_2}{A'}
	}
	{M\ \vdash\ \onlyThrough{A_1\ \wedge\ A_	1'}{A_2}{A\ \wedge\ A'}}
	\quad(\textsc{OT-Conj1})
	\and
\infer
	{
	M\ \vdash\ \onlyThrough{A_1}{A_2}{A} \\
	M\ \vdash\ \onlyThrough{A_1}{A_2'}{A'}
	}
	{M\ \vdash\ \onlyThrough{A_1}{A_2\ \wedge\ A_2'}{A\ \wedge\ A'}}
	\quad(\textsc{OT-Conj2})
	\and
\infer
	{
	M\ \vdash\ \onlyThrough{A_1}{A'}{\texttt{false}} \\
	M\ \vdash\ \onlyThrough{A_1}{A_2}{A\ \vee\ A'}
	}
	{M\ \vdash\ \onlyThrough{A_1}{A_2}{A}}
	\quad(\textsc{OT-Contra1})
	\and
\infer
	{
	M\ \vdash\ \onlyThrough{A'}{A_2}{\texttt{false}} \\
	M\ \vdash\ \onlyThrough{A_1}{A_2}{A\ \vee\ A'}
	}
	{M\ \vdash\ \onlyThrough{A_1}{A_2}{A}}
	\quad(\textsc{OT-Contra2})
\end{mathpar}
\caption{Only Through Connective Rules}
\label{f:only_through_connective}
\end{figure}

Figure \ref{f:only_if2} provides the proof rules for ``Only If''.

\begin{figure}[p]
\footnotesize
\begin{mathpar}
\infer
	{
	M\ \vdash\ A \longrightarrow A'\\
	M\ \vdash\ \onlyIf{A'}{A_1}{A_2}
	}
	{M\ \vdash\ \onlyIf{A}{A_1}{A_2}}
	\quad(\textsc{OI-Consequence1})
	\and
\infer
	{
	M\ \vdash\ A \longrightarrow A'\\
	M\ \vdash\ \onlyIf{A_1}{A'}{A_2}
	}
	{M\ \vdash\ \onlyIf{A_1}{A}{A_2}}
	\quad(\textsc{OI-Consequence2})
	\and
\infer
	{
	M\ \vdash\ A \longrightarrow A'\\
	M\ \vdash\ \onlyIf{A_1}{A_2}{A}
	}
	{M\ \vdash\ \onlyIf{A_1}{A_2}{A'}}
	\quad(\textsc{OI-Consequence3})
	\and
\infer
	{
	M\ \vdash\ \onlyIf{A_1}{A_2}{A} \\
	M\ \vdash\ \onlyIf{A_1'}{A_2}{A'}
	}
	{M\ \vdash\ \onlyIf{A_1\ \vee\ A_	1'}{A_2}{A\ \vee\ A'}}
	\quad(\textsc{OI-Disj1})
	\and
\infer
	{
	M\ \vdash\ \onlyIf{A_1}{A_2}{A} \\
	M\ \vdash\ \onlyIf{A_1}{A_2'}{A'}
	}
	{M\ \vdash\ \onlyIf{A_1}{A_2\ \vee\ A_2'}{A\ \vee\ A'}}
	\quad(\textsc{OI-Disj2})
	\and
\infer
	{
	M\ \vdash\ \onlyIf{A_1}{A_2}{A} \\
	M\ \vdash\ \onlyIf{A_1'}{A_2}{A'}
	}
	{M\ \vdash\ \onlyIf{A_1\ \wedge\ A_	1'}{A_2}{A\ \wedge\ A'}}
	\quad(\textsc{OI-Conj1})
	\and
\infer
	{
	M\ \vdash\ \onlyIf{A_1}{A_2}{A} \\
	M\ \vdash\ \onlyIf{A_1}{A_2'}{A'}
	}
	{M\ \vdash\ \onlyIf{A_1}{A_2\ \wedge\ A_2'}{A\ \wedge\ A'}}
	\quad(\textsc{OI-Conj2})
	\and
\infer
	{
	M\ \vdash\ \onlyThrough{A_1}{A_2}{A_2}
	}
	{M\ \vdash\ \onlyIf{A_1}{A_2}{A_2}}
	\quad(\textsc{OI-Invariant)}
	\and
\infer
	{
	M\ \vdash\ \onlyThrough{A_1}{A_2}{A_3} \\
	M\ \vdash\ \onlyIf{A_1}{A_3}{A}
	}
	{M\ \vdash\ \onlyIf{A_1}{A_2}{A}}
	\quad(\textsc{OI-Trans)}
\end{mathpar}
\caption{Only if}
\label{f:only_if2}
\end{figure}

\newpage

\subsection{Bank Account Example}

Now, to demonstrate the use of the rules in Figures \ref{f:only_through_base}, \ref{f:only_through_connective}, and \ref{f:only_if2}, we construct 
a proof of the bank account example. First we rewrite the bank account example using the triples define here.
The original specification is written below.
\begin{lstlisting}[mathescape=true]
Spec $\triangleq$ $\forall$ a b. [ a : Account $\wedge$ a.balance = b $\wedge$
               will $\langle\ \neg$ a.balance $\neq$ b $\rangle$
               $\longrightarrow$
               $\exists$ o.[$\langle$ external o $\rangle$ $\wedge$ $\langle$ o access a $\rangle$]
\end{lstlisting}
We now rewrite it as this
\begin{lstlisting}[mathescape=true]
Spec $\triangleq$  { a : Account $\wedge$ a.balance = b } $\trans{}$ { $\langle\ \neg$ a.balance $\neq$ b $\rangle$ }
               only if
          [ $\langle$ external o $\rangle$ $\wedge$ $\langle$ o access a $\rangle$ ]
\end{lstlisting}
Proof:
\vspace{3mm}
\footnotesize
\hrule
\vspace{3mm}
$$
	\{a : \texttt{Account}\ \wedge a.\texttt{balance} = b\}
$$
$$\trans{}$$
$$
	\{\neg\ a.\texttt{balance} \neq b\}
$$
$$\texttt{only through}$$
$$
	\exists x.[\mut{x}{a}{\texttt{balance}}]
$$
\hfill by \textsc{OT-Mut}
\hrule
\vspace{3mm}
$$
	\{a : \texttt{Account}\ \wedge a.\texttt{balance} = b\ \wedge\  \langle \texttt{internal}\ a \rangle\}
$$
$$\trans{}$$
$$
	\{\neg\ a.\texttt{balance} \neq b\}
$$
$$\texttt{only through}$$
$$
	{\exists x.[\mut{x}{a}{\texttt{balance}}]}
$$
\hfill by \textsc{OT-Consequence1}
\hrule
\vspace{3mm}
$$
	\{a : \texttt{Account}\ \wedge a.\texttt{balance} = b\ \wedge\  \langle \texttt{internal}\ a \rangle\}
$$
$$\trans{}$$
$$
	\{\neg\ a.\texttt{balance} \neq b\}
$$
$$\texttt{only through}$$
$$
	{\exists x.[\mut{x}{a}{\texttt{balance}}\ \wedge\ \langle \texttt{internal}\ x \rangle]}
$$
\hfill by \textsc{OT-Mut-Int}
\hrule
\vspace{3mm}
$$
	\{a : \texttt{Account}\ \wedge a.\texttt{balance} = b\ \wedge\  \langle \texttt{internal}\ a \rangle\}
$$
$$\trans{}$$
$$
	\{\neg\ a.\texttt{balance} \neq b\}
$$
$$\texttt{only through}$$
$$
	{\exists o.[\langle o\ \texttt{calls}\ a.\texttt{deposit}(\_, \_) \rangle\ \vee\ \langle o\ \texttt{calls}\ \_.\texttt{deposit}(a, \_) \rangle]}
$$
\hfill by \textsc{OT-Consequence3}
\hrule
\vspace{3mm}
(a)
$$
	\{a : \texttt{Account}\ \wedge a.\texttt{balance} = b\ \wedge\  \langle \texttt{internal}\ a \rangle\}
$$
$$\trans{}$$
$$
	\{\neg\ a.\texttt{balance} \neq b\}
$$
$$\texttt{only through}$$
$$
	{\exists o.[\langle o\ \texttt{access}\ a\rangle\ \wedge\ \langle \texttt{external}\ o \rangle]}
$$
\hfill by \textsc{OT-Consequence3}
\hrule
\vspace{3mm}
\normalsize
\noindent now we show that access to bank accounts cannot be leaked:
\vspace{3mm}
\footnotesize
\hrule
\vspace{3mm}
$$
	\{\neg\ \langle o\ \texttt{access}\ a\rangle\ \wedge\ \langle \texttt{external}\ o \rangle\}
$$
$$\trans{}$$
$$
	\{\langle o\ \texttt{access}\ a\rangle\ \wedge\ \langle \texttt{external}\ o \rangle\}
$$
$$\texttt{only through}$$
$$
	\exists x.[\gives{x}{o}{a}]
$$
\hfill by \textsc{OT-Gives}
\hrule
\vspace{3mm}
$$
	\{\neg\ \langle o\ \texttt{access}\ a\rangle\ \wedge\ \langle \texttt{external}\ o \rangle\ \wedge\ \langle \texttt{encapsulated}\ a\rangle\}
$$
$$\trans{}$$
$$
	\{\langle o\ \texttt{access}\ a\rangle\ \wedge\ \langle \texttt{external}\ o \rangle\}
$$
$$\texttt{only through}$$
$$
	\exists x.[\gives{x}{o}{a}\ \wedge\ \langle \texttt{internal}\ x\rangle]
$$
\hfill by \textsc{OT-Gives-Int}
\hrule
\vspace{3mm}
$$
	\{\neg\ \langle o\ \texttt{access}\ a\rangle\ \wedge\ \langle \texttt{external}\ o \rangle\ \wedge\ \langle \texttt{encapsulated}\ a\rangle\}
$$
$$\trans{}$$
$$
	\{\langle o\ \texttt{access}\ a\rangle\ \wedge\ \langle \texttt{external}\ o \rangle\}
$$
$$\texttt{only through}$$
$$
	\texttt{false}
$$
\hfill by \textsc{OT-Consequence3}
\hrule
\vspace{3mm}
$$
	\{\neg\ \langle o\ \texttt{access}\ a\rangle\ \wedge\ \langle \texttt{external}\ o \rangle\ \wedge\ \langle \texttt{encapsulated}\ a\rangle\}
$$
$$\trans{}$$
$$
	\{\neg(\neg\ \langle o\ \texttt{access}\ a\rangle\ \wedge\ \langle \texttt{external}\ o \rangle\ \wedge\ \langle \texttt{encapsulated}\ a\rangle)\}
$$
$$\texttt{only through}$$
$$
	\texttt{false}
$$
\hfill by \textsc{OT-Consequence2}
\hrule
\vspace{3mm}
\normalsize
{\noindent\color{red} Julian: There are still some lumps that need ironing out here, or maybe i'm just struggling to think through the proof.
I think some proof rules are needed for quantified assertions.
Ultimately we hope to end up here:}
\vspace{3mm}
\hrule
\vspace{3mm}
(b)
$$
	\{a : \texttt{Account}\ \wedge a.\texttt{balance} = b\ \wedge\  \langle \texttt{internal}\ a \rangle\}
$$
$$\trans{}$$
$$
	\{\exists o.[\langle o\ \texttt{access}\ a\rangle\ \wedge\ \langle \texttt{external}\ o \rangle]\}
$$
$$\texttt{only through}$$
$$
	{\exists o.[\langle o\ \texttt{access}\ a\rangle\ \wedge\ \langle \texttt{external}\ o \rangle]}
$$
\vspace{3mm}
\hrule
\vspace{3mm}
\noindent{\color{red}Julian: then we use (a) and (b) to give us:}
\vspace{3mm}
\hrule
\vspace{3mm}
$$
	\{a : \texttt{Account}\ \wedge a.\texttt{balance} = b\}
$$
$$\trans{}$$
$$
	\{\neg\ a.\texttt{balance} \neq b\}
$$
$$\texttt{only through}$$
$$
	{\exists o.[\langle o\ \texttt{access}\ a\rangle\ \wedge\ \langle \texttt{external}\ o \rangle]}
$$
\hfill by \textsc{OI-Invariant}
\hrule
\vspace{3mm}

\newpage

\subsection{Safe Example}

We recall the original Safe specification:
\begin{lstlisting}[mathescape=true]
Spec $\triangleq$ $\forall$ s scr. [ s : Safe $\wedge$ s.treasure $\neq$ null $\wedge$
               will $\langle$ s.treasure = null $\rangle$
               $\longrightarrow$
               $\exists$ o.[$\langle$ external o $\rangle$ $\wedge$ $\langle$ o access s.secret $\rangle$]
\end{lstlisting}
We now rewrite it as this
\begin{lstlisting}[mathescape=true]
Spec $\triangleq$  { s : Safe $\wedge$ s.treasure $\neq$ null } $\trans{}$ { s.treasure = null}
               only if
          [$\exists$ o.[ $\neg$ s.is_internal(o) $\wedge$ $\langle$ o access s.secret $\rangle$ ]]
\end{lstlisting}
Note, this is the modified version of the Safe that uses a ghost field to define what is internal to Safe. 
In this case, this is the safe, and the secret.

\noindent Proof:
\vspace{3mm}
\footnotesize
\hrule
\vspace{3mm}
$$
	\{s : \texttt{Safe} \wedge s.\texttt{treasure} \neq \texttt{null}\}
$$
$$\trans{}$$
$$
	\{s.\texttt{treasure} = \texttt{null} \}
$$
$$\texttt{only through}$$
$$
	\exists x.[\mut{x}{s}{\texttt{treasure}}]
$$
\hfill by \textsc{OT-Mut}
\hrule
\vspace{3mm}
$$
	\{s : \texttt{Safe} \wedge s.\texttt{treasure} \neq \texttt{null}\}
$$
$$\trans{}$$
$$
	\{s.\texttt{treasure} = \texttt{null} \}
$$
$$\texttt{only through}$$
$$
	\exists o. [\langle \texttt{external}\ o \rangle\ \wedge\ \langle o\ \texttt{calls}\ s.\texttt{take}(s.\texttt{secret}) \rangle]
$$
\hfill by \textsc{OT-Consequence3}
\hrule
\vspace{3mm}
(a)
$$
	\{s : \texttt{Safe} \wedge s.\texttt{treasure} \neq \texttt{null}\}
$$
$$\trans{}$$
$$
	\{s.\texttt{treasure} = \texttt{null} \}
$$
$$\texttt{only through}$$
$$
	\exists o. [\langle \texttt{external}\ o \rangle\ \wedge\ \langle o\ \texttt{access}\ s.\texttt{secret} \rangle]
$$
\hfill by \textsc{OT-Consequence3}
\hrule
\vspace{3mm}
\normalsize
\noindent{\color{red}Julian:We need to prove the following invariant:}
\vspace{3mm}
\footnotesize
\hrule
\vspace{3mm}
(b)
$$
	\{s : \texttt{Safe} \}
$$
$$\trans{}$$
$$
	\{\exists o. [\langle \texttt{external}\ o \rangle\ \wedge\ \langle o\ \texttt{access}\ s.\texttt{secret} \rangle]\}
$$
$$\texttt{only through}$$
$$
	\exists o. [\langle \texttt{external}\ o \rangle\ \wedge\ \langle o\ \texttt{access}\ s.\texttt{secret} \rangle]
$$
\hrule
\vspace{3mm}
\normalsize
\noindent{\color{red}Julian: Then by (a) and (b) we have:}
\vspace{3mm}
\footnotesize
\hrule
\vspace{3mm}
(b)
$$
	\{s : \texttt{Safe}  \wedge s.\texttt{treasure} \neq \texttt{null}\}
$$
$$\trans{}$$
$$
	\{\exists o. [\langle \texttt{external}\ o \rangle\ \wedge\ \langle o\ \texttt{access}\ s.\texttt{secret} \rangle]\}
$$
$$\texttt{only if}$$
$$
	\exists o. [\langle \texttt{external}\ o \rangle\ \wedge\ \langle o\ \texttt{access}\ s.\texttt{secret} \rangle]
$$
\hfill by \textsc{OI-Invariant} and \textsc{OI-Trans}
\hrule
\vspace{3mm}
\normalsize

\newpage

\subsection{Temporal Invariants}
Both the proof of Bank Account and the Safe examples rely on the invariant of encapsulation over time.
In the bank account proof, we rely on the fact that it is not possible for internal components to 
expose access to an account, and thus that if an external object has access to an account in the future, then 
some external object must have access to the account in the present moment. 
i.e.
$$\neg\langle \texttt{encapsulated}\ a  \rangle$$
is invariant. Specifically the following triple must be true:
$$\onlyThrough{a\ :\ \texttt{Account}}{\neg\langle \texttt{encapsulated}\ a  \rangle}{\neg\langle \texttt{encapsulated}\ a  \rangle}$$
We then use the rule \textsc{OI-Invariant} to arrive at 
$$\onlyIf{a\ :\ \texttt{Account}}{\neg\langle \texttt{encapsulated}\ a  \rangle}{\neg\langle \texttt{encapsulated}\ a  \rangle}$$

In the Safe example, we rely on the fact that the safe does not expose
its secret. Define the following predicate:
$$\langle x\ \texttt{encapsulated\_in}\ y\rangle\ =\ \forall\ z.[\neg \langle z\ \texttt{access}\ x \rangle\ \vee\ y.\texttt{is\_internal}(z)]$$
Now the following invariant is required to hold for the specification to be satisfied:
\small
$$\onlyThrough{s\ :\ \texttt{Safe}}{\neg\langle s.\texttt{secret}\ \texttt{encapsulated\_in}\ s  \rangle}{\neg\langle s.\texttt{secret}\ \texttt{encapsulated\_in}\ s  \rangle}$$
\normalsize
We then use \textsc{OI-Invariant} to get
\small
$$\onlyIf{s\ :\ \texttt{Safe}}{\neg\langle s.\texttt{secret}\ \texttt{encapsulated\_in}\ s  \rangle}{\neg\langle s.\texttt{secret}\ \texttt{encapsulated\_in}\ s  \rangle}$$
\normalsize

It isn't immediately clear how to deal with these invariants, as they both involve quantification, but they both form key parts of the proof 
of satisfaction in their respective examples.


\bibliographystyle{abbrv}
\bibliography{main}
	
	\end{document}