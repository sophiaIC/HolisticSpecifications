\section{Appendix to Section \ref{s:preserve} -- Preservation of Satisfaction }
\label{app:preserve}

 
\beginProof{lemma:addr:expr}

We first prove that for  any $M$ $A$, $\sigma$

\begin{enumerate}
\item
To show that  $ \satisfiesA{M}{\sigma}{A}\ \ \ \ \Longleftrightarrow \ \ \ \ \ \satisfiesA{M}{\sigma}{A[{\interpret \sigma x}/x]} $ 

The proof goes by induction on the structure of $A$,   application of  Defs.  \ref{def:chainmail-semantics}, \ref{def:chainmail-protection-from}, and  \ref{def:chainmail-protection}.
%\item Assume that   $\satisfiesA{M}{\sigma}{A[{\interpret \sigma x}/x]} $. To show that  $ \satisfiesA{M}{\sigma}{A}$. 


\item
To show that $ \satisfiesA{M}{\sigma}{A}   \ \ \ \Longleftrightarrow\ \ \ \satisfiesA{M}{\sigma[\prg{cont}\mapsto stmt]}{A}$ 

The proof goes by induction on the structure of $A$,   application of  Defs.  \ref{def:chainmail-semantics}, \ref{def:chainmail-protection-from}, and  \ref{def:chainmail-protection}.


\end{enumerate}

The lemma itself then follows form (1) and (2) proven above.

\completeProof

 
In addition to what is claimed in Lemma  \ref{lemma:addr:expr}, it  also holds that 
\begin{lemma}
$\eval{M}{\sigma}{{\re}}{\alpha}  \ \ \Longrightarrow\ \  [ \ \satisfiesA{M}{\sigma}{A} \  \Longleftrightarrow\   \  \satisfiesA{M}{\sigma}{A[\alpha/\re]} \  \  ]$
\end{lemma}

\begin{proof} by induction on the structure of $A$,   application of  Defs.  \ref{def:chainmail-semantics}, \ref{def:chainmail-protection-from}, and  \ref{def:chainmail-protection}, and , and auxiliary lemma \ref{aux:lemma:vars:eval}.

\end{proof}

\subsection{Stability}

%The proof of lemma \ref{lemma:addr:expr} is by induction on the structure of $A$.

We first give complete definitions for the concepts of $  \Stable {\_]}$ and $\Pos {\_}$

\vspace{.2cm}

% An assertion is {\emph {stable}}, written as $\Stable A$ if it does not contain $\inside {\re}$ assertions:
\begin{definition}
\label{def:Basic}
%\begin{figure}[hbt]
[$\Stable{\_}$] assertions: %  defined below

$
\begin{array}{l}
 \begin{array}{c}
  \Stable {\inside{\re}}  \triangleq  false \\
    \Stable {\protectedFrom \re {\overline {u}}} =  
  \Stable  {\internal \re} =  %   \PushAS y  {(\external \re)} & \triangleq &   {\external \re}
    \Stable {\re}=   
     \Stable {\re:C}\   \triangleq \    true
 \end{array}
  \\
 \begin{array}{lcl}
 \Stable  {A_1  \wedge  A_2}\  \triangleq\     \Stable  { A_1}  \wedge    \Stable  {A_2}    &
\lclSPACE  &  
 \Stable  {\forall x:C.A} =\Stable  {\neg A} \   \triangleq\   \Stable A
 \end{array}
 \end{array}
$
\label{f:Basic}
 \end{definition}


 \begin{definition}
% \begin{figure}[hbt]
[$\Pos{\_}$] assertions: %  defined below

$
\begin{array}{l}
 \begin{array}{c}
  \Pos {\inside{\re}} =  \Pos {\protectedFrom \re {\overline {u}}} =  
  \Pos  {\internal \re} =   
    \Pos {\re}=   
     \Pos {\re:C}\   \triangleq \    true
     \\
% \end{array}
%  \\
% \begin{array}{lll}
 \Pos  {A_1  \wedge  A_2}\  \triangleq\     \Pos  { A_1}  \wedge    \Pos  {A_2}  
  \\ 
 \Pos  {\forall x:C.A}   \triangleq\   \Pos A
\\ % \   &
  \Pos {\neg A}  \triangleq \Stable A % \Neg A
\end{array}
 \end{array}
 $
 \label{def:Pos}
\end{definition}

%\begin{definition}
%[$\Neg{\_}$] assertions:
%
%$
%\begin{array}{l}
% \begin{array}{lll}
%  \Neg {\inside{\re}}\  \triangleq \    false &    \Neg {\protectedFrom \re {\overline {u}}} =  
%  \Neg  {\internal \re} =   
%    \Neg {\re}=   
%     \Neg {\re:C}\   \triangleq \    true
%\\ 
% \Neg  {A_1  \wedge  A_2}\  \triangleq\     \Neg  { A_1}  \wedge    \Neg  {A_2}   \ \   
% &    
% \Neg  {\forall x:C.A}   \triangleq\   \Neg A \ \ 
%&
%  \Neg {\neg A}  \triangleq \Pos A
% \end{array}
% \end{array}
%$
%\label{def:Neg}
%\end{definition}

The definition  of $\Pos{\_}$ is  less  general than would be   possible. \Eg $(\inside {x} \rightarrow  x.f=4) \rightarrow xf.3=7$  does not satisfy our definition of $\Pos {\_}$.
We have given these less general definitions in order to simplify both our defintions and our proofs.

\beginProof {l:preserve:asrt} 
Take any  state  $\sigma$, frame  $\phi$,  assertion  $A$,  
 
 
\begin{itemize}
\item 
To show\\
  $\Stable{A}\ \wedge \  \fv(A)=\emptyset \ \ \  \Longrightarrow \ \  \ \  [\ \ M, \sigma \models A \ \ \Longleftrightarrow \ \  M,{\PushSLong \phi \sigma} \models A\ \ ]$

By induction on the structure of the definition of $\Stable{A}$.

\item 
To show\\
 $\Pos{A}\ \wedge \  \fv(A)=\emptyset \  \wedge \     {M\cdot\Mtwo \models {\PushSLong \phi \sigma}}\ \wedge\ 
  \ M, \sigma \models A \  \wedge \  M, {\PushSLong \phi \sigma} \models  \intThis \ \ \Longrightarrow$ \\
  $\strut \hspace{2cm}  \ \  M,{\PushSLong \phi \sigma} \models A $

By induction on the structure of the definition of $\Pos{A}$.
The only interesting case is when $A$ has the form $\inside {\re}$. Because 
$ fv(A)=\emptyset$, we know that $\interpret {\sigma} {\re}$=$\interpret {\PushSLong \phi \sigma}  {\re}$. Therefore, we assume that 
 $\interpret {\sigma} {\re} = \alpha$ for some $\alpha$, assume that $ M,  \sigma  \models \inside \alpha$, and want to show that  $ M,{\PushSLong \phi \sigma} \models \inside \alpha$. 
 From $   {M\cdot\Mtwo \models {\PushSLong \phi \sigma}}$ we obtain that
 $Rng(\phi) \subseteq Rng(  \sigma)  $. 
 From this, we obtain that
  $  \LRelevantO {\PushSLong \phi \sigma} \subseteq \LRelevantO  {\sigma}$.
  The rest follows by unfolding and folding Def. \ref{def:chainmail-protection}.

  \end{itemize}
 
\completeProof




\subsection{Encapsulation}

{
Proofs of adherence to {\SpecLang specifications  hinge on the expectation that some, 
specific, assertions cannot be invalidated unless some 
} internal (and thus known) computation took place. 
{We call such assertions   \emph{encapsulated}.}
}
We define the  judgement,  $M\ \vdash  \encaps{A}$, in terms of the judgment  $M; \Gamma \vdash \encaps A  ; \Gamma'$ from Fig. \ref{f:encaps:aux}.
This judgements ensures   that any objects whose fields  are read  in the validation of $A$ are internal, 
that $\protectedFrom {\_} {\_}$  does not appear in $A$,  and that protection assertions (ie $\inside{}$ or $\protectedFrom {\_} {\_}$) do not appear in negative positions in $A$. The second environment in this judgement, $\Gamma'$, is used to keep track of any variables introduces in that judgment, \eg we would have that\\
$\strut \hspace{1cm} M_{good}, \emptyset\ \vdash\ \encaps{a:\prg{Account}\wedge  k:\prg{Key} \wedge \inside{a.\prg{key}} \wedge a.\prg{key}\neq k}; \ (a:\prg{Account}, k:\prg{Key}$.


We assume a type judgment $M; \Gamma \vdash e :  \prg{intl}$ which says that in the context of $\Gamma$, the expression $e$ belongs to a class from $M$.
We also assume that the judgement $M; \Gamma \vdash e :  \prg{intl}$ can deal with ghostfields -- namely, ghost-methods have to be type checked in the contenxt of $M$ and therefor they will only read the state of internal objects.
Note that it is possible for $M; \Gamma \vdash \encaps {\re}; \Gamma'$ to hold and 
$M; \Gamma \vdash  e : \prg{intl}$ not to hold -- \cf rule {\sc{Enc\_1}}.


\begin{figure}[thb]
$
\begin{array}{l}
\begin{array}{lclcl}
\inferruleSDNarrow 
{~ \strut  {\sc{Enc\_1}}}
{  
\begin{array}{l}
M; \Gamma \vdash \re : \prg{intl} \\
M; \Gamma \vdash \encaps{\re};\  \Gamma'
\end{array}
}
{
M; \Gamma \vdash \encaps{\re.f};\  \Gamma'
}
& &
\inferruleSDNarrow 
{~ \strut  {\sc{Enc\_2}}}
{  
\begin{array}{l}
  \\
M; \Gamma \vdash \encaps{\re};\  \Gamma'
\end{array}
}
{
M; \Gamma \vdash \encaps{\re: C};\  (\Gamma', \re:C)
}
& &
\inferruleSDNarrow 
{~ \strut  {\sc{Enc\_3}}}
{   
\begin{array}{l}
M; \Gamma \vdash \encaps{A};\ \Gamma'  \\
 A \mbox{\ does\ not\ contain\ $\inside{\_}$}
% Chooped, superfluous because if 1st requirement
%\\ A \mbox{\ does\ not\ contain\ \ $\protectedFrom {\_} {\_}$}
\end{array}
}
{
M; \Gamma \vdash \encaps{ \neg A};\  \Gamma'  
}
\\ \\
\inferruleSDNarrow 
{~ \strut  {\sc{Enc\_4}}}
{  
\begin{array}{l}
M; \Gamma \vdash \encaps{A_1};\ \Gamma''   \\
  M; \Gamma'' \vdash \encaps{ A_2};\ \Gamma' 
  \end{array} 
}
{
M; \Gamma \vdash \encaps{A_1 \wedge A_2};\  \Gamma'
}
& &
\inferruleSDNarrow 
{~ \strut  {\sc{Enc\_5}}}
{  
\\
M; \Gamma, {x:C} \vdash \encaps {A};\ \Gamma' 
}
{
M; \Gamma \vdash \encaps {\forall {x:C}. A};\  \Gamma'
}
& & 
\inferruleSDNarrow 
{~ \strut  {\sc{Enc\_6}}}
{  
\\
M; \Gamma \vdash \encaps{\re};\  \Gamma'
}
{
M; \Gamma \vdash \encaps{\re: \prg{extl} };\  \Gamma'
}
\end{array}
\\ \\
\inferruleSDNarrow 
{~ \strut  {\sc{Enc\_7}}}
{  
M; \Gamma \vdash \encaps{\re}; \Gamma' 
}
{
M; \Gamma \vdash \encaps{\inside{\re}}; \ \Gamma' 
}
\end{array}
$
\caption{The judgment $M; \Gamma \vdash \encaps  {A}; \Gamma'$}
\label{f:encaps:aux}
\label{f:encaps}
\end{figure}


 

\begin{definition}[An assertion $A$ is \emph{encapsulated} by module $M$] $~$ \\
\label{d:encaps:sytactic}
\begin{itemize}
\item 
$M \vdash \encaps{A}  \ \   \triangleq  \ \  \exists \Gamma.[\ M; \emptyset \vdash \encaps{A}; \Gamma\ ]$ \ \  as defined in Fig. \ref{f:encaps}.
 \end{itemize}
  \end{definition}

To motivate the design of our judgment $M; \Gamma \vdash \encaps{A}; \Gamma'$,  we first give a semantic notion of encapsulation:


\begin{definition}  An assertion $A$ is semantically encapsulated by module $M$:
\label{d:encaps:sem}

\begin{itemize}
\item
$
%\begin{equation} 
    M\ \models \encaps{A}\ \   \triangleq  \   
     \forall \Mtwo, \sigma, \sigma'.[   \ \  \satisfiesA{M}{\sigma}{(A  \wedge \externalexec)}\  \wedge\ { \leadstoBounded {M\madd\Mtwo}  {\sigma}{\sigma'}} % \\    \ \ \ \ \ \   
        \  \Longrightarrow\  
    {M},{\sigma'}\models {\as \sigma A} \ \  ]
%  \end{equation}
  $
\end{itemize}
\end{definition}

\noindent
\textbf{More on Def. \ref{d:encaps:sem}} {If the definition \ref{d:encaps:sem} or in lemma \ref{d:encaps} we had used the more general execution, $\leadstoOrig  {M\madd\Mtwo}  {\sigma}{\sigma'}$, rather than the scoped execution,  $\leadstoBounded {M\madd\Mtwo}  {\sigma}{\sigma'}$,
 then fewer assertions would have been encapsulated.}
Namely, assertions like    $\inside {x.f}$ would not be encapsulated.
Consider, \eg, a heap $\chi$, with objects $1$, $2$, $3$ and $4$, where  $1$, $2$ are external, and $3$, $4$ are internal, and  $1$ has fields pointing to $2$ and $4$, and $2$ has a field pointing to $3$, and $3$ has a field $f$ pointing to $4$. Take  state $\sigma$=$(\phi_1\!\cdot\!\phi_2,\chi)$, where $\phi_1$'s receiver is $1$,  $\phi_2$'s receiver is $2$,   and there are no local variables. 
We have  $...\sigma\models \externalexec \wedge \inside {3.f}$. 
We  return from the most recent all, 
getting  $\leadstoOrig  {...}  {\sigma}{\sigma'}$ where $\sigma'=(\phi_1,\chi)$; and have   $...,\sigma'\not\models  \inside {3.f}$.

\begin{example}
\label{ex:not:encaps}
For an assertion $A_{bal}\  \triangleq\ a:\prg{Account}\wedge a.\prg{balance}=b$, % can only be invalidated through internal  methods.  %on internal objects.
and modules \ModB and  \ModC  from \S~\ref{s:outline}, we have  \ \ \ $\ModB\ \models\ \encaps{ A_{bal} }$, \ \ \ and \ \ \ $\ModB\ \models\ \encaps{ A_{bal} }$.
\end{example}


\begin{example} Assume   further modules, $\ModD$ and $\ModE$,  which  use ledgers mapping  accounts to their balances, and export functions that update this map. In  $\ModD$ the ledger is  part of the {internal} module, %\emph{not} protected, while 
while in $\ModE$ it is part of the  {external} module.
Then  \ \ $\ModD \ \not\models\encaps{ A_{bal}} $, \ \  and \ \ $\ModE  \models \encaps{ A_{bal}} $.
Note that in both $\ModD$ and $\ModE$, the term \prg{a.balance} is a ghost field. 
\end{example}

\begin{note} Relative protection % (a variable is protected from another one) 
is not encapsulated, (\eg $M \not\models {\encaps{\protectedFrom{x}{y}}}$), even though    absolute protection is
(\eg $M \models \encaps{\inside{x}}$).
Encapsulation of an assertion does not imply encapsulation of its negation; 
 for example,  $M \not\models {\encaps{\neg\inside{x}}}$.
\end{note}

\noindent
\textbf{More on Def. \ref{d:encaps:sytactic}} This definition is less permissive than necessary. 
For example $M \not\vdash \encaps{\neg ( {\neg \inside {x}})}$ even though 
 $M  \models \encaps{\neg ( {\neg \inside {x}})}$.
 Namely, 
$\neg (\neg \inside {x}) \equiv  \inside {x}$ and $M \vdash {\encaps{  \inside {x}}}$.
A more permissive, sound, definition, is not difficult, but not the main aim of this work.
We gave this, less permissive definition, in order to simplify the definitions and the proofs. 


 \vspace{.1cm}

 \beginProof{lem:encap-soundness} This says that $M \vdash {\encaps A}$ implies that $M \vdash {\encaps A}$.
 \\
 We fist prove that\\
 $\strut \ \ \ \ $  (*) Assertions $A_{poor}$ which do not contain $\inside {\_}$ or $\protectedFrom {\_} {\_}$ are preserved by any external step.\\
 Namely, such an assertion only depends on the contents of the fields of internal objects, and these are not modified  by external steps.
 Such an $A_{poor}$ is defined essentially through
\\ 
 $
\begin{syntax}
\syntaxElement{\ \ \ \  A_{poor}}{}
		{
		\syntaxline
				{{\re}}
				{{\re} : C}
				{\neg A_{poor}}
				{A_{poor}\ \wedge\ A_{poor}}
				{\all{x:C}{A_{poor}}}
				{\external{{\re}}}
		\endsyntaxline
		}
\endSyntaxElement\\
\end{syntax}
$
\\
 We can prove (*) by induction on the structure of $A_{poor}$ and case analysis on the execution step.
  
\vspace{.05cm}   
\noindent
We then prove Lemma \ref{lem:encap-soundness} by induction on the structure of $A$. 

\noindent 
--- The cases {\sc{Enc\_1}}, {\sc{Enc\_2}},  and {\sc{Enc\_6}} 
are  straight application of (*). 

\noindent 
--- The case {\sc{Enc\_3}} also follows from (*), because any $A$ which satisfies $\encaps {A}$ and which does not contain $\inside {\_}$
is an $A_{poor}$ assertion.

\noindent 
--- The cases {\sc{Enc\_4}} and {\sc{Enc\_5}}  follow  by induction hypothesis.

\noindent 
--- The case {\sc{Enc\_7}} is more interesting. \\
We assume that $\sigma$ is an external state,   that $\leadstoBounded {...} {\sigma} {\sigma'}$, and that $.. \sigma \models \inside \re$.
By definition, the latter means that\\
 $\strut \ \ \ \ $  (**) no locally reachable external object in $\sigma$ has a field ponting to $\re$, \\
 $\strut \ \ \ \ \ \ \ \ \ \ \ $ nor is $\re$ one of the variables.\\
We proceed by case analysis on the step $\leadstoBounded {...} {\sigma} {\sigma'}$.
\\
- If that step was an assignment to a local variable $x$, then this does not affect $\inside {\as \sigma \re}$ because in $\interpret \sigma {\re}$=$\interpret {\sigma'} {\as \sigma {\re}}$,
and $... \sigma' \models x \neq re$.
\\
- If that step was an assignment to an external object's field, of the form $x.f :=y$ then this does not affect $\inside {\as \sigma \re}$ either.
This is so, because $\encaps {\re}$ gives that $\interpret \sigma {\re}$=$\interpret \sigma' {\as \sigma {\re}}$ -- namely the evaluation of $\re$ does not read $x$'s fields, since $x$ is external. 
And moreover, the assignment $x.f :=y$ cannot create a new, unprotected  path to $\re$ (unprotected means here that the penultimate element in that path is external), because then we would have had in $\sigma$  an unprotected path from $y$ to $\re$.
\\
- If that step was a method call, then we apply lemma \ref{lemma:push:N} which says that all objects reachable in $\sigma'$ were already reachable in $\sigma$.
\\
- Finally, we do not consider method return (\ie the rule {\sc{Return}}),   because we are looking at $\leadstoBounded {...} {\_} {\_}$ execution steps rather than $\leadstoOrig  {...}  {\_}{\_}$ steps.
\hspace{3cm}
\completeProof

 

%\TODO{{\red{TODO}: Say that there are more ways to define the encaps judgement. ALSO: show the rule for %the ghostfields. }}

