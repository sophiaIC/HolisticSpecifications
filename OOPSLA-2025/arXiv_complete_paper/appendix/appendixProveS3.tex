

\subsection{Demonstrating that $M_{good} \vdash \SThree$, and that $M_{fine} \vdash \SThree$}
 \label{s:app:example:proofs:S3}


 \subsection{Extending the specification $\SThree$}
\label{s:extend:spec:three}

As in \S \ref{s:extend:spec}, we redefine $\SThree$ so that it also describes the behaviour of method \prg{send}. and have:
\\
%{\sprepost
%		{\strut \ \ \ \ \ \ \ \ \ S_{3a}} 
%		{ \prg{a}:\prg{Account}, \prg{e}:\prg{external}, \prg{b}:\prg{int} \ \wedge \protectedFrom{\prg{a.key}} {\prg{e} }\ \wedge\ \prg{a}.\balance \geq\prg{b}\  } % \wedge \prg{a.blnce}=\prg{b} }
%		 {\prg{private Shop}}
%		 {\prg{send}}
%		 {\prg{buyer}:\prg{external},\prg{anItem}:\prg{Item} }
%		 { \protectedFrom{\prg{a.key}} {e} \wedge   \prg{a}.\balance \geq\prg{b}\ } 
%		 { \protectedFrom{\prg{a.key}} {e}\wedge   \prg{a}.\balance \geq\prg{b}\ } 
%}
%\\
%{\sprepost
%		{\strut \ \ \ \ \ \ \ \ \ S_{2b}} 
%		{ \prg{a}:\prg{Account} \wedge \prg{a.blnce} =\prg{b} }
%		 {\prg{private Shop}}
%		 {\prg{send}}
%		 {\prg{buyer}:\prg{external}, \prg{anItem}:\prg{Item} }
%		 { \prg{a.blnce} =\prg{b} }
%		{   \prg{a.blnce} =\prg{b} }
%}
\\
$\strut  \SPSP  \SThreeStrong \ \  \triangleq \ \ \ \SThree \ \wedge \ S_{2a} \ \wedge \ S_{2b} $

%$\strut  \SPSP  \STwo\ \  \triangleq \ \ \TwoStatesN   \ \wedge\  
%{ \mprepostN {\prg{a:Account}\wedge \protectedFrom{\prg{a.key}} {\prg{buyer}}}
%                     {A_2} {A_3} }\ $ 


  

\begin{lemma}[module $M_{good}$  satisfies $\SThreeStrong$]
\label{lemma:S3}
\label{l:Mgood:S3}
$M_{good} \vdash \SThreeStrong$
\end{lemma}
\begin{proofO}
%We construct our proof tree using a top down approach.  
In order to prove that 
$$M_{good} \vdash \TwoStatesN {\prg{a}:\prg{Account}, b:\prg{int} }{\  \inside{\prg{\prg{a.key}}} \wedge \prg{a.\balance}\geq b\ }$$
we have to apply  \textsc{Invariant} from Fig. \ref{f:wf}.
 That is, for each  class $C$  defined in $M_{good}$, and for each public method $m$ in $C$, with parameters $\overline{y:D}$, we have to prove that they satisfy the corresponding quadruples.
 

% \small
%\begin{align*}
%M_{good}\ \vdash \ \ &   \{ \ \prg{this}:\prg{C},\, \overline{y:D},\, \prg{a}:\prg{Account}\, \wedge\,
%		             {\inside{\prg{a.key}}}\ \wedge\       \protectedFrom {\prg{a.key}} {(\prg{this},\overline y)} \  \} \\
%		& \SPT  \prg{C}::\prg{m}.\prg{body}\  \\
%		&
%                   \{\ {\inside{\prg{a.key}}}\ \wedge\ \ \protectedFrom {\prg{a.key}} {\prg{res}}   \ \}\ ||\ \{\ {\inside{\prg{a.key} } } \ \} \\
%\end{align*}


\normalsize
Thus, we need to prove  three Hoare quadruples: one for \prg{Shop::buy}, one for  \prg{Account::transfer}, and one for  \prg{Account::set}.  That is, we have to prove that
 \small
\begin{align*}
\text{(31?)}  \ \ \ \ M_{good}\ \vdash  \  \ 
		&	\{  \ \Alocals, \, \prg{a}:\prg{Account}, b:\prg{int} \, \wedge\, {\inside{\prg{a.key}}} \, \wedge \, \protectedFrom {\prg{a.key}} {\Ids} \wedge \prg{a.\balance}\geq b \  \} \\
		& \SPT \prg{Shop}::\prg{buy}.\prg{body}\ \\  
		& \{ {\inside{\prg{a.key}}}\ \wedge\ {\PushASLong {\prg{res}} {\inside{\prg{a.key}}}}  \wedge \prg{a.\balance}\geq b \} \ \ \  || \ \ \ 
		   \{ {\inside{\prg{a.key}}}  \wedge \prg{a.\balance}\geq b \}
\\
\text{(32?)}  \ \ \ \ M_{good} \vdash \ 
		&	\{  \ \Alocalstr, \, \prg{a}:\prg{Account}\, , b:\prg{int} \, \wedge\,  {\inside{\prg{a.key}}} \, \wedge \, \protectedFrom {\prg{a.key}} {\Idstr}  \ \wedge \prg{a.\balance}\geq b \  \} \\
		& \SPT \prg{Account}::\prg{transfer}.\prg{body}\ \\  
		& \{ {\inside{\prg{a.key}}}\ \wedge\ {\PushASLong {\prg{res}} {\inside{\prg{a.key}}} } \wedge \prg{a.\balance}\geq b \  \} \ \ \  || \ \ \ 
		   \{ {\inside{\prg{a.key}}} \wedge \prg{a.\balance}\geq b \  \}
\\
\text{(33?)}  \ \ \ \ M_{good} \vdash \ 
		&	\{  \ \Alocalsset, \, \prg{a}:\prg{Account}\, , b:\prg{int} \, \wedge\,  {\inside{\prg{a.key}}} \, \wedge \, \protectedFrom {\prg{a.key}} {\Idsset}  \ \wedge \prg{a.\balance}\geq b \  \} \\
		& \SPT \prg{Account}::\prg{set}.\prg{body}\ \\  
 		& \{ {\inside{\prg{a.key}}}  \wedge\ {\PushASLong {\prg{res}} {\inside{\prg{a.key}}}}   \wedge \prg{a.\balance}\geq b \  \}\ \ \  || \ \ \ 
		   \{ {\inside{\prg{a.key}}} \wedge \prg{a.\balance}\geq b \  \} 
\end{align*}
\normalsize
where we are using ? to indicate that this needs to be proven, and 
where we are using the shorthands $\Alocals$,   $\Ids$, $\Alocalstr$, $\Idstr$, $\Alocalsset$ as defined earlier.

 \end{proofO}
 
The proofs for $M_{fine}$ are similar.

%
 We outline the proof of (31?) in Lemma \ref{l:buy:sat:S3}. %  in \S \ref{s:buy:sat:S3}.
 We outline the proof of (32?) in  Lemma \ref{l:transfer:sat:S3}.

\subsubsection{Proving that \prg{Shop::buy} from $M_{good}$ satisfies $\SThreeStrong$ and also $S_4$}
\label{s:buy:sat:S3}

\begin{lemma}[function $M_{good}::\prg{Shop}::\prg{buy}$  satisfies $\SThreeStrong$ and also $S_4$]
\label{l:buy:sat:S3}
 
\begin{align*}
\text{(31)}  \ \ \ \ M_{good}\ \vdash  \  \ 
		&	\{  \ \Alocals, \, \prg{a}:\prg{Account}, b:\prg{int}, \, \wedge\, {\inside{\prg{a.key}}} \, \wedge \, \protectedFrom {\prg{a.key}} {\Ids} \wedge \prg{a.\balance}\geq b \  \} \\
		& \SPT \prg{Shop}::\prg{buy}.\prg{body}\ \\  
		& \{ {\inside{\prg{a.key}}}\ \wedge\ {\PushASLong {\prg{res}} {\inside{\prg{a.key}}}}  \wedge \prg{a.\balance}\geq b \} \ \ \  || \ \ \ 
		   \{ {\inside{\prg{a.key}}}  \wedge \prg{a.\balance}\geq b \}
\end{align*}

\end{lemma}

\begin{proofO}
Note that (31) is a proof that $M_{good}::\prg{Shop}::\prg{buy}$  satisfies $\SThreeStrong$ and also  hat $M_{good}::\prg{Shop}::\prg{buy}$  satisfies $S_4$. This is so, because application of {\sc{[Method]}} on $S_4$ gives us exactly the proof obligation from (31).

This proof is similar to the proof of lemma \ref{l:buy:sat}, with the extra requirement here that we need to argue about balances not decreasing.
To do this, we will leverage the assertion about balances given in $S_3$.

We will use the shorthand $\Alocalsb \triangleq \Alocals, \,\prg{a}:\prg{Account}, b:\prg{int}$.
We will split the proof into 1) proving that statements 10, 11, 12 preserve the protection of \prg{a.key} from the \prg{buyer}, 2) proving that the external call 

\step{1st Step: proving statements 10, 11, 12}

We apply the underlying Hoare logic and prove that the statements on lines 10, 11, 12 do not affect the value of \prg{a.key} nor that of \prg{a.\balance}.  Therefore, for a $z,z'\notin \{ \prg{price}, \prg{myAccnt}, \prg{oldBalance} \}$, we have 

\begin{align*}
\text{(40)}  \ \ \ \ {M_{good} \vdash_{ul}} 
		&	\{  \ \Alocalsb\  \wedge\ z=\prg{a.key} \wedge\ z'=\prg{a.\balance}  \} \\
		&   \SPT \prg{price:=anItem.price}; \\  
		&   \SPT \prg{myAccnt:=this.accnt}; \\  
                 &   \SPT \prg{oldBalance := myAccnt.blnce};\\
		& \{ z=\prg{a.key} \wedge\ z'=\prg{a.\balance} \}
\end{align*}

We then apply {\sc{Embed\_UL}}, {\sc{Prot-1}} and {\sc{Prot-2}} and {\sc{Combine}} and and {\sc{Types-2}} on (10) and use the shorthand $\stmtsP$ for the statements on lines 10, 11 and 12, and obtain: 
\\
\begin{align*}
\text{(41)}  \ \ \ \ M_{good} \vdash 
		&	\{  \ \Alocalsb\  \wedge\ {\inside{\prg{a.key}}} \, \wedge\, \protectedFrom{\prg{buyer}} {\prg{a.key}} \wedge\ z'=\prg{a.\balance}  \} \\
		& \SPT \stmtsP\ \\  
		& \{ \ {\inside{\prg{a.key}}}  \, \wedge\, \protectedFrom{\prg{buyer}} {\prg{a.key}} \wedge\ z'=\prg{a.\balance}   \}
\end{align*}



We apply  {\sc{Mid}}  on (11) and obtain 
\begin{align*}
\text{(42)}  \ \ \ \ M_{good} \vdash 
		&	\{  \ \Alocalsb\, \wedge\, \protectedFrom {\prg{a.key}} {\prg{buyer}}\ \wedge\ z'=\prg{a.\balance}  \} \\
		& \SPT \stmtsP\ \\  
		& \{ \ \Alocalsb\, \wedge \  {\inside{\prg{a.key}}} \, \wedge\, \protectedFrom{\prg{buyer}} {\prg{a.key}}  \ \wedge\ z'=\prg{a.\balance}  \} \ \ || \\
		& \{ \ {\inside{\prg{a.key}}}\  \wedge\ z'=\prg{a.\balance} \}
\end{align*}

\step{2nd Step: Proving the External Call}

 

We now need to prove that the external method call \prg{buyer.pay(this.accnt, price)} protects the \prg{key}, and does nit decrease the balance, i.e.
\small
\begin{align*}
\text{(43?)} \ \ \ M_{good} \vdash & \{ \ \Alocalsb \   \wedge\    {\inside{\prg{a.key}}} \, \wedge\, \protectedFrom {\prg{a.key}} {\prg{buyer}} \wedge\ z'= \prg{a.\balance}  \} \\
		  		& \SPT  \prg{tmp := buyer.pay(myAccnt, price)}\ \\  
		  		& \{ \ \ \ \Alocalsb \ \wedge\ {\inside{\prg{a.key}}} \, \wedge\, \protectedFrom{\prg{buyer}} {\prg{a.key}} \wedge \ \prg{a.\balance}\geq z'\  \} \ \ \  || \ \ \  \\
		  		&   \{ \   {\inside{\prg{a.key}}}\ \wedge \  \prg{a.\balance}\geq z'  \}
\end{align*}
\normalsize

We use that $M \vdash \TwoStatesN  {\prg{a}:\prg{Account},\prg{b}:\prg{int}, }  {\inside{\prg{a.key}} \wedge \prg{a.\balance}\geq z'}   $
 and  obtain
 \\
 \small
\begin{align*}
\text{(44)} \ \ \ M_{good} \vdash & \{ \ \prg{buyer}:\prg{external},\,  {\inside{\prg{a.key}}} \, \wedge\, 
\protectedFrom {\prg{a.key}} {(\prg{buyer},\prg{myAccnt},\prg{price})} \  \wedge\ z'\geq \prg{a.\balance}  \} \\
		  		& \SPT  \prg{tmp := buyer.pay(myAccnt, price)}\ \\  
		  		& \{ \ \inside{\prg{a.key}} \, \wedge\, 
\protectedFrom {\prg{a.key}} {(\prg{buyer},\prg{myAccnt},\prg{price})}\ \wedge\ z'\geq \prg{a.\balance}  \} \ \ \  || \ \ \  \\
		  		&   \{ \   {\inside{\prg{a.key}}}\  \wedge\ z'\geq \prg{a.\balance}  \}
\end{align*}
\normalsize 
 
In order to obtain (43?) out of (44), we use the type system and type declarations and obtain that all objects transitively reachable from
\prg{myAccnt} or \prg{price} are scalar or internal. Thus, we 
apply \textsc{Prot-Intl}, %and \textsc{Prot-Int}$_1$,  
and obtain\\
$
\begin{array}{llll}
& (45) & & M_{good} \vdash \Alocalsb \wedge  {\inside{\prg{a.key}}}\  \longrightarrow\ \protectedFrom {\prg{a.key}} {\prg{myAccnt}} 
\\
& (46) & & M_{good} \vdash \Alocalsb \wedge  {\inside{\prg{a.key}}}\  \longrightarrow\ \protectedFrom {\prg{a.key}} {\prg{price}} 
\\
& (47) & & M_{good} \vdash \Alocalsb \wedge  z'= \prg{a.\balance}\   \longrightarrow\  z'\geq \prg{a.\balance} 
\end{array}
$

We apply {\textsc{Consequ}} on (44), (45), (46) and (47) and obtain (43)!

\normalsize

\step{3nd Step: Proving the Remainder of the Body}

 We now need to prove that lines 15-19 of the method preserve the protection of \prg{a.key}, and do not decrease \prg{a.balance}.
 We outline the  remaining proof in less detail.
 
 We prove the internal call on line 16, using the method specification for \prg{send}, using $S_{2a}$ and $S_{2b}$, and applying rule {\sc{[Call\_Int]}}, and obtain

 \small
\begin{align*}
\text{(48)} \ \ \ M_{good}\  \vdash \ & \{ \ \prg{buyer}:\prg{external},\, \prg{item}:\prg{Intem} \wedge {\inside{\prg{a.key}}} \, \wedge\, 
\protectedFrom {\prg{a.key}} {(\prg{buyer}}  \  \wedge\ z'= \prg{a.\balance}  \} \\
		  		& \SPT  \prg{tmp := this.send(buyer,Item)}\ \\  
		  		& \{ \ \inside{\prg{a.key}} \, \wedge\, 
\protectedFrom {\prg{a.key}} {\prg{buyer}} \ \wedge\ z'= \prg{a.\balance}  \} \ \ \  || \ \ \  \\
		  		&   \{ \   {\inside{\prg{a.key}}}\  \wedge\ z'= \prg{a.\balance}  \}
\end{align*}
\normalsize  
 

We now need to prove that the external method call \prg{buyer.tell("You have not paid me")} also protects the \prg{key}, and does nit decrease the balance. We can do this by applying the rule about protection from strings  {\sc{[Pror\_Str]}}, the fact that $M_{good} \vdash S_{3}$, and rule  {\sc{[Call\_Extl\_Adapt]}} and obtain:


 \small
\begin{align*}
\text{(49)} \ \ \ M_{good}\  \vdash \ & \{ \ \prg{buyer}:\prg{external},\, \prg{item}:\prg{Intem} \wedge {\inside{\prg{a.key}}} \, \wedge\, 
\protectedFrom {\prg{a.key}} {(\prg{buyer}}  \  \wedge\ z'/geq  \prg{a.\balance}  \} \\
		  		& \SPT  \prg{tmp:=buyer.tell("You have not paid me")}\ \\  
		  		& \{ \ \inside{\prg{a.key}} \, \wedge\, 
\protectedFrom {\prg{a.key}} {\prg{buyer}} \ \wedge\ z'\geq \prg{a.\balance}  \} \ \ \  || \ \ \  \\
		  		&   \{ \   {\inside{\prg{a.key}}}\  \wedge\ z'\geq  \prg{a.\balance}  \}
\end{align*}
\normalsize 

We can now apply  {\sc{[If\_Rule}}, and {\sc{[Conseq}} on (49) and (50),  and obtain

 \small
\begin{align*}
\text{(50)} \ \ \ M_{good}\  \vdash \ & \{ \ \prg{buyer}:\prg{external},\, \prg{item}:\prg{Intem} \wedge {\inside{\prg{a.key}}} \, \wedge\, 
\protectedFrom {\prg{a.key}} {(\prg{buyer}}  \  \wedge\ z'\geq  \prg{a.\balance}  \} \\
		  		& \SPT  \prg{if} ... \prg{then}\\
				& \SPT \SPT  \prg{tmp:=this.send(buyer,anItem)}\ \\  
				& \SPT  \prg{else}\\
				& \SPT \SPT  \prg{tmp:=buyer.tell("You have not paid me")}\ \\  
		  		& \{ \ \inside{\prg{a.key}} \, \wedge\, 
\protectedFrom {\prg{a.key}} {\prg{buyer}} \ \wedge\ z'\geq \prg{a.\balance}  \} \ \ \  || \ \ \  \\
		  		&   \{ \   {\inside{\prg{a.key}}}\  \wedge\ z'\geq  \prg{a.\balance}  \}
\end{align*}
\normalsize 

The rest follows through application of {\sc{[Prot\_Int}}, and {\sc{[Seq]}}.



\end{proofO}

\begin{lemma}[function $M_{good}::\prg{Account}::\prg{transfer}$ satisfies $S_3$]
\label{l:transfer:sat:S3}
 \small
 \begin{align*}
\text{(32)}  \ \ \ \ M_{good} \vdash \ 
		&	\{  \ \Alocalstr, \, \prg{a}:\prg{Account}\, , b:\prg{int} \, \wedge\,  {\inside{\prg{a.key}}} \, \wedge \, \protectedFrom {\prg{a.key}} {\Idstr}  \ \wedge \prg{a.\balance}\geq b \  \} \\
		& \SPT \prg{Account}::\prg{transfer}.\prg{body}\ \\  
		& \{ {\inside{\prg{a.key}}}\ \wedge\ {\PushASLong {\prg{res}} {\inside{\prg{a.key}}} } \wedge \prg{a.\balance}\geq b \  \} \ \ \  || \ \ \ 
		   \{ {\inside{\prg{a.key}}} \wedge \prg{a.\balance}\geq b \  \}
\end{align*}
\normalsize


\end{lemma}

\begin{proofO}
We will use   the shorthand $stmts_{28-33}$ for the statements in the body of \prg{transfer}. 
We will prove   the preservation of protection, separately from the balance not decreasing when the key is protcted.

For the former, applying the steps in the proof of Lemma \ref{l:transfer:S2},  we obtain

\small
 \begin{align*}
\text{(21)}  \ \ \ \ M_{good} \vdash \ 
		&	\{  \ \Alocalstr, \, \prg{a}:\prg{Account}\, \wedge\,  {\inside{\prg{a.key}}} \, \wedge \, \protectedFrom {\prg{a.key}} {\Idstr}  \  \} \\
		&  \SPT   stmts_{28-33} \\
		& \{ {\inside{\prg{a.key}}}\ \wedge\ {\PushASLong {\prg{res}} {\inside{\prg{a.key}}}}  \} \ \ \  || \ \ \ 
		   \{ {\inside{\prg{a.key}}} \}
\end{align*}
\normalsize

For the latter, we  rely on the underlying Hoare logic to ensure that no balance decreases, except perhaps that of the receiver, in which case its key was not protected.
Namely, we have that

\small
 \begin{align*}
 \text{(71)}  \ \ \ \ M_{good} \vdash_ul \ 
		&	\{  \ \Alocalstr, \, \prg{a}:\prg{Account}\, \wedge\,  \prg{a.\balance}=b\ \wedge \ (\prg{this}\neq \prg{a} \vee prg{this}.\prg{key}\neq \prg{key}' ) \  \} \\
		&  \SPT   stmts_{28-33} \\
		& \{ \prg{a.\balance}\geq b \}
\end{align*}
\normalsize

We apply rules {\sc{Embed\_UL}} and   {\sc{Mid}} on (71), and obtain

\small
 \begin{align*}
 \text{(72)}  \ \ \ \ M_{good} \vdash  \ 
		&	\{  \ \Alocalstr, \, \prg{a}:\prg{Account}\, \wedge\,  \prg{a.\balance}=b\ \wedge \ (\prg{this}\neq \prg{a} \vee prg{this}.\prg{key}\neq \prg{key}' ) \  \} \\
		&  \SPT   stmts_{28-33} \\
		& \{ \prg{a.\balance}\geq b \} \ \ \  || \ \ \  \{ \prg{a.\balance}\geq b \}
\end{align*}
\normalsize

Moreover, we have 

\small
$\begin{array}{llll}
 \text{(73)} \ \ \  & M_{good}\  & \vdash  \ & \protectedFrom {\prg{a.key}} {\Idstr}\ \  \rightarrow \ \ \protectedFrom {\prg{a.key}} {\prg{key'}} \\
 \text{(74)} &  M_{good}  & \vdash  & \protectedFrom {\prg{a.key}} {\prg{key'}} \ \  \rightarrow\ \  \prg{a.key} \neq \prg{key'}  \\
  \text{(75)} &  M_{good}  & \vdash  &    \prg{a.key} \neq  \prg{key'}  \ \ \rightarrow \ \ \prg{a}\neq \prg{this} \vee \prg{this.key}\neq \prg{key}'
\end{array}
$

normalsize

Applying (73), (74), (75) and {\sc{Conseq}} on (72) we obtain:

\small
 \begin{align*}
 \text{(76)}  \ \ \ \ M_{good} \vdash  \ 
		&	\{  \ \Alocalstr, \, \prg{a}:\prg{Account}\, \wedge\,  \prg{a.\balance}=b\ \wedge \ \protectedFrom {\prg{a.key}} {\Idstr}\ \  \} \\
		&  \SPT   stmts_{28-33} \\
		& \{ \prg{a.\balance}\geq b \} \ \ \  || \ \ \  \{ \prg{a.\balance}\geq b \}
\end{align*}
\normalsize

We combine  (72) and (76) through {\sc{Combine}} and obtain (32).

\end{proofO}

\subsection{Dealing with polymorphic function calls}
\label{app:polymorphic}

The case split rules together with the rule of consequence allow our Hoare logic to formally reason about polymorphic calls, where the receiver may be internal or external.

We demonstrate this through an example where we may have an external receiver, or a receiver from a class $C$. Assume we had a module $M$ with a scoped invariant (as in A), and an internal method specification as in (B). 

$\begin{array}{lclcl}
& (A)& M & \vdash & \TwoStatesN{y_1:D}{A} \\
& (B) & M & \vdash & \mprepostLong {A_1}{ \prg{p}\ C} {m}   {y_1: D} {A_2} { \parallel \{A_3\}}  \\
\\
  \end{array}
$
\\
Here \prg{p} may be \prg{private} or \prg{ublic}; the argument apples either way.

\vspace{.1cm}

\noindent 
Assume also implications as in (C)-(H)

$\begin{array}{lclcl}
&  (C) & M & \vdash & A_0 \ \rightarrow \ \PushASLong {(y_0,y_1)}{A}\\
&  (D) & M & \vdash &  \PushASLong {(y_0,y_1)}{A} \rightarrow A_4\\
 & (E) &  M & \vdash & A \rightarrow A_5\\
  & (F) &  M & \vdash & A_0 \rightarrow A_1[y_0/\prg{this}]\\
  & (G) &  M & \vdash & A_2[y_0,u/\prg{this},res] \rightarrow A_4 \\
    & (H) &  M & \vdash & A_3  \rightarrow A_5 
\\
  \end{array}
$

Then, by application of {\sc{Call\_Ext\_Adapt}}  on (A) we obtain (I)

$
\SPT (I)\ \ {  \hprovesN {M} 
						{ \  y_0:external, y_1: D \wedge  {\PushASLong {(y_0,y_1)} {A}}\ }  
						 { \ u:=y_0.m(y_1)\    }
					         { \   \PushASLong {(y_0,y_1)}{A} \ } 
						{  \ A \ }  }
						\\
						$
						
By application of the rule of consequence on (I) and (C), (D), and (E), we obtain

$
\SPT (J)\ \ {  \hprovesN {M} 
						{ \  y_0:external, y_1: D \wedge  A_0\ }  
						 { \ u:=y_0.m(y_1)\    }
					         { \   A_4 \ } 
						{  \ A_5\ }  }
						\\
						$

Then, by application of {\sc{[Call\_Intl]}}  on (B) we obtain (K)

$
\SPT (K)\ \ {  \hprovesN {M} 
						{ \  y_0:C, y_1: D \wedge  A_1[y_0/\prg{this}]\ }  
						 { \ u:=y_0.m(y_1)\    }
					         { \  A_2[y_0,u/\prg{this},res] \ } 
						{  \ A_3 \ }  }
						\\
						$
						
By application of the rule of consequence on (K) and (F), (G), and (H), we obtain

$
\SPT (L)\ \ {  \hprovesN {M} 
						{ \  y_0:C, y_1: D \wedge  A_0\ }  
						 { \ u:=y_0.m(y_1)\    }
					         { \  A_4 \ } 
						{  \ A_5 \ }  }
						\\
						$

By case split, {\sc{[Cases]}}, on (J) and (L), we obtain

$
\SPT (polymoprhic)\ \ {  \hprovesN {M} 
						{ \  (y_0: external \vee y_0:C), y_1: D \wedge  A_0\ }  
						 { \ u:=y_0.m(y_1)\    }
					         { \  A_4 \ } 
						{  \ A_5 \ }  }
						\\
						$

