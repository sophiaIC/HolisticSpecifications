 \newcommand{\ttt}{\prg{true}}
\newcommand{\ff}{\prg{false}}
\newcommand{\unkn}{\prg{b???}}
\newcommand{\bv}{\prg{bval}}

\newcommand{\sdparagraph}[1]{{\vspace{.3cm} \noindent \textit{#1}\ }}

\newcommand{\prg}[1]{{\mbox{\tt{#1}}}}

\newcommand{\m}{\prg{m}}
 \newcommand{\f}{\prg{f}}
  \newcommand{\e}{\prg{e}}
 \renewcommand{\c}{\prg{C}}
 \renewcommand{\v}{\prg{v}}
  \newcommand{\x}{\prg{x}}
  \newcommand{\p}{\prg{p}}
   \newcommand{\y}{\prg{y}}
      \newcommand{\uu}{\prg{u}}
  %  \newcommand{\z}{\prg{z}}
  \newcommand{\this}{\prg{this}}
  \newcommand{\caller}{\kw{caller}}
   \newcommand{\nullK}{\prg{null}}
\newcommand{\addr}{\ensuremath{\alpha}}


\newcommand{\fldMap}{\textit{fldMap}}

\newcommand{\forget}[1]{}
\newcommand{\etc}{{\it etc.}}
\newcommand{\eg}{{\it e.g.\,}}
\newcommand{\ie}{{\it i.e.\,}}

% new macros for holistic
\newcommand{\Future}[1] {\ensuremath{{\mathsf{will}}\langle \,#1\,\rangle}}
\newcommand{\Using}[2]{\ensuremath{\langle\, #1\, \mathsf{in}\, #2\, \rangle}}
\newcommand{\External}[1] {\ensuremath{{\mathsf{external}}\langle\,  #1\, \rangle}}
\newcommand{\Internal}[1] {\ensuremath{{\mathsf{internal}}\langle\,  #1\, \rangle}}
\newcommand{\Changes}[1]{\ensuremath{{\mathsf{changes}}\langle\,#1\,\rangle}}
\newcommand{\CanAccessTr}[2]{\ensuremath{\langle\, #1\, \mathsf{access}^*\, #2\, \rangle}}
\newcommand{\CanAccess}[2]{\ensuremath{\langle\, #1\, \mathsf{access}\, #2\, \rangle}}

\newcommand{\Calls}[4]{\ensuremath{\langle\, #1\, \mathsf{calls}\, #3.#2\lp#4\rp\, \rangle}}%(\!({#1})\!)}
\newcommand{\Next}[1] {\ensuremath{{\mathsf{next}}\langle \,#1\,\rangle}}%(\!({#1})\!)}
\newcommand{\PrevId}{\ensuremath{{\mathcal{P}}\textrm{\textit{rev}}}}
\newcommand{\Prev}[1] {\ensuremath{{\mathsf{prev}}\langle \,#1\,\rangle}}%(\!({#1})\!)}
\newcommand{\Past}[1]  {\ensuremath{{\mathsf{was}}\langle \,#1\,\rangle}}



% old macros for holistic
%\newcommand{\Future}[1] 
%{{{\mathcal W}\!ill}\langle#1\rangle}%{\lozenge\, #1}% {\bullet #1}% {{{\mathcal F}}(#1)} % {{{\mathcal B}}(#1)}
%\newcommand{\Using}[2]{{\mathcal W}ith\langle\ #2,\,#1\ \rangle}
% \newcommand{\External}[1]{{\mathcal E}xternal\langle #1\rangle}
%\newcommand{\Using}[2]{#1\,{\mathcal U}sing\, #2}
%\newcommand{\Using}[2]{#1\,{\mathcal U}sing\, #2} %{{{\mathcal U}}(#1,#2)}
% \newcommand{\Changes}[1]{\ensuremath{\mathcal{C}\textrm{\textit{hange}}\langle#1\rangle}}
%\newcommand{\CanAccessTr}[2]{\ensuremath{\mathcal{A}}\textrm{\textit{ccess}}^*\langle #1,#2\rangle}
%\newcommand{\CanAccess}[2]{\ensuremath{{\mathcal{A}}\textrm{\textit{ccess}}}\langle #1,#2\rangle}%(#1,#2)}
%\newcommand{\Calls}[4]{\ensuremath{{\mathcal{C}}\textrm{\textit{alls}}}\langle #1,#2,#3,#4\rangle}%(\!({#1})\!)}
%\newcommand{\Next}[1]{\ensuremath{{\mathcal{N}}\textrm{\textit{ext}}}\langle #1\rangle}%(\!({#1})\!)}
%\newcommand{\PrevId}{\ensuremath{{\mathcal{P}}\textrm{\textit{rev}}}}
%\newcommand{\Prev}[1]{\ensuremath{{\mathcal{P}}\textrm{\textit{rev}}}\langle #1\rangle}%(\!({#1})\!)}
%\newcommand{\Caller}{\ensuremath{{\mathcal{C}}\textrm{\textit{aller}}}}

%\newcommand{\VisibleLit}{\ensuremath{\mathcal{V}\textrm{\textit{isible}}}}
%\newcommand{\Next}[1]{\ensuremath{{\mathcal{N}}\textrm{\textit{ext}}}\langle #1\rangle}%(\!({#1})\!)}
%\newcommand{\PrevId}{\ensuremath{{\mathcal{P}}\textrm{\textit{rev}}}}
%\newcommand{\Prev}[1]{\ensuremath{{\mathcal{P}}\textrm{\textit{rev}}}\langle #1\rangle}%(\!({#1})\!)}
%\newcommand{\Caller}{\ensuremath{{\mathcal{C}}\textrm{\textit{aller}}}}
% \newcommand{\Past}[1] {{{\mathcal W}\!as}\langle#1\rangle}% {\nabla #1} %{\lozenge\!\!\!\!\-\!\!-\,#1}


\newcommand{\SigmaUsing}[2]{#1\@ #2} %{{{\mathcal U}}(#1,#2)}
%{\lozenge\!\!\!\!\!\circ  #1} % {\lozenge\!\!\!\!\-\!\!- #1} %{\upupsilon #1}  %{\nabla #1} %{\circ #1}%  {{{\mathcal P}}(#1)}
\newcommand{\Initial}[1] {{{\mathcal I}\!nitial}\langle#1\rangle}

\newcommand{\Pol}[1] {{\ensuremath{\prg{Pol}\_{\prg{#1}}}}}

\newcommand{\strongImplies}{\leqq} %{{ \,^\sqsubset\!\!\!_{\sim}\, }}
\newcommand{\weakImplies}{\lessapprox} %{{ \,^\sqsubset\!\!\!_{\sim}\, }}
\newcommand{\frames}{~\kw{frames}~}

\newcommand{\appref}[1]{see App.~\ref{#1}}

\newcommand{\sE}{{\prg{e}}}
\newcommand{\varMap}{{\ensuremath{\beta}}}

\newcommand{\Lang} {\ensuremath{{\mathcal L}{_1}}}
\newcommand{\LangOO} {\ensuremath{{\mathcal L}{_{\tt {oo}}}}\xspace}
\newcommand{\Chainmail} {\ensuremath{{\mathcal C}hainmail}\xspace}

% ------------------------------------------------------------------
%                                             positions, separations
\newcommand{\cf}{{\it c.f.~}}
\newcommand{\HYPHENA}{{\em-- }}
\newcommand{\HYPHENB}{{\em-- }}
\newcommand{\SP}{{\hspace{.1in}}}
\newcommand{\s}{{\hspace{.01in}}}

\newcommand{\obeys}{\,\textbf{\textrm{obeys}}\,}
\newcommand{\StrongDom}{\ensuremath{\mathcal{S}\textrm{\textit{trong}}{\mathcal{D}}\textrm{\textit{om}}}}
\newcommand{\Dom}{\ensuremath{\mathcal{D}}\textrm{\textit{om}}}


\newcommand{\Gives}{\ensuremath{\mathcal{G}\textrm{\textit{ives}}}}
%{\ensuremath{\mathcal{C}\textrm{\textit{an}}{\mathcal{A}}\textrm{\textit{ccess}}}(#1,#2)}
\newcommand{\A}{\ensuremath{A}}
\newcommand{\B}{\ensuremath{B}}
\newcommand{\Arising}[1]{{\mathcal{A}}\textrm{\textit{rising}}(#1)}

 %------------------------ syntax tables

\newcommand{\syntax}[1]{\prg{{\it #1}}}
\newcommand{\BBC}{$::=$} %in syntactic definitions
\newcommand{\SOR}{\ensuremath{\ \mid\ }} % BNF or
\newcommand{\MID}{{\SPsmall ~ \mid ~ \SPsmall }} % in sets


\newcommand{\pre}{\ensuremath{_{{pre}}}}   %kjx no \sc  in math mode
\newcommand{\post}{\ensuremath{_{{post}}}} %kjx no \sc  in math mode
\newcommand{\PRE}{\pre}
\newcommand{\POST}{\post}

 \newcommand{\interp}[2]{{\ensuremath{\lfloor{ {#1}}\rfloor_{#2}}}}
% \newcommand{\interpBL}[1]{{\lceil   {#1}  \rfloor}}

% ------------------------------------------------------------------
%                                              keywords, program text
\newcommand{\kw}[1]{\prg{#1}} % {{\bf{\sf {#1}}}}
\newcommand{\kwN}[1]{{\bf{\sf {#1}}}}
\newcommand{\returnKW}{{\bf{\sf {return}}}}
\newcommand{\newKW}{\mbox{\bf{\sf{new}}}}

\newcommand{\lit}[1]{{\prg {#1}\xspace}}
\newcommand{\com}{\ensuremath{\prg{//}}}


\newcommand{\ass}{\mbox{{\kw {:=}}\,}}
\newcommand{\semi}{\mbox{{\kw {;}}\ }}
\newcommand{\comma}{\mbox{{\kw {,}}\,}}
\newcommand{\lb}{\prg{\mbox{\tt{\bf{\{ }}}}}
\newcommand{\rb}{\prg{\mbox{\tt{\bf{\} }}}}}
\newcommand{\lp}{\prg{\mbox{\tt{\bf{( }}}}}
\newcommand{\rp}{\prg{\mbox{\tt{\bf{) }}}}}
%\newcommand{\thisL}{{\lit {this}}}% no~around it
\newcommand{\nullKW}{{\lit {null}}~}
\newcommand{\true}{{\lit {true}}~}
\newcommand{\false}{{\lit {false}}~}
\newcommand{\return}{{\kw {return}}\s}

 \newcommand{\M}{\prg{\ensuremath{\prg{M}}}}
  \newcommand{\Prog}[1]{\M{#1}}

\newcommand{\mkpair}{\fatsemi}
\newcommand{\subconf}{\ensuremath{\sqsubseteq}}
\newcommand{\restrct}[2]{\ensuremath{#1\!\!\downarrow\!_{#2}}}
\newcommand{\adapt}{\ensuremath{\!\triangleleft\!}}
\newcommand{\link}{\!\circ\!}

\newcommand{\ClassOf}[2] {\ensuremath{{\mathcal C}{\mathit{lass}}(#1)_{#2}}}

% --- assertions and expressions - simple

\newcommand{\SA}{\ensuremath{B}}%{\ensuremath{\prg{B}}}
\newcommand{\SAPrime}{\ensuremath{B'}}

\newcommand{\SE}{\ensuremath{\prg{e}}}
\newcommand{\SEPrime}{\ensuremath{\prg{e}'}}
\newcommand{\SEOne}{\ensuremath{\prg{e}_1}}
\newcommand{\SETwo}{\ensuremath{\prg{e}_2}}




%\newcommand{\Prog}[1]  {{\ensuremath{\prg{M}{{\prg{#1}}}}}}
    % {\prg{P}}

\newcommand{\expandexp}[1]{}

\newcommand{\oo}{object-oriented}
\newcommand{\mExtS}{\ensuremath{\Downarrow}}

% re-classification expression
\newcommand{\cm}[1]{\this{\prg{\ensuremath{\mExtS}}}\prg{#1}}

\newcommand{\refDef}[1]{Defintion \ref{{#1}}}

% structuring macros
\newcommand{\EndDefLemma}{\noindent $\bigtriangleup$}

 %-------------------- implies, and, or, iff, etc -----------------
  \newcommand{\AND}{{\SPsmall {\mbox{and}} \SPsmall}}
\newcommand{\WITH}{{\SPsmall {\mbox{with}} \SPsmall}}
 \newcommand{\IFF}{{\SP {\mbox{ if }} \SP}}
\newcommand{\OR}{{\SPsmall {\mbox{or}} \SPsmall}}
\newcommand{\implies}{{\ensuremath{\longrightarrow}}}
\newcommand{\upd}{{\mapsto}}

\newcommand{\SAF}{\ensuremath{W}} % {\prg{AFs}}






%Macros for inference rules
\newcommand{\inferencerule}[2]{
\begin{array}{l} #1 \\ \hline #2 \end{array}
}

\newcommand{\inferenceruleN}[3]
{
\begin{array}{l}
%\SP\SP\SP\SP\SP\SP\SP\SP
\SP\SP\SP\SP\SP  {\sf #1}
\\ #2  \\ \hline   #3
  \end{array}
}

\newcommand{\inferenceruleNM}[3]
{
\begin{array}{l}
\SP\SP\SP\SP\SP \SP\SP\SP\SP\SP\SP\SP\SP  {\sf #1}
\\ #2  \\ \hline   #3
  \end{array}
}

\newcommand{\inferenceruleNN}[3]
{
\begin{array}{l}
\SP\SP\SP\SP\SP\SP\SP\SP
\SP\SP\SP\SP\SP\SP\SP\SP
\SP\SP\SP\SP\SP\SP\SP\SP
\SP\SP\SP\SP\SP\SP\SP\SP
\SP\SP\SP\SP\SP\SP\SP\SP
\SP\SP
   {\sf #1}
\\ #2  \\ \hline   #3
  \end{array}
}

%===========================================================================
%  Definition-Lemma-Theorem-Proof
%
% Adaptation of LaTeX's theorem environment; can be used as a command
% (eg just \Lemma not \begin{Lemma}) and no italicisation; also works
% with ptmac; result numbering is uniform within subsections and can be
% suppressed.
%
\newif\ifNumberResults\NumberResultstrue
\def\@@opargbegintheorem#1#2#3{\@@@@begintheorem{\bf\@@thmname{#1}{#2}(#3)}}
\def\@@begintheorem#1#2{\@@@@begintheorem{\bf\@@thmname{#1}{#2}}}
\def\@@@@begintheorem#1{\par\removelastskip\smallskip\noindent{#1}}
\def\@@thmname#1#2{#1\ \ifNumberResults#2\ \fi}

% similarly \Proof or \begin{Proof}...\end{Proof}
% prefer proofs with statements if possible - hence \penalty700
%\let\qedsymbol\S% make it \square or \blacksquare if you like for kb
\let\qedsymbol \Box
\def\qed{\hfill{$\qedsymbol$}}
\def\Proof{\par\removelastskip\smallskip\penalty700\noindent{\bf Proof}\enskip}
\def\endProof{\qed\penalty-700 \smallskip}
\let\endproof\endProof

%   The actual words

\newtheorem{theo}{Theorem}
% \newtheorem{definition}[theo]{Definition}
%\newtheorem{example}[theo]{Example}
\newtheorem{mylemma}[theo]{Lemma}
%\newtheorem{conjecture}[theo]{Conjecture}
% \newtheorem{theorem}{Theorem}
% \newtheorem{note}[theo]{Note}
 \newtheorem{observation}[theo]{Observation}


%--------------------------------- the ones that Susan introduced
\newcommand{\SF}{{\prg S}}
\newcommand{\z}{{\prg z}}
\newcommand{\pu}{{\prg u}}
\newcommand{\pb}{{\prg b}}
\newcommand{\acc}{{\prg a}}
\newcommand{\bal}{{\prg{balance}}}
\newcommand{\zs}{{\prg{zs}}}

\newcommand{\Fields}[3]{\ensuremath{{\mathcal F}(}\Prog{#1},\prg{#2},
\prg{#3}\ensuremath{)} }
\newcommand{\FieldIds}[2]{\ensuremath{{\mathcal F}{\it {s}}(\Prog{#1},\prg{#2})}}
\newcommand{\Meths}[3]{\ensuremath{{\mathcal M}(}\Prog{#1},\prg{#2},
\prg{#3}\ensuremath{)} }





\newcommand{\WideFig}[3]
{
\begin{figure*}[t]
\begin{center}
\noindent
\fbox{
\begin{minipage}{4.7 in}
{#1} % the contents
\end{minipage}
}
\caption{#2}
\label{#3}
\end{center}
\end{figure*}
}


\newcommand{\WideFigWhere}[4] % you can specify where it should appear!
{
\begin{figure*}[{#4}]
\begin{center}
\noindent
\fbox{
\begin{minipage}{5. in}
{#1} % the contents
\end{minipage}
}
\caption{#2}
\label{#3}
\end{center}
\end{figure*}
}

\newcommand{\BigWideFigWhere}[4] % you can specify where it should appear!
{
\begin{figure*}[{#4}]
\begin{center}
\noindent
{\normalsize
\hrule
\begin{minipage}{5. in}
{#1} % the contents
\end{minipage}
\hrule
}
\caption{#2}
\label{#3}
\end{center}
\end{figure*}
}

\newcommand{\NotTooWideFigWhere}[4] % you can specify where it should appear!
{
\begin{figure*}[{#4}]
\begin{center}
\noindent
\fbox{
\begin{minipage}{4.3 in}
{#1} % the contents
\end{minipage}
}
\caption{#2}
\label{#3}
\end{center}
\end{figure*}
}


\newcommand{\opsemExprFig}
{\BigWideFigWhere {\opsemExpr} {Execution of expressions\MD}
{opsemTrad} {htbp} }



\newcommand{\mlc}{ }%{\heartsuit}
%\newcommand{\mcl}{ }%{\heartsuit}
\newcommand{\mc}{ }%{\heartsuit}

\newcommand{\BigNotTooWideFigWhere}[4] % you can specify where it should appear!
{
\begin{figure*}[{#4}]
\begin{center}
\noindent
{\normalsize
\hrule
\begin{minipage}{4.3 in}
{#1} % the contents
\end{minipage}
\hrule
}
\caption{#2}
\label{#3}
\end{center}
\end{figure*}
}

 \newcommand{\hlc}[2][yellow]{{%
    \colorlet{foo}{#1}%
    \sethlcolor{foo}\hl{#2}}%\prg{m}''
}

%]})
%}
