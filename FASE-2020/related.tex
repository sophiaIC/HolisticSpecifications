\paragraph{Behavioural Specification Languages} 

Hatcliff et al.\ \cite{behavSurvey2012} provide an excellent survey of
contemporary specification approaches.  With a lineage back to Hoare
logic \cite{Hoare69}, Meyer's Design by Contract \cite{Meyer97} was the
first popular attempt to bring verification techniques to
object-oriented programs as a ``whole cloth'' language design in
Eiffel.  Several more recent specification languages are now making
their way into practical and educational use, including JML
\cite{Leavens-etal07}, Spec$\sharp$ \cite{BarLeiSch05}, Dafny
\cite{dafny} and Whiley \cite{whiley15}. Our approach builds upon
these fundamentals, particularly Leino \& Shulte's
%\kjx{and Naumann's} 
formulation of
two-state invariants \cite{usingHistory}, and Summers and
Drossopoulou's Considerate Reasoning \cite{Considerate}.
%
In general, these approaches assume a closed system, where modules
can be trusted to co{\"o}perate. In this paper we aim to
% illustrate the kinds of techniques required
work
in an open system where modules'
invariants must be protected irrespective of the behaviour of the rest
of the system.

%% \sd{\Chainmail assertions are} guarantees upheld throughout program execution. 
%% Other systems which give such ``permanent'' guarantees are  type systems, 
%% which ensure that well-formed programs  always produce well-formed runtime
%% configurations, or information flow control systems \cite{infoflow}, which ensure that values 
%% classified as high  will not be passed into contexts classified as low. 
%% Such  guarantees %made by types or information flow control
%%  are  practical to check, but   too coarse grained
%% for the purpose of fine-grained,  module-specific specifications. 


%% \Chainmail\ specifications can cross-cut the code they are
%% specifying; \sd{therefore,} they are related to
%% aspect-oriented specification
%% languages such as AspectJML \cite{AspectJML} and AspectLTL
%% \cite{AspectLTL}.
%% %
%% AspectJML is an aspect-oriented extension to JML;
%%  in much the same way that AspectJ is an aspect-oriented extension to
%% Java \cite{AspectJ}.  AspectJML offers AspectJ-style pointcuts 
%% that allow the definition of crosscutting specifications, such as 
%% shared pre- or post-conditions for a range of method calls. 
%% % SD removed the below, as I do not understand it.
%% % These crosscutting specifications can be checked dynamically along with
%% % traditional object-oriented JML assertions. In contrast, \Chainmail\
%% %specifications naturally cross-cut implementation and specification
%% %modules without any special notation, although, lacking wildcards,
%% %\Chainmail\ is not as flexible as AspectJML. 
%% % % SD removed the below, because I do not think it is important
%% %To our knowledge, the
%% %semantics of AspectJML have yet to be defined formally, although
%% %earlier work by Molderez and Janssens describes the formal core of a
%% %similar language \cite{DbCAspectJ}.

%% AspectLTL \cite{AspectLTL} is a specification language based on Linear
%% Temporal Logic (LTL). \sd{It} %AspectLTL 
%% adds cross-cutting aspects to more
%% traditional LTL module specifications: these aspects can further
%% constrain specifications in modules. In that sense, AspectLTL and
%% \Chainmail\ %both 
%% \sd{use} similar implicit join point models, rather than
%% importing AspectJ style explicit pointcuts as in AspectJML.
%% %% % SD removed the below, because I do not think it is important
%% %  AspectLTL
%% %has a formal definition, as does \Chainmail; unlike \Chainmail,
%% %AspectLTL has support for automated reasoning with an efficient
%% %synthesis algorithm.

%% % \paragraph{Concurrent Reasoning} Deny-Guarantee \cite{DenyGuarantee}
%% % distinguishes between assertions guaranteed by a thread, and actions
%% % denied to all other threads. Deny properties correspond to our
%% % requirements that certain properties be preserved by all code linked
%% % to the current module. Compared with our work, deny-guarantee assumes
%% % co{\"o}peration: composition is legal only if  threads adhere  to
%% % their deny properties. In our work, a module has to be robust  and
%% % ensure that these properties cannot be affected by  other code. 


%% %Finally, 
%% \sd{Our} work is also related to the causal obligations in Helm et
%% al.'s behavioural contracts \cite{helm90}. Causal obligations allow
%% programmers to specify e.g.\ that whenever one object receives a
%% message (such as a subject in the Observer pattern having its value
%% changed) that object must send particular messages off to other objects
%% (e.g.\ the subject must notify its observers). \Chainmail's control
%% %SD: not "control flow"
%%  operator % (`$\Calls{\_} {\_} {\_} {\_} $) 
%%  %allows  programmers to make
%%  \sd{supports}  similar specifications, (e.g. 
%%  ${\Calls{\_}  {\prg{setValue}} {\prg{s}} {\prg{v}}}  \rightarrow \Future{\Calls{\prg{s}}{\prg{notify}}{\prg{s.observer}}{\prg{v}}}$ --- when a subject receives a \prg{setValue} method,
%%   it must ``forward'' those messages to the observer.

\paragraph{Defensive Consistency}

%cute but wrong.
%To misparaphrase Tolstoy, secure systems are all alike;
%every insecure system is insecure in its own way
%\cite{WikipediaAnnaKareninaPrinciple}.

In an open world, we cannot rely on the kindness of strangers: rather
we have to ensure our code is correct regardless of whether it
interacts with friends or foes.
Attackers 
\textit{``only have to be lucky once''} while secure systems 
\textit{``have to be lucky always''} \cite{IRAThatcher}.
% SD 
Miller \cite{miller-esop2013,MillerPhD} defines the necessary approach
as \textbf{defensive consistency}: \textit{``An object is defensively
  consistent when it can defend its own invariants and provide correct
  service to its well behaved clients, despite arbitrary or malicious
  misbehaviour by its other clients.''}  Defensively consistent
modules are particularly hard to design, to write, to understand, and
to verify: but
% they have the great advantage that
they make it much
easier to make guarantees about systems composed of multiple components
\cite{Murray10dphil}.


\paragraph{Object Capabilities and Sandboxes.}
{{\em Capabilities} as a means to support the development of concurrent and distributed system  were developed in the 60's
by Dennis and Van Horn \cite{Dennis66}, and were adapted to the
programming languages setting in the 70's \cite{JamesMorris}. 
{\em Object capabilities} were first introduced~\cite{MillerPhD} in the early 2000s},
 and many recent % work attempts to manage
studies manage
to verify  safety or correctness of object capability programs.
Google's Caja \cite{Caja} applies   sandboxes, proxies, and wrappers
 to limit components'
access to \textit{ambient} authority.
% --- that is, capabilities that
%can be obtained from the wider environment, rather than being granted
%to a component explicitly.
Sandboxing has been validated
formally: Maffeis et al.\ \cite{mmt-oakland10} develop a model of
JavaScript, demonstrate that it obeys two principles of
object capability systems
%  (``connectivity begets connectivity'' and
%``no authority amplification''), and then % uses these principles to
and show  how untrusted applications can be prevented from interfering with
the rest of the system. 
Recent programming languages % and web systems
\cite{CapJavaHayesAPLAS17,CapNetSocc17Eide,DOCaT14} including Newspeak
\cite{newspeak17}, Dart \cite{dart15}, Grace \cite{grace,graceClasses}
and Wyvern \cite{wyverncapabilities} have adopted the object
capability model.

%% \paragraph{Verification of Dynamic Languages}
%% A few formal verification frameworks  address JavaScript's highly
%% dynamic, prototype-based semantics. Gardner et al.\ \cite{Gardner12}
%%  developed a formalisation of JavaScript based on separation logic
%% % that they have used
%% and verified   examples. Xiong and Qin et
%% al.\ \cite{XiongPhd,Qin11}  worked on similar lines.
%% % More substantially,
%% Swamy et al.\ \cite{JSDijkstraMonad}  recently
%% developed a mechanised verification technique for JavaScript based on
%% the Dijkstra Monad in the F* programming language.  Finally, Jang et
%% al.\ \cite{Quark} % have %  managed to provide
%% developed a machine-checked proof of
%% five important properties of a web browser --- again similar to our
%% % \prg{any\_code} 
%% invariants --- such as
%% % \textit{``no tab may interfere with
%% %  another tab''} and 
%% \textit{``cookies may not be shared across
%%   domains''} by writing the minimal kernel of the browser in Haskell.
  
%%   \paragraph{JavaScript analyses.}
%% More practically, 
%% Karim et al. apply static analysis on
%% Mozilla's JavaScript Jetpack extension framework \cite{adsafe}, including
%%  pointer analyses. % In a different direction,
%% Bhargavan et al.\ \cite{DefJS}
%% extend language-based sandboxing techniques to support defensive
%% components that can execute successfully  in otherwise untrusted
%% environments.   Politz et
%% al.\ \cite{ADsafety} use a JavaScript type checker to check
%% properties such as
%% % \textit{``widgets cannot obtain direct references
%%  % to DOM nodes''} and
%%  \textit{``multiple widgets on the same page
%%   cannot communicate.''}
%% % --- somewhat similar in spirit to our \textbf{Pol\_4}.
%% Lerner et al.\ extend this system to ensure browser
%% extensions observe \textit{``private mode''} browsing conventions,
%% such as that \textit{``no private browsing history retained''}
%% \cite{Lerner2013b}.  Dimoulas et al.\ \cite{DPCC14} generalise the
%% language and type checker based approach to enforce explicit policies,
%% % although the policies  are restricted to
%% that  describe  which components  may
%% access, or may influence the use of, particular capabilities.
%% Alternatively, Taly et al.\ \cite{secureJS}
%% model  JavaScript APIs in Datalog, and then
%% carry out a Datalog search for an ``attacker'' from the set of all
%% valid API calls. 



\paragraph{Verification of Object Capability Programs}
Murray made the first attempt to formalise defensive consistency and
correctness~\cite{Murray10dphil}.  Murray's model was rooted in
counterfactual causation~\cite{Lewis_73}: an object is defensively
consistent when the addition of untrustworthy clients cannot cause
well-behaved clients to be given incorrect service.  Murray formalised
defensive consistency very abstractly, over models of (concurrent)
object-capability systems in the process algebra CSP~\cite{Hoare:CSP},
without a specification language for describing effects, such as what
it means for an object to provide incorrect service.  Both Miller and
Murray's definitions are intensional, describing what it means for an
object to be defensively consistent.


Dro\-sso\-pou\-lou and Noble \cite{capeFTfJP,capeFTfJP14} have
analysed Miller's Mint and Purse example \cite{MillerPhD} 
% SD Chope details by
% expressing it in Joe-E 
% a Java subset without reflection and static
%fields, 
%and in Grace \cite{capeFTfJP14}, 
and discussed the six
capability policies 
% that characterise the correct behaviour of the
% program, 
as proposed in \cite{MillerPhD}.
%We argued that these policies require a novel
%approach to specification, and showed some first ideas on how to use
%temporal logic.
In %  an unpublished technical report
\cite{WAS-OOPSLA14-TR}, {they} % Drossopoulou and Noble
sketched a  specification language,  \sd{used}  it to  
specify the six policies from \cite{MillerPhD}, % however,
%{their} partial formalisation showed that % they allowed
\sd{showed} that several possible interpretations were possible, %.  They also 
\sd{and} uncovered
the need for another four further policies.
%  and formalised them as well, showing how different implementations of the underlying Mint and Purse
% systems coexist with different policies \cite{capeIFM14},
They also
  sketched how 
a trust-sensitive 
example (the escrow exchange) could be verified in an open world
\cite{swapsies}. 
% In contrast, our work focuses on the semantics of the  \Chainmail\ specification
% language and how it can be used to provide holistic specifications for
% robust programs.
\sd{Their work does not support the concepts of control, time, or space, as in \Chainmail,
but it offers a primitive expressing trust.}
 
Devriese et al.\ \cite{dd}  have deployed
   \sd{powerful} %rather more complex
  theoretical techniques to address similar problems:  % Devrise et al.\ 
  \sd{They} show how step-indexing, Kripke worlds, and representing objects
as state machines with public and private transitions can be used to
reason about % object-oriented programs in general.
\sd{object capabilities}.
Devriese have demonstrated solutions to a range of exemplar problems,
including the DOM wrapper (replicated in our
section~\ref{sect:example:DOM}) and a mashup application.
% Although the formal techniques are much more sophisticated than we
%apply here, and consequently 
% not true can e.g.\ reason about recursion where we
%cannot, there are some similarities, e.g.\ with the 
\sd{Their} distinction
between public and private transitions %being related 
\sd{is similar} to the
distinction between internal and external objects.

More recently, Swasey et al.\ \cite{ddd}  designed OCPL, a logic
for object capability patterns, that supports specifications and
proofs for object-oriented systems in an open world.  
% The key idea here is to 
\sd{They} % say it simpler
draw on verification techniques for security and
information flow: separating internal implementations (``high values''
which must not be exposed to attacking code) from interface objects
(``low values'' which may be exposed).  OCPL supports defensive
consistency % (Swasey et al.\ use 
(\sd{they} use the term ``robust safety'' from the
security community \cite{Bengtson}) via a proof system that ensures
low values can never leak high values to external attackers. 
%\susan{How does this imply that high values can be exposed?}
%\james{typo fixed: it's LOW values that can be exposed}
This means that low values \textit{can} be exposed to external code,
and the behaviour of the system is described by considering attacks only
on low values.  %OCPL is a program logic, and Swasey
\sd{They} use that logic to
prove a number of object-capability patterns, including
sealer/unsealer pairs, the caretaker, and a general membrane.

Schaefer et al.\ \cite{schaeferCbC} have recently
% taken a similar approach to Swasey,
% adding support for
\sd{added}  support for information-flow security % in a setting 
\sd{using} refinement to ensure correctness (in this case confidentiality) by
construction. 
% Although designed to support
% confidentialty, it seems likely that the information-flow guarantees
% could also be used to ensure robustness.  
By enforcing encapsulation, \sd{all} % used to say both
these approaches share similarity with techniques such as
ownership types \cite{ownalias,NobPotVitECOOP98}, which also
protect internal implementation objects from accesses that cross
encapsulation boundaries.  Banerjee and Naumann demonstrated that by
ensuring confinement, ownership
systems can enforce representation independence (a property close to
``robust safety'') some time ago \cite{Banerjee:2005}.

 
\Chainmail\ differs from Swasey, Schaefer's, and Devriese's work in a number of ways:
% \citet{ddd} and \citet{schaeferCbC} 
\sd{They} are primarily concerned \sd{with} %about
mechanisms that ensure encapsulation (aka 
confinement) while we abstract away from any mechanism via the
$\External{}$ predicate. 
\sd{They use powerful mathematical techniques
% , such as Kripke worlds and step-indexing 
which  the users need  to understand in order to write their specifications,
while \Chainmail users only need  to understand  first order logic and 
the holistic operators presented in this paper.}
% While \Chainmail's $\Using{}{}$ is related to Banerjee
% and Naumann's region sets, the assertion languages here are mostly
% traditional (extensions of) Hoare logics --- Swasey et al.\ build on a
%concurrent separation logic. 
\sd{ Finally, none of these systems offer the kinds of
holistic assertions addressing control flow, change, or temporal
operations that are at the core of \Chainmail's approach.
}

Scilla \cite{scillaOOPSLA19} is a minimalistic typed functional
language for writing smart contracts that compiles to the Ethereum
bytecode. Scilla's semantic model is restricted, assuming actor based
communication and restricting recursion,  thus facilitating static
analysis of Scilla contracts and ensuring termination.
Scilla is able to demonstrate that a number of popular Ethereum
contracts avoid type errors, out-of-gas resource failures, and
preservation of virtual currency. 
Scilla's semantics are defined formally, but have not yet been represented in a
mechanised model.

%% \kjx{NPChecker \cite{NPcheckerOOPSLA19} analyses Ethereum smart
%% contracts to detect bugs related to nondeterministic
%% execution. NPChecker undertakes an information flow
%% analysis to detect potential read-write hazards
%% particularly reentrancy and ordering dependencies.
%% \textbf{We don't do concurrency. Do we need this one? I don't think so}
%% }


Finally, the recent VerX tool is able to verify a range of
specifications for solidity contracts automatically \cite{VerX}.
Similar to \Chainmail, VerX has a specification language based on
temporal logic.  VerX offers three temporal operators (always, once,
prev) but only within a past modality, while \Chainmail\ has two
temporal operators, both existential, but with both past and future
modalities.   VerX specifications can also include predicates that
model the current invocation on a contract (similar to \Chainmail's
``calls''), can access variables, and compute sums (only) over
collections. \Chainmail\ is strictly more expressive as a
specification language, including quantification over objects and sets
(so can compute arbitrary reductions on collections) and of course
specifications for permission (``access''), space (``in'') and
viewpoint (``external'') which have no analogues in VerX. 
Unlike \Chainmail, VerX includes a practical tool that has
been used to verify   a hundred properties across case studies of
twelve Solidity contracts.
%\textbf{(ideally also say something about proof status)}}



% SD chopped as did not like
%As with Swasey et al.\ this work does not provide a holistic
%assertion language like \Chainmail.
% SD Chopped, as it sounds as if their is not real code, which is debatable
% and what is an extensional framework? they would say that theirs is too.
%In contrast, \Chainmail\ is
%meant for describing and reasoning about real code, and we provide an
%expressive, extensional framework for evaluating defensive consistency
%in actual open systems.
%


%%%%%%%%%%%%%%%%%%%%%%%%%%%%%%%%%%%%%%%%%%%%%%%%%%%%%%%%%%%%
%%NOTES:
%% the other thing this section needs to do, particularly with Devrise, is to lay out precisely the way our work is more limited than theirs:
%% (Swasey, I'm more and more convinced, is just ownership-via-a-proof-system) 
%% we don't step-index, don't have logical relations, etc: what do we lose by NOT having those things
%% (or what do we gain by having those things...

%% The "deep" comparison with Swasey and with Devirese (and also
%% information flow control and temporal logics) needs to say why these
%% works are not as good (expressive? easy to understand?) as ours.
%% Currently the Related work just mentions them, but does not answer the
%% question as to why our work is important when theirs already has been
%% published.




%% *Difference between Spec Languages and Chainmail*  One way to tackle
%%  this would be to enumerate which elements of Chainmail appear at
%%  other works, which do not, and claim that Chainmail’s novelty is the
%%  good combination of these elements


%% Eg: reflection about contents of stack and heap (in classical Hoare
%% Logics), two state assertions (JML etc), invariants (Hoare and Meyer),
%% internal/external (Liskov?, Noble et al,modules in Neumann and also
%% O’Hearn), time (temporal logic, but they do not have the other stuff),
%% Control (none?), Space (in Sep. logic, and in effects, buyt the
%% meaning is different), Permissions (our earlier work, and less
%% flexible approaches such as owenrship types and perhaps also
%% oinformation flow control), Authority (effect systems and modifies
%% clauses, and perhaps also Bierman&Parkison abstract predicates, but
%% there it is tied to pre-post conditions.


%% Also, point out difference between our invariants and Hoare
%% triples. Subtle and needs thinking







%%%%%%%%%%%%%%%%%%%%%%%%%%%%%%%%%%%%%%%%%%%%%%%%%%%%%%%%%%%%

%% Neither effort addresses specification languages for security and
%% robustness, provides Hoare logics to reason about object-capability
%% programs.

%% , model protocols that dynamically ascribe trust
%% \cite{swapsies,lefthand} or quantify the damage an untrustworthy
%% object can do.






% \kjx{History-Based Specification and Verification of Scalable
%  Concurrent and Distributed Systems --- ICFEM15}


% \paragraph{Specifying Design Patterns}

% Techniques for specifying Design Patterns go back at least to 
% Helm's contracts \cite{Helm92}.

% more importantly: work on formalisation of design patterns.
% (again look at JC grant, even if refs are 5 years old)
% let's be shameless here...



% This search is similar to the quantification over
% potential code snippets in our model.
% The problem posed by the Escrow example is that it establishes a two-way
% dependency between trusted and untrusted systems --- precisely the
% kind of dependencies these techniques prevent.

% %These approaches are all based on static analyses.
%  The WebSand
% \cite{flowcaps11,sabelfeld-inlining2012} and Jeeves \cite{jeeves2012}
% projects use dynamic techniques to monitor safe execution of information flow policies.
%  Richards et al.\ \cite{FlacJS}   extended this approach by
% incorporating explicit dynamic ownership of objects (and thus of
% capabilities) and policies that may examine the history of objects'
% computations. While these dynamic techniques can restrict or terminate
% the execution of a component that breaches its security policies, they
% cannot guarantee in advance that such violations can never happen.
% While information flow policies are concerned with the flow of objects (and thus also capabilities)
% across the program code, our work is more concerned with the identification of the objects which protect
% the services.

%Compared with all these approaches, our work   focuses on
%\textit{general} techniques for specifying (and ultimately verifying)
%capability policies, whereas these systems are generally much more
%\textit{specific}: focusing on one (or a small number) of actual
%policies. % This seems to be because contemporary object capability
%programming is primarily carried out in JavaScript, but
% There are few

 
% \paragraph{Relational models of trust.}
% Artz and Gil \cite{artz-trust-survey-2007} survey various
% types of trust in computer science generally, although trust has also
% been studied in specific settings, 
% %
% ranging from peer-to-peer systems \cite{aberer-trust-p2p-2001} and
% cloud computing \cite{habib-trust-cloud-2011} 
% to mobile ad-hoc networks \cite{cho-trust-survey-adhocnets-2011}, 
% the internet of things \cite{lize-trust-IoT-2014}, 
% online dating \cite{norcie-trust-online-dating},
% and as a component of a wider socio-technical system
% \cite{cho-trust-sustainable-2013,walter-trust-cloud-ecis2013}. 
% %
% Considering trust (and risk) in systems design, Cahill et al.'s overview
% of the \textsc{Secure} project \cite{cahill-trust-pervasive-2003}
% gives a good introduction to both theoretical and practical issues of
% risk and trust, including a qualitative analysis of an e-purse
% example. This project builds on Carbone's trust model
% \cite{carbone-formal-trust-2003} which offers a core semantic model of
% trust based on intervals to capture both trust and uncertainty in that
% trust. Earlier Abdul-Rahman proposed using separate relations for
% trust and recommendation in distributed systems
% \cite{abdul-rahman-distributed-trust-1998}, more recently Huang and
% Nicol preset a first-order formalisation that makes the same
% distinction \cite{huang-formal-semantics-trust-calculus-2010}.
% Solhaug and St{\o}len \cite{solhaug-trust-uncertainty-2011} 
% consider how risk and trust are related to uncertainties over
% actual outcomes versus knowledge of outcomes.
% Compared with our work, these approaches produce models of trust
% relationships between high-level system components 
% (typically treating risk as uncertainty in trust) 
% but do not link those relations to the system's code. 



% \paragraph{Logical models of trust.}
% \sd{A detailed study of how web-users decide whether to trust appears in \cite{GilArtz}.}
% \sd{Starting with \cite{Lampson92},} various different logics have been used to measure trust in different
% kinds of systems.
% Murray and Lowe \cite{murray10-infoflow} model object capability
% programs in CSP, and use a model checker to ensure program executions
% do not leak authority.
% Carbone et al.\ \cite{carbone-formal-trust-2011}
% use linear temporal logic to model specific trust relationships in service
% oriented architectures.
% Ries et al.\ \cite{habib-trust-CertainLogic-2011} evaluate trust under
% uncertainty by evaluating Boolean expressions in terms of real values
% for average rating, certainty, and initial expectation.
% % Perhaps closer to our work, Aldini
% Aldini \cite{aldini-calculus-trust-IFIPTM2014} describes a temporal logic for
% trust that supports model checking to verify some trust properties.
% Primiero and Taddeo \cite{primiero-modal-theory-trust-2011} have
% developed a modal type theory that treats trust as a second-order
% relation over base relations between
% counterparties. Merro and Sibilio
% \cite{merro-calculus-trust-adhoc-facs2011} developed a trust model for
% a process calculus based on labelled transition systems.
% Compared with our proposal, these approaches use
% process calculi or other abstract logical models of systems, rather
% than engaging directly with the system's code.






%%%% %%%% %%%% %%%% %%%% %%%% %%%% %%%% %%%% %%%% %%%% %%%% %%%% %%%% 
%%%% %%%% %%%% %%%% %%%% %%%% %%%% %%%% %%%% %%%% %%%% %%%% %%%% %%%% 





