%Susan:Please read first bit as I have just written it
%\se{When you write a module that is to be used with other code, the last thing you want to happen is that some other code uses it to cause effects that you never intended. Our specification language \Chainmail has been designed, so that developers whose modules are going to be used in the wild, have the language to constrain the usage of their code. In addition to classical function by function specification techniques, we have shown that a holistic or whole program approach is needed to make open world code robust. We have shown} 
% going to the old one, as running out of space.
% also, the new one brings new words, and I think all th words in concluson should have appeared earlier
In this paper we have motivated the need for holistic specifications,
presented the \Chainmail specification language for writing such
specifications, and shown 
how \Chainmail can be used to give holistic
specifications of key exemplar problems: the bank account, the
wrapped DOM, and the DAO.

To focus on the key attributes of a holistic specification language,
% we have tried to keep the
\sd{we have kept  \Chainmail simple, only requiring an understanding of first order logic.}
\sd{We believe that the holistic features (permission, control, time, space and viewpoint),
are intuitive concepts %for ptogrammers. 
when reasoning informally, and were pleased to have been able to provide their
formal semantics in what  we  argue is a simple manner.}
% below not true, we do have recusrions  
%do not even support recursive procedures to avoid circularities in the
%metatheory, let alone concurrency, exceptions, distribution,
%networking, etc. 

\sd{The development of the semantics of \Chainmail assertions posed several interesting 
challenges, \eg the treatment of the open world requires two-module execution
and the concept of external objects,
recursion is confined to ghostfields and assertions require termination of included expressions,
space required the concept of restricting runtime configurations,
and time required adaptation operators which apply bindings from one configuration to another.}  

\sd{\Chainmail is powerful enough to express many key examples from the
literature; nevertheless, it lacks several important features: It provides 
recursion  only in a restricted form, it has  has a rather inflexible notion of
module and does not support hierarchies of modules, and knows nothing about
concurrency or distribution.  We plan to remove these restrictions by applying
techniques such as step-indexing \cite{stepindex}, but hope to keep that in the
model of \Chainmail without exposing it to the programmer.  We are also
 interested in extending \Chainmail\ to situations
where internal modules are typed, but the external modules are untyped.
%
We also plan to extend \Chainmail to support reasoning about
conditional trust in programs, and to quantify the risks involved in
interacting with untrustworthy software \cite{swapsies}.
}

\sd{In order to make these kinds of specifications
practically useful,  we plan to develop logics for proving adherence to holistic specs, as well
as logics for using holistic specs in proofs. We want to develop 
dynamic monitoring  and model checking techniques for our specifications. 
And finally, we plan to automate reasoning with these logics.}
