%\documentclass{article}[14pt]
 \documentclass[11pt]{article}
%\usepackage{multicol, blindtext}

%\documentclass[twocolumn]{article}[14pt]
\usepackage{multicol}
\usepackage{lipsum}
 
\usepackage{url}

\usepackage{a4wide}
 \usepackage{pxfonts}  % for the black diamond!
 
\usepackage{xcolor}
\usepackage{soul}

\newcommand{\hlc}[2][yellow]{{%
    \colorlet{foo}{#1}%
    \sethlcolor{foo}\hl{#2}}%\prg{m}''
}

 
\usepackage{rotating}
\usepackage{enumerate}

\usepackage{natbib}
\setlength{\bibsep}{0.0pt}

%\pdfpagewidth=210truemm
%\pdfpageheight=297truemm{0.8cm}

 \addtolength{\topmargin}{-2.4cm}
 \addtolength{\textheight}{3.1cm}
 \headheight     0.0cm
 \errorcontextlines=-1
 \advance\textwidth by 2.1cm
  \advance\oddsidemargin by -1.2cm
 \advance\evensidemargin by -0.6cm
\usepackage[T1]{fontenc}
\usepackage[scaled=1]{helvet}
\renewcommand*\familydefault{\sfdefault}
\usepackage{microtype}


\newcommand{\forget}[1]{} % {{#1}}
\pagestyle{empty}

\newcommand{\etc}{{\it etc.}}
\newcommand{\eg}{{\it e.g.\,}}
\newcommand{\ie}{{\it i.e.\,}}


%\usepackage[usenames]{color}

\usepackage{times}
 \usepackage{latexsym}
\usepackage{listings}
\definecolor{dkgreen}{rgb}{0,0.6,0}
\definecolor{gray}{rgb}{0.5,0.5,0.5}
\definecolor{mauve}{rgb}{0.58,0,0.82}

\lstset{ %
  language=Java,                % the language of the code
  basicstyle=\footnotesize\tt,           % the size of the fonts that are used for the code
  numbers=left,                   % where to put the line-numbers
  numberstyle=\tiny\color{dkgreen},  % the style that is used for the line-numbers
  stepnumber=1,                   % the step between two line-numbers. If it's 1, each line
                                  % will be numbered
  numbersep=5pt,                  % how far the line-numbers are from the code
  backgroundcolor=\color{white},      % choose the background color. You must add \usepackage{color}
  showspaces=false,               % show spaces adding particular underscores
  showstringspaces=false,         % underline spaces within strings
  showtabs=false,                 % show tabs within strings adding particular underscores
  frame=single,                   % adds a frame around the code
  rulecolor=\color{black},        % if not set, the frame-color may be changed on line-breaks within not-black text (e.g. commens (green here))
  tabsize=2,                      % sets default tabsize to 2 spaces
  captionpos=b,                   % sets the caption-position to bottom
  breaklines=true,                % sets automatic line breaking
  breakatwhitespace=false,        % sets if automatic breaks should only happen at whitespace
  title=\lstname,                   % show the filename of files included with \lstinputlisting;
                                  % also try caption instead of title
  keywordstyle=\color{blue},          % keyword style
  commentstyle=\color{gray},       % comment style
  stringstyle=\color{mauve},         % string literal style
  escapeinside={\%*}{*)},            % if you want to add LaTeX within your code
  morekeywords={private,public,final,this,throw,new,||,to,def,any}               % if you want to add more keywords to the set
}

\newcommand{\sparagraph}[1]{{\noindent {\bf {#1}}} \noindent} % {{ \vspace{.05in} \noindent {\textit{\textbf{#1:}}}}}
\newcommand{\dparagraph}[1]{{ \vspace{.04in} \noindent {\bf {#1}}} \noindent}
\newcommand{\sd}[1]{#1}
\newcommand{\jn}[1]{#1} % {{\color{green}{#1}}}
\newcommand{\prg}[1]{{\mbox{{{\tt #1}}}}}%{\textttt {#1}}}
\newcommand{\mprg}[1]{\tt #1}
  \newcommand{\RoSpec}{${\cal C}hainmail$}

\newcommand{\FigAuthorize}
{\begin{figure*}[!b]
  \fbox{   
 \begin{minipage}{\textwidth} 
 $Authorized(\prg{e},\prg{p},\prg{p}',\prg{m})$\ \ $\equiv$   $ \left\{
                            \begin{array}{l}
                              \exists \prg{m}'. Prev(\ Call(\prg{p},\prg{e}.\prg{authorize}(\prg{p}',\prg{m}')\ )\ \ \wedge\ \ \prg{m}'\geq \prg{m}    \\
                              \mbox{ \hlc[yellow]{In the previous step, {\tt p} authorized {\tt p'} for some amount greater or equal  {\tt m}}}\\
                             \vee \\
                              Prev(\ \ \ Authorized(\prg{e},\prg{p},\prg{p}',\prg{m})  \ \    \wedge   \ \ 
                                \neg  \exists \prg{p}''.Call(\prg{p}',\prg{e.transferFrom(p,p'',\_)},\_ ) \ \  \  )\\
                               \mbox{ \hlc[yellow]{In the previous step, {\tt p'} was authorized for {\tt m} 
                                and did not transfer   on behalf of  {\tt p}  }} 
			\\
                               \vee \\
                              \exists \prg{m}', \prg{m}'', \prg{p}''. [ \ Prev(\ \ \  Authorized(\prg{e},\prg{p},\prg{p}',\prg{m}')  \ \wedge\ 
                          %    \strut \hspace{2.95cm} 
                              Call(\prg{p}',\prg{e}.\prg{transferFrom}(\prg{p},\prg{p}'',\prg{m}''),\_ ) \ \ \ )  \ \ \\ 
                               \strut \hspace{1.9cm} \wedge \ \ \prg{m}''\leq \prg{m}'-\prg{m} \ \ \ \  ]  \\
                               \mbox{ \hlc[yellow]{In the previous step, {\tt p'} was authorized for {\tt m'}, and  transfered  on behalf of {\tt p}}} \\
                                 %  \strut \hspace{1.0cm} 
                              \mbox{ \hlc[yellow]{some amount smaller than   {\tt m'}-{\tt m}}}\\
                        \end{array} 
                         \right.$
\strut ~ \\
% \begin{minipage}{\textwidth}                   
%\prg{Pol\_ERC20\_withdraw} \ $\equiv$ \\ 
$\forall \prg{e}:\prg{ERC20}.\forall \prg{p}:\prg{Any}.\forall \prg{m}:\prg{Nat}.$\\
%\strut \hspace{1.4cm} 
\strut \hspace{0.2cm}
$[\ \ \  \ \ Next(\prg{e}.\prg{balance(p)})=\prg{e}.\prg{balance(p)}-\prg{m}$ \\  
  \strut \hspace{0.8cm}  \mbox{ \hlc[yellow]{If, \  in the next step {\tt p}'s balance decreases by {\tt m},}}\\
\strut \hspace{0.8cm} \ \ \ $\longrightarrow$\\
\strut \hspace{0.8cm} $\exists \prg{p}',\prg{p}'':\prg{Any}.$\\
%\strut \hspace{1.8cm} 
\strut \hspace{0.8cm}
$[\  \ Call(\prg{p},\prg{e.transfer}(\prg{p}',\prg{m}),\_)) \,  \vee\,  
  Call(\prg{p}'',\prg{e.transferFrom}(\prg{p},\prg{p}',\prg{m}),\_)\   \wedge\    Authorized(\prg{e},\prg{p},\prg{p}'',\prg{m}) \ ] \ \ \  \ \ ]$\\
 \strut \hspace{0.8cm}  \mbox{ \hlc[yellow]{then, \ in the current step  either  {\tt p} itself, or a {\tt p''}  authorized for {\tt m}, instigates a transfer of {\tt m}}}
\end{minipage}
% \end{minipage}
}
\label{DefAuthorized}
\caption{Authorization and ERC20 guarantees on transfer  in  \RoSpec  --  \hlc[yellow]{informal explanations in yellow}}                     
 \end{figure*}                         
%  \vspace{.09cm}  
  }

\newcommand{\FigEscrow}
{\begin{figure*}[!b]
  \fbox{
 \begin{minipage}{\textwidth} 
 $\prg{e}\ {\bf obeys}\ \prg{ValidEscrow}$ \ \ $\equiv$ 
\\
$\begin{array}{l}
\prg{true}\\
 \strut \hspace{0.5cm} \   \prg{ \{ \ res := e.exchange(c1,c2,k,k')   \    \} } \\         
\ensuremath{ [ } \ \  \prg{res}=\prg{true} \wedge\, \prg{c1}\ {\bf obeys}\ \prg{ValidClient} \,   \wedge\, \prg{c2}\ {\bf obeys}\ \prg{ValidClient} \ \   
\\
 \strut \hspace{0.2cm} 
 \rightarrow  \mbox{ \hlc[yellow]{ {\tt c1}, {\tt c2} had sufficient funds \& transfers between them took place, other {\tt ValidClient} not affected}}\ \ ] \  \ \ \wedge \\
\ensuremath{ [ } \ \  \prg{res}=\prg{true}  \  \wedge\  \neg( \prg{c1}\ {\bf obeys}\ \prg{ValidClient}) \   \wedge\ \neg( \prg{c2}\ {\bf obeys}\ \prg{ValidClient} )\\ 
  \strut \hspace{0.2cm} 
 \rightarrow  \mbox{ \hlc[yellow]{{\tt c1}, {\tt c2} may be affected, but all {\tt ValidClient} not affected}} \ \ ] \  \ \ \wedge
 \\
\ensuremath{ [ } \ \    \prg{res}=\prg{false}  \wedge\, \prg{c1}\ {\bf obeys}\ \prg{ValidClient} \,   \wedge\, \prg{c2}\ {\bf obeys}\ \prg{ValidClient} \\
 \strut \hspace{0.2cm} 
 \rightarrow  \mbox{ \hlc[yellow]{ {\tt c1} or {\tt c2} had insufficient funds,  {\tt c1} and {\tt c2} and  all  {\tt ValidClient} not affected}}\ \ ] \  \ \ \wedge
 \\
\ensuremath{ [ } \ \     \prg{res}=\prg{false}\  \wedge\  (\  \  \prg{c1}\ {\bf obeys}\ \prg{ValidClient}  \   \veebar\   \prg{c2}\ {\bf obeys}\ \prg{ValidClient}\ \  ) \\
  \strut \hspace{0.2cm} 
 \rightarrow    \mbox{ \hlc[yellow]{ all  {\tt ValidClient} not affected}}\ \ ]  
\end{array}$ 
\end{minipage}
}
\label{DefEscrow}
\caption{Sketch of Escrow specification  --  \hlc[yellow]{yellow part described informally}}                     
 \end{figure*}                         
 % \vspace{.09cm}  
}


\begin{document}
 \begin{multicols}{1}



\title{ \vspace{-1.3cm}
Holistic Specifications for Robust Smart Contracts
 \vspace{-0.3cm}
 }
 

\author{ 
  \small{Sophia Drossopoulou, Imperial College London, UK}  \\
}
  
 \date{     }
  \end{multicols} 
  

 
\FigAuthorize

%  \vspace{-10cm}

\begin{multicols}{2}
 %\newcommand{\RoSpec}{{${\cal R}o{\cal S}mart{\cal S}pec$}}%{{${\cal H}ol{\cal S}spec$}}
 % {{${\cal R}obust{\cal S}pec$}}%{{${\cal H}ol{\cal S}spec$}}



\maketitle
% \vspace{-6cm}  -- makes no difference

 \vspace{-0.5cm}  
   
\sparagraph{Vision}

\noindent
%Smart contracts carry a great promise of wide accessibility and transparency, but current
%implementations are often not robust~\cite{multisigBugOne,multisigBugTwo,DaoBug,Ponzi}.  
We work on methodologies to deliver {\em robust} smart contracts. We call robust 
those smart contracts  which
behave correctly in the {\em open} world, i.e., when used  in unanticipated ways, and  by unknown 
participants.  

We propose that robustness should be at the forefront of smart contract development: It should be 
an {\em explicitly} specified concern,  and developers should use code-reviews and tools to guarantee 
adherence to it.

%\noindent
%Smart contracts holding considerable assets
%are used in the {\em open} world by unknown users in % all possible 
%unanticipated ways, and once deployed they cannot be modified. Thus, they
%are vulnerable to all kinds of attacks, and large amounts of moneys have been
%lost through careless or malicious programming~\cite{multisigBugOne,multisigBugTwo,DaoBug,Ponzi}. 

Traditional specification languages do not adequately express robustness,
as they were designed for the {\em closed} world.
%They were developed for the closed world, without unanticipated use in mind, and 
They attach pre- and post- conditions to each function of a contract, and thus
 give {\em sufficient} conditions for some effect to take place. They  are explicit about 
each individual function;  but are {\em implicit} about the overall behaviour emerging from the combination of 
all functions.
 Instead, in  the open world, 
 we are primarily interested in {\em necessary} conditions  (all possible causes)
 for effects, and in the  
behaviour of  a contract as a {\em whole}.


%Therefore, smart contracts need to be {\em robust} against all possible use scenarios.
%To achieve this,  we argue  that   robustness 
%aspects should be put at the forefront of the contract development:
%Robustness should be specified eas a separate concern, and 
%developers should be required to 
%demonstrate adherence to that concern.  

% 
%Traditional function pre- and post- conditions  are not appropriate here, as they were designed for the
%{\em closed} world:
%they give {\em sufficient} conditions for some effect to take place.  Instead, in
% the open world, we are interested in {\em necessary} conditions  
% for such effects.

% can best be characterized through novel specification concepts; 
% the development of  robustness specifications, and a proof of adherence of contracts to such specifications will make smart contracts 
%much more reliable.


\dparagraph{Holistic Specifications} % in\ \RoSpec}

\noindent
% To support the characterization of robustness, 
We have developed \RoSpec,  a  
{\em holistic} specification language which extends traditional %, functional 
 specification languages \cite{Leavens-etal07} with %aspects of robustness.
 novel concerns about:
%In particular: % we  have introduced concepts for: 

\noindent 
\vspace{-14pt}
\begin{description}% [topsep=0pt,itemsep=-1ex,partopsep=1ex,parsep=1ex]
 \setlength{\itemsep}{0pt}
  \item {Control}:\  $Call(\prg{x},\prg{y.f}(\prg{zs}),\prg{n})$ expresses that 
 currently,   \prg{x}  is the caller, \prg{y} the callee, \prg{f} the function, \prg{zs} the arguments, and  \prg{n} money
 was attached,
  \item {Time}:\ $Next$  and $Prev$ %, and $Fut$,  $Past$ 
         describe the next and previous %, and some future or past 
         snapshot of execution, 
   \item {Space}:\  $In(A,S)$ means that assertion $A$ holds when considering only  the objects from $S$.
   \item{Trust}:  \prg{x} {{\bf obeys}}  $A$ says that \prg{x} behaves according to the specification $A$.
\end{description}
\vspace{-4pt}
%The time aspects of \RoSpec~ 
Time   allows us   to reflect on %the current execution snapshot, but also 
all possible execution  traces. % (sequences of snapshots), 
Control   allows us to describe the causes of effects. %  who many cause what effect.
Space is used to distinguish permission and authority~\cite{MillerPhD}.
Trust   allows us to distinguish the expectations in the case where \prg{x} behaves according to $A$, from the risks 
in the case where it does not ~\cite{swapsies}.

\dparagraph{An example: ERC20 transfers} % in the ERC20}

\noindent
 ERC20~\cite{ERC20} keeps track of participants' tokens;  tokens may be transferred between participants, provided
 the transfer was instigated by the account holder, or somebody authorized by them.


%\vspace{.06cm}
%\noindent
%$Authorized(\prg{e},\prg{p},\prg{p}',\prg{m})$\ \ $\equiv$ \
%
% $ \left\{
%                            \begin{array}{l}
%                              Prev(\ Call(\prg{p},\prg{e.authorize(p,m')\ )}\  \wedge\  \prg{m}\leq \prg{m}'    \\
%                             \vee \\
%                              Prev(\ Authorized(\prg{e},\prg{p},\prg{p}',\prg{m})\ ) \,  \,   \wedge\,   \\
%                               \ \  \neg (Prev(Call(\prg{p}',\prg{e.transferFrom(p,cl'',m'')},\_ ) ) )\\
%                               \vee \\
%                              Prev(Authorized(\prg{e},\prg{p},\prg{p}',\prg{m}')  \,  \,   \wedge\, \prg{m}\leq \prg{m}'-\prg{m}''\, \wedge \, \\
%                               \ \  Prev(Call(\prg{p}',\prg{e.transferFrom(p,cl'',m'')},\_ ) ) 
%                        \end{array} 
%                         \right.$
% 
% \vspace{.09cm}  
% 
% \noindent
 
  
 In Fig. 1 
 % \ref{DefAuthorized} the ref does not work because of my trick
 we  define % in \RoSpec~
 what it means for $\prg{p}'$ to be authorized to spend \prg{m} tokens on behalf of $\prg{p}$: At some point in the
past,  \prg{p} gave authority to $\prg{p}'$  to spend on their behalf  a number %of tokens  
which is larger or equal   \prg{m} 
plus the sum of  tokens 
spent so far by $\prg{p}' $ on the behalf of \prg{p}. %on \prg{p}'s behalf 
%
%Wrt transfer of tokens, the  \RoSpec~specification from 

We also require that % below says that %~policy \prg{Pol\_ERC20\_tranfer} from below says:  
any decrease in  a participant's balance  %with an \prg{ERC20}   
 (i.e.,  $Next(\prg{e}.\prg{balance}(\prg{p}))$=$...$)  
is caused either by a transfer instigated by the account holder themselves (i.e., $Call(\prg{p},...)$), or by
a transfer instigated by another participant $\prg{p}''$  (i.e., $Call(\prg{p}''...$) who   had   been given authority earlier.
 % by the account holder 
%($Authorized(\prg{p},\prg{p}'',\prg{m})$).  
 
%\vspace{.06cm}
%
%\noindent  
%%\prg{Pol\_ERC20\_withdraw} \ $\equiv$ \\ 
%\strut \hspace{0.3cm} $\forall \prg{e}:\prg{ERC20}.\forall \prg{p}:\prg{Any}.\forall \prg{m}:\prg{Nat}.$\\
%\strut \hspace{0.3cm} $[\ \ Next(\prg{e}.\prg{balance(p)})=\prg{e}.\prg{balance(p)}-\prg{m}$ \\ %.\forall\prg{m}:\prg{Nat}.$\\
%\strut \hspace{0.4cm} \ \ \ $\longrightarrow$\\
%\strut \hspace{0.4cm} $\exists \prg{p}',\prg{p}'':\prg{Any}.$ \\
%\strut \hspace{0.4cm} $[\ \  Call(\prg{p},\prg{e.transfer(p'.m),\_)})\  \ \  \vee\, $\\
%\strut \hspace{0.4cm} $\ \ \ \ Call(\prg{p}'',\prg{e.transferFrom(p,cl',m),\_})\  \wedge\ $\\
%\strut \hspace{0.4cm} $\ \ \ \ Authorized(\prg{e},\prg{p},\prg{p}'',\prg{m}) \ ] \ \  ]$\\
%\strut \hspace{0.3cm} $] $
%%\strut \hspace{0.5cm} \ \ \ $\prg{d}.\prg{ether}\geq \prg{m}\ \wedge$ $\ Fut(Call(\prg{d.send(p)},m))\ \ ] $ 
%
%%\noindent
%%In summary, any decrease of balance is instigated either by the account holder or by somebody authorized to make such a transfer on their behalf. 
%
%\vspace{.05cm}

 This holistic specification  gives  to  account holders an "authorization-guarantee": their balance cannot decrease unless they themselves,  or somebody they had  authorized, instigates a transfer of tokens. Moreover, authorization is not transitive.
 
\FigEscrow 

\dparagraph{Comparison with Traditional Specifications}

%Traditional  specifications %for the ERC20 example would consist of 
%describe the behaviour of each function separately even for a single object. 
%They  give pre- and postconditions, % for a function,
% and thus describe a  {\em sufficient} condition for the effect of the particular function. 
\noindent
As stated earlier,  traditional specifications give {\em sufficient} conditions, e.g., if \prg{p''}  is authorized and executes \prg{transferFrom}, then   the balance decreases. But they are {\em implicit} about the overall behaviour and the   {\em necessary} conditions, 
e.g., what are all the possible actions that can cause a decrease of balance?
%effect (e.g, a decrease of balance) can only take place if the necessary conditions hold (e.g., the owner called \prg{transfer} or somebody authorized  called  \prg{transferFrom}). 

  
With traditional  specifications, to obtain the "authorization-guarantee", one would need to inspect the pre- and post- conditions of all the functions in the contract, and determine which ones decrease balances, and then determine which ones affect authorizations. Moreover, with traditional  specifications, nothing stops the next release of the contract to add, e.g., a method which 
allows participants to share their authority, and thus
 violate the "authorization-guarantee".

 
\dparagraph{Progress and Project Aims}

\noindent
In \cite{holistic} we  developed   \RoSpec, and   
% used it  to specify 
specified robustness aspects 
of  popular  
patterns from object-capabilities~\cite{MillerPhD} and smart contracts: 
the membrane~\cite{membranesJavascript}, %~\cite{membranesJavascript,membranes}, 
%(nobody can access an entity within the membrane, unless the membrane allows it), 
 Mint-and-Purse ~\cite{ELang,SolidityCoin},
 %(you cannot affect the currency of a Mint --Coin in Ethereum--  unless you have access --and the key-- to the Mint),  
DOM-wrappers~\cite{dd},
% (you cannot modify a DOM Node if all accesses to this DOM Node go through wrappers whose 
%height is smaller than their distance to the Node)  
ERC20~\cite{ERC20}, and  DAO~\cite{DAO}.
 

We now  want to apply our ideas to larger case studies. We also want to
develop methodologies with which
 to demonstrate that contracts adhere to
\RoSpec\ specifications, including  
  argued code-reviews, testing~\cite{testing}, inference systems,
and tool-based program verification~\cite{dafny,Z3}. 
Results on small, class-based languages and methodologies for code-reviews would take one and a half highly-qualified
person-years, moving to delegation-based would take another person-year, 
the development of verification logics one person-year, 
first stab at verification tools another two person-years, and
same amount for the testing infrastructure.
We collaborate with Susan Eisenbach (Imperial),  James Noble (VU Wellington) and 
 Toby Murray (Univ. Melbourne).

  
 

\vspace{-0.5cm}
{ \footnotesize{
{
  \setlength{\itemsep}{0pt} %hack into  bbl file
\bibliographystyle{abbrv}
\bibliography{Case}

}}}

\dparagraph{Appendix A:   DAO}
% 
% \noindent
in a similar style as that  of the ERC20 spec earlier, we can write a \RoSpec~specification 
requiring that  DAO ~\cite{DAO}  holds as much ether as the sum of
the   its clients' balances, and that balances may only be affected by clients joining or leaving, and
projects being approved or repaying.
This   precludes the famous  re-entrancy bug~\cite{DaoBug}.  

\dparagraph{Appendix B:  Escrow}
% 
% \noindent
Given clients \prg{c1} and \prg{c2} holding accounts in two currencies, an Escrow agent  
can exchange  \prg{k} of one currency against $\prg{k}'$ in the other,
%From the clients'   point of view the transfer is atomic, 
%The agent makes the exchange and reports success only if \prg{c1} and \prg{c2} had %has to take place atomically, and only if
%  
provided  \prg{c1} and \prg{c2}  had sufficient funds~\cite{proxiesECOOP2013}.
% A  Secure ECMAScript implementation was proposed in

One expects that the clients behave according to  spec \prg{ValidClient} --
e.g., do not try to withdraw more than entitled.
%, acknowledge receipt of funds, etc.
But what are the risks if some clients do not adhere to  \prg{ValidClient}, and there 
 is no central authority to certify them?
We sketch a spec for this  % the associated risks 
in Figure 2 (more in ~\cite{swapsies}):  
If the agent reports success (by returning \prg{true}), then either:\ (a)  both clients 
  satisfy  \prg{ValidClient} and had sufficient funds and 
the exchange did take place
 or,\  (b) \ neither
  satisfy  \prg{ValidClient}. 
Similarly, if the agent reports failure (by returning \prg{false}), then either (c) they both satisfy  \prg{ValidClient}  but did not have
sufficient funds, or (d) exactly one satisfies \prg{ValidClient}. 
Case (b) was surprising, esp. as  there is no way to implement the agent so as to internally distinguish  (a) from (b).
Nevertheless, the risk is limited: in all cases except (a), we have the guarantee that all %
\prg{ValidClient}  % except \prg{c1} and \prg{c2} 
are not affected.
% In (a) all \prg{ValidClient}    except \prg{c1} and \prg{c2}   are not affected.
  
  
%and all other \prg{ValidClient}'s are unaffected; \ or, neither
%  satisfy  \prg{ValidClient}, in which case we know nothing about the clients, but have the
% guarantee that no other \prg{ValidClient} is affected.  
%Similarly, if the agent reports fgailure (by returning \prg{false}), then either they were both good by did not have
%sufficient funds, or one of them was bad. 
%
%%\dparagraph{Appendix: Another Example, DAO}
%
%\noindent
%Consider a  simplified version of the DAO~\cite{DAO}:
%It keeps the moneys of a set of clients, and will refund them when they call the function \prg{repay}. 
%
%The \RoSpec~policy {Pol\_DAO\_withdraw} from below says:  
% If anybody (a \prg{p}) has a balance of \prg{m} at a \prg{DAO} contract \prg{d},
%and if \prg{p} calls \prg{repay} on \prg{d}, then  
% \prg{d} is required to hold at least \prg{m} ether at the time of that call ($\prg{d}.\prg{ether}\geq \prg{m}$), and will eventually send \prg{m} back to \prg{p} (expressed as $Fut(Call(\prg{d.send(p)},m)$):
%
%%Formally: 
%\vspace{.05cm}
%
%\noindent  
%\prg{Pol\_DAO\_withdraw} \ $\equiv$ \\ 
%\strut \hspace{0.5cm} $\forall \prg{d}:\prg{DAO}.\forall \prg{p}:\prg{Any}.\forall\prg{m}:\prg{Nat}.$\\
%\strut \hspace{0.5cm} $[\ \  Call(\prg{p},\prg{d.repay(),\_})\, \wedge\, \prg{d.Balance(p)}=\prg{m} $\\ 
%\strut \hspace{0.5cm} \ \ \ $\longrightarrow$\\
%\strut \hspace{0.5cm} \ \ \ $\prg{d}.\prg{ether}\geq \prg{m}\ \wedge$ $\ Fut(Call(\prg{d.send(p)},m))\ \ ] $ 
%
%\noindent
%The requirement that \prg{d} holds at least \prg{m} ether precludes the known re-entrancy bug~\cite{DaoBug}, that is, a contract
%which satisfies \prg{Pol\_DAO\_withdraw} will always have enough money to satisfy all \prg{repayment} requests. 
%
%\vspace{.005cm}
%
%We can now define  what it means for \prg{p} to have a  \prg{Balance} at  \prg{d}. The \prg{Balance}  is \prg{0} if the previous call was
%a repayment; it is \prg{m} if  the previous call was \prg{p} joining \prg{d} and paying in \prg{m}. More cases are needed to reflect the financing and repayments of proposals, but they can be expressed with the concepts described so far.
%
%\noindent
%$\prg{d.Balance(p)}$\ = \ $ \left\{
%                            \begin{array}{ll}
%                             \prg{0}, & \hbox{if}\ Prev(Call(\prg{p},\prg{d.repay(),\_})    \\
%                             \prg{m},  & \hbox{if}\  Prev(Call(\prg{p},\prg{d.join(),m}))   \\
% %                            m+n, & \hbox{if}\ Prev(Call(\prg{p},\prg{payIn(m),\_})\ \\
% %                              &  \ \strut \ \  \prg{d.Balance(p)}=n) \\
%                             .. & \hbox{if}\  ...
%                           \end{array} 
%                         \right.$
%

\end{multicols}
\end{document}


