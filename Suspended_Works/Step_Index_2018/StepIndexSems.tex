\documentclass[acmsmall,screen]{acmart}
  \citestyle{acmauthoryear}

  % \documentclass[acmsmall,10pt]{acmart}\settopmatter{printfolios=true} % ,review
% \citestyle{acmauthoryear}
\usepackage{subcaption}
\usepackage[T1]{fontenc}
\usepackage[utf8]{inputenc}
\usepackage[british]{babel}
\usepackage{xspace, listings, lstcustom, wrapfig, graphicx, enumerate}
\usepackage{paralist}
\usepackage{color, colortbl, relsize}
\usepackage{rotating}
\usepackage{pifont}
\usepackage{multirow}
\usepackage{soul}
\usepackage{tcolorbox}
\usepackage[scaled=.9, light]{zlmtt}
\usepackage{siunitx}
\usepackage{setspace}

   \newcommand{\ttt}{\prg{true}}
\newcommand{\ff}{\prg{false}}
\newcommand{\unkn}{\prg{b???}}
\newcommand{\bv}{\prg{bval}}


\newcommand{\prg}[1]{{\mbox{\tt{#1}}}}
 \newcommand{\prgCol}[1]{#1}

 \newcommand{\forget}[1]{}
\newcommand{\etc}{{\it etc.}}
\newcommand{\eg}{{\it e.g.\,}}
\newcommand{\ie}{{\it i.e.\,}}

\newcommand{\Future}[1]{\lozenge\, #1}% {\bullet #1}% {{{\mathcal F}}(#1)} % {{{\mathcal B}}(#1)}
\newcommand{\Using}[2]{#1\,\kw{in}\, #2} %{{{\mathcal U}}(#1,#2)}
\newcommand{\SigmaUsing}[2]{#1\@ #2} %{{{\mathcal U}}(#1,#2)}
\newcommand{\Past}[1] {\nabla #1} %{\lozenge\!\!\!\!\-\!\!-\,#1}
%{\lozenge\!\!\!\!\!\circ  #1} % {\lozenge\!\!\!\!\-\!\!- #1} %{\upupsilon #1}  %{\nabla #1} %{\circ #1}%  {{{\mathcal P}}(#1)}
\newcommand{\Initial}[1] {{{\mathcal I}\!nitial}(#1)}

\newcommand{\Pol}[1] {{\ensuremath{\prg{Pol}\_{\prg{#1}}}}}
%\newcommand{\MOne} {{\ensuremath{\prg{M}_{\prg{BA}}}}}
%\newcommand{\MTwo} {{\ensuremath{\prg{M}_{\prg{BA}'}}}}

\newcommand{\strongImplies}{\leqq} %{{ \,^\sqsubset\!\!\!_{\sim}\, }}
\newcommand{\weakImplies}{\lessapprox} %{{ \,^\sqsubset\!\!\!_{\sim}\, }}
\newcommand{\frames}{~\kw{frames}~}

\newcommand{\appref}[1]{see App.~\ref{#1}}

%\newcommand{\sE}{{\prg{e}}}

\newcommand{\LangOO} {\ensuremath{{\mathcal L}ang{_{\tt oo}}}}

% ------------------------------------------------------------------
%                                             positions, separations
\newcommand{\cf}{{\it c.f.~}}
%\newcommand{\HYPHENA}{{\em-- }}
%\newcommand{\HYPHENB}{{\em-- }}
%\newcommand{\SP}{{\hspace{.1in}}}
%\newcommand{\s}{{\hspace{.01in}}}

%\newcommand{\obeys}{\,\textbf{\textrm{obeys}}\,}
%\newcommand{\StrongDom}{\ensuremath{\mathcal{S}\textrm{\textit{trong}}{\mathcal{D}}\textrm{\textit{om}}}}
%\newcommand{\Dom}{\ensuremath{\mathcal{D}}\textrm{\textit{om}}}

\newcommand{\Changes}[1]{\ensuremath{\mathcal{C}\textrm{\textit{hanges}}(#1)}}
\newcommand{\VisibleLit}{\ensuremath{\mathcal{V}\textrm{\textit{isible}}}}

\newcommand{\Gives}{\ensuremath{\mathcal{G}\textrm{\textit{ives}}}}
\newcommand{\MayCall}{\ensuremath{\mathcal{M}\textrm{\textit{ay}}{\mathcal{C}}\textrm{\textit{all}}}}
%\newcommand{\Dom}{\ensuremath{\mathcal{D}\textrm{\textit{om}}}}
\newcommand{\MayRead}{\ensuremath{\mathcal{M}\textrm{\textit{ay}}{\mathcal{R}}\textrm{\textit{ead}}}}
\newcommand{\MayAccess}{\ensuremath{\mathcal{M}\textrm{\textit{ay}}{\mathcal{A}}\textrm{\textit{ccess}}}}
\newcommand{\CanAccess}[2]{\ensuremath{{\mathcal{A}}\textrm{\textit{ccess}}}(#1,#2)}
\newcommand{\Calls}[1]{\ensuremath{{\mathcal{C}}\textrm{\textit{alls}}}(\prg{#1})}
\newcommand{\Caller}{\ensuremath{{\mathcal{C}}\textrm{\textit{aller}}}}
%{\ensuremath{\mathcal{C}\textrm{\textit{an}}{\mathcal{A}}\textrm{\textit{ccess}}}(#1,#2)}
\newcommand{\WillAccessThrough}{\ensuremath{\mathcal{W}\textrm{\textit{ill}}{\mathcal{A}}\textrm{\textit{ccess}}{\mathcal{T}}\!\!\textrm{\textit{hrough}}}}
\newcommand{\modelsWithO}{\models\!\!\!\!{_{_{_{\tiny{\mathcal O}}}}}}
\newcommand{\A}{\ensuremath{A}}
\newcommand{\SA}{\ensuremath{{^{\small{\prg{s}}}\! A}}}
\newcommand{\SE}{\ensuremath{{^{\small{\prg{s}}}\!e}}}
\newcommand{\SEOne}{\ensuremath{{^{\small{\prg{s}}}\!e}}}
\newcommand{\SETwo}{\ensuremath{{^{\small{\prg{s}}}\!e'}}}
\newcommand{\B}{\ensuremath{B}}
\newcommand{\Arising}{{\mathcal{A}}\textrm{\textit{rising}}}

 %------------------------ syntax tables

\newcommand{\syntax}[1]{\prg{{\it #1}}}
\newcommand{\BBC}{$::=$} %in syntactic definitions
\newcommand{\SOR}{\ensuremath{\ \mid\ }} % BNF or
\newcommand{\MID}{{\SPsmall ~ \mid ~ \SPsmall }} % in sets


\newcommand{\pre}{\ensuremath{_{{pre}}}}   %kjx no \sc  in math mode
\newcommand{\post}{\ensuremath{_{{post}}}} %kjx no \sc  in math mode
\newcommand{\PRE}{\pre}
\newcommand{\POST}{\post}

%\newcommand{\eval}[2]{{\ensuremath{\langle{ {#1}}\rangle_{#2}}}}
\newcommand{\interp}[2]{{\ensuremath{\lfloor{ {#1}}\rfloor_{#2}}}}
%\newcommand{\interpBL}[1]{{\lceil   {#1}  \rfloor}}
%  { \langle \!\langle { {#1} \rangle \!\rangle}\! }
% { \langle   { {#1}  \rangle} }
% ------------------------------------------------------------------
%                                             English abbreviations

% ------------------------------------------------------------------
%                                              keywords, program text
\newcommand{\kw}[1]{{\bf{\sf {#1}}}}
%{\mbox{\prgCol{\rm{\bf {#1}}}}}
\newcommand{\lit}[1]{{\prg {#1}\xspace}}
\newcommand{\com}{\ensuremath{\prg{//}}}
%\newcommand{\cnf}{\ensuremath{\kappa}}

 \newcommand{\code}[1]{{\mbox{\tt{#1}}}}
\newcommand{\M}{\ensuremath{\prg{M}}}
%\newcommand{\C}{\ensuremath{\prg{C}}}

%\newcommand{\ext}{\mbox{\,\,{\kw {extends}}\,\,}}
%\newcommand{\extends}{\mbox{\,\,{\kw {extends}}\,\,}}
%\newcommand{\ass}{\mbox{{\kw {:=}}\,}}
\newcommand{\semi}{\mbox{{\kw {;}}\ }}
%\newcommand{\comma}{\mbox{{\kw {,}}\,}}
%\newcommand{\dotK}{\mbox{{\kw {.}}}}
%\newcommand{\class}{\mbox{{\kw {class}}\,\,}}
%\newcommand{\stat}{\mbox{{\kw {state}}}}
%\newcommand{\trans}{\mbox{{\kw {root}}}}
%            % {\mbox{{\kw {root state}}}}
%            % was {\mbox{{\kw {abs-state}}}}
\newcommand{\lb}{\prgCol{\mbox{\tt{\bf{\{ }}}}}
\newcommand{\rb}{\prgCol{\mbox{\tt{\bf{\} }}}}}
\newcommand{\lp}{\prgCol{\mbox{\tt{\bf{( }}}}}
\newcommand{\rp}{\prgCol{\mbox{\tt{\bf{) }}}}}
 





\newcommand{\assertTC}[2]{{\M{#1} \typCol{\vdash} {\prg{#2}}\s
    \typCol{\DDiamond_{r} }}}
\newcommand{\assertSC}[2]{{\M{#1} \typCol{\vdash} {\prg{#2}}\s
    \typCol{\DDiamond_{s}} }}




% \newcommand{\M}[1]  {{\ensuremath{\prg{M}{{\prg{#1}}}}}}
    % {\prg{P}}
%\newcommand{\Env}[1]{\envCol{\ensuremath{\Gamma{#1}}}}
% \newcommand{\state}[1]{\stCol{\ensuremath{\sigma???{#1}}}}
%\newcommand{\stackFrame}[1]{\stCol{\ensuremath{\phi{#1}}}}
%\newcommand{\heap}[1]{\stCol{\ensuremath{\chi{#1}}}}

%\newcommand{\expr}[1]{{\ensuremath{\prg{e{#1}}}}}
%\newcommand{\fld}[1]{{\ensuremath{\prg{f{#1}}}}}
%\newcommand{\param}{{\ensuremath{\prg{x}}}}
%
%\newcommand{\clss}[1]{\ensuremath{\prg{c}{\prg{#1}}}}
%\newcommand{\clssD}[1]{\ensuremath{\prg{d}{{#1}}}}
%\newcommand{\type}[1]{\ensuremath{\prg{t{#1}}}}
%
%\newcommand{\val}[1]{{\ensuremath{\prg{v}{\prgCol{#1}}}}}
%\newcommand{\res}[1]{{\ensuremath{\prg{dv}{#1}}}}
%\newcommand{\valOrDev}[1]{{\ensuremath{\prg{r}{#1}}}}


%\newcommand{\expandexp}[1]{}
%
%\newcommand{\oo}{object-oriented}
%\newcommand{\mExtS}{\ensuremath{\Downarrow}}
%
%% re-classification expression
%\newcommand{\cm}[1]{\this{\prgCol{\ensuremath{\mExtS}}}\prg{#1}}






% ------------------------------------------------------------------
%                                             identifiers in the examples
%                     ---------------------
%                                      Stack
 %                     ---------------------
%                                   Employee
\newcommand{\Empl}{\prg {Empl}}
\newcommand{\Pers}{\prg {Pers}}
\newcommand{\E}{{_\prg {e}}}
\newcommand{\Man}{\prg {Boss}}
\newcommand{\Stud}{\prg {Stdt}}
\newcommand{\Scholar}{\prg {Scholar}}
\newcommand{\sal}{\prg {sal}}
\newcommand{\bYear}{\prg {bYear}}
\newcommand{\frnd}{\prg {frnd}}
% \newcommand{\marks}{\prg {marks}}
\newcommand{\pay}{\prg {fee}}
\newcommand{\setP}{\prg {set}}
\newcommand{\String}{\prg {String}}
\newcommand{\hobby}{\prg {hobby}}
\newcommand{\intg}{\kw {int}}
\newcommand{\boolg}{\kw {bool}}
\newcommand{\ii}{\prg {i}}
\newcommand{\prom}{\prg {promote}}
\newcommand{\mkStud}{\prg {study}}
\newcommand{\dem}{\prg {demote}}
\newcommand{\grad}{\prg {employ}}
\newcommand{\assist}{\prg {assist}}
\newcommand{\amount}{\prg {amount}}

\newcommand{\Phil}{\prg {Phil}}
\newcommand{\Book}{\prg {Book}}
\newcommand{\Person}{\prg {Person}}
\newcommand{\fav}{\prg {favourite}}

\newcommand{\aMan}{\prg {billy}}
\newcommand{\anotherMan}{\prg {bob}}
\newcommand{\aMutMan}{\prg {bea}}
\newcommand{\employees}{employees}
\newcommand{\aStud}{\prg {steve}}
\newcommand{\aPers}{\prg {peter}}
\newcommand{\aStudEmpl}{\prg {mary}}
\newcommand{\anoStudEmpl}{\prg {judy}}
\newcommand{\yetAnoStudEmpl}{\prg {betty}}
\newcommand{\yetYetAnoStudEmpl}{\prg {joe}}

\newcommand{\exprog}{_{\prg{ei}}}



% structuring macros
\newcommand{\EndDefLemma}{\noindent $\bigtriangleup$}



\newcommand{\para}[1]{{\ensuremath{\prg{x}{#1}}}}


\newcommand{\methAndBodyNew}{\ensuremath{
\prg{\type{}~m \lp \type{_1}~\param{}\rp\   \ \lb
~\expr{}~\rb} }}
\newcommand{\methAndBodyNewOne}{\ensuremath{
\prg{\type{}~m \lp \type{_1}~\param{}\rp\   \ \lb
~\expr{'}~\rb} }}
\newcommand{\methAndBodyNewTwo}{\ensuremath{
\prg{\type{}~m \lp \type{_1}~\param{}\rp\   \ \lb
~\expr{''}~\rb} }}
\newcommand{\methAndBodyNewThree}{\ensuremath{
\prg{\type{''}~m \lp \type{_1''}~\param{}\rp\   \ \lb
~\expr{''}~\rb} }}
\newcommand{\methAndBodyNewPrime}{\ensuremath{
\prg{\type{}~m \lp  \type{_1}~\para{} \rp\   \
\lb~\expr{'}~\rb} }}
\newcommand{\methAndBodyNewPrimeAll}{\ensuremath{
\prg{\type{'}~m \lp  \type{_1'}~\para{} \rp\   \
\lb~\expr{'}~\rb} }}
\newcommand{\methAndBodyNewP}{\ensuremath{ % slight diff in Primes from above
\prg{\type{}~m \lp  \type{_1}~\para{} \rp\   \
\lb~\expr{'}~\rb} }}
\newcommand{\methAndBodyFromOneSig}{\ensuremath{
\prg{\type{}~m \lp  \type{_1}~\para{} \rp\ \eff{} \lb~ ... ~\rb}
}}

%-----------------

\newcommand{\Body}[2]{\ensuremath{\mathcal{B}ody(#1,\prg{#2})}}

\newcommand{\T}[1]{{\ensuremath{\type{}{#1}}}}

% find the declaration of an identifier
\newcommand{\LookUp}[2]{\ensuremath{ {#1}({\prg{#2}}) }}
\newcommand{\LookUpEnv}[2]{{ \Env{#1}({\prg{#2}}) }}
\newcommand{\LookUpState}[2]{{ \ensuremath{\sigma{#1}}({\prg{#2}}) }}
\newcommand{\LookUpHeap}[2]{{ \ensuremath{\heap{#1}}({\prg{#2}}) }}
\newcommand{\LookUpBody}[2]{{ {\M{#1}}({\prg{#2}}) }}
%\newcommand{\LookUpClass}[2]
%        { \ensuremath{{\mathcal C}%{\mathcal D}%{\it ef}
%        (}\M{#1},\prg{#2}\ensuremath{)}}
\newcommand{\LookUpField}[3]
        {\ensuremath{{\mathcal F}{\mathcal D}% {\it f}
                (}\M{#1},\prg{#2},\prg{#3}\ensuremath{)}}
\newcommand{\LookUpMethod}[3]
        {\ensuremath{{\mathcal M}{\mathcal D}% {\it f}
         (}\M{#1},\prg{#2},\prg{#3}\ensuremath{)}}
\newcommand{\Undef}{{\ensuremath{\mathcal U\!}{\it df}}}
\newcommand{\ClassOf}[2] { \ensuremath{{\mathcal C}{\mathit{lass}}(#1)_{#2}}}

%-------------------Part Three: Types ...
% Notation for types ( open and close types).


\newcommand{\subclass}{\typCol{\sqsubseteq}}
\newcommand{\sclass}{\subclass}
\newcommand{\widen}{\typCol{\ensuremath {\le}}}

%-------------------- implies, and, or, iff, etc -----------------
\newcommand{\inset}[3]{\prg{#1}\!\in\!\prg{#2},\ldots,\prg{#3}}
% do not change above, it is also used with \forall, \exists
%\newcommand{\IMPLIES}{{\SP \Longrightarrow \SP}}
 \newcommand{\AND}{{\SPsmall {\mbox{and}} \SPsmall}}
\newcommand{\WITH}{{\SPsmall {\mbox{with}} \SPsmall}}

%\newcommand{\IFF}{{\SPsmall {\mbox{iff}} \SPsmall}}
\newcommand{\IFF}{{\SP {\mbox{ iff }} \SP}}

\newcommand{\OR}{{\SPsmall {\mbox{or}} \SPsmall}}
\renewcommand{\implies}{{\ensuremath{\longrightarrow}}}
\newcommand{\upd}{{\mapsto}}

%----------------------- Conformance --------------------
\newcommand{\stateConf}[3]
  {\M{#1},\Env{#2}
  \rtjCol{\ensuremath{\vdash}}\,\ensuremath{\state{#3}}\,\rtjCol{\ensuremath{\DDiamond}}  }
\newcommand{\stackFrameHeapConf}[4]
  {\M{#1},\Env{#2}
  \rtjCol{\ensuremath{\vdash}}\,\ensuremath{\stackFrame{#3},\heap{#4}}\,\rtjCol{\ensuremath{\DDiamond}}  }
\newcommand{\stackFrameHeapConfS}[4]
  {\M{#1},\Env{#2}
  \rtjCol{\ensuremath{\vdash}}\,\ensuremath{ {#3},\heap{#4}}\,\rtjCol{\ensuremath{\DDiamond}}  }

\newcommand{\stackFrameHeapConfNOT}[4]
  {\M{#1},\Env{#2}
  \rtjCol{\ensuremath{\not\vdash}}\,\ensuremath{\stackFrame{#3},\heap{#4}}\,\rtjCol{\ensuremath{\DDiamond}}  }

\newcommand{\stackFrameHeapConfLong}[4]
  {\M{#1}, {#2}
  \rtjCol{\ensuremath{\vdash}}\,\ensuremath{\stackFrame{#3},\heap{#4}}\,\rtjCol{\ensuremath{\DDiamond}}  }
\newcommand{\stackFrameHeapConfLongNOT}[4]
  {\M{#1}, {#2}
  \rtjCol{\ensuremath{\not\vdash}}\,\ensuremath{\stackFrame{#3},\heap{#4}}\,\rtjCol{\ensuremath{\DDiamond}}  }


\newcommand{\stateConfEnv}[3]
  {\M{#1}, {#2}
  \rtjCol{\ensuremath{\vdash}}\,\ensuremath{\state{#3}}\,\rtjCol{\ensuremath{\DDiamond}}  }
  \newcommand{\stateConfLong}[3]
  { {#1}, {#2}
        \rtjCol{\ensuremath{\vdash}}\,\ensuremath{\state{#3}}\,\rtjCol{\ensuremath{\DDiamond} } }
\newcommand{\stateConfLongNot}[3]
  { {#1}, {#2}
        \rtjCol{\ensuremath{\vdash}\!\!\!\not}\ \ensuremath{\state{#3}}\,
         \rtjCol{\ensuremath{\DDiamond} } }
\newcommand{\conf}[4]
  {\M{#1},\ensuremath{\heap{#2}}
        \rtjCol{\ensuremath{\vdash}} \ensuremath{\prg{#3}} \rtjCol{\ensuremath{\lhd}}
        \ensuremath{\prg{#4}}}
\newcommand{\confW}[4]
  {\M{#1},\ensuremath{\heap{#2}}
        \rtjCol{\ensuremath{\vdash}} \ensuremath{\prg{#3}} \rtjCol{<:}
        \ensuremath{\prg{#4}}}
% states conforming to each other:


\newcommand{\confState}[5]
    % Prog, eff, thsiMut, state, state
  {\confStateL{\M{#1}}{\eff{#2}}{\clss{#3}}{\state{#4}}{\state{#5}}}
  \newcommand{\confStateP}[4]
    % Prog, eff, state, state
  {\ensuremath{\M{#1},{\eff{#2}}\vdash{\state{#3}}\lhd{\state{#4}}}}
\newcommand{\confStateL}[5]
  {\ensuremath{#1,{#2},{#3}\vdash{#4}\lhd{#5}}}
\newcommand{\confStateLP}[5]
  {\ensuremath{#1,{#2}\vdash{#3}\lhd{#4}}}




% ------------------------------------------- Type assertions ----------------
\newcommand{\assert}[3]{\prg{#1} \vdash \prg{#2}~:~\prg{#3}}
% \newcommand{\assertT}[5] {\M{#1}, \Env{#2} \   \ensuremath{\vdash} \ \prg{#4}~
% \ensuremath{:~\prg{#5}}}
\newcommand{\EFFSEP}{\typCol{[\!]}}

\newcommand{\assertTAll}[6]
    % program, env, expr, type, mut, eff
{\M{#1}, \Env{#2} \   \typCol{\ensuremath{\vdash}} \ \prg{#3}~
\ensuremath{\typCol{:~} \prg{#4} } }% ~\EFFSEP~ {#5}~\EFFSEP~{#6} }}
\newcommand{\assertTAllS}[6]
% with subsumption
    % program, env, expr, type, mut, eff
{\M{#1}, \Env{#2} \   \typCol{\ensuremath{\vdash_s}} \
\prg{#3}~
\ensuremath{\typCol{:~} \prg{#4} } }% ~\EFFSEP~ {#5}~\EFFSEP~{#6} }}
\newcommand{\assertTAllR}[6]
% with subsumption
    % program, env, expr, type, mut, eff
{\M{#1}, \Env{#2} \   \typCol{\ensuremath{\vdash_r}} \
\prg{#3}~
\ensuremath{\typCol{:~} \prg{#4} } }% ~\EFFSEP~ {#5}~\EFFSEP~{#6} }}
\newcommand{\assertTAllL}[6]
    % program, env, expr, type, mut, eff
{{#1}, {#2} \   \typCol{\ensuremath{\vdash}} \ \prg{#3}~
\ensuremath{\typCol{:~}} \prg{#4} }%~\EFFSEP~ {#5}~\EFFSEP~{#6} }}
\newcommand{\assertTFour}[4]
    % for the explanations, ignore P and \eff
{ {#1} \   \typCol{\ensuremath{\vdash}} \ \prg{#2}~
\ensuremath{\typCol{:~} \prg{#3} ~\EFFSEP~ {#4}  }}
% same as above, but in 2 lines!
% do not remove! it is difficult to program, so
% leave it in, even if temporarily not used!
\newcommand{\assertTAllTwoL}[6]
    % program, env, expr, type, mut, eff
{\M{#1}, \Env{#2} \   \ensuremath{\vdash} \ \prg{#3}~
\ensuremath{:~\\ \SP\SP\SP\SP\SP
   \prg{#4} }}
\newcommand{\assertTAllEnvS}[6]
    %as before, but the environment is given in full
{\M{#1}, {#2} \   \typCol{\ensuremath{\vdash}_s} \ \prg{#3}~
\ensuremath{\typCol{:~}}\prg{#4}  }
\newcommand{\assertTAllEnvR}[6]
    %as before, but the environment is given in full
{\M{#1}, {#2} \   \typCol{\ensuremath{\vdash}_r} \ \prg{#3}~
\ensuremath{\typCol{:~}}\prg{#4}  }
\newcommand{\assertTAllEnv}[6]
    %as before, but the environment is given in full
{\M{#1}, {#2} \   \typCol{\ensuremath{\vdash}} \ \prg{#3}~
\ensuremath{\typCol{:~}}\prg{#4}  }




 % Effects


% receiver mutations

\newcommand{\mutOr}[1] % takes as parameter the program
        {\ensuremath{\sqcup\!{_{{#1}}}}} % binary on muts

\newcommand{\assertc}[3]{\ensuremath{\M{#1}
 \typCol{\vdash}\, \prg{#2}\, \subclass\, \prg{#3}}}

 \newcommand{\assertw}[3]{\ensuremath{\M{#1} \typCol{\vdash}
\prg{#2}\widen\prg{#3}}}
\newcommand{\assertion}[3]{\ensuremath{#1 \typCol{\vdash} \prg{#2}~#3}}
\newcommand{\notStateConfEnv}[3]
  {\M{#1},\prg{#2}\,
  \rtjCol{\ensuremath{\not\vdash}}\,\ensuremath{\state{#3}}\,\rtjCol{\ensuremath{\DDiamond}}  }
 \newcommand{\notConf}[4]
  {\M{#1},\ensuremath{\heap{#2}}
        \rtjCol{\ensuremath{\not\vdash}} \ensuremath{\prg{#3}} \rtjCol{\ensuremath{\lhd}}
        \ensuremath{\prg{#4}}}







%Macros for inference rules
\newcommand{\inferencerule}[2]{
\begin{array}{l} #1 \\ \hline #2 \end{array}
}

\newcommand{\inferenceruleN}[3]
{
\begin{array}{l}
% \SP\SP\SP\SP\SP\SP\SP\SP
% \SP\SP\SP\SP\SP\SP\SP\SP
\SP\SP\SP\SP\SP\SP\SP\SP
\SP\SP\SP\SP\SP\SP  {\sf #1}
\\ #2  \\ \hline   #3
  \end{array}
}

\newcommand{\inferenceruleNN}[3]
{
\begin{array}{l}
\SP\SP\SP\SP\SP\SP\SP\SP
\SP\SP\SP\SP\SP\SP\SP\SP
\SP\SP\SP\SP\SP\SP\SP\SP
\SP\SP\SP\SP\SP\SP\SP\SP

   {\sf #1}
\\ #2  \\ \hline   #3
  \end{array}
}

%===========================================================================
%  Definition-Lemma-Theorem-Proof
%
% Adaptation of LaTeX's theorem environment; can be used as a command
% (eg just \Lemma not \begin{Lemma}) and no italicisation; also works
% with ptmac; result numbering is uniform within subsections and can be
% suppressed.
%
\newif\ifNumberResults\NumberResultstrue
\def\@@opargbegintheorem#1#2#3{\@@@@begintheorem{\bf\@@thmname{#1}{#2}(#3)}}
\def\@@begintheorem#1#2{\@@@@begintheorem{\bf\@@thmname{#1}{#2}}}
\def\@@@@begintheorem#1{\par\removelastskip\smallskip\noindent{#1}}
\def\@@thmname#1#2{#1\ \ifNumberResults#2\ \fi}

% similarly \Proof or \begin{Proof}...\end{Proof}
% prefer proofs with statements if possible - hence \penalty700
%\let\qedsymbol\S% make it \square or \blacksquare if you like for kb
\let\qedsymbol \Box
\def\qed{\hfill{$\qedsymbol$}}
\def\Proof{\par\removelastskip\smallskip\penalty700\noindent{\bf Proof}\enskip}
\def\endProof{\qed\penalty-700 \smallskip}
\let\endproof\endProof

%   The actual words

%\newtheorem{theo}{Theorem} this is necessay if we want eg \newtheorem{definition}[theo]{Definition} to work 
\newtheorem{definition}{Definition} %  \newtheorem{definition}[theo]{Definition}
\newtheorem{example}{Example} %  \newtheorem{example}[theo]{Example}
%\newtheorem{lemma}{Lemma} % \newtheorem{mylemma}[theo]{Lemma}
%\renewtheorem{mylemma}{Lemma} % \newtheorem{mylemma}[theo]{Lemma}
\newtheorem{conjecture}{Conjecture} %\newtheorem{conjecture}[theo]{Conjecture}
\newtheorem{theorem}{Theorem} %\newtheorem{theo}{Theorem}
\newtheorem{note}{Note} % \newtheorem{note}[theo]{Note}
\newtheorem{observation}{Observation} %  \newtheorem{observation}[theo]{Observation}


%--------------------------------- the ones that Susan introduced
\newcommand{\z}{{\prg z}}

\newcommand{\Fields}[3]{\ensuremath{{\mathcal F}(}\\Mg{#1},\prg{#2},
\prg{#3}\ensuremath{)} }
\newcommand{\FieldIds}[2]{\ensuremath{{\mathcal F}{\it {s}}(\M{#1},\prg{#2})}}
\newcommand{\Meths}[3]{\ensuremath{{\mathcal M}(}\M{#1},\prg{#2},
\prg{#3}\ensuremath{)} }







 


\newcommand{\rewriteLong}[1]
{\rtCol{\ensuremath{\ \leadsto\!\!\!\!\!\!_{_{#1}\,\ \ }}}}
\newcommand{\rewrite}[1]
{\rtCol{\ensuremath{\ \leadsto\!\!\!\!\!\!_{_{\M{#1}}\,\ \ }}}}
\newcommand{\rewriteP}
    {{\ensuremath{\ \ \leadsto\!\!_{_{\prg{P}}}\,\,\,}}}
 % {\rtCol{\ensuremath{\ \ \leadsto \!\!\!\!_{{\M{}}\,\ \ }}}}
% \newcommand{\Conf}[2]% configurations: < #1, #2 >
% inside the mathmode
% { \langle \prg{#1}, {#2} \rangle }
% {  \prg{#1}\ensuremath{,}{#2}  }
%\newcommand{\greenComma}{\rtCol{\ensuremath{,}}}
%\newcommand{\ConfL}[3]
%{  \prg{#1}\greenComma\stackFrame{#2}\greenComma\heap{#3}  }
%\newcommand{\ConfLLong}[3]
%{  \prg{#1}\greenComma {#2}\greenComma\heap{#3}  }
%\newcommand{\ConfR}[2]
%{  \prg{#1}\greenComma\heap{#2}  }
%

% ------------------------------------- Well formed, unique acyclic ---------------
 
%{\ensuremath{\M{#1} \vdash {\prg{#2}}\s \DDiamond_{\it {eff}}  }}

  \newcommand{\nullPEC}{\lit {nullPntrExc}}



%\newcommand{\Obj}{\prg{Object}}
%\newcommand{\void}{\kw{void}}
%\newcommand{\newK}{{\kw {new}}~} % {{\kw {new}}}% no ~ around it
%\newcommand{\newKW}{~{\kw {new}}~}
%




\newcommand{\back}{{$\!\!\!\!\!\!\!$}}


\newcommand{\WideFig}[3]
{
\begin{figure*}[t]
\begin{center}
\noindent
\fbox{
\begin{minipage}{4.7 in}
{#1} % the contents
\end{minipage}
}
\caption{#2}
\label{#3}
\end{center}
\end{figure*}
}


\newcommand{\WideFigWhere}[4] % you can specify where it should appear!
{
\begin{figure*}[{#4}]
\begin{center}
\noindent
\fbox{
\begin{minipage}{5. in}
{#1} % the contents
\end{minipage}
}
\caption{#2}
\label{#3}
\end{center}
\end{figure*}
}

\newcommand{\BigWideFigWhere}[4] % you can specify where it should appear!
{
\begin{figure*}[{#4}]
\begin{center}
\noindent
{\normalsize
\hrule
\begin{minipage}{5. in}
{#1} % the contents
\end{minipage}
\hrule
}
\caption{#2}
\label{#3}
\end{center}
\end{figure*}
}

\newcommand{\NotTooWideFigWhere}[4] % you can specify where it should appear!
{
\begin{figure*}[{#4}]
\begin{center}
\noindent
\fbox{
\begin{minipage}{4.3 in}
{#1} % the contents
\end{minipage}
}
\caption{#2}
\label{#3}
\end{center}
\end{figure*}
}


\newcommand{\opsemExprFig}
{\BigWideFigWhere {\opsemExpr} {Execution of expressions\MD}
{opsemTrad} {htbp} }



\newcommand{\mlc}{ }%{\heartsuit}
%\newcommand{\mcl}{ }%{\heartsuit}
\newcommand{\mc}{ }%{\heartsuit}

\newcommand{\BigNotTooWideFigWhere}[4] % you can specify where it should appear!
{
\begin{figure*}[{#4}]
\begin{center}
\noindent
{\normalsize
\hrule
\begin{minipage}{4.3 in}
{#1} % the contents
\end{minipage}
\hrule
}
\caption{#2}
\label{#3}
\end{center}
\end{figure*}
}

 

%]})
%}


% \setcopyright{rightsretained}
%\acmPrice{}
%\acmDOI{10.1145/3133896}
%\acmYear{2018}
%\copyrightyear{2018}
%\acmJournal{PACMPL}
%\acmVolume{1}
%\acmNumber{????}
%\acmArticle{72}
%\acmMonth{10}
%
%\citestyle{acmauthoryear}
%
%
%
%\copyrightyear{2017}
%\copyrightdata{978-1-nnnn-nnnn-n/yy/mm}
%\doi{nnnnnnn.nnnnnnn}



%\usepackage[usenames]{color}

\usepackage{times}
 \usepackage{latexsym}
\usepackage{listings}
\definecolor{dkgreen}{rgb}{0,0.6,0}
\definecolor{gray}{rgb}{0.5,0.5,0.5}
\definecolor{mauve}{rgb}{0.58,0,0.82}


\lstset{ %
  language=Java,                % the language of the code
  mathescape=true,
  basicstyle=\footnotesize\tt,           % the size of the fonts that are used for the code
  numbers=left,                   % where to put the line-numbers
  numberstyle=\tiny\color{dkgreen},  % the style that is used for the line-numbers
  stepnumber=1,                   % the step between two line-numbers. If it's 1, each line
                                  % will be numbered
  numbersep=5pt,                  % how far the line-numbers are from the code
  backgroundcolor=\color{white},      % choose the background color. You must add \usepackage{color}
  showspaces=false,               % show spaces adding particular underscores
  showstringspaces=false,         % underline spaces within strings
  showtabs=false,                 % show tabs within strings adding particular underscores
  frame=single,                   % adds a frame around the code
  rulecolor=\color{black},        % if not set, the frame-color may be changed on line-breaks within not-black text (e.g. commens (green here))
  tabsize=2,                      % sets default tabsize to 2 spaces
  captionpos=b,                   % sets the caption-position to bottom
  breaklines=true,                % sets automatic line breaking
  breakatwhitespace=false,        % sets if automatic breaks should only happen at whitespace
  title=\lstname,                   % show the filename of files included with \lstinputlisting;
                                  % also try caption instead of title
  keywordstyle=\color{blue},          % keyword style
  commentstyle=\color{gray},       % comment style
  stringstyle=\color{mauve},         % string literal style
  escapeinside={\%*}{*)},            % if you want to add LaTeX within your code
  morekeywords={private,public,final,this,throw,new,||,to,def,any,fun,fld,abstract,policy,specification,ghost,field}
         % if you want to add more keywords to the set
}

% \newcommand{\scd}[1]{{\color{blue}{#1}}}
% \newcommand{\jn}[1]{{\color{green}{#1}}}
% \newcommand{\sd}[1]{{\color{dkgreen}{#1}}}
% \newcommand{\sd}[1]{{{#1}}}

\begin{document}

%\preprintfooter{internal memo}

\author{Sophia D} \author{Holistic Group} \author{help from Alex S. and Krysia B.} \affiliation{zzz}


%\authorinfo{Sophia Drossopoulou$^1$, James Noble$^{2,1}$, Toby Murray$^4$, Mark Miller$^3$, Shupeng Loh$^1$, Susan Eisenbach$^1$}{$^1$Imperial College London, $^2$Victoria University Wellington, $^3$Google Inc, $^3$NICTA and UNSW.}{}


\title{Step-index like semantics for our assertions}
\subtitle{Work in progress}


\begin{abstract}

How we define the semantics for assertions through step-indexing or similar ideas. The ensuing logic is not classical.
We propose some alternatives, and discuss their ramifications.
In this version we do not include the \obeys-predicate.
\end{abstract}


\maketitle


\section{Introduction}

\subsection{Preliminaries}

We have the following entities:
\begin{itemize}
\item expressions $e$ with the expected syntax, and
\item assertions $A$,
\item modules $\M$ which hold the  definitions of classes, methods and functions, and the function $\Body{\M}{\prg{id}}$ which returns the body of a function or a predicate \prg{id}, as defined in \M.
\item states $\sigma$ which contain a stack of frames. where each frame has the receiver, the method id, the arguments, the local variables, and the code being executed
\end{itemize}

The above are more or less as described in the July documents. For clarity we define assertions $A$  (we skip some parts of the syntax for the time being)

\begin{definition}
[Assertions] have the following syntactic form

$A$ \ :;= \ \prg{true} \ | \ \prg{false} \ | \ \prg{e} \ | \ $A \rightarrow A$ \ | \ $A \vee A$ \ | \  $\neg A $ \ |  \prg{PredId{e}}.

\end{definition}

\subsection{Examples}

In this paper we will consider  functions, predicates and states, described below.

\noindent
The function   \prg{acyclic}  returns \prg{true}
if the list is acyclic, and is defined defined as \\
\strut \ \ \ \ \prg{acyclic(x)} \ \ $\equiv$\\
\strut \ \ \ \ {\small{\prg{if x:List then (if x.next==null then true else acyclic(x.next)) else false}}}.
\\
The function  \prg{length}  returns the length of a list, and is defined defined as \\
\strut \ \ \ \ \prg{length(x)} \ \ $\equiv$\\
\strut \ \ \ \ {\small{\prg{if x:List then (if x.next==null then 0 else 1+length(x.next)) else 0}}}


\noindent
We also define the predicates \prg{Finite} and  $\prg{Finite}'$ similar to the function \prg{acylcic},  as follows:\\
\strut \ \ \ \ \prg{Finite(x)}\ \ \ $\equiv$     \ \ \  $\prg{x.next}\neq\prg{null} \ \rightarrow\ \prg{Finite(x.next)})$.\\
% \strut \ \ \ \  \ \ \ {\small{$(\prg{x.next==null} \rightarrow \prg{true}) \wedge  (\prg{x.next!=null} \rightarrow \prg{Finite(x.next)})$}}.\\
\strut \ \ \ \ $\prg{Finite}'(\prg{x})$\ \ \ $\equiv$     \ \ \  $\prg{x.next}=\prg{null} \ \vee\  \prg{Finite}'(\prg{x.next})$.\\
And we define  the    contradictory predicate \prg{Cntrdct},\\
\strut \ \ \ \  $\prg{Cntrdct}(\prg{x})\ \ \ \equiv \ \ \ \neg \prg{Cntrdct}(\prg{x})$. \\

\noindent,
Moreover, we assume a state $\sigma_{ac}$ such that
$\sigma_{ac}(\prg{z.next.nex}t)$=\prg{null}\footnote{This needs to be written better.}
And we assume another state,
$\sigma_{cyc}$, where \prg{z} is a cyclic \prg{List} object, and
$\sigma_{cyc}(\sigma_{cyc}(\prg{z}),\prg{next})=\sigma_{cyc}(\prg{z})$.

\subsection{What we wiil do}

We now define the judgments
\begin{itemize}
\item $\M, \sigma, e \leadsto_k v$ which says that $e$ evaluates to $v$ in up to $k$ steps
\item $\M, \sigma \models_k A$  which says that validity of $A$ in  $\M, \sigma$ can be established in up to $k$ steps.
\item $\M, \sigma \models  A$  which says that $A$ is valid in $\sigma$.

\item $\M, \sigma\leadsto_k \sigma'$ which says that state $\sigma$ evaluates to new state $\sigma'$ in up to $k$ steps
\end{itemize}

We will not give complete definitions here, but show the most salient parts below.



\section{Evaluation of expressions}


\begin{definition}
\label{def:expr:step:index}
Given a module $\M$, a state $\sigma$, and a natural number $k$ we define the judgment $\M, \sigma, e \leadsto_k v$ by cases on \prg{e} as follows:
\begin{itemize}
\item
$\M, \sigma, \prg{x} \leadsto_k v$ \ \ \  iff \ \ \  $\prg{x}\in Dom(\sigma)$  and $\sigma(\prg{x})$=$v$, for any value of $k$.
\item
$\M, \sigma, \prg{e.f} \leadsto_k v$\ \ \  iff \ \ \  $\M, \sigma, \prg{e} \leadsto_{k} \iota$ for some $\iota$,  % and $k\geq 1$,
and $\sigma(\iota,\prg{f})$=$v$.
\item
$\M, \sigma, \prg{e1}==\prg{e2} \leadsto_k \prg{true}$\ \ \  iff \ \ \  $\M, \sigma, \prg{e1} \leadsto_{k} v$, \ and \ $\M, \sigma, \prg{e2} \leadsto_{k} v$ for some $v$.
\item
$\M, \sigma, \prg{e1}==\prg{e2} \leadsto_k \prg{false}$\ \ \  iff \ \ \  $\M, \sigma, \prg{e1} \leadsto_{k} v1$, \ and \ $\M, \sigma, \prg{e2} \leadsto_{k} v2$, and $v1\neq v2$. % for some $v1$, $v2$.
\item
$\M, \sigma, \prg{if\ e0 then\ e1\ else e2} \leadsto_k v$\ \ \  iff \ \ \    $\M, \sigma \leadsto_{k-1} \prg{true} $ \ and\ $\M, \sigma, \prg{e1} \leadsto_{k-1} v$ \\
\strut \hspace{2.6in} or\ $\M, \sigma \leadsto_{k-1} \prg{false} $\  and\ $\M, \sigma, \prg{e2} \leadsto_{k-1} v$,\\
\strut \hspace{2.6in}  and  $k\geq 1$.
\item
$\M, \sigma, \prg{fnId(e)} \leadsto_k v$\ \ \  iff \ \ \    $\M, \sigma, \prg{e} \leadsto_{k-1} v'$  and $\Body{\M}{\prg{fnId}}$=$\prg{e}'$ \\
 \strut \hspace{1.55in} and  $\M, \sigma[\prg{z}\mapsto v'], \prg{e'}[\prg{z}/\prg{x}]  \leadsto_{k-1} v $ and \prg{z}   free in \prg{e} and  $\sigma$,\\
 \strut \hspace{1.55in} and $k\geq 1$.
\end{itemize}

 We also define, more generally:
\begin{itemize}
\item
$\interp {e}{\M,\sigma}$=$v$ \ \  iff \ \  $\exists k.\forall j\geq k. \,  \M, \sigma, \prg{e} \leadsto_j v$.
\end{itemize}
\end{definition}

Note that expressions have no side-effects. The evaluation $\leadsto_k$ is monotonic, in the sense that if execution of up to $k$ steps returns a result, then any larger number of steps returns the same result.

\begin{mylemma}[Expression Evaluation is monotonic]
\label{lemma:expr:eval:monotinic}
For any module $\M$, any state $\sigma$, expression \prg{e},  any natural numbers $k$, $j$ and values $v$ and $v'$ we have:
\begin{itemize}
\item
$\M, \sigma, \prg{e} \leadsto_k v$ \ \ \  $\longrightarrow$ \ \ \  $\M, \sigma, \prg{e} \leadsto_{k+j} v$.
\item
$\M, \sigma, \prg{e} \leadsto_j v \ \wedge\ \M, \sigma, \prg{e} \leadsto_k v'$  \ \ \  $\longrightarrow$ \ \ \  $v$=$v'$
\end{itemize}

\end{mylemma}

\begin{proof}
The lemma is easy to prove. Just a sketch below

\begin{itemize}
\item
By well-founded ordering introduced on the structure of \prg{e} and value of $k$\footnote{Here to find and use the right term from the step-index lit. - but it is well-founded ordering thing.}. Only the base case, where \prg{e}=\prg{v}, and the function calls are  a bit interesting cases.
\item
Similar argument.
\end{itemize}
\end{proof}


 In some cases,  there exists no $k$ and $v$ such that $\M, \sigma, \prg{x} \leadsto_k v$; therefore the expression $\interp {e}{\M,\sigma}$ is not always defined. For example,   the terms $\interp {\prg{acyclic(z)}}{\M,\sigma_{cyc}}$
and  $\interp {\prg{lenght(z)}}{\M,\sigma_{cyc}}$ are undefined.\footnote{In Summers-Drossopoulu-ECOOP11 we had special error values for such cases.} And the term $\interp {\prg{if acyclic(z)  then false else true}}{\M,\sigma_{cyc}}$, which corresponds to the negation of \prg{acyclic} is also undefined.


\section{Code Execution}



We now define what it means to execute code.\footnote{Or perhaps we can be parametric with that.  This sections is not that interesting for the time being.}

\begin{definition}
\label{def:execution:step:index}
Given a module $\M$ a state $\sigma$ and a natural number $k$ we define the judgment $\M, \sigma  \leadsto_k \sigma'$ by cases on $\sigma$.\prg{code} as follows:
\begin{itemize}
\item
$\M, \sigma  \leadsto_k \sigma$ for all $k$
\item
For all $k$, \\
 if \  $\sigma[\prg{x},\prg{f}\mapsto\sigma(\prg{y}),\prg{code}\mapsto \prg{stmts}]  \leadsto_k \sigma'$, \ then \
$\sigma[\prg{code} \mapsto \prg{x.f=y; stmts}] \leadsto_{k+1} \sigma'$
\item
... skipping the rest
\end{itemize}
\end{definition}

\begin{mylemma}[Code execution is monotonic]
For any module $\M$, any states $\sigma$, $\sigma'$, and $\sigma''$, and any natural numbers $k$, $j$ we have:

\label{lemma:exec:monotinic}
$\M, \sigma  \leadsto_k \sigma'$ and $\M, \sigma  \leadsto_j \sigma''$
and $j\geq k$ imply that % $\sigma'=\sigma''$ or
$\M, \sigma'  \leadsto_{j-k} \sigma''$.
\end{mylemma}
\begin{proof}
TODO.
\end{proof}

\section{Semantics of assertions through step-indexing}

Here step-indexing is important and interesting.
%
The basic structure of the definition has to follow the LICS style, and add a case for predicates.\footnote{Note the in Kripke predicate logics the predicates have an interpretation, and no bodies.}
There is a large design space:

\begin{enumerate}
\item
Do we define implication a) in terms of the "next" world, as in Kripke semantics, or b) in terms of the previous world, as in the LICS paper?
\item
For the base case, i.e.  for $\M, \sigma \models_0 A$ , do we have
\begin{enumerate}
\item $\M, \sigma \models_0 A$ always (as in LICS), or
\item  $\M, \sigma \models_0 A$ never (this is inductive, and firs the execution of expressions),
\item   $\M, \sigma \models_0 A$ iff $A$ is an expression $e$ and $\M, \sigma, e \leadsto_0 \prg{true}$.
\end{enumerate}
\item
For implication, we have\\
\strut   $ \ \ \hspace{.6in} $ a)  \ \  $\M, \sigma \models_k A \rightarrow A'$  \ \ \  iff \ \ \   $\M, \sigma \models_{k-1} A$ implies that $\M, \sigma \models_{k-1}   A'$, or
\\
\strut   $ \ \ \hspace{.6in} $ b) \ \ $\M, \sigma \models_k A \rightarrow A'$  \ \ \  iff \ \ \     $\forall j< k.\ \M, \sigma \models_{j} A$ implies that $\M, \sigma \models_{j}   A'$
\item
For disjunction do we have\\
\strut   $ \ \ \hspace{.6in} $ a) \ \  $\M, \sigma \models_k A \vee A'$  \ \ \  iff \ \ \   $\M, \sigma \models_{k} A$ or $\M, \sigma \models_{k}   A'$, or
\\
\strut   $ \ \ \hspace{.6in} $ b) \ \ $\M, \sigma \models_k A \vee A'$  \ \ \  iff \ \ \     $\forall j< k.\ \M, \sigma \models_{j} A$ or $\M, \sigma \models_{j}   A'$

\item
For validity, do we have \\
\strut   $ \ \ \hspace{.6in} $ a) \ \  $\M, \sigma \models  A'$  \ \ \  iff \ \ \   $\forall k\in \mathbb{N}.\ \M, \sigma \models_{k} A$, \ \ or \\
\strut   $ \ \ \hspace{.6in}$  b) \ \ $\M, \sigma \models  A'$  \ \ \
  iff \ \ \  $\exists k.\ \forall j\geq  k.\ \M, \sigma \models_{j} A$
\end{enumerate}

Wrt (1), I cannot see how the definition in terms of a future world could  work for u, because most of  predicates (e.g. \prg{Finite}) have an inductive (least fixpoint reading). Therefore I discard this avenue. Similarly,
So, there are 18 alternatives.

In the below, I am choosing 2c)-3a)-4b)-5b). I believe that the only other sensible alternative is to chose 2c)-3a)-4a)-5b.
\footnote{I do not like 2a)  as it would give that  $\M, \sigma \models_0 \neg   A$ as well as $\M, \sigma \models_0    A$. Similarly, 2b) gives $\M, \sigma \models_0  \neg A$ as well as $\M, \sigma \models_0 \neg \neg A$}
\footnote{In the LICS paper they start with $\M, \sigma \models_0 A$ always. But they do not have predicates, let alone recursive ones. But if we did that, then we would get, as shown below, that $\M, \sigma \models \prg{Acyclic(z)}$. And it makes sense to have a base case of invalid, as we have not yet dedicated enough fuel to investigate...
}

\subsection{Step-indexing with 2c)-3a)-4b)-5b}

 \begin{definition}
\label{def:assetion:step:index}
 Given a module $\M$ a state $\sigma$ and a natural number $k$ we define the judgment $\M, \sigma \models_k A$ by cases on $A$  and $k$  as follows:

\noindent
For all $k$ we have
% \begin{enumerate}[a]
\begin{itemize}
\item
\label{def:assrt:expr}
$\M, \sigma \models_k e$ \ \ \  iff \ \ \   $\M, \sigma, e  \leadsto_{k} \prg{true}$.\footnote{This rule implies
that $\M, \sigma \models_k \prg{true}$  always and $\M, \sigma \models_k \prg{false}$.}
% \footnote{should it be iff $\M, \sigma, e  \leadsto_{k-1} \prg{true}$? I do not think so, but to remember}
\end{itemize}

\noindent
For $k\geq 1$ we define
% \begin{enumerate}[a]
\begin{itemize}
\item
\label{def:assrt:impl}
$\M, \sigma \models_k A \rightarrow A'$  \ \ \  iff \ \ \   $\M, \sigma \models_{k-1} A$ implies that $\M, \sigma \models_{k-1}   A'$. %, and $k\geq 1$.
\item
\label{def:assrt:vee}
$\M, \sigma \models_k A \vee A'$  \ \ \  iff \ \ \   $\forall j\leq k.\ \M, \sigma \models_{k}   A$\ or\   $\forall j\leq k.\ \M, \sigma \models_{k}   A'$. %\item
%\label{def:assrt:Future}
%$\M, \sigma \models_k \Future{A}$ \ \ \  iff \ \ \   $\exists \sigma'.\, \M, \sigma  \leadsto_{k-1} \sigma'$ and $\M, \sigma' \models_{k-1}  {A}$.\footnote{We can ignore this case for the time being, as the issues artise even without looking at this case.}
%\item
%\label{def:assrt:past}
%$\M, \sigma \models_k \Past{A}$ \ \ \  iff \ \ \   ... like, and a bit more complex than above, doable.
%\item
%\label{def:assrt:Using}
%$\M, \sigma \models_k \Using{A}{\prg{S}}$ \ \ \  iff \ \ \   $\M, \sigma |_{S} \models_k  A$ where $\interp{\prg{S}}{\M,\sigma}$=$S$
%\footnote{We need to think a bit about $\interp{S}{\M,\sigma}$. Also, should it be $\M, \sigma |_{S} \models_{k-1}  A$, and does it make a difference?}
\item
\label{def:assrt:Pred}
$\M, \sigma \models_k {\prg{PredId}(\prg{e})}$\ \ \  iff \ \ \   $\exists v.\, \M, \sigma, \prg{e} \leadsto_{k-1} v$  and $\Body{\M}{\prg{PredId}}$=$\prg{A}$ \\
 \strut \hspace{1.55in} and  $\M, \sigma[\prg{z}\mapsto v']  \models_{k-1} \prg{A}[\prg{z}/\prg{x}]  $ and \prg{z}   free in \prg{A} and  $\sigma$.
% \strut \hspace{1.55in} and $k\geq 1$.
%\end{enumerate}
\end{itemize}

 As well, we define
 \begin{itemize}
  \item
 $\M, \sigma \not\models_k A$\ \  iff\ \  $\M, \sigma\models_k   A$ does not hold. \end{itemize}

And also
\begin{itemize}
\item
$\M, \sigma \models  A$\ \  iff\ \  $\exists k.\forall j\geq k. \,  \M, \sigma\models_j  A$.
% \item
% $\M, \sigma \not\models  A$\ \  iff\ \  $\exists k.\forall j\geq k. \,  \M, \sigma\not\models_j  A$.
\end{itemize}

\end{definition}

\begin{note}[The meaning of negation] We use $\neg A$ as shorthand for $A \rightarrow \prg{false}$. Therefore, we have that
\ \ $\M, \sigma \models_k \neg A$   \  iff \    $\M, \sigma \models_{k}   A \rightarrow \prg{false}$, which means that  \ \
 $\M, \sigma \models_k \neg A$   \  iff \   $\M, \sigma \not\models_{k}   A $.

\end{note}


\noindent{\bf{Discussion}}


\begin{example}[acyclic and Finite]
In the acyclic configuration $\sigma_{acyc}$, we can establish that x is finite after three steps. That is


\begin{tabular}{llll}
\ \ \ \ & (A) & $\M,\sigma_{acyc}  \models_0   \prg{x.next.next}=\prg{null} $,& by def \ref{def:assetion:step:index}, base case.\\
\ \ \ \ & (A') & $\M,\sigma_{acyc}  \not \models_0   \prg{x.next}=\prg{null} $,& by def \ref{def:assetion:step:index}, base case.\\
\ \ \ \ & (B) & $\M,\sigma_{acyc} \not\models_0  \prg{Finite}(\prg{x.next})$,& by def \ref{def:assetion:step:index}, base case.\\
\ \ \ \ & (C) & $\M,\sigma_{acyc}  \not\models_0   \prg{Finite}(\prg{x})$,& by def \ref{def:assetion:step:index}, base case.\\
\ \ \ \ & (D) & $\M,\sigma_{acyc}   \models_1   \prg{Finite}(\prg{x.next})$,& by A, and def \ref{def:assetion:step:index} ind.  case for predicates,\\
\ & & & and inductive case for implications\\
\ \ \ \ & (E) & $\M,\sigma_{acyc}   \not\models_1   \prg{Finite}(\prg{x})$,& by A' and C, and def \ref{def:assetion:step:index}  ind.  case for predicates\\
\ & & & and inductive case for implications\\
\ \ \ \ & (F) & $\M,\sigma_{acyc}    \models_2   \prg{Finite}(\prg{x})$,& by A' and D, and def \ref{def:assetion:step:index}  ind.  case for predicates\\
\ & & & and inductive case for implications\\
\ \ \ \ & (G) & $\M,\sigma_{acyc}    \models_3   \prg{Finite}(\prg{x})$,& by A' and F,  and def \ref{def:assetion:step:index}  ind.  case for predicates\\
\ & & & and inductive case for implications\\
& ... & ...
\end{tabular}

Therefore,    $\M,\sigma_{acyc}    \models_k   \prg{Finite}(\prg{x})$, for all $k \geq 2.$, which gives us that
  $\M,\sigma_{acyc}    \models   \prg{Finite}(\prg{x})$.


\vspace{.01in}
In the cyclic configuration, as we saw earlier, $\interp{\prg{acyclic(z)}}{\sigma_{cyc}}$ is undefined. Nevertheless, we  $\M, \sigma_{cyc} \models_k \neg \prg{Finite}({z})$ for all $k$.
  We also have that $\M, \sigma_{cyc} \models_k \neg \prg{length}(z)=3$, even though  \prg{length} and also $\prg{length}(z)=3$ are ubdefined
 in the context of $\sigma_{cyc}$.
This gives us   also that
$\M, \sigma \models \neg \prg{Finite}({z})$, and $\M, \sigma_{cyc} \models_k \neg \prg{length}(z)=3$.
\end{example}

\begin{example}[Cntrdr]

 We have $\M,\sigma \not\models  \prg{Cntrdct}(\prg{z})$ as well as $\M,\sigma \not\models    \neg\prg{Cntrdct}(\prg{z})$.
 Namely,

\begin{tabular}{llll}
\ \ \ \ & (0) & $\M,\sigma \not\models_0  \prg{Cntrdct}(\prg{z})$,& by def \ref{def:assetion:step:index}, base case.\\
& (1) &  $\M,\sigma \models_1   \prg{Cntrdct}(\prg{z})$, &  by (0) and def \ref{def:assetion:step:index}, inductive case for predicates.
\\
& (2) & $\M,\sigma \models_2   \neg \prg{Cntrdct}(\prg{z})$, & by (1) and def \ref{def:assetion:step:index}, inductive  case for predicates,\\
& & & negation, implication and \prg{false}\\
& (3)&  $\M,\sigma \models_3    \prg{Cntrdct}(\prg{z})$, & by (2) and def \ref{def:assetion:step:index}, inductive step case for predicates, \\
& & & negation and implication.
\\
& & ...
\end{tabular}

So we obtain that\\
\strut \hspace{.34in}(F1) \hspace{.3in}   $\M,\sigma \models_k   \prg{Cntrdct}(\prg{z})$ for all odd $k$, and $\M,\sigma \models_k   \neg\prg{Cntrdct}(\prg{z})$ for  all even $k$.

Therefore,\\
\strut \hspace{.3in}(F2) \hspace{.3in}  $\M,\sigma \not\models   \prg{Cntrdct}(\prg{z})$ als well as $\M,\sigma \not\models   \neg\prg{Cntrdct}(\prg{z})$. \\
And also:\\
\strut \hspace{.3in}(F3) \hspace{.3in}  $\M,\sigma \not\models   \prg{Cntrdct}(\prg{z}) \vee   \neg\prg{Cntrdct}(\prg{z})$.


\end{example}

\begin{example}[$\prg{Acyclic}'$]

In the acyclic configuration $\sigma_{acyc}$ we cannot estabish the predicate $\prg{Acyclic}'$. This is because of our choice of 4b. In more detail.

\begin{tabular}{llll}
\ \ \ \ & (A) & $\M,\sigma_{acyc}  \not \models_k   \prg{x.next}=\prg{null} $,& for all $k$,   by def \ref{def:assetion:step:index}, base case.\\
\ \ \ \ & (B) & $\M,\sigma_{acyc}  \not\models_0   \prg{Finite}'(\prg{x})$,& by def \ref{def:assetion:step:index}, base case.\\
\ \ \ \ & (C) & $\M,\sigma_{acyc}   \not\models_1   \prg{Finite}'(\prg{x})$,& by A, B, and def \ref{def:assetion:step:index} ind.  case for predicates,\\
\ & & & and inductive case for disjunction\\
\ \ \ \ & (D) & $\M,\sigma_{acyc}   \not\models_2   \prg{Finite}'(\prg{x})$,& by A, C, and def \ref{def:assetion:step:index} ind.  case for predicates,\\
\ & & & and inductive case for disjunction\\

\end{tabular}

Therefore $\forall k. \M,\sigma_{acyc}   \not\models_k   \prg{Finite}'(\prg{x})$, which gives that $\M,\sigma_{acyc}   \not\models    \prg{Finite}'(\prg{x})$

\end{example}

\begin{observation}[Lack of Monotonicity]

 In contrast with LICS'09 (and all other step-index works I know), the judgment $\M, \sigma \models_k  A$ is not monotonic with $k$. Namely, it is possible to have that $\M, \sigma \models_k  A$, as well as
$\M, \sigma \not\models_{k+j}  A$.
\\
An example can be seen in (F1).

\end{observation}

\begin{observation}[Lack of Modus Ponens]

We do not have modus ponens. A counterexample appears in (F2) and in (F3.

$Question$ Do we have modus ponens "in the steps", ie it does not hold for all $k$ that either
$\M, \sigma \models_k  A$ or $\M, \sigma \models_k  \neg(A)$?


\end{observation}


\begin{observation}[Distributivity of Disjunction]

In our system, disjunction distributes, ie we have

\begin{mylemma}[Disjunction is distributive] For all $A$.

\begin{itemize}
\item
 $\M, \sigma \models_k  A \vee A' $ implies $\M, \sigma \models_k  A $ or  $\M, \sigma \models_k A' $ for all $k$.
 \item
 $\M, \sigma \models  A \vee A' $ implies $\M, \sigma \models   A $ or  $\M, \sigma \models  A' $
\end{itemize}
 \end{mylemma}


Note that if we had chosen 4a) rather than 4b), the above would not hold, Namely, we call $\models_{alt}$ the inference system 2c)-3a)-4b)-5b), and obtain:
\\
\strut \hspace{.3in}(F4) \hspace{.3in}   $\M,\sigma \models_{alt,k}   \prg{Cntrdct}(\prg{z})\ \vee\   \neg\prg{Cntrdct}(\prg{z})$ for  all   $k$.

Which gives that,\\
\strut \hspace{.3in}(F5) \hspace{.3in}   $\M,\sigma \models_{alt}   \prg{Cntrdct}(\prg{z})\ \vee\   \neg\prg{Cntrdct}(\prg{z})$.

And also:\\
\strut \hspace{.3in}(F6) \hspace{.3in}  $\M,\sigma \not\models_{alt}   \prg{Cntrdct}(\prg{z})$  and   $\M,\sigma \not\models_{alt}   \neg\prg{Cntrdct}(\prg{z})$.

\end{observation}

\begin{conjecture}[Consistency]
For all $\M$, $A$, $A'$ and $\sigma$.

\begin{itemize}
\item
 It is never the case that $\M, \sigma \models  A$  and  $\M, \sigma \models   \neg A $.
 \item
 $\M, \sigma \models  A \rightarrow A' $ and  $\M, \sigma \models  A$ entails  $\M, \sigma \models  A'$
 \item
  $\M, \sigma \models  \neg A \vee A' $ entails  $\M, \sigma \models   A \rightarrow A' $
  \item
 $\M, \sigma \models  A \vee A' $ implies $\M, \sigma \models   A $ or  $\M, \sigma \models  A' $
\end{itemize}

\end{conjecture}
\begin{proof} This is a sketch

\begin{itemize}
\item
 We will show that for all $k$ it is impossible to have $\M, \sigma \models_k  A  $ and  $\M, \sigma \models_k  \neg A$.
 \item
 Expecting this to work by unfolding the definitions.
 \item
Expecting this to work by unfolding the definitions. This property is less crucial tan the other two, but holds in intuitionistic logic.
 \item
 This property is less crucial tan the other two, but holds in intuitionistic logic.
\end{itemize}
\end{proof}

\subsection{Considering holistic aspects}

In  Definition \ref{def:assetion:step:index} we have not considered the holistic aspects of assertions. We will therefore extend the definition as below

\begin{definition}
\label{def:assetion:hol:step:index}
 Given a module $\M$, a state $\sigma$, and a natural number $k$, we define the judgment $\M, \sigma \models_k A$ by cases on $A$  and $k$  as follows, where $k>0$ and arbitrary:

 % \begin{enumerate}[a]
\begin{itemize}
\item
all cases from Definition \ref{def:assetion:step:index}, and in addition:
\item
\label{def:assrt:obeys:basic}
$\M, \sigma \models_0 x \obeys \prg{S}$ \ \ \  always.
\item
\label{def:assrt:obeys}
$\M, \sigma \models_k x \obeys \prg{S}$ \ \ \  iff \\
  for all  $\sigma'$, for all    assertions $A$ from \prg{S},\ \ if  $ \M, \sigma  \leadsto_{k-1} \sigma'$ then  $\M, \sigma' \models_{k-1}  {A[\prg{this}\mapsto \prg{x}]}$.

\item
\label{def:assrt:Future}
$\M, \sigma \models_k \Future{A}$ \ \ \  iff \ \ \   $\exists \sigma'.\, \M, \sigma  \leadsto_{k-1} \sigma'$ and $\M, \sigma' \models_{k-1}  {A}$.
\item
\label{def:assrt:past}
$\M, \sigma \models_k \Past{A}$ \ \ \  iff \ \ \   ... like, and a bit more complex than above, doable.
\item
\label{def:assrt:Using}
$\M, \sigma \models_k \Using{A}{\prg{S}}$ \ \ \  iff \ \ \   $\M, \sigma |_{S} \models_k  A$ where $\interp{\prg{S}}{\M,\sigma}$=$S$
\footnote{We need to think a bit about $\interp{S}{\M,\sigma}$. Also, should it be $\M, \sigma |_{S} \models_{k-1}  A$, and does it make a difference?}
\end{itemize}

 \end{definition}
\begin{note} that we treat predicates differently to specifications. Namely, for all predicates $\prg{P}$, we have that $\M, \sigma \not\models_0 \prg{P}(\prg{x})$, while for any specification, \prg{S}, we have that $\M, \sigma \models_0 x \obeys \prg{S}$. That is, we define validity of predicates inductively, and validity of assertions co-inductively. We take the smallest fixpoints for predicates and take the largest fixpoints for assertions.
\end{note}


------------

\section{Semantics of assertions through unfolding}

Here an alternative way of defining validity of assertions -- less well thought at the moment.
We define unfolding of assertions in the obvious way, and  then validity of an assertion through validity of finite unfolding:

\begin{definition}
\label{def:unfold}
 We define the relation $\M, A \leadsto_k A'$ as follows, where $k\in \mathbb{N}$:

\noindent
 \begin{itemize}
\item
$\M, x \leadsto_k x$,  and $val \leadsto_k val$  for any k, and for all $val\in\{ \prg{null}, \prg{true}, \prg{false} \}$.
\item
$\M, e.f \leadsto_k e'.f$ if $e \leadsto_k e'$, for any k.
\item
$\M, func(e) \leadsto_{k+1} e'$ if $e \leadsto_k e''$,  and $FBody(fun)[x \mapsto e'']   \leadsto_k e'$
\item
$\M,  A \rightarrow A' \leadsto_{k+1} A'' \rightarrow A'''$  \ \ \  iff \ \ \   $ \M, A  \leadsto_{k}  A''$ and $ \M, A'  \leadsto_{k} A'''$
\item
$ \M, A \vee A' \leadsto_{k+1} A'' \vee A'''$  \ \ \  iff \ \ \   $ \M, A  \leadsto_{k}  A''$ and $\M,  A'  \leadsto_{k} A'''$
\item
$  {\prg{PredId}(\prg{e})} \leadsto_{k+1} A'$\ \ \  iff \ \ \   $e \leadsto_k e'$,  and $FBody(\M, \prg{PredId})[x \mapsto e']   \leadsto_k A'$\end{itemize}


\end{definition}

The $\leadsto_k$ relation is deterministic\footnote{better word needed}, i.e. $A \leadsto_k A'$ and $A \leadsto_k A''$ implies that $A'$=$A''$.


\begin{definition}
\label{def:three:values}
 We define three valued logic carrier set $\{ \ttt, \ff, \unkn \}$, and operations $\vee$, $\rightarrow$ as follows, where $\bv$ is an arbitrary value from $\{ \ttt, \ff, \unkn \}$:
 
 \noindent
 \begin{itemize}
\item The operator $\vee$
\begin{itemize}
\item $\ttt \vee \bv$ $\triangleq$ $\tt$
\item $\ff \vee \bv$ $\triangleq$ $\bv$
\item $\unkn \vee \bv$ $\triangleq$ $\bv$
\end{itemize}

\item The operator $\rightarrow$
\begin{itemize}
\item $\ttt \rightarrow \bv$ $\triangleq$ $\bv$
\item $\ff \rightarrow \bv$ $\triangleq$ $\ttt$
\item $\unkn \rightarrow \bv$ $\triangleq$ $\bv$
\end{itemize}
\end{itemize}

The operators $\wedge$   and $\neg$ can be encoded.
 \end{definition}

 \begin{definition}
\label{def:assertion:unfold}
Given   a state $\sigma$ and an assertion $A$ we define  $\eval {A} {\sigma}$ as follows

\begin{itemize}
\item
$\eval { \prg{PredId}(\prg{e})} {\sigma}$  $\triangleq$    $\unkn$
\item
 $\eval {e} {\sigma}$  $\triangleq$ $\bv$\ iff \  $\emptyset, \sigma, e \leadsto _0 \bv$.\footnote{Notice that we have the empty module $\emptyset$,
and so cannot unfold functions in this judgment.}
\item
$\eval {A \rightarrow A'} {\sigma}$  $\triangleq$    $\eval {A}{\sigma}  \rightarrow \eval {A'}{\sigma}$
\item
$\eval {A \vee A'} {\sigma}$  $\triangleq$    $\eval {A}{\sigma}  \vee \eval {A'}{\sigma}$%\item
\end{itemize}

 Given a module $\M$ a state $\sigma$ and a natural number $k$ we define the judgment  $\eval { \prg{PredId}(\prg{e})} {\sigma,k}$   as follows:

\begin{itemize}
\item
 $\eval { A} {\sigma,0}$ $\triangleq$   $\eval {A'} {\sigma}$, where $A \leadsto_0 A'$.
 \item
 $\eval { A\wedge A'} {\sigma,k+1}$ $\triangleq$   $\eval {A' \wedge A''} {\sigma}$, where $A \leadsto_k A'$ and $A \leadsto_{k+1} A''$

 \item
 $\eval { A\rightarrow A'} {\sigma,k+1}$ $\triangleq$   $\eval { A\rightarrow A'} {\sigma,k} \wedge \eval {A''\rightarrow A'''} {\sigma} $, where $A \leadsto_k A''$ and $A' \leadsto_{k+1} A'''$
\end{itemize}

 As well, we define
 \begin{itemize}
  \item
 $\M, \sigma \not\models_k A$\ \  iff\ \  $\eval {A'} {\sigma,j} \neq \ttt$.
  \end{itemize}

And also
\begin{itemize}
\item
$\M, \sigma \models  A$\ \  iff\ \  $\exists k.\forall j\geq k.  \eval {A} {\sigma,j} = \ttt$.
% \item
% $\M, \sigma \not\models  A$\ \  iff\ \  $\exists k.\forall j\geq k. \,  \M, \sigma\not\models_j  A$.
\end{itemize}

\end{definition}



\subsection{Discussion}


\begin{example}[acyclic and Finite]
In the acyclic configuration $\sigma_{acyc}$, we can establish that x is finite after three steps. That is\\


\begin{tabular}{llll}
\ \ \ \ & (A) & $\prg{Finite}(\prg{x}) \leadsto_2 $\\
& & $\prg{x.next }\neq\prg{null} \rightarrow $\\
& & $~ ~ \hspace{.2in}( \prg{x.next.next}\neq\prg{null} \rightarrow \prg{Finite}(\prg{x}.\prg{next}.\prg{next})$
 & by def \ref{def:unfold}, for all $k\geq 2$ \\
 \ \ \ \ & (B) & $\eval {\prg{Finite}(\prg{x})}{\M,\sigma_{acyc} ,1} =\unkn $ & by def \ref{def:unfold}, and  def \ref{def:assertion:unfold}.\\
\ \ \ \ & (C) & $\eval {\prg{Finite}(\prg{x})}{\M,\sigma_{acyc} ,2} =\ttt $ & by (A) and def \ref{def:assertion:unfold}.\\
\ \ \ \ & (D) & $\eval {\prg{Finite}(\prg{x})}{\M,\sigma_{acyc} ,k}  =\ttt  $ & for all $k\geq 2$, by (A) and def \ref{def:assertion:unfold}.\\
 & (E) & $\M,\sigma_{acyc} \models \prg{Finite}(\prg{x})  $,& by (D) and def \ref{def:assertion:unfold}.\\
& ... & ...
\end{tabular}

Similarly, we have

\begin{tabular}{llll}
\ \ \ \ & (A) & $\prg{Finite}'(\prg{x}) \leadsto_k$ \\
& & $\prg{x.next}=\prg{null} \vee $\\
& &  $(\prg{x.next.next}=\prg{null} \vee \prg{Finite}'(\prg{x}.\prg{next}.\prg{next}) )$ & by def \ref{def:unfold}, base case, if $k\geq 2$ \\
 \ \ \ \ & (B) & $\eval {\prg{Finite}'(\prg{x})}{\M,\sigma_{acyc} ,1} =\unkn $ &by def \ref{def:unfold}, and  def \ref{def:assertion:unfold}.\\
\ \ \ \ & (C) & $\eval {\prg{Finite}'(\prg{x})}{\M,\sigma_{acyc} ,2} =\ttt $ & by (A) and def \ref{def:assertion:unfold}.\\
\ \ \ \ & (D) & $\eval {\prg{Finite}'(\prg{x})}{\M,\sigma_{acyc} ,k}  =\ttt  $ & for all $k\geq 2$, by (A) and def \ref{def:assertion:unfold}.\\
 & (E) & $\M,\sigma_{acyc} \models \prg{Finite}(\prg{x})  $,& by (D) and def \ref{def:assertion:unfold}.\\
& ... & ...
\end{tabular}

\end{example}
  
\begin{appendix}

\section{Appendix}

\subsection{Continuing the example}
\begin{example}[continued]
The rest needs to be revisited and adapted.

\vspace{.04in}
In the cyclic configuration, as we saw earlier, $\interp{\prg{acyclic(z)}}{\sigma_{cyc}}$ is undefined. Nevertheless, we  $\M, \sigma_{cyc} \models_k \neg \prg{Finite}({z})$ for all $k$.
  We also have that $\M, \sigma_{cyc} \models_k \neg \prg{length}(z)=3$, even though  \prg{length} and also $\prg{length}(z)=3$ are undefined
 in the context of $\sigma_{cyc}$.
This gives us   also that
(*) \ \ $\M, \sigma \models \neg \prg{Finite}({z})$, and also $\M, \sigma_{cyc} \models_k \neg \prg{length}(z)=3$.
\end{example}

\begin{example}[Cntrdr]

 We have $\M,\sigma \not\models  \prg{Cntrdct}(\prg{z})$ as well as $\M,\sigma \not\models    \neg\prg{Cntrdct}(\prg{z})$.
 \end{example}


\begin{observation}[Monotonicity]

I think now we have that  $\M, \sigma \models_k  A$ implies $\M, \sigma \not\models_{k+j}  A$.
\\
To think more.

\end{observation}

\begin{observation}[Lack of Modus Ponens]

We do not have modus ponens. A counterexample appears in (*).


\end{observation}


\begin{observation}[Distributivity of Disjunction]

In our system, disjunction distributes, ie we have

\begin{mylemma}[Disjunction is distributive] For all $A$.

\begin{itemize}
\item
 $\M, \sigma \models_k  A \vee A' $ implies $\M, \sigma \models_k  A $ or  $\M, \sigma \models_k A' $ for all $k$.
 \item
 $\M, \sigma \models  A \vee A' $ implies $\M, \sigma \models   A $ or  $\M, \sigma \models  A' $
\end{itemize}
 \end{mylemma}

TODO check!

\end{observation}

\begin{conjecture}[Consistency]
For all $\M$, $A$, $A'$ and $\sigma$.

\begin{itemize}
\item
 $\M, \sigma \models  A$  ana  $\M, \sigma \models   A $ never holds.
 \item
 $\M, \sigma \models  A \rightarrow A' $ and  $\M, \sigma \models  A$ entails  $\M, \sigma \models  A'$
 \item
  $\M, \sigma \models  \neg A \vee A' $ entails  $\M, \sigma \models   A \rightarrow A' $
  \item
 $\M, \sigma \models  A \vee A' $ implies $\M, \sigma \models   A $ or  $\M, \sigma \models  A' $
\end{itemize}

\end{conjecture}

TODO check!

%\begin{proof} This is a sketch

%\begin{itemize}
%\item
% We will show that for all $k$ it is impossible to have $\M, \sigma \models_k  A  $ and  $\M, \sigma \models_k  \neg A$.
% \item
% Expecting this to work by unfolding the definitions.
% \item
%Expecting this to work by unfolding the definitions. This property is less crucial tan the other two, but holds in intuitionistic logic.
% \item
% This property is less crucial tan the other two, but holds in intuitionistic logic.
%\end{itemize}
 %\end{proof}

\subsection{Considering holistic aspects}

In  Definition \ref{def:assetion:step:index} we have not considered the holistic aspects of assertions. We will therefore extend the definition as below

\begin{definition}
\label{def:assetion:hol:step:index}
 Given a module $\M$, a state $\sigma$, and a natural number $k$, we define the judgment $\M, \sigma \models_k A$ by cases on $A$  and $k$  as follows, where $k>0$ and arbitrary:

\begin{itemize}
\item
$ x \obeys \prg{S}\leadsto_{k+1} A'$ \ \ \  iff \ \ \
  for some $A$ from \prg{S},\ \    $A [\prg{this}\mapsto \prg{x}] \leadsto_{k} A'$.
\item
$\sigma \models_{naive} x \obeys \prg{S}$ \ \ \  is true.
\item
$\M, \sigma \models  A$\ \  iff\ \  $\exists k.  \forall j>k.  \forall A'. A \leadsto_{j} A'\ \implies \ \ \M, \sigma\models_{naive}  A'$.

\end{itemize}

 \end{definition}

More thoughts needed here. Notice that in the non-holistic part we asked something weaker, ie we had that 
$\exists k.  \forall j>k.  \exists A'. A \leadsto_{j} A'\ \wedge \ \ \M, \sigma\models_{naive}  A$.

\end{appendix}

\end{document}

