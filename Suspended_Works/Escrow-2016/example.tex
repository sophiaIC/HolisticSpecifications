\section{Escrow Exchange }
\label{section:example}

Figure~\ref{fig:DealV1} shows a first attempt to implement an escrow
exchange, also shown in previous work \cite{miller-esop2013,capeFTfJP14}.
%
%
We model both money and goods by \prg{Purses} (a resource model
proposed in E~\cite{ELang}).  The call
%
\lstinline+dst.deposit(amt, src)+
%
will either transfer \lstinline+amt+ resources from the \lstinline+src+
purse to the \lstinline+dst+ purse and return true, or do nothing and
return false.  A new, empty purse can be created at any time by asking
an existing purse to \prg{sprout} --- the new purse has a zero balance
but can then be filled via \prg{deposit}.

\begin{figure}[htb]
% do not change the lines, as hard coded in explanations
% or if you do, update explanations
\begin{lstlisting}
method deal_version1( ) {

   // make temporary money Purse
   escrowMoney = sellerMoney.sprout
   // make temporary goods Purse
   escrowGoods = buyerGoods.sprout

   res = escrowMoney.deposit(price, buyerMoney)
   if (!res) then
        // insufficient money in buyerMoney
        // or different money mints
        { return false }

   // sufficient money; same mints.
   // price transferred to escrowMoney
   res = escrowGoods.deposit(amt, sellerGoods)
   if (!res) then
        // insufficient goods in sellerGoods
        // or different goods mints
        {  // undo the goods transaction
           buyerMoney.deposit(price,escrowMoney)
           return false }

   // price in escrowMoney; amt in escrowGoods.
   // now complete the transaction
   sellerMoney.deposit(price, escrowMoney)
   buyerGoods.deposit(amt, escrowGoods)
   return true
}
\end{lstlisting}
\vspace*{-7mm}
\caption{First attempt at Escrow Exchange deal method}
\label{fig:DealV1}
\end{figure}

The goal of the escrow is to exchange \texttt{amt} goods for
\texttt{price} money, between the purses of a seller and buyer.
To make the exchange transactional, we use % will use an additional pair of
two private \textit{escrow} purses, one for on each side of the
transaction (money and goods). 
%Rather than swapping money and goods
%between buyer's and seller's purses in one go, the buyer's money and
%seller's goods are moved first into escrow purses, and then from the
%escrow purses into the final destinations.  In this way, we only
%complete the second half of the transaction when we are sure enough
%money and goods are securely in the escrow purses. If the transaction
%needs to be abandoned halfway through, we can return the buyer's money
%from the escrow purse without any reference to the seller.
%
Lines 3--6 of Figure~\ref{fig:DealV1} show how we first set up the
escrow purses, by sprouting two new purses (\prg{escrowMoney} and
\prg{escrowGoods}) from their respective input purses.  

It is
important that the escrow purses are newly created within the method,
and cannot have been manipulated or retained by the buyer or seller,
which is why the escrow asks \prg{sellerMoney} to make one, and
\prg{buyerGoods} to make the other.  The requirements of an open
system means that the escrow method cannot have the escrow purses
before the transaction, because 
the escrow cannot know the right kind of purses to create, 
and there is no central trusted authority
that could provide them.
Buyers and sellers cannot provide escrows purses directly, 
precisely because we must assume 
they don't trust each other: if they did, 
they wouldn't need to use an escrow.

Second, we attempt to escrow the buyer's money by transferring it from
the \prg{buyer}\-\prg{Money} purse into the new \prg{escrowMoney} purse ---
line 8.  If this \prg{deposit} request returns true, then the money
will have been transferred. If the deposit fails we abort the
transaction.
%
Third, we attempt to escrow the seller's goods --- line \tobym{16}, again by
depositing them into the other escrow purse.  If we are unsuccessful,
we again abort the transaction, after we have returned the escrowed
money to the buyer --- lines 21 and 22.
%
At this point (line \tobym{26}) the deal method should have sole access
to sufficient money and goods in the escrow purses.  The method
completes the transaction by transferring the escrowed money and goods
into the respective destination purses --- lines 26 and 27. Thanks to
the escrow purses, these transfers should not fail, and indeed, if
\prg{deal\_version1} is called in good faith it will carry out the transaction
correctly.  Unfortunately, we cannot assume good faith in a mutually
untrusting open system.


\paraC{The failure of \prg{deal\_version1}}
The % \prg{deal\_version1} 
method in Figure~\ref{fig:DealV1} does not behave
correctly % --- in fact, 
in an open system. %, it \textit{cannot}.
%SD the "it cannot" makes sense only if we had a spec for deal.
The critical
problems are assumptions about trust: both the code and the
specification implicitly trust the purse objects with which they interact.
%Considering both the \prg{Purse} and \prg{Escrow} specifications: what
%happens if a purse or escrow is asked to interact with an
%untrustworthy purse?  How much risk is involved: just the potentially
%untrustworthy purse? That purse plus any other purse it knows about,
%or interacts with (e.g.\ both are passed into the same method)?  Any
%purse (or indeed any object) anywhere in the system?
%
%Classical approaches to specification and verification have no notion
%of the risks involved when an object that does not meet its
%specification. All bets are off: the world ends.
%

Imagine if \prg{sellerMoney} was a malicious, untrustworthy object. At
line \tobym{4}, the \prg{sprout} call could itself return a malicious object,
which would then be stored in \prg{escrowMoney}. At line \tobym{8},
%
\lstinline+escrowMoney.deposit(price, buyerMoney)+
%
would let the malicious \prg{escrowMoney} purse steal all the
money out of \prg{buyerMoney} purse, and still return \prg{false}. As
a result, the seller would lose all their money, and receive no goods!
Even if the seller was more cautious, and themselves sprouted a
special temporary purse with a balance of exactly \prg{price} to pass
in as \prg{sellerMoney}, they would still lose all this money without
any recompense.

Perhaps there is something else we could do --- a \prg{trusted} method
on every object, say, that returns \prg{true} if the object is
trusted, and \prg{false} otherwise?
% , like tainting in Perl
% \cite{Perl-tainging} or the dataflow checks that secure web browsers
% \cite{existing-web-broweser-cites}?
  The problem, of course, is that
an object that is untrustworthy is, well, untrustworthy: we cannot
expect a \prg{trusted} method ever to return \prg{false}. 
This leads
to our definition of trust: trust is {\em hypothetical}, and in
relation to some specification of expected behaviour.

%Thus, trust is always with respect to some specification.}

% SD: do not like the beliow, because we do not ever ascertain that something is
% untrustworthy
% an untrustworthy object: a untrustworthy
% object is an object that does not obey its specification.


% \begin{center}
% *~~~~~~~~~~~~~~~~~~~~~~~~~*~~~~~~~~~~~~~~~~~~~~~~~~~*
% \end{center}
