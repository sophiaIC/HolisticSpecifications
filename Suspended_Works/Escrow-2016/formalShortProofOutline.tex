
\begin{figure*} 
\begin{lstlisting}
   true
        {  var escrowMoney := sellerMoney.sprout  }
   $\sM\pre \obeys \PS \longrightarrow$ 
           $(\ \eM \obeys\, \PS\  \wedge$  
             $CanTrade(\eM,\sM)\  \wedge$
             $\eM.\bal = 0\  \wedge\   $
             $\forall p\in\pre \GP.\  p.\bal\pre=p.\bal\ \wedge $
             $ \sM  \obeys \PS\ ) \ \ \ \ \wedge $
    $ \forall p:\pre \GP.$
       $(\, p.\bal\pre=p.\bal \vee  \MayAccess\pre(\sM,p) \, )\ \ \wedge$
    $ \forall z:\pre \Obj.$
       $(\, \MayAccess(\eM,z) \longrightarrow \MayAccess\pre(\sM,z)\, ) \  \wedge$
    $ \forall z,y:\pre \Obj.\ $
       $ (\,  \MayAccess(\,z,y\,)\  \longrightarrow\   $
            $(\, \MayAccess\pre(\, z, y\,)\  \vee$
                    $\MayAccess\pre(\, \sM, y\,) \wedge$
                    $\MayAccess\pre(\, \sM, z\,) \, ) \ \ )$
    $\HoareCSep$
    true
   \end{lstlisting}
%\vspace*{-7mm}
\caption{Hoare tuple for first step in \prg{deal}}
\label{fig:DealV3:S1}
\end{figure*}


% \paragraph{First Step} 
\kjx{We now use our Hoare Logic to prove the key steps of the escrow
  protocol, establishing mutual trust and delineating the risk. Here we have space to show just
  one-way trust between the escrow and seller in full: the remaining reasoning
  to establish mutual trust is outlined in the technical report \cite{appendix}.}
%
\kjx{Figure \ref{fig:DealV3:S1} shows the  Hoare tuple
for the first statement in method \prg{deal} (line 4 from Figure \ref{fig:DealV2})}. 
Lines 3-8 of Figure \ref{fig:DealV3:S1} describe the postcondition in case  \eM\ indeed \obeys \PS, while lines 9-17 make absolutely no assumption
about the trustworthiness, or provenance, of \eM.

\noindent
% We obtain lines 3-7 from  policy \prg{Pol\_sprout}  and application of  \ruleN{Meth-Call-1} and \ruleN{Cons-2}, while line 8 requires a framing step.

\noindent
By \prg{Pol\_sprout}  and \ruleN{Meth-Call-1} we obtain that \\
$  % (1) \\
  \HoareNLSP {A}
      {\sM \obeys  \PS}
       {\eM:=\sM.\prg{sprout}}
      { \eM  \obeys  \PS\  \wedge  ... rest ...} 
         {\true} %  {\true}
$

\noindent
By application   \ruleN{Cons-2} on the above we obtain\\
$  
 \HoareNLSP {B}
      {\true}
       {\eM:=\sM.\sprout}
      { \sM\pre \obeys\, \PS \rightarrow \ \\
      \HoareSP\SP\SP (\ \eM  \obeys \PS\  \wedge \ ... rest ...\, )}
       { \true }  $

\noindent
To obtain line 8, we apply a basic framing rule %a la Reynolds 
(\ruleN{frame-general} in \cite{appendix}) and get\\ % that \\
\SP $\Hoare {...}  {\eM:=\sM.\sprout} {\eM\pre=\eM} {...} $, \\ and
then, in conjunction with \ruleN{code-invar-2}, \ruleN{Cons-2}   we also obtain that\\ 
$\HoareNLSP {C} 
     {\true}  
     {\eM:=\sM.\sprout} 
     {\eM\pre\obeys\PS \rightarrow \eM\obeys\PS} 
     {...}
$ \\
We can then apply a conjunction rule (\ruleN{Conj} in \cite{appendix}) on \ruleN{B} and  \ruleN{C}, and obtain the postcondition as in 4-8.\\
To obtain 9-11, we will apply several of the code-agnostic rules. After all, here we cannot appeal to the specification of \sprout, as we do not
know whether \sM\ adheres to \PS. We start by application of \ruleN{meth-Call-2}, and a consequence rule (\ruleN{Cons-1} in \cite{appendix}):\\
$\HoareNLSP {D} 
     {\true}  
     {\eM:=\sM.\sprout} 
     {\true} 
     {\forall z. \ \MayAccess(\sM,z) \rightarrow \MayAccess\pre(\sM,z) }
$ \\
By applying  the fact that $\forall u,v,w,\ \MayAccess(u,v)\ \wedge\ \MayAccess(v,w) \rightarrow \MayAccess(u,w)$, and conjunction  and inference rules on \ruleN{D}, we get:\\
$\HoareNLSP {E} 
     {\neg \MayAccess(\sM,p)}  
     {\eM:=\sM.\sprout} 
     {\true} 
     {\forall z. \ \MayAccess(\sM,z) \rightarrow \neg \MayAccess(z,p) }
$ \\
By application of rule  \ruleN{code-invar-1}, we obtain:\\
$\HoareNLSP {F} 
     {\true}  
     {\eM:=\sM.\sprout} 
     {\true} 
     {\forall p. (\, p\obeys\PS \rightarrow (\forall z. \, \MayAffect(z,p.\bal) \rightarrow \MayAccess(z,p)) \, )}
$ \\
Through a combination of \ruleN{E} and \ruleN{F} and application of conjunction, and application of \ruleN{frame-meth-call}, we obtain that\\
$\HoareNLSP {G} 
      {\neg \MayAccess(\sM,p)}  
     {\eM:=\sM.\sprout} 
     {\true} 
     {p\obeys\pre \PS \rightarrow (p.\bal=p.\bal\pre) }
$ \\
Now by applying \ruleN{Cons-2} on \ruleN{F}, we obtain\\
$\HoareNLSP {H} 
      { \true}  
     {\eM:=\sM.\sprout} 
     {\true} 
     {\forall p. \, p\obeys\pre \PS \rightarrow \\
      \SP\SP\SP\SP (\ p.\bal=p.\bal\pre  \ \vee\  \MayAccess(\sM,p) \ ) }
$
\\
We now apply \ruleN{Cons-1} from \cite{appendix} to conjoin the invariant and postcondition, obtaining\\
$\HoareNLSP {I} 
      { \true}  
     {\eM:=\sM.\sprout} 
          {\forall p. \, p\obeys\pre \PS \rightarrow \\
      \SP\SP\SP\SP (\ p.\bal=p.\bal\pre  \ \vee\  \MayAccess(\sM,p) \ ) }
      {\true} $
Last, we obtain lines 11-12 from \ruleN{Meth-Call-2}. We also obtain lines 13-17 from \ruleN{Meth-Call-2}, and   \ruleN{Cons-1} from \cite{appendix}.

