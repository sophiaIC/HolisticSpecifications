%-----------------------------------------------------------------------------
%
%               Template for sigplanconf LaTeX Class
%
% Name:         sigplanconf-template.tex
%
% Purpose:      A template for sigplanconf.cls, which is a LaTeX 2e class
%               file for SIGPLAN conference proceedings.
%
% Guide:        Refer to "Author's Guide to the ACM SIGPLAN Class,"
%               sigplanconf-guide.pdf
%
% Author:       Paul C. Anagnostopoulos
%               Windfall Software
%               978 371-2316
%               paul@windfall.com
%
% Created:      15 February 2005
%
%-----------------------------------------------------------------------------
\documentclass[preprint,10pt]{sigplanconf}
% \documentclass[preprint,10pt,nocopyrightspace]{sigplanconf}
%KJX - for some reason, turning off preprint gives me an error!

% The following \documentclass options may be useful:

% preprint      Remove this option only once the paper is in final form.
% 10pt          To set in 10-point type instead of 9-point.
% 11pt          To set in 11-point type instead of 9-point.
% authoryear    To obtain author/year citation style instead of numeric.

\usepackage{graphicx}
\usepackage{amssymb} % mathtools,
\usepackage[reqno]{amsmath}
\usepackage{listings}
\usepackage{url}
\usepackage{enumitem}
\usepackage{grace}
\usepackage{balance}
\usepackage{hyperref}
\usepackage{lineno}



\special{papersize=8.5in,11in}
\setlength{\pdfpageheight}{\paperheight}
\setlength{\pdfpagewidth}{\paperwidth}

% \conferenceinfo{PLAS 2014}{6th July  2015, Prague, Czech Republic}
\copyrightyear{2015}
\copyrightdata{978-1-nnnn-nnnn-n/yy/mm}
\doi{nnnnnnn.nnnnnnn}

% Uncomment one of the following two, if you are not going for the
% traditional copyright transfer agreement.

%\exclusivelicense                % ACM gets exclusive license to publish,
                                  % you retain copyright

%\permissiontopublish             % ACM gets nonexclusive license to publish
                                  % (paid open-access papers,
                                  % short abstracts)

% \titlebanner{DRAFT:  }        % These are ignored unless
\preprintfooter{work in progress}   % 'preprint' option specified.

\authorinfo{Sophia Drossopoulou$^1$, James Noble$^{2,1}$, Toby Murray$^4$, Mark Miller$^3$}{$^1$Imperial College London, $^2$Victoria University Wellington, $^3$Google Inc, $^3$NICTA and UNSW.}{}

\usepackage{natbib}
\usepackage{bbm}
\usepackage{framed}
\usepackage{graphics}
\definecolor{shadecolor}{rgb}{1,0.8,0.3}
\usepackage{amsthm} % it conflicts with llncs
\usepackage{epsfig}
\usepackage{xspace}
%\usepackage{latexsym}
\usepackage{amssymb}
\usepackage{amsfonts}
%\usepackage{amstext}
% \usepackage{txfonts}
\usepackage{url}
%\usepackage{xypic}
%\usepackage{makeidx}  % allows for indexgeneration
\usepackage{listings} % for code
  \usepackage{multirow}
  \usepackage{qsymbols}
  \usepackage{amsmath}
% \usepackage{MnSymbol}


\newcommand{\prg}[1]{{\mbox{\tt{#1}}}}
\newcommand{\forget}[1]{}
\newcommand{\etc}{{\it etc.}}
\newcommand{\eg}{{\it e.g.\,}}
\newcommand{\ie}{{\it i.e.\,}}



 \newcommand{\ttt}{\prg{true}}
\newcommand{\ff}{\prg{false}}
\newcommand{\unkn}{\prg{b???}}
\newcommand{\bv}{\prg{bval}}


\newcommand{\prg}[1]{{\mbox{\tt{#1}}}}
 \newcommand{\prgCol}[1]{#1}

 \newcommand{\forget}[1]{}
\newcommand{\etc}{{\it etc.}}
\newcommand{\eg}{{\it e.g.\,}}
\newcommand{\ie}{{\it i.e.\,}}

\newcommand{\Future}[1]{\lozenge\, #1}% {\bullet #1}% {{{\mathcal F}}(#1)} % {{{\mathcal B}}(#1)}
\newcommand{\Using}[2]{#1\,\kw{in}\, #2} %{{{\mathcal U}}(#1,#2)}
\newcommand{\SigmaUsing}[2]{#1\@ #2} %{{{\mathcal U}}(#1,#2)}
\newcommand{\Past}[1] {\nabla #1} %{\lozenge\!\!\!\!\-\!\!-\,#1}
%{\lozenge\!\!\!\!\!\circ  #1} % {\lozenge\!\!\!\!\-\!\!- #1} %{\upupsilon #1}  %{\nabla #1} %{\circ #1}%  {{{\mathcal P}}(#1)}
\newcommand{\Initial}[1] {{{\mathcal I}\!nitial}(#1)}

\newcommand{\Pol}[1] {{\ensuremath{\prg{Pol}\_{\prg{#1}}}}}
%\newcommand{\MOne} {{\ensuremath{\prg{M}_{\prg{BA}}}}}
%\newcommand{\MTwo} {{\ensuremath{\prg{M}_{\prg{BA}'}}}}

\newcommand{\strongImplies}{\leqq} %{{ \,^\sqsubset\!\!\!_{\sim}\, }}
\newcommand{\weakImplies}{\lessapprox} %{{ \,^\sqsubset\!\!\!_{\sim}\, }}
\newcommand{\frames}{~\kw{frames}~}

\newcommand{\appref}[1]{see App.~\ref{#1}}

%\newcommand{\sE}{{\prg{e}}}

\newcommand{\LangOO} {\ensuremath{{\mathcal L}ang{_{\tt oo}}}}

% ------------------------------------------------------------------
%                                             positions, separations
\newcommand{\cf}{{\it c.f.~}}
%\newcommand{\HYPHENA}{{\em-- }}
%\newcommand{\HYPHENB}{{\em-- }}
%\newcommand{\SP}{{\hspace{.1in}}}
%\newcommand{\s}{{\hspace{.01in}}}

%\newcommand{\obeys}{\,\textbf{\textrm{obeys}}\,}
%\newcommand{\StrongDom}{\ensuremath{\mathcal{S}\textrm{\textit{trong}}{\mathcal{D}}\textrm{\textit{om}}}}
%\newcommand{\Dom}{\ensuremath{\mathcal{D}}\textrm{\textit{om}}}

\newcommand{\Changes}[1]{\ensuremath{\mathcal{C}\textrm{\textit{hanges}}(#1)}}
\newcommand{\VisibleLit}{\ensuremath{\mathcal{V}\textrm{\textit{isible}}}}

\newcommand{\Gives}{\ensuremath{\mathcal{G}\textrm{\textit{ives}}}}
\newcommand{\MayCall}{\ensuremath{\mathcal{M}\textrm{\textit{ay}}{\mathcal{C}}\textrm{\textit{all}}}}
%\newcommand{\Dom}{\ensuremath{\mathcal{D}\textrm{\textit{om}}}}
\newcommand{\MayRead}{\ensuremath{\mathcal{M}\textrm{\textit{ay}}{\mathcal{R}}\textrm{\textit{ead}}}}
\newcommand{\MayAccess}{\ensuremath{\mathcal{M}\textrm{\textit{ay}}{\mathcal{A}}\textrm{\textit{ccess}}}}
\newcommand{\CanAccess}[2]{\ensuremath{{\mathcal{A}}\textrm{\textit{ccess}}}(#1,#2)}
\newcommand{\Calls}[1]{\ensuremath{{\mathcal{C}}\textrm{\textit{alls}}}(\prg{#1})}
\newcommand{\Caller}{\ensuremath{{\mathcal{C}}\textrm{\textit{aller}}}}
%{\ensuremath{\mathcal{C}\textrm{\textit{an}}{\mathcal{A}}\textrm{\textit{ccess}}}(#1,#2)}
\newcommand{\WillAccessThrough}{\ensuremath{\mathcal{W}\textrm{\textit{ill}}{\mathcal{A}}\textrm{\textit{ccess}}{\mathcal{T}}\!\!\textrm{\textit{hrough}}}}
\newcommand{\modelsWithO}{\models\!\!\!\!{_{_{_{\tiny{\mathcal O}}}}}}
\newcommand{\A}{\ensuremath{A}}
\newcommand{\SA}{\ensuremath{{^{\small{\prg{s}}}\! A}}}
\newcommand{\SE}{\ensuremath{{^{\small{\prg{s}}}\!e}}}
\newcommand{\SEOne}{\ensuremath{{^{\small{\prg{s}}}\!e}}}
\newcommand{\SETwo}{\ensuremath{{^{\small{\prg{s}}}\!e'}}}
\newcommand{\B}{\ensuremath{B}}
\newcommand{\Arising}{{\mathcal{A}}\textrm{\textit{rising}}}

 %------------------------ syntax tables

\newcommand{\syntax}[1]{\prg{{\it #1}}}
\newcommand{\BBC}{$::=$} %in syntactic definitions
\newcommand{\SOR}{\ensuremath{\ \mid\ }} % BNF or
\newcommand{\MID}{{\SPsmall ~ \mid ~ \SPsmall }} % in sets


\newcommand{\pre}{\ensuremath{_{{pre}}}}   %kjx no \sc  in math mode
\newcommand{\post}{\ensuremath{_{{post}}}} %kjx no \sc  in math mode
\newcommand{\PRE}{\pre}
\newcommand{\POST}{\post}

%\newcommand{\eval}[2]{{\ensuremath{\langle{ {#1}}\rangle_{#2}}}}
\newcommand{\interp}[2]{{\ensuremath{\lfloor{ {#1}}\rfloor_{#2}}}}
%\newcommand{\interpBL}[1]{{\lceil   {#1}  \rfloor}}
%  { \langle \!\langle { {#1} \rangle \!\rangle}\! }
% { \langle   { {#1}  \rangle} }
% ------------------------------------------------------------------
%                                             English abbreviations

% ------------------------------------------------------------------
%                                              keywords, program text
\newcommand{\kw}[1]{{\bf{\sf {#1}}}}
%{\mbox{\prgCol{\rm{\bf {#1}}}}}
\newcommand{\lit}[1]{{\prg {#1}\xspace}}
\newcommand{\com}{\ensuremath{\prg{//}}}
%\newcommand{\cnf}{\ensuremath{\kappa}}

 \newcommand{\code}[1]{{\mbox{\tt{#1}}}}
\newcommand{\M}{\ensuremath{\prg{M}}}
%\newcommand{\C}{\ensuremath{\prg{C}}}

%\newcommand{\ext}{\mbox{\,\,{\kw {extends}}\,\,}}
%\newcommand{\extends}{\mbox{\,\,{\kw {extends}}\,\,}}
%\newcommand{\ass}{\mbox{{\kw {:=}}\,}}
\newcommand{\semi}{\mbox{{\kw {;}}\ }}
%\newcommand{\comma}{\mbox{{\kw {,}}\,}}
%\newcommand{\dotK}{\mbox{{\kw {.}}}}
%\newcommand{\class}{\mbox{{\kw {class}}\,\,}}
%\newcommand{\stat}{\mbox{{\kw {state}}}}
%\newcommand{\trans}{\mbox{{\kw {root}}}}
%            % {\mbox{{\kw {root state}}}}
%            % was {\mbox{{\kw {abs-state}}}}
\newcommand{\lb}{\prgCol{\mbox{\tt{\bf{\{ }}}}}
\newcommand{\rb}{\prgCol{\mbox{\tt{\bf{\} }}}}}
\newcommand{\lp}{\prgCol{\mbox{\tt{\bf{( }}}}}
\newcommand{\rp}{\prgCol{\mbox{\tt{\bf{) }}}}}
 





\newcommand{\assertTC}[2]{{\M{#1} \typCol{\vdash} {\prg{#2}}\s
    \typCol{\DDiamond_{r} }}}
\newcommand{\assertSC}[2]{{\M{#1} \typCol{\vdash} {\prg{#2}}\s
    \typCol{\DDiamond_{s}} }}




% \newcommand{\M}[1]  {{\ensuremath{\prg{M}{{\prg{#1}}}}}}
    % {\prg{P}}
%\newcommand{\Env}[1]{\envCol{\ensuremath{\Gamma{#1}}}}
% \newcommand{\state}[1]{\stCol{\ensuremath{\sigma???{#1}}}}
%\newcommand{\stackFrame}[1]{\stCol{\ensuremath{\phi{#1}}}}
%\newcommand{\heap}[1]{\stCol{\ensuremath{\chi{#1}}}}

%\newcommand{\expr}[1]{{\ensuremath{\prg{e{#1}}}}}
%\newcommand{\fld}[1]{{\ensuremath{\prg{f{#1}}}}}
%\newcommand{\param}{{\ensuremath{\prg{x}}}}
%
%\newcommand{\clss}[1]{\ensuremath{\prg{c}{\prg{#1}}}}
%\newcommand{\clssD}[1]{\ensuremath{\prg{d}{{#1}}}}
%\newcommand{\type}[1]{\ensuremath{\prg{t{#1}}}}
%
%\newcommand{\val}[1]{{\ensuremath{\prg{v}{\prgCol{#1}}}}}
%\newcommand{\res}[1]{{\ensuremath{\prg{dv}{#1}}}}
%\newcommand{\valOrDev}[1]{{\ensuremath{\prg{r}{#1}}}}


%\newcommand{\expandexp}[1]{}
%
%\newcommand{\oo}{object-oriented}
%\newcommand{\mExtS}{\ensuremath{\Downarrow}}
%
%% re-classification expression
%\newcommand{\cm}[1]{\this{\prgCol{\ensuremath{\mExtS}}}\prg{#1}}






% ------------------------------------------------------------------
%                                             identifiers in the examples
%                     ---------------------
%                                      Stack
 %                     ---------------------
%                                   Employee
\newcommand{\Empl}{\prg {Empl}}
\newcommand{\Pers}{\prg {Pers}}
\newcommand{\E}{{_\prg {e}}}
\newcommand{\Man}{\prg {Boss}}
\newcommand{\Stud}{\prg {Stdt}}
\newcommand{\Scholar}{\prg {Scholar}}
\newcommand{\sal}{\prg {sal}}
\newcommand{\bYear}{\prg {bYear}}
\newcommand{\frnd}{\prg {frnd}}
% \newcommand{\marks}{\prg {marks}}
\newcommand{\pay}{\prg {fee}}
\newcommand{\setP}{\prg {set}}
\newcommand{\String}{\prg {String}}
\newcommand{\hobby}{\prg {hobby}}
\newcommand{\intg}{\kw {int}}
\newcommand{\boolg}{\kw {bool}}
\newcommand{\ii}{\prg {i}}
\newcommand{\prom}{\prg {promote}}
\newcommand{\mkStud}{\prg {study}}
\newcommand{\dem}{\prg {demote}}
\newcommand{\grad}{\prg {employ}}
\newcommand{\assist}{\prg {assist}}
\newcommand{\amount}{\prg {amount}}

\newcommand{\Phil}{\prg {Phil}}
\newcommand{\Book}{\prg {Book}}
\newcommand{\Person}{\prg {Person}}
\newcommand{\fav}{\prg {favourite}}

\newcommand{\aMan}{\prg {billy}}
\newcommand{\anotherMan}{\prg {bob}}
\newcommand{\aMutMan}{\prg {bea}}
\newcommand{\employees}{employees}
\newcommand{\aStud}{\prg {steve}}
\newcommand{\aPers}{\prg {peter}}
\newcommand{\aStudEmpl}{\prg {mary}}
\newcommand{\anoStudEmpl}{\prg {judy}}
\newcommand{\yetAnoStudEmpl}{\prg {betty}}
\newcommand{\yetYetAnoStudEmpl}{\prg {joe}}

\newcommand{\exprog}{_{\prg{ei}}}



% structuring macros
\newcommand{\EndDefLemma}{\noindent $\bigtriangleup$}



\newcommand{\para}[1]{{\ensuremath{\prg{x}{#1}}}}


\newcommand{\methAndBodyNew}{\ensuremath{
\prg{\type{}~m \lp \type{_1}~\param{}\rp\   \ \lb
~\expr{}~\rb} }}
\newcommand{\methAndBodyNewOne}{\ensuremath{
\prg{\type{}~m \lp \type{_1}~\param{}\rp\   \ \lb
~\expr{'}~\rb} }}
\newcommand{\methAndBodyNewTwo}{\ensuremath{
\prg{\type{}~m \lp \type{_1}~\param{}\rp\   \ \lb
~\expr{''}~\rb} }}
\newcommand{\methAndBodyNewThree}{\ensuremath{
\prg{\type{''}~m \lp \type{_1''}~\param{}\rp\   \ \lb
~\expr{''}~\rb} }}
\newcommand{\methAndBodyNewPrime}{\ensuremath{
\prg{\type{}~m \lp  \type{_1}~\para{} \rp\   \
\lb~\expr{'}~\rb} }}
\newcommand{\methAndBodyNewPrimeAll}{\ensuremath{
\prg{\type{'}~m \lp  \type{_1'}~\para{} \rp\   \
\lb~\expr{'}~\rb} }}
\newcommand{\methAndBodyNewP}{\ensuremath{ % slight diff in Primes from above
\prg{\type{}~m \lp  \type{_1}~\para{} \rp\   \
\lb~\expr{'}~\rb} }}
\newcommand{\methAndBodyFromOneSig}{\ensuremath{
\prg{\type{}~m \lp  \type{_1}~\para{} \rp\ \eff{} \lb~ ... ~\rb}
}}

%-----------------

\newcommand{\Body}[2]{\ensuremath{\mathcal{B}ody(#1,\prg{#2})}}

\newcommand{\T}[1]{{\ensuremath{\type{}{#1}}}}

% find the declaration of an identifier
\newcommand{\LookUp}[2]{\ensuremath{ {#1}({\prg{#2}}) }}
\newcommand{\LookUpEnv}[2]{{ \Env{#1}({\prg{#2}}) }}
\newcommand{\LookUpState}[2]{{ \ensuremath{\sigma{#1}}({\prg{#2}}) }}
\newcommand{\LookUpHeap}[2]{{ \ensuremath{\heap{#1}}({\prg{#2}}) }}
\newcommand{\LookUpBody}[2]{{ {\M{#1}}({\prg{#2}}) }}
%\newcommand{\LookUpClass}[2]
%        { \ensuremath{{\mathcal C}%{\mathcal D}%{\it ef}
%        (}\M{#1},\prg{#2}\ensuremath{)}}
\newcommand{\LookUpField}[3]
        {\ensuremath{{\mathcal F}{\mathcal D}% {\it f}
                (}\M{#1},\prg{#2},\prg{#3}\ensuremath{)}}
\newcommand{\LookUpMethod}[3]
        {\ensuremath{{\mathcal M}{\mathcal D}% {\it f}
         (}\M{#1},\prg{#2},\prg{#3}\ensuremath{)}}
\newcommand{\Undef}{{\ensuremath{\mathcal U\!}{\it df}}}
\newcommand{\ClassOf}[2] { \ensuremath{{\mathcal C}{\mathit{lass}}(#1)_{#2}}}

%-------------------Part Three: Types ...
% Notation for types ( open and close types).


\newcommand{\subclass}{\typCol{\sqsubseteq}}
\newcommand{\sclass}{\subclass}
\newcommand{\widen}{\typCol{\ensuremath {\le}}}

%-------------------- implies, and, or, iff, etc -----------------
\newcommand{\inset}[3]{\prg{#1}\!\in\!\prg{#2},\ldots,\prg{#3}}
% do not change above, it is also used with \forall, \exists
%\newcommand{\IMPLIES}{{\SP \Longrightarrow \SP}}
 \newcommand{\AND}{{\SPsmall {\mbox{and}} \SPsmall}}
\newcommand{\WITH}{{\SPsmall {\mbox{with}} \SPsmall}}

%\newcommand{\IFF}{{\SPsmall {\mbox{iff}} \SPsmall}}
\newcommand{\IFF}{{\SP {\mbox{ iff }} \SP}}

\newcommand{\OR}{{\SPsmall {\mbox{or}} \SPsmall}}
\renewcommand{\implies}{{\ensuremath{\longrightarrow}}}
\newcommand{\upd}{{\mapsto}}

%----------------------- Conformance --------------------
\newcommand{\stateConf}[3]
  {\M{#1},\Env{#2}
  \rtjCol{\ensuremath{\vdash}}\,\ensuremath{\state{#3}}\,\rtjCol{\ensuremath{\DDiamond}}  }
\newcommand{\stackFrameHeapConf}[4]
  {\M{#1},\Env{#2}
  \rtjCol{\ensuremath{\vdash}}\,\ensuremath{\stackFrame{#3},\heap{#4}}\,\rtjCol{\ensuremath{\DDiamond}}  }
\newcommand{\stackFrameHeapConfS}[4]
  {\M{#1},\Env{#2}
  \rtjCol{\ensuremath{\vdash}}\,\ensuremath{ {#3},\heap{#4}}\,\rtjCol{\ensuremath{\DDiamond}}  }

\newcommand{\stackFrameHeapConfNOT}[4]
  {\M{#1},\Env{#2}
  \rtjCol{\ensuremath{\not\vdash}}\,\ensuremath{\stackFrame{#3},\heap{#4}}\,\rtjCol{\ensuremath{\DDiamond}}  }

\newcommand{\stackFrameHeapConfLong}[4]
  {\M{#1}, {#2}
  \rtjCol{\ensuremath{\vdash}}\,\ensuremath{\stackFrame{#3},\heap{#4}}\,\rtjCol{\ensuremath{\DDiamond}}  }
\newcommand{\stackFrameHeapConfLongNOT}[4]
  {\M{#1}, {#2}
  \rtjCol{\ensuremath{\not\vdash}}\,\ensuremath{\stackFrame{#3},\heap{#4}}\,\rtjCol{\ensuremath{\DDiamond}}  }


\newcommand{\stateConfEnv}[3]
  {\M{#1}, {#2}
  \rtjCol{\ensuremath{\vdash}}\,\ensuremath{\state{#3}}\,\rtjCol{\ensuremath{\DDiamond}}  }
  \newcommand{\stateConfLong}[3]
  { {#1}, {#2}
        \rtjCol{\ensuremath{\vdash}}\,\ensuremath{\state{#3}}\,\rtjCol{\ensuremath{\DDiamond} } }
\newcommand{\stateConfLongNot}[3]
  { {#1}, {#2}
        \rtjCol{\ensuremath{\vdash}\!\!\!\not}\ \ensuremath{\state{#3}}\,
         \rtjCol{\ensuremath{\DDiamond} } }
\newcommand{\conf}[4]
  {\M{#1},\ensuremath{\heap{#2}}
        \rtjCol{\ensuremath{\vdash}} \ensuremath{\prg{#3}} \rtjCol{\ensuremath{\lhd}}
        \ensuremath{\prg{#4}}}
\newcommand{\confW}[4]
  {\M{#1},\ensuremath{\heap{#2}}
        \rtjCol{\ensuremath{\vdash}} \ensuremath{\prg{#3}} \rtjCol{<:}
        \ensuremath{\prg{#4}}}
% states conforming to each other:


\newcommand{\confState}[5]
    % Prog, eff, thsiMut, state, state
  {\confStateL{\M{#1}}{\eff{#2}}{\clss{#3}}{\state{#4}}{\state{#5}}}
  \newcommand{\confStateP}[4]
    % Prog, eff, state, state
  {\ensuremath{\M{#1},{\eff{#2}}\vdash{\state{#3}}\lhd{\state{#4}}}}
\newcommand{\confStateL}[5]
  {\ensuremath{#1,{#2},{#3}\vdash{#4}\lhd{#5}}}
\newcommand{\confStateLP}[5]
  {\ensuremath{#1,{#2}\vdash{#3}\lhd{#4}}}




% ------------------------------------------- Type assertions ----------------
\newcommand{\assert}[3]{\prg{#1} \vdash \prg{#2}~:~\prg{#3}}
% \newcommand{\assertT}[5] {\M{#1}, \Env{#2} \   \ensuremath{\vdash} \ \prg{#4}~
% \ensuremath{:~\prg{#5}}}
\newcommand{\EFFSEP}{\typCol{[\!]}}

\newcommand{\assertTAll}[6]
    % program, env, expr, type, mut, eff
{\M{#1}, \Env{#2} \   \typCol{\ensuremath{\vdash}} \ \prg{#3}~
\ensuremath{\typCol{:~} \prg{#4} } }% ~\EFFSEP~ {#5}~\EFFSEP~{#6} }}
\newcommand{\assertTAllS}[6]
% with subsumption
    % program, env, expr, type, mut, eff
{\M{#1}, \Env{#2} \   \typCol{\ensuremath{\vdash_s}} \
\prg{#3}~
\ensuremath{\typCol{:~} \prg{#4} } }% ~\EFFSEP~ {#5}~\EFFSEP~{#6} }}
\newcommand{\assertTAllR}[6]
% with subsumption
    % program, env, expr, type, mut, eff
{\M{#1}, \Env{#2} \   \typCol{\ensuremath{\vdash_r}} \
\prg{#3}~
\ensuremath{\typCol{:~} \prg{#4} } }% ~\EFFSEP~ {#5}~\EFFSEP~{#6} }}
\newcommand{\assertTAllL}[6]
    % program, env, expr, type, mut, eff
{{#1}, {#2} \   \typCol{\ensuremath{\vdash}} \ \prg{#3}~
\ensuremath{\typCol{:~}} \prg{#4} }%~\EFFSEP~ {#5}~\EFFSEP~{#6} }}
\newcommand{\assertTFour}[4]
    % for the explanations, ignore P and \eff
{ {#1} \   \typCol{\ensuremath{\vdash}} \ \prg{#2}~
\ensuremath{\typCol{:~} \prg{#3} ~\EFFSEP~ {#4}  }}
% same as above, but in 2 lines!
% do not remove! it is difficult to program, so
% leave it in, even if temporarily not used!
\newcommand{\assertTAllTwoL}[6]
    % program, env, expr, type, mut, eff
{\M{#1}, \Env{#2} \   \ensuremath{\vdash} \ \prg{#3}~
\ensuremath{:~\\ \SP\SP\SP\SP\SP
   \prg{#4} }}
\newcommand{\assertTAllEnvS}[6]
    %as before, but the environment is given in full
{\M{#1}, {#2} \   \typCol{\ensuremath{\vdash}_s} \ \prg{#3}~
\ensuremath{\typCol{:~}}\prg{#4}  }
\newcommand{\assertTAllEnvR}[6]
    %as before, but the environment is given in full
{\M{#1}, {#2} \   \typCol{\ensuremath{\vdash}_r} \ \prg{#3}~
\ensuremath{\typCol{:~}}\prg{#4}  }
\newcommand{\assertTAllEnv}[6]
    %as before, but the environment is given in full
{\M{#1}, {#2} \   \typCol{\ensuremath{\vdash}} \ \prg{#3}~
\ensuremath{\typCol{:~}}\prg{#4}  }




 % Effects


% receiver mutations

\newcommand{\mutOr}[1] % takes as parameter the program
        {\ensuremath{\sqcup\!{_{{#1}}}}} % binary on muts

\newcommand{\assertc}[3]{\ensuremath{\M{#1}
 \typCol{\vdash}\, \prg{#2}\, \subclass\, \prg{#3}}}

 \newcommand{\assertw}[3]{\ensuremath{\M{#1} \typCol{\vdash}
\prg{#2}\widen\prg{#3}}}
\newcommand{\assertion}[3]{\ensuremath{#1 \typCol{\vdash} \prg{#2}~#3}}
\newcommand{\notStateConfEnv}[3]
  {\M{#1},\prg{#2}\,
  \rtjCol{\ensuremath{\not\vdash}}\,\ensuremath{\state{#3}}\,\rtjCol{\ensuremath{\DDiamond}}  }
 \newcommand{\notConf}[4]
  {\M{#1},\ensuremath{\heap{#2}}
        \rtjCol{\ensuremath{\not\vdash}} \ensuremath{\prg{#3}} \rtjCol{\ensuremath{\lhd}}
        \ensuremath{\prg{#4}}}







%Macros for inference rules
\newcommand{\inferencerule}[2]{
\begin{array}{l} #1 \\ \hline #2 \end{array}
}

\newcommand{\inferenceruleN}[3]
{
\begin{array}{l}
% \SP\SP\SP\SP\SP\SP\SP\SP
% \SP\SP\SP\SP\SP\SP\SP\SP
\SP\SP\SP\SP\SP\SP\SP\SP
\SP\SP\SP\SP\SP\SP  {\sf #1}
\\ #2  \\ \hline   #3
  \end{array}
}

\newcommand{\inferenceruleNN}[3]
{
\begin{array}{l}
\SP\SP\SP\SP\SP\SP\SP\SP
\SP\SP\SP\SP\SP\SP\SP\SP
\SP\SP\SP\SP\SP\SP\SP\SP
\SP\SP\SP\SP\SP\SP\SP\SP

   {\sf #1}
\\ #2  \\ \hline   #3
  \end{array}
}

%===========================================================================
%  Definition-Lemma-Theorem-Proof
%
% Adaptation of LaTeX's theorem environment; can be used as a command
% (eg just \Lemma not \begin{Lemma}) and no italicisation; also works
% with ptmac; result numbering is uniform within subsections and can be
% suppressed.
%
\newif\ifNumberResults\NumberResultstrue
\def\@@opargbegintheorem#1#2#3{\@@@@begintheorem{\bf\@@thmname{#1}{#2}(#3)}}
\def\@@begintheorem#1#2{\@@@@begintheorem{\bf\@@thmname{#1}{#2}}}
\def\@@@@begintheorem#1{\par\removelastskip\smallskip\noindent{#1}}
\def\@@thmname#1#2{#1\ \ifNumberResults#2\ \fi}

% similarly \Proof or \begin{Proof}...\end{Proof}
% prefer proofs with statements if possible - hence \penalty700
%\let\qedsymbol\S% make it \square or \blacksquare if you like for kb
\let\qedsymbol \Box
\def\qed{\hfill{$\qedsymbol$}}
\def\Proof{\par\removelastskip\smallskip\penalty700\noindent{\bf Proof}\enskip}
\def\endProof{\qed\penalty-700 \smallskip}
\let\endproof\endProof

%   The actual words

%\newtheorem{theo}{Theorem} this is necessay if we want eg \newtheorem{definition}[theo]{Definition} to work 
\newtheorem{definition}{Definition} %  \newtheorem{definition}[theo]{Definition}
\newtheorem{example}{Example} %  \newtheorem{example}[theo]{Example}
%\newtheorem{lemma}{Lemma} % \newtheorem{mylemma}[theo]{Lemma}
%\renewtheorem{mylemma}{Lemma} % \newtheorem{mylemma}[theo]{Lemma}
\newtheorem{conjecture}{Conjecture} %\newtheorem{conjecture}[theo]{Conjecture}
\newtheorem{theorem}{Theorem} %\newtheorem{theo}{Theorem}
\newtheorem{note}{Note} % \newtheorem{note}[theo]{Note}
\newtheorem{observation}{Observation} %  \newtheorem{observation}[theo]{Observation}


%--------------------------------- the ones that Susan introduced
\newcommand{\z}{{\prg z}}

\newcommand{\Fields}[3]{\ensuremath{{\mathcal F}(}\\Mg{#1},\prg{#2},
\prg{#3}\ensuremath{)} }
\newcommand{\FieldIds}[2]{\ensuremath{{\mathcal F}{\it {s}}(\M{#1},\prg{#2})}}
\newcommand{\Meths}[3]{\ensuremath{{\mathcal M}(}\M{#1},\prg{#2},
\prg{#3}\ensuremath{)} }







 


\newcommand{\rewriteLong}[1]
{\rtCol{\ensuremath{\ \leadsto\!\!\!\!\!\!_{_{#1}\,\ \ }}}}
\newcommand{\rewrite}[1]
{\rtCol{\ensuremath{\ \leadsto\!\!\!\!\!\!_{_{\M{#1}}\,\ \ }}}}
\newcommand{\rewriteP}
    {{\ensuremath{\ \ \leadsto\!\!_{_{\prg{P}}}\,\,\,}}}
 % {\rtCol{\ensuremath{\ \ \leadsto \!\!\!\!_{{\M{}}\,\ \ }}}}
% \newcommand{\Conf}[2]% configurations: < #1, #2 >
% inside the mathmode
% { \langle \prg{#1}, {#2} \rangle }
% {  \prg{#1}\ensuremath{,}{#2}  }
%\newcommand{\greenComma}{\rtCol{\ensuremath{,}}}
%\newcommand{\ConfL}[3]
%{  \prg{#1}\greenComma\stackFrame{#2}\greenComma\heap{#3}  }
%\newcommand{\ConfLLong}[3]
%{  \prg{#1}\greenComma {#2}\greenComma\heap{#3}  }
%\newcommand{\ConfR}[2]
%{  \prg{#1}\greenComma\heap{#2}  }
%

% ------------------------------------- Well formed, unique acyclic ---------------
 
%{\ensuremath{\M{#1} \vdash {\prg{#2}}\s \DDiamond_{\it {eff}}  }}

  \newcommand{\nullPEC}{\lit {nullPntrExc}}



%\newcommand{\Obj}{\prg{Object}}
%\newcommand{\void}{\kw{void}}
%\newcommand{\newK}{{\kw {new}}~} % {{\kw {new}}}% no ~ around it
%\newcommand{\newKW}{~{\kw {new}}~}
%




\newcommand{\back}{{$\!\!\!\!\!\!\!$}}


\newcommand{\WideFig}[3]
{
\begin{figure*}[t]
\begin{center}
\noindent
\fbox{
\begin{minipage}{4.7 in}
{#1} % the contents
\end{minipage}
}
\caption{#2}
\label{#3}
\end{center}
\end{figure*}
}


\newcommand{\WideFigWhere}[4] % you can specify where it should appear!
{
\begin{figure*}[{#4}]
\begin{center}
\noindent
\fbox{
\begin{minipage}{5. in}
{#1} % the contents
\end{minipage}
}
\caption{#2}
\label{#3}
\end{center}
\end{figure*}
}

\newcommand{\BigWideFigWhere}[4] % you can specify where it should appear!
{
\begin{figure*}[{#4}]
\begin{center}
\noindent
{\normalsize
\hrule
\begin{minipage}{5. in}
{#1} % the contents
\end{minipage}
\hrule
}
\caption{#2}
\label{#3}
\end{center}
\end{figure*}
}

\newcommand{\NotTooWideFigWhere}[4] % you can specify where it should appear!
{
\begin{figure*}[{#4}]
\begin{center}
\noindent
\fbox{
\begin{minipage}{4.3 in}
{#1} % the contents
\end{minipage}
}
\caption{#2}
\label{#3}
\end{center}
\end{figure*}
}


\newcommand{\opsemExprFig}
{\BigWideFigWhere {\opsemExpr} {Execution of expressions\MD}
{opsemTrad} {htbp} }



\newcommand{\mlc}{ }%{\heartsuit}
%\newcommand{\mcl}{ }%{\heartsuit}
\newcommand{\mc}{ }%{\heartsuit}

\newcommand{\BigNotTooWideFigWhere}[4] % you can specify where it should appear!
{
\begin{figure*}[{#4}]
\begin{center}
\noindent
{\normalsize
\hrule
\begin{minipage}{4.3 in}
{#1} % the contents
\end{minipage}
\hrule
}
\caption{#2}
\label{#3}
\end{center}
\end{figure*}
}

 

%]})
%}



\usepackage{times}
 \usepackage{latexsym}
\usepackage{listings}
\definecolor{dkgreen}{rgb}{0,0.6,0}
\definecolor{gray}{rgb}{0.5,0.5,0.5}
\definecolor{mauve}{rgb}{0.58,0,0.82}


\lstset{ %
  language=Java,                % the language of the code
  mathescape=true,
  basicstyle=\footnotesize\tt,           % the size of the fonts that are used for the code
  numbers=left,                   % where to put the line-numbers
  numberstyle=\tiny\color{dkgreen},  % the style that is used for the line-numbers
  stepnumber=1,                   % the step between two line-numbers. If it's 1, each line
                                  % will be numbered
  numbersep=5pt,                  % how far the line-numbers are from the code
  backgroundcolor=\color{white},      % choose the background color. You must add \usepackage{color}
  showspaces=false,               % show spaces adding particular underscores
  showstringspaces=false,         % underline spaces within strings
  showtabs=false,                 % show tabs within strings adding particular underscores
  frame=single,                   % adds a frame around the code
  rulecolor=\color{black},        % if not set, the frame-color may be changed on line-breaks within not-black text (e.g. commens (green here))
  tabsize=2,                      % sets default tabsize to 2 spaces
  captionpos=b,                   % sets the caption-position to bottom
  breaklines=true,                % sets automatic line breaking
  breakatwhitespace=false,        % sets if automatic breaks should only happen at whitespace
  title=\lstname,                   % show the filename of files included with \lstinputlisting;
                                  % also try caption instead of title
  keywordstyle=\color{blue},          % keyword style
  commentstyle=\color{gray},       % comment style
  stringstyle=\color{mauve},         % string literal style
  escapeinside={\%*}{*)},            % if you want to add LaTeX within your code
  morekeywords={module,obeys,satisfies,field,fields,predicate,private,public,final,this,throw,new,||,to,method,def,any,specification,policies,policy,then,where}               % if you want to add more keywords to the set
}




%  {\footnote{{\color{red}{Sophia removed the following:\ }}{#1}{\color{red}{Sophia's reason:\ }}{#2}}}


\usepackage{enumitem}
\setlist{nolistsep}

\newcommand{\definitionautorefname}{Definition}
\renewcommand{\theoremautorefname}{Theorem}
\newcommand{\lemmaautorefname}{Lemma}

\begin{document}





 \title{More Reasoning about Risk and Trust  in an Open Word\\
(Appendix)}

\maketitle



% \begin{abstract}
%   This is the appendix
% \end{abstract}

% \category{D.3.1}{Programming Languages}{Formal Definitions and Theory}
% \category{D.4.6}{Security and Protection}{Verification}
% \category{K.4.4}{Electronic Commerce}{Payment Schemes, Security}


\section{Introduction}

This is the companion appendix to our work \textit{``Reasoning about
  Risk and Trust in an Open World''}. We give here the full
definitions of \LangOO, \Chainmail, our Hoare logic, prove
soundness of our Hoare logic, and then prove that our escrow exchange
implementation establishes mutual trust while managing risk.

\footnote{Explain that we have made changes to model since  previous version.  Perhaps outline them? Perhaps
have new title.}

\paragraph{Status} There are lots of rough  edges. But what we want to concentrate on at the moment is the introduction of the oracles, $\O$, in order to give meaning to the $\obeys$ predicate, and to express the soundness of the Hoare logic.

On the other hand, some other (smaller) things are inconsistent, and will be updated once we are certain about the circularity issue. These include
\begin{enumerate}
\item How many specifications may a class declare to implement? Currently in some places it is one, it others it is many
\item Constructors -- do we allow constructors with bodies, and explicit pre- snf post- conditions in the specifications? We introduced them later in the day, and need to overhaul the oper semantics, $\Reach$ etc, or drop them altogether (ie no pre-post conditions in the Hoare logic).
\item The treatment of {\em unknown} as a possible interpretation and value of an expression, or an assertion. I think this does not pose problems, but I am not sure it is all consistent. Also, unsure whether we need the distinction between {\em unknown} and {\em undefined} -- bit SD thinks that we can make all this to work - after all similar problems appear in other systems (while the oracle-related issues are novel).
\item
How many parameters to a method or constructor? I now propose that we write the syntax with any number thereof, but already ope semantics onwards we require exactly one, and say this simplifies the exposition.
\item
Is it \prg{stmt} or \prg{stmts}?
\end{enumerate}

\paragraph{Status as of 14.10.2017} Some more changes need to be made.

\begin{enumerate}
\item Modules have to be stratified, and linking has to satitsfy the stratification.
\item We do not need the distinction between unknown and undefined. :-)
\item
SD thinks that in the formal system classes should satisfy one spec, and methods should have one parameter, and we can say in the text that the extension  is easy, and that the examples are allowed to use more than one.
\item
Is it \prg{stmt} or \prg{stmts}?
\end{enumerate}

\paragraph{High level explanations} Say that we deal with the impredicative nature of specification by only supporting the unfolding of definitions, and not the folding of them. 


%\appendix


%already defined in LLNCS
 \newtheorem{definition}{Definition}
 \newtheorem{example}{Example}
 \newtheorem{lemma}{Lemma}
 \newtheorem{theorem}{Theorem}

\newcommand{\HoareFigOne}
{
\begin{figure*}
$
\begin{array}{l}
\begin{array}{lcl}
\inferenceruleN {varAsg} {
} {
     \Hoare   {\kw{true}}
     {\kw{var}\,  \prg{v} \kw{:=} \prg{a} } { \prg{v}=\prg{a}\pre } {\kw{true}}
\\
     \Hoare   {\kw{true}}
      {  \prg{v} \kw{:=} \prg{a} } { \prg{v}=\prg{a}\pre }  {\kw{true}}
}
& &
\inferenceruleN {fieldAsg} {

}{
     \Hoare
     {\kw{true}}
      { \kw{this.}\prg{f}  \kw{:=} \prg{a} }
      { \kw{this.}\prg{f}   = \prg{a}\pre}
       {\kw{true}}
}
\\
\\
\inferenceruleN {cond-1} {
 \A \rightarrow_{\M} \prg{cond} \\
  \Hoare  {\A} {  \prg{stmts}_1\  } {\B} {\B' }
} {
   \Hoare  {\A} { {\kw{if}}\, \prg{cond}\,  {\kw{then}}\, \prg{stmts}_1\, {\kw{else}}\, \prg{stmts}_2  } {\B} {\B'}
   }
& &
\inferenceruleN {cond-2} {
 \A \rightarrow_{\M} \neg\prg{cond} \\
  \Hoare  {\A} {  \prg{stmts}_2\  } {\B} {\B'}
} {
   \Hoare  {\A} { {\kw{if}}\, \prg{cond}\,  {\kw{then}}\, \prg{stmts}_1\, {\kw{else}}\, \prg{stmts}_2  } {\B} {\B'}
   }
   \\
   \\
   \inferenceruleN {skip} {
} {
   \Hoare  {\A} { {\kw{skip}} } {\A} {\kw{true}}
   }
\end{array}
\\
%\\
 {
  \inferenceruleNN {newObj} {
  \M( {\cal S}(\M,\prg{C})) \ = \   \kw{spec}\ S\  \lb\    ....,\  \A \,\lb \,  {\kw{new}(\prg{x}_1,... \prg{x}_n)}\, \rb\, \B,\  ....\  \rb
 \hspace{2in}

}{
       \Hoare
        {     \A[\prg{a}_1/\prg{x}_1,... \prg{a}_n/\prg{x}_n]  }
         { \prg{x}\,  \kw{:=} \, \kw{new}\lp \prg{a}_1... \prg{a}_n \rp }
         %{      \B[(\prg{a}_1)\pre/\prg{x}_1,... (\prg{a}_n)\pre/\prg{x}_n]  }
         {      \B[\prg{a}_1/\prg{x}_1,... \prg{a}_n/\prg{x}_n,\prg{x}/\prg{res}] 
                \ \wedge\ \prg{x}\obeys 
                 {\cal S}(\M,\prg{C}) }
          { \kw{true} }
}
}
 % OLD VERSION
%{
% \inferenceruleNN {newObj} {
%    \prg{f}_1,... \prg{f}_n \mbox{ are the fields of } \M(\prg{C})
%\hspace{2in}
%  \prg{S}\in  {\cal S}(\M,\prg{C})
%}{
%      \Hoare
%       { \kw{true} }
%    { \prg{x}\,  \kw{:=} \, \kw{new}\, \prg{C}\lp \prg{a}_1... \prg{a}_n \rp }
%   { \prg{x.f}_{1}=(\prg{a}_1)\pre \, \wedge ...  \wedge\, \prg{x.f}_{n}=(\prg{a}_n)\pre
%    \  \wedge\ \prg{x}: \prg{C}\ \ \wedge\ \   \prg{x} \obeys \prg{S}  }
%       { \kw{true} }
%}
%}
\\ \\
  \inferenceruleNN {meth-call-1} {
     \M(S) \,= \,  \kw{spec}\ S\  \lb\  \overline{Policy}, \A \,\lb \, \prg{this.m(par)}\, \rb\, \B, \overline{Policy'}\  \rb
% \SP\SP \SP\SP\SP
  %  \B''[\prg{x}/\prg{this},\prg{y}/\prg{par}] = \B'''
    } {
    \Hoare  {\prg{x}\obeys S\wedge  \A[\prg{x}/\prg{this},\prg{y}/\prg{par}]} {\prg{v} \, \kw{:=}\, \prg{x.m(y)}} {\B[\prg{x}/\prg{this},\prg{y}/\prg{par},\prg{v}/\prg{res}]}   {{\kw{true}}} % {\B'''}
}
  \\
  \\
  \\
   \inferenceruleNN {meth-call-2} {
         \B \equiv \forall \prg{z}:\pre \Obj.\ [ \ \MayAccess({\prg{v},\prg{z}}) \rightarrow
    (\, \MayAccess\pre(\prg{x},\prg{z})\, \vee \, \MayAccess\pre(\prg{y},\prg{z})\, )\ \ ] \\
   \B' \equiv   \forall \prg{z},\prg{u}:\pre \Obj. \ [ \ \ \MayAccess({\prg{u},\prg{z}}) \rightarrow \\
         ~   \hspace{2.3in}  ( \MayAccess\pre(\prg{u},\prg{z}) \ \ \ \vee  \\
                   ~   \hspace{2.7in}[\  (\MayAccess\pre({\prg{x},\prg{z}})\vee \MayAccess\pre({\prg{y},\prg{z}}) )\ \wedge \\
                   ~   \hspace{2.8in}  (\MayAccess\pre({\prg{x},\prg{u}})\vee \MayAccess\pre({\prg{y},\prg{u}}) ) \ ] \ ) \ \  ]\\
   ~   \hspace{0.25in} \wedge \\
   ~   { \hspace{0.25in} \forall \prg{z}:\pre \Obj.\forall \prg{u}: \Obj.}\\
    ~   \hspace{1.7in}[\ \ { \New(\prg{u}) \wedge (\ \MayAccess(\prg{z},\prg{u})\vee\MayAccess(\prg{u},\prg{z})\ )}\ \  \ \rightarrow  \\
     ~   \hspace{2.6in} {(\ \MayAccess\pre(\prg{x},\prg{z})\vee\MayAccess\pre(\prg{y},\prg{z})\ ) }  \ \ \  ]
                 } {
    \Hoare  {\true} {\prg{v} \, \kw{:=}\, \prg{x.m(y)}} {\B}
           { \B'  }
}
\\ \\
   \inferenceruleNN {frame-methCall} {
           \Hoare  {\A \ {\wedge\ \prg{e}=\prg{e}} }   {\prg{x.m(y)}}  {\kw{true}}  {\forall \prg{z}. (\ \MayAffect(\prg{z},\prg{e})  \rightarrow \B'({\prg{z}})\ )\ \ \wedge }
          \\
% trick for layout
		\SP\SP\SP\SP\SP \SP\SP\SP\SP\SP \SP\SP\SP\SP\ \ \   \
		{\forall \prg{z}.(\ (\MayAccess\pre(\prg{x},\prg{z}) \vee \MayAccess\pre(\prg{y},\prg{z})
           \vee \New(\prg{z}) \ )%(\prg{z}:\prg{Object}\wedge \neg ( \prg{z}:\pre\prg{Object}) ) \ )
           \ \rightarrow\  \neg \B'(\prg{z})\ }
}
{
          \Hoare  {{\A}\ {\wedge\ \prg{e}=\prg{e}} } {\prg{x.m(y)}}   {\prg{e}=\prg{e}\pre}  {\kw{true}}
}
\\
\\
\inferenceruleNN{sequence} {
        \Hoare  {\A}{\stmts_1}{\B_1}{\B'}  \ \SP\SP\SP        \Hoare  {\A_2}{\stmts_2}{ \B_2}{\B'}
        % \\
       \ \SP\SP\SP
         {\A,\B_1\rightarrow_{\M,\O} \kw{true},\A_2}    \ \SP\SP\SP  {\B_1,\B_2\rightarrow_{\M,\O} \B}
        }
     {
       \Hoare  {\A}{\stmts_1; \, \stmts_2}{ \B}{\B'}
        }
\end{array}
$
\caption{Hoare Logic -- Basic rules of the language -- we assume that the module \M\ is globally given}
\label{fig:HoareLogicBasic}
\end{figure*}
}

\newcommand{\HoareFigTwo}
{
\begin{figure*}
$\begin{array}{l}
\begin{array}{lcl}
   \\
\\
   \inferenceruleN {frame-general} {
          \Hoare  {\A}{\stmts}{\B} {\B'}
           \\
        {\A  \rightarrow_{\M}   \stmts\disj  \A'     \SP\SP\SP\    \A  \rightarrow_{\M}   \stmts\ddisj  \A''}
} {
          \Hoare  {\A\wedge \A'} {\stmts} {\B\wedge{\A'}}  {\B' \wedge{\A''}}
}
& &
\inferenceruleN {Conj} {
 { \Hoare  {\A_1}{\stmts}{ \B_1}{\B_3}}\\
{  \Hoare  {\A_2}{\stmts}{ \B_2}{\B_4}}
} {
   { \Hoare  {\A_1\wedge \A2 }{\stmts}{ \B_1 \wedge \B_2 }{\B_3 \wedge \B_4}   }
 }
\\
\\
    \inferenceruleN {Cons-1} { % REFACTORED
          \Hoare  {\A}{\stmts}{ \B}{\B'} \SP\SP\SP {\A' \rightarrow_{\M} \A}
        \\
     {\B \rightarrow_{\M} \B''} \ \  \SP\SP\SP\SP\SP\SP\SP\SP {\B' \rightarrow_{\M} \B'''}
     }
     {
       \Hoare  {\A'}{\stmts}{ {\B''\wedge\B'}}{{\B'''}}
        }
&
&
    \inferenceruleN {Cons-2} {  % NEW
          \Hoare  {\A}{\stmts}{ \B}{\B''}   \\
           {\A',\B' \rightarrow_{\M} \A,\kw{true}}
     }
     {
       \Hoare  {\A'}{\stmts}{{\B' \rightarrow \B}}{\B''}
        }
%\\ \\
%\inferenceruleN {Cons-3}
%{ % was Cons-1
%  \Hoare  {\A}{\stmts}{ \B}{\B'} %
%  % SD put in Cons-4 \SP\SP\SP\SP\SP \toby{\A' \rightarrow_{\M} \A}
%  \\
% \A, \B \rightarrow_{\M} \kw{true},\A'
% % I droppped the \wedge \B', because Cons-1 does it
%} {
%   \Hoare  {{\A}}{\stmts}{ \A'}{\B'}
%   }
%& &
%  \inferenceruleN {Cons-4} { % was Cons-2
%          \Hoare  {\A}{\stmts}{ \A'}{\B'} % \SP\SP\SP \toby{\A'' \rightarrow_{\M} \A}
%        \\
%     \toby{{\A},\A' \rightarrow_{\M} \B} %\SP\SP\SP\SP\SP\SP\SP \toby{\B' \rightarrow_{\M} \B''}
%     }
%     {
%       \Hoare  {\A}{\stmts}{ \B}{\toby{\B'}}
%        }
   \\
   \\
 \inferenceruleN {code-invar-1} {
            { \M( S) \equiv \  \kw{spec}\ S\  \lb\  \overline{Policy}, P, \overline{Policy'}\  \rb}
  }
     {
       \Hoare   {{\kw{true}}} {\stmts} {{\kw{true}}} { {\prg{x} \obeys S} \rightarrow P[\prg{this}/\prg{x}]}
      }
    & &
  \inferenceruleN {code-invar-2} {

     }{
    \Hoare    {{e} \obeys S}  {\stmts}    {\kw{true}}  {{e}\pre \obeys S}
      }
  \\
   \\
  \inferenceruleN {quantifiers-1} {
             x \mbox{ does not appear in } \A, \mbox{ or } \B, \mbox{ or } \prg{stmts}
     \\
       \Hoare   {\A} {\stmts}{ \B'}{\B}
      }
   {
    \Hoare      {\A} {\stmts}{\forall x.\B'}{\B}
      }
           & &
  \inferenceruleN {quantifiers-2} {
             x \mbox{ does not appear in } \A, \mbox{ or  }  \B  % the ' does not show otherwise
             \mbox{ or } \prg{stmts}

     \\
       \Hoare   {\A} {\stmts}{ \B}{\B'}
      }
   {
    \Hoare      {\A} {\stmts}{\B}{\forall x.\B'}
      }
 \end{array}
  \end{array}
$
\caption{Hoare Logic -- we assume that the module \M\ is globally given}
\label{fig:HoareLogic}
\end{figure*}
}


\section{Formal Definition of the language \LangOO}

 \subsection{Modules and Classes}
 \label{formal:modules}

  \LangOO\ modules map class identifiers to class descriptions,
  {and specification identifiers} to specification descriptions.



\begin{definition}[Modules]
We define modules $\M\in\syntax{Module}$ as follows   \\  % to force line break

\begin{tabular}  {@{}l@{\,}c@{\,}ll}
\syntax{Module} \ \  &    =   & \syntax{ClassId}   \ $\cup$\  \syntax{SpecId}\   \\
& &  $\longrightarrow$ \\
& & ( \syntax{ClassDescr}    
\ $\cup$\ \syntax{Specification}  )
 \end{tabular}
\\ 
such that for any any module  $\M$, class identifier \prg{C}, and specification identifier \prg{S},  we have $\M(\prg{C})\in \syntax{ClassDescr}$ or undefined, and $\M(\prg{S})\in \syntax{Specification}$ or undefined.
We describe \syntax{ClassDescr}s in Definition \ref{def:syntax:classes}, and  \syntax{Specification}s in Defintion \ref{def:syntax:classes}.
\end{definition}



\paragraph{Classes}

We define the syntax of class definitions below. {Class definitions serve two purposes:
They describe how a class contributes to the runtime behaviour of a program: for this we use its field and method declarations.
They also describe how a class contributes to the
semantics of its module: which specifications it claims to obey,
how functions and  predicates are to be interpreted when applied to its objects\footnote{TODO: say earlier that  functions and predicates are interpreted depending on the class of receiver},
and how ghost fields are to be interpreted.\footnote{TODO: perhaps use model field and not ghost field.}
Note that ghost fields are interpreted according to the syntax of expressions - \syntax{Expr} - which will be given later.}

Note that the language is untyped. Method bodies consist of sequences of statements; these can be field read or field assignments (only allowed if the object is \prg{this} -- i.e. as in Smalltalk), conditionals, and method calls. All else can be encoded.

 {The keyword \kw{private} indicates that a class, say \prg{C},  is  private to its module. This means that even though anybody who has access to  objects of class \prg{C} may  call any methods on these objects, only classes from the same module as \prg{C} may construct objects of class \prg{C}.
   In terms of the capabilities literature, this means that the capability to create new objects of that class is hidden within this module.
   This form of privacy is enforced through the linking operator defined in Definition \ref{formal:Linking}.}

 \begin{definition}[Classes, Methods, Args]
\label{def:syntax:classes}
We define the syntax of modules below.

\begin{tabular}{lcll}
 \syntax{ClassDescr}   &   \BBC  &   $ [ $  \kw{private} $ ]$  \kw{class}  \syntax{ClassId} \\
 & &  \hspace{0.7in}  \kw{implements}  \syntax{SpecId}\\
 & &   \lb\,  $($\ \kw{field} \syntax{FieldId}\ $)^*$ \\
 & &  \hspace{0.16in}    $($\ \syntax{methBody}\ $)^*$ \\
 & &  \hspace{0.16in}  {$($\ \syntax{FunDescr}\ $)^*$}  \\
  & &  \hspace{0.16in}  $($\ \syntax{PredDescr}\ $)^*$  \\
 & &  \hspace{0.16in}   $($\ \kw{ghost}   \syntax{GhFldId}\ = \ \syntax{Expr} \ $)^*$  \ \rb
\\
\syntax{methBody} &\BBC&
     \kw{method}    \syntaxMeth\lp \syntax{ParId}$^*$ \rp\\
 & & \hspace{0.1in}  \lb\, \syntax{Stmts} \semi   \kw{return}  \syntax{Arg}  \,
    \rb
 \\
 \syntax{Stmts}  &\BBC&  \syntax{Stmt}     ~\SOR~  \syntax{Stmt} \semi \syntax{Stmts} \\
\syntax{Stmt}    &\BBC&     \kw{var} \syntax{VarId}  {\kw{:=}} \syntax{Rhs}\\
&  ~\SOR~ &     \syntax{VarId} {\kw{:=}} \syntax{Rhs} \\
&  ~\SOR~ &    \kw{this}.\syntax{FieldId} {\kw{:=}} \syntax{Rhs} \\
&  ~\SOR~ &   \kw{if}  \syntax{Arg}  \kw{then} \syntax{Stmts} \kw{else} \syntax{Stmts}\\
&  ~\SOR~ &   \kw{skip}\\
\syntax{Rhs} & \BBC&    {\syntax{Arg}}{\kw{.}}\syntax{MethId}\lp  \syntax{Arg}$^*$ \rp    ~\SOR~   \syntax{Arg}  \\
&  ~\SOR~ &   {\kw{new} \syntax{ClassId}\lp \, \syntax{Arg}$^*$\, \rp} \\
 \syntax{Arg} &\BBC&  \syntax{ParId} ~\SOR~ \syntax{VarId} ~\SOR~ {\kw{this}} \\
&  ~\SOR~ & {\kw{this}}.\syntax{FieldId}
 \end{tabular}
\end{definition}

Note that \LangOO\, supports a limited form of protection: the syntax
  supports  reading of fields of any object, but restricts each object
  to being able to modify only its \emph{own} fields.






\paragraph{Lookup} We define  method lookup function, $\cal M$ which returns the corresponding method definition given a class and a method identifier, specification lookup function $\cal S$ which returns the set of specification identifiers which a class is claiming to be implementing, and  $\cal P$ .... and $\cal G$  ...
\begin{definition}[Lookup] The lookup functions $\cal M$, $\cal S$, and $\cal P$ are defined as follows, where we assume that \prg{C} is a class identifier,   \prg{m} a method identifier,
$P$ a predicate identifier, and \prg{f} a function identifier : $ ~ $ \\

\noindent
$
\Meths {} {\prg{C}} {m}      =       \kw{method}\, m\, \lp p_1, ... p_n \rp
\lb stms \semi\, \kw{return}\ {\syntax{a}} \rb
 $

\noindent
 iff \ $ \M(\prg{C}) =  p\ \kw{class}\, \prg{C}\, \ .... \lb ...  $\\
$ ~ $  \hspace{.8in} $ \kw{method}\, m\, \lp p_1, ... p_n \rp
\lb stms \semi\, \kw{return}\ {\syntax{a}} \rb$\\
$ ~ $ \hspace{.7in} $ ... \rb.$
\\
\noindent
undefined, otherwise. $ ~ $ \\


\noindent
 $
   {\cal S} ({\M}, {\prg{C}})\      = \     \{ \ \syntax{S} \ \}
 $

\noindent
 iff \ $ \M(\prg{C}) \ =\   p\ \kw{class}\  \prg{C} \ \kw{implements}\ \syntax{S}  \lb   ...  \rb $\\
\noindent
undefined, otherwise. $ ~ $ \\


\noindent
 $
   {\cal P} ({\M}, {\prg{C}}, {Q})\      =  \     \kw{predicate}\, Q\, \lp p_1, ... p_n \rp
\lb\ \A \ \rb
 $

\noindent
 iff \ $ \M(\prg{C})\ =\ p\  \kw{class}\  \prg{C} \ .... \lb ...  $\\
$ ~ $  \hspace{.8in} $ \kw{predicate}\, Q\, \lp p_1, ... p_n \rp
\lb\ \A \ \rb $\\
$ ~ $ \hspace{.7in} $ ... \rb.$
\\
\noindent
undefined, otherwise.$ ~ $ \\


\noindent
 $
   {\cal G} ({\M}, {\syntax{C}}, {\mathit{gf}})\      =       expr $

\noindent
 iff \ $ \M(\syntax{C}) =\ p\  \kw{class}\  \syntax{C} \ .... \lb ...  $\\
$ ~ $  \hspace{.87in} $ \kw{ghost}\,   {\mathit{gf}}\, \kw{=}\  expr $\\
$ ~ $ \hspace{.7in} $ ... \rb.$
\\
\noindent
undefined, otherwise.

 $ ~ $ \\
\noindent
{In all the above, $p$ is either empty, or the annotation \kw{private}. }
\end{definition}


  \subsection{Execution  in \LangOO}



In order to define execution in  \LangOO, we define runtime states, $\sigma$, the interpretation of arguments in such states,
$\interp {a} {\sigma} $, and finally a relation of the form $\M, \sigma, stms \leadsto \sigma'$.\footnote{TODO: rename private to internal; rename ghost field to model field}.

\begin{figure*}
$\begin{array}{l}
\begin{array}{lcl}
 \inferenceruleN {methCall\_OS} { 
 \interp{\syntax{a}}{\phi\cdot\chi} = \iota \\
   \forall i\in\{1..n\}.\ \ \ \   \interp{{\syntax{a}}_i}{\phi\cdot\chi} = \syntax{val}_i 
 \\
\Meths {} {$\chi(\iota)\downarrow_1$} {m} \  =  \\
  ~ \hspace{.1in} \kw{method}\ m\lp par_1, \ldots par_n \rp
 \lb stms \semi\, \kw{return}\ \syntax{a}' \rb
 \\
 \phi''=\this \mapsto \iota, par_1 \mapsto val_1, \ldots par_n \mapsto val_n
 \\
 \M,\, \phi''\cdot\chi, stmts \ \leadsto  \phi'\cdot\chi' 
}
{
  \M,\, \phi\cdot\chi,\, \syntax{a}{\kw{.}}\syntax{m}\lp {\syntax{a}_1,\ldots\syntax{a}_n}\rp  \ \leadsto\ 
    \chi', \interp{\syntax{a'}} {\phi' \cdot\chi'}
}
& &
\begin{array}{l}
\inferenceruleN {Arg\_OS} {
       } {
     \M,\, \phi\cdot\chi,\, {  \syntax{a}} 
        \ \leadsto\    \chi,\interp{\syntax{a}}{\phi\cdot\chi}  
}
\\
\inferenceruleN {\sd{New\_OS}} {
      \addr {} \mbox{ is new in}\ \chi \\
       \prg{f}_1, ... \prg{f}_n \mbox{ are the fields defined in } \syntax{C}
} {
  \M{},\,
  \phi\cdot\chi,\, {\kw{new}\, \syntax{C} \lp \syntax{a}_1, ... \syntax{a}_n} \rp \\
        \ \leadsto\  
        \chi[ \addr \mbox{\mapsto} (\syntax{C}, \prg{f}_1 
    \!\!\mapsto\!\!        \interp{\syntax{a}_1}{\phi,\sigma}
        ..\prg{f}_n\!\!\mapsto\!\! \interp{\syntax{a}_n}{\phi,\sigma})]\, 
        % \addr
}
\end{array}
\\ \\
 \inferenceruleN {varAsg-1\_OS} {
       \M,\, \phi\cdot\chi,\, {\syntax{e}}
        \ \leadsto\  \sd{\chi'},\,\syntax{val}
} {
    \M,\, \phi\cdot\chi,\, {\kw{var}}\,\syntax{v} {\kw{:=}} \syntax{e}
        \ \leadsto\  \phi[v\mapsto val]\sd{\cdot\chi'}        \\
}
& &
\inferenceruleN {varAsg-2\_OS} {
       \M,\, \phi\cdot\chi,\, {\syntax{e}}
        \ \leadsto\  \sd{\chi'} ,\, \syntax{val}
} {
    \M,\, \phi\cdot\chi,\, {\syntax{v} {\kw{:=}} \syntax{e}}
        \ \leadsto\  \phi[v\mapsto val]\sd{\cdot\chi'}
        \\
}
\\ \\
\inferenceruleN {fieldAsg\_OS} {
       \M,\, \phi\cdot\chi,\, {\syntax{e}}
        \ \leadsto\  \phi\cdot\chi',\, \syntax{val}
} {
    \M,\, \phi\cdot\chi,\, {  {\kw{this.}}\syntax{f} {\kw{:=}}\ \syntax{e}}
        \ \leadsto\  \phi \cdot\chi'[\phi({\kw{this}}),\syntax{f}\mapsto \syntax{val}]
        \\
}
& &

\inferenceruleN {sequence\_OS} {

       \M,\, \sd{\sigma}, {\syntax{stmt}} 
        \ \leadsto\  \sd{\sigma''} 
        \\
       \M,\, \sd{\sigma''} ,\, {\syntax{stmts}} 
        \ \leadsto\  \sd{\sigma'} 
} {
    \M,\, \sd{\sigma} ,\, {   \syntax{stmt} \,\semi\, \syntax{stmts}   }
        \ \leadsto\ \sd{\sigma'} 
        }
\\
\\
\inferenceruleN {cond-True\_OS} {
\interp{\syntax{a}}{\sd{\sigma} } = \kw{true}
\\
       \M,\, \sd{\sigma},\, {\syntax{stmts}}_1
        \ \leadsto\  \sd{\sigma'} 
} {
    \M,\, \sd{\sigma},\, { {\kw{if}}\, \syntax{a}\,  {\kw{then}}\, \syntax{stmts}_1\, {\kw{else}}\, \syntax{stmts}_2  }
        \ \leadsto\  \sd{\sigma'}
}
& &
\inferenceruleN {cond-False\_OS} {
\interp{\syntax{a}}{\sd{\sigma} } = \kw{false}
\\
       \M,\, \sd{\sigma} ,\, {\syntax{stmts}}_2
        \ \leadsto\  \sd{\sigma'} 
} {
    \M,\, \sd{\sigma},\, { {\kw{if}}\, \syntax{a}\,  {\kw{then}}\, \syntax{stmts}_1\, {\kw{else}}\, \syntax{stmts}_2  }
        \ \leadsto\  \sd{\sigma'} 
}
\\
\\
\inferenceruleN {skip\_OS} {
 } {
    \M,\, \sd{\sigma},\, { {\kw{skip} }}
        \ \leadsto\  \sd{\sigma} 
}

 \end{array}
\end{array}
$
\label{fig:Execution}
\caption{Operational Semantics}
\end{figure*}


  \paragraph{Runtime state}
\label{formal:state} The runtime state~$\sigma$ consists of a stack frame~$\phi$, and a heap $\chi$. A stack frame  is  a mapping from
  receiver (\this) to its address, and from {the local variables ($\syntax{VarId}$) and parameters ($\syntax{ParId}$) to their values.}  Values are integers,
 the booleans \kw{true} or \kw{false}, addresses, or \kw{null}. Addresses are ranged over by $\iota$. The heap maps addresses to objects. Objects are tuples consisting of the class of the object, and a  mapping from field identifiers onto values.


\begin{tabular}{lll}
$\sigma \in $ \syntax{state}  & =  &  \syntax{frame}  $\times$ \syntax{heap} \\
 $\phi \in$   \syntax{frame}  & =  &   \syntax{{StackId}} $\longrightarrow$ \syntax{val} \\
 $\chi \in$   \syntax{heap}  & =  \ \ \ &  \syntax{addr} $\longrightarrow$ \syntax{object} \\
 $\prg{v}\in$ \syntax{val} & = &  $\{$   \nullK, \kw{true}, \kw{false}  $\}$    $\cup\  \syntax{addr}\ \cup \ \mathbb{N}$ \\
 \syntax{object}\ \ & = &  \syntax{ClassId} $\times$ ( \ \syntax{FieldId} $\longrightarrow$ \syntax{val} \ ) \ \SP \SP \\
$\iota, \iota',..$  & $\in$ & $\syntax{addr}$  \\
% \syntax{res}  & = & $\{\ \nullPEC, \  \stuck\ \}$\ $\cup$\ \\
{\syntax{StackId}}  & = & {$\{\ \kw{this}\  \}$\ $\cup$\ \syntax{VarId}  $\cup$\ \syntax{ParId}}\\
\end{tabular}
\footnoteC{{we used to support null pointer exceptions and stuck, but see comment below}}

 \paragraph{The Operational Semantics of \LangOO}
\label{formal:semantics}

\newcommand{\cons}{\mathit{::}}
\newcommand{\st}{\ensuremath{\mathit{st}}}
\newcommand{\fs}{\ensuremath{\mathit{fs}}}
\newcommand{\traverseFields}[3]{\mathit{traverseFields}_{#1}(#2,#3)}
{We define $\interp{a}{\sigma}$, the {\em interpretation} of  {an argument~$a\!\in\!\syntax{Arg}$} in a state $\sigma$ as follows. 

\begin{definition}[Interpretation] For a state {$\sigma = ({\phi}%\cons\st
,\chi)$} we define the partial function\\
$\strut \ \ \ \ \interp {\_} {\_} :  state \times Path \rightarrow Value$\\
as follows:

$\begin{array}{lcl}
\interp {\kw{null}} {\sigma} & = & {\kw{null}}
\\
\interp {\kw{true}} {\sigma} & = & {\kw{true}}
\\
{\interp {\kw{false}} {\sigma}} & = & {\kw{false}}
\\
\interp {{x}} {\sigma} & = & {\phi(x) \quad \mathrm{(for\   x \in\ } {\syntax{StackId}})},
\\
& & \mbox{ undefined,} \quad \ \ \ \mbox{otherwise}.
\\
 {\interp {p\kw{.}f} {\sigma}}  & = & {\chi(\interp {p} {\sigma})(f)}   \quad   \mbox{if} \interp {p} {\sigma}
\mbox{ is defined,} \\
& &  \quad \quad  \quad \quad  \quad \quad \mbox{and}\,  $f$\, \mbox{is a field of} \interp {\kw{this}} {\sigma}
\\
& & \mbox{ undefined,} \quad \ \ \ \mbox{otherwise}
\end{array}$
where \\
\begin{tabular}{lcll}
 $p \in$ \syntax{path}   &   \BBC  &    $x$  ~\SOR~ p.f
 \end{tabular}

 \noindent
We also define the  lookup of the class of an expression,
$\begin{array}{lcl}
$~$   \ {\mathcal C}lass(e)_{\sigma} & = & (\sigma\downarrow_2(\interp {e} {\sigma}))\downarrow_1 \quad \ \ \  \mbox{if \interp {e} {\sigma}\ defined}
\\
& & \mbox{ undefined,} \quad \ \ \ \mbox{otherwise}
\end{array}$

\newcommand{\emptylist}{\epsilon}
\noindent
\end{definition}

Execution uses module \M, and  maps a runtime state $\sigma$ and statements
\syntax{stmts} (respectively a right hand side $rhs$) % (code in the general case)
onto a new state $\sigma'$ (respectively a new heap $\chi'$ and
a value). {We therefore do not give execution rules for things like null-pointer-exception, or stuck execution. This allows us to keep the system simple; it will be easy to extend the semantics to a fully-fledged language.}
\begin{definition}

 
Execution  of \LangOO\ statements and expressions is defined in figure \ref{fig:Execution}, and has the following shape:
\begin{tabular}{lcl}
 ${\rewriteLong {}}\s$ &  :  &    \syntax{Module}  $\times$  \syntax{state}  $\times$   {\syntax{Stmts}}
  \ \  $\longrightarrow $ \ \     {{\syntax{state}}}
\\
${\rewriteLong {}}\s$ &  :  &    \syntax{Module}  $\times$  \syntax{state}  $\times$   {{\syntax{Rhs}}}
  \ \  $\longrightarrow $ \ \     \syntax{heap} $\times$ \syntax{val}

\end{tabular}

\noindent
\end{definition}

{Note that execution is undefined -- will be stuck -- if we try to access fields or call methods which are not part of the object, and also, if we execute a conditional where the condition is not a boolean (neither \kw{true} nor \kw{false}). Such situations are possible, since \LangOO\, is untyped. Wrt to security the question arises whether the language then guards against denial of service attacks - pass the wrong kind of object, and get stuck. The answer is: yes in theory.  Even though we could guard against these by throwing exceptions, we cannot guard against denial of service attacks when passed an object on which "our" code will execute a method which will loop forever. All this does not affect the validity of our approach, since we are only claiming partial correctness.\footnote{TODO: Add similar disclaimer in the Soundness theorem}.
}

 

\paragraph{Arising Configurations}

Policies need to be satisfied in all configurations which may arise during execution of some program. This leads us the concept of {\em arising} configuration. Arising configurations allow us to restrict the set of configurations we need to consider. For example, in a program where a class does not export visibility to a field, the constructor initialises the field to say $0$, and all method calls increment that field, the arising configurations will only consider states where the field is positive.


We  can now define $\Arising(\M)$ as the set of runtime configurations
which may be reached during execution of some initial context ($\sigma_0$,$\code_0$),
with the module \M expanded with {\em any} module $\M'$.
 A context is initial if its heap contains only objects of class \prg{Object}.
%, and its stack contains only one frame]
 % and the code contains exclusively method calls of methods defined in $M$. }
%   form
%   consider all configurations which may be reached from such initial configurations.
  %  which are well-typed
% under the assumptions that \kw{x} and \this\ denote objects of class \prg{Object}.
%These concepts are defined in App. \ref{formal:arising}.
%The {\em arising} configurations are those which may be reached by executing an initial configuration, where initial configurations all configurations that may be encountered at the start of program execution. %, \cf \ref{def:initial}.

\begin{definition}[Arising and Initial configurations] $ $ We define the mappings \\
$\begin{array}{lcl}
\SP {\mathcal{I}nit} & \ : \ & {\syntax{Module}} \   \longrightarrow\  \mathcal{P}( {\syntax{state} \times \syntax{Stmt}}  )
\\
 \Arising & : &  {\syntax{Module}}    \longrightarrow \mathcal{P}( {\syntax{state}} \times \syntax{Stmts}  )
\end{array}$

\noindent
as follows:\\
$\begin{array}{l@{\,}l}
 {\mathcal{I}nit}(\M)   =  &  \{ \ (\ \sigma_0, \kw{new}\ \clss{}.\prg{m}\lp\kw{new}\  \clss{'} \rp) \ |    \ \clss{},\clss{'}\in dom( \M) \\
    &  \SP  \mbox{where}\  \sigma_0\ =\ ( (\iota,\kw{null}),\chi_0), \\
    & \SP  \mbox{and}\   \chi_0(\iota)=(\prg{Object}, \emptyset)\  \}
 \\
 \Arising(\M)    =   & \    \bigcup_{ (\sigma,\code{})\in  \mathcal{I}nit (\M)}  \Reach(\M, \sigma, \code{})
\end{array}$
\end{definition}


\noindent
Initial configuration should be as ``minimal'' as possible, We therefore construct a heap which has only one object, and execute a method call on a newly created
object, with another newly created object as argument.




\begin{figure*}
$
\begin{array}{lcl}
%  \Reach(\M{}, \sigma,\kw{a}) & \ =\  & \{ \ ( \kw{skip}, \sigma ) \}  %  \{ \ (\kw{a},\sigma) \ \}
  \\
 \Reach(\M{}, \sigma, \prg{v}\kw{:=} \kw{new}\, \clss{}\lp \prg{a}_1,...\prg{a}_n \rp) & = &  \{ \ ( \prg{v}\kw{:=}\kw{new}\, \clss{}\lp \prg{a}_1,...\prg{a}_n \rp,\sigma), \ ( \kw{skip}, \sigma' ) \}
 \\
 & &  \mbox{where }  \M,\sigma, \prg{v}\kw{:=} \kw{new}\, \clss{}\lp \prg{a}_1,...\prg{a}_n \rp \leadsto \sigma'
 \\
\Reach(\M{}, \sigma,\prg{stmt}\semi\prg{stmts}) & = &
\Reach(\M,\sigma,\prg{stmt}) \cup \Reach(\M, \sigma',\prg{stmts})
\\ & & \mbox{where }  \M,\sigma, \prg{stmt} \leadsto \sigma'
\\
\Reach(\M{}, \sigma,\prg{v}\kw{:=}\prg{a}) & = &
 \{ ( \prg{v}\kw{:=}\prg{a}, \sigma), \ ( \kw{skip}, \sigma' ) \}
 \\
 & &  \mbox{where }  \M,\sigma, \prg{v}\kw{:=} \prg{a} \leadsto \sigma'
\\

\Reach(\M{}, \sigma,\prg{v} \kw{:=} \prg{a}\kw{.}\prg{m}\lp \prg{a}_1,...\prg{a}_n \rp) & = &
\{ \ (\prg{v} \kw{:=}\prg{a}\kw{.}\prg{m}\lp \prg{a}_1,...\prg{a}_n \rp, \sigma ) , \ ({\kw {skip}}, \sigma''' ) \  \} \ \cup
\ \Reach(\M{}, \sigma',\prg{stmts}) \\
& &  \mbox{where}\ \sigma = \phi \cdot\chi\ \mbox{and}
 \interp{\syntax{a}}{\phi\cdot\chi} = \iota\ \mbox{and} \\
& & \Meths {} {$\chi(\iota)\downarrow_1$} {m} \  =  \\
& &  ~ \hspace{.1in} \kw{method}\ m\lp par_1, \ldots par_n \rp
 \lb stms \semi\, \kw{return}\ \syntax{a}' \rb
\ \mbox{and}\\
& & \phi'=\this \mapsto \iota, par_1 \mapsto  \interp{{\syntax{a}}_1}{\phi\cdot\chi}, \ldots par_n \mapsto  \interp{{\syntax{a}}_n}{\phi\cdot\chi}
\ \mbox{and}\\
& & \M,\, \phi'\cdot\chi, stmts \ \leadsto  \sigma''
\ \mbox{and} \ \sigma'''= (\sigma\downarrow_1[\prg{v} \mapsto \interp{a'}{\sigma''}],  \sigma''\downarrow_2)
\\
\Reach(\M{}, \sigma,{\kw {skip}}) & = &
\{ \ ({\kw {skip}}, \sigma ) \  \}   \\
\Reach(\M{}, \sigma,{\kw {if}}\, \prg{a}\, {\kw {then}}\, \prg{stmts}_1\, {\kw {else}}\, \prg{stmts}_2\, ) & = &
\{ \ ( {\kw {if}}\, \prg{a}\, {\kw {then}}\, \prg{stmts}_1\, {\kw {else}}\, \prg{stmts}_2, \sigma ) , \  \}  \  \cup \
\Reach(\M{}, \sigma,\prg{stmts''})\\
& & \mbox{ where } \prg{stmts''}=\prg{stmts}_1 \mbox{ if } \interp{a}{\sigma}={\kw {true}},\ \mbox{ otherwise }
\prg{stmts''}=\prg{stmts}_2
\end{array}
$
\caption{Reachable Configurations\label{fig:reach}}
\end{figure*}


\paragraph{Reachable Configurations}
\label{formal:reachable}

A configuration is reachable from another configuration, if the former may be required for the evaluation of the latter after any number of steps.

$
\SP  \Reach \ :
\ {\syntax{Module} \times \syntax{state} \times \syntax{Stmts}}  \\
\SP\SP\SP\SP \SP\SP\SP\SP \longrightarrow \mathcal{P}( {\syntax{Stmts} \times \syntax{state}}  )
$

%\kjx{again: e or Stmts??? fix here, fix in the paper} STmts

\noindent
In figure \ref{fig:reach} we define the function  $\Reach$  by cases on the structure of the expression, and depending on the execution of the statement. The set $\Reach(\M, \sigma, \stmts)$
  collects all configurations reachable during execution of $\sigma, \stmts$. Note that the function  $\Reach(\M, \sigma, \stmts) $ is defined, even when the execution should diverge; of course then it may be an infinite set.
  The definedness of  $\Reach(\M, \sigma, \stmts) $  is important, because it allows us to give meaning to capability policies without requiring termination.


 
\begin{lemma}[$\Reach$ and $\leadsto$]\label{lemma:ReachAndOpSem}
For all $\M$, $\M'$, $\sigma$, $\sigma'$, $\sigma'$,and \prg{stmt}, $\prg{stmt}'$, and $\prg{stmt}''$:
\begin{itemize}
\item 
If $\M, \sigma, \prg{stmt} \leadsto \sigma'$, then
$(\_,\sigma') \in \Reach (\M,\sigma,\prg{stmt})$.
\item 
If  
$(\prg{stmt}',\sigma') \in \Reach (\M,\sigma,\prg{stmt})$,  and\\
 $(\prg{stmt}'',\sigma'') \in \Reach (\M,\sigma',\prg{stmt}')$,  then\\
$(\prg{stmt}''',\sigma'') \in \Reach (\M,\sigma,\prg{stmt})$.
\item
If $\M*\M'$ is defined, and $(\prg{stmt}',\sigma') \in \Reach (\M,\sigma,\prg{stmt})$, then
$(\prg{stmt}',\sigma') \in \Reach (\M*\M',\sigma,\prg{stmt})$.
\item
If $\M*\M'$ is defined, then  $\Arising (\M)\subseteq\Arising (\M*\M')$.

\end{itemize}
\end{lemma}
\begin{Proof}
By structural induction on $\leadsto$ and the definition of $\Reach$ and $\Arising$.
\end{Proof}

\paragraph{Notation} We shall use $\sigma'\in \Reach (\M,\sigma,\prg{stmt})$ as a shorthand for $(\_,\sigma')\in \Reach (\M,\sigma,\prg{stmt})$ and $\sigma'\in \Arising (\M)$ as a shorthand for $(\_,\sigma')\in \Arising (\M)$.



\section{The Specification Language Chainmail}
 \label{sec:SpecLan}

Our specifications and policies are expressed in terms of one-state as well as two-state
assertions. To express the state in which an expression is evaluated, we annotate it with a \t-subscript. For example, given $\sigma$ and $\sigma'$ where $\interp{x}{sigma}$={4},  and $\interp{x}{sigma}$={3}, we have  $\M,\sigma,\sigma' \modelsWithO x\pre {-} x\post = 1$.
% {The specification language uses functions and  predicates  defined below.}.
% {In function bodies
% \footnote{a better term than \syntax{FArg}?}  we allow the use of values, as well as known mathematical operators such as $+$, $-$ etc. Similarly, in predicate bodies we allow the use of known mathematical operators such as $\leq$, $\leq$ etc. \footnote{{SD asks: do we need to allow $\wedge$, $\not$ etc explicitly? Perhaps they can all be encoded with what we have}}
% }\footnote{{TODO: Find better names than FunBody and PredBody}}

\subsection{Expressions} We first define expressions, $\sE\in\syntax{Expr}$,
% and assertions \syntax{\A},
which depend on  {\em one state} only.
We allow the use of mathematical operators, like $+$
and $-$, and we use the identifier  $\mathit{f}$    to indicate
functions whose value depends on the state (eg the function
\prg{length} of a list). We use  identifier  $\syntax{Q}$, or   $\syntax{P}$  to
range over {predicate  identifiers}.
%  whose validity depends on the state (eg the predicate \prg{Acyclic} for a list).

The difference between expressions $\sE$, and arguments $\mathit{ar}$, as from
the earlier section, is that
expressions may depend  ghost information, which is not
stored explicitly in the state $\sigma$
but can be deduced from it ---  e.g.\ the length of a list
that is not stored with the list. {Thus, expressions may read the fields from
any object reachable from \kw{this}, or
any argument or local variable; they may access ghost fields, and they may call functions.}

\begin{definition}[Expressions]
$ ~ $ \\

\noindent
% \begin{figure}[h]
\begin{tabular}{lcll}
{\syntax{Path}}  &   \BBC  &   \syntax{ParId} ~\SOR~ \syntax{VarId} ~\SOR~ {\kw{this}}
    \SOR   \syntax{Path}{\kw{.}} \syntax{FieldId}
\\
 \syntax{\sExpr} \ \  &   \BBC  &   {\syntax{Arg}}  ~\SOR~ {\syntax{Path}}  ~\SOR~ \syntax{Val}  \\
  &  ~\SOR~   & ~\SOR~ \syntax{\sExpr} $+$ \syntax{\sExpr}  ~\SOR~ ... ~\SOR~
 \syntax{f}(\syntax{\sExpr}$^*$) \\
&     ~\SOR~ & \kw{if} \syntax{\sExpr} \kw{then} \syntax{\sExpr} \kw{else} \syntax{\sExpr}
 \\
\syntax{funDescr} \ \  &   \BBC  &   \kw{function}  \syntax{f}\lp \syntax{ParId}$^*$ \rp\lb~   \syntax{\sExpr}~ \rb
 \end{tabular}
\label{fig:syntax:functions}
% \end{figure}

\end{definition}

We now define the value  of such expressions through the interpretation function
$\interp{e}{\M,\sigma} $. {We distinguish between the case where the function returns a value, is undefined, or unknown. The latter case arises only when the expression mentions functions which are unknown in the current module.}

\begin{definition}[Interpretations]
We define the interpretation of expressions, as a {\em partial} function:

\noindent
$
\begin{array}{c}
% \begin{array}{lcl}
\  \interp{\cdot}{}\, : \, \syntax{\sExpr} \times   \syntax{Module} \times   \syntax{state}
% \times \syntax{state}
\rightarrow \syntax{Value}
% \times \syntax{state}
% \end{array}
 \end{array}
$

\noindent
using the notation  $\interp{\cdot}{\M,\sigma}$: \\


\begin{itemize}
\item
 $\interp{val}{\M,\sigma} $    = $val$,\  for all values $val\ {\in \syntax{Val}}$.
\item
 $\interp{{a}}{\M,\sigma} $    = $\interp{{a}{} }{\sigma}${, for all arguments $a\ \in \syntax{Arg}$}.
 \item
 {${\interp {p\kw{.}f} {\M,\sigma} } $} = $ {\chi(\interp {p} {\sigma})(f)}$  {if} $\interp {p} {\sigma}$
  is defined,
  \\   $~$ \hspace{.2in}  and  $f$ is a field of $\interp {p} {\sigma}$
\\
undefined,  otherwise.
\item
{$ \interp {p\kw{.}\mathit{gf}} {\M,\sigma}$} = $ \interp {e[p/\prg{this}]} {\M,\sigma} $ if  $\interp {p} {\M,\sigma}$
 is defined,  \\
% $~ $   \hspace{.2in}  and $\mathit{gf}$ is a ghost field of that object,\\
 $~ $   \hspace{.2in}  and  $\syntax{e}$= ${\cal G}(\M,{\cal C}lass(\interp {p} {\sigma}, \sigma),\mathit{gf}))$,
\\ {unknown,}\ \  if $\interp {p} {\sigma}, \sigma$ defined, \\
$~ $   \hspace{.2in}  and
$\M({\cal C}lass(\interp {p} {\sigma}, \sigma))$   undefined
\\  undefined,\ \    otherwise.

\item
  $\interp{\syntax{\sE_1}+ \syntax{\sE_2}}{\M,\sigma}$ =
 $\interp{\syntax{\sE_1}}{\M,\sigma} + \interp{\syntax{\sE_2}}{\M,\sigma}$,
 \\
 $~ $   \hspace{.1in}  if $\interp{\syntax{\sE_1}}{\M,\sigma}$
 and $\interp{\syntax{\sE_2}}{\M,\sigma}$ are numbers,\\
 undefined, $~ $   \hspace{.1in}  if $\interp{\syntax{\sE_1}}{\M,\sigma}$
 or $\interp{\syntax{\sE_2}}{\M,\sigma}$ are undefined, \\
$~ $  \hspace{.2in}   or not numbers
 \item
  $\interp{\syntax{f}(\syntax{\sE_1},...\syntax{\sE_n})}{\M,\sigma}$ =
 $\interp{\syntax{\sE}[\syntax{\sE_1}/\prg{this},\sE_2/\syntax{p_2}, ... \syntax{\sE_n}/\syntax{p_n}]}{\M,\sigma}$
 \\   $ ~ $ $ \hspace{.2in} $ where
 $ \M(f, {\cal C}lass(\sE,\sigma)) =  \kw{function} ~\syntax{f}~\lp  \syntax{p_2}...\syntax{p_n}  \rp\lb~   \syntax{\sE}~ \rb $,
\\
undefined, $~ $   \hspace{.1in} if $\exists i$.  $\interp{\syntax{\sE_i}}{\M,\sigma}$ undefined,\\
% $~ $   \hspace{.2in} or $\interp{\syntax{\sE}[\syntax{\sE_1}/\syntax{p_1}, ... \syntax{\sE_n}/\syntax{p_n}]}{\M,\sigma}$  undefined,
%\\
{unknown,}  $~ $   \hspace{.1in} if  $\exists i$.  $\interp{\syntax{\sE_i}}{\M,\sigma}$ unknown, \\ $~ $   \hspace{.2in} or
if $ \M({\cal C}lass(\sE_1,\sigma),f )$ undefined.

 \item
$\interp{ \kw{if} ~\syntax{\sE_0} ~\kw{then} ~ \syntax{\sE_1} ~\kw{else} ~\syntax{\sE_2}}{\M,\sigma} $\\
$ ~ $ $ \hspace{.2in} $=$\interp{ \ \syntax{\sE_1} }{\M,\sigma} $, \ \   \ \ if $\interp{ \ \syntax{\sE_0} }{\M,\sigma} $=\kw{true}, \\
$ ~ $ $ \hspace{.2in} $=$\interp{ \ \syntax{\sE_2} }{\M,\sigma} $, \ \  \ \  if $\interp{ \ \syntax{\sE_0} }{\M,\sigma} $=\kw{false}.
\\
{undefined,} if $\interp{ \ \syntax{\sE_0} }{\M,\sigma} $ undefined, \\
$ ~ $   \hspace{.2in} or
$\interp{ \ \syntax{\sE_0} }{\M,\sigma} $ defined but not \kw{true} or \kw{false}.
\\
{unknown,} if $\interp{ \ \syntax{\sE_0} }{\M,\sigma} $ unknown.
% \item TODO: Say something about further predefined functions
 \end{itemize}


\end{definition}

\noindent
Thus,  the interpretation  $\interp {e} {\M,\sigma}$ is undefined if $e$ refers to non-existing variables, or fields, or calls a infinite recursive function. For example, $\interp {length(x)} {\M,\sigma}$ is undefined if $x$ is not defined in $\sigma$, or if $\interp {x} {\sigma}$ points to a cyclic structure, and the $length$ function is defined with the obvious meaning in the class of $x$.}
{On the other hand,  the interpretation  is unknown, if any subterm of $e$ refers objects whose class is not defined in $\M$. For example, if the class of $x$ is not defined in $\M$, then $\interp {length(x)} {\M,\sigma}$ is unknown.} Undefined expressions remain undefined in the context of larger modules, but unknown expressions may become known. For example, consider a module $M'$ whichj contains a definition for t the class of $x$, then $\interp {length(x)} {\M'*M,\sigma}$ is no longer unknown; it may be defined or undefined.




\subsection{One-state assertions} We now define the syntax of one-state assertions, i.e. assertions whose
validity depends on {\em one} state.
{These assertions include the standard assertions, such as expressions $e$, comparisons $e>e'$, and the usual operators on expressions. They also support the usual boolean combinators, ie implication, negation etc.\footnote{All these may be encode through $\rightarrow$, the value
 \kw{false} and $\wedge$, so we only defined these.}
One state assertions also include predicates  specific to our concerns:} the assertions $\MayAffect$ and $\MayAccess$ which we   use to
model risk, the assertion $\sE:\prg{C}$ which expresses class membership, and the assertion
$\sE\obeys\prg{S} $ which expresses adherence to a specification. The two former  predicates
are {\em hypothetical}, in that they talk about the
  potential effect of execution of code, or of the existence of paths to connect two objects.
In particular, the  predicate $\MayAffect$   ascertains whether its first
parameter may execute code which affects the second one,  while
the predicate $\MayAccess$
%, and $\MayPublicAccess$
 ascertains whether
its first parameter has {\em any} path to the second one.


\noindent
\begin{definition}[One-state Assertions]
$ ~ $ \\

\begin{tabular}{lcll}
\syntax{\A} \ \  &   \BBC  &   \syntax{\sExpr}  ~\SOR~   \syntax{\sExpr} $\geq$ \syntax{\sExpr} ~\SOR~  ...
\\
& ~\SOR~ &  \syntax{Q}(\syntax{\sExpr}$^*$)
 \\
& ~\SOR~ &     \syntax{\A} $\wedge$ \syntax{\A} ~\SOR~
{\syntax{\A} $\rightarrow$ \syntax{\A}} \\
& ~\SOR~ &  $\exists \prg{x}.$\syntax{\A} ~\SOR~ $\forall \prg{x}.$\syntax{\A}
 \\
& ~\SOR~ & \syntax{\sExpr}\kw{:}\syntax{ClassId}
\\
& ~\SOR~ & $\MayAffect$ \lp\syntax{\sExpr},\syntax{\sExpr}\rp
\\
& ~\SOR~ & $\MayAccess$\lp\syntax{\sExpr},\syntax{\sExpr}\rp
\\
& ~\SOR~ & \syntax{\sExpr} \obeys \syntax{SpecId} \\
\\
 \syntax{PredDescr} \ \  &   \BBC  &  \kw{predicate}  \syntax{Q}\lp \syntax{ParId}$^*$ \rp\lb~   \syntax{\A}~ \rb
 \end{tabular}
\label{fig:syntax:functions}
\end{definition}

We next define  validity of one-state assertions through a judgement of the form
$\M,\sigma  \modelsWithO \A$,
which expresses that the assertion $\A$ holds for module $\M$, state $\sigma$ and using {the oracle \O.}\footnote{want better name than oracle.} We first define what is an oracle.
 
\begin{definition}
\label{def:orcacle}
\label{def:oracle}
A function \\
$\strut \ \ \ \ $ $\O : Module \times ClassId \ \longrightarrow \ \{ true, false \}$\\
 is called an {\em oracle} for module $\M$, if it is defined for exactly the classes defined in $\M$.  
 \begin{itemize}
\item  $\M \vdash \O\$ iff  $$dom(\O) = Classes(\M)$. 
\end{itemize}
\end{definition}
 
 

 The oracle $\O$ is needed to give meaning to the assertion $\M,\sigma  \models\withO e \obeys  \prg{S}$.  We appeal to \prg{C}, the class of the object indicated by $e$, 
and appeal to the oracle as to whether $\O(\M,\prg{C},\prg{S})$ holds.
%We do not define an algorithm for $\O$  
Later in   this paper,  cf. Def. \ref{def:C:adheres:to:A}, we define soundness of such oracles to means that whenever the oracle judges that a class \prg{C} indeed adheres to a specification \prg{S}, then all objects of this class behave according to \prg{S}.
%Moreover, we  argue that it is possible to develop such algorithms, and we outline how one can indeed ascertain that some classes adhere to their specs ....\footnote{We need to write thjis section} 

 Note that whether an object
obeys a spec depend on the state $\sigma$ only insofar as the class of the object is concerned, and not on the contents of the object's fields or any other fields. This may seem strange - surely an object cannot fullfill its obligations unless its state is well-formed? However, we are only concerned with reachable states, and in these, if the class has been implemented in a robust manner, the object's state will be appropriate\footnote{TODO explain this betterm as it is crucial.}

Note also, that judging that $\M,\sigma  \not\modelsWithO \sE \obeys  \prg{S}$ does not imply that there exists some witness 
that the object will have behaviour contradicting \prg{S}.
 In fact, it is even possible to have that a class's behaviour indeed adheres to the spec at a semantic level, but the oracle still returns $false$. 
 \footnote{ TODO expand. }This treatment allows us to avoid circularities in the definitions with relatively light-weight means, 
 as opposed to using notion like ... as in Devries  \footnote{ TODO expand.  TODO perhaps move some of this to the section ojn Hoare  Logic. }
 

\begin{definition}[Validity of one-state assertions] % --  $\MayAffect$ and $\MayAccess$] $ ~ $
\label{def:one-state-assertion:valid}
%{We assume a sound judgment of the form $\M  \vdash\,\prg{C}: \A $}.
Given an oracle $\O \subseteq Module \times ClassId$,   
the validity of an  assertion \syntax{\A}, is defined through the {\em partial} judgments:
$ ~ $
\\

\noindent
$ \  \ \ \models \ \subseteq \,  \syntax{Module} \times   \syntax{state} \times Oracle  \times  Assertion$
\\
$\strut \  \   \not\models \ \subseteq \,  \syntax{Module} \times   \syntax{state} \times Oracle  \times   Assertion 
% {\syntax{A}}
% SD: A is a metavariable, and Assertion is a set
$
\\

\noindent
using the notations  $\M,\sigma \modelsWithO \A$ and $\M,\sigma \not\modelsWithO \A$: \\


\begin{itemize}
 \item
 $\M,\sigma  \modelsWithO \syntax{\sE}$, \ \ \  if  $\interp{ \syntax{\sE}}{\M,\sigma}=\kw{true}$, \\
  {$\M,\sigma  \not\modelsWithO \syntax{\sE}$, \ \ \  if  $\interp{ \syntax{\sE}}{\M,\sigma}=\kw{false}$,\\
  $\M,\sigma  \not\modelsWithOQ \syntax{\sE}$, \ \ \ otherwise.}

   \item  $\M,\sigma  \modelsWithO \syntax{\sE_1}\geq\syntax{\sE_2}$, \ \ \  if $\interp{ \syntax{\sE_1}}{\M,\sigma}$
  $\geq$ $\interp{ \syntax{\sE_2}}{\M,\sigma}$, \\
  {$\M,\sigma  \not\modelsWithO \syntax{\sE}$, \ \ \ $\interp{ \syntax{\sE_1}}{\M,\sigma}$
  $<$ $\interp{ \syntax{\sE_2}}{\M,\sigma}$,
  \\
  undefined, \ \ \ \ if $\interp{ \syntax{\sE_1}}{\M,\sigma}$ or $\interp{ \syntax{\sE_1}}{\M,\sigma}$ is undefined, \\
  \hspace{1in} or not a number.
}
%  \item
% $\M,\sigma   \modelsWithO {\syntax{Q}(\syntax{\sE_1},...\syntax{\sE_n})}$ iff \\
%   $\M,\sigma  \modelsWithO {\syntax{A}[\syntax{\sE_1}/\syntax{p_1}, ... \syntax{\sE_n}/\syntax{p_n}]}$
% \\
% if  $ \M(Q) =  \kw{predicate} ~\syntax{Q}~\lp \syntax{p_1}...\syntax{p_n} \rp\lb~   \syntax{\A}~ \rb $,
 \item
 {
 $\M,\sigma   \modelsWithO {\syntax{Q}(\syntax{\sE_0},\syntax{\sE_1},...\syntax{\sE_n})}$, \ \ \ if \\
 $~$  \hspace{.3in} $\M,\sigma  \modelsWithO  \syntax{A}[\syntax{\sE_0}/\syntax{this,\syntax{\sE_1}/\syntax{p_1}, ... \syntax{\sE_n}/\syntax{p_n}]}$
 \\
  $\M,\sigma   \not\modelsWithO {\syntax{Q}(\syntax{\sE_0},\syntax{\sE_1},...\syntax{\sE_n})}$, \ \ \ if \\
 $~$  \hspace{.3in} $\M,\sigma  \not\modelsWithO  \syntax{A}[\syntax{\sE_0}/\syntax{this,\syntax{\sE_1}/\syntax{p_1}, ... \syntax{\sE_n}/\syntax{p_n}]}$
 \\
 if ${\cal P}(\M,\interp {\sE_0}{\sigma}\downarrow_1, \syntax{Q})   =  \kw{predicate} ~\syntax{Q}~\lp \syntax{p_1}...\syntax{p_n} \rp\lb~   \syntax{\A}~ \rb $,
 \\
  undefined, \ \ \ \ if ${\cal P}(\M,\interp {\sE_0}{\sigma}\downarrow_1, \syntax{Q})$ undefined,\\
   \hspace{1in} or if  $\M,\sigma   \modelsWithO {\syntax{A}(\syntax{\sE_0},\syntax{\sE_1},...\syntax{\sE_n})}$ undefined.
}

    \item
 $\M,\sigma  \modelsWithO \syntax{\A_1}\wedge\syntax{\A_2}$, \ \  if   $\M,\sigma  \modelsWithO \syntax{\A_1}$ and  $\M,\sigma  \modelsWithO \syntax{\A_2}$,
 \\
 {$\M,\sigma \not \modelsWithO \syntax{\A_1}\wedge\syntax{\A_2}$, \ \  if   $\M,\sigma  \not\modelsWithO \syntax{\A_1}$ or  $\M,\sigma  \not\modelsWithO \syntax{\A_2}$}
 \\
 undefined, if  $\M,\sigma  \modelsWithO \syntax{\A_1}$ or  $\M,\sigma  \modelsWithO \syntax{\A_2}$ is undefined.

      \item
$\M,\sigma  \modelsWithO \syntax{\A_1} \rightarrow \syntax{\A_2}$, \ \  if   $\M,\sigma  \modelsWithO \syntax{\A_1}$ and  $\M,\sigma  \modelsWithO \syntax{\A_2}$,\\
$~$ \hspace{1in} or $\M,\sigma  \not\modelsWithO \syntax{\A_1}$.
 \\
  $\M,\sigma  \not\modelsWithO \syntax{\A_1} \rightarrow \syntax{\A_2}$, \ \  if   $\M,\sigma  \modelsWithO \syntax{\A_1}$ and  $\M,\sigma  \not\modelsWithO \syntax{\A_2}$ ,
 \\
\sdO{$\M,\sigma  \modelsWithOQ \syntax{\A_1} \rightarrow \syntax{\A_2}$, \ \ otherwise}


 \item
  $\M,\sigma  \modelsWithO \exists \prg{x}.\syntax{\A}$ iff for some address $\iota$ and some {fresh} variable $\prg{z}{\!\in\!\syntax{VarId}}$,  we have $\M,\sigma[\prg{z}{\mapsto}\iota] \modelsWithO  \syntax{\A[\prg{z}/\prg{x}}]$.
  \\
{$\M,\sigma \not\modelsWithO \exists \prg{x}.\syntax{\A}$ iff for all address $\iota$ and  {fresh} variable  $\prg{z}{\!\in\!\syntax{VarId}}$,  we have $\M,\sigma[\prg{z}{\mapsto}\iota] \not\modelsWithO  \syntax{\A[\prg{z}/\prg{x}}]$.}
\\
$\M,\sigma  \modelsWithOQ \exists \prg{x}.\syntax{\A}$, otherwise.

  \item
  $\M,\sigma  \modelsWithO \forall \prg{x}.\syntax{\A}$ iff for all addresses $\iota\!\in dom\!(\sigma)$, and fresh variable $\prg{z}$, we have  $\M,\sigma[\prg{z}{\mapsto}\iota] \modelsWithO  \syntax{\A[\prg{z}/\prg{x}}]$.
  \\
{  $\M,\sigma  \not\modelsWithO \forall \prg{x}.\syntax{\A}$ iff there exists an address  $\iota\!\in dom\!(\sigma)$, and fresh variable $\prg{z}$, such that  $\M,\sigma[\prg{z}{\mapsto}\iota] \not\modelsWithO  \syntax{\A[\prg{z}/\prg{x}}]$.
  \\
  undefined, otherwise.}

  \item
 $\M,\sigma  \modelsWithO  \syntax{\sE}\kw{:}\prg{C}$, \ \  if $\sigma(\interp{ \syntax{\sE}}{\M,\sigma})\downarrow_1 = \prg{C}$.
 \\
 {
 $\M,\sigma  \not\modelsWithO  \syntax{\sE}\kw{:}\prg{C}$, \ \  if $\sigma(\interp{ \syntax{\sE}}{\M,\sigma})\downarrow_1 \neq \prg{C}$.
 \\
 $\M,\sigma \modelsWithOQ  \syntax{\sE}\kw{:}\prg{C}$, \  if $\interp{ \syntax{\sE}}{\M,\sigma}$ undefined, or\\
\strut \hspace{1in} $\sigma(\interp{ \syntax{\sE}}{\M,\sigma})\notin dom(\sigma\downarrow_2)$.}

 \item
$\M,\sigma  \modelsWithO \MayAffect \lp\syntax{\sE},\syntax{\sE'}\rp$, \ \ if
{$\interp {\prg{\sE} }{\M,\sigma}$ and $\interp {\prg{\sE'} }{\M,\sigma}$ are defined,} and there exists a method \prg{m}, arguments $\bar{\prg{a}}$, state $\sigma'$, identifier \prg{z}, such that
    $ \M, \sigma[\prg{z}\mapsto \interp {\syntax{\sE}} {\M,\sigma}], \syntax{z}\prg{.m}(\bar{\prg{a}}) \leadsto   \sigma'$, and   $\interp {\syntax{\sE'}} {\M,\sigma} \neq  \interp {\syntax{{\sE'}}} {\M,\sigma\downarrow_1,\sigma'\downarrow_1}    $.\\
 $\M,\sigma  \modelsWithO \MayAffect \lp\syntax{\sE},\syntax{\sE'}\rp$, \ \ undefined if
{$\interp {\prg{\sE} }{\M,\sigma}$ or $\interp {\prg{\sE'} }{\M,\sigma}$ are undefined}.
\\
{$\M,\sigma \not\modelsWithO \MayAffect \lp\syntax{\sE},\syntax{\sE'}\rp$, \ \ otherwise.}

\item
$\Prog{},\sigma \modelsWithO { \MayAccess}(\prg{\sE},\prg{\sE'})$,   \ if \  {$\interp {\prg{\sE} }{\M,\sigma}$ and $\interp {\prg{\sE'} }{\M,\sigma}$ are defined,} and there exist  fields  $\prg{f}_1$,... $\prg{f}_n$, such that
      $\interp{\prg{z}.\prg{f}_1...\prg{f}_n}{\M,\sigma[\prg{z}\mapsto \interp {\syntax{\sE}} {\M,\sigma}]}= \interp {\prg{\sE'} }{\M,\sigma}$.
\\
 $\M,\sigma  \modelsWithOQ  {\MayAccess}(\prg{\sE},\prg{\sE'})$, \ \  if
 $\interp {\prg{\sE} }{\M,\sigma}$  or $\interp {\prg{\sE'} }{\M,\sigma}$ \\
\strut \hspace{1in} is undefined.
\\
{$\M,\sigma \not\modelsWithO  {\MayAccess}(\prg{\sE},\prg{\sE'})$, \ \ otherwise.}

  \item
{$\M, \sigma  \modelsWithO \sE \obeys \prg{S}   $},  \  {if ${\cal S}(\prg{C},\M)= \prg{S}$ and $\O(\prg{C})$}
   \\
 {$\M, \sigma  \not\modelsWithO \sE \obeys \prg{S}   $},  \     {if ${\cal S}(\prg{C},\M)\neq \prg{S}$ or $\neg \O(\prg{C})$}
 \\
\strut \hspace{1in} where \prg{C}=${\cal C}lass(\sE,\sigma)$.  
  \\
{$\M, \sigma  \modelsWithOQ \sE \obeys \prg{S}   $},  \   if
  $\interp{ \syntax{\sE}}{\M,\sigma}$ undefined,  \\
\strut  \hspace{1in} or ${\cal C}lass(\sE,\sigma) \notin dom(\M)\cap dom(\O)$
%\SP  $\ \ \ \forall\, (\sigma,\code)\!\in\!\Arising(M).\ \forall  i\!\!\in\! \!\{1..n\}.$\\%  \wedge
%\SP  $\ \ \  \forall\,\sigma',\code'.\ (\sigma',\code')\!\in\!\Reach(M,\sigma,\code).$\\
%\SP  $\ \ \ \ \ \ \ \ \ \SP M, \sigma'[z\mapsto\interp{e}{\sigma}]  \modelsWithO \Policy_i[ z / {\kw{this}}]$ \\
%where $z$ is a fresh variable in $\sigma'$, and where
% we assume that $\PolSpecId$ was defined as \\ %  that   \PolSpecId was introduced by\\
%$ \ \prg{specification}\ \PolSpecId\  \{\ \Policy_1, ... \Policy_n\ \}$, \\

 \end{itemize}
 In the above, the  notation   {$\sigma[v{\mapsto} \iota]$} is shorthand for
$(\phi[v{\mapsto}\iota],\chi)$ for a state $\sigma = (\phi,\chi)$.
 \end{definition}

 {Note that the definition of  $\M, \sigma  \modelsWithO \sE \obeys \prg{S}   $   only depends on whether 
  \prg{C},    the {\em class} of the object denoted by $\sE$,  is considered by the oracle $\O$ to satisfy  \prg{S}. In particular, $\M, \sigma  \modelsWithO \sE \obeys \prg{S}   $  does not imply that $\sE$  satisfies the policies from  \prg{S}.  In order to have this guarantee we also need to know that $\O$ is sound - as defined in Def \ref{def:sound:oracle}.}

{Some examples to do with undefined assertions and preservation under linking can be found in section
\ref{problems:undefinedAssert}. In particular, in a state $\sigma$
where \prg{x} points to an object of class \prg{c}, and module $\M$
 where  class \prg{c} is undefined, and any specification identifier \prg{S}, we have that
$\M, \sigma  \modelsWithO \prg{x} \obeys \prg{S}$ is undefined, and
$\M, \sigma  \modelsWithO \prg{x} \obeys \prg{S} \rightarrow \kw{true}$ is also undefined.
 }

{However, the situation that the class of an object in the state is undefined in a module cannot
happen  in states that have been reached by execution of a module:}

\begin{lemma}
For all modules $\M$, all states $\sigma$, and specification identifiers \prg{S}:

\begin{itemize}
\item
{If $\sigma\in\Arising(\M)$  and $\prg{x}\in dom (\sigma)$, then \\  $\M, \sigma \modelsWithO \prg{x} \obeys \prg{S}$ is defined.}\\
\end{itemize}
\end{lemma}
\begin{proof}
We first prove that in any $\sigma\in\Arising(\M)$,  any object in
$\sigma$ has a class that is defined in $\M$. This can be proven
by induction on the execution. It holds because objects can only
be crated if $\M$ contains a class declaration for the respective class.
We then argue that under this condition the judgment $\M, \sigma \modelsWithO \prg{x} \obeys \prg{S}$ is well-defined for any \prg{S}, provided that
  \end{proof}

{Therefore, for any $\O$, if $\sigma\!\in\!\Arising(\M)$  and $\prg{x}\!\in\!dom (\sigma)$, then  either $\M, \sigma \modelsWithO \prg{x} \obeys \prg{S}$ or $\M, \sigma \not\modelsWithO \prg{x} \obeys \prg{S}$.
This is so, because we always either $\O(\M,\prg{C},\prg{S})=true$, or  $\O(\M,\prg{C},\prg{S})=false$.  }


{TODO Say somewhere that even though it is possible to construct sound $\O$s, the $\O$ is not unique.}

\subsection{Two state assertions} Two-state assertions\footnoteC{TODO add citation} allow us to compare properties of two different states, and thus say, e.g. that $\prg{p.balance}\post = \prg{p.balance}\pre + 10$. To differentiate between the two states we use the subscripts  \textsf{pre} and  \textsf{post}.

\noindent
\begin{definition}[Two-state Assertions]
$ ~ $ \\

% \begin{figure}[h]
\begin{tabular}{lcll}
 {\textit{t}} &   ::= & \textsf{pre} \ \     | \ \ \textsf{post}  \ \ | \ \ $\epsilon$
 \\
\syntax{\B} \ \  &   \BBC  &   \syntax{\A}$_{{\textit{t}}}$ \\
&  ~\SOR~  &\syntax{\sExpr}$_{\textit{t}}$ $\geq$ \syntax{\sExpr}$_{{\textit{t}}}$ ~\SOR~ ... \\
& \SOR & $\New(\syntax{\sExpr})$ \\
&  ~\SOR~  & \syntax{\B}  $\wedge$ \syntax{\B}  ~\SOR~ ... \\
& ~\SOR~ &  $\exists \prg{x}.$\syntax{\B}  ~\SOR~ $\forall \prg{x}.$\syntax{\B} . \\
 \end{tabular}
\label{fig:syntax:two:state:assert}
% \end{figure}
\end{definition}

\noindent
Given the syntax from above, we can express assertions like\\
 $ ~ $ \ \ $ \ \ \ \ $  $\forall \prg{p}.\prg{p}:\pre\prg{Purse}.$\\
$ ~ $ \ \ $ \ \ \ \ $ $[\ \prg{p.bank}\pre=\pre\prg{RBS} \rightarrow $\\
$ ~ $ \ \ $ \ \ \ \ \ \ \  \ $ $\prg{p.balance}\pre =\prg{p.balance}\post\  ]$,\\
 to require that the \prg{balance} of any \prg{Purse} belonging to \prg{RBS} is immutable across the to states.  
 
 \paragraph{Validity of one-state, two-state assertions, and policies}
We now define validity of two state assertions, ...

\begin{definition}
[Validity of Two-state assertions] We define the judgment \\
$
\begin{array}{c}
\begin{array}{lcl}
\ \ \  \models   &   \subseteq &   Module \times state \times state \\
& & \times\, Oracle\, \times {TwoStateAssertion}
\end{array}
\end{array}
$

\noindent
using the notation $\Prog{},\sigma,\sigma'\modelsWithO \B $\ as follows
\begin{itemize}
\item
$\Prog{},\sigma,\sigma'\modelsWithO \syntax{\A}_{\textit{t}}$
iff   $\Prog{},\sigma''\modelsWithO \syntax{\A}$, \\
where  $\sigma''=\sigma$ if  {{\textit{t}}}=\textsf{pre}, and  $\sigma''=\sigma'$ otherwise.
\item
$\Prog{},\sigma,\sigma'\modelsWithO \syntax{\sE}_{\textit{t}} \geq \sE'_{\textit{t'}}$,
    iff $\interp{\syntax{\sE}}{\M,\sigma_1} \geq \interp{\syntax{\sE'}}{\M,\sigma_2}$,\\
    where $\sigma_1=\sigma$ if  {{\textit{t}}}=\textsf{pre}, and  $\sigma_1=\sigma'$ otherwise,\\
    and $\sigma_2=\sigma$ if  {{\textit{t}}}$'$=\textsf{pre}, and  $\sigma_2=\sigma'$ otherwise.
\item
$\Prog{},\sigma,\sigma'\modelsWithO \New(\syntax{\sE})$
\ iff \
$\interp{\syntax{\sE}}{\M,\sigma'}\in dom(\sigma')\setminus dom(\sigma)$
\item
$\Prog{},\sigma,\sigma'\modelsWithO  \syntax{\B}_1   \wedge \syntax{\B}_2$   iff\\
 $\Prog{},\sigma,\sigma'\modelsWithO  \syntax{{\B}}_1 $
    and   $\Prog{},\sigma,\sigma'\modelsWithO  \syntax{{\B}}_2  $.
\item
  $\M,\sigma,{\sigma'}  \modelsWithO \exists \prg{x}.\syntax{\B}$ iff for some address $\iota$ and fresh variable $\prg{z}$,  we have $\M,\sigma[\prg{z}\mapsto\iota], {\sigma'[\prg{z}\mapsto\iota]} \modelsWithO  \syntax{\B[\prg{z}/\prg{x}}]$.
  \item
  $\M,\sigma,{\sigma'}  \modelsWithO \forall \prg{x}.\syntax{\B}$ iff for all addresses $\iota$ %\!\in dom\!(\sigma)$, 
  and fresh variable $\prg{z}$, we have $\M,\sigma[\prg{z}\!\mapsto\!\iota],\sigma'[\prg{z}\!\mapsto\!\iota]  \modelsWithO  \syntax{\B[\prg{z}/\prg{x}}]$.

\end{itemize}

\end{definition}

\noindent
  For example, for states $\sigma_1$, $\sigma_2$ where
$\interp  {\prg{x.balance}} {\sigma_1}  = 4$  and $\interp
{\prg{x.balance}} {\sigma2}  = 14$,
we have\\
%  $\Prog{},\sigma,\sigma' \not\modelsWithO \prg{x.balance}\PRE  \geq 10$, and
%  $\Prog{},\sigma_1,\sigma_2 \modelsWithO \prg{x.balance}\POST  \geq 10$.  And we would also have that
\noindent  $\Prog{},\sigma_1,\sigma_2\modelsWithO \prg{x.balance}\POST = \prg{x.balance}\PRE+10$.



\subsection{Policies and Specification} are expressed in terms of one-state assertions $\A$, $A'$, etc. and two state assertions $\B$, $\B''$ etc.\footnoteC{Should probably introduce this naming earlier on.}

Policies can have one of the three following forms:
1)~invariants of the form $\A$, which require that $\A$ holds at all visible states of a
program; or
2)\  $\A\, \lb \, \prg{code}\, \rb\, \B$, which require that execution of \prg{code} in any state which satisfies $\A$ will lead to a state
 which satisfies $\B$ wrt the original state\footnoteC{How else to talk of validity of two-state assertions?};
 or
 3)\ $\A\, \{\prg{any\_code}\}\, \B$ which, similar to two state invariants, requires that execution of {\em any} code in a state which satisfies $\A$ will lead to a state which satisfies $\B$.


\begin{definition}[Policies and Specifications]
\label{def:polAndSpec}
 $ ~ $ \\
 $
\begin{array}{lclcl}
Policy & \BBC & \ \A \ | \  \A \ \{ \prg{res}=\prg{this.m(par)} \}\  \B \ \\
 & | \ & \   \A \ \{ \prg{res}=\prg{new(x1,..xn)} \}\  \B \\
 & | \ & \ \ \A \ \{ \kw{any\_code} \}\ \B
\\
Specification & \  \BBC  & \  \prg{spec}\ SpcId \, \lb\, Policy^*\, \rb
\end{array}
$
\end{definition}

\footnote{Do we want to policies for constructors $ \A \ \{ \prg{res}=\prg{new(x1,..xn)} \}\  \B$ and do we still want to policy
$\A \ \{ \prg{any\_code} \}\ \B$ -- we can add both; the next chapter is incomplete wrt the two latter ones}
\footnoteC{TODO: Give examples of policies. Probably refer to the main part of paper where we have the policies already, Perhaps there are some examples further to those in the paper we would like to talk about}.

\subsection{Connectivity Properties}

A large amount of research into is concerned with properties of connectivity. For this we will first prove lemmas about connectivity.

\begin{lemma}[Only Connectivity Begets Connectivity for Existing Objects] For any module $\M$, oracle $\O$, frame $\phi$,
\label{lemma:connectivity-existing}\ \\
\begin{itemize}
\item  {If $\M, \phi \cdot \chi, \prg{stmts} \leadsto \phi' \cdot \chi'$ and if $\{x,z\}\subseteq dom(\chi)$, and
$\M, \phi' \cdot \chi' \modelsWithO   \MayAccess(x,z)$,  then\\
$\M, \phi \cdot \chi  \modelsWithO   \MayAccess(x,z)$ or
$\exists w, y\in dom(\chi) \cap dom(\phi).\
\M, \phi \cdot \chi \modelsWithO   \MayAccess(w,x) \land \MayAccess(y,z)$.}

\item {If $\M, \phi \cdot \chi, \mathit{rhs} \leadsto \chi'$ and if
$\{x,z\}\subseteq dom(\chi)$, and
$\M, \phi \cdot \chi' \modelsWithO   \MayAccess(x,z)$ then\\
$\M, \phi \cdot \chi  \modelsWithO   \MayAccess(x,z)$ or
$\exists w, y\in dom(\chi) \cap dom(\phi).\
\M, \phi \cdot \chi \modelsWithO   \MayAccess(w,x) \land \MayAccess(y,z)$.}

\item {If $\phi' \cdot \chi' \in \Reach(M,\phi \cdot \chi,\stmts)$
and if $\{x,z\} \subseteq dom(\chi)$, and
$\M, \phi' \cdot \chi' \modelsWithO   \MayAccess(x,z)$ then\\
$\M, \phi \cdot \chi  \modelsWithO   \MayAccess(x,z)$ or
$\exists w, y\in dom(\chi) \cap dom(\phi).\
\M, \phi \cdot \chi \modelsWithO   \MayAccess(w,x) \land \MayAccess(y,z)$.}

\end{itemize}
\end{lemma}

This lemma expresses a basic property of object-capability
systems that ``only connectivity begets connectivity''~\cite{MillerPhD},
{and says that the only way that some existing object~$z$ can become
accessible by some existing~$x$
through executing statements $\prg{stmts}$ or expression~$\mathit{rhs}$
is if each is already accessible to some~$y$ and~$w$ respectively that are
both in the current stack frame~$\phi$.}
\begin{proof}
{
The first two bullets we prove
by structural induction over the mutual inductive definitions of $\leadsto$
for statements and expressions. The final bullet is then proved by
structural induction over statements using the definition of $\Reach$.
}
\end{proof}

The following lemma expresses the only connectivity begets connectivity
property for newly created objects.

\begin{lemma}[Only Connectivity Begets Connectivity for New Objects]
\label{lemma:connectivity-new}\ \\
\begin{itemize}
\item {If $\M, \phi \cdot \chi, \prg{stmts} \leadsto \phi' \cdot \chi'$ and if $x \in dom(\chi)$ and $z \not\in dom(\chi)$, and
$\M, \phi' \cdot \chi' \modelsWithO   \MayAccess(x,z) \lor \MayAccess(z,x)$, then\\
$\exists w \in dom(\chi) \cap dom(\phi).\
\M, \phi \cdot \chi \modelsWithO   \MayAccess(w,x)$.}

\item {If $\M, \phi \cdot \chi, \mathit{rhs} \leadsto \chi'$ and if
$x \in dom(\chi)$ and $z \not\in dom(\chi)$, and
$\M, \phi \cdot \chi' \modelsWithO   \MayAccess(x,z) \lor \MayAccess(z,x)$, then\\
$\exists w \in dom(\chi) \cap dom(\phi).\
\M, \phi \cdot \chi \modelsWithO   \MayAccess(w,x)$.}


\item {If $\phi' \cdot \chi' \in \Reach(M,\phi \cdot \chi,\stmts)$
and if $x \in dom(\chi)$ and $z \not\in dom(\chi)$, and
$\M, \phi' \cdot \chi' \modelsWithO   \MayAccess(x,z) \lor \MayAccess(z,x)$, then\\
$\exists w \in dom(\chi) \cap dom(\phi).\
\M, \phi \cdot \chi \modelsWithO   \MayAccess(w,x)$.}

\end{itemize}
\end{lemma}

It says that if an existing object~$x$ gains access to an object
$z$ by executing $\stmts$ or $\mathit{rhs}$ in which $z$ is newly created, then
$x$ must have already been accessible to some objecct $w$ in the initial stack
frame~$\phi$.

\begin{proof}
Analogous to \autoref{lemma:connectivity-existing}.
\end{proof}








\input{MoreLinkingAndWellFormed}
\section{Linking and Well-formed Modules}
\label{sect:Linking}

\subsection{Well-formed, closed, and stratified modules}

In this section we define what it means for a module to be well-formed, closed and  stratified. Examples are given in sections \ref{problems:wellFormedModule} and \ref{problems:stratifiedModule}.

We first define some basic terms: A specification $\SPC$ {\em uses} another specification $\SPC'$, if the body of $\SPC$ mentions $\SPC'$.
A module $\M$ {\em uses} a  specification $\SPC$ if one of the specification defined in $\M$ uses   $\SPC$. A module implements a
specification $\SPC$ if one of the classes in $\M$ claims that is implements $\SPC$.

\begin{definition}
\label{def:uses}
\label{def:impls}
For a module $\M$ and specification identifiers $\SPC$ and $\SPC'$ we define
\begin{itemize}
   \item
   $\M \vdash \SPC \uses \SPC'$\ \ iff\ \  $\M(\SPC)$ contains a term of the form $e \obeys \SPC'$
   \item
   $\Muses {\M} {\SPC}$\ \ iff\ \  $\M \vdash \SPC' \uses \SPC$ for some $\SPC'$, or there exists a class $\CP$, such that ${\cal S}(\M,\CP)=\SPC$.
   \item
   $\M \vdash \SPC \usestr \SPC'$\ \ iff\ \  $\SPC$=$\SPC'$, or $\exists \SPC.\,\M \vdash \SPC'' \uses \SPC''\, \wedge \, \M \vdash \SPC'' \usestr \SPC'$.
   \item
   $\Mimpls {\M} {\SPC}$\ \ iff\ \   $\exists \CP.\,{\cal S}(\M,\CP)=\SPC$.
 \end{itemize}

\end{definition}

 We  define a  module to be {\em well-formed} if it contains descriptions for all the specification identifiers used in the module.

\begin{definition}
\label{def:well:formed:module}
We say that a module $\M$   is well-formed iff
\begin{itemize}
   \item
   $\forall \SPC.\ [\ \Muses {\M} {\SPC} \, \rightarrow\, {\SPC}\in dom(\M)\ ]$.
\end{itemize}
\end{definition}

\noindent
We can prove that   $\M$ is well-formed iff  $\forall \SPC,\SPC'.\  [\, \M \vdash \SPC \usestr \SPC'\ \rightarrow \SPC'\in dom(\M)\ ]$.


 Based on these concepts, we define {\em closed modules} to be those which contain implementations for all
 specifications used in  specifications  implemented in the module.

\begin{definition}
\label{def:closed}
For a module $\M$ we say that $\Closed{\M}$ iff for all ${\SPC}$ and ${\SPC}'$
\begin{itemize}
   \item
   $\Mimpls {\M} {\SPC} \,\wedge\, \M \vdash \SPC \uses \SPC'\ \ \rightarrow \ \ \Mimpls {\M} {\SPC'}$.
  \end{itemize}
\end{definition}

\noindent
We can easily show that $\Closed{\M}$ iff $\Mimpls {\M} {\SPC}$ and  $\M \vdash \SPC \usestr \SPC'$  then $\Mimpls {\M} {\SPC}$

We now define modules to be {\em stratified} if whenever the module implements
 specifications $\SPC$ and $\SPC'$ such that  $\SPC$ uses $\SPC'$, then it also implements all
 ``intermediate'' specifications, \ie those which are used by $\SPC$ and in their turn use  $\SPC'$.

\begin{definition}
\label{def:stratified}
For a module $\M$ we say that  $\Stratified{\M}$ iff
\begin{itemize}
   \item
   $\M$ is well-formed, and
   \item
   $\Mimpls {\M} {\SPC} \,\wedge\, \M \vdash \SPC \uses \SPC''\ \ \,\wedge\, \M \vdash \SPC'' \uses \SPC'
   \,\wedge\, \Mimpls {\M} {\SPC'} \ \ \rightarrow \ \ \Mimpls {\M} {\SPC''}$.


 \end{itemize}
\end{definition}

 \noindent
 We can show that in a stratified module, specifications which mutually use each other, are either both implemented in the  module, or both not implemented. That is, if $\Stratified{\M}$ then
$\Mimpls {\M} {\SPC}\, \wedge\,  \M \vdash \SPC \usestr \SPC' \, \wedge\,  \M \vdash \SPC \usestr \SPC'\ \ \rightarrow \ \ \Mimpls {\M} {\SPC'}$.

\begin{lemma}
\label{lemma:closed:is:Stratified}
For any module $\M$: $\Closed{\M} \ \rightarrow \ \Stratified{\M}.$
\end{lemma}

The opposite implication does not hold. A counterexample can be found in section \ref{problems:stratifiedModule}.





\subsection{Linking}

We can now define linking of modules. Linking is an operation that takes two modules, and creates a module which corresponds  to the union of the two. We use the concept of module linking 
to model the open world, where our module $\M$ whose code we know, will be executed together with further modules whose code we do not know.


We now define linking of modules, $M*M'$, to be the  union of their respective mappings, provided that the  domains of the two modules are disjoint wrt to classes, that the modules give the same definitions for specification identifiers, {and that linking preserves the privacy of the constructors of any classes which had been declared as private.}

\begin{definition}[Linking]
\label{formal:Linking}
Linking  of  modules $\M$ and $\M{'}$ is

\noindent $\ \ \ * \ : \ Module \times Module \ \longrightarrow \ Module  $
\noindent

$
\M * \M{'}  =\ \left\{
\begin{array}{l}
                        \M *_{aux} \M{'},\ \ \   \hbox{if}\  \ WFL(\M,\M{'})\\
\bot  \ \ \ \mbox{otherwise.}
\end{array}
                    \right.\\
                    \\
(\M *_{aux} \M{'})(\clss{}) =\ \left\{
\begin{array}{l}
 \M(id),\ \ \ \mbox{if }
\M(id) \mbox{ is defined}\\
\M{'}(id) \ \ \  \mbox{otherwise.}
\end{array}
                    \right.
                    $
\\
\\
{$WFL(\M,\M{'})$ }    \ $\equiv $ \\ 

\begin{itemize}
\item
$ ~ $   $dom(\M)\!\cap\!dom(\M{'})\!\cap\!\syntax{ClassId}\!=\!\emptyset\ \ \wedge\ $ 
\item $ ~ $   $\forall \prg{S}\in \syntax{SpecId}\!\cap\! dom(\M)\!\cap\! dom(\M{'}).\  \M(\prg{S})=\M(\prg{S}')  $ 
\item $ ~ $  $WFP(\M,\M') \ \wedge  \ WFP(\M',\M)$
\item
    $\Mimpls {\M_i} {\SPC} \ \rightarrow \Mimpls {\M_j} {\SPC}$ for $i\neq j\in \{1,2\}$
\item
   $\Mimpls {\M_1} {\SPC} \,\wedge\, \M_1 \vdash \SPC \uses \SPC'\ \   \ \ \rightarrow \ \ \neg \Mimpls {\M_2} {\SPC'}$.
 \end{itemize}  
 
{$WFP(\M,\M')$ } \ \   \ $\equiv $ \ \\

 \begin{itemize}
\item
 $ \forall \prg{C}.  \ \Private{\M}{\prg{C}} \ \rightarrow \ \ \neg \Creates{\M}{\prg{C}}$
  \end{itemize}  
\end{definition}



Add to the requirement that $\M_1$ and $\M_2$ implement disjoint specifications, and $\M_2$ does not implement any specification which is used by a specification implemented  in $\M_1$:
 
 
 
{In the above, the predicate $WFL(\M,\M{'})$ asserts that linking of the modules $\M$ and $\M{'}$ is well-defined. It requires that 1) classes are not defined more than once, 2) specifications may have been defined more than once, but then their bodies must be identical\footnote{We need to expand this, to also require that such "duplicate" specifications do not mention classes}, and 3) no module can call private constructors\footnote{Why nor also private methods?} from another module. This means, that the runtime system enforces this form of privacy.   For example, if in module $\M{_{mp}}$ we define \prg{Purse} as \kw{private} and \prg{Mint} as not private,
the call of the \prg{Purse} constructor is restricted to only insider the module $\M{_{mp}}$, while the call of \prg{Mint} is unrestricted.
This means, that only objects of classes defined in the module  $\M{_{mp}}$ may create \prg{Purse}s, while  clients of $\M{_{mp}}$ may create
objects of class \prg{Mint}. In effect, the creation of \prg{Mint} is publicly available, but the creation of  \prg{Purse}s is restricted}
We place some conditions for module linking to be defined: We require that the two modules do not contain implementations for the same class identifiers, and if they contain bodies for the same specification identifiers, then the bodies should be identical. Moreover, we require that the two modules do not contain implementations for the same specification, and that the second module should not contain implementation for specifications used by the first. The former two requirements are uncontroversial -- otherwise we would have conflicts between the to modules. The last two requirements are 
....


\subsection{Linking preserves properties}
\sdO{In this section we prove that linking preserves properties related to execution, validity and well-formedness.
Note however, that all these lemmas are predicated on  the linking operation being defined, that is
${\cal M}( \M,\prg{C},\prg{m}) $ =  ${\cal M}( \M'*\M,\prg{C},\prg{m})$ is a shorthand for
If ${\cal M}( \M,\prg{C},\prg{m})$ is defined, then  ${\cal M}( \M,\prg{C},\prg{m}) $ =  ${\cal M}( \M'*\M,\prg{C},\prg{m})$.
}

{The following lemma says that lookup is preserved by linking more modules.}

{
\begin{lemma}
\label{lemma:link:preserves:lookup}
For   modules $\M$ and $\M'$ such that and $\M'*\M$ is defined, and class identifier \prg{C}. method identifier \prg{m}, predicate identifier \prg{Q}, and ghost field identifier \prg{gf}, we have

\begin{itemize}
\item
%If ${\cal M}( \M,\prg{C},\prg{m})$ is defined, then
 ${\cal M}( \M,\prg{C},\prg{m}) $ =  ${\cal M}( \M'*\M,\prg{C},\prg{m})$.
\item
%If ${\cal S}( \M,\prg{C})$ is defined, then
 ${\cal S}( \M,\prg{C})$  =  ${\cal S}( \M'*\M,\prg{C})$.
\item
% If ${\cal P}( \M,\prg{C},\prg{Q})$ is defined, then
 ${\cal P}( \M,\prg{C},\prg{Q}) $ =  ${\cal P}( \M'*\M,\prg{C},\prg{Q})$.
\item
%If ${\cal G}( \M,\prg{C},\prg{gf})$ is defined, then
${\cal G}( \M,\prg{C},\prg{gf}) $ =  ${\cal G}( \M'*\M,\prg{C},\prg{gf})$.
\end{itemize}
\end{lemma}
}
{\begin{proof} By application of the definitions.
\end{proof}
}



The following lemma says that execution is preserved by linking more modules.

{
\begin{lemma}
\label{lemma:link:preserves:execution}
For all state $\sigma$ and $\sigma'$, modules $\M$ and $\M'$, and statements $stmts$, such that and $\M'*\M$ is defined:
\\
If $\M, \sigma, stmts \leadsto \sigma'$, then $\M'*\M, \sigma, stmts \leadsto \sigma'$.
\end{lemma}
}
{\begin{proof} By structural induction on the execution $\leadsto$ and lemma \ref{lemma:link:preserves:lookup}.
\end{proof}
}

{The value of a defined expression is preserved by linking. The same applies to undefinedness.}

{\begin{lemma}
\label{lemma:link:preserves:exprs}
For any   state $\sigma$, expression $e$, and modules $\M$ and $\M'$, where $\M'*\M$ is defined:\\
\begin{itemize}
\item
If  $\interp {e} {\M,\sigma}$ is defined,  then $\interp {e} {\M,\sigma}$=$\interp {e} {\M'*\M,\sigma}$.
\item
{If  $\interp {e} {\M,\sigma}$ is undefined,  then $\interp {e} {\M'*\M,\sigma}$ is undefined.}
\end{itemize}
\end{lemma}

\begin{proof}
By structural induction on $e$ and   lemma \ref{lemma:link:preserves:lookup}.
\end{proof}
}

 The opposite implication does not hold. It is possible to have, say  $\interp {e} {\M2*\M1,\sigma}$=$3$, but  $\interp {e} {\M1,\sigma}$ is undefined. Also, it is possible to have $\interp {e} {\M,\sigma}$ unknown,  but $\interp {e} {\M'*\M,\sigma}$   defined (or also undefined).

{Validity of one-state assertions is preserved by linking, as expressed by the lemma below:}

{\begin{lemma}
\label{lemma:link:preserves:exprs}
For any state $\sigma$, assertion $\A$, and modules $\M$ and $\M'$, where $\M'*\M$ is defined:\\
\begin{itemize}
\item
If  $\M,\sigma  \modelsWithO  \A$ holds, then  $\M'*\M,\sigma  \modelsWithO  \A$.
\item
If  $\M,\sigma  \not\modelsWithO  \A$ holds, then  $\M'*\M,\sigma  \not\modelsWithO  \A$.
\end{itemize}
\end{lemma}
\begin{proof}
By structural induction on $\A$ and   lemmas \ref{lemma:link:preserves:lookup} and \ref{lemma:link:preserves:exprs}. In several inductive cases we
use the fact that valictydity of the subterms must be defined, For example, in the case of implication,
i.e. when $\A$ has the form $\A_1 \rightarrow \A_2$ we  use the fact that validity of $\A_1$ must be defined.  \end{proof}




 \begin{lemma}[properties of linking]
 For any modules $\M_0$, and $\M_1$ and $\M_2$, such that $\M_1*\M_2$ is defined:
 \label{lemma:linking:properties}
 \begin{itemize}
     \item
    If $\M_1$ and $\M_2$ are stratified,   then $\M_1*\M_2$ is also stratified.
    \item
    If $\M_0*\M_1$ is defined,   then $\M_0*\M_1*\M_2$ is also stratified.
    \item
    $\M_1*(\M_2*\M_3)$=$(\M_1*\M_2)*\M_3$.
    \item
    If $\M_0*\M_1$ is closed,   then there exists an $\M_3$ such that $\M_3$ is closed and
    $\M_3*\M_0*\M_1*\M_2$ is also closed, and moreover $\M_3*\M_0*\M_1$=$\M_0*\M_1*\M_3$
    \item
    If $\M*\M'*\M''$ is closed,   then  $\M$  is closed and
    $\M*\M'$ is also closed.
 \end{itemize}

 \end{lemma}

 From now on, we implicitly expect all modules to be stratified.


\sdO{We require modules to be {\em stratified}, namely we avoid cyclic dependencies among  specifications in different modules.
In this section we define what it means for modules to be stratified, what it means to link modules, and describe the properties preserved by linking. }

\subsection{Good Oracle}
 An oracle is good if ...

 We fist define oracle concatenation in the obvious way:
 \begin{definition}[Oracle Concatenation]
 \label{def:good:oracle}
 For two oracles  $\O$ and $\O'$  their concatenation is undefined iff
  $dom(\O)\cap dom(\O') \neq \emptyset$
 and otherwise
 \begin{itemize}
      \item
      If $\CP\in dom(\O)$ then $\O*\O'(\CP)=\O(\CP)$
     \item
    If $\CP\in dom(\O')$ then $\O*\O'(\CP)=\O'(\CP)$
 \end{itemize}
 \end{definition}

 \begin{definition}[Good Oracle]
 \label{def:good:oracle}
 For a module $\M$ and an oracle $\O$, we say that $\M \models \O$ iff
 \begin{itemize}
     \item
      $dom(\O)=\Classes{\M}$
     \item
   If $\M$ is closed, then \ \
   $\forall \M'.\forall \sigma\in\Arising{\M*\M'}. \forall \CP\in \Classes{\M}. \forall Pol\in\Policies{\M}.$\\
     $\M,\sigma \modelsWithO x:\CP \rightarrow Pol[x/\this]$
    \item
    If $\M$ is not closed, then $\forall \M'$, such that $\M'*\M$ is closed, and for all
    $\O'$ such that $\M'\models\O'$.
    $\forall \M''.\forall \sigma\in\Arising{\M'*\M*\M''}. \forall \CP\in \Classes{\M}. \forall Pol\in\Policies{\M}.$\\
     $\M'*\M,\sigma \modelsWith{\O'*\O} x:\CP \rightarrow Pol[x/\this]$
 \end{itemize}
 \end{definition}

 We can prove that if $\M$ is not closed, then $\M \models \O$ iff $\M*'*\M \models \O'*\O$ for all
 $\M'$ such that $\M'*\M$ is closed, and $\O'$ such that $\M'\models\O'$.


  \begin{lemma}[properties of good oracles]
 For any modules  $\M$, $\M'$, $\M_1$, $\M_2$, and oracles $\O$, $\O_1$ and $\O_2$:
 \label{lemma:good:properties}
 \begin{itemize}
     \item
    If $\M \models \O$ then there exists a $O'$ such that $\M*\M'\models \O*\O'$
    \item
    If $\M_1 \models \O_1$ and $\M_2 \models \O_2$, then $\M_1*\M_2 \models \O_1*\O_2$.
 \end{itemize}
  \end{lemma}
  \begin{proof}
 \begin{itemize}
     \item
      If $\M*\M'$ is undefined, then the lemma holds trivially. If $\M*\M'$ is  defined, then take
  $\O'$ to be the oracle mapping every class to false.
    \item
    Again, if $\M_1*\M_2$ is undefined, then the lemma holds trivially. If $\M_1*\M_2$ is  defined, then
    use lemma \ref{lemma:linking:properties}
   \end{itemize}
  \end{proof}

  Does the opposite hold??



\HoareFigOne  

\section{Hoare Logic}

We define the Hoare Logic that allows us to prove adherence to
policies. In order to reflect that the code to be verified is executed
in an open system, and that it calls code whose specification and
trustworthiness is unknown to the code being verified, we augment the
Hoare triples, so that not only do they guarantee some property to
hold {\em after} execution of the code, but also guarantee that some
property is preserved {\em during} execution of the
code.



A Hoare tuple in our system has either the format\\
$ ~ \ \ \ \ \ \ \ \  \ \ \ \  $   \HoareExpl{\A} {\prg{stms}} {\M} {\A'} {\B},\\ 
 or the format\\
$ ~ \ \ \ \ \ \ \ \  \ \ \ \  $   \HoareExpl{\A} {\prg{stms}} {\M} {\B'} {\B},\\  
The former promises that execution of \prg{stms} in any
state which satisfies $A$~will lead to a state which satisfies $\A'$.
The latter promises that execution of \prg{stms} in any
state which satisfies $A$~will lead to a state where the relation of the old and new state is described by $\B$.
Both the former and latter tuples also promise that
%throughout execution of this \prg{stms}, 
the relation between the initial state, and any of the 
the intermediate states reached by execution of  \prg{stms} will be described by $\B$. 

The execution of \prg{stmts}  may call methods defined in \M, and the
predicates appearing in $\A$, $\A'$, and $\B$, may use  predicates
as defined in \M. 
 When the module \M\ is implicit from the context
 % , or ???\footnote{of no particular importance?} 
 we use the shorthand
\Hoare{\A} {\prg{stms}}  {\A'} {\B}. 

\sdJ{*** I commented out the stuff about the logical variables; I hope we can do without ***}

%As is usual in many Hoare logics~\cite{Kleymann:phd} we introduce {\em logical variables} into our assertions. We assume that
%these have the form \lvar, \lvar', and that they come from a separate domain. We also
% assume that there exists a function \Lvars, which returns all the
% logical variables within an assertion. For example $\Lvars(\prg{p1.balance}=\lvar ) =\{\,\lvar\,\}$.
% % \wedge \prg{p2.balance}>\lvar')=\{\lvar, \lvar'\}$
%\footnote{Make sure we have said earlier that \val stands for a value.\sd{Just noticed that I sometimes uses \prg{v} for variables, and some times for values. Arghh}}
% 
%\begin{definition}[Validity of Hoare Tuples]\label{defn:validity}
%$ $ \\
%
%  \begin{itemize}
%  \item
%$M  \models \HoareImpl{\A} {\prg{stms}} {\A'} {\B}  $ \ \ \  iff\\
%\SP $ \Lvars(\A)=\Lvars(\A')=\{\, \overline{\lvar}\, \}\ \wedge \ \forall \M',\sigma, \overline{\val}.$\\
%\SP\SP $ (\sigma,\_)\in \Arising(\M*\M')$\\
%\SP\SP $ \wedge\ \M*\M',  \sigma{[\overline{\lvar}\mapsto\overline{\val}]} \models \A$\\
%\SP\SP $ \wedge\  \M\!*\!\M', \sigma, \prg{stms}  \leadsto  \sigma'$\\
%\SP\SP\SP \ \ \ $\longrightarrow$ \ \ \ \\
%\SP\SP   $ \M\!*\!\M'\!, \sigma'{[\overline{\lvar}\mapsto\overline{\val}]} \models \A' $\\
%\SP\SP    $\wedge$ \\
%\SP\SP  $\forall {\sigma''}\!\in\!\Reach(M,\sigma,\stmts).\ \M\!*\!\M', \sigma, {\sigma''} \models \B   $
%  \item
%$M  \models  \HoareImpl{\A} {\prg{stms}} {\B'} {\B}  $ \ \ \  iff\\
%\SP $ \Lvars(\A)=\Lvars(\A')=\{\, \overline{\lvar}\, \}\ \wedge \ \forall \M',\sigma, \overline{\val}.$\\
%\SP\SP $ (\sigma,\_)\in \Arising(\M*\M')$\\
%\SP\SP $ \wedge\ \M*\M',  \sigma{[\overline{\lvar}\mapsto\overline{\val}]} \models \A$\\
%\SP\SP $ \wedge\  \M\!*\!\M', \sigma, \prg{stms}  \leadsto  \sigma'$\\
%\SP\SP\SP \ \ \ $\longrightarrow$ \ \ \ \\
%\SP\SP   $ \M\!*\!\M'\!, \sigma{[\overline{\lvar}\mapsto\overline{\val}]}, \sigma'{[\overline{\lvar}\mapsto\overline{\val}]} \models \B'$\\
%\SP\SP   $\wedge$ \\
%\SP\SP  $\forall {\sigma''}\!\in\!\Reach(M,\sigma,\stmts).\ \M\!*\!\M', \sigma, {\sigma''} \models \B   $
%\end{itemize}
%\end{definition}
%
%Note that the definition from above does not support the use of logical variables in the invariant part of the tuple, $\B$. Even though it would have been possible to accommodate for this in our formal model, it would slightly complicate the expositions, and so far we have not found a need to do that.
%
%
%\footnoteC{TODO:  Explain why we want both logical variables as well as two-state assertions, and why two state assertions cannot easily be encoded into logical variables.}

\subsection{Hoare Rules}

 We define the Hoare rules in figure
\ref{fig:HoareLogicBasic} for the language constructs, while in figure \ref{fig:HoareLogic} we give the rules for framing, the rules for consequence, and rules about invariants preserved during execution of a statement.\footnote{\sd{Notice that we have no rule for object creation; these would like rules for method calls; while they do  not pose special challenges, they would increase the size of our system and we leave this to further work.}}

% \noindent 
We first consider the rules from figure \ref{fig:HoareLogicBasic}:
 The rules  \ruleN{VarAsg}  and  \ruleN{FieldAsg} are not surpising. The annotations $\_$\pre\ and $\_$\post\ explain the use of $\prg{a}\pre$, and allow us to talk in the postcondition about values in the pre-state. For example, we would obtain \\
 $\HoareNL {\kw {true}}  {\prg{this.f=this.f+3}} {\prg{this.f}=\prg{this.f}\pre+3} {{\kw {true}}} {}$.

The rules  \ruleN{Cond-1} and \ruleN{Cond-2} describe conditional statements, and are standard.

\forget{
The rule  \ruleN{NewObj} guarantees that the fields of the new object will contain the values of the arguments as read in the old state, and that the new object will belong to class \prg{C}. More importantly, it also guarantees that the new object will satisfy   the specification of \prg{C} as given in module \M. Note that in contrast to rule here we are in a ``closed'' world. We know therefore that the new object will obey the specification of its class, as given in the module. 
\footnoteC{\sd{SUPERIMPORTANT: This rule is unsound! We either need to add constructors into the specification, and check their preconditions when we create a new object, and also check the constructors when we check that a module is well-formed, or drop object creation altogether. I have left the rule and the stuff in for the time being. Note that we do not need this rule for the Escrow proof, but we would need it if we were to prove Purse, but Purse is currently beyond our reach.}}
}

 The   rule   \ruleN{meth-call-1}  describes method call. \footnoteC{   ... TODO explain that we use one possible pair of Pre/Post conditions, explain the replacements, } \footnote{We have no invariant part in the spec of a method,  but it would not be difficult to extend the system to support this.}
%I have not changed this, as it would be too much work. But perhaps we should.}



 On the other hand, rule   \ruleN{meth-call-2} is unusual in a Hoare logic setting; it expresses that  ``only connectivity
begets connectivity'' . The terms was coined by Mark Miller\footnoteC{add citation}  and is used widely in the capabilities literature. To our knowledge, this property has not been expressed in a Hoare logic. The reason, is, we believe, that Hoare logics so far have been developed with the closed world assumption, in the sense that all methods (or functions) called come from code which has a specification, and which has been verified.  


\HoareFigTwo

The rule  \ruleN{Frame-methCall} is also unusual; note that its precondition is \kw{true}. This means that we make no assumptions about the receiver of the method call; this allows us to reason in an {\em open} setting.  Even though we do not know what the behaviour method \prg{m} will be, we still have some conditions which can guarantee  that $A'$ will be preserved. These conditions are that anything that was accessible from the receiver \prg{x} or argument of \prg{z} at the time of the method call, or anything that is newly created during execution of the method body,  does not satisfy the prerequisites necessary to affect $A'$.\footnote{Notes that  $\sigma'\!\in\!\Reach(M,\sigma,\stmts)$ is a shorthand for $\sigma'.  (\sigma',\_')\!\in\!\Reach(M,\sigma,\stmts)$. }
% ofPerhaps we have done so already in earlier chapters.}

 The last rule in figure \ref{fig:HoareLogicBasic} is  \ruleN{Sequence}. \sd{It
%  requires that the first premise uses the format    $\Hoare{\A}{\prg{stms}_1}{\A'}{\B'}$ 
% TODO say why this is necessary. TODO  say how rule  \ruleN{Cons-1} is useful here. 
requires that the precondition  and the postcondition of the first statements, \ie $\A$\, and $\B_1$, imply the precondition of the second statements, ie $\A_2$, and that the combined effects described by the two-state assertion in the postconditions of $\stmts_1$ and $\stmts_2$, $\B_1$\, followed by $\B_2$, imply the postcondition of the  sequence, \ie $\B$. }

\sd{The standard entailment,  \ie $\A\, \rightarrow_\M\, \A'$,   guarantees that any state which satisfies $\A$\, also satisfies $\A'$. We extend the notion to cater for two state assertions, and have three new forms of entailment, described in Definition \ref{def:entail}. The requirement  $\A,\B_1\, \rightarrow_\M\, {\bf true}, \A_2$ guarantees that for any pair of states if the former states satisfies $\A$ and the two together satisfy $\B_1$, then the second state will also satisfy $\A_2$, \cf  Definition \ref{def:entail}.\ref{entailPQPP}. The requirement $\B_1,\B_2 \rightarrow_\M\, \B$ guarantees for any three states, if the first two together satisfy $\B_1$, and the second and third together satisfy $\B_2$, then the first and third will satisfy $\B$,  \cf  Definition \ref{def:entail}.\ref{entailQQQ}.  For example, with \ref{def:entail}.\ref{entailPQPP} we have $\prg{x}=5,\prg{x}\post=\prg{x}+2 \rightarrow_\M  \kw{true},\prg{x}=7$, while with  \ref{def:entail}.\ref{entailQQQ} we have $\prg{x}\post=\prg{x}+4,\prg{x}\post=\prg{x}+2 \rightarrow_\M  \prg{x}\post=\prg{x}+6$ for any module $\M$.}

\sdJ{\begin{definition}[Entailment]
\label{def:entail}
$ $ \\

\begin{enumerate} 
\item
\label{entailPP}
$\A\, \rightarrow_\M\, \A'$\ \  iff\ \\
$\forall   \M'. \forall (\_,\sigma)\in \Arising(\M*\M').$ \\
 $\M*\M', \sigma \models  \A$ \ implies \ $\M*\M',  \sigma \models  \A'$.
\item
\label{entailQQ}
$\B\, \rightarrow_\M\, \B'$\ \  iff\ \\
$\forall   \M'. \forall (\_,\sigma),(\_,\sigma')\in \Arising(\M*\M').$ \\
$\M*\M', \sigma, \sigma' \models  \toby{\B}$ \ implies \   $\M*\M', \sigma, \sigma' \models  \B'$.
\item
\label{entailPQPP}
$\A, \B \rightarrow_\M \A',\A''$\ \  iff\ \\
$\forall   \M'. \forall (\_,\sigma),(\_,\sigma')\in \Arising(\M*\M').$ \\
$\M*\M', \sigma\models \A\ \wedge\ \M*\M', \sigma,\sigma' \models \B $ \ 
implies \ $ \M*\M',\sigma \models \A' \ \wedge\   \M*\M',\sigma' \models \A''$ 
\item
\label{entailPPQ}
$\A, \A' \rightarrow_\M \B$\ \  iff\ \\
$\forall   \M'. \forall (\_,\sigma),(\_,\sigma')\in \Arising(\M*\M').$ \\
$\M*\M',\sigma\models \A\ \wedge\  \M*\M',\sigma' \models \A'$ \ implies \ $\M*\M',\sigma,\sigma' \models \B$. 
\item
\label{entailQQQ}
$\B, \B' \rightarrow_\M \B''$ \ \  iff\ \\
$\forall   \M'. \forall (\_,\sigma),(\_,\sigma')\in \Arising(\M*\M').$ \\
$\M*\M',\sigma,\sigma' \models \B\ \wedge\   \M*\M'\sigma',\sigma'' \models \B'$ implies $\M*\M',\sigma,\sigma'' \models \B''$ 
\end{enumerate}
\end{definition}
}

\sdJ{Note that $\A\, \rightarrow_\M\, \A'$ is equivalent with 
$\M \models  \A \rightarrow \A'$. TODO: what about the rest? Can they be expressed more succinctly?
}


We now turn our attention to the structural\footnoteC{\sd{to check that this is the term}} rules from figure   \ref{fig:HoareLogic}. 

Rule  \ruleN{Frame-General} allows us to frame onto a tuple any assertion that has not been affected by the code. \footnoteC{TODO: cite reynolds' paper for this}. For this, we  need two notions of some code being disjoint from an assertion:

\begin{definition}[Disjointness]\label{defn:disjointness}
$ $ \\
\begin{itemize}
\item
\sd{$\M,\sigma \models \prg{stms} \ddisj \A$ iff }\\
$\M,\sigma  \models \A$ \  $\wedge$ \  $\forall \sigma'\in\Reach(\M,\stmts,\sigma). \ \M,\sigma'  \models \A$.

\item
$\M,\sigma \models \prg{stms} \disj \A$ iff \\
$\M,\sigma  \models \A$ \  $\wedge$ \  $\M,\sigma, \prg{stms} \leadsto \sigma'$ \ \ $\rightarrow$ \ \ $\M,\sigma'  \models \A$.
\end{itemize}
\end{definition}

For example $\prg{x=7}\disj\prg{x:=x+1; x:=x-1}$ holds for all states and modules,  but  $\prg{x=7}\ddisj\prg{x:=x+1; x:=x-1}$ never holds. In general, framing is an undecidable problem, but we can prove some very basic properties, eg that assignment to a variable does not affect all other variables, nor other paths. Note, that in order to express this property we are making use of logical variables.

\begin{lemma}
For all modules \M, and states $\sigma$,

\begin{itemize}
\item
If \prg{x} and \prg{y} are textually different variables, then\\
$\M,\sigma \models \prg{x=a}\ddisj \prg{y}:=\prg{a'}$.
\footnoteC{\sd{Thank you for taking care of :=, Toby. And I think you have pushed it throughout rest of appendix. WOW!}}
\item
\sd{If \prg{x} is not a prefix of the path \prg{p}, then} \\
$\M,\sigma \models \prg{p.f=a}\ddisj \prg{x}:=\prg{a'}$ \footnoteC{\sd{Need to make sure that logical variable are part of \prg{a}}}. 
\item
\sd{If $\M,\sigma \models \prg{stms} \ddisj \A$ then  $\M,\sigma \models \prg{stms}  \disj \A$.}
\end{itemize}
\end{lemma}

The rule  \ruleN{Conj} allows us to combine different Hoare tuples for the same code, and follows standard Hoare logics.

 Interestingly, our system has {\em four} rules of consequence. The fist rule, \ruleN{Cons-1}, is largely standard, as it allows us to strengthen the precondition \A, and weaken the postcondition \B, and invariant $\B'$. A novelty of this rule, however, is that it allows the invariant to be conjoined to the postcondition; this is sound, because the invariant is guaranteed to hold throughout execution of the code, and thus also after it. 

\sd{For \ruleN{Cons-1} we use the entailment  $\A\, \rightarrow_\M\, \A'$, which guarantees that any state which satisfied $\A$\, also satisfies $\A'$, and that of the form  $\B\, \rightarrow_\M\, \B'$ which guarantees that any pair of states which together satisfy $\B$ also satisfy $\B'$. This is described in Definition \ref{def:entail}.}


 The next rule,   \ruleN{Cons-2}, is unusual, in that it allows us to {\em weaken} the precondition, while adding a hypothesis $\B'$ to the postcondition, such that the original postcondition, $\B$, is only guaranteed if $\B'$ holds. The rule is sound, because we also require that the new precondition $\A'$ together with the new postcondition $\B'$ guarantee that the original precondition holds in the pre-state.   
The judgment $\A,\B\, \rightarrow_\M\, \A',\A''$ is defined in  in Definition \ref{def:entail}. 
For example, we can use this rule to take\\
$\HoareNL {\prg{p1}\, \obeys\, Purse} {\prg{p2:=p1.sprout}} {\prg{p2}\, \obeys\, Purse}  {\kw{true}}$\\ and deduce that\\
$\HoareNL {\kw{true}} {\prg{p2:=p1.sprout}} {\prg{p1}\pre\, \obeys\, Purse \rightarrow \prg{p2}\, \obeys\, Purse}  {\kw{true}}$.

The next two rules, \ruleN{Cons-3} and \ruleN{Cons-4}, allow us to swap between tuples
 where the postcondition is a one-state assertion, \ie $\Hoare {\A} {\prg{stms}} {\A'} {\B'}$ and
that where the postcondtion is a one state assertion, \ie $\Hoare {\A} {\prg{stms}} {\B} {\B'}$. 


% perhaps some part of following text to be resurrected later
% We have the standard definitions of entailment of the form  $\A\, \rightarrow_\M\, \A'$ and $\toby{\B'}\, \rightarrow_\M\, \B$\, but we also need an unusual definition of entailment, namely: \sd{The third entailment from below, ie $\A, \B\wedge\B' \rightarrow_\M \A',\A''$ allows us to turn an assertion of the form   $\Hoare{\A}{\prg{stms}}{ \B}{\B'}$ to an assertion of the from  $\Hoare{\A}{\prg{stms}}{ \A'}{\B'}$, as shown in rule ??? in figure ??? . Moreover, the fourth entailment, ie ... TODO complete all this}






The following lemma is an example entailment.
\begin{lemma} For all modules $\M$:
$ ~ $

\noindent$ \MayAccess(\prg{x},\prg{y}) \wedge \MayAccess({\prg{y},\prg{z}})  \rightarrow_\M \ \MayAccess(\prg{x},\prg{z}) $.

\end{lemma} 

The two last rules in \ref{fig:HoareLogic} are concerned with adherence to specification.

The rule \ruleN{Code-Invar-1} expresses that throughout execution of any code, in all intermediate states, for any variable \prg{x} for which we know that it \obeys a specification $S$, we know that it satisfies any of $S$'s stated policies. 

The rule \ruleN{Code-Invar-2} guarantees that any term $e$ which has been shown to be pointing to an object which \obeys a specification $S$ will continue satisfying the specification throughout execution of any \prg{stms}. 


\subsection{Soundness}

We first demonstrate that judgments made in the context of a module are preserved when we link a larger module. 
In lemma \ref{lemma:linkentail}, we state that entailment is preserved by linking:

\begin{lemma}
\label{lemma:linkentail}
$ $ \\

\begin{itemize}
\item
$\A\, \rightarrow_\M\, \A'$\ implies that\  $\A  \rightarrow_{\M*\M'}\, \A'$.
\item
$\B\, \rightarrow_\M\, \B'$\ implies that\  $\B\, \rightarrow_{\M*\M'}\,  \B'$
 \item
$\A,\A'\, \rightarrow_\M\, \B$\ implies that\  $\A,\A'\, \rightarrow_{\M*\M'}\,  \B$
\item
$\B,\B'\, \rightarrow_\M\, \B''$\ implies that\  $\B,\B'\, \rightarrow_{\M*\M'}\,  \B''$
\end{itemize}
\end{lemma}

In lemma \ref{lemma:hl:preserve}  we state that  derivability and validity of Hoare tuples is preserved for larger modules
\begin{theorem}[Linking preserves derivations and validity]
\label{lemma:hl:preserve}
For all modules $\M$, $\M'$.

\begin{itemize}
\item
If $\HoareExpl {\A} {\prg{stms}}  {\M} {\A'} {\B}$ , then\\ $\HoareExpl {\A} {\prg{stms}}  {\M\!*\!\M'} {\A'} {\B}$.
\item
{If $\M\models \HoareImpl {\A} {\prg{stms}}  {\A'} {\B}$, then \\
$\M\!*\!\M'\models \HoareImpl {\A} {\prg{stms}}  {\A'} {\B}$}\footnoteC{\sd{Toby, this was what the second bullet was meant to say}  }
\end{itemize}
\end{theorem}

We now define what it means for a method body, and a class definition to adhere to its specification

We say that a method \prg{m} defined a class \prg{C} adheres to is specification, \\
\SP\SP $\M \vdash \prg{C},\prg{m}$\\
if we able to show that the body of \prg{m} when executed in a state that satisfies \A,  the difference between the initial and final state is described by \B, and will preserve \B', 
 where \A\ and \B' \  and \B\ are the method's pre,   postcondition, and invariant.
 Moreover, we say that a class adheres to its specification\\
\SP\SP $\M \vdash \prg{C}$\\
 of all its methods adhere to their specification.
 Finally, a module adheres to its specification, \\
\SP\SP $\M \vdash \M$\\ 
if all the classes in \M adhere to their specifications. \footnoteC{\sd{Toby you wrote: \toby{This is phrased as a definition. But
doesn't proving a module's
adherence to a specification involve some kind of inductive argument?
It feels to me like it is some kind of inductive argument that would justify
adding the $\wedge \kw{this} \obeys Spec$ precondition as, from my
understanding, this precondition is added to capture the idea that when
proving the correctness of method \texttt{m} we implicitly assume correctness
of all of \texttt{m}'s callees, right?} Does the new Def. read better?}}

 
  
\begin{definition}[Proving code's adherence to specification]
~ \\

\begin{itemize}
\item
$\M, \prg{C} \vdash\, \A \,\lb \, \prg{this.m(par)}\, \rb\, \B$ \SP iff    \\
we can prove that\\
 $\HoareExpl  {\A\, \wedge\,  \sdJ{\kw{this} : \prg{C}}}  {\prg{stmts}} {\M} {\B[\prg{a}/\prg{res}]} {\sd{\kw{true}} }$ %  {\B'}$
\\
where\\
  $ \Meths {} {\prg{C}} {\prg{m}}$ = 
$\kw{method} \ \prg{m}\lp  \prg{par}  \rp\ \lb\,    \prg{stmts}\semi\, \kw{return}\, \prg{a}\, \rb$. 

\item
$\M, \prg{C} \vdash\, \A $ \SP iff    \\
forall $\M'$, %if $\M * \M'$ is defined, then, 
and for all $(\sigma,\code)\in \Arising(\M'*\M)$,\footnote{Note that is $\M * \M'$ is undefined, then the set 
 $\Arising(\M*\M')$ is empty, and the assertion is trivially satisfied.}
\\
\sdJ{$\M*\M', \sigma \models\kw{this} : \prg{C}  \rightarrow \A$}

\item
$\M, \prg{C} \vdash\, \prg{S}$  \SP iff    \\
for all policies $Pol\in \M{} (\prg{S})$,  \footnote{Perhaps this needs to be expressed better?} 
we have\ \ $\M, \prg{C} \vdash\, Pol$
 
\item
$\M \vdash \prg{C}$ \SP iff  \\
for all \prg{S}, with  $\M{} (\prg{C}) $ = $\kw{class}\,\prg{C}\,\kw{satisfies}\, ..., \prg{S}, ...\, \lb ... \rb$, we have that\ \ \ 
 $\M, \prg{C} \vdash \prg{S}$.
 \item
$  \vdash \M$ iff\\
 $\M \vdash \prg{C}\ \ $  
  for all classes \prg{C} from \M
\end{itemize}

\end{definition}

\sdJ{*** Very important: 2nd bullet point above. This one worries me. ****}

Below we are defining and proving the soundness of our Hoare logic. \scd{Note that we do not require that  $\M \vdash \M$, because
we do not model object creation. If we had object creation in our system, we would have needed that requirement, and the proof of soundness would have required slightly more complex proof techniques such as a generation lemma, or double induction.}

\begin{lemma}
\label{lemma:connectivity}
If $\M, \sigma, \prg{stmts} \leadsto \sigma'$ and if $\z\in dom(\sigma)$, and 
$\M, \sigma' \models \MayAccess(..)$ then
$\M, \sigma  \models \MayAccess(..)$ or ....
\end{lemma}

This lemma expresses the basic axiom of object-capability
systems that ``only connectivity begets connectivity''~\cite{MillerPhD},
\begin{proof}
By structural induction over the derivation of $\M, \sigma, \prg{stmts} \leadsto \sigma'$.
\end{proof}

\begin{theorem}[Soundness of the Hoare Logic]
\label{lemma:hl:soundness}
 For any modules  \M and $\M'$, code \prg{stms}, assertions $\A$, $\A'$ and $\B$ and $\B'$. 
If 
\begin{enumerate}
\item   
$\vdash \M$, and 
% \item If \ $\HoareExpl {\A} {\prg{stms}}{\M}  {\A'} {\B}$\\ then  $\M \models \HoareImpl {\A} {\prg{stms}} {\A'} {\B}$.
\item   \ $\HoareExpl {\A} {\prg{stms}}{\M}  {\B'} {\B}$, and 
\item
\sdJ{$\M, \sigma \models \A$}, and 
\item
\sdJ{$\M * \M', \sigma, \prg{stms} \leadsto \sigma'$}
\end{enumerate}
then
\begin{enumerate} 
\item
$\M, \sigma, \sigma' \models \B'$,  and
\item
$\forall {\sigma''}\!\in\!\Reach(M,\sigma,\stmts).\ \M\!*\!\M', \sigma, {\sigma''} \models \B   $
\end{enumerate}
\end{theorem}

\sdJ{Note that  the first and second requirement above only talk of module $\M$ which has been verified, and
which is used to prove the tuple $\HoareExpl {\A} {\prg{stms}}{\M}  {\B'} {\B}$. However execution of the \prg{stmts} is in the context of the linked program $\M'$, and validity of the assertion $\A$ is again wrt to both $\M$ and $\M'$ -- ACTUALLY SD NOT SURE ABOUT THE LATTER} 
\begin{proof}
We fix the modules $\M$ and $\M'$. 
 
\sdJ{The proof proceeds by well-founded induction.
We define a well-founded ordering $\prec$ which orders tuples of states, statements, one-state assertions, and two two-state assertions, ie\\
$\mbox{ } \ \ \prec\ \  \subseteq\  \ (\ state\ \times\ \syntax{Stmts}\ \times\ \syntax{OneStateAssert}\ \times$
\\
$  \mbox{ } \hspace{.49in}  \syntax{TwoStateAssert}\  \times\ \syntax{TwoStateAssert} \ )^2
$
\\
This ordering $\prec$ is the 
 smallest relation which satisfies the following two requirements\footnote{need to express better}
\\
For all $\sigma$, $\sigma'$ $\prg{stmts}$, $\prg{stmts'}$,$\A$, $\B$, $\B'$,  $\A'$, $\B''$, $\B'''$:
\\
If $\M*\M', \sigma, \prg{stmts} \leadsto \sigma''$ in fewer steps than $\M*\M', \sigma', \prg{stmts}' \leadsto \sigma'''$\footnote{This should be expressed better, but is clear}, then \\
$  \mbox{ } \hspace{.2in} (\sigma, \prg{stmts}, \A, \B, \B') \prec (\sigma', \prg{stmts'}, \A', \B'', \B''')$
\\
If the proof of $\HoareExpl {\A} {\prg{stms}}{\M}  {\B} {\B'}$ requires the proof of 
$\HoareExpl {\A'} {\prg{stms}}{\M}  {\B''} {\B'''}$ through one of the steps from Figure 4\footnote{DANGEROUS, need to know that we cannot introduce cycles! but should be doable}, then\\
 $  \mbox{ } \hspace{.2in}  (\sigma, \prg{stmts}, \A, \B, \B') \prec (\sigma, \prg{stmts}, \A', \B'', \B''')$
\\
We now argue that the relation is well-founded, ie there are no cycles. *** some work here *** 
}
 
 

We proceed by case analysis on the last step in the derivation of $\HoareExpl {\A} {\prg{stms}}{\M}  {\B'} {\B}$.
\begin{description}
%\item[1st Case] $\prec$ because of a shorter execution. We now apply a case analysis on the last step from the Hoare logic to obtain  $\HoareExpl {\A} {\prg{stms}}{\M}  {\B'} {\B}$
%
%\begin{description}
\item[Case] \ruleN{varAsg}, \ruleN{fieldAsg}, \ruleN{fieldAsg}, \ruleN{cond-1} and \ruleN{cond-2} all
 follow %trivially 
 %SD: Never say that somthing is trivial - except for the validity of $false \rightarrow A$
 from the operational semantics of \LangOO; the latter two cases also require application of the induction hypothesis. \item[Case] \ruleN{meth-call-1}. 
 \sdJ{This gives that \\
 5. $\prg{stmts}$ has the form\ $ \prg{v:=x.m(y)}$, and that \\
 6. $\A$ $\equiv$
 $x \obeys \prg{S} \wedge \A'[\prg{x}/\prg{this}, \prg{y}/\prg{par}]$, where \\
 7.   $\M(S)$ $=$ $\kw{specification}\, \prg{S}\, \lb ..., \A' \lb \prg{this.m(par)} \B'', ... \rb$, and where\\
8. $\B$ $\equiv$
  $ \B''[\prg{x}/\prg{this}, \prg{y}/\prg{par}, \prg{v}/\prg{res}]$, and\\
 9.  $\B'$ $\equiv$ $\kw{true}$. \\
From 5. and the operational semantics we obtain that \\
10. $\M*\M', \phi'\cdot\chi, \prg{stmts'} \leadsto \sigma''$, where  \\
11. $\Meths {} {\prg{C}} {\prg{m}} = .. \lb\, \prg{stmts'}; \kw{return}\, a\, \rb $,   and\\
 12. $\sigma'=\sigma''[\prg{v} \mapsto \interp {a} {\sigma''}]$ .
... More steps here ...\\
From 6. and 3., and by definition ???, we obtain that\\
yy. $\sigma(x)\downarrow_1=\prg{C}$, and \\
vv. \prg{C} is defined to satisfy \prg{S}.\\
From 7, vv, and because of 1. we also obtain that\\
zz. $\HoareExpl {\prg{this}:\prg{C} \wedge \A} {\prg{stms'}}{\M}  {\B''[\prg{a}/\prg{res}]} {\kw{true}}$\\
From 10, and .. we obtain that\\
uu. $(\phi'\cdot\chi,\prg{stms''}, x \obeys \prg{S} \wedge \A'[\prg{x}/\prg{this}, \prg{y}/\prg{par}], ..., ...)
 \prec (\sigma, {\prg{stms}}, {\A}, {\B'},{\B}) $.\\
Therefore, by application of inductive hypothesis, we obtain 
... more here ..}

\item[Case] \ruleN{meth-call-2} 
\sdJ{Follows from lemma \ref{lemma:connectivity}.}

\item[Case] \ruleN{frame-methCall} 
\sdJ{needs work}
%Is similar to \ruleN{meth-call-2} in
%that it expresses a basic axiom of object-capability languages, namely that
%in order to cause some visible effect, one must have access to an object able
%to perform the effect. Coupled with ``only connectivity begets connectivity'',
%this implies that a method can cause some effect only if the caller has
%(transitive) access to some object able to cause the effect (including
%perhaps the callee).
\item[Case] \ruleN{sequence} follows from the definition of \linebreak
$\Reach(M,\sigma,\prg{code$_1$; code$_2$})$ and the definition of validity of
Hoare tuples (\autoref{defn:validity}).
\item[Case] \ruleN{frame-general} 
Follows by the definition of $\disj$ and $\ddisj$.
%\end{description}
%\item[2nd Case] $\prec$ because of a structural rule. Therefore, ....
%\begin{description}
\item[Case] \ruleN{cons-1} follows from the definition of entailment
(\autoref{def:entail}) and the fact that\\ $(\sigma,\prg{stms}) \in \Reach(M,\sigma,\prg{stms})$.
\item[Case] \ruleN{cons-2} follows because $\sigma,\sigma' \models Q' \rightarrow Q$ if and only iff $\sigma,\sigma' \models Q$ assuming $\sigma,\sigma' \models Q'$.
\item[Case] \ruleN{cons-3} and \ruleN{cons-4} follow straightforwardly from
the definition of entailment and Hoare tuple validity.
\item[Case] \ruleN{code-invar-1} follows because the definition of policy
satisfaction for one-state-assertions~$\A$ requires that $\A$ holds for all
internally-reachable states~$\sigma'$ via $\Reach$.
\item[Case] \ruleN{code-invar-2} follows straightforwardly from the
definition of Hoare tuple validity and 2-state-assertion validity.
%\end{description}
\end{description}
  \end{proof}




 


 




\section{\tobym{Illuminating Examples}}
\label{examples}
\toby{Here we consider a suite of examples that shed light on the
various subtle aspects of our approach.}

\subsection{Well-formed and Closed Modules}
\label{problems:wellFormedModule}

Consider the following module definition.

 \begin{lstlisting}
  module M$_1$
       $\equiv$
  specification S$_2${
     field f
     policy Pol$_2$
         this.f obeys S_1
  }
\end{lstlisting}
\noindent
Module M$_1$ is closed, but not well-formed, because it messes a definition for $\SPC_1$, used by $\SPC_2$.

If we add the definition of $\SPC_2$, we make the module well-formed:

\begin{lstlisting}
  module M$_2$
       $\equiv$
  specification S$_1${
     field g
     policy Pol$_1$
         this.g > 4
  }
  specification S$_2${
     ... $\mbox{as earlier}$ ...
  }
\end{lstlisting}
Module M$_1$ is closed and well-formed.
 
 If we add an implementation for $\SPC_2$ (and not $\SPC_1$) then the module will   not be closed:

\begin{lstlisting}
  module M$_3$
       $\equiv$
  specification S$_1${
     ... $\mbox{as earlier}$ ... }
  specification S$_2${
     ... $\mbox{as earlier}$ ...  }
  class C$_2$ implements S$_2$ {
     ...
  }
\end{lstlisting}
\noindent 
Module $\M_3$ is well-formed but not closed.

But if we add an implementation for $\SPC_1$    then the module will be    closed (regardless of whether we also have an implementation of  $\SPC_2$):

\begin{lstlisting}
  module M$_4$
       $\equiv$
  specification S$_1${
     ... $\mbox{as earlier}$ ...  }
  specification S$_2${
     ... $\mbox{as earlier}$ ... }
  class C$_1$ implements S$_1$ {
     ...
  }
\end{lstlisting}
\noindent 
Module $\M_4$ is well-formed and closed.

 
\subsection{Stratified Modules}
\label{problems:stratifiedModule}

Consider the following module definition, where $\M_1 \vdash \SPC_2 \uses \SPC_2$, and 
$\M_1 \vdash \SPC_3 \uses \SPC_1$.

  \begin{lstlisting}
  module M$_1$
       $\equiv$
  specification S$_1${
     field f
     policy Pol$_1$
         this.f > 5
  specification S$_2${
      ... x  obeys S_1  ... }
  specification S$_3${
      ... y obeys S_2 ... 
  class C$_1$ implements S_$1$ { 
      ...
  }
  class C$_{1a}$ implements S_$1$ { 
      ...
  }      
\end{lstlisting}
\noindent
In \M$_1$ we implement $\SPC_1$, and therefore \M$_1$ is well-formed,  closed, and stratified.


Now consider what happens if we implement $\SPC_3$ without implementing $\SPC_2$ which is used by $\SPC_3$.

 \begin{lstlisting}
  module M$_2$
       $\equiv$
  specification S$_1${
     field f
     policy Pol$_1$
         this.f > 5
  specification S$_2${
      ... x  obeys S_1  ... }
  specification S$_3${
      ... y obeys S$_2$ ... 
  class C$_3$ implements S$_3$ { 
      ...
  }
\end{lstlisting}
\noindent
 \M$_2$ is well-formed, stratified, but not  closed.

Now consider  \M$_3$, where we implement $\SPC_1$ and $\SPC_3$ but not $\SPC_2$.

 \begin{lstlisting}
  module M$_3$
       $\equiv$
  specification S$_1${
     ... $\mbox{as earlier}$  ... }  
  specification S$_2${
      ... $\mbox{as earlier}$  ... }
  specification S$_3${
      ... $\mbox{as earlier}$  ... }
  class C$_1$ implements S$_1$ { 
      ...
  }
  class C$_3$ implements S$_3$ { 
      ...
  }      
\end{lstlisting}
\noindent
 Therefore M$_3$ is well-formed, not closed, and not stratified.

As we shall see later, there exists a module which we can link to $\M_3$ and obtain a closed module, while it is impossible to do that with $\M_2$.

Finally, in $\M_4$ we introduce a cycle between $\SPC_1$, $\SPC_2$, and $\SPC_3$, and only implement $\SPC_1$: 
\begin{lstlisting}
  module M$_4$
       $\equiv$
  specification S$_1${
     ... x obeys S$_3$ ... }
  specification S$_2${
      ... x obeys S$_1$ ... }
  specification S$_3${
      ... x obeys S$_2$ ... }
  class C$_1$ implements S_$1$ { 
      ...
  }      
\end{lstlisting}
\noindent
Module M$_4$ is well-formed, not closed, and not stratified.

\subsection{Problems with stratified linking}
\label{problems:stratified:problems}
Stratified linking gives several very nice properties, but it creates problems, as explained below.
Consider the following three modules

 \begin{lstlisting}
  module M$_1$
       $\equiv$
  specification S$_1${  }
  class C$_1$ satisfies S$_1${  }
  
  module M$_2$
       $\equiv$
  specification S$_1${  }
  class C$_2$ satisfies S$_2${  }
  
  module M$_3$
       $\equiv$
  specification S$_1${  }
  class C$_1$ satisfies S$_1${  }
  class C$_2$ satisfies S$_2${  }
  \end{lstlisting}
  
\noindent
Consider oracles $\O_1$,  $\O_3$ such that\\
$\strut ~ \ \ \ \O_1=\CP_1 \mapsto true$ while $\O_3=\CP_1 \mapsto true, \CP_2 \mapsto true$
\\
Both oracles are sound in the respective modules, ie $\M_1 \models \O_1$ and $\M_3 \models \O_3$


We then have that $\M_1 \modelsWith{\O_1} x \obeys \SPC_1 \rightarrow x:\CP_1$, but
$\M_3 \not\modelsWith{\O_3} x \obeys \SPC_1 \rightarrow x:\CP_1$, which is a bit wierd.
Currently there is no problem with that, because the linking of $\M_1$ and $\M_2$ is undefined. 
But this in not a satisfactory
solutions, because it would forbid me to argue that an implementation of $\SPC$ behaves as promised in the context of other
implementations of $\SPC$. 

For example,   if I prove that $\CP_{purse\_a}$ developed in $\M_a$ indeed implements $\prg{Purse}$, and
I have another module $\M_b$, which provides a  $\CP_{purse\_b}$ also proven to implement   $\prg{Purse}$, then
how can I argue that  $\CP_{purse\_a}$ upholds all promises if linked with  $\M_b$? 

So, in the long term we will need to make linking less restrictive. To do this, we will need to either restrict assertions 
so that they do not allow $\obeys$ in {\em some} negative positions (we have to be careful with that, because we want to allow, e.g.
that $p \obeys \prg{Purse} \rightarrow p.balance>0$), or we will have to make the meaning of $\_ \modelsWith{\O} \_$ more
demanding, so that t $\M_1 \not\modelsWith{\O_1} x \obeys \SPC_1 \rightarrow x:\CP_1$

Perhaps we have to bite the bullet, and forbid $\obeys$ in {\em all} negative positions. And instead, only allow unfolding of the 
specifications, e.g. Say that $\A\, \rightarrow_{\M}\, \A'$ iff ... $\A \wedge \A''$ where $\A''$ unfolds all the specifications?

\subsection{Unknown Assertions}
THIS SUBSECTION PROBABLY HAS TO BE SCRAPPED.
\label{problems:undefinedAssert}
Consider the following module definition.

 \begin{lstlisting}
  module M$_1$
       $\equiv$
  specification S$_1${
    }
  class C$_1$ satisfies S$_1${
       }
  module M$_2$
       $\equiv$
  class C$_2$ satisfies S$_1${
      }
  specification S$_1${
    }
  }
\end{lstlisting}

\noindent
Consider now some mappings $\O_1$,  $\O_2$ such that\\
  $~ \ \ \ \ \  $  $\O_1(\MP_1,\CP_1,\SPC_1)=true$, \ \ and \\  $~ \ \ \ \ \  $  $\O_1(\MP_1*\MP_2,\CP_1,\SPC_1)=false$, \ \
 and\\
 $~ \ \ \ \ \  $  $\O_2(\MP_1,\CP_1,\SPC_1)=false$, \ \  and\\   $~ \ \ \ \ \  $  $\O_2(\MP_1*\MP_2,\CP_1,\SPC_1)=true$.\\
Neither $\O_1$, nor $\O_2$ are oracles, since they do not satisfy the requirements from Definition \ref{def:orcacle}.

Now onsider now some mappings $\O_3$,  $\O_4$ such that\\
 $~ \ \ \ \ \  $  $\O_3(\MP_1*\M',\CP_1,\SPC_1)=true$  for all $\M'$;\footnote{We assume that there also exist ab empty module, so as to also obtain $\O_3(\MP_1,\CP_1,\SPC_1)=true$}\\
$~ \ \ \ \ \ $  $\O_3(\MP_2*\M',\CP_2,\SPC_1)$ for all $\M'$;\\
 $~ \ \ \ \ \ $ $\O_3(\_,\_,\_)=false$, otherwise.
\\
$~ \ \ \ \ \ $ $\O_4(\MP_1*\M',\CP_1,\SPC_1)=false$  for all $\M'$;
\\ $~ \ \ \ \ \ $ $\O_3(\MP_2*\M',\CP_2,\SPC_2)=true$ for all $\M'$;
\\$~ \ \ \ \ \ $  $\O_4(\_,\_,\_)=false$, otherwise.
\\
Both $\O_3$ and $\O_4$ are oracles.
Note that  $\O_4$  treats $\MP_1$, $\CP_1$ , and $\SPC_1$ differently from
$\MP_2$, $\CP_2$ , and $\SPC_2$, even though they are structurally isomorphic. Nevertheless, $\O_4$ is
an oracle.  Moreover, both $\O_3$ and $\O_4$ are {\em sound} oracles.

We now consider the application of these oracles to judge validity.
Assume a state $\sigma_1$ such that ${\cal C}lass(x)_{\sigma_1}=\prg{C}_1$,
and ${\cal C}lass(x)_{\sigma_1}=\prg{C}_2$.
Then we have that \\
$~ \strut \ \ \ $ $\M_1, \sigma_1 \modelsWith{\O_4} x \obeys \prg{S}_1$, \ \ while\\
$~ \strut \ \ \ $  $\M_1, \sigma_1 \modelsWith{\O_4} y \obeys \prg{S}_1$ is unknown.
\footnote{If we adopted  Alex's clever definition however,  we would have $\M_1, \sigma_1 \models y \obeys \prg{S}_1 \rightarrow  x \obeys \prg{S}_1$ holds.-- Still to think whether we want that.}
\\
On the other hand:\\
$~ \strut \ \ \ $ $\M_2*\M_1, \sigma_1 \modelsWith{{\O_3}} x \obeys \prg{S}_1 \rightarrow y \obeys \prg{S}_1$, \ \ \ but \\
$~ \strut \ \ \ $  $\M_2*\M_1, \sigma_1 \not\modelsWith{{\O_4} }x \obeys \prg{S}_1 \rightarrow y \obeys \prg{S}_1$.\\ Therefore,
we have that\\
 $~ \strut \ \ \ $ $\models  x \obeys \prg{S}_1 \not\longrightarrow_{\M_2*\M_1} y \obeys \prg{S}_1$.


\subsection{Infinite execution}

Consider the following module definition.

 \begin{lstlisting}
  module M$_1$
       $\equiv$
  specification S$_1${
     policy P$_1$:
        true
           { this.m() }
        false
  }
  class C$_1$ satisfies S$_1${
      method m(){
          this.m()
      }
  }
\end{lstlisting}

Specification $\prg{S}_1$ is clearly nonsensical, as it asserts the
unsatisfiable postcondition \prg{false}. However, perhaps surprisingly,
 a mapping  $\O_1$ such that $\O_1(\MP_1*\M',\CP_1,\SPC_1)=true$ for all modules $\M'$,
and $\O_1(\_,\_,\_)=false$, otherwise, is an oracle, and moreover is a {\em sound} oracle!
 This circularity is not
a soundness issue,   since \prg{m()} never terminates.
Our   logic only requires \emph{partial correctness}.

\subsection{Contradiction in the Specification}

Our specification language allows one to express apparently contradictory
specifications, such as the following.

 \begin{lstlisting}
  module M$_2$
       $\equiv$
  specification S$_2${
     policy P$_2$:
        $\neg$ this $\obeys$ S$_2$
  }
  class C$_2$ satisfies S$_2${
      ...
  }
\end{lstlisting}

Note that while class $\prg{C}_2$ claims to satisfy $\prg{S}_2$, the
contradiction in policy $\prg{P}_2$ means that we would be very surprised if there was a possibility to satisfy it.
 Consider any oracle $\O_1$ such that $\O_1(\MP_2,\CP_2,\SPC_2)=true$ -- such an oracle is {\em not} sound.

Here is why: Assume a state $\sigma_1$ such that ${\cal C}lass(x)_{\sigma_1}=\prg{C}_2$, and assume that
$\O_1$ was sound.
Then, by definition of $\models$, we obtain that $\M_2, \sigma_1 \modelsWith{\O_1}  x \obeys \prg{S}_2 $. Because $\O_1$ is sound, we can unfold the definition of $\prg{S}_2$, and ontain $\M_2, \sigma_1 \not\modelsWith{\O_1}  x \obeys \prg{S}_2 $.
This is a contradiction, and therefore, $\O_1$ is not sound.

Note that we do not obtain a contradiction if we have an oracle  $\O_2$ such that  $\O_2(\MP_2,\CP_2,\SPC_2)=false$. Namely, we are not allowed the fold the definition, and thus we cannot obtain that $\M_2, \sigma_1 \modelsWith{\O_2}  x \obeys \prg{S}_2 $.

%
%Note that $o \obeys \prg{S}_2$ implies  that $\neg(o \obeys \prg{S}_2)$, but the opposite does not hold, ie
% $\neg(o \obeys \prg{S}_2)$ does not imply that $o \obeys \prg{S}_2$.
% This is so because $\obeys$ is always judged syntactically, i.e. wrt the
%specifications that the class of the $o$ claims to satisfy and not wrt the behaviour of $o$.
%
%Therefore, the case analysis $o \obeys \prg{S}_2 \ \vee\ \neg(o \obeys \prg{S}_2)$
%does not introduce a problem.
%Namely, the case analysis   is equivalent to ``class of $o$ claims it satisfies   $\prg{S}_2$,
%or  class of $o$ does not claim it   satisfies   $\prg{S}_2$.''
%Also, by application of the implications from the previous paragraph,
%$o \obeys \prg{S}_2 \ \vee\ \neg(o \obeys \prg{S}_2)$ implies  (but is not equivalent to) $\neg(o \obeys \prg{S}_2 )\ \vee\ \neg(o \obeys \prg{S}_2)$.

\subsection{Monotonicity and Invariants}

Consider the following module $\prg{M}_3$ that declares a single
specification~$\prg{S}_3$
and class~$\prg{C}_4$ that claims to satisfy $\prg{S}_3$.

 \begin{lstlisting}
  module M$_3$
       $\equiv$
  specification S$_3${ }
  specification S$_4$
  	policy Pol_1
	       $\forall$ o. o:Object. o obeys S$_3$
  }
  class C$_3$ satisfies S$_3${
      ...
  }
\end{lstlisting}

In the context of this module \emph{alone} all objects will be of class
$\prg{C}_4$, and so each will obey $\prg{S}_3$. However, the assertion
$\prg{M}_3 \modelsWithO  \prg{C}_4 :\forall o:Object.\ o\ \obeys\ \prg{S}_3$ does \emph{not} hold for {\em any} oracle $\O$. This is so, because \autoref{def:invariant} requires that the assertion holds in all configuration arising from all possible extensions of $\prg{M}_3$, by linking
$\prg{M}_3$ against all possible $\M'$. Naturally some of these $\M'$ not contain the definition of  $\prg{S}_3$, and by \autoref{def:oracle}  the oracle $\O$ will say that $\O(\M,\prg{C})=false$ for any class \prg{C} defined in $\M'$.

\subsection{Negative Positions}

Quite often\footnote{TODO find citations}, we need to restrict the assertions appearing in negative positions. But  this is not necessary here. Nevertheless, in general, when $\obeys$ appears in a negative position in the postcondition of a method specification, the specification is unsatisfiable.

In the example below,    $\prg{x}  \obeys \SPC{_4}$  appears in a negative position.

 \begin{lstlisting}
  module M$_4$
       $\equiv$
  specification S$_4${
       policy Pol$_{4a}$
           true
               { res=this.m(x) }
           x $\obeys$ S$_4$ $\longrightarrow$ res

       policy Pol$_{4b}$
           true
               { res=this.m(x) }
           x $\obeys$ S$_4$ $\longleftarrow$ res
  }
  class C$_4$ satisfies S$_4${
      method m(x){ ... }
  }
\end{lstlisting}

In general, it is impossible to satisfy  $\prg{Pol}_{4a}$ unless the method \prg{m} always returns true. Namely, this specification requires
that the method should recognize all objects which satisfy $\SPC_4$. Since there are many classes that can satisfy this specification, and not  all such classes are known to the module $\M_4$, it is impossible to write a method body for \prg{m} such that it satisfies this requirment.

Policy $\prg{Pol}_{4b}$ is even more difficult to satisfy.

\vspace{.1in}

However, in the below we   strengthen the specification  $\SPC{_4}$ from he example above, so that $\prg{Pol}_{4a}$ became satisfiable.
Namely, we added the policy $\prg{Pol}_{4c}$ which requires that objects which satisfy  $\prg{Spec}_4$ return $true$ when executing method \prg{check()}. On the other hand, policy  $\prg{Pol}_{4b}$ remains unsatisfiable.


\begin{lstlisting}
  module M$_{4a}$
       $\equiv$
  specification S$_4${
       policy Pol$_{4a}$
           true
               { res=this.m(x) }
           x $\obeys$ S$_4$ $\longrightarrow$ res

       policy Pol$_{4b}$
           true
               { res=this.m(x) }
           x $\obeys$ S$_4$ $\longleftarrow$ res

       policy Pol$_{4c}$
           true
               { res=this.test() }
          res
  }
  class C$_4$ satisfies S$_4${
      method m(x){ z=x.check(); return z }
  }
\end{lstlisting}



\vspace{.1in}
On the other hand, the following spec is satisfiable (in fact, we have similar policies for Purse-s). Here we have an abstract predicate \prg{AP} on the right hand side of the implication. Since the abstract predicate will be made concrete by the module, it can be defined in such a way as to make it possible to satisfy the specification. Note that the predicate \prg{AP} is a  {\em binary} relation, while $\obeys$ is unary.
\begin{lstlisting}
  module M$_5$
       $\equiv$
  specification S$_5${
       abstract predicate AP(o1,o2)

       policy Pol$_5$
           this.AP(o) $\rightarrow$ o $\obeys$ S$_5$

       policy Pol$_6$
           true
               { res this.m(x) }
          this.AP(o)  $\longleftrightarrow$ res
  }
  class C$_5$ satisfies S$_5${
      ...
  }
\end{lstlisting}

\subsection{Cyclic Definitions}


 \begin{lstlisting}
  module M$_6$
       $\equiv$
  specification S$_6${
       field next
       policy Pol$_6$
           this.next$\neq$ null $\rightarrow$ this.next $\obeys$ S$_7$
  }
  specification S$_7${
       field next
       policy Pol$_7$
           this.next$\neq$ null $\rightarrow$ this.next $\obeys$ S$_6$
  }
  private class C$_6$ satisfies S$_6$ {
      field next
      method m( ){ this.next = new C$_7$(null) }
  }
  private class C$_7$ satisfies S$_7$ {
      field next
      method m( ){ this.next = new C$_6$(null) }
  }
  class D{
    ... new C$_6$(null)
    ...
  }
\end{lstlisting}

The specifications $\SPC{_6}$ and  $\SPC{_7}$ are  cyclic.
An oracle $\O$  such that $\O(M_6, \prg{C}_6,\prg{S}_6)=true$ is sound only if
  in each reachable state $\sigma$, if ${\cal C}lass (x)_\sigma)=\prg{C}_6$ and
$\interp{x.next}{\sigma}\neq {\kw{ null}}$, then
$\O(M_6, \prg{C},\prg{S}_7)=true$ where
$\prg{C}={\cal C}lass (\prg{x.next})_\sigma$.
 In the particular case, given the code of the classes $\prg{C}_6$, this
amounts to requiring that $\O(M_6, \prg{C}_7,\prg{S}_7)$.


 \begin{lstlisting}
  module M$_8$
       $\equiv$
  specification S$_8${
       policy Pol$_8$
           $\forall$x. x$\,\obeys\,$S$_9$  $\rightarrow$ x$\,\obeys\,$S$_9$
  }
  specification S$_9${
       policy Pol$_9$
        ... $\mbox{some unsatisfiable requirement}$ ....
  }
  class C$_{surprise}$ satisfies S$_8$ {
  }
\end{lstlisting}

Perhaps unsurprisingly, there exists a sound oracles $\O$, such that
\ \  $\O(\MP_8,\CP_{surprise},\SPC_8)=true$,









%
%\onecolumn
\section{June 2016 EscrowProof2}


\begin{figure*}[htb]
\begin{lstlisting}
method dealV2(  ) // returns Boolean
{
  //setup and validate Money purses
  var escrowMoney := sellerMoney.sprout
  var res := escrowMoney.deposit(0, sellerMoney)
  if (!res) then {return false}
  res := buyerMoney.deposit(0, escrowMoney)
  if (!res) then {return false}
  res := escrowMoney.deposit(0, buyerMoney)
  if (!res) then {return false}

  //setup and validate Goods purses
  var escrowGoods := buyerGoods.sprout
  res := escrowGoods.deposit(0, buyerGoods)
  if (!res) then {return false}
  res := sellerGoods.deposit(0, escrowGoods)
  if (!res) then {return false}
  res := escrowGoods.deposit(0, sellerGoods)
  if (!res) then {return false}

  // start the actual exchange
  res := escrowMoney.deposit(price, buyerMoney)
  if (!res) then {return false}
  res := escrowGoods.deposit(amt, sellerGoods)
  if (!res) then {
    // undo the transaction
    buyerMoney.deposit(price, escrowMoney)
    return false}
  // transfer from the two escrows to two accounts
  sellerMoney.deposit(price, escrowMoney)
  buyerGoods.deposit(amt, escrowGoods)

  return true
}
\end{lstlisting}
%\vspace*{-7mm}
\caption{Consolidated i.e.\ full length version of the \prg{deal\_V2} method}
\label{fig:DealV2Consolidated}
\end{figure*}


\begin{figure*}[htb]
\begin{lstlisting}
method deal( DOWEWANTARGUMENTS )
{
  //setup and validate Money purses
  var escrowMoney := sellerMoney.sprout
  var res := escrowMoney.deposit(0, sellerMoney)
  if (res) then  {
    res :=  buyerMoney.deposit(0, escrowMoney)
    if (res) then  {
      res := escrowMoney.deposit(0, buyerMoney)
      if (res) then  {

        // set up and validate Goods purses
        var escrowGoods := buyerGoods.sprout
        res := escrowGoods.deposit(0, buyerGoods)
        if (res) then  {
          res := sellerGoods.deposit(0, escrowGoods)
          if (res) then  {
            res :=  escrowGoods.deposit(0, sellerGoods)
            if (res) then  {

              // start the actual exchange
              res := escrowMoney.deposit(price, buyerMoney)
              if (res) then  {
                res := escrowGoods.deposit(amt, sellerGoods)
                if (!res) then  {
                  // undo the transaction
                  buyerMoney.deposit(price, escrowMoney)
                } else {
                  // transfer from the two escrows to two accounts
                  sellerMoney.deposit(price, escrowMoney)
                  buyerGoods.deposit(amt, escrowGoods)
                }
              } else skip
            } else skip
          } else skip
        } else skip
      } else skip
    } else skip
  }
  return res
}
\end{lstlisting}
%\vspace*{-7mm}
\caption{Revised \prg{deal} method expressed without {\prg{return}} statements}
\label{fig:DealV3}
\end{figure*}

\newcommand{\bothfigs}{fig.~\ref{fig:DealV2Consolidated} or \ref{fig:DealV3}}

\newpage

\begin{figure*}[htb]
\begin{lstlisting}[escapechar=@]
true
     @\textbf{ \{ var escrowMoney := sellerMoney.sprout \} }@
sellerMoney$\pre$ $\obeys$ ValidPurse $\rightarrow$ (
   sellerMoney $\obeys$ ValidPurse $\wedge$
   // TRUST
   escrowMoney $\obeys$ ValidPurse $\wedge$ CanTrade(sellerMoney, escrowMoney) $\wedge$
   // FUNCTIONAL
   escrowMoney.balance=0) $\wedge$
   // RISK
$\forall$p.[ p$\obeys\PRE$ValidPurse $\rightarrow$ p.balance=p.balance$\pre$ $\wedge$ escrowMoney $\neq$ p ]  $\wedge$
$\forall$o:$\pre$Object. $\forall$p$\obeys$$\pre$ValidPurse. [ $\MayAccess$(o,p) $\rightarrow$ $\MayAccess\pre$(o,p) ]
\end{lstlisting}
\caption{Hoare tuple for first step in \prg{deal} (Done with lstisting)}
\label{fig:DealV3:S1}
\end{figure*}


\newcommand{\MACROsproutOld}[2]{{
{\tt\footnotesize
\fbox{\parbox{0.8\textwidth}{
\internallinenumbers
\resetlinenumber[1]
\begin{verse}
  true \\
   ~~~~~~\textbf{ \{ var #1 := #2.sprout \} }\\
   #2$\pre$ $\obeys$ ValidPurse $\rightarrow$ (\\
   ~~~~#2 $\obeys$ ValidPurse $\wedge$ \\
   ~~~~// TRUST \\
   ~~~~#1 $\obeys$ ValidPurse $\wedge$ CanTrade(#2, #1) $\wedge$\\
   ~~~~// FUNCTIONAL  \\
   ~~~~#1.balance=0 )$\wedge$\\
   // RISK\\
   $\forall$p.[ p$\obeys\PRE$ValidPurse $\rightarrow$ p.balance=p.balance$\pre$ $\wedge$ #1 $\neq$ p ]  $\wedge$\\
   $\forall$o:$\pre$Object. $\forall$p$\obeys$$\pre$ValidPurse. [ $\MayAccess$(o,p) $\rightarrow$ $\MayAccess\pre$(o,p) ] \\
\end{verse}
}}
}}}

\begin{figure*}[htb]
\MACROsproutOld{escrowMoney}{sellerMoney}
\caption{MACRO VERSION: Hoare tuple for first step in \prg{deal} - \sdJJ{has some errors}}
\label{fig:DealV3:S1:old}
\end{figure*}
 

\newcommand{\MACROsproutNew}[2]{{
{\tt\footnotesize
\fbox{\parbox{0.8\textwidth}{
\internallinenumbers
\resetlinenumber[1]
\begin{verse}
  true \\
   ~~~~~~\textbf{ \{ var #1 := #2.sprout \} }\\
   #2$\pre$ $\obeys$ ValidPurse $\rightarrow$ (\\
   ~~~~#2 $\obeys$ ValidPurse $\wedge$ \\
   ~~~~// TRUST \\
   ~~~~#1 $\obeys$ ValidPurse $\wedge$ CanTrade(#2, #1) $\wedge$\\
   ~~~~// FUNCTIONAL  \\
   ~~~~#1.balance=0 )$\wedge$\\
   // RISK\\
   $\forall$p.[ p$\obeys\PRE$ValidPurse $\rightarrow$ 
        ( [ p.balance=p.balance$\pre$ $\wedge$ #1 $\neq$ p ] $\vee$\\  
        ~~~~ ~~~~ ~~~~ ~~~~~~~~ ~~~~ ~~~~ ~~~~ $\MayAccess\pre$(#1,p) ) ]  \\
   $\wedge$\\
   $\forall$z:$\pre$Object.$\forall$u:Object. \\
   ~~~~[ $\MayAccess$(z,u) $\rightarrow$ ($\MayAccess\pre$(z,u)$\vee$ ($\MayAccess\pre$(#2,z)$\wedge\MayAccess\pre$(#2,u)) ] \\
   $\HoareCSep$\\
   true
\end{verse}
}}
}}}

\begin{figure*}[htb]
\MACROsproutNew{escrowMoney}{sellerMoney}
\caption{MACRO VERSION: Hoare tuple for first step in \prg{deal} - \sdJJ{corrected}}
\label{fig:DealV3:S1}
\end{figure*}


 


% \newcommand{\MACROvalidate}[2]{{
% {\tt\footnotesize
% \fbox{\parbox{0.8\textwidth}{
% \internallinenumbers
% \resetlinenumber[1]
% \begin{verse}
%       true\\
%       ~~~~~~\textbf{ \{ res := #1.deposit(0, #2) \} }\\
%       (#1$\pre$ $\obeys$ ValidPurse $\rightarrow$  #1 $\obeys$ ValidPurse) $\wedge$ \\
%       // TRUST AND FUNCTIONAL\\
%       res $\equiv$ CanTrade(#1,#2)$\PRE{}$  $\wedge$\\
%       // RISK\\
%       $\forall$p.[ p$\obeys\PRE{}$ValidPurse$\,\rightarrow\,$ p.balance=p.balance$\pre$ ] $\wedge$\\
%       $\forall$o:$\pre$Object.\ $\forall$p$\obeys\pre$ValidPurse.\
%          [ $\MayAccess$(o,p) $\rightarrow$ $\MayAccess\pre$(o,p) ] )\\
% \end{verse}
% }}
% }}}

% \begin{figure*}[htb]
% \MACROvalidate{escrowMoney}{sellerMoney}
% \caption{WRONG MACRO VERSION: Hoare tuple for step 2 in \prg{deal}}
% \label{fig:DealV3:S2}
% \end{figure*}




% \newcommand{\MACROdeposit}[3]{{
% {\tt\footnotesize
% \fbox{\parbox{0.8\textwidth}{
% \internallinenumbers
% \resetlinenumber[1]
% \begin{verse}
%       #2$\in \mathbb{N}$\\
%       ~~~~~~\textbf{ \{ res := #1.deposit(#2, #3) \} }\\
%       #1$\pre$ $\obeys$ ValidPurse $\rightarrow$ [\\
%       #1 $\obeys$ ValidPurse $\wedge$\\
%       (res $\rightarrow$ \\
%       ~~~~// TRUST\\
%       ~~~~CanTrade(#1,#3)$\PRE{}$  $\wedge$\\
%       ~~~~// FUNCTIONAL  \\
%       ~~~~0$\leq$#2$\leq$#3.balance$\PRE{}\ \wedge$\\
%       ~~~~#1.balance=#1.balance$\PRE$+#2 $\wedge$\\
%       ~~~~#3.balance=#3.balance$\PRE$-#2  $\wedge$\\
%       ~~~~//RISK\\
%       ~~~~$\forall$p.[ p$\obeys$$\pre$ValidPurse $\wedge$ p$\notin\{$#1,#3$\}\,\rightarrow$\\
%       ~~~~~~~~ p.balance=p.balance$\pre$ ]  $\wedge$\\
%       ~~~~$\forall$o:$\pre$Object. $\forall$p$\obeys$$\pre$ValidPurse.
%                [ $\MayAccess$(o,p) $\rightarrow$ $\MayAccess\pre$(o,p) ]~~)\\
%       $\vee$\\
%       ($\neg$res $\rightarrow$ \\
%       ~~~~// TRUST and FUNCTIONAL  \\
%       ~~~~$\neg$[ CanTrade(#1,#3)$\PRE{}$ $\wedge$ 0$\leq$#2$\leq$#3.balance$\PRE{}$ ] $\wedge$\\
%       ~~~~// RISK\\
%       ~~~~$\forall$p.[ p$\obeys\PRE{}$ValidPurse$\,\rightarrow\,$ p.balance=p.balance$\pre$ ] $\wedge$\\
%       ~~~~$\forall$o:$\pre$Object. $\forall$p$\obeys$$\pre$ValidPurse.
%                [ $\MayAccess$(o,p) $\rightarrow$ $\MayAccess\pre$(o,p) ]~~)~~]\\
% \end{verse}
% }}
% }}}

% \begin{figure*}[htb]
% \MACROdeposit{escrowMoney}{price}{buyerMoney}
% \caption{WRONG MACRO VERSION: Hoare tuple for step 9 in \prg{deal}}
% \label{fig:DealV3:S9}
% \end{figure*}

\cleardoublepage
\newpage

We now outline the most salient steps from the proof of the
Escrow --- fig.~\ref{fig:DealV2Consolidated} shows the code of the
whole deal method.
To simplify the Hoare rules our formal language does not support
returning from the inside of a method. Fig.~\ref{fig:DealV3}
shows a re-written version of the method using only conditional
statements: note that the line numbers match
fig.~\ref{fig:DealV2Consolidated} as much as possible.

\subsection{Step 1}\footnote{James updated on 29th June, and Sophia revisited on 30th June}

The pre- and post-conditions for the first method call that
\lstinline+sprout+'s the \lstinline+escrowMoney+ purse from the
\lstinline+sellerMoney+ purse (line 4 in
\bothfigs) are shown in fig.~\ref{fig:DealV3:S1}.
The postcondition is concerned with two scenarios: what if \prg{escrowMoney} is valid purse, and what if it it not?



\subsubsection{What if \prg{escrowMoney} is valid purse}

First, from \lstinline+Policy_sprout+ in the \lstinline+ValidPurse+
specification (fig.~\ref{fig:ValidPurse}) by straightforwardly
applying \ruleN{meth-call-1} we obtain (1A):

\begin{tabular}{lll}
{\bf 1A} & &
\begin{lstlisting}[escapechar=@]
sellerMoney $\obeys$ ValidPurse
        @\textbf{ \{ var escrowMoney := sellerMoney.sprout \} }@
   // TRUST
escrowMoney $\obeys$ ValidPurse $\wedge$ CanTrade(sellerMoney, escrowMoney) $\wedge$
   // FUNCTIONAL
escrowMoney.balance=0 $\wedge$
   // RISK
$\forall$p.[ p$\obeys\PRE$ValidPurse $\rightarrow$ p.balance=p.balance$\pre$ $\wedge$ escrowMoney $\neq$ p ]  $\wedge$
$\forall$o:$\pre$Object. $\forall$p$\obeys$$\pre$ValidPurse. [ $\MayAccess$(o,p) $\rightarrow$ $\MayAccess\pre$(o,p) ]  
   $\HoareCSep$
true
\end{lstlisting}
\end{tabular}

\noindent (1A) includes \lstinline+sellerMoney $\obeys$ ValidPurse+ in
the pre-condition.  We can "move" this into the postcondition by
applying \ruleN{Cons-2} ---  allowing us to weaken the precondition
to \lstinline+true+ (1B):
 

\begin{tabular}{lll}
{\bf 1B} & & \begin{lstlisting}[escapechar=@]
true
        @\textbf{ \{ var escrowMoney := sellerMoney.sprout \} }@
sellerMoney$\pre$ $\obeys$ ValidPurse $\rightarrow$ (
   // TRUST
   escrowMoney $\obeys$ ValidPurse $\wedge$ CanTrade(sellerMoney, escrowMoney) $\wedge$
   // FUNCTIONAL
   escrowMoney.balance=0 $\wedge$
   // RISK
   $\forall$p.[ p$\obeys\PRE$ValidPurse $\rightarrow$ p.balance=p.balance$\pre$ $\wedge$ escrowMoney $\neq$ p ]  $\wedge$
   $\forall$o:$\pre$Object. $\forall$p$\obeys$$\pre$ValidPurse. [ $\MayAccess$(o,p) $\rightarrow$ $\MayAccess\pre$(o,p) ] )
   $\HoareCSep$
true
\end{lstlisting}
\end{tabular}


We need to ensure that we remember that \lstinline+sellerMoney+ still $\obeys$
\texttt{ValidPurse} in the post-condition, ie line 4 below. We apply on (1B) the rules
\ruleN{code-invar-2}, \ruleN{Cons-1}), and \ruleN{Cons-2}) to get (1C):

\begin{tabular}{lll}
{\bf 1C} & &
\begin{lstlisting}[escapechar=@]
true
        @\textbf{ \{ var escrowMoney := sellerMoney.sprout \} }@
sellerMoney$\pre$ $\obeys$ ValidPurse $\rightarrow$ (
   sellerMoney $\obeys$ ValidPurse $\wedge$
   // TRUST
   escrowMoney $\obeys$ ValidPurse $\wedge$ CanTrade(sellerMoney, escrowMoney) $\wedge$
   // FUNCTIONAL
   escrowMoney.balance=0 $\wedge$
   // RISK
   $\forall$p.[ p$\obeys\PRE$ValidPurse $\rightarrow$ p.balance=p.balance$\pre$ $\wedge$ escrowMoney $\neq$ p ]  $\wedge$
   $\forall$o:$\pre$Object. $\forall$p$\obeys$$\pre$ValidPurse. [ $\MayAccess$(o,p) $\rightarrow$ $\MayAccess\pre$(o,p) ] )
   $\HoareCSep$
true
\end{lstlisting}
\end{tabular}

\noindent
We can use \ruleN{code-invar-2} and the invariance of $\obeys$ to show
that at this point we have already established the one-way trust
relationship (1-ONEWAY):

\begin{lstlisting}[backgroundcolor=\color{yellow}]
sellerMoney $\obeys$ ValidPurse $\rightarrow$  escrowMoney $\obeys$ ValidPurse
\end{lstlisting}

So far so good: but the problem is that the RISK is only bounded if
the \lstinline+sellerMoney+ purse obeys its specification.   Because
of object-capability safety we are able to reason about connectivity even when
objects do not obey their specifications

\subsubsection{What if we have no evidence that \prg{escrowMoney} is valid purse}

\paragraph{Who may affect} The first question we tackle is who may affect a purse's balance.
For this,  \ruleN{code-invar-1} lets us import the policy
\lstinline+Pol_protect_balance+ from \lstinline+ValidPurse+:  

\begin{tabular}{lll}
{\bf 1D} & & \begin{lstlisting}[escapechar=@]
true
        @\textbf{ \{ var escrowMoney := sellerMoney.sprout \} }@
   //RISK
true
   $\HoareCSep$
$\forall$ p. [ p $\obeys$ ValidPurse $\rightarrow$ $\forall$ o:Object.[$\MayAffect$(o,p.balance) $\rightarrow$ $\MayAccess$(o,p)] ]
\end{lstlisting}
\end{tabular}

By application of rules \ruleN{code-invar-1}, \ruleN{Cons} and \ruleN{Cons-1} on (1D) we obtain

\begin{tabular}{lll}
{\bf 1E} & & 
\begin{lstlisting}[escapechar=@]
p $\obeys$ ValidPurse 
    @\textbf{ \{ var escrowMoney := sellerMoney.sprout \} }@
true
    $\HoareCSep$
$\forall$z.[$\MayAffect$(z, p.balance ) $\rightarrow$ $\MayAccess$(z,p)]  
\end{lstlisting}
\end{tabular}




\vspace{.02in}
\paragraph{How does accessibility evolve} The next question we tackle is 
changes in accessibility after the first line.
We start by application of \ruleN{meth-Call-2} ---
this says that the only way \sdJJ{accessibility can increase} is by the
\lstinline+sellerMoney+ having \sdJJ{access to objects, and passing these to other objects.}.
We combine \ruleN{meth-Call-2} with \ruleN{Cons-1} and obtain:\footnote{\sdJJ{This is different from the old version, because of the
quantifiers $\forall$ z: $\Obj$. $\forall$ u:$\pre$ $\Obj$. in line 6 -- we allow for new objects too}}
 

\begin{tabular}{lll}
{\bf 1F} & &
\begin{lstlisting}[escapechar=@]
true
        @\textbf{ \{ var escrowMoney := sellerMoney.sprout \} }@
   //RISK
$\forall$ z:$\pre$ $\Obj$.[ $\MayAccess$(escrowMoney, z) $\rightarrow$ $\MayAccess\pre$(sellerMoney, z) ]
   $\HoareCSep$
$\forall$ z:$\Obj$.$\forall$ u:$\Obj$.
    [  $\MayAccess$(z,u) $\rightarrow$
         [$\MayAccess\pre$(z,u)$\,\vee\,$ ($\MayAccess\pre$(sellerMoney,u)$\wedge\MayAccess\pre$(sellerMoney,z)]   ]
\end{lstlisting}
\end{tabular}

\vspace{.02in}
\paragraph{When does the balance of a valid purse change?}

We put together (1E), (1F) through rule \ruleN{Conj}, apply \ruleN{Cons-1} and obtain (1G) below. To do that we need to prove that 
$\forall z:\Obj.\forall u: \Obj.
[ \MayAccess(z, u)\rightarrow[\MayAccess\pre(z, u) \,\vee\, \MayAccess\pre(sellerMoney, u)]$ and $\neg\MayAccess\pre(sellerMoney, p)$  imply that 
$\forall z:\Obj.[\MayAccess(sellerMoney, z) \rightarrow
\neg\MayAccess(z, p)$. This is shown through transitivity of the $\MayAccess$ relation -- connectivity begets connectivity. 


\begin{tabular}{lll}
{\bf 1G} & &
\begin{lstlisting}[escapechar=@]
p $\obeys$ ValidPurse $\wedge\ \neg\MayAccess$(sellerMoney, p) 
        @\textbf{ \{ var escrowMoney := sellerMoney.sprout \} }@
   //RISK
true
   $\HoareCSep$
$\forall$z.[$\MayAffect$(z,p.balance) $\rightarrow$ $\MayAccess$(z,p)]
   $\wedge$
$\forall$ z:$\Obj$.[ $\MayAccess$(sellerMoney,z) $\rightarrow$ $\neg\MayAccess$(z,p) ]
\end{lstlisting}
\end{tabular}
 

We can now  apply on (1G) the rule \ruleN{Frame-MethCall} to establish that
 good purses inaccessible to \lstinline+sellerMoneyt+ cannot be changed by calling
\lstinline+sprout+. We obtain (1H): 

\begin{tabular}{lll}
{\bf 1H} & &
\begin{lstlisting}[escapechar=@]
p $\obeys$ ValidPurse $\wedge\ \neg\MayAccess$(sellerMoney, p) 
        @\textbf{ \{ var escrowMoney := sellerMoney.sprout \} }@
   //RISK
p.balance = p.balance\pre  
   $\HoareCSep$
true   
   \end{lstlisting}
\end{tabular}

And by applying \ruleN{Cons-2}, and \ruleN{Quantifiers-1}  on (1H) we obtain (1I):

\begin{tabular}{lll}
{\bf 1I} & &
\begin{lstlisting}[escapechar=@]
true 
        @\textbf{ \{ var escrowMoney := sellerMoney.sprout \} }@
   //RISK
$\forall$ p. [ p $\obeys$ ValidPurse $\rightarrow$ [ p.balance = p.balance$\pre\,\vee\,  \MayAccess$(sellerMoney, p) ] ]
   $\HoareCSep$
true   
   \end{lstlisting}
\end{tabular}



\vspace{.02in}
\subsubsection{Putting it all together } We now combine the results from (1C), (1I), (1F) - lines 6, 7 and 8, and obtain: 

\begin{tabular}{lll}
{\bf 1K} & &
\begin{lstlisting}[escapechar=@]
true
        @\textbf{ \{ var escrowMoney := sellerMoney.sprout \} }@
sellerMoney$\pre$ $\obeys$ ValidPurse $\rightarrow$ (
   sellerMoney $\obeys$ ValidPurse $\wedge$
   // TRUST
   escrowMoney $\obeys$ ValidPurse $\wedge$ CanTrade(sellerMoney, escrowMoney) $\wedge$
   // FUNCTIONAL
   escrowMoney.balance=0 $\wedge$
   // RISK
   $\forall$p.[ p$\obeys\PRE$ValidPurse $\rightarrow$ p.balance=p.balance$\pre$ $\wedge$ escrowMoney $\neq$ p ]  $\wedge$
   $\forall$o:$\pre$Object. $\forall$p$\obeys$$\pre$ValidPurse. [ $\MayAccess$(o,p) $\rightarrow$ $\MayAccess\pre$(o,p) ] )
$\wedge$
$\forall$ p. [ p $\obeys$ ValidPurse $\rightarrow$ [ p.balance = p.balance$\pre\,\vee\,  \MayAccess$(sellerMoney, p) ] ]
$\wedge$
$\forall$ z:$\Obj$.$\forall$ u:$\Obj$.
    [  $\MayAccess$(z,u) $\rightarrow$
         [$\MayAccess\pre$(z,u) $\,\vee\,$ ($\MayAccess\pre$(sellerMoney,u)$\wedge\MayAccess\pre$(sellerMoney,z)]   ]
$\HoareCSep$
true
\end{lstlisting}
\end{tabular}
 

\subsection{Step 2}

We next must consider the second method call (
%
\lstinline+escrowMoney.deposit(0, sellerMoney)+,
%
line 5 in \bothfigs) where the
\lstinline+escrowMoney+ purse in turn validates the
\lstinline+sellerMoney+ purse.
There are two policies describing the behaviour of the
\lstinline+deposit+ method in the
\lstinline+ValidPurse+ specification (fig.~\ref{fig:ValidPurse}).
\lstinline+Pol_deposit_1+ applies when a deposit reports success (by
returning \lstinline+true+) and \lstinline+Pol_deposit_2+ when a
deposit reports failure.  Because we're only interested in validating
the purses, the amount we are attempting to transfer is 0, and so we
can combine these policies, giving the pre- and post-conditions
in fig.~\ref{fig:DealV3:S2}.

The overall shape of the argument follows step 1, except that we need
to consider the additional \lstinline+src+ argument of the deposit.

Applying \ruleN{meth-call-1} and \ruleN{Conj}
to \lstinline+Pol_deposit_1+ and \lstinline+Pol_deposit_2+
we obtain (2A):

\begin{lstlisting}[escapechar=@]
escrowMoney $\obeys$ ValidPurse
       @\textbf{ \{ res := escrowMoney.deposit(0, sellerMoney) \} }@
      // TRUST
res $\equiv$ CanTrade(escrowMoney,sellerMoney)$\PRE{}$  $\wedge$
      // RISK AND FUNCTIONAL
$\forall$p.[ p$\obeys\PRE{}$ValidPurse$\,\rightarrow\,$ p.balance=p.balance$\pre$ ] $\wedge$
$\forall$o:$\pre$Object. $\forall$p$\obeys\pre$ValidPurse. [ $\MayAccess$(o,p) $\rightarrow$ $\MayAccess\pre$(o,p) ]
   $\HoareCSep$
true
\end{lstlisting}

\noindent (2A) again includes \lstinline+escrowMoney $\obeys$ ValidPurse+ in
the pre-condition.  We can "move" this into the postcondition by
applying \ruleN{Cons-2} ---  allowing us to weaken the precondition
to \lstinline+true+ (2B):

\begin{lstlisting}[escapechar=@]
true
       @\textbf{ \{ res := escrowMoney.deposit(0, sellerMoney) \} }@
escrowMoney$\pre$ $\obeys$ ValidPurse $\rightarrow$ (
      // TRUST
      res $\equiv$ CanTrade(escrowMoney,sellerMoney)$\PRE{}$  $\wedge$
      // RISK AND FUNCTIONAL
      $\forall$p.[ p$\obeys\PRE{}$ValidPurse$\,\rightarrow\,$ p.balance=p.balance$\pre$ ] $\wedge$
      $\forall$o:$\pre$Object. $\forall$p$\obeys\pre$ValidPurse.
         [ $\MayAccess$(o,p) $\rightarrow$ $\MayAccess\pre$(o,p) ] )
   $\HoareCSep$
true
\end{lstlisting}

We need to ensure that we remember that \lstinline+escrowMoney+ still $\obeys$
\texttt{ValidPurse} in the post-condition. We apply
\ruleN{code-invar-2}, \ruleN{Cons-1}) to get (2C):

\begin{lstlisting}[escapechar=@]
true
       @\textbf{ \{ res := escrowMoney.deposit(0, sellerMoney) \} }@
escrowMoney$\pre$ $\obeys$ ValidPurse $\rightarrow$ (
      escrowMoney $\obeys$ ValidPurse $\wedge$
      // TRUST
      res $\equiv$ CanTrade(escrowMoney,sellerMoney)$\PRE{}$  $\wedge$
      // RISK AND FUNCTIONAL
      $\forall$p.[ p$\obeys\PRE{}$ValidPurse$\,\rightarrow\,$ p.balance=p.balance$\pre$ ] $\wedge$
      $\forall$o:$\pre$Object. $\forall$p$\obeys\pre$ValidPurse.
         [ $\MayAccess$(o,p) $\rightarrow$ $\MayAccess\pre$(o,p) ] )
   $\HoareCSep$
true
\end{lstlisting}


We again need to bound the risk whether or not \lstinline+escrowMoney+
obeys its specification. Here we must take the argument purse into account.
\kjx{James --- ignoring the 0 passed in, but worrying about res.}
We again apply \ruleN{meth-Call-2} (2D):

\begin{lstlisting}[escapechar=@]
true
        @\textbf{ \{ res := escrowMoney.deposit(0, sellerMoney) \} }@
   //RISK
$\forall$ z:$\pre$ $\Obj$. $\MayAccess$(res, z)  $\rightarrow$ ($\MayAccess\pre$(escrowMoney, z) $\vee$ $\MayAccess\pre$(sellerMoney, z))
   $\HoareCSep$
$\forall$ z, u:$\pre$ $\Obj$. ($\MayAccess$(u, z) $\rightarrow$
      [$\MayAccess\pre$(u, z) $\vee$ (( $\MayAccess\pre$(escrowMoney, z) $\vee$ $\MayAccess\pre$(sellerMoney, z)  ) $\wedge$
                             ($\MayAccess\pre$(escrowMoney, u) $\vee$ $\MayAccess\pre$(sellerMoney, u)  )) ] )
\end{lstlisting}


Then \ruleN{code-invar-1} lets us import \lstinline+Pol_protect_balance+: (2E)

\begin{lstlisting}[escapechar=@]
true
         @\textbf{ \{ res := escrowMoney.deposit(0, sellerMoney) \} }@
    //RISK
true
    $\HoareCSep$
$\forall$ z. z $\obeys$ ValidPurse $\rightarrow$ $\forall$ o:Object. [$\MayAffect$(o,z.balance) $\rightarrow$ $\MayAccess$(o,z) ]
\end{lstlisting}


The we combine (2D) and (2E) to get (2F):

\begin{lstlisting}[escapechar=@]
true
        @\textbf{ \{ res := escrowMoney.deposit(0, sellerMoney) \} }@
   //RISK
$\forall$ z:$\pre$ $\Obj$. $\MayAccess$(res, z)  $\rightarrow$ ($\MayAccess\pre$(escrowMoney, z) $\vee$ $\MayAccess\pre$(sellerMoney, z))
   $\HoareCSep$
[ $\forall$ z, u:$\pre$ $\Obj$. ($\MayAccess$(u, z) $\rightarrow$
      [$\MayAccess\pre$(u, z) $\vee$
       (( $\MayAccess\pre$(escrowMoney, z) $\vee$ $\MayAccess\pre$(sellerMoney, z)  ) $\wedge$
        ($\MayAccess\pre$(escrowMoney, u) $\vee$ $\MayAccess\pre$(sellerMoney, u)  )) ] ) ] $\wedge$
[ $\forall$ z. z $\obeys$ ValidPurse $\rightarrow$
       $\forall$ o:Object. ($\MayAffect$(o,z.balance) $\rightarrow$ $\MayAccess$(o,z) ) ]
\end{lstlisting}


\kjx{James --- Is this assumption right?}  We apply \ruleN{method-call-2}
again, but hypothesizing that there is some other \kjx{ValidPurse}  $p$ that
neither sellerMoney nor escrowMoney can access (2G):


\begin{lstlisting}[escapechar=@]
$\forall$ p. p $\obeys$ ValidPurse. $\neg$ $\MayAccess$(escrowMoney, p) $\wedge$ $\neg$ $\MayAccess$(sellerMoney, p)
         @\textbf{ \{ res := escrowMoney.deposit(0, sellerMoney) \} }@
    //RISK
$\forall$ z:$\pre$ $\Obj$. $\MayAccess$(res, z)  $\rightarrow$ ($\MayAccess\pre$(escrowMoney, z) $\vee$ $\MayAccess\pre$(sellerMoney, z))
   $\HoareCSep$
$\forall$ z, u:$\pre$ $\Obj$. ($\MayAccess$(u, z) $\rightarrow$
      [$\MayAccess\pre$(u, z) $\vee$
       (( $\MayAccess\pre$(escrowMoney, z) $\vee$ $\MayAccess\pre$(sellerMoney, z)  ) $\wedge$ ($\MayAccess\pre$(escrowMoney, u) $\vee$ $\MayAccess\pre$(sellerMoney, u)  )) ] )
\end{lstlisting}

We need to show the assumption is preserved, i.e. the is no purse they couldn't get to before, but can get to afterwards: given
%
\lstinline+$\neg$ $\MayAccess\pre$(escrowMoney, p) $\wedge$ $\neg$ $\MayAccess\pre$(sellerMoney, p)+
%
we have to show
%
\lstinline+$\neg$ $\MayAccess$(escrowMoney, p) $\wedge$ $\neg$ $\MayAccess$(sellerMoney, p)+


The reasoning is the same as the 1G step: contradiction on the invariant: to conclude
$\MayAccess\texttt{(escrowMoney, p)}$  then either
$\MayAccess\pre\texttt{(escrowMoney, p)}$  (contradiction) or
$\MayAccess\pre\texttt{(escrowMoney, z)}$ where $z = p$ (contradition);
and similarly
$\MayAccess\texttt{(sellerMoney, p)}$  then either
$\MayAccess\pre\texttt{(sellerMoney, p)}$  (contradiction) or
$\MayAccess\pre\texttt{(sellerMoney, z)}$ where $z = p$ (contradition).

\begin{lstlisting}[escapechar=@]
$\forall$ p: p $\obeys$ ValidPurse. $\neg$ $\MayAccess$(escrowMoney, p) $\wedge$ $\neg$ $\MayAccess$(sellerMoney, p)
         @\textbf{ \{ res := escrowMoney.deposit(0, sellerMoney) \} }@
    //RISK
$\forall$ z:$\pre$ $\Obj$. $\MayAccess$(res, z)  $\rightarrow$ ($\MayAccess\pre$(escrowMoney, z) $\vee$ $\MayAccess\pre$(sellerMoney, z))
   $\HoareCSep$
   //What the rule gives us
$\forall$ z, u:$\pre$ $\Obj$. ($\MayAccess$(u, z) $\rightarrow$
      [$\MayAccess\pre$(u, z) $\vee$
       (( $\MayAccess\pre$(escrowMoney, z) $\vee$ $\MayAccess\pre$(sellerMoney, z)  ) $\wedge$ ($\MayAccess\pre$(escrowMoney, u) $\vee$ $\MayAccess\pre$(sellerMoney, u)  )) ] )
   //WHAT SOPHIA WANTS
$\forall$ z:$\pre$ $\Obj$. $\MayAccess$(escrowMoney, z) $\rightarrow$ $\neg \MayAccess$(z,p) $\wedge$
$\forall$ z:$\pre$ $\Obj$. $\MayAccess$(sellerMoney, z) $\rightarrow$ $\neg \MayAccess$(z,p)
   //WHAT James thinks we can get (which implies sophia's condition above)
$\neg \MayAccess\texttt{(escrowMoney, p)} \wedge \neg \MayAccess\texttt{(sellerMoney, p)}$
\end{lstlisting}


Again we cross our fingers and invoke \ruleN{frame-methCall}:

$\begin{array}{l}
   \inferenceruleNNP {frame-methCall} {
          \Hoare{\A}   {v:=\prg{x.m(y)}} {{\true}}  {\B} \\
B \equiv \forall \prg{z}. (\ \MayAffect(\prg{z},\A')  \rightarrow \B'({\prg{z}})\ )\ \ \wedge
          \\
% trick for layout
		\SP\SP  \
		{\forall \prg{z}.(\ (\MayAccess\pre(\prg{x},\prg{z}) \vee \MayAccess\pre(\prg{y},\prg{z})
           \vee \New(\prg{z}) \ )%(\prg{z}:\prg{Object}\wedge \neg ( \prg{z}:\pre\prg{Object}) ) \ )
           \ \rightarrow\  \neg \B'(\prg{z})\ )}
}
{
          \Hoare{{\A\wedge\A'}} {\prg{v:=x.m(y)}} {\A'} {{\true}}
}
\end{array}{l}$

The ``precondition'' is: $A \equiv \neg \MayAccess\texttt{(escrowMoney, p)} \wedge \neg \MayAccess\texttt{(sellerMoney, p)}$

The property being preseved is:
 $A' \equiv \texttt{p.balance} = \texttt{p.balance}\pre$.

We hypothesize:
%
$B'(z) \equiv  \MayAccess\texttt{(z , p)}$

We need (2H):

\begin{lstlisting}[escapechar=@]
$\forall$ p: p $\obeys$ ValidPurse. $\neg \MayAccess$(escrowMoney, p) $\wedge$  $\neg \MayAccess$(sellerMoney, p)
    @\textbf{ \{ res := escrowMoney.deposit(0, sellerMoney) \} }@
true
    $\HoareCSep$
$\forall$z.($\MayAffect$(z, p.balance ) $\rightarrow$ $\MayAccess$(z,p)) $\wedge$
$\forall$z.(($\MayAccess$(escrowMoney,z) $\vee$ $\MayAccess$(sellerMoney,z) $\vee$ New(z))$\rightarrow$ $\neg \MayAccess$(z,p))
\end{lstlisting}

We know the first conjunct because \lstinline+p+ is a good purse by \lstinline+Pol_protect_balance+.
We know the second by transitivity of $\MayAccess$,
and \lstinline+p+ is in the pre-state so it cannot be new;

Thus we obtain (2I):

\begin{lstlisting}[escapechar=@]
$\forall$ p: p $\obeys$ ValidPurse. $\neg \MayAccess$(escrowMoney, p) $\wedge$  $\neg \MayAccess$(sellerMoney, p)
    @\textbf{ \{ res := escrowMoney.deposit(0, sellerMoney) \} }@
p.balance = p.balance$\pre$
    $\HoareCSep$
true
\end{lstlisting}


By \ruleN{CONS-2} (2J):

\begin{lstlisting}[escapechar=@]
true
    @\textbf{ \{ res := escrowMoney.deposit(0, sellerMoney) \} }@
$\forall$ p: p $\obeys\pre$ ValidPurse. ($\neg \MayAccess$(escrowMoney, p) $\wedge \neg \MayAccess$(sellerMomey, p) ) $\rightarrow$ p.balance = p.balance$\pre$
    $\HoareCSep$
true
\end{lstlisting}

Rewriting the implication (2K):

\begin{lstlisting}[escapechar=@]
true
    @\textbf{ \{ res := escrowMoney.deposit(0, sellerMoney) \} }@
$\forall$ p: p $\obeys\pre$ ValidPurse.  ($\MayAccess$(escrowMoney, p) $\vee$ $\MayAccess$(sellerMomey, p) )  $\vee$ p.balance = p.balance$\pre$
    $\HoareCSep$
true
\end{lstlisting}

Finally, \ruleN{conj} lets us combine (2C), (2D) and (2K) for (2L):

\begin{lstlisting}[escapechar=@]
true
    @\textbf{ \{ res := escrowMoney.deposit(0, sellerMoney) \} }@
//2C
escrowMoney$\pre$ $\obeys$ ValidPurse $\rightarrow$ (
      escrowMoney $\obeys$ ValidPurse $\wedge$
      // TRUST
      res $\equiv$ CanTrade(escrowMoney,sellerMoney)$\PRE{}$  $\wedge$
      // RISK AND FUNCTIONAL
      $\forall$p.[ p$\obeys\PRE{}$ValidPurse$\,\rightarrow\,$ p.balance=p.balance$\pre$ ] $\wedge$
      $\forall$o:$\pre$Object. $\forall$p$\obeys\pre$ValidPurse. [ $\MayAccess$(o,p) $\rightarrow$ $\MayAccess\pre$(o,p) ] )  $\wedge$
//2D
$\forall$ z:$\pre$ $\Obj$. $\MayAccess$(res, z)  $\rightarrow$ ($\MayAccess\pre$(escrowMoney, z) $\vee$ $\MayAccess\pre$(sellerMoney, z) $\wedge$
//2K
$\forall$ p: p $\obeys\pre$ ValidPurse. ($\MayAccess$(escrowMoney, p) $\vee$ $\MayAccess$(sellerMoney, p)  $\vee$ p.balance = p.balance$\pre$ )
   $\HoareCSep$
//2D
$\forall$ z, u:$\pre$ $\Obj$. ($\MayAccess$(u, z) $\rightarrow$
      [$\MayAccess\pre$(u, z) $\vee$
       (( $\MayAccess\pre$(escrowMoney, z) $\vee$ $\MayAccess\pre$(sellerMoney, z)  ) $\wedge$ ($\MayAccess\pre$(escrowMoney, u) $\vee$ $\MayAccess\pre$(sellerMoney, u)  )) ] )
\end{lstlisting}

\subsubsection{Combining Step 1 and Step 2}

Rule \ruleN{sequence} lets us combine 1L and 2L --- the main thing is that because
%
\lstinline+$\MayAccess$(escrowMoney, z) $\rightarrow$ $\MayAccess\pre$(sellerMoney, z)+
%
we can elide escrowMoney from the risk, and if
%
\lstinline+sellerMoney$\pre$ $\obeys$ ValidPurse+
%
then \lstinline+res+ must be \lstinline+true+.
%
giving (2M):

\begin{lstlisting}[escapechar=@]
true
    @\textbf{ \{ var escrowMoney := sellerMoney.sprout }@
    @\textbf{     res := escrowMoney.deposit(0, sellerMoney) \} }@
sellerMoney$\pre$ $\obeys$ ValidPurse $\rightarrow$ (
   // TRUST
   sellerMoney $\obeys$ ValidPurse $\wedge$  escrowMoney $\obeys$ ValidPurse $\wedge$ CanTrade(sellerMoney, escrowMoney) $\wedge$
   // FUNCTIONAL
   escrowMoney.balance=0 $\wedge$
   res $\equiv$ true
   // RISK (such as it is)
   $\forall$p.[ p$\obeys\PRE$ValidPurse $\rightarrow$ p.balance=p.balance$\pre$ $\wedge$ escrowMoney $\neq$ p ] 
   $\forall$o:$\pre$Object. $\forall$p$\obeys\pre$ValidPurse. [ $\MayAccess$(o,p) $\rightarrow$ $\MayAccess\pre$(o,p) ]
) $\wedge$
//RISK
$\forall$ z:$\pre$ $\Obj$. $\MayAccess$(escrowMoney, z) $\rightarrow$ $\MayAccess\pre$(sellerMoney, z) $\wedge$
$\forall$ z, u:$\pre$ $\Obj$. ($\MayAccess$(u, z) $\rightarrow$
    [$\MayAccess\pre$(u, z) $\vee$ (($\MayAccess\pre$(sellerMoney, z)  ) $\wedge$ ($\MayAccess\pre$(sellerMoney, u)  )) ] ) $\wedge$
//2D
$\forall$ z:$\pre$ $\Obj$. $\MayAccess$(res, z)  $\rightarrow$ $\MayAccess\pre$(sellerMoney, z) $\wedge$
//2K
$\forall$ p: p $\obeys\pre$ ValidPurse. ($\MayAccess$(escrowMoney, p) $\vee$ $\MayAccess$(sellerMoney, p)  $\vee$ p.balance = p.balance$\pre$ )
$\forall$ z, u:$\pre$ $\Obj$. ($\MayAccess$(u, z) $\rightarrow$
      [$\MayAccess\pre$(u, z) $\vee$
       (( $\MayAccess\pre$(escrowMoney, z) $\vee$ $\MayAccess\pre$(sellerMoney, z)  ) $\wedge$ ($\MayAccess\pre$(escrowMoney, u) $\vee$ $\MayAccess\pre$(sellerMoney, u)  )) ] )
   $\HoareCSep$
true
\end{lstlisting}


\subsubsection{End of Step 2}

Note that at this, by \ruleN{Cond-1}, if \lstinline+res+ is true, we
have again established a one-way trust relationship, in the opposite
direction to Step 1: (2-ONEWAY):

\begin{lstlisting}[backgroundcolor=\color{yellow}]
res $\rightarrow$ (sellerMoney$\pre$ $\obeys$ ValidPurse $\rightarrow$  escrowMoney$\pre$ $\obeys$ ValidPurse)
\end{lstlisting}

\noindent based on the definition of \lstinline+CanTrade+, or rather
its specification by the policies \lstinline+My_Purses_Valid+ and
\lstinline+My_Purses_CanTrade+. \kjx{James thinks he would rather
  have the obeys clause in the specification. not buried in CanTrade.}

\kjx{James has no idea if the following is right:}
Then by \ruleN{Cons-1}, we can assume
the trust relationship from Step 1 (1-ONEWAY) into the precondition,
and \ruleN{code-invar-2} carries that in to post-condition;
\ruleN{sequence} ensures (1-ONEWAY) as precondition is satisifed.
Thus, at this point, we have established the two way trust (2-TWOWAY):

\begin{lstlisting}[backgroundcolor=\color{red}]
res $\rightarrow$ (sellerMoney$\pre$ $\obeys$ ValidPurse $\longleftrightarrow$  escrowMoney$\pre$ $\obeys$ ValidPurse)
\end{lstlisting}

For Step 3 to step 31 we will be under the condition that
\lstinline+res+ is true, so in that portion of the code this
assumption holds.

\subsubsection{Error Exit}

Looking ahead slightly, what if we reach line 6 of \bothfigs\ with
%
\lstinline+res $\equiv$ false+?
%
Considering (2M), we must have
%
\lstinline+$\neg$ sellerMoney$\pre$ $\obeys$ ValidPurse+
%
the risk is given by the unconditional conjuncts from 2M.

\subsection{Steps 3 and 4}

Steps 3 and 4, lines 6 to 10 of \bothfigs, are basically the same as Step 2, except validating
\lstinline+buyerMoney+ against \lstinline+escrowMoney+.

\kjx{What follows is mostly so I can track my working (or you check it), this doesn't even have to appear in the TR.}

We work from 2L with the POWER of CUT and PASTE to get 3L:

\begin{lstlisting}[escapechar=@]
true
    @\textbf{ \{ res := buyerMoney.deposit(0, escrowMoney) \} }@
buyerMoney$\pre$ $\obeys$ ValidPurse $\rightarrow$ (
      buyerMoney $\obeys$ ValidPurse $\wedge$
      // TRUST
      res $\equiv$ CanTrade(buyerMoney,escrowMoney)$\PRE{}$  $\wedge$
      // RISK AND FUNCTIONAL
      $\forall$p.[ p$\obeys\PRE{}$ValidPurse$\,\rightarrow\,$ p.balance=p.balance$\pre$ ] $\wedge$
      $\forall$o:$\pre$Object. $\forall$p$\obeys\pre$ValidPurse. [ $\MayAccess$(o,p) $\rightarrow$ $\MayAccess\pre$(o,p) ] )  $\wedge$
$\forall$ z:$\pre$ $\Obj$. $\MayAccess$(res, z)  $\rightarrow$ ($\MayAccess\pre$(buyerMoney, z) $\vee$ $\MayAccess\pre$(escrowMoney, z) $\wedge$
$\forall$ p: p $\obeys\pre$ ValidPurse. ($\MayAccess$(buyerMoney, p) $\vee$ $\MayAccess$(escrowMoney, p)  $\vee$ p.balance = p.balance$\pre$ )
   $\HoareCSep$
$\forall$ z, u:$\pre$ $\Obj$. ($\MayAccess$(u, z) $\rightarrow$
      [$\MayAccess\pre$(u, z) $\vee$
       (( $\MayAccess\pre$(buyerMoney, z) $\vee$ $\MayAccess\pre$(escrowMoney, z)  ) $\wedge$ ($\MayAccess\pre$(buyerMoney, u) $\vee$ $\MayAccess\pre$(escrowMoney, u)  )) ] )
\end{lstlisting}

and then 4L:

\begin{lstlisting}[escapechar=@]
true
    @\textbf{ \{ res := escrowMoney.deposit(0, buyerMoney) \} }@
escrowMoney$\pre$ $\obeys$ ValidPurse $\rightarrow$ (
      escrowMoney $\obeys$ ValidPurse $\wedge$
      // TRUST
      res $\equiv$ CanTrade(escrowMoney,buyerMoney)$\PRE{}$  $\wedge$
      // RISK AND FUNCTIONAL
      $\forall$p.[ p$\obeys\PRE{}$ValidPurse$\,\rightarrow\,$ p.balance=p.balance$\pre$ ] $\wedge$
      $\forall$o:$\pre$Object. $\forall$p$\obeys\pre$ValidPurse. [ $\MayAccess$(o,p) $\rightarrow$ $\MayAccess\pre$(o,p) ] )  $\wedge$
$\forall$ z:$\pre$ $\Obj$. $\MayAccess$(res, z)  $\rightarrow$ ($\MayAccess\pre$(escrowMoney, z) $\vee$ $\MayAccess\pre$(buyerMoney, z) $\wedge$
$\forall$ p: p $\obeys\pre$ ValidPurse. ($\MayAccess$(escrowMoney, p) $\vee$ $\MayAccess$(buyerMoney, p)  $\vee$ p.balance = p.balance$\pre$ )
   $\HoareCSep$
$\forall$ z, u:$\pre$ $\Obj$. ($\MayAccess$(u, z) $\rightarrow$
      [$\MayAccess\pre$(u, z) $\vee$
       (( $\MayAccess\pre$(escrowMoney, z) $\vee$ $\MayAccess\pre$(buyerMoney, z)  ) $\wedge$ ($\MayAccess\pre$(escrowMoney, u) $\vee$ $\MayAccess\pre$(buyerMoney, u)  )) ] )
\end{lstlisting}

\subsubsection{End of Step 4}

Rule \ruleN{sequence} lets us combine 2M, 3L, and 4L to get 4M ---
\kjx{I'm NOT showing all the rules, doing it by inspection (i.e.\
  guessing) becuase it's just a sequence and simplification). I
  think. \ldots}.  (4M):

\newcommand{\init}{_\textit{0}}

\kjx{should some of the $e\pre$ become $e\init$?   I only started using init$\init$ around step 10}

\begin{lstlisting}[escapechar=@]
true
    @\textbf{ \{ var escrowMoney := sellerMoney.sprout }@
    @\textbf{     res := escrowMoney.deposit(0, sellerMoney) }@
    @\textbf{     if (res) then \{ }@
    @\textbf{      res := buyerMoney.deposit(0, escrowMoney) }@
    @\textbf{       if (res) then \{ }@
    @\textbf{         res := escrowMoney.deposit(0, buyerMoney) }@
//THIS IS A GUESS
((sellerMoney$\pre$ $\obeys$ ValidPurse) $\vee$  (buyerMoney $\obeys$ ValidPurse)) $\rightarrow$ (
   // TRUST
   res $\equiv$ CanTrade(sellerMoney, buyerMoney) $\wedge$
   CanTrade(sellerMoney, buyerMoney) $\rightarrow$ CanTrade(escrowMoney, buyerMoney) $\wedge$
   // FUNCTIONAL
   escrowMoney.balance=0 $\wedge$
   // RISK (such as it is)
   $\forall$p.[ p$\obeys\PRE$ValidPurse $\rightarrow$ p.balance=p.balance$\pre$ $\wedge$ escrowMoney $\neq$ p ] 
   $\forall$o:$\pre$Object. $\forall$p$\obeys\pre$ValidPurse. [ $\MayAccess$(o,p) $\rightarrow$ $\MayAccess\pre$(o,p) ]
) $\wedge$

//RISK
$\forall$ z:$\pre$ $\Obj$. $\MayAccess$(res, z)  $\rightarrow$ ($\MayAccess\pre$(sellerMoney, z) $\vee$ $\MayAccess\pre$(buyerMoney, z) $\wedge$
$\forall$ p: p $\obeys\pre$ ValidPurse. ($\MayAccess$(sellerMoney, p) $\vee$ $\MayAccess$(buyerMoney, p)  $\vee$ p.balance = p.balance$\pre$ )
   $\HoareCSep$
$\forall$ z, u:$\pre$ $\Obj$. ($\MayAccess$(u, z) $\rightarrow$
      [$\MayAccess\pre$(u, z) $\vee$
       (( $\MayAccess\pre$(sellerMoney, z) $\vee$ $\MayAccess\pre$(buyerMoney, z)  ) $\wedge$ ($\MayAccess\pre$(sellerMoney, u) $\vee$ $\MayAccess\pre$(buyerMoney, u)  )) ] )
\end{lstlisting}

Note once again that if we get here with \lstinline+res = true+, we have established hypothetical two-way trust
between all three purses, and thus between the buyer and seller purses in particular (4-TWOWAY):

\begin{lstlisting}[backgroundcolor=\color{red}]
sellerMoney$\pre$ $\obeys$ ValidPurse $\longleftrightarrow$  escrowMoney$\pre$ $\obeys$ ValidPurse $\longleftrightarrow$  buyerMoney$\pre$ $\obeys$ ValidPurse
\end{lstlisting}

\subsubsection{Error Exit}

I don't want to be that fine-grained so again, so we take risk is that from 4M, i.e. the worst case,
for exiting at any of the intermediate steps.

\subsection{Steps 5-8}

Steps 5 to 8, lines 12 to 20 of \bothfigs, are exactly the same as
steps 1-4, except validating \lstinline+buyerGoods+ against
\lstinline+escrowGoods+.  The POWER of CUT and PASTE can give us 8M
\kjx{which I'd copy and paste \textbf{were I surer the above was
    correct} although were I surer they above was correct I wouldn't
  need to cut and paste it.}


\subsection{Step 9}

Step 9, line 22 of \bothfigs,
%
\lstinline+res := escrowMoney.deposit(price, buyerMoney)+
%
is the first actual deposit with a non-zero balance in the escrow
code.  The reasoning forthis step is essentially the same as for step
2, however there we elided the amount since it was always zero --- so
must revisit steps 2A-2C, this time also handling the \lstinline+amt+.
The risk does not depend on the amount transferred, so steps 2D-2L are
unchanged: I've elided it here.

Thus (9A) from \ruleN{meth-call-1} and \ruleN{Conj} with \lstinline+Pol_deposit_1+ and \lstinline+Pol_deposit_2+: the postcondition is a big conjunction of both policies (9A):

\begin{lstlisting}[escapechar=@]
(escrowMoney $\obeys$ ValidPurse) $\wedge$ amt$\in \mathbb{N}$
       @\textbf{ \{ res := escrowMoney.deposit(price, buyerMoney) \} }@
res $\rightarrow$ (
    // TRUST
    CanTrade(escrowMoney,buyerMoney)$\PRE{}$  $\wedge$
    // FUNCTIONAL
    0$\leq$price$\leq$buyerMoney.balance$\PRE{}\ \wedge$
    escrowMoney.balance=escrowMoney.balance$\PRE$+price $\wedge$
    buyerMoney.balance=buyerMoney.balance$\PRE$-price  $\wedge$
    //RISK
    $\forall$p.[ p$\obeys$$\pre$ValidPurse $\wedge$ p$\notin\{$escrowMoney,buyerMoney$\}\,\rightarrow$
       p.balance=p.balance$\pre$ ]  $\wedge$
    $\forall$o:$\pre$Object. $\forall$p$\obeys$$\pre$ValidPurse.
       [ $\MayAccess$(o,p) $\rightarrow$ $\MayAccess\pre$(o,p) ]   )
$\wedge$
$\neg$res $\rightarrow$ (
    // TRUST and FUNCTIONAL
    $\neg$[ CanTrade(escrowMoney,buyerMoney)$\PRE{}$ $\wedge$ 0$\leq$price$\leq$buyerMoney.balance$\PRE{}$ ] $\wedge$
    // RISK
    $\forall$p.[ p$\obeys\PRE{}$ValidPurse$\,\rightarrow\,$ p.balance=p.balance$\pre$ ] $\wedge$
    $\forall$o:$\pre$Object. $\forall$p$\obeys$$\pre$ValidPurse.
       [ $\MayAccess$(o,p) $\rightarrow$ $\MayAccess\pre$(o,p) ]   )
   $\HoareCSep$
true
\end{lstlisting}



\kjx{at this point I really don't know what I'm doing}

\noindent \ruleN{sequence} means that by the start of line 22 we have
\lstinline+escrowMoney $\obeys$ ValidPurse+ (folowing from 8M); for
now we ignore \lstinline+amt$\in \mathbb{N}$+.

As in step 2 and the other deposit steps, we move the trust assumption
to the precondition, and carry trust over into the postcondition
via  \ruleN{Cons-2}l, \ruleN{code-invar-2}, \ruleN{Cons-1}) to get (9C):

At the end of step 1L, we had \lstinline+escrowMoney.balance = 0+;
that step preserves \lstinline+buyerMoney+'s initial balance (which
I'll write \lstinline+buyerMoney\init+) all subsequent steps preserve
every valid purse's balance, including \lstinline+escrowMoney+ and
\lstinline+buyerMoney+: and \ruleN{code-invar-2} means we can carry
the assumption that they purses are valid through each step (see 4M, 8M).

\begin{lstlisting}[escapechar=@]
amt$\in \mathbb{N}$
       @\textbf{ \{ res := escrowMoney.deposit(price, buyerMoney) \} }@
(escrowMoney $\obeys$ ValidPurse $\wedge$  res $\rightarrow$ (
    // TRUST
    CanTrade(escrowMoney,buyerMoney)$\PRE{}$  $\wedge$
    // FUNCTIONAL
    0$\leq$price$\leq$buyerMoney.balance$\init\ \wedge$
    escrowMoney.balance=price $\wedge$
    buyerMoney.balance=buyerMoney.balance$\init$-price  $\wedge$
    //RISK
    $\forall$p.[ p$\obeys\init$ValidPurse $\wedge$ p$\notin\{$escrowMoney,buyerMoney$\}\,\rightarrow$
       p.balance=p.balance$\init$ ]  $\wedge$
    $\forall$o:$\pre$Object. $\forall$p$\obeys$$\pre$ValidPurse.
       [ $\MayAccess$(o,p) $\rightarrow$ $\MayAccess\init$(o,p) ]   ) )
$\wedge$
(escrowMoney $\obeys$ ValidPurse $\wedge$  $\neg$res $\rightarrow$ (
    // TRUST and FUNCTIONAL
    $\neg$[ CanTrade(escrowMoney,buyerMoney)$\PRE{}$ $\wedge$ 0$\leq$price$\leq$buyerMoney.balance$\PRE{}$ ] $\wedge$
    // RISK
    $\forall$p.[ p$\obeys\PRE{}$ValidPurse$\,\rightarrow\,$ p.balance=p.balance$\pre$ ] $\wedge$
    $\forall$o:$\pre$Object. $\forall$p$\obeys$$\pre$ValidPurse.
       [ $\MayAccess$(o,p) $\rightarrow$ $\MayAccess\pre$(o,p) ]   ) ) )
   $\HoareCSep$
true
\end{lstlisting}


\subsubsection{Error Exit}

Line 23 takes an error exit if the result is true.  Have to deal with those later.

\subsection{Step 10}

Line 24
%
\lstinline+res := escrowGoods.deposit(amt, sellerGoods)+
%
is reached only if step 9 returned true, can be handled in a similar
way to step 9. We again use \ruleN{cond-1} and \ruleN{sequence}:

\begin{lstlisting}[escapechar=@]
(escrowMoney $\obeys$ ValidPurse $\wedge$ escrowGoods $\obeys$ ValidPurse) $\longrightarrow$ (
    // TRUST
    CanTrade(escrowMoney,buyerMoney)  $\wedge$
    // FUNCTIONAL
    0$\leq$price$\leq$buyerMoney.balance$\init$ $\wedge$
    escrowMoney.balance=price $\wedge$
    buyerMoney.balance=buyerMoney.balance$\init$-price  $\wedge$
    //RISK
    $\forall$p.[ p$\obeys$$\pre$ValidPurse $\wedge$ p$\notin\{$escrowMoney,buyerMoney$\}\,\rightarrow$
       p.balance=p.balance$\pre$ ]  $\wedge$
    $\forall$o:$\pre$Object. $\forall$p$\obeys$$\pre$ValidPurse.
       [ $\MayAccess$(o,p) $\rightarrow$ $\MayAccess\pre$(o,p) ] )
       @\textbf{    \{  res := escrowGoods.deposit(amt, sellerGoods) \} }@
(escrowMoney $\obeys$ ValidPurse $\wedge$ escrowGoods $\obeys\pre$ ValidPurse $\wedge$  res $\rightarrow$ (
    // TRUST
    CanTrade(escrowGoods,sellerGoods)$\PRE{}$  $\wedge$
    // FUNCTIONAL
    0$\leq$amt$\leq$sellerGoods.balance$\init\ \wedge$
    escrowGoods.balance=amt $\wedge$
    sellerGoods.balance=sellerGoods.balance$\init$-amt  $\wedge$
    //RISK
    $\forall$p.[ p$\obeys$$\pre$ValidPurse $\wedge$ p$\notin\{$escrowGoods,sellerGoods$\}\,\rightarrow$
       p.balance=p.balance$\init$ ]  $\wedge$
    $\forall$o:$\init$Object. $\forall$p$\obeys$$\init$ValidPurse.
       [ $\MayAccess$(o,p) $\rightarrow$ $\MayAccess\init$(o,p) ]    $\wedge$
    escrowMoney.balance=price $\wedge$  buyerMoney.balance=buyerMoney.balance$\init$-price //implied by the above line ) )
$\wedge$
(escrowGoods $\obeys$ ValidPurse $\wedge$  $\neg$res $\rightarrow$ (
    // TRUST and FUNCTIONAL
    $\neg$[ CanTrade(escrowGoods,sellerGoods)$\PRE{}$ $\wedge$ 0$\leq$amt$\leq$sellerGoods.balance$\init{}$ ] $\wedge$
    // RISK
    $\forall$p.[ p$\obeys\PRE{}$ValidPurse$\,\rightarrow\,$ p.balance=p.balance$\pre$ ] $\wedge$  //cannot be init
    $\forall$o:$\pre$Object. $\forall$p$\obeys$$\init$ValidPurse.
       [ $\MayAccess$(o,p) $\rightarrow$ $\MayAccess\init$(o,p) ]   ) )
   $\HoareCSep$
true
\end{lstlisting}

A reminder again that I'm ignoring the residual risk: this is the same as calculated in 4M above for money,
replicated at 8M for goods if anyone cares\ldots.

\subsection{Step 11}

If step 10 has failed (returned false) then goods have failed to be moved into the escrow,
so the amount has not been escrowed from \lstinline+sellerGoods+ but
because step 9 has completed we have
%
\lstinline+buyerMoney.balance=buyerMoney.balance$\init$-price+
%
. We need to undo that transaction, which is what step 11 does;
by \ruleN{sequence} and \ruleN{cond-1} --- it is nothing but the reverse of step 9.
Note know this transction will succeed because we know \lstinline+escrowMoney.balance+ contains \lstinline+price+
from step 9, preserved through step 10.   We also use the twoway trust to talk about \lstinline+buyerGoods+ rather than
\lstinline+escrowGoods+, ditto for money (11M):

\begin{lstlisting}[escapechar=@]
(escrowMoney $\obeys$ ValidPurse $\wedge$ escrowGoods $\obeys$ ValidPurse) $\longrightarrow$ (
    // REASON FOR FAILURE (failure conjunct of Step 10 postcondition)
    $\neg$[ CanTrade(buyerGoods,sellerGoods)$\PRE{}$ $\wedge$ 0$\leq$amt$\leq$sellerGoods.balance$\init{}$ ] $\wedge$
    // TRUST
    CanTrade(sellerMoney,buyerMoney)  $\wedge$
    // FUNCTIONAL
    0$\leq$price$\leq$buyerMoney.balance$\init$ $\wedge$
    escrowMoney.balance=price $\wedge$
    buyerMoney.balance=buyerMoney.balance$\init$-price  $\wedge$
    //RISK
    $\forall$p.[ p$\obeys$$\pre$ValidPurse $\wedge$ p$\notin\{$escrowMoney,buyerMoney$\}\,\rightarrow$
       p.balance=p.balance$\init$ ]  $\wedge$
    $\forall$o:$\pre$Object. $\forall$p$\obeys$$\pre$ValidPurse.
       [ $\MayAccess$(o,p) $\rightarrow$ $\MayAccess\init$(o,p) ] )
       @\textbf{ \{ res := escrowMoney.deposit(price, buyerMoney) \} }@
(escrowMoney $\obeys$ ValidPurse $\wedge$ escrowGoods $\obeys$ ValidPurse) $\longrightarrow$ (
    // REASON FOR FAILURE (failure conjunct of Step 10 postcondition)
    $\neg$[ CanTrade(buyerGoods,sellerGoods)$\PRE{}$ $\wedge$ 0$\leq$amt$\leq$sellerGoods.balance$\init{}$ ] $\wedge$
    // TRUST
    CanTrade(sellerMoney,buyerMoney)$\pre$  $\wedge$
    // FUNCTIONAL
    0$\leq$price$\leq$buyerMoney.balance$\init$ $\wedge$
    escrowMoney.balance=0 $\wedge$
    buyerMoney.balance=buyerMoney.balance$\init$  $\wedge$
    //RISK
    $\forall$p.[ p$\obeys$$\pre$ValidPurse $\wedge$ p$\notin\{$escrowMoney,buyerMoney$\}\,\rightarrow$
       p.balance=p.balance$\pre$ ]  $\wedge$
    $\forall$o:$\pre$Object. $\forall$p$\obeys$$\pre$ValidPurse.
       [ $\MayAccess$(o,p) $\rightarrow$ $\MayAccess\pre$(o,p) ] )
   $\HoareCSep$
true
\end{lstlisting}

\noindent which ends up saying nothing has changed at all.
Then by a series of applications of \ruleN{cond-2} \lstinline+deal+ will return \lstinline+false+ and,
with the postcondition (11):

\begin{lstlisting}[escapechar=@]
true
       @\textbf{ \{ res := deal \} }@
(sellerMoney $\obeys$ ValidPurse $\wedge$ buyerGoods $\obeys$ ValidPurse) $\longrightarrow$ (
    // REASON FOR FAILURE (failure conjunct of Step 10 postcondition)
    $\neg$[ CanTrade(buyerGoods,sellerGoods)$\PRE{}$ $\wedge$ 0$\leq$amt$\leq$sellerGoods.balance$\init{}$ ] $\wedge$
    // TRUST
    CanTrade(sellerMoney,buyerMoney)$\pre$  $\wedge$
    // FUNCTIONAL
    0$\leq$price$\leq$buyerMoney.balance$\init$ $\wedge$
    //RISK
    $\forall$p.[ p$\obeys\init{}$ValidPurse$\,\rightarrow\,$ p.balance=p.balance$\init$ ] $\wedge$
    $\forall$o:$\pre$Object. $\forall$p$\obeys$$\init$ValidPurse.
       [ $\MayAccess$(o,p) $\rightarrow$ $\MayAccess\init$(o,p) ]   ) )
   $\HoareCSep$
true
\end{lstlisting}

This is the nastiest subcase of the second case of the \lstinline+ValidEscrow+ specification,
because we had to reverse out the escrow. The other subcases are that the Step 9 money transfer fails:
we just finish with the poscondition from step 9, which just adds the reason for failure to the postcondition above.

\begin{lstlisting}
    $\neg$[ CanTrade(sellerMoney,buyerMoney)$\PRE{}$ $\wedge$ 0$\leq$price$\leq$buyerMoney.balance$\PRE{}$ ] $\wedge$
\end{lstlisting}

I'm pretty sure that gets us to case 2 of \lstinline+ValidEscrow+

\subsection{Steps 12 and 13}

Steps 9 and 10 above exemplify the reasoning necessary to prove steps
12 and 13; as with step 11 we know both deposits will succeed because
both goods and money have been successfully escrowed (postcondition of
step 10; preservation of balance when dealing with trusted purses).

I'm pretty sure that gets us to case 1 of \lstinline+ValidEscrow+

\subsection{Error Exits}

Errors exits are covered by unconditional risk e.g. shown in the postconditions of 4M.
Errors where once side fails to validate will be caught at steps 1-8;
That seems to get us to case 3 of \lstinline+ValidEscrow+.  I think.

\subsection{Fool me twice, shame on me}

Finally case 4 relies that all the postconditions are conditional on whether some purse objets its spec.
In such cases (hah!) the residual risk (see 4M above) applies.

\kjx{I can think a bit more about these if someone sanity-checks the above}

%\section{June 2016 PurseProof2}

We start by showing the specification for \prg{ValidMint}s. We require BLA, BLA ...

\begin{figure*}[hbt]
\begin{lstlisting}[escapechar=&]
specification ValidPurse {
  ghost field balance // Number


  policy Purse_wf 
      m $\obeys$ Mint $\wedge$ amt$\in \mathbb{N}$
        &\textbf{ \{ res := new This(amt, m) \} }&
      res $\obeys$ ValidPurse 

  abstract predicate CanTrade(p1,p2)

  policy Pol_deposit_1     //   1$^{st}$ case:
      amt$\in \mathbb{N}$
        &\textbf{ \{ res := this.deposit(amt, src) \} }&
      res $\rightarrow$ (
          // TRUST
          CanTrade(this,src)$\PRE{}$  $\wedge$
          // FUNCTIONAL  
          0$\leq$amt$\leq$src.balance$\PRE{}\ \wedge$
          this.balance=this.balance$\PRE$+amt $\wedge$
          src.balance=src.balance$\PRE$-amt  $\wedge$
          //RISK
          $\forall$p.[ p$\obeys$$\pre$ValidPurse $\wedge$ p$\notin\{$this,src$\}\,\rightarrow$
               p.balance=p.balance$\pre$ ]  $\wedge$
          $\forall$o:$\pre$Object. $\forall$p$\obeys$$\pre$ValidPurse.
               [ $\MayAccess$(o,p) $\rightarrow$ $\MayAccess\pre$(o,p) ]   )

  policy Pol_deposit_2     //   2$^{nd}$ case:
      amt$\in \mathbb{N}$
        &\textbf{ \{ res := this.deposit(amt, src) \} }&
       $\neg$res $\rightarrow$ (
          // TRUST and FUNCTIONAL  
          $\neg$[ CanTrade(this,src)$\PRE{}$ $\wedge$ 0$\leq$amt$\leq$src.balance$\PRE{}$ ] $\wedge$
          // RISK
          $\forall$p.[ p$\obeys\PRE{}$ValidPurse$\,\rightarrow\,$ p.balance=p.balance$\pre$ ] $\wedge$
          $\forall$o:$\pre$Object. $\forall$p$\obeys$$\pre$ValidPurse.
               [ $\MayAccess$(o,p) $\rightarrow$ $\MayAccess\pre$(o,p) ]   )
\end{lstlisting}
\caption{\prg{ValidPurse} specification now with extra stuff
  \kjx{can't see the extra stuff}}
\label{fig:ValidPurseX}
\end{figure*}




\begin{figure*}[hbt]
\begin{lstlisting}[escapechar=&]
specification ValidLedger {
  ghost field map // Map[[Purse, Number]]

  policy Ledger_wf 
      true
        &\textbf{ \{ res := new This \} }&
      res $\obeys$ ValidLedger $\wedge$ res.map = $\emptyset$
  
  policy Ledger_get1
      true
        &\textbf{ \{ res := this.get(k) \} }&
      (k,v) $\in$ map $\rightarrow$ res = v

  policy Ledger_get2
      true
        &\textbf{ \{ res := this.get(k) \} }&
      (k,v) $\not\in$ map $\rightarrow$ res = null

  policy Ledger_put1
      true
        &\textbf{ \{ res := this.put(k,v) \} }&
      res = ((k,v) $\in$ map) 

  policy Ledger_put2
      true
        &\textbf{ \{ res := this.put(k,v) \} }&
      $\forall$j $\in$ &\textit{dom}&(map$\pre$) : ((k,v) $\in$ map = (k,v) $\in$ map$\pre$) $\wedge$ (j = k)

  policy Ledger_contains 
      true
        &\textbf{ \{ res := this.contains(k) \} }&
      res = (k $\in$ &\textit{dom}&(map))
\end{lstlisting}
\caption{\prg{ValidLedger} specification?}
\label{fig:ValidLedger}
\end{figure*}



\begin{minipage}{\textwidth}
\begin{lstlisting}
 specification ValidMint {

   abstract predicate MyPurses(m,p);
   
   policy My_Purses_CanTrade
		$\forall$ p1, p2. MyPurses(this,p1) $\wedge$ MyPurses(this,p2) $\rightarrow$ CanTrade(p1,p2)
   
   policy My_Purses_Valid
		$\forall$ p. MyPurses(this,p)  $\rightarrow$ p $\obeys$ ValidPurse

   policy Purse_creation
        amount: Number
   	     { res = this.newPurse(amount )  }
        MyPurses(prs)

   policy Valid_deposit
        amount: Number
              { res = this.deposit(to, amount, from)  }
        res $\rightarrow$ MyPurses(this,to) $\wedge$ MyPurses(this,from) $\wedge$ amount < ... $\wedge$ ....     

   policy Valid_deposit
        amount: Number
              { res = this.deposit(to, amount, from)  }
        $\neg$ (res) -> $\neg$( MyPurses(this,to) $\wedge$ MyPurses(this,from) $\wedge$ amount < ... $\wedge$ ....     )
                               $\wedge$ "no effect outside Bank"

    policy Valid_balance_1
        true
            { res =balance(prs) }
        res $\in$ Number $\leftrightarrow$ prs $\obeys$ ValidPurse  
  
    policy Sprout
        true
          { res = this.sprout }
      $\neg$MyPurses(thus,res)$\pre$ $\wedge$   MyPurses(this,res)  $\wedge$ res.balance = 0
  
    policy Create
    true
          { res=This.new  }
    res $\obeys$ ValidMint

}
\end{lstlisting}
\end{minipage}
 



%
%\cite{swapsiesGotGotNeed}


\setlength{\bibsep}{0.0pt}
\bibliographystyle{abbrv} % {plainnat}
\bibliography{Case}

\end{document}
%   \section{Material which may be removed or needs to be reorganised}

\subsection{TODO: reogranize}

TODO: some of it may go to section 4, and some to related work.

To specify capability policies, we must be able to specify both rely
and deny policies.  For \rely~properties, we can draw from
specification languages for functional properties, {\it e.g.}
JML~\cite{Leavens-etal07}, or
separation-logic-based~\cite{IDF,MattAlex}, enhanced so as to also
talk about indirect properties.



The persistent nature of \deny~properties necessitates  refining the
relations between different instants in time,
\eg a change in the balance of a purse
is preceded by a transfer, which in its turn, again, is preceded by
the mint creating  the two purses. In more detail:
if the balance of a purse of $p1$ decreases by $amt$ over its immediately previous value, then the immediately preceding step executed $p1.deposit(p2,mt)$,  and at some times prior to that step,  the purses $p1$ and $p2$ were created by  the same mint. The annotation $val_{prev}$ is meant to indicate a value  {\em immediately before} the event in question, and the annotation $\theta_{prec}$ is meant to indicate an event  {\em immediately preceding} the event in question. Thus, we express this policy as follows:\\


Another facet of \deny~properties is the different modes of
causality. For example, does {\bf Pol\_2} mean that a change in the currency implies that
the mint object was accessible, or, more strongly, that the mint executed a method?
Classical approaches to ownership, for example, support the latter
approach\cite{ownVerif}.
%
%

Object invariants \cite{Mey88,Parkinson07,pubsdoc:invariants-iwaco09}  are relevant, % in specifications,
 {\it e.g.} an object's sealer and unsealer must come from the same mint.
%  $\forall p1,p2: p1.sealer==p2.sealer \rightarrow p1.unselarer==p2.unsealer$.
Monotonic properties are relevant too, \eg   {\bf Pol\_3} says that the currency can only grow. Such properties
are akin to history invariants \cite{usinghistory}.
Accommodating for object and history invariants  poses the challenge of deciding
 at which point they may be broken/must be restored \cite{BarnettNaumann04,objInvars};
%we will apply our experience from
% surveying known approaches and establishing soundness
known approaches  follow different, but fixed rules~\cite{DrossoFrancaMuellerSummers08}, we shall investigate whether the rules could
be part of user-defined policies.
%



To address the crucial issue of capabilities leaking from trusted to
untrusted code, we can apply techniques drawn from ownership types~\cite{ClaPotNobOOPSLA98,dynamicOwn,multiple,rolesForOwners}.
Ownership types restrict heap topology to manage access between
objects,
and have generally been used to support encapsulation and
concurrency.
By applying ownership to capabilities, we
will be able to support many deny policies directly:
{\bf Pol\_4} and {\bf Pol\_5}, for example, or the policy of a client
object of the currency system: that if an object owns its purse,
no other objects should be able to access that purse.



We also expect to be able to employ
% We will develop
effects systems~\cite{LuPotPOPL06} to restrict interactions between
sets of object, \eg $\forall m,m':Mint. m\neq m' \rightarrow m.Purse()
\# m'.Purse()$ says that \prg{Purse}s created by different \prg{Mint}s
will not affect each other. These systems can expand our earlier work
on effects and isolation \cite{ClaDroOOPLSA02,rolesForOwners}.



%
%In \cite{LinearLogicKnowledge} support for linear and non-linear assertions, principals who affirm facts,
%while in \cite{Bauer07consumablecredentials} actions require credentials, and credentials may be consumed.
%The above systems only specify the properties of a system, but do not prove that code has these properties.
%These works define protocols, and prove invariants of the protocol ({\it e.g.} that the sum on moneys in a bak are constant). They do not, however, check that a set of methods indeed, adhere to that protocol, nor do they prove that this set is tamper-proof, in these sense that further extensions to that set might break the protocol.
%
%Also provenance: in the case of the confused deputy, we could specify which are the only lines of code which may affect the contents of the statistics file.
%
%
%Also use \cite{murray10dphil} where they also specify situations which will {\em not} happen, ext 2.7.1.


We expect to define the semantics of the specification language by
means of satisfiability of assertions in the context of a given stack
and heap \cite{MattAlex}.
%For the various possible meanings of encapsulation, we will  build on our earlier work on encapsulation in the presence of universe types. \cite{pubsdoc:universe-types-encapsulation}.
 For \deny~policies, we will  have to expand the approach,
 define satisfiability over the history of executions.~ \cite{monitoring}.
 % for the causal policies we will draw on our experience from  \cite{pubsdoc:universe-types-encapsulation}.
 %
% The specification language may initially come in two flavours: one for the Java-version, and one for the dynamic language, but we hope that in large parts they will coincide.

\subsection{Language Features}

We are also considering the extent to which particular language
constructs can support reasoning about policies --- both for extant
features and potential novel features.  We have already seen how
reasoning about object-capability programming in the class-based Java
style (in Fig.1) differs in some important respects from a
lexically-scoped E style (in Fig.2): we would like to extend this
analysis to understand particular constructs in more detail.

We have begun collecting programming language idioms   often used in capability programs
(\eg sealers, revocation, membranes \cite{membranes,membranesJavascript,MillerPhD}, {\em e.t.c.}), and
 identify idioms which have the same effect ({\em e.g.,} the use of field \prg{mint}  in Fig.1 has the same effect as that of \prg{sealer}/\prg{unsealer} in Fig.2).
%
We will % try to
lift idioms to more succinct, abstract language features.

% Such an idiom is
Consider the use of the field \prg{mint} in the code in Fig.1: its
purpose is to ensure
that  no transactions involve \prg{Purse}s from different  \prg{Mint}s.  This is enforced through  the \prg{private} % field
annotation (line 4), % for \prg{mint}, the  %field
 initialisation (lines 9 and 13), and check (line  17).  The idiom
 would be directly expressible
  more directly in a variation of ownership
  types~\cite{ClaPotNobOOPSLA98} which allowed for dynamic checks for
  owners~\cite{dynamicOwn,aliasICSE2013}. Making  the \prg{Mint} the owner of the  \prg{Purse}s and
  replacing  the field declaration, the initialisation, and the check
  mentioned above through one type argument to \prg{Mint}   would reduce
  the code by 30\%.
 % The variation of ownership types mentioned here will be developed in this workpackage.
Crucially, it would also prevent a purse from
ever leaking its mint capability to an untrusted object.

An  extension of the money example is that \prg{Purse}s should belong
to \prg{Person}s, and that  \prg{Person}s should not have access to   \prg{Purse}s belonging
to  other  \prg{Person}s.
Thus, a person    \prg{p1}  wanting to pay person  \prg{p2},  could create a \prg{Purse}, pay some amount into it,
and then make it belonged to \prg{p2}, thus ensuring that the purse can be safely passed around and not be tampered with.
% object should be encapsulated within a  \prg{Person} object.
%This is another application of ownership types, which gives us that a \prg{Purse} has a \prg{Mint} as well as
This can be modelled by  multiple ownership, here with a \prg{Mint}
and a \prg{Person} owner \cite{multiple,rolesForOwners}
although we need to discriminate
to allow for different treatment of, and different roles for, the
different owners: the \prg{Person} owner guarantees encapsulation, is
checked statically, and is  mutable, while the  \prg{Mint} owner makes
no encapsulation guarantees, is checked dynamically, and is
immutable.
%Such combinations of roles for owners is novel territory for programming languages.
%
%Observe that
Encapsulation is often implicitly present in programs written in
dynamic languages: In Fig.2, all \prg{purse}s created by the same \prg{
  mint} share, and do not leak further, the same \prg{sealer}/\prg{
  unsealer} pair. The code could be made more succinct and more
abstract --- not to mention more secure --- with dynamically checked owners\cite{dynamicOwn}.



More generally, we are interested in the role of restrictive features in supporting \deny~policies. We expect to expand these features and allow  dynamically enforced versions of the restrictions: dynamic application and revocation of the restrictions,  dynamic  linking of ownership domains\cite{ODs},  dynamic merger or dissolution of ownership boxes, \etc Such features have their counterpart in the dynamic treatment of
 capabilities,\eg revocation, membranes, and proxies
 \cite{membranes,membranesJavascript,proxiesECOOP2013}.
These kind of  dynamically enforced  properties inspired by static type
systems seems to offer interesting opportunities for cross-pollination
between  static and dynamic  languages.

% We will work on more explicit connections between policy and code, and  will develop special   annotations to identify which parts of the
% code support which policies.
% % Later on,
% % Taking inspiration from AOP \cite{AOP},
% %we will try to develop  powerful weaving operations, which will allow  separation of the concerns policy code separately from   functionality code, so that the
% %"pre-weave" code is clear and explicit, while the "woven" code can be implicit and tangled.

% \Methodology  We will  determine sufficiently interesting  subsets Joe-E and Grace, % , for which
%  develop operational semantics,  a type system of  Joe-E,
%  and  build  on the subsets to  formalize the new language features.
% We will be discussing language features with the Grace team.





