\section{Conclusions and Further Work}
\label{section:conclusion}

In this paper we addressed the questions of specification of risk, trust, and reasoning about such specifications. To answer these questions, we contributed:
\begin{itemize}
\item {\em Hypothetical} predicates $\obeys$ to model trust,
  $\MayAccess$ and $\MayAffect$ to model risk, and their formal semantics.
\item {\em Open Assertions} and {\em Open Policies} whose validity
  must be guaranteed, even when linked with  {\em any} other code.
\item {\em Formal models} of \LangOO\, and \Chainmail.
\item \kjx{{\em Hoare four-tuples} that make invariants explicit.}  
\item \scd{A {\em Hoare logic} incorporating code agnostic inference rules.}
\item \tobym{\emph{Formal reasoning} to prove key steps of the Escrow Exchange.}
\end{itemize}



%In this paper we have shown how explicitly modelling risk and trust
%supports reasoning about object-capability programs in an open world.
%We introduced the $\obeys$ predicate to model trust, and the
%$\MayAccess$ and $\MayAffect$ predicates to model risk. We showed how
%these constructs can be used to specify and argue informally about the
%design of escrow exchanges ---  real-life grown-up, swapsies
%--- using hypothetical and conditional reasoning.  Finally we defined
%the featherweight programming and specification languages \LangOO\ and
%\Chainmail, and use them to prove that the escrow implementation based
%on mutual trust meets its specification.

%%
%%% James Killed before POST
%%
%
% During this work, we considered  encapsulation as a low-level mechanism, which
%  should be transparent to specifications.   To our surprise we have observed multiple times that
% unless encapsulation  percolates to the specification level, specifications end up being too weak and wordy.
%
\noindent In further work we will extend our approach 
to deal with
concurrency, distribution, exceptions, networking, aliasing, and
encapsulation.
%What if the system is distributed
%(e.g.\ each participant on different host). What if the participants
%are distributed each over several hosts? And what if we have a
%participant distributed over several hosts and perhaps not all of them
%trusted.
Finally, we hope to develop automated reasoning
techniques to make these kinds of specifications practically useful.


% Discuss the unsatisfactory parts of the spec, and of Pol\_7. Dealing
% with these is further work.  Further work: concurrency, more cases,
% more details, automated reasoning

% Further work: concurrency, more cases, more details, automated
% reasoning.

%%%
%%% thisis really good but not enough time to handle properly
%%%
% The specification from figure \ref{fig:JoeEPurse} does not describe
% what changes the call of \prg{Escrow::deal} may make to
% accessibility. This is important, because after the call of deal we
% need to know what extra knowledge the potentially malicious
% participants have given to each other. In particular, we want to be
% able to ascertain that the malicious participants do not gain more
% accessibility than the sum of what was accessible to them before the
% call.
