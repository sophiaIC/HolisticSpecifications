\section{DOM Wrappers -- code and traditional Specification}
\label{DOM:traditional}

\kjx{this stuff should go with the DOM example,
  either in motivation, or as second example after formalism?}

\kjx{merge in when we first talk about language:}
We write our code in a small, fictitious, class-based, dynamically
typed, imperative programming language. All functions are public, \ie
may be called by any other object, and all fields are private, \ie may
be read or written only by the object itself. 

\begin{figure}[htb]
\begin{tabular}{llll}
&
\begin{minipage}{0.45\textwidth}
\begin{lstlisting}
class Node(par,prop){

  fld parnt = par;
  fld property = prop;
  fld children = new array[10];
  fld maxChild=0;
  children[maxChild++]=this
  
  func getParent(){
       parnt 
  }  
  func getChild(i){
       children[i] 
  }
  func setProperty(prp){
       property = prp 
  }  
  func setChild(nd){
       children[i] = nd
  }  
}
\end{lstlisting}
\end{minipage}
& & 
\begin{minipage}{0.40\textwidth}
\begin{lstlisting}
class Wrapper(nd,hgt){

  fld node=nd;
  fld height=hgt;

  func setPropety(i,prp){
    if (i>height){ 
       return 
    } 
    else  
    {  nd=node;  
       while (i>0){
          nd=nd.getParent();
          i--;    };
        nd.setProperty(prp); }
  }    
  func getChild(i){ 
    Wrapper(node.getChild(i),
                    height+1); 
 }                           
}
\end{lstlisting}
\end{minipage}
\end{tabular}
 \vspace*{-7mm}
\caption{\prg{Node}  and \prg{Wrapper} }
\label{fig:DOM}
\end{figure}

The figure needs some beautification

\begin{figure}[htb]
\begin{lstlisting}
class Wrapper(nd,hgt){

  fld node=nd;
  fld height=hgt;

  func setPropety(i,prp)
  // PRE:  true
  // POST:
  //     (  i not a number or i>this.height ) --> modifies nothing  
  //     &&
  //     i<=this.height -->  (  modifies = { this.node.parnt^i.property }
  //                          &&
  //                          this.node,parnt^i.property == prp )
  {
    if (i>height){     return     } 
    else  
    {  nd=node;  
       while (i>0){   nd=nd.getParent();  i--;    };
        nd.setProperty(prp); }
  }    
  func getChild(i)
  // PRE:  true
  // POST:  modifies: nothing
  // RETURNS: i-th child of this
  
  {    Wrapper(node.getChild(i),   height+1);   }                           
}
\end{lstlisting}
 \vspace*{-7mm}
\caption{\prg{Wrapper} specification}
\label{fig:WrapperSpec}
\end{figure}

In Figure \ref{fig:WrapperSpec} we  outline a specification of the class \prg{Wrapper} in the ``traditional style''. Each function has a pore and a post conditon. Because the functions may be called by any, unchecked code, the preconditions are not stronger than \prg{true}. The functions need to behave correctly under all possible circumstances.  The post-condition of \prg{setPropery} guarnatees that if the argument is not a number or if it is larger than the \prg{height} of the wrapper, then nothing is modified. If on the other hand, the argument is smaller or equal to the \prg{height}, then the \prg{property} of the \prg{i}-th parent will be set to \prg{prp}, and nothing else will be modified.
