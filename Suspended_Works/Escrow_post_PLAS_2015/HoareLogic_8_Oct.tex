\section{Hoare Logic}

We define the Hoare Logic that allows us to prove adherence to policies. In order to reflect that the code to be verified is executed in an open system, and that it calls code whose specification and trustworthiness is unknown to the code being verified\footnote{TODO: Untanhgle the sentence}, we augment the Hoare triples, so that not only do they gurantee some property to hold {\em after} execution of the code, but also with the assertion that some property is preserved {\em during} execution of the code.\footnote{Shall we say that we augmented/adapted the notion of invariant from object invariant to ``code invariant''?}
  
Our Hoare tuples have the format \HoareExpl{\P} {\prg{code}} {\M} {\Q}
{\Q'}, and it promises that ... in the presence of code and predicates
as defined in \M. When the module \M\ is implicit from the context, or
???\footnote{of no particular importance?} we use the shorthand
\Hoare{\P} {\prg{code}}  {\Q} {\Q'}.\footnote{TO-DISCUSS: Is it OK to
  have an implicit module \M in the Hoare logic, esp in view of the
  lemma \ref{lemma:hl:preserve}? \kjx{Yes! Makes it look much nicer,
    isn't a problem to think about}}



\begin{definition}[Validity of Hoare Tuples] 
$ $ \\

  \begin{itemize}
  \item
$M  \models \HoareImpl{\P} {\prg{code}} {\Q} {\Q'}  $ \ \ \  iff\\
\SP $\forall \M',\sigma.$\\
\SP $ (\sigma,\_)\in \Arising(\M*\M')\ \wedge \M*\M', \sigma \models \P$\\
\SP \ \ \ $\longrightarrow$ \ \ \ \\
\SP \ \ \  $(\M\!*\!\M', \sigma, \prg{code}  \leadsto \prg{res}, \sigma'   \ \rightarrow  \  \M\!*\!\M'\!, \sigma, \sigma' \models \Q\, )$\\
\SP\ \ \   $\wedge$ \\
\SP\ \ \  $\forall \sigma'\!\in\!\Reach(M,\sigma,\code).\ \M\!*\!\M', \sigma, \sigma' \models \Q'   $

\end{itemize}
\end{definition}




We define the Hoare rules in figure
\ref{fig:HoareLogic}. \footnote{TODO: The str-1 ... str-2 rules are
  only a rough sketch. Perhaps they can and should be amalgamated into
  one rule. Also, perhaps there are more such rules.}
\footnote{\toby{rule \textsc{meth-call-2} is that just ``only connectivity
begets connectivity''?}}
\footnote{\toby{In rule \textsc{framing-1}, the definition of disjointness
(Definition~\autoref{defn:disjointness}) talks only about initial and final states, 
not intermediate ones. So I fear it is not strong enough to help reason
about preservation of the invariants. However, it perhaps is strong enough
 to prove the version of \textsc{framing-1} that Sophia sent in her email
on October 6th.}}
\footnote{TODISCUSS: Rule FRAMING-1 is a pain, as it requires
  knowledge about \prg{code} as well as \Q'', but I do not see how
  this can be avoided. }
\footnote{TODISCUSS: Rule FRAMING-2 is at the
  heart of our work. I am surprised it looks so short. It is different
  from the one in the photo James and I had sent in July; the old one
  does not work. To check its application  in the proof steps for
  \prg{Escrow}.\kjx{arg, why do I feel I should think about this?  but
  really I don't want to, want to focus on stuff on the critical path
  with what time and energy I have}}\footnote{TODO: Say that  $\sigma'\!\in\!\Reach(M,\sigma,\code)$ is a shorthand for $\sigma'.  (\sigma',\_')\!\in\!\Reach(M,\sigma,\code)$.}
\footnote{\toby{In rule \textsc{varAss-2}, doesn't that rule require that
$Q$ to not mention the value of \texttt{v}?} Maybe for \textsc{varAss-1} too?}
\footnote{TODO: Explain that $PRE$ and $POST$ and $PRSRV$ look up the method's pre and post conditions}

We need the notion of disjointness, and of framing.
\begin{definition}[Disjointness]\label{defn:disjointness}
$ $ \\

$\M,\sigma \models \prg{code} \# \P$ iff \\
$\M,\sigma  \models \P$ \  $\wedge$ \  $\M,\sigma, \prg{code} \leadsto \sigma', \prg{v}$ \ \ $\rightarrow$ \ \ $\M,\sigma'  \models \P$
\end{definition}


We have the standard definitions of entailment of the form $\P\, \rightarrow_\M\, \P'$ and $\Q\, \rightarrow_\M\, \Q'$\, but we also need an unusual definition of entailment:\footnote{TODO: What do the others think?} of the form $\P, \Q\, \rightarrow_\M\, \P'$. The former two say .... the latter says ....

\begin{definition}[Entailment]
$ $ \\

\begin{itemize}
\item
$\P\, \rightarrow_\M\, \P'$\ \  iff\ \\
$\forall  \sigma$. $\M, \sigma \models  \P$    $\rightarrow$ $\M,  \sigma \models  \P'$
\item
$\Q\, \rightarrow_\M\, \Q'$\ \  iff\ \\
$\forall  \sigma, \sigma'$.  $\M, \sigma, \sigma' \models  \Q$ $\rightarrow$ $\M, \sigma, \sigma' \models  \Q'$ 
\item
$\P, \Q\, \rightarrow_\M\, \Q'$\ \  iff\ \\
$\forall  \sigma, \sigma'$. $\M, \sigma \models  \P$  $\wedge$ $\M, \sigma, \sigma' \models  \Q$ $\rightarrow$ $\M, \sigma, \sigma' \models  \Q'$
\end{itemize}
\end{definition}

\newcommand{\HoareFigOne}
{
\begin{figure*}
$
\begin{array}{l}
\begin{array}{lcl}
\inferenceruleN {varAsg} {
} {
     \Hoare   {\kw{true}}
     {\kw{var}\,  \prg{v} \kw{:=} \prg{a} } { \prg{v}=\prg{a}\pre } {\kw{true}}
\\
     \Hoare   {\kw{true}}
      {  \prg{v} \kw{:=} \prg{a} } { \prg{v}=\prg{a}\pre }  {\kw{true}}
}
& &
\inferenceruleN {fieldAsg} {

}{
     \Hoare
     {\kw{true}}
      { \kw{this.}\prg{f}  \kw{:=} \prg{a} }
      { \kw{this.}\prg{f}   = \prg{a}\pre}
       {\kw{true}}
}
\\
\\
\inferenceruleN {cond-1} {
 \A \rightarrow_{\M} \prg{cond} \\
  \Hoare  {\A} {  \prg{stmts}_1\  } {\B} {\B' }
} {
   \Hoare  {\A} { {\kw{if}}\, \prg{cond}\,  {\kw{then}}\, \prg{stmts}_1\, {\kw{else}}\, \prg{stmts}_2  } {\B} {\B'}
   }
& &
\inferenceruleN {cond-2} {
 \A \rightarrow_{\M} \neg\prg{cond} \\
  \Hoare  {\A} {  \prg{stmts}_2\  } {\B} {\B'}
} {
   \Hoare  {\A} { {\kw{if}}\, \prg{cond}\,  {\kw{then}}\, \prg{stmts}_1\, {\kw{else}}\, \prg{stmts}_2  } {\B} {\B'}
   }
   \\
   \\
   \inferenceruleN {skip} {
} {
   \Hoare  {\A} { {\kw{skip}} } {\A} {\kw{true}}
   }
\end{array}
\\
%\\
 {
  \inferenceruleNN {newObj} {
  \M( {\cal S}(\M,\prg{C})) \ = \   \kw{spec}\ S\  \lb\    ....,\  \A \,\lb \,  {\kw{new}(\prg{x}_1,... \prg{x}_n)}\, \rb\, \B,\  ....\  \rb
 \hspace{2in}

}{
       \Hoare
        {     \A[\prg{a}_1/\prg{x}_1,... \prg{a}_n/\prg{x}_n]  }
         { \prg{x}\,  \kw{:=} \, \kw{new}\lp \prg{a}_1... \prg{a}_n \rp }
         %{      \B[(\prg{a}_1)\pre/\prg{x}_1,... (\prg{a}_n)\pre/\prg{x}_n]  }
         {      \B[\prg{a}_1/\prg{x}_1,... \prg{a}_n/\prg{x}_n,\prg{x}/\prg{res}] 
                \ \wedge\ \prg{x}\obeys 
                 {\cal S}(\M,\prg{C}) }
          { \kw{true} }
}
}
 % OLD VERSION
%{
% \inferenceruleNN {newObj} {
%    \prg{f}_1,... \prg{f}_n \mbox{ are the fields of } \M(\prg{C})
%\hspace{2in}
%  \prg{S}\in  {\cal S}(\M,\prg{C})
%}{
%      \Hoare
%       { \kw{true} }
%    { \prg{x}\,  \kw{:=} \, \kw{new}\, \prg{C}\lp \prg{a}_1... \prg{a}_n \rp }
%   { \prg{x.f}_{1}=(\prg{a}_1)\pre \, \wedge ...  \wedge\, \prg{x.f}_{n}=(\prg{a}_n)\pre
%    \  \wedge\ \prg{x}: \prg{C}\ \ \wedge\ \   \prg{x} \obeys \prg{S}  }
%       { \kw{true} }
%}
%}
\\ \\
  \inferenceruleNN {meth-call-1} {
     \M(S) \,= \,  \kw{spec}\ S\  \lb\  \overline{Policy}, \A \,\lb \, \prg{this.m(par)}\, \rb\, \B, \overline{Policy'}\  \rb
% \SP\SP \SP\SP\SP
  %  \B''[\prg{x}/\prg{this},\prg{y}/\prg{par}] = \B'''
    } {
    \Hoare  {\prg{x}\obeys S\wedge  \A[\prg{x}/\prg{this},\prg{y}/\prg{par}]} {\prg{v} \, \kw{:=}\, \prg{x.m(y)}} {\B[\prg{x}/\prg{this},\prg{y}/\prg{par},\prg{v}/\prg{res}]}   {{\kw{true}}} % {\B'''}
}
  \\
  \\
  \\
   \inferenceruleNN {meth-call-2} {
         \B \equiv \forall \prg{z}:\pre \Obj.\ [ \ \MayAccess({\prg{v},\prg{z}}) \rightarrow
    (\, \MayAccess\pre(\prg{x},\prg{z})\, \vee \, \MayAccess\pre(\prg{y},\prg{z})\, )\ \ ] \\
   \B' \equiv   \forall \prg{z},\prg{u}:\pre \Obj. \ [ \ \ \MayAccess({\prg{u},\prg{z}}) \rightarrow \\
         ~   \hspace{2.3in}  ( \MayAccess\pre(\prg{u},\prg{z}) \ \ \ \vee  \\
                   ~   \hspace{2.7in}[\  (\MayAccess\pre({\prg{x},\prg{z}})\vee \MayAccess\pre({\prg{y},\prg{z}}) )\ \wedge \\
                   ~   \hspace{2.8in}  (\MayAccess\pre({\prg{x},\prg{u}})\vee \MayAccess\pre({\prg{y},\prg{u}}) ) \ ] \ ) \ \  ]\\
   ~   \hspace{0.25in} \wedge \\
   ~   { \hspace{0.25in} \forall \prg{z}:\pre \Obj.\forall \prg{u}: \Obj.}\\
    ~   \hspace{1.7in}[\ \ { \New(\prg{u}) \wedge (\ \MayAccess(\prg{z},\prg{u})\vee\MayAccess(\prg{u},\prg{z})\ )}\ \  \ \rightarrow  \\
     ~   \hspace{2.6in} {(\ \MayAccess\pre(\prg{x},\prg{z})\vee\MayAccess\pre(\prg{y},\prg{z})\ ) }  \ \ \  ]
                 } {
    \Hoare  {\true} {\prg{v} \, \kw{:=}\, \prg{x.m(y)}} {\B}
           { \B'  }
}
\\ \\
   \inferenceruleNN {frame-methCall} {
           \Hoare  {\A \ {\wedge\ \prg{e}=\prg{e}} }   {\prg{x.m(y)}}  {\kw{true}}  {\forall \prg{z}. (\ \MayAffect(\prg{z},\prg{e})  \rightarrow \B'({\prg{z}})\ )\ \ \wedge }
          \\
% trick for layout
		\SP\SP\SP\SP\SP \SP\SP\SP\SP\SP \SP\SP\SP\SP\ \ \   \
		{\forall \prg{z}.(\ (\MayAccess\pre(\prg{x},\prg{z}) \vee \MayAccess\pre(\prg{y},\prg{z})
           \vee \New(\prg{z}) \ )%(\prg{z}:\prg{Object}\wedge \neg ( \prg{z}:\pre\prg{Object}) ) \ )
           \ \rightarrow\  \neg \B'(\prg{z})\ }
}
{
          \Hoare  {{\A}\ {\wedge\ \prg{e}=\prg{e}} } {\prg{x.m(y)}}   {\prg{e}=\prg{e}\pre}  {\kw{true}}
}
\\
\\
\inferenceruleNN{sequence} {
        \Hoare  {\A}{\stmts_1}{\B_1}{\B'}  \ \SP\SP\SP        \Hoare  {\A_2}{\stmts_2}{ \B_2}{\B'}
        % \\
       \ \SP\SP\SP
         {\A,\B_1\rightarrow_{\M,\O} \kw{true},\A_2}    \ \SP\SP\SP  {\B_1,\B_2\rightarrow_{\M,\O} \B}
        }
     {
       \Hoare  {\A}{\stmts_1; \, \stmts_2}{ \B}{\B'}
        }
\end{array}
$
\caption{Hoare Logic -- Basic rules of the language -- we assume that the module \M\ is globally given}
\label{fig:HoareLogicBasic}
\end{figure*}
}

\newcommand{\HoareFigTwo}
{
\begin{figure*}
$\begin{array}{l}
\begin{array}{lcl}
   \\
\\
   \inferenceruleN {frame-general} {
          \Hoare  {\A}{\stmts}{\B} {\B'}
           \\
        {\A  \rightarrow_{\M}   \stmts\disj  \A'     \SP\SP\SP\    \A  \rightarrow_{\M}   \stmts\ddisj  \A''}
} {
          \Hoare  {\A\wedge \A'} {\stmts} {\B\wedge{\A'}}  {\B' \wedge{\A''}}
}
& &
\inferenceruleN {Conj} {
 { \Hoare  {\A_1}{\stmts}{ \B_1}{\B_3}}\\
{  \Hoare  {\A_2}{\stmts}{ \B_2}{\B_4}}
} {
   { \Hoare  {\A_1\wedge \A2 }{\stmts}{ \B_1 \wedge \B_2 }{\B_3 \wedge \B_4}   }
 }
\\
\\
    \inferenceruleN {Cons-1} { % REFACTORED
          \Hoare  {\A}{\stmts}{ \B}{\B'} \SP\SP\SP {\A' \rightarrow_{\M} \A}
        \\
     {\B \rightarrow_{\M} \B''} \ \  \SP\SP\SP\SP\SP\SP\SP\SP {\B' \rightarrow_{\M} \B'''}
     }
     {
       \Hoare  {\A'}{\stmts}{ {\B''\wedge\B'}}{{\B'''}}
        }
&
&
    \inferenceruleN {Cons-2} {  % NEW
          \Hoare  {\A}{\stmts}{ \B}{\B''}   \\
           {\A',\B' \rightarrow_{\M} \A,\kw{true}}
     }
     {
       \Hoare  {\A'}{\stmts}{{\B' \rightarrow \B}}{\B''}
        }
%\\ \\
%\inferenceruleN {Cons-3}
%{ % was Cons-1
%  \Hoare  {\A}{\stmts}{ \B}{\B'} %
%  % SD put in Cons-4 \SP\SP\SP\SP\SP \toby{\A' \rightarrow_{\M} \A}
%  \\
% \A, \B \rightarrow_{\M} \kw{true},\A'
% % I droppped the \wedge \B', because Cons-1 does it
%} {
%   \Hoare  {{\A}}{\stmts}{ \A'}{\B'}
%   }
%& &
%  \inferenceruleN {Cons-4} { % was Cons-2
%          \Hoare  {\A}{\stmts}{ \A'}{\B'} % \SP\SP\SP \toby{\A'' \rightarrow_{\M} \A}
%        \\
%     \toby{{\A},\A' \rightarrow_{\M} \B} %\SP\SP\SP\SP\SP\SP\SP \toby{\B' \rightarrow_{\M} \B''}
%     }
%     {
%       \Hoare  {\A}{\stmts}{ \B}{\toby{\B'}}
%        }
   \\
   \\
 \inferenceruleN {code-invar-1} {
            { \M( S) \equiv \  \kw{spec}\ S\  \lb\  \overline{Policy}, P, \overline{Policy'}\  \rb}
  }
     {
       \Hoare   {{\kw{true}}} {\stmts} {{\kw{true}}} { {\prg{x} \obeys S} \rightarrow P[\prg{this}/\prg{x}]}
      }
    & &
  \inferenceruleN {code-invar-2} {

     }{
    \Hoare    {{e} \obeys S}  {\stmts}    {\kw{true}}  {{e}\pre \obeys S}
      }
  \\
   \\
  \inferenceruleN {quantifiers-1} {
             x \mbox{ does not appear in } \A, \mbox{ or } \B, \mbox{ or } \prg{stmts}
     \\
       \Hoare   {\A} {\stmts}{ \B'}{\B}
      }
   {
    \Hoare      {\A} {\stmts}{\forall x.\B'}{\B}
      }
           & &
  \inferenceruleN {quantifiers-2} {
             x \mbox{ does not appear in } \A, \mbox{ or  }  \B  % the ' does not show otherwise
             \mbox{ or } \prg{stmts}

     \\
       \Hoare   {\A} {\stmts}{ \B}{\B'}
      }
   {
    \Hoare      {\A} {\stmts}{\B}{\forall x.\B'}
      }
 \end{array}
  \end{array}
$
\caption{Hoare Logic -- we assume that the module \M\ is globally given}
\label{fig:HoareLogic}
\end{figure*}
}
 

Note that   entailment is preserved by linking:
\begin{lemma}
$ $ \\

\begin{itemize}
\item
$\P\, \rightarrow_\M\, \P'$\ implies that\  $\P  \rightarrow_{\M\!*\!\M'}\, \P'$.
\item
$\Q\, \rightarrow_\M\, \Q'$\ implies that\  $\Q\, \rightarrow_{\M\!*\!\M'}\,  \Q'$

$\P, \Q\, \rightarrow_\M\, \P'$\   implies that  $\P, \Q\, \rightarrow_{\M\!*\!\M'}\, \P'$\
\end{itemize}
\end{lemma}

Examples of some entailments

\begin{lemma}[Entailments] For all modules $\M$:
$ ~ $

\begin{enumerate}
\item
$ \MayAccess(\prg{x},\prg{y}) \wedge \MayAccess\pre({\prg{y},\prg{z}}) \ \rightarrow_\M \ \MayAccess(\prg{x},\prg{z}) $.
\item
...
\end{enumerate}

\end{lemma}

Also,  derivability and validity of Hoare tuples is preserved for larger modules
\begin{theorem}[Linking preserves derivations and validity]
\label{lemma:hl:preserve}
For all modules $\M$, $\M'$.
\begin{itemize}
\item
If $\HoareExpl {\P} {\prg{code}}  {\M} {\Q} {\Q'}$ , then\\ $\HoareExpl {\P} {\prg{code}}  {\M\!*\!\M'} {\Q} {\Q'}$.
 
\item
If $\HoareExpl {\P} {\prg{code}}  {\M} {\Q} {\Q'}$, then \\ $\M \models \HoareImpl {\P} {\prg{code}} {\Q} {\Q'}$. \footnote{\toby{This second bullet just expresses soundness, no?}}
\end{itemize}
\end{theorem}

Soundness

\begin{theorem}[Soundness of the Hoare Logic]
\label{lemma:hl:soundness}
 For all modules \M, code \prg{code} and assertions $\P$, $\Q$ and $\Q'$:\\
 
\noindent
If $\HoareExpl {\P} {\prg{code}}{\M}  {\Q} {\Q'}$ then $\M \models \HoareImpl {\P} {\prg{code}} {\Q} {\Q'}$.  
\end{theorem}

\begin{definition}[Proving code's adherence to specification]
In order to prove that a method adheres to its specification, 
  where \P\ and \Q \  and \Q'\ are the method's pre,   postcondition, and invariant, 
and where 
the method's body is
\prg{stmts}\semi \kw{return} \prg{a}
we need to derive that \\
  $ \SP\SP\SP$ $\Hoare {\P\wedge \kw{this} \obeys Spec} {\prg{stmts}}  {\Q''} {\Q'''} $\\
and that \\
  $ \SP\SP\SP$  $\Q''\equiv \Q[\prg{a}/\prg{res}] \ $\\
 and\\
  $ \SP\SP\SP$  $\Q'''\equiv \Q'[\prg{a}/\prg{res}] \ $
\end{definition}\footnote{\toby{This is phrased as a definition. But
doesn't proving a module's
adherence to a specification involve some kind of inductive argument?
It feels to me like it is some kind of inductive argument that would justify
adding the $\wedge \kw{this} \obeys Spec$ precondition as, from my
understanding, this precondition is added to capture the idea that when
proving the correctness of method \texttt{m} we implicitly assume correctness
of all of \texttt{m}'s callees, right?}}

 
\footnote{Should we distinguish in \M\ between code and specification? I think we have to!}
 
