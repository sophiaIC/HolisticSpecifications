\section{A Formal Model of Trust and Risk}
\label{section:formal}
 
In this section we provide an overview of our core programming
language, \LangOO, our specification language, \Chainmail, and our
Hoare logic. \kjx{The Hoare logic uses \textit{four-tuples} because it
  includes an invariant that must be preserved during the execution of
  a statement as well as a postcondition established afterwards.}
\kjx{We also outline a key step required to prove} that
\prg{deal\_version2} meets the \prg{ValidEscrow} specification:
\kjx{we prove that two purses can establish mutual trust, and formally delineate
the risk.}  Many
details are relegated to our technical report \cite{appendix}; here we
adopt its numbering for definitions.




\subsection{\LangOO}

We define a small object oriented language, \LangOO\ 
(Featherweight Object Capability Language, not to be confused
with FOCAL \cite{FOCAL-69}). \LangOO\ 
 supports classes, fields and methods.
(Figures~\ref{fig:DealV1} and~\ref{fig:DealV2} are effectively examples of \LangOO.)
%
\LangOO\ is  memory-safe: it does not allow
addresses to be forged, or non-existent methods or fields to be
called, read or written.   \LangOO\ is dynamically typed: it does not check
that the arguments to a method call or a field write are of the
appropriate type either statically or dynamically: similar to 
JavaScript, Grace, E, and Dart's unchecked mode.
 

% \LangOO\  supports
Modules, $M$,  are  mappings from class
identifiers to \LangOO\ class definitions, and from predicate identifiers to \Chainmail\ assertions as described in section \ref{sect:Chainmail}.
The  linking operator $*$ combines these definitions, provided that 
the modules' mappings have separate domains, and performs no other checks. 
This reflects the open world setting,
where objects of different provenance interoperate without a central
authority.
For example, taking $M_{p}$ as a module implementing purses, and $M_e$
as another module implementing the escrow,
$M_{p}*M_e$ is defined but $M_{e}*M_e$ is not.

\LangOO\  enforces a weak form of privacy for fields; % are \prg{private}; 
only the receiver may modify these fields,  and anybody may read them.
% SD what it said before is wrong, and did not come from Appendix :-(
%THIS IS STILL PROBABLY WRONG!

The operational semantics of  \LangOO\ takes a module $\M$  
and a runtime state $\sigma = \syntax{frame} \times \syntax{heap}$ 
and maps statements onto a new 
state $\sigma'$.
% This is not needed for the main paper
% and right hand sides of assignments (\ie paths, values, and method calls) to a value and new state and a value. 
%The
% operational semantics has the following shape: 

\vspaceSD

\setcounter{definition}{5}
\begin{definition}[Shape of Execution]

\begin{tabular}{lcl}
 ${\rewriteLong {}}\s$ &  :  &    \syntax{Module}  $\times$  \syntax{state}  $\times$   {\syntax{Stmts}}
  \ \  $\longrightarrow $ \ \     {\syntax{state}}
%\\
% ${\rewriteLong {}}\s$ &  :  &    \syntax{Module}  $\times$  \syntax{state}  $\times$   {{\syntax{Rhs}}}
%  \ \  $\longrightarrow $ \ \     \syntax{heap} $\times$ \syntax{val}  
\end{tabular}
\end{definition}




 
\paragraph{Arising and Reachable Configurations}  

Policies need to be satisfied in all {\em configurations} (pairs of states and statements) which may
arise during execution of the program. For example, if a program contains a class which has field which is not exported, and
where this field is initialized to $0$ by the constructor, and incremented by $3$ in all method calls, then in the arising configurations the
value of this field is guaranteed to be a multiple of $3$. Thus, through the concept of arising configurations we can 
ignore configurations which are guarantee not to arise.

To define arising configurations we need the concept of initial configuration, and reachability. 
 A configuration is
\emph{reachable} from some starting configuration if it is reached
during the evaluation of the starting configuration after any number
of steps.     We define
the function 
\mbox{$\Reach:\syntax{Module} \times \syntax{state} \times \syntax{Stmts} 
\longrightarrow \mathcal{P}(\syntax{state} \times \syntax{Stmts})$}
by cases on the structure of
the statements. Note that  $\Reach(\M, \sigma, \stmts) $ is defined, 
even when the execution should diverge. This is important, because it allows us to give meaning to capability policies without requiring termination. 


\noindent We then define $\Arising(\M)$ as the set of runtime configurations
which may be reached during execution of some initial context
($\sigma_0$,$\code_0$).

\setcounter{definition}{6}
\begin{definition}[Arising and Initial configurations] 
$ ~ $

$\begin{array}{lcl}
\SP {\mathcal{I}nit} & \ : \ & {\syntax{Module}} \   \longrightarrow\  \mathcal{P}( {\syntax{state} \times \syntax{Stmt}}  )
\\
 \Arising & : &  {\syntax{Module}}    \longrightarrow \mathcal{P}( {\syntax{state}} \times \syntax{Stmts}  )
\end{array}$

$\begin{array}{ll}
 {\mathcal{I}nit}(\M)   =  &  \{ \ (\ \sigma_0, \kw{new}\ \clss{}.\prg{m}\lp\kw{new}\  \clss{'} \rp) \ |    \ \clss{},\clss{'}\in dom( \M),\\
 & \ \ \ 
   \mbox{where}\  \sigma_0\ =\ ( (\iota_0,\kw{null}),\chi_0),    \mbox{and}\   \chi_0(\iota)=(\prg{Object}, \emptyset)\  \}
 \\
 \Arising(\M)    =   & \    \bigcup_{ (\sigma,\code{})\in  \mathcal{I}nit (\M)}  \Reach(\M, \sigma, \code{})
\end{array}$
\end{definition}




\subsection{\Chainmail}
\label{sect:Chainmail}

\Chainmail\ is a specification language where a specification is
a conjunction of a set of named policies.
(Figures~\ref{fig:ValidPurse} and~\ref{fig:ValidEscrow} are examples of
\Chainmail\ specifications.)

\Chainmail\ policies are based on one-state assertions ($\A$) and two-state assertions ($\B$).  To express the state in which an expression
is evaluated, we annotate it with a subscript. For example, $x>1$ is a one-state, and 
$ x\pre - x\post = 1$ is a two-state assertion. Validity of an assertion is defined in the usual manner, \eg 
in a state $\sigma$ with  $\sigma(x)=4$ we have   $M,\sigma \models x>1$. If we also have $\sigma'(x)=3$,
then we obtain    $\M,\sigma,\sigma' \models x\pre - x\post =
1$.  \Chainmail\ specifications may also express ghost information,
which is not stored explicitly in the state $\sigma$ but can be
deduced from it --- e.g.\ the length of a null-terminated string.



Policies can have one of the three following forms:  
1) invariants of the form $\A$, which   require that $\A$ holds at all visible states of a 
program; or  
2)\  $\A\, \lb \, \prg{code}\, \rb\, \B$, which require that execution of \prg{code} in any state satisfying $\A$ will lead to a state
satisfying $\B$ wrt the original state
 or  
 3)\ $\A\, \lb \,  {\prg{any\_code}}\, \rb \, \B$ which requires that execution of {\em any} 
 code in a state satisfying $\A$ will lead to a state satisfying $\B$ wrt the original state.

\vspaceSD

\setcounter{definition}{11}
\begin{definition}[Policies]
 $ ~ $ \\
 $
\begin{array}{lclcl}
Policy & \BBC & \ \A \ | \  \A \ \{ \prg{code} \}\  \B \ | \ \  \A \ \{ \prg{any\_code} \}\ \B
\\
PolSpec & \  \BBC  & \  \prg{specification}\ S \, \lb\, Policy^*\, \rb
\end{array}
$
\end{definition}


One-state assertions include assertions about  expressions (such as $\leq$, $>$ \etc) and four additional
assertions: $\syntax{\sExpr}\obeys\syntax{SpecId}$ to model trust,
i.e.\ that an object confirms to a specification; and $\MayAccess$ and
$\MayAffect$ to model risk, i.e.\ whether one object may access
another, or alter a property.  These are {\em hypothetical}, in that
they talk about the potential effects or behaviour of code: we cannot
somehow evaluate their truth-value when executing the program.  The
fourth assertion $\syntax{\sExpr}\kw{:}\syntax{ClassId}$ 
simply tests class membership. 
 
 
Validity of one-state assertions is expressed through the judgment
$\M,\sigma \models \A$.  The key case is that some expression obeys a
specification if it satisfies that specification's policies in all
reachable configurations arising from the module.

 

\vspace*{1mm}\noindent\textbf{(from Definition 13)}:

\begin{itemize}
  \item
 $\M,\sigma  \models  \syntax{\sE}\kw{:}\prg{C}$ iff  $\sigma(\interp{ \syntax{\sE}}{\M,\sigma})\downarrow_1 = \prg{C}$.
 
 \item 
$\M,\sigma  \models \MayAffect \lp\syntax{\sE},\syntax{\sE'}\rp$ iff  
% there exist fields  $\prg{f}_1$,... $\prg{f}_n$,  
there exist  method \prg{m}, arguments $\bar{\prg{a}}$, state $\sigma'$, identifier \prg{z}, such that
    $ \M, \sigma[\prg{z}\mapsto \interp {\syntax{\sE}} {\M,\sigma}], \syntax{z}\prg{.m}(\bar{\prg{a}}) \leadsto   \chi'$, and   $\interp {\syntax{\sE'}} {\M,\sigma} \neq  \interp {\syntax{\toby{\sE'}}} {\M,\sigma\downarrow_1,\chi'}    $.
\item
$\Prog{},\sigma \models { \MayAccess}(\prg{\sE},\prg{\sE'})$   \ iff \  there exist  fields $\bar{\prg{f}}$, % $\prg{f}_1$,... $\prg{f}_n$, 
  such that
     %  $\interp{\prg{z}.\prg{f}_1...\prg{f}_n}{\M,\sigma[\prg{z}\mapsto \interp {\syntax{\sE}} {\M,\sigma}]}= \interp {\prg{\sE'} }{\M,\sigma}$.
     $\interp{\prg{z}.\ensuremath{\bar{\prg{f}}}}{\M,\sigma[\prg{z}\mapsto \interp {\syntax{\sE}} {\M,\sigma}]}= \interp {\prg{\sE'} }{\M,\sigma}$
  \item
$\M, \sigma  \models \sE \obeys \PolSpecId   $ \  iff  \\
\SP  $\ \ \   \forall\, (\sigma,\code)\!\in\!\Arising(M).\   \forall  i\!\!\in\! \!\{1..n\}. \ \forall\,\sigma',\code'.$  \\
\SP  $\ \ \    (\sigma',\code')\! \in \Reach(M,\sigma,\code)  \longrightarrow    M, \sigma'[z\mapsto\interp{e}{\sigma}]  \models \Policy_i[ z / {\kw{this}}]$ \\
where $z$ is a fresh variable in $\sigma'$, and where
 we assume that $\PolSpecId$ was defined as \ \  %  that   \PolSpecId was introduced by\\
$ \ \prg{specification}\ \PolSpecId\  \{\ \Policy_1, ... \Policy_n\ \}$. \\
\end{itemize}
% \end{definition}

Two-state assertions allow us to compare properties of two different
states. Validity of two-state assertions $\M,\sigma,\sigma' \models \B$
is defined similarly to one-state assertions, using cases.
%
We can now define adherence to policy, $\M,\sigma \models_{pol}  \Policy$:
%    which
% ensures that the requirements of $\Policy$ are satisfied in any context arising from $\M$.

\vspaceSD
\setcounter{definition}{14}
\begin{definition}[Adherence to Policies] ~ ~ % \\
% $ $ \\

\begin{itemize}
\item
$\M,\sigma  \models_{pol} \A$   \  iff \ 
$\M, \sigma \models \A$
\item
$\M,\sigma    \models_{pol} \A\, \{ \prg{code} \}\, \B$ \  iff \\ 
 $ ~ $ \hspace{.2in} 
 $(\ \M, \sigma \models \A\ \ \wedge\  \M, \sigma, \prg{code}  \leadsto \sigma' $ %  \\
 %  $ ~ $   \hspace{.6in}  
 $ \ \ \  \longrightarrow $ 
%  \\  $ ~ $   \hspace{.2in}  
$  \ \ \ \M, \sigma, \sigma' \models \B\ \ )$
\item
{$\M, \sigma \models_{pol} \A \ \{ \prg{any\_code} \}\ B$\ iff} \\
 $ ~ $ \hspace{.2in}  $ \forall \prg{code}.\ (\  (\sigma,\prg{code} )\in \Arising(\M)\ \wedge\  \M, \sigma \models \A\  \wedge\  \M, \sigma, \prg{code}  \leadsto  \sigma' $ \\
  $ ~ $   \hspace{.6in}  $ \ \ \  \longrightarrow $
  % \\   $ ~ $   \hspace{.2in}  $ 
   $\ \ \ \M, \sigma, \sigma' \models \B\ \ )$
\end{itemize}
\end{definition}

% SD removed it, ad not central and moved the for all modules to later on
%\setcounter{definition}{14}
%\begin{definition}[Strong Adherence to Policies] ~ ~ % \\
%% $ $ \\
%
%\noindent $\M   \models_{pol} Policy$   \  iff \ \\
%\SP $\forall \M', \sigma, \code. \ 
%(\ (\sigma,\code)\in \Arising(\M'*\M)\ \longrightarrow \M'*\M, \sigma \, \models Policy\ )$
%\end{definition}


\subsection{Hoare Logic}

The Hoare logic allows us to prove adherence to policies. In order to
reflect that the code to be verified is executed in an open system,
and that it calls code whose specification and trustworthiness is
unknown to the code being verified, we use Hoare four-tuples rather
than Hoare triples, so that not only do they guarantee a postcondition
holds {\em after} execution of the code, but also guarantee that
\kjx{an invariant} is preserved {\em during} execution of the code.
\kjx{These invariants are critical to modelling risk, as they let us 
talk about the absence of temporary but unwanted effects caused on
objects during execution.}% e.g.\ reasoning  that valid purses
%always correctly protect their balances.}


\kjx{A Hoare four-tuple  is either \mbox{\HoareExpl{\A}
  {\prg{stms}} {\M} {\A'} {\B}} (executing 
\prg{stms} in any state satisfying $\A$~will lead to a state which
satisfies $\A'$) or \mbox{\HoareExpl{\A} {\prg{stms}} {\M} {\B'} {\B}}
(executing \prg{stms} in any state satisfying
$\A$~will lead to a state where the relation of the old and
new state is described by $\B'$).}  \kjx{Critically,} both promise that the relation between
the initial state, and {\em any} of the   intermediate states reached by
execution of \prg{stms}, will \kjx{maintain the invariant} $\B$.
The execution of \prg{stmts}  may call methods defined in \M, and the
predicates appearing in $\A$, $\A'$, $\B'$, and $\B$, may use  predicates
  defined in \M. 
When \M\ is implicit from the context,
we use the shorthand
\Hoare{\A} {\prg{stms}}  {\A'} {\B}. 


In order to model open systems, we require that after linking {\em any}
module with the module at hand, the policy will be satisfied. As
stated in \cite{JamesMorris}, \textit{``A programmer should be able to prove
that his programs have various properties and do not malfunction,
solely on the basis of what he can see from his private bailiwick.''}

% SD removed as not cnetral to our argument
% We introduce {\em logical variables} into our assertions. We assume that
%these have the form \lvar, \lvar', and that they come from a separate domain. We also
% assume that there exists a function \Lvars, which returns all the
% logical variables within an assertion. For example
% $\Lvars(\prg{p1.balance}=\lvar ) =\{\,\lvar\,\}$. Taking account of
% logical variables we can define the validity of tuples formally:
 
\vspace{-.05in}
\begin{definition}[Validity of Hoare Four-Tuples]\label{defn:validity}

\noindent
%  \begin{itemize}
%  \item
%$M  \models \HoareImpl{\A} {\prg{stms}0.5} {\A'} {\B}  $ \ \ \  iff\\
%\SP $ \Lvars(\A)=\Lvars(\A')=\{\, \overline{\lvar}\, \}\ \wedge \ \forall \M',\sigma, \overline{\val}.$\\
%\SP\SP $ (\sigma,\_)\in \Arising(\M*\M')$\\
%\SP\SP $ \wedge\ \M*\M',  \sigma{[\overline{\lvar}\mapsto\overline{\val}]} \models \A$\\
%\SP\SP $ \wedge\  \M\!*\!\M', \sigma, \prg{stms}  \leadsto \prg{res}, \sigma'$\\
%\SP\SP\SP \ \ \ $\longrightarrow$ \ \ \ \\
%\SP\SP   $ \M\!*\!\M'\!, \sigma'{[\overline{\lvar}\mapsto\overline{\val}]} \models \A' $\\
%\SP\SP    $\wedge$ \\
%\SP\SP  $\forall {\sigma''}\!\in\!\Reach(M,\sigma,\stmts).\ \M\!*\!\M', \sigma, {\sigma''} \models \B   $
%  \item
$M  \models  \HoareImpl{\A} {\prg{stms}} {\B'} {\B}  $ \ \ \  iff \ \ 
  $\forall \M',\sigma$.\\
\SP $ (\sigma,\_)\in \Arising(\M*\M')\  \ \wedge  \ \ \M*\M',  \sigma \models \A \  \ \wedge \ \   \  \M\!*\!\M', \sigma, \prg{stms}  \leadsto \prg{res}, \sigma'$\\
\SP\SP\SP\SP\SP\SP\SP\SP\SP\SP\SP \ \ \ $\longrightarrow$ \ \ \ \\
\SP $M\!*\!\M'\!, \sigma, \sigma' \models \B' \ \ \ \wedge \ \ \forall {\sigma''}\!\in\!\Reach(M,\sigma,\stmts).\ \M\!*\!\M', \sigma, {\sigma''} \models \B $
% \end{itemize}
\end{definition}


\renewcommand{\true}{true}

\begin{figure*} 
$\begin{array}{l}
 \begin{array}{lcl}
      \inferenceruleNP {varAsg} {
} {
         \Hoare {\true}
      {  \prg{v} \kw{:=} \prg{a} } { \prg{v}=\prg{a}\pre }  {\true}
}
& \SP\SP  &
\inferenceruleN{fieldAsg} {
 
}{
     \Hoare
     {\true}
      { \kw{this.}\prg{f}  \kw{:=} \prg{a} }
      { \kw{this.}\prg{f}   = \prg{a}\pre}
       {\true}
}
\end{array}
\\ \\
  \inferenceruleNNP {meth-call-1} {
    { \M(S) \,= \,  \kw{spec}\ S\  \lb\  \overline{Policy}, \A \,\lb \, \prg{this.m(par)}\, \rb\, \B, \overline{Policy'}\  \rb}
% \SP\SP \SP\SP\SP
  %  \B''[\prg{x}/\prg{this},\prg{y}/\prg{par}] = \B'''
    } {
    \Hoare{\prg{x}\obeys S\wedge  \A[\prg{x}/\prg{this},\prg{y}/\prg{par}]} {\prg{v} \, \kw{:=}\, \prg{x.m(y)}} {\B[\prg{x}/\prg{this},\prg{y}/\prg{par},\prg{v}/\prg{res}]}   {{\true}} % {\B'''}
  }
  \\
  \\
   \inferenceruleNNP {meth-call-2} {
         \B \equiv \forall \prg{z}:\pre \Obj.\ \MayAccess({\prg{v},\prg{z}}) \rightarrow  
    (\, \MayAccess\pre(\prg{x},\prg{z})\, \vee \, \MayAccess\pre(\prg{y},\prg{z})\, )\\
   \B' \equiv   \forall \prg{z},\prg{u}:\pre \Obj. \ ( \ \MayAccess({\prg{u},\prg{z}}) \rightarrow \\
       $ ~ $ \hspace{1.3in}  ( \MayAccess\pre(\prg{u},\prg{z}) \ \ \ \vee  \\
                 $ ~ $ \hspace{1.7in}(\  (\MayAccess\pre({\prg{x},\prg{z}})\vee \MayAccess\pre({\prg{y},\prg{z}}) )\ \wedge \\
                 $ ~ $ \hspace{1.8in}  (\MayAccess\pre({\prg{x},\prg{u}})\vee \MayAccess\pre({\prg{y},\prg{u}}) ) \ ) \ ) \ \  )  } {
    \Hoare{\true} {\prg{v} \, \kw{:=}\, \prg{x.m(y)}} {\B}
           { \B'  }
}
\\ \\
   \inferenceruleNNP {frame-methCall} {
          \Hoare{\A}   {v:=\prg{x.m(y)}} {{\true}}  {\B} \\
B \equiv \forall \prg{z}. (\ \MayAffect(\prg{z},\A')  \rightarrow \B'({\prg{z}})\ )\ \ \wedge 
          \\
% trick for layout
		\SP\SP  \ 
		{\forall \prg{z}.(\ (\MayAccess(\prg{x},\prg{z}) \vee \MayAccess(\prg{y},\prg{z})
           \vee \New(\prg{z}) \ )%(\prg{z}:\prg{Object}\wedge \neg ( \prg{z}:\pre\prg{Object}) ) \ )
           \ \rightarrow\  \neg \B'(\prg{z})\ )}
}
{
          \Hoare{{\A\wedge\A'}} {\prg{v:=x.m(y)}} {\A'} {{\true}}
}
\\ \\
\begin{array}{lcl}
 \inferenceruleB {\SP\SP\SP\SP\SP\SP\SP\SP\SP\SP}   {code-invar-1} {
            { \M( S) \equiv \  \kw{spec}\ S\  \lb\  \overline{Policy}, P, \overline{Policy'}\  \rb}
            \\
            \B \equiv {\forall \prg{x}.(\, \prg{x} \obeys S \rightarrow P[/\prg{x}/\prg{this}]\, )}
  }
     {
       \Hoare {{\true}} {\stmts} {{\true}} {\B}
      }
      & &
\inferenceruleB {\SP\SP\SP\SP\SP\SP\SP\SP\SP\SP}   {code-invar-2} {           ~ \\
     }{
    \Hoare  {e \obeys S}  {\stmts}    {\true}  {e\pre \obeys S}
      }

\end{array}
\\ \\
\begin{array}{lcccr}
  \inferenceruleNP{Cons-2} {  % NEW
          \Hoare{\A}{\stmts}{ \B}{\B''}   \\
           \sd{\A',\B' \rightarrow_\M \A,\_}   
     }
     {
       \Hoare{\A'}{\stmts}{\sd{\B'\!\rightarrow\!\B}}{\B''}
        }
        &\ \ &
\inferenceruleNP {Cons-3} 
{ 
  \Hoare{\A}{\stmts}{ \B}{\B'} %
  \\
 {{\A}, \B \rightarrow_\M \_,\A'}  
} {
   \Hoare{{\A}}{\stmts}{ \A'}{\B'}
   }
&    &
  \inferenceruleNP {Cons-4} { % was Cons-2 
          \Hoare{\A}{\stmts}{ \A'}{\B'}  
        \\
     {{\A},\A' \rightarrow_\M \B}  
     }
     {
       \Hoare{\A}{\stmts}{ \B}{{\B'}}
        }
 \end{array}
\\ \\
\inferenceruleNNP{sequence} {
        \Hoare{\A}{\stmtss_1}{\B_1}{\B'}  \ \SP        \Hoare{\A_2}{\stmtss_2}{ \B_2}{\B'}
        % \\
       \ \SP
         {\A,\B_1\rightarrow_\M \_,\A_2}    \ \SP  {\B_1,\B_2\rightarrow_\M \B}
        }
     {
       \Hoare{\A}{\stmtss_1; \, \stmtss_2}{ \B}{\B'}
        } 
\end{array} 
$
\caption{A selection of Hoare Logic rules; we assume that the module \M\ is globally given}
\label{fig:HoareLogic}
\end{figure*}

% \noindent
Figure \ref{fig:HoareLogic} shows a selection of our Hoare rules. It starts with two familiar Hoare Logic rules: In \ruleN{varAsg}  and \ruleN{fieldAsg}  the postconditions
use the previous value of the
right-hand-side, and thus  allow us to deduce, \eg: \\ % that\\
\SP \SP \Hoare {\kw{this}.f=21} {\kw{this}.f=2*\kw{this}.f} {\kw{this}.f=42}  {\true}.
\\
%\vspace{.1in}
%
\scd{\ruleN{meth-call-1} is also familiar, as it reasons over method calls under the assumption that the 
receiver   \obeys a specification $S$, and that the current state satisfies the precondition of 
\prg{m} as defined in $S$.}
% by making use of the \obeys assertion.

The remaining rules are more salient.

\ruleN{meth-call-2} expresses the basic axiom of object-capability
systems that ``only connectivity begets connectivity''~\cite{MillerPhD}.
\tobym{It promises in the postcondition that the result of the method call~\prg{v} 
cannot expose access to any object \prg{z} that wasn't
accessible initially to the method call's receiver~\prg{x} or argument~\prg{y}.
Additionally, it also promises that,
{\em during} execution of the method, accessibility does
not change, unless the participants (here \prg{z}  and \prg{u})
were accessible to the receiver and/or the argument {\em before}
the call. Note that this latter promise is made via the invariant (fourth)  rather than  the postcondition (third) part
of the Hoare-tuple. Note also that this rule is applicable {\em even if we know nothing} about the 
receiver of the call:}
\kjx{this rule and the invariants are critical to reasoning about risk.}

\ruleN{code-invar-1} allows reasoning under the hypothesis that
an object~$o$ \obeys its specification~$S$: in this case~$o$ can be trusted
to act in accordance with~$S$ always.


\ruleN{frame-methCall} also
expresses an axiom of object-capability languages, namely that
in order to cause some visible effect, one must have access to an object able
to perform the effect. Coupled with ``only connectivity begets connectivity'',
this implies that a method can cause some effect only if the caller has
(transitive) access to some object able to cause the effect (including
perhaps the callee).


The remaining rules each make use of the entailment judgement~$\rightarrow_M$,
which allows converting back and forth
between one-state and two-state
assertions and comes in number of flavours; the relevant ones are
defined as follows.

\setcounter{definition}{18}
\begin{definition}[Entailment]
\label{def:entail}
$ $ \\

\begin{enumerate} 
\item
\label{entailPQPP}
{
$\A, \B \rightarrow_\M \A',\A''$\ \  iff\ \\
$\forall \sigma, \sigma'. \ \sigma\models \A\ \wedge\  \sigma,\sigma' \models \B \ \ \longrightarrow\ \ \sigma \models \A' \ \wedge\  \sigma' \models \A''$}
\item
\label{entailPPQ}
{$\A, \A' \rightarrow_\M \B$}\ \  iff\ \\
{$\forall \sigma, \sigma'. \ \sigma\models \A\ \wedge\  \sigma' \models \A'\  \ \longrightarrow\ \  \sigma,\sigma' \models \B$}
\item
\label{entailQQQ}
{$\B, \B' \rightarrow_\M \B''$}\ \  iff\ \\
{$\forall \sigma, \sigma', \sigma''. \  \sigma,\sigma' \models \B\ \wedge\   \sigma',\sigma'' \models \B'\  \ \longrightarrow\ \  \sigma,\sigma'' \models \B''$}
\end{enumerate}
\end{definition}

The rules \ruleN{cons-3} and \ruleN{cons-4} make use of the entailment
judgement to allow converting between one- and two-state postconditions
during Hoare logic reasoning.
To reason across sequenced computations $\stmtss_1; \stmtss_2$,
the \ruleN{sequence} rule requires finding a one-state assertion~$\A_2$ 
that holds after $\stmtss_1$ and is the precondition of $\stmtss_2$. It uses
the entailment $\A, \B_1 \rightarrow_M \_,\A_2$ to require that
$\stmtss_1$'s execution guarantees $\A_2$, and the entailment
$B_1, B_2 \rightarrow_M B$ to require that the combined execution
of $\stmtss_1$ and $\stmtss_2$ guarantees the top-level postcondition~$B$.

 % \scd{Soundnss of the Hoare rules is stated below, and proven in \cite{appendix}}

\noindent

\setcounter{theorem}{2}
\begin{theorem}[Soundness of the Hoare Logic]

 For all modules \M, statements \prg{stms} and assertions $\A$, $\B$ and $\B'$ ,  
\\
$\SP\SP$ If \ \ $\HoareExpl {\A} {\prg{stms}}{\M}  {\B'} {\B}$, \   then\  $\M \models \HoareImpl {\A} {\prg{stms}} {\B'} {\B}$.  

\end{theorem}

\noindent
\scd{The theorem is proven in \cite{appendix}.}


\vspace{.05in}
\noindent 
\scd{In summary, we have four ''code agnostic'' rules ---
% SD I am happy if you find a better word than code-agnistiv, 
rules which 
are applicable regardless of the underlying code. Rules \tobym{\ruleN{frame-methCall} 
and \ruleN{meth-call-2}}  express  restrictions on the effects of a
method call.
\kjx{Normally such restrictions stem from the specification of the
  method being called, but here we can argue in the absence of any
  such specifications, allowing us to reason about risk even in open systems.}
Rules \ruleN{Code-Invar-1}  and \ruleN{Code-Invar-2} are applicable on {\em any} code, and allow us to
assume that an object which \obeys a specification $S$, satisfies all policies from $S$, and that the
object, once trusted, will always be {\kw {obeying}} $S$.}


\subsection{\kjx{Proving Mutual Trust}}
\label{section:formalPROOF}

\begin{figure*} 
\begin{lstlisting}
   true
        {  var escrowMoney := sellerMoney.sprout  }
   $\sM\pre \obeys \PS \longrightarrow$ 
           $(\ \eM \obeys\, \PS\  \wedge$  
             $CanTrade(\eM,\sM)\  \wedge$
             $\eM.\bal = 0\  \wedge\   $
             $\forall p\in\pre \GP.\  p.\bal\pre=p.\bal\ \wedge $
             $ \sM  \obeys \PS\ ) \ \ \ \ \wedge $
    $ \forall p:\pre \GP.$
       $(\, p.\bal\pre=p.\bal \vee  \MayAccess\pre(\sM,p) \, )\ \ \wedge$
    $ \forall z:\pre \Obj.$
       $(\, \MayAccess(\eM,z) \longrightarrow \MayAccess\pre(\sM,z)\, ) \  \wedge$
    $ \forall z,y:\pre \Obj.\ $
       $ (\,  \MayAccess(\,z,y\,)\  \longrightarrow\   $
            $(\, \MayAccess\pre(\, z, y\,)\  \vee$
                    $\MayAccess\pre(\, \sM, y\,) \wedge$
                    $\MayAccess\pre(\, \sM, z\,) \, ) \ \ )$
    $\HoareCSep$
    true
   \end{lstlisting}
%\vspace*{-7mm}
\caption{Hoare tuple for first step in \prg{deal}}
\label{fig:DealV3:S1}
\end{figure*}


% \paragraph{First Step} 
\kjx{We now use our Hoare Logic to prove the key steps of the escrow
  protocol, establishing mutual trust and delineating the risk. Here we have space to show just
  one-way trust between the escrow and seller in full: the remaining reasoning
  to establish mutual trust is outlined in the technical report \cite{appendix}.}
%
\kjx{Figure \ref{fig:DealV3:S1} shows the  Hoare tuple
for the first statement in method \prg{deal} (line 4 from Figure \ref{fig:DealV2})}. 
Lines 3-8 of Figure \ref{fig:DealV3:S1} describe the postcondition in case  \eM\ indeed \obeys \PS, while lines 9-17 make absolutely no assumption
about the trustworthiness, or provenance, of \eM.

\noindent
% We obtain lines 3-7 from  policy \prg{Pol\_sprout}  and application of  \ruleN{Meth-Call-1} and \ruleN{Cons-2}, while line 8 requires a framing step.

\noindent
By \prg{Pol\_sprout}  and \ruleN{Meth-Call-1} we obtain that \\
$  % (1) \\
  \HoareNLSP {A}
      {\sM \obeys  \PS}
       {\eM:=\sM.\prg{sprout}}
      { \eM  \obeys  \PS\  \wedge  ... rest ...} 
         {\true} %  {\true}
$

\noindent
By application   \ruleN{Cons-2} on the above we obtain\\
$  
 \HoareNLSP {B}
      {\true}
       {\eM:=\sM.\sprout}
      { \sM\pre \obeys\, \PS \rightarrow \ \\
      \HoareSP\SP\SP (\ \eM  \obeys \PS\  \wedge \ ... rest ...\, )}
       { \true }  $

\noindent
To obtain line 8, we apply a basic framing rule %a la Reynolds 
(\ruleN{frame-general} in \cite{appendix}) and get\\ % that \\
\SP $\Hoare {...}  {\eM:=\sM.\sprout} {\eM\pre=\eM} {...} $, \\ and
then, in conjunction with \ruleN{code-invar-2}, \ruleN{Cons-2}   we also obtain that\\ 
$\HoareNLSP {C} 
     {\true}  
     {\eM:=\sM.\sprout} 
     {\eM\pre\obeys\PS \rightarrow \eM\obeys\PS} 
     {...}
$ \\
We can then apply a conjunction rule (\ruleN{Conj} in \cite{appendix}) on \ruleN{B} and  \ruleN{C}, and obtain the postcondition as in 4-8.\\
To obtain 9-11, we will apply several of the code-agnostic rules. After all, here we cannot appeal to the specification of \sprout, as we do not
know whether \sM\ adheres to \PS. We start by application of \ruleN{meth-Call-2}, and a consequence rule (\ruleN{Cons-1} in \cite{appendix}):\\
$\HoareNLSP {D} 
     {\true}  
     {\eM:=\sM.\sprout} 
     {\true} 
     {\forall z. \ \MayAccess(\sM,z) \rightarrow \MayAccess\pre(\sM,z) }
$ \\
By applying  the fact that $\forall u,v,w,\ \MayAccess(u,v)\ \wedge\ \MayAccess(v,w) \rightarrow \MayAccess(u,w)$, and conjunction  and inference rules on \ruleN{D}, we get:\\
$\HoareNLSP {E} 
     {\neg \MayAccess(\sM,p)}  
     {\eM:=\sM.\sprout} 
     {\true} 
     {\forall z. \ \MayAccess(\sM,z) \rightarrow \neg \MayAccess(z,p) }
$ \\
By application of rule  \ruleN{code-invar-1}, we obtain:\\
$\HoareNLSP {F} 
     {\true}  
     {\eM:=\sM.\sprout} 
     {\true} 
     {\forall p. (\, p\obeys\PS \rightarrow (\forall z. \, \MayAffect(z,p.\bal) \rightarrow \MayAccess(z,p)) \, )}
$ \\
Through a combination of \ruleN{E} and \ruleN{F} and application of conjunction, and application of \ruleN{frame-meth-call}, we obtain that\\
$\HoareNLSP {G} 
      {\neg \MayAccess(\sM,p)}  
     {\eM:=\sM.\sprout} 
     {\true} 
     {p\obeys\pre \PS \rightarrow (p.\bal=p.\bal\pre) }
$ \\
Now by applying \ruleN{Cons-2} on \ruleN{F}, we obtain\\
$\HoareNLSP {H} 
      { \true}  
     {\eM:=\sM.\sprout} 
     {\true} 
     {\forall p. \, p\obeys\pre \PS \rightarrow \\
      \SP\SP\SP\SP (\ p.\bal=p.\bal\pre  \ \vee\  \MayAccess(\sM,p) \ ) }
$
\\
We now apply \ruleN{Cons-1} from \cite{appendix} to conjoin the invariant and postcondition, obtaining\\
$\HoareNLSP {I} 
      { \true}  
     {\eM:=\sM.\sprout} 
          {\forall p. \, p\obeys\pre \PS \rightarrow \\
      \SP\SP\SP\SP (\ p.\bal=p.\bal\pre  \ \vee\  \MayAccess(\sM,p) \ ) }
      {\true} $
Last, we obtain lines 11-12 from \ruleN{Meth-Call-2}. We also obtain lines 13-17 from \ruleN{Meth-Call-2}, and   \ruleN{Cons-1} from \cite{appendix}.


