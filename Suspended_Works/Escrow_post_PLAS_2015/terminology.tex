\section{Trust, Risk, and the Open World}

\kjx{NO LONGER INCLUDED: IDEAS EDITED INTO INTRO}


The key feature of an open system is that it can interact with unknown
potentially malicious code.  We consider our system to be comprised of
a module \M, and model an open environment by as an unknown,
potentially malicious module \M'.  
The system as a whole is M dynamically linked with \M', $\M*\M'$.
THen, given some Policy (an invariant on the behaviour of the
system) we have: 

\begin{definition}
M $\models$ Policy    iff
$\forall M', \kappa, code.$
       $(\kappa,code) \in Arising(M'*M)$
                  $\Rightarrow   M'*M, \kappa \models$  Policy
\end{definition}

\kjx{What's the right LaTeX formatting here? DO we need to say what
  $\kappa$ and ``code'' are? Do we want this here?}
 
$\Arising(M'*M)$ restricts configurations to be considered to only those that are reachable though execution of M'*M.

\paraB{Trust.}   Our notion of trust is that if an object $o$ is
trusted, we can take it as obeying a specification Policy:

\begin{definition}
Definition 
$M, \kappa \models o \obeys\ Policy$    iff
                                        $\forall M'.  M'*M, \kappa \models Policy[o/this]$
\end{definition}

The ``$\obeys$'' predicate is hypothetical: there no central authority
that can assign trustworthiness (or not) to objects; there is no
``trust'' bit that we can test. Rather, ``$o \obeys\ Policy$'' is an
assumption that may or may not be true, and we use that assumption to
reason by cases. If we trust an object, we can use the object's
specification $Policy$ to determine the results of a method call on
that object. If we don't trust the object, we reason about the
confirmation of the system to determine the maximum amount of damage
the call could do: the risk.

\paraB{Risk.}  Risks are effects against which we want to guard our
objects: bounds on the potential damage caused by calls to untrusted
objects.  The key to delinating risks are two further hypothetical
predicates $\MayAccess$ and $\MayAffect$, which describe whether an
object may access or may modify a certain property respectively.
For example, we can state that a \lstinline+deposit+ request can on a
trusted purse will not increase the accessibility of any trusted purse
in the system:

\begin{lstlisting}[escapechar=@]
  policy Pol_deposit_1     //   1$^{st}$ case:
      amt$\in \mathbb{N}$
        @\textbf{ \{ res = dest.deposit(amt, src) \} }@
      res $\rightarrow$ (
              @\ldots@
              $\forall$p.(p$\obeys$$\pre$ValidPurse @\ldots@
              $\forall$o:$\pre$Object. $\forall$ p$\obeys$$\pre$ValidPurse.  $\MayAccess$(o,p) $\rightarrow$ $\MayAccess\pre$(o,p)  )
\end{lstlisting}

\noindent or that a valid purse's balance can only be changed by an
object that can somehow access that purse object:a

\begin{lstlisting}
  policy Pol_protect_balance
      $\forall$ o,p: p$\obeys$ValidPurse $\wedge$ o:Object
      $\rightarrow$ [$\MayAffect$(o,p.balance) $\rightarrow$ $\MayAccess$(o,p)]
\end{lstlisting}
