\section{Proof of \prg{Escrow:deal}}


We now outline the most salient steps from the proof of the \prg{Escrow}. Note that out formally defined language does not support returning from the inside of a method - we did this to simplify the Hoare rules. Therefore, in \autoref{fig:DealV3} we re-write the mothod \prg{deal} so that it obeys this  syntactic restriction.

\begin{figure*}[htb]
\begin{lstlisting}
method deal(  )  
{
  //setup and validate Money purses
  escrowMoney := sellerMoney.sprout
  res := escrowMoney.deposit(0, sellerMoney)
  if res then {
    res :=  buyerMoney.deposit(0, escrowMoney)
    if res then {
      res := escrowMoney.deposit(0, buyerMoney)     
      if res then {
        // set up and validate Goods purses
        escrowGoods := buyerGoods.sprout
        res := escrowGoods.deposit(0, buyerGoods) 
        if res then {
          res := sellerGoods.deposit(0, escrowGoods) 
          if res then {
            res :=  escrowGoods.deposit(0, sellerGoods) 
            if res then {
              // start the actual exchange
              res := escrowMoney.deposit(price, buyerMoney)
              if res then {
                res := escrowGoods.deposit(amt, sellerGoods)
                if res then{
                  // transfer from the two escrows to two accounts
                  sellerMoney.deposit(price, escrowMoney)
                  buyerGoods.deposit(amt, escrowGoods)
                } else {
                  // undo the transaction
                  buyerMoney.deposit(price, escrowMoney)
                }
              } else skip                        
            } else skip
          } else skip 
        } else skip 
      } else skip
    } else skip 
  }
  return res  
}  
\end{lstlisting}
%\vspace*{-7mm}
\caption{Revised \prg{deal} method expressed without {\prg{return}} statements}
\label{fig:DealV3}
\end{figure*}
\subsection{Preliminaries}

We first create some admissible rules, useful for our reasoning.

Firstly, because logical variables cannot be assigned to, we have that   $\Hoare  {{\lvar} }  {\stmts}   {\kw{true}}  {{\lvar}=\lvar\pre}$ for any \stmts; therefore,  the following rules are admissible for any logical variable \lvar, and specification $S$:\\
$  \inferenceruleN {code-inavr-3} {
           
     }{
    \Hoare  {{\lvar} \obeys S}  {\prg{code}}   {\kw{true}}  {{\lvar} \obeys S}
     }
$

\vspace{.1in}
Similarly, through application of \ruleN{Frame-General}, if $\prg{z}\neq\prg{x}$, we get 
$ \Hoare  {\prg{z}=\lvar}  {\prg{x:=rhs}}   {\prg{z} =\lvar}  {\kw{true}}  $, which also gives that
$ \Hoare  {\kw{true}}  {\prg{x:=rhs}}   {\prg{z} =\prg{z}\pre}  {\kw{true}}  $. Then, 
by \ruleN{Code-Invar-2}  and \ruleN{Cons-1} we obtain that \\
$  \inferenceruleN {obeys-invar} {
           \prg{z}\neq \prg{x}
     }{
    \Hoare  {{\prg{z}} \obeys S}  {\prg{x:=rhs}}   {\kw{true}}  {{\prg{z}} \obeys S}
     }
$


\vspace{.1in}
\noindent
%This rule follows from\textsc{code-invar-2}, and ...


\subsection{First Step}

The pre and postconditions for the first line from the code, ie for line 4 from Figure \ref{fig:DealV3} are described in figure \ref{fig:DealV3:S1}.
Drawing on the \prg{Pol\_sprout} policy of the 
\prg{ValidPurse} specification, this step is obtained as follows:

Firstly, by application of \ruleN{obeys-ivar} and \ruleN{Cons-4} we obtain
\\
$  (0) 
\\ \HoareNL
      {\prg{true}}
       {\eM:=\sM.\prg{sprout}}
      { \sM\pre  \obeys \PS\ \rightarrow \\
      \SP\SP\SP \sM\ \obeys \PS}
         {\kw{true}} %  {\kw{true}}
$
 \vspace{.1in}

\begin{figure*}[htb]
\begin{lstlisting}
   true
        {  var escrowMoney := sellerMoney.sprout  }
   $\sM\pre \obeys \PS \longrightarrow (\ \eM \obeys \PS\  \wedge$  
                                      $CanTrade(\eM,\sM)\  \wedge$
                                      $\eM.\bal = 0\  \wedge\   $
                                      $\forall p\in\pre \GP. p.\bal\pre=p.\bal\ \wedge $
                                      $ \sM  \obeys \PS\ ) \ \ \ \ \wedge $
    $ \forall p:\pre \GP.  (\, p.\bal\pre=p.\bal\ \vee\ \MayAccess\pre(\sM,p) \, )\ \ \ \  \ \ \ \    \wedge$
    $ \forall z:\pre \Obj.\ (\,  \MayAccess(\eM,z)\  \longrightarrow\   \MayAccess\pre(\sM,z)\, ) \ \ \ \ \wedge$
    $ \forall z,y:\pre \Obj.\ (\,  \MayAccess(\,z,y\,)\  \longrightarrow\   $
              $  (\, \MayAccess\pre(\, z, y\,)\  \vee\  \MayAccess\pre(\, \sM, y\,) \wedge  \MayAccess\pre(\, \sM, z\,) \, ) $
    $\HoareCSep$
    true
   \end{lstlisting}
%\vspace*{-7mm}
\caption{Hoare tuple for first step in \prg{deal}}
\label{fig:DealV3:S1}
\end{figure*}

. 

%\textsc{code-invar-1}, 
\noindent
Then, from the specification of \sprout in $\PS$, and the rule \ruleN{Meth-Call-1} we obtain that \\
$  (1) 
\\ \HoareNL
      {\sM \obeys \PS}
       {\eM:=\sM.\prg{sprout}}
      { \eM  \obeys \PS\  \wedge \\
        CanTrade(\eM,\sM)\  \wedge \\ 
        \forall\,p:\pre\GP. p.\bal=p.\bal\pre} 
         {\kw{true}} %  {\kw{true}}
$
 \vspace{.1in}
\noindent
Then, from (1), and application of \ruleN{Cons-2}, we obtain\\
$ (2) \\
 \HoareNL
      {\kw{true}}
       {\eM:=\sM.\sprout}
      { \sM\pre \obeys \PS \rightarrow\\
       \SP\SP \ \ (\ \eM  \obeys \PS\  \wedge \\
        \SP\SP\SP CanTrade(\eM,\sM)\  \wedge \\ 
        \SP\SP\SP \forall \,p\in\pre\GP.\ p.\bal=p.\bal\pre\ )}
       { \kw{true} }  $
 \vspace{.1in}
 
\noindent
Also, by application of \ruleN{Code-Invar-1}, and the specification of \PS, we  have that\\
$ (3) \\
 \HoareNL
      {\kw{true}}
       {\eM:=\sM.\sprout}
       { \kw{true} }
       {  \forall\,p\in\pre\GP,o:\Obj.\ \\
        \ \ \  (\, \MayAffect(o,p.\bal) \rightarrow \MayAccess(o,p) \ )}
        $
 \vspace{.1in}
 
\noindent
 By application of \ruleN{Meth-Call-2} and \ruleN{Frame-Meth-Call} and (3) we obtain\footnoteC{here I am not sure about!!!}\\
$ (4) \\
 \HoareNL
      {\kw{true}}
       {\eM:=\sM.\sprout}
       { \kw{true} }
       {  \forall\,p\in\pre\GP\ \\
        \ \ \  (\, p.\bal=p.\bal\pre \vee \\
        \ \ \ \ \ \MayAccess\pre(\sM,o) \ )}
        $ 
 \vspace{.1in}
  
\noindent
 Finally, by application of \ruleN{Meth-Call-2}   we obtain\\
$ (5) \\
 \HoareNL
      {\kw{true}}
       {\eM:=\sM.\sprout}
       { \kw{true} }
      { \forall z,y:\pre \Obj.\ (\,  \MayAccess(\,z,y\,)\  \longrightarrow\   \\
              \SP (\, \MayAccess\pre(\, z, y\,)\  \vee\  \\
              \SP \ \ \MayAccess\pre(\, \sM, y\,) \wedge\\
              \SP \ \   \MayAccess\pre(\, \sM, z\,) \, ) }
$
\vspace{.1in}

By application of \ruleN{Cons-1}, and \ruleN{Conj} on (0), (2), (4), and (5), we obtain the pre-postconditions from Figure \ref{fig:DealV3}.
 
 
 
 % \vspace{.1in}



\subsection{Second Step}
The pre and postconditions for the second step are described in figure
\ref{fig:DealV3:S2}. The main differences between
figures \ref{fig:DealV3:S1} and \ref{fig:DealV3:S2} are a reflection of the
differences between the policies \prg{Pol\_sprout} and
\prg{Pol\_deposit\_1} and \prg{Pol\_deposit\_2} in the
\prg{ValidPurse} specification.  Functionally, \prg{deposit} may
succeed or fail, indictated by its return value \prg{res}, while
\prg{sprout} always succeeds; \prg{deposit} may change the balances of
participant purses, while \prg{sprout} may not.  

Crucially for us, the trust essentially the same in both cases:\\
\mbox{\lstinline+src$\obeys\PRE$ValidPurse $\wedge$ CanTrade(this,src)$\PRE{}$+}\\
\noindent and the risk is very similar --- slightly more complex for
\prg{deposit} which may modify the two purses:\\
\noindent \mbox{\lstinline+$\forall$p.(p$\obeys$$\pre$ValidPurse
  $\wedge$p$\notin\{$this,src$\}\,\rightarrow$+}\\
\mbox{~~~~\lstinline+p.balance=p.balance$\pre$)  $\wedge$+}\\
\noindent  but otherwise may not increase risk:\\
\noindent\mbox{\lstinline+$\forall$o:$\pre$Object. $\forall$p$\obeys$$\pre$ValidPurse.$\MayAccess$(o,p)
  $\rightarrow$+}\\
\noindent\mbox{~~~~\lstinline+$\MayAccess\pre$(o,p)  )+}\\
\noindent Thus, the reasoning for this step can be justified in
similar ways to those that from figure \ref{fig:DealV3:S1}. 



\begin{figure*}[htb]
\begin{lstlisting}
    true
          {  res=escrowMoney.deposit(0, sellerMoney)  }
   $\eM\pre \obeys \PS  \longrightarrow (  \forall p\in\pre \GP. p.\bal\pre=p.\bal ) $
   $\eM\pre \obeys \PS \wedge \prg{res}\  \rightarrow\  (\, \sM \obeys \PS\,  \wedge$
   $ \forall p:\pre \GP.  (\, p.\bal\pre=p.\bal\ \vee\ \MayAccess\pre(\sM,p) \, )\ \ \ \  \ \ \ \    \wedge$
   $ \forall z:\pre \Obj.\ (\,  \MayAccess(\eM,z)\  \longrightarrow\   \MayAccess\pre(\eM,z)\, ) \ \ \ \ \wedge$
   $ \forall z,y:\pre \Obj.\ (\,  \MayAccess(\,z,y\,)\  \longrightarrow\   $
              $  (\, \MayAccess\pre(\, z, y\,)\  \vee\  \MayAccess\pre(\, \sM, y\,) \wedge  \MayAccess\pre(\, \sM, z\,) \, ) $
    $\HoareCSep$
    true
 \end{lstlisting}
%\vspace*{-7mm}
\caption{Hoare tuple for second step in \prg{deal}}
\label{fig:DealV3:S2}
\end{figure*}

 


\newpage
\subsection{Step 1 and Step 2 Establish Mutual Trust}

When we combine step 1 and step 2 we obtain the Hoare tuple from figure \ref{fig:DealV3:S3}. Here we make use of the results from figure \ref{fig:DealV3:S1} and figure \ref{fig:DealV3:S2}, and combine them through the \ruleN{Sequence} rule. For example, we use our invariants entailment  $\longrightarrow_\M$, whereby for any module $\M$:\\
 $\forall z:\pre \Obj.\ (\,  \MayAccess(\eM,z)  \rightarrow$\\
 $\SP\SP\SP  \MayAccess\pre(\sM,z)\, ),\ $\\
$\forall z:\pre \Obj.\ (\,  \MayAccess(\sM,z)\  \rightarrow$ \\
$\SP\SP\SP   \MayAccess\pre(\eM,z)\, ), $\  \\
$\longrightarrow_\M$ \\
 {\kw {true}}, \\
 $\forall z:\pre \Obj.\ (\,  \MayAccess(\eM,z)\  \rightarrow$\\
 $\SP\SP\SP    \MayAccess\pre(\sM,z)\,).$

\begin{figure*}[htb]
\begin{lstlisting}
    true
        {  var escrowMoney := sellerMoney.sprout 
           res :=  escrowMoney.deposit(0, sellerMoney)  }
    $\res\ \longrightarrow \ \sM\pre \obeys \PS \longleftrightarrow  \eM \obeys \PS\  \ \  \wedge$  
    $\sM\pre \obeys \PS \ \longrightarrow (\  CanTrade(\eM,\sM)\  \wedge$
                                         $\eM.\bal = 0\  \wedge\   $
                                         $\forall p\in\pre \GP. p.\bal\pre=p.\bal\ \wedge $
                                         $ \sM  \obeys \PS\ ) \ \ \ \ \wedge $
    $ \neg \res\ \longrightarrow\ \neg(\sM\pre \obeys \PS) \ \ \ \  \ \ \ \    \wedge$                                  
    $ \forall p:\pre \GP.  (\, p.\bal\pre=p.\bal\ \vee\ \MayAccess\pre(\sM,p) \, )\ \ \ \  \ \ \ \    \wedge$
    $ \forall z:\pre \Obj.\ (\,  \MayAccess(\eM,z)\  \longrightarrow\   \MayAccess\pre(\sM,z)\, ) \ \ \ \ \wedge$
    $ \forall z:\pre \Obj.\ (\,  \MayAccess(\sM,z)\  \longrightarrow\   \MayAccess\pre(\sM,z)\, ) \ \ \ \ \wedge$
    $ \forall z,y:\pre \Obj.\ (\,  \MayAccess(\,z,y\,)\  \longrightarrow\   $
              $  (\, \MayAccess\pre(\, z, y\,)\  \vee\  \MayAccess\pre(\, \sM, y\,) \wedge  \MayAccess\pre(\, \sM, z\,) \, ) $
    $\HoareCSep$
    true
   \end{lstlisting}
%\vspace*{-7mm}
\caption{Hoare tuple for first and second step in \prg{deal}}
\label{fig:DealV3:S3}
\end{figure*}

\vspace*{1cm}  
 
These two steps combined prove that we have now established 
mutual trust between these two purses.  This is expressed in
line 4 of
figure~\ref{fig:DealV3:S3}:\\
\mbox{\lstinline+$\res\ \longrightarrow$+}\\
\mbox{~~~~\lstinline+$\sM\pre \obeys \PS$+}\\
\mbox{~~~~~~~~~~~~~~~~\lstinline+$\longleftrightarrow \eM \obeys \PS$+}\\
\noindent The bulk of the proof proceeds similarly, with lines 6-18 of
figure~\ref{fig:DealV3}
requiring the same reasoning to establish the remaining mutual trust
relationships, first by including the remaining money purse, and then
between all the goods purses. 

Finally lines 20-30 complete the escrow exchange by exchanging money
and goods.  The core reasoning here is completely straightforward, as
trust is already established between all concerned purses --- although
of course we also have to handle the cases where trust is not
established, on paths where a \prg{deposit} call fails. We have to
continue to reason about the risk, but since only \prg{deposit} and
\prg{sprout} calls are involved, this reasoning is no different to
that of the first and second step.

