%\appendix

 
%already defined in LLNCS
 \newtheorem{definition}{Definition}
 \newtheorem{example}{Example}
 \newtheorem{lemma}{Lemma}
 \newtheorem{theorem}{Theorem}

\newcommand{\HoareFigOne}
{
\begin{figure*}
$
\begin{array}{l}
\begin{array}{lcl}
\inferenceruleN {varAsg} {
} {
     \Hoare   {\kw{true}}
     {\kw{var}\,  \prg{v} \kw{:=} \prg{a} } { \prg{v}=\prg{a}\pre } {\kw{true}}
\\
     \Hoare   {\kw{true}}
      {  \prg{v} \kw{:=} \prg{a} } { \prg{v}=\prg{a}\pre }  {\kw{true}}
}
& &
\inferenceruleN {fieldAsg} {

}{
     \Hoare
     {\kw{true}}
      { \kw{this.}\prg{f}  \kw{:=} \prg{a} }
      { \kw{this.}\prg{f}   = \prg{a}\pre}
       {\kw{true}}
}
\\
\\
\inferenceruleN {cond-1} {
 \A \rightarrow_{\M} \prg{cond} \\
  \Hoare  {\A} {  \prg{stmts}_1\  } {\B} {\B' }
} {
   \Hoare  {\A} { {\kw{if}}\, \prg{cond}\,  {\kw{then}}\, \prg{stmts}_1\, {\kw{else}}\, \prg{stmts}_2  } {\B} {\B'}
   }
& &
\inferenceruleN {cond-2} {
 \A \rightarrow_{\M} \neg\prg{cond} \\
  \Hoare  {\A} {  \prg{stmts}_2\  } {\B} {\B'}
} {
   \Hoare  {\A} { {\kw{if}}\, \prg{cond}\,  {\kw{then}}\, \prg{stmts}_1\, {\kw{else}}\, \prg{stmts}_2  } {\B} {\B'}
   }
   \\
   \\
   \inferenceruleN {skip} {
} {
   \Hoare  {\A} { {\kw{skip}} } {\A} {\kw{true}}
   }
\end{array}
\\
%\\
 {
  \inferenceruleNN {newObj} {
  \M( {\cal S}(\M,\prg{C})) \ = \   \kw{spec}\ S\  \lb\    ....,\  \A \,\lb \,  {\kw{new}(\prg{x}_1,... \prg{x}_n)}\, \rb\, \B,\  ....\  \rb
 \hspace{2in}

}{
       \Hoare
        {     \A[\prg{a}_1/\prg{x}_1,... \prg{a}_n/\prg{x}_n]  }
         { \prg{x}\,  \kw{:=} \, \kw{new}\lp \prg{a}_1... \prg{a}_n \rp }
         %{      \B[(\prg{a}_1)\pre/\prg{x}_1,... (\prg{a}_n)\pre/\prg{x}_n]  }
         {      \B[\prg{a}_1/\prg{x}_1,... \prg{a}_n/\prg{x}_n,\prg{x}/\prg{res}] 
                \ \wedge\ \prg{x}\obeys 
                 {\cal S}(\M,\prg{C}) }
          { \kw{true} }
}
}
 % OLD VERSION
%{
% \inferenceruleNN {newObj} {
%    \prg{f}_1,... \prg{f}_n \mbox{ are the fields of } \M(\prg{C})
%\hspace{2in}
%  \prg{S}\in  {\cal S}(\M,\prg{C})
%}{
%      \Hoare
%       { \kw{true} }
%    { \prg{x}\,  \kw{:=} \, \kw{new}\, \prg{C}\lp \prg{a}_1... \prg{a}_n \rp }
%   { \prg{x.f}_{1}=(\prg{a}_1)\pre \, \wedge ...  \wedge\, \prg{x.f}_{n}=(\prg{a}_n)\pre
%    \  \wedge\ \prg{x}: \prg{C}\ \ \wedge\ \   \prg{x} \obeys \prg{S}  }
%       { \kw{true} }
%}
%}
\\ \\
  \inferenceruleNN {meth-call-1} {
     \M(S) \,= \,  \kw{spec}\ S\  \lb\  \overline{Policy}, \A \,\lb \, \prg{this.m(par)}\, \rb\, \B, \overline{Policy'}\  \rb
% \SP\SP \SP\SP\SP
  %  \B''[\prg{x}/\prg{this},\prg{y}/\prg{par}] = \B'''
    } {
    \Hoare  {\prg{x}\obeys S\wedge  \A[\prg{x}/\prg{this},\prg{y}/\prg{par}]} {\prg{v} \, \kw{:=}\, \prg{x.m(y)}} {\B[\prg{x}/\prg{this},\prg{y}/\prg{par},\prg{v}/\prg{res}]}   {{\kw{true}}} % {\B'''}
}
  \\
  \\
  \\
   \inferenceruleNN {meth-call-2} {
         \B \equiv \forall \prg{z}:\pre \Obj.\ [ \ \MayAccess({\prg{v},\prg{z}}) \rightarrow
    (\, \MayAccess\pre(\prg{x},\prg{z})\, \vee \, \MayAccess\pre(\prg{y},\prg{z})\, )\ \ ] \\
   \B' \equiv   \forall \prg{z},\prg{u}:\pre \Obj. \ [ \ \ \MayAccess({\prg{u},\prg{z}}) \rightarrow \\
         ~   \hspace{2.3in}  ( \MayAccess\pre(\prg{u},\prg{z}) \ \ \ \vee  \\
                   ~   \hspace{2.7in}[\  (\MayAccess\pre({\prg{x},\prg{z}})\vee \MayAccess\pre({\prg{y},\prg{z}}) )\ \wedge \\
                   ~   \hspace{2.8in}  (\MayAccess\pre({\prg{x},\prg{u}})\vee \MayAccess\pre({\prg{y},\prg{u}}) ) \ ] \ ) \ \  ]\\
   ~   \hspace{0.25in} \wedge \\
   ~   { \hspace{0.25in} \forall \prg{z}:\pre \Obj.\forall \prg{u}: \Obj.}\\
    ~   \hspace{1.7in}[\ \ { \New(\prg{u}) \wedge (\ \MayAccess(\prg{z},\prg{u})\vee\MayAccess(\prg{u},\prg{z})\ )}\ \  \ \rightarrow  \\
     ~   \hspace{2.6in} {(\ \MayAccess\pre(\prg{x},\prg{z})\vee\MayAccess\pre(\prg{y},\prg{z})\ ) }  \ \ \  ]
                 } {
    \Hoare  {\true} {\prg{v} \, \kw{:=}\, \prg{x.m(y)}} {\B}
           { \B'  }
}
\\ \\
   \inferenceruleNN {frame-methCall} {
           \Hoare  {\A \ {\wedge\ \prg{e}=\prg{e}} }   {\prg{x.m(y)}}  {\kw{true}}  {\forall \prg{z}. (\ \MayAffect(\prg{z},\prg{e})  \rightarrow \B'({\prg{z}})\ )\ \ \wedge }
          \\
% trick for layout
		\SP\SP\SP\SP\SP \SP\SP\SP\SP\SP \SP\SP\SP\SP\ \ \   \
		{\forall \prg{z}.(\ (\MayAccess\pre(\prg{x},\prg{z}) \vee \MayAccess\pre(\prg{y},\prg{z})
           \vee \New(\prg{z}) \ )%(\prg{z}:\prg{Object}\wedge \neg ( \prg{z}:\pre\prg{Object}) ) \ )
           \ \rightarrow\  \neg \B'(\prg{z})\ }
}
{
          \Hoare  {{\A}\ {\wedge\ \prg{e}=\prg{e}} } {\prg{x.m(y)}}   {\prg{e}=\prg{e}\pre}  {\kw{true}}
}
\\
\\
\inferenceruleNN{sequence} {
        \Hoare  {\A}{\stmts_1}{\B_1}{\B'}  \ \SP\SP\SP        \Hoare  {\A_2}{\stmts_2}{ \B_2}{\B'}
        % \\
       \ \SP\SP\SP
         {\A,\B_1\rightarrow_{\M,\O} \kw{true},\A_2}    \ \SP\SP\SP  {\B_1,\B_2\rightarrow_{\M,\O} \B}
        }
     {
       \Hoare  {\A}{\stmts_1; \, \stmts_2}{ \B}{\B'}
        }
\end{array}
$
\caption{Hoare Logic -- Basic rules of the language -- we assume that the module \M\ is globally given}
\label{fig:HoareLogicBasic}
\end{figure*}
}

\newcommand{\HoareFigTwo}
{
\begin{figure*}
$\begin{array}{l}
\begin{array}{lcl}
   \\
\\
   \inferenceruleN {frame-general} {
          \Hoare  {\A}{\stmts}{\B} {\B'}
           \\
        {\A  \rightarrow_{\M}   \stmts\disj  \A'     \SP\SP\SP\    \A  \rightarrow_{\M}   \stmts\ddisj  \A''}
} {
          \Hoare  {\A\wedge \A'} {\stmts} {\B\wedge{\A'}}  {\B' \wedge{\A''}}
}
& &
\inferenceruleN {Conj} {
 { \Hoare  {\A_1}{\stmts}{ \B_1}{\B_3}}\\
{  \Hoare  {\A_2}{\stmts}{ \B_2}{\B_4}}
} {
   { \Hoare  {\A_1\wedge \A2 }{\stmts}{ \B_1 \wedge \B_2 }{\B_3 \wedge \B_4}   }
 }
\\
\\
    \inferenceruleN {Cons-1} { % REFACTORED
          \Hoare  {\A}{\stmts}{ \B}{\B'} \SP\SP\SP {\A' \rightarrow_{\M} \A}
        \\
     {\B \rightarrow_{\M} \B''} \ \  \SP\SP\SP\SP\SP\SP\SP\SP {\B' \rightarrow_{\M} \B'''}
     }
     {
       \Hoare  {\A'}{\stmts}{ {\B''\wedge\B'}}{{\B'''}}
        }
&
&
    \inferenceruleN {Cons-2} {  % NEW
          \Hoare  {\A}{\stmts}{ \B}{\B''}   \\
           {\A',\B' \rightarrow_{\M} \A,\kw{true}}
     }
     {
       \Hoare  {\A'}{\stmts}{{\B' \rightarrow \B}}{\B''}
        }
%\\ \\
%\inferenceruleN {Cons-3}
%{ % was Cons-1
%  \Hoare  {\A}{\stmts}{ \B}{\B'} %
%  % SD put in Cons-4 \SP\SP\SP\SP\SP \toby{\A' \rightarrow_{\M} \A}
%  \\
% \A, \B \rightarrow_{\M} \kw{true},\A'
% % I droppped the \wedge \B', because Cons-1 does it
%} {
%   \Hoare  {{\A}}{\stmts}{ \A'}{\B'}
%   }
%& &
%  \inferenceruleN {Cons-4} { % was Cons-2
%          \Hoare  {\A}{\stmts}{ \A'}{\B'} % \SP\SP\SP \toby{\A'' \rightarrow_{\M} \A}
%        \\
%     \toby{{\A},\A' \rightarrow_{\M} \B} %\SP\SP\SP\SP\SP\SP\SP \toby{\B' \rightarrow_{\M} \B''}
%     }
%     {
%       \Hoare  {\A}{\stmts}{ \B}{\toby{\B'}}
%        }
   \\
   \\
 \inferenceruleN {code-invar-1} {
            { \M( S) \equiv \  \kw{spec}\ S\  \lb\  \overline{Policy}, P, \overline{Policy'}\  \rb}
  }
     {
       \Hoare   {{\kw{true}}} {\stmts} {{\kw{true}}} { {\prg{x} \obeys S} \rightarrow P[\prg{this}/\prg{x}]}
      }
    & &
  \inferenceruleN {code-invar-2} {

     }{
    \Hoare    {{e} \obeys S}  {\stmts}    {\kw{true}}  {{e}\pre \obeys S}
      }
  \\
   \\
  \inferenceruleN {quantifiers-1} {
             x \mbox{ does not appear in } \A, \mbox{ or } \B, \mbox{ or } \prg{stmts}
     \\
       \Hoare   {\A} {\stmts}{ \B'}{\B}
      }
   {
    \Hoare      {\A} {\stmts}{\forall x.\B'}{\B}
      }
           & &
  \inferenceruleN {quantifiers-2} {
             x \mbox{ does not appear in } \A, \mbox{ or  }  \B  % the ' does not show otherwise
             \mbox{ or } \prg{stmts}

     \\
       \Hoare   {\A} {\stmts}{ \B}{\B'}
      }
   {
    \Hoare      {\A} {\stmts}{\B}{\forall x.\B'}
      }
 \end{array}
  \end{array}
$
\caption{Hoare Logic -- we assume that the module \M\ is globally given}
\label{fig:HoareLogic}
\end{figure*}
}


\section{Formal Definition of the language \LangOO}

 \subsection{Modules and Linking}
 \label{formal:modules}

  \LangOO\ modules map class identifiers to class descriptions, function  identifiers to  function descriptions,  and predicate identifiers to predicate descriptions. We also require implicitly for any module  $\M$, class identifier \c, function identifier \f, and predicate identifier \P, that 
that $\M(\c)\in \syntax{ClassDescr}$ or undefined, that $\M(\f)\in \syntax{FunDescr}$ or undefined,  and $\M(\P)\in \syntax{PredDescr}$ or undefined. 



\begin{definition}[Modules]
 $ $ \\  % to force line break
 
\begin{tabular}  {@{}l@{\,}c@{\,}ll}
\\
\syntax{Module} \ \  &    =   &    \syntax{ClassId}  $\cup$ \syntax{FunId} $\cup$ \syntax{PredId}   $\cup$ \syntax{SpecId}\   \\
& &  $\longrightarrow$ \\
& & ( \syntax{ClassDescr}  $\cup$ \syntax{FuncDescr} $\cup$ \syntax{PredDescr}\\
& & \ \   $\cup$ \syntax{Specification}  )
 \end{tabular}
\end{definition}

\noindent 
We  define linking of modules, $M*M'$, to be the  union of their respective mappings, provided that the  domains of the two modules are disjoint:

\begin{definition}[Linking and Lookup]
\label{formal:Linking}
Linking  of  modules $\M$ and $\M{'}$ is 

\noindent $\ \ \ * \ : \ Module \times Module \ \longrightarrow \ Module  $
\noindent

$
\M * \M{'}  =\ \left\{
\begin{array}{l}
                        \M *_{aux} \M{'},\ \ \   \hbox{if}\  \ dom(\M)\!\cap\!dom(\M{'})\!=\!\emptyset\\
\bot  \ \ \ \mbox{otherwise.}
\end{array}
                    \right.\\
(\M *_{aux} \M{'})(\clss{}) =\ \left\{
\begin{array}{l}
 \M(id),\ \ \ \mbox{if }
\M(id) \mbox{ is defined}\\
\M{'}(id) \ \ \  \mbox{otherwise.}
\end{array}
                    \right.
                    $
\end{definition}


 

%%this was originally redefining macro M for some reason  
%%the original macro had no arguments
%%this one has got one. 
%%actually I see no reason whatsoever for this defintion
%%and removing it makes everything work
%%
%%\newcommand{\M}[1]{{\textit{M}{#1}}} %%KJX: WHY Sopha, WHY?  
%%

\paragraph{Classes}
 
 We define the syntax ....

\begin{definition}[Classes, Methods, Args]
\label{def:syntax:classes}
We define the syntax of modules below.

\begin{tabular}{lcll}
 \syntax{ClassDescr}   &   \BBC  &  \kw{class}  \syntax{ClassId} \\
 & & \hspace{0.1in} \lb\,  (\kw{fld} \syntax{FieldId})$^*$   (\ \syntax{methBody}\ )$^*$ \ \rb
\\
\syntax{methBody} &\BBC&
     \kw{method}    \syntaxMeth\lp \syntax{ParId}$^*$ \rp\\
 & & \hspace{0.1in}  \lb\, \syntax{Stmts} \semi   \kw{return}  \syntax{Arg}  \, 
    \rb
 \\
 \syntax{Stmts}  &\BBC&  \syntax{Stmt}     ~\SOR~  \syntax{Stmt} \semi \syntax{Stmts} \\
\syntax{Stmt}    &\BBC&     \kw{var} \syntax{VarId}  {\kw{:=}} \syntax{Rhs}\\
&  ~\SOR~ &     \syntax{VarId} {\kw{:=}} \syntax{Rhs} \\
&  ~\SOR~ &    \kw{this}.\syntax{FieldId} {\kw{:=}} \syntax{Rhs} \\
&  ~\SOR~ &   \kw{if}  \syntax{Arg}  \kw{then} \syntax{Stmts} \kw{else} \syntax{Stmts}\\
&  ~\SOR~ &   \kw{skip}\\
\syntax{Rhs} & \BBC&    \sd{\syntax{Arg}}{\kw{.}}\syntax{MethId}\lp  \syntax{Arg}$^*$ \rp    ~\SOR~   \syntax{Arg}  \\
&  ~\SOR~ &   \sd{\kw{new} \syntax{ClassId}\lp \, \syntax{Arg}$^*$\, \rp} \\
 \syntax{Arg} &\BBC&  \syntax{Path} ~\SOR~ \kw{true} ~\SOR~ \kw{false} ~\SOR~ \kw{null} \\
\syntax{Path}    &\BBC& \syntax{ParId} ~\SOR~ \syntax{VarId} ~\SOR~ {\kw{this}} \\
&   \SOR &  \syntax{Path}{\kw{.}} \syntax{FieldId} 
 \end{tabular}
\end{definition} 

Note that \LangOO\, supports a limited form of protection: the syntax
  supports  reading of fields of any object, but restricts each object
  to being able to modify only its \emph{own} fields.
 
\footnoteC{** TODO explain the syntax, and say why this language is so minimal. In particular why we do model object creation *** Also explain that we permit \kw{return} only at the end of a method body and how it looks in the Escrow. Say what we do not object creation, but these will come.
}
  

\paragraph{Method Lookup} We define the method lookup function, $\cal M$ which returns the corresponding method definition given a class and a method identifier. 
\begin{definition}[Lookup] The lookup function
% $\cal M$ returns the corresponding method definition, if it is defined.

%  \begin{center}
$
% \begin{array}{lcl}
   \Meths {} {\clss{}} {m}      =       \kw{method}\, m\, \lp p_1, ... p_n \rp
\lb stms \semi\, \kw{return}\ \sd{\syntax{a}} \rb  
%  \end{array}
 $
% \end{center}

\noindent
 iff \ $ \M({\clss{}}) =  \kw{class}\  \syntax{c}\lb ...  $\\
$ ~ $  \hspace{.7in} $ \kw{method}\, m\, \lp p_1, ... p_n \rp
\lb stms \semi\, \kw{return}\ \sd{\syntax{a}} \rb$\\
$ ~ $ \hspace{.6in} $ ... \rb.$
\\
\noindent
undefined, otherwise.
\end{definition}

 
  \subsection{Execution of  \LangOO}

\begin{figure*}
$\begin{array}{l}
\begin{array}{lcl}
 \inferenceruleN {methCall\_OS} { 
 \interp{\syntax{a}}{\phi\cdot\chi} = \iota \\
%   \forall i\in\{1..n\}.\ \ \ \   \interp{{\syntax{a}}_i}{\phi\cdot\chi} = \syntax{val}_i 
% \\
\Meths {} {$\chi(\iota)\downarrow_1$} {m} \  =  \\
  ~ \hspace{.1in} \kw{method}\ m\lp par_1, \ldots par_n \rp
 \lb stms \semi\, \kw{return}\ \syntax{a}' \rb
 \\
 \phi''=\this \mapsto \iota, par_1 \mapsto  \interp{{\syntax{a}}_1}{\phi\cdot\chi}, \ldots par_n \mapsto  \interp{{\syntax{a}}_n}{\phi\cdot\chi}
 \\
 \M,\, \phi''\cdot\chi, stmts \ \leadsto  \phi'\cdot\chi' 
}
{
  \M,\, \phi\cdot\chi,\, \syntax{a}{\kw{.}}\syntax{m}\lp {\syntax{a}_1,\ldots\syntax{a}_n}\rp  \ \leadsto\ 
    \chi', \interp{\syntax{a'}} {\phi' \cdot\chi'}
}
& &
\begin{array}{l}
\inferenceruleN {Arg\_OS} {
       } {
     \M,\, \phi\cdot\chi,\, {  \syntax{a}} 
        \ \leadsto\    \chi,\interp{\syntax{a}}{\phi\cdot\chi}  
}
\\
\inferenceruleN {\sd{New\_OS}} {
      \addr {} \mbox{ is new in}\ \chi \\
       \prg{f}_1, ... \prg{f}_n \mbox{ are the fields defined in } \syntax{C}
} {
  \M{},\,
  \phi\cdot\chi,\, {\kw{new}\, \syntax{C} \lp \syntax{a}_1, ... \syntax{a}_n} \rp \\
        \ \leadsto\  
        \chi[ \addr \mbox{\mapsto} (\syntax{C}, \prg{f}_1 
    \!\!\mapsto\!\!        \interp{\syntax{a}_1}{\phi,\sigma}
        ..\prg{f}_n\!\!\mapsto\!\! \interp{\syntax{a}_n}{\phi,\sigma})]\, 
        % \addr
}
\end{array}
\\ \\
 \inferenceruleN {varAsg-1\_OS} {
       \M,\, \phi\cdot\chi,\, {\syntax{e}}
        \ \leadsto\  \sd{\chi'},\,\syntax{val}
} {
    \M,\, \phi\cdot\chi,\, {\kw{var}}\,\syntax{v} {\kw{:=}} \syntax{e}
        \ \leadsto\  \phi[v\mapsto val]\sd{\cdot\chi'}        \\
}
& &
\inferenceruleN {varAsg-2\_OS} {
       \M,\, \phi\cdot\chi,\, {\syntax{e}}
        \ \leadsto\  \sd{\chi'} ,\, \syntax{val}
} {
    \M,\, \phi\cdot\chi,\, {\syntax{v} {\kw{:=}} \syntax{e}}
        \ \leadsto\  \phi[v\mapsto val]\sd{\cdot\chi'}
        \\
}
\\ \\
\inferenceruleN {fieldAsg\_OS} {
       \M,\, \phi\cdot\chi,\, {\syntax{e}}
        \ \leadsto\  \phi\cdot\chi',\, \syntax{val}
} {
    \M,\, \phi\cdot\chi,\, {  {\kw{this.}}\syntax{f} {\kw{:=}}\ \syntax{e}}
        \ \leadsto\  \phi \cdot\chi'[\phi({\kw{this}}),\syntax{f}\mapsto \syntax{val}]
        \\
}
& &

\inferenceruleN {sequence\_OS} {

       \M,\, \sd{\sigma}, {\syntax{stmt}} 
        \ \leadsto\  \sd{\sigma''} 
        \\
       \M,\, \sd{\sigma''} ,\, {\syntax{stmts}} 
        \ \leadsto\  \sd{\sigma'} 
} {
    \M,\, \sd{\sigma} ,\, {   \syntax{stmt} \,\semi\, \syntax{stmts}   }
        \ \leadsto\ \sd{\sigma'} 
        }
\\
\\
\inferenceruleN {cond-True\_OS} {
\interp{\syntax{a}}{\sd{\sigma} } = \kw{true}
\\
       \M,\, \sd{\sigma},\, {\syntax{stmts}}_1
        \ \leadsto\  \sd{\sigma'} 
} {
    \M,\, \sd{\sigma},\, { {\kw{if}}\, \syntax{a}\,  {\kw{then}}\, \syntax{stmts}_1\, {\kw{else}}\, \syntax{stmts}_2  }
        \ \leadsto\  \sd{\sigma'}
}
& &
\inferenceruleN {cond-False\_OS} {
\interp{\syntax{a}}{\sd{\sigma} } = \kw{false}
\\
       \M,\, \sd{\sigma} ,\, {\syntax{stmts}}_2
        \ \leadsto\  \sd{\sigma'} 
} {
    \M,\, \sd{\sigma},\, { {\kw{if}}\, \syntax{a}\,  {\kw{then}}\, \syntax{stmts}_1\, {\kw{else}}\, \syntax{stmts}_2  }
        \ \leadsto\  \sd{\sigma'} 
}
\\
\\
\inferenceruleN {skip\_OS} {
 } {
    \M,\, \sd{\sigma},\, { {\kw{skip} }}
        \ \leadsto\  \sd{\sigma} 
}

 \end{array}
\end{array}
$
\label{fig:Execution}
\caption{Operational Semantics}
\end{figure*}
 

  \paragraph{Runtime state}
\label{formal:state} The runtime state~$\sigma$ consists of a stack frame~$\phi$, and a heap $\chi$. A stack frame  is  a mapping from 
  receiver (\this) to its address, and from {the local variables ($\syntax{VarId}$) and parameters ($\syntax{ParId}$) to their values.}  Values are integers, 
 the booleans \kw{true} or \kw{false}, addresses, or \kw{null}. Addresses are ranged over by $\iota$. The heap maps addresses to objects. Objects are tuples consisting of the class of the object, and a  mapping from field identifiers onto values.


\begin{tabular}{lll}
$\sigma \in $ \syntax{state}  & =  &  \syntax{frame}  $\times$ \syntax{heap} \\
 $\phi \in$   \syntax{frame}  & =  &   \syntax{{StackId}} $\longrightarrow$ \syntax{val} \\
 $\chi \in$   \syntax{heap}  & =  \ \ \ &  \syntax{addr} $\longrightarrow$ \syntax{object} \\
 $\prg{v}\in$ \syntax{val} & = &  $\{$   \nullK, \kw{true}, \kw{false}  $\}$    $\cup\  \syntax{addr}\ \cup \ \mathbb{N}$ \\
 \syntax{object}\ \ & = &  \syntax{ClassId} $\times$ ( \ \syntax{FieldId} $\longrightarrow$ \syntax{val} \ ) \ \SP \SP \\
$\iota, \iota',..$  & $\in$ & $\syntax{addr}$  \\
% \syntax{res}  & = & $\{\ \nullPEC, \  \stuck\ \}$\ $\cup$\ \\
{\syntax{StackId}}  & = & {$\{\ \kw{this}\  \}$\ $\cup$\ \syntax{VarId}  $\cup$\ \syntax{ParId}}\\
\end{tabular}
\footnoteC{{we used to support null pointer exceptions and stuck, but see comment below}}

 \paragraph{The Operational Semantics of \LangOO}
\label{formal:semantics}

\newcommand{\cons}{\mathit{::}}
\newcommand{\st}{\ensuremath{\mathit{st}}}
\newcommand{\fs}{\ensuremath{\mathit{fs}}}
\newcommand{\traverseFields}[3]{\mathit{traverseFields}_{#1}(#2,#3)}
{We define $\interp{a}{\sigma}$, the {\em interpretation} of \jn{an argument~$a \in \syntax{Arg}$} in a state $\sigma$ as follows. }

\begin{definition}[Interpretation] For a state {$\sigma = ({\phi}%\cons\st
,\chi)$} we define

$\begin{array}{lcl}
\interp {\kw{null}} {\sigma} & = & {\kw{null}}
\\
\interp {\kw{true}} {\sigma} & = & {\kw{true}}
\\
{\interp {\kw{false}} {\sigma}} & = & {\kw{false}}
\\
\interp {{x}} {\sigma} & = & {\phi(x) \quad \mathrm{(for\   x \in\ } {\syntax{StackId}})}
%\ \cup\ \syntax{ParId}\mathrm{)}}
\\
{\interp {p\kw{.}f} {\sigma}} & = & {\chi(\interp {p} {\sigma})(f)}   \quad \quad \mbox{is} \interp {p} {\sigma}
\mbox{ is defined,} 
\\
& & \mbox{ undefined,} \quad \quad \mbox{otherwise} 
%\\
%{\interp {x\kw{.}\fs\kw{.}f} {\sigma}} & = &  
 %{\chi(\interp {x\kw{.}\fs} {\sigma})(f)}
  %{\interp {x\kw{.}\fs} {\sigma}} & = & 
 % {\traverseFields{\chi}{\interp{x}{\sigma}}{\fs}} 
\end{array}$ 

\newcommand{\emptylist}{\epsilon}
\noindent
% SD the bolow is no longer needed {Here $\fs$ is a non-empty \kw{.}-separated list of \syntax{FieldId}s.}
\end{definition}

Execution uses module \M, and  maps a runtime state $\sigma$ and statements 
\syntax{stmts} (respectively a right hand side $rhs$) % (code in the general case)
onto a new state $\sigma'$ (respectively a new heap $\chi'$ and
a value). {We therefore do not give execution rules for things like null-pointer-exception, or stuck execution. This allows us to keep the system simple; it will be easy to extend the semantics to a fully-fledged language.} 
\begin{definition}

%result.
Execution  of \LangOO\ statements and expressions is defined in figure \ref{fig:Execution}, and has the following shape: 
\begin{tabular}{lcl}
 ${\rewriteLong {}}\s$ &  :  &    \syntax{Module}  $\times$  \syntax{state}  $\times$   \toby{\syntax{Stmts}}
  \ \  $\longrightarrow $ \ \     \sd{{\syntax{state}}} 
\\
${\rewriteLong {}}\s$ &  :  &    \syntax{Module}  $\times$  \syntax{state}  $\times$   {\toby{\syntax{Rhs}}}
  \ \  $\longrightarrow $ \ \     \syntax{heap} $\times$ \syntax{val}  
 
\end{tabular}

\noindent
\end{definition}
 
\paragraph{Arising and Reachable Configurations}  
\label{formal:reachable}

Policies need to be satisfied in all configurations which may arise during execution of some program. This leads us the concept of {\em arising} configuration. Arising configurations allow us to restrict the set of configurations we need to consider. For example, in a program where a class does not export visibility to a field, the constructor initialises the field to say $0$, and all method calls increment that field, the arising configurations will only consider states where the field is positive.

A configuration is reachable from another configuration, if the former may be required for the evaluation of the latter after any number of steps.   

$
\SP  \Reach \ :
\ \toby{\syntax{Module} \times \syntax{state} \times \syntax{Stmts}}  \\
\SP\SP\SP\SP \SP\SP\SP\SP \longrightarrow \mathcal{P}( \toby{\syntax{state} \times \syntax{Stmts}}  )
$

%\kjx{again: e or Stmts??? fix here, fix in the paper} STmts

\noindent
In figure \ref{fig:reach} we define the function  $\Reach$  by cases on the structure of the expression, and depending on the execution of the statement. The set $\Reach(\M, \sigma, \stmts)$
  collects all configurations reachable during execution of $\sigma, \stmts$. Note that the function  $\Reach(\M, \sigma, \stmts) $ is defined, even when the execution should diverge. This is important, because it allows us to give meaning to capability policies without requiring termination. 
%In case of divergence, $\Reach(\M, \sigma,\stmtssfra) $ will be an infinite set. 
 
 
 
We  then define $\Arising(\M)$ as the set of runtime configurations
which may be reached during execution of some initial context ($\sigma_0$,$\code_0$).
 A context is initial if its heap contains only objects of class \prg{Object}.
%, and its stack contains only one frame]
 % and the code contains exclusively method calls of methods defined in $M$. }
%   form 
%   consider all configurations which may be reached from such initial configurations.
  %  which are well-typed
% under the assumptions that \kw{x} and \this\ denote objects of class \prg{Object}.
%These concepts are defined in App. \ref{formal:arising}.
%The {\em arising} configurations are those which may be reached by executing an initial configuration, where initial configurations all configurations that may be encountered at the start of program execution. %, \cf \ref{def:initial}.

\begin{definition}[Arising and Initial configurations] $ $ We define the mappings \\
$\begin{array}{lcl}
\SP {\mathcal{I}nit} & \ : \ & \toby{\syntax{Module}} \   \longrightarrow\  \mathcal{P}( \toby{\syntax{state} \times \syntax{Stmt}}  )
\\
 \Arising & : &  \toby{\syntax{Module}}    \longrightarrow \mathcal{P}( \toby{\syntax{state}} \times \syntax{Stmts}  )
\end{array}$

\noindent
as follows:\\
$\begin{array}{l@{\,}l}
 {\mathcal{I}nit}(\M)   =  &  \{ \ (\ \sigma_0, \kw{new}\ \clss{}.\prg{m}\lp\kw{new}\  \clss{'} \rp) \ |    \ \clss{},\clss{'}\in dom( \M) \\
    &  \SP  \mbox{where}\  \sigma_0\ =\ ( (\iota,\kw{null}),\chi_0), \\
    & \SP  \mbox{and}\   \chi_0(\iota)=(\prg{Object}, \emptyset)\  \}
 \\
 \Arising(\M)    =   & \    \bigcup_{ (\sigma,\code{})\in  \mathcal{I}nit (\M)}  \Reach(\M, \sigma, \code{})
\end{array}$
\end{definition}


\noindent
Initial configuration should be as ``minimal'' as possible, We therefore construct a heap which has only one object, and execute a method call on a newly created
object, with another newly created object as argument. 




\begin{figure*} 
$
\begin{array}{lcl}
%  \Reach(\M{}, \sigma,\kw{a}) & \ =\  & \{ \ ( \kw{skip}, \sigma ) \}  %  \{ \ (\kw{a},\sigma) \ \}
  \\
 \Reach(\M{}, \sigma, \prg{v}\kw{:=} \kw{new}\, \clss{}\lp \prg{a}_1,...\prg{a}_n \rp) & = &  \{ \ ( \prg{v}\kw{:=}\kw{new}\, \clss{}\lp \prg{a}_1,...\prg{a}_n \rp,\sigma), \ ( \kw{skip}, \sigma' ) \}
 \\
 & &  \mbox{where }  \M,\sigma, \prg{v}\kw{:=} \kw{new}\, \clss{}\lp \prg{a}_1,...\prg{a}_n \rp \leadsto \sigma'
 \\
\Reach(\M{}, \sigma,\prg{stmt}\semi\prg{stmts}) & = &  
\Reach(\M,\sigma,\prg{stmt}) \cup \Reach(\M, \sigma',\prg{stmts})  
\\ & & \mbox{where }  \M,\sigma, \prg{stmt} \leadsto \sigma'
\\
\Reach(\M{}, \sigma,\prg{v}\kw{:=}\prg{a}) & = & 
 \{ ( \prg{v}\kw{:=}\prg{a}, \sigma), \ ( \kw{skip}, \sigma' ) \}
 \\
 & &  \mbox{where }  \M,\sigma, \prg{v}\kw{:=} \prg{a} \leadsto \sigma'
\\
\Reach(\M{}, \sigma,\prg{v} \kw{:=} \prg{a}\kw{.}\prg{m}\lp \prg{a}_1,...\prg{a}_n \rp) & = &
\{ \ (\prg{v} \kw{:=}\prg{a}\kw{.}\prg{m}\lp \prg{a}_1,...\prg{a}_n \rp, \sigma ) , \ ({\kw {skip}}, \sigma''' ) \  \} \ \cup 
\ \Reach(\M{}, \sigma',\prg{stmts}) \\ 
& &  \mbox{where}
% \ \chi=\sigma\downarrow_1, \mbox{and}\ \sigma'=(\kw{this}\mapsto \interp{a_1}{\sigma},x_1 \mapsto \interp{a_n}{\sigma}..x_n \mapsto \interp{a_n}{\sigma}),\chi)\ 
 \\ 
& &  \mbox{and} \  
BLA

% \Meths {} 
%      {\chi(\interp{a_1}{\sigma})\downarrow_1} 
%      {\prg{m}}  
      = ... \lp\ \prg{stmts}\kw{;} \, \kw{return} \prg{a}\ \rp   
     \  \mbox{and}  \\
     & & 
       \M,\sigma', \prg{stmts}  \leadsto \sigma''   \  \mbox{and} \ \sigma'''= (\sigma\downarrow_1[\prg{v} \mapsto \interp{a}{\sigma''}],  \sigma''\downarrow_2)
\\
\Reach(\M{}, \sigma,{\kw {skip}}) & = &
\{ \ ({\kw {skip}}, \sigma ) \  \}   \\ 
\Reach(\M{}, \sigma,{\kw {if}}\, \prg{a}\, {\kw {then}}\, \prg{stmts}_1\, {\kw {else}}\, \prg{stmts}_2\, ) & = &
\{ \ ( {\kw {if}}\, \prg{a}\, {\kw {then}}\, \prg{stmts}_1\, {\kw {else}}\, \prg{stmts}_2, \sigma ) , \  \}  \  \cup \
\Reach(\M{}, \sigma,\prg{stmts''})\\
& & \mbox{ where } \prg{stmts''}=\prg{stmts}_1 \mbox{ if } \interp{a}{\sigma}={\kw {true}},\ \mbox{ otherwise } 
\prg{stmts''}=\prg{stmts}_2
\end{array}
$
\caption{Reachable Configurations\label{fig:reach}}
\end{figure*} 



 
   
