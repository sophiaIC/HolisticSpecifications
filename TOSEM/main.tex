\documentclass[acmsmall]{acmart}
%\usepackage{amssymb}
\usepackage{stmaryrd}

%\usepackage{subcaption} %kjx
\usepackage[utf8]{inputenc}
\usepackage[british]{babel}
\usepackage{xspace, listings, lstcustom, wrapfig, graphicx, enumerate}
\usepackage{paralist}
\usepackage{color,colortbl, relsize}
\usepackage{rotating}
\usepackage{multirow}
\usepackage[normalem]{ulem}
\usepackage{soul}
\usepackage{tcolorbox}
\usepackage[scaled=.9, light]{zlmtt}
\usepackage{siunitx}
\usepackage{setspace}
\usepackage{enumitem}
\usepackage{graphicx}

% Note: the code for hyperlinked images for orcids must come before  \newcommand{\ttt}{\prg{true}}
\newcommand{\ff}{\prg{false}}
\newcommand{\unkn}{\prg{b???}}
\newcommand{\bv}{\prg{bval}}


\newcommand{\prg}[1]{{\mbox{\tt{#1}}}}
 \newcommand{\prgCol}[1]{#1}

 \newcommand{\forget}[1]{}
\newcommand{\etc}{{\it etc.}}
\newcommand{\eg}{{\it e.g.\,}}
\newcommand{\ie}{{\it i.e.\,}}

\newcommand{\Future}[1]{\lozenge\, #1}% {\bullet #1}% {{{\mathcal F}}(#1)} % {{{\mathcal B}}(#1)}
\newcommand{\Using}[2]{#1\,\kw{in}\, #2} %{{{\mathcal U}}(#1,#2)}
\newcommand{\SigmaUsing}[2]{#1\@ #2} %{{{\mathcal U}}(#1,#2)}
\newcommand{\Past}[1] {\nabla #1} %{\lozenge\!\!\!\!\-\!\!-\,#1}
%{\lozenge\!\!\!\!\!\circ  #1} % {\lozenge\!\!\!\!\-\!\!- #1} %{\upupsilon #1}  %{\nabla #1} %{\circ #1}%  {{{\mathcal P}}(#1)}
\newcommand{\Initial}[1] {{{\mathcal I}\!nitial}(#1)}

\newcommand{\Pol}[1] {{\ensuremath{\prg{Pol}\_{\prg{#1}}}}}
%\newcommand{\MOne} {{\ensuremath{\prg{M}_{\prg{BA}}}}}
%\newcommand{\MTwo} {{\ensuremath{\prg{M}_{\prg{BA}'}}}}

\newcommand{\strongImplies}{\leqq} %{{ \,^\sqsubset\!\!\!_{\sim}\, }}
\newcommand{\weakImplies}{\lessapprox} %{{ \,^\sqsubset\!\!\!_{\sim}\, }}
\newcommand{\frames}{~\kw{frames}~}

\newcommand{\appref}[1]{see App.~\ref{#1}}

%\newcommand{\sE}{{\prg{e}}}

\newcommand{\LangOO} {\ensuremath{{\mathcal L}ang{_{\tt oo}}}}

% ------------------------------------------------------------------
%                                             positions, separations
\newcommand{\cf}{{\it c.f.~}}
%\newcommand{\HYPHENA}{{\em-- }}
%\newcommand{\HYPHENB}{{\em-- }}
%\newcommand{\SP}{{\hspace{.1in}}}
%\newcommand{\s}{{\hspace{.01in}}}

%\newcommand{\obeys}{\,\textbf{\textrm{obeys}}\,}
%\newcommand{\StrongDom}{\ensuremath{\mathcal{S}\textrm{\textit{trong}}{\mathcal{D}}\textrm{\textit{om}}}}
%\newcommand{\Dom}{\ensuremath{\mathcal{D}}\textrm{\textit{om}}}

\newcommand{\Changes}[1]{\ensuremath{\mathcal{C}\textrm{\textit{hanges}}(#1)}}
\newcommand{\VisibleLit}{\ensuremath{\mathcal{V}\textrm{\textit{isible}}}}

\newcommand{\Gives}{\ensuremath{\mathcal{G}\textrm{\textit{ives}}}}
\newcommand{\MayCall}{\ensuremath{\mathcal{M}\textrm{\textit{ay}}{\mathcal{C}}\textrm{\textit{all}}}}
%\newcommand{\Dom}{\ensuremath{\mathcal{D}\textrm{\textit{om}}}}
\newcommand{\MayRead}{\ensuremath{\mathcal{M}\textrm{\textit{ay}}{\mathcal{R}}\textrm{\textit{ead}}}}
\newcommand{\MayAccess}{\ensuremath{\mathcal{M}\textrm{\textit{ay}}{\mathcal{A}}\textrm{\textit{ccess}}}}
\newcommand{\CanAccess}[2]{\ensuremath{{\mathcal{A}}\textrm{\textit{ccess}}}(#1,#2)}
\newcommand{\Calls}[1]{\ensuremath{{\mathcal{C}}\textrm{\textit{alls}}}(\prg{#1})}
\newcommand{\Caller}{\ensuremath{{\mathcal{C}}\textrm{\textit{aller}}}}
%{\ensuremath{\mathcal{C}\textrm{\textit{an}}{\mathcal{A}}\textrm{\textit{ccess}}}(#1,#2)}
\newcommand{\WillAccessThrough}{\ensuremath{\mathcal{W}\textrm{\textit{ill}}{\mathcal{A}}\textrm{\textit{ccess}}{\mathcal{T}}\!\!\textrm{\textit{hrough}}}}
\newcommand{\modelsWithO}{\models\!\!\!\!{_{_{_{\tiny{\mathcal O}}}}}}
\newcommand{\A}{\ensuremath{A}}
\newcommand{\SA}{\ensuremath{{^{\small{\prg{s}}}\! A}}}
\newcommand{\SE}{\ensuremath{{^{\small{\prg{s}}}\!e}}}
\newcommand{\SEOne}{\ensuremath{{^{\small{\prg{s}}}\!e}}}
\newcommand{\SETwo}{\ensuremath{{^{\small{\prg{s}}}\!e'}}}
\newcommand{\B}{\ensuremath{B}}
\newcommand{\Arising}{{\mathcal{A}}\textrm{\textit{rising}}}

 %------------------------ syntax tables

\newcommand{\syntax}[1]{\prg{{\it #1}}}
\newcommand{\BBC}{$::=$} %in syntactic definitions
\newcommand{\SOR}{\ensuremath{\ \mid\ }} % BNF or
\newcommand{\MID}{{\SPsmall ~ \mid ~ \SPsmall }} % in sets


\newcommand{\pre}{\ensuremath{_{{pre}}}}   %kjx no \sc  in math mode
\newcommand{\post}{\ensuremath{_{{post}}}} %kjx no \sc  in math mode
\newcommand{\PRE}{\pre}
\newcommand{\POST}{\post}

%\newcommand{\eval}[2]{{\ensuremath{\langle{ {#1}}\rangle_{#2}}}}
\newcommand{\interp}[2]{{\ensuremath{\lfloor{ {#1}}\rfloor_{#2}}}}
%\newcommand{\interpBL}[1]{{\lceil   {#1}  \rfloor}}
%  { \langle \!\langle { {#1} \rangle \!\rangle}\! }
% { \langle   { {#1}  \rangle} }
% ------------------------------------------------------------------
%                                             English abbreviations

% ------------------------------------------------------------------
%                                              keywords, program text
\newcommand{\kw}[1]{{\bf{\sf {#1}}}}
%{\mbox{\prgCol{\rm{\bf {#1}}}}}
\newcommand{\lit}[1]{{\prg {#1}\xspace}}
\newcommand{\com}{\ensuremath{\prg{//}}}
%\newcommand{\cnf}{\ensuremath{\kappa}}

 \newcommand{\code}[1]{{\mbox{\tt{#1}}}}
\newcommand{\M}{\ensuremath{\prg{M}}}
%\newcommand{\C}{\ensuremath{\prg{C}}}

%\newcommand{\ext}{\mbox{\,\,{\kw {extends}}\,\,}}
%\newcommand{\extends}{\mbox{\,\,{\kw {extends}}\,\,}}
%\newcommand{\ass}{\mbox{{\kw {:=}}\,}}
\newcommand{\semi}{\mbox{{\kw {;}}\ }}
%\newcommand{\comma}{\mbox{{\kw {,}}\,}}
%\newcommand{\dotK}{\mbox{{\kw {.}}}}
%\newcommand{\class}{\mbox{{\kw {class}}\,\,}}
%\newcommand{\stat}{\mbox{{\kw {state}}}}
%\newcommand{\trans}{\mbox{{\kw {root}}}}
%            % {\mbox{{\kw {root state}}}}
%            % was {\mbox{{\kw {abs-state}}}}
\newcommand{\lb}{\prgCol{\mbox{\tt{\bf{\{ }}}}}
\newcommand{\rb}{\prgCol{\mbox{\tt{\bf{\} }}}}}
\newcommand{\lp}{\prgCol{\mbox{\tt{\bf{( }}}}}
\newcommand{\rp}{\prgCol{\mbox{\tt{\bf{) }}}}}
 





\newcommand{\assertTC}[2]{{\M{#1} \typCol{\vdash} {\prg{#2}}\s
    \typCol{\DDiamond_{r} }}}
\newcommand{\assertSC}[2]{{\M{#1} \typCol{\vdash} {\prg{#2}}\s
    \typCol{\DDiamond_{s}} }}




% \newcommand{\M}[1]  {{\ensuremath{\prg{M}{{\prg{#1}}}}}}
    % {\prg{P}}
%\newcommand{\Env}[1]{\envCol{\ensuremath{\Gamma{#1}}}}
% \newcommand{\state}[1]{\stCol{\ensuremath{\sigma???{#1}}}}
%\newcommand{\stackFrame}[1]{\stCol{\ensuremath{\phi{#1}}}}
%\newcommand{\heap}[1]{\stCol{\ensuremath{\chi{#1}}}}

%\newcommand{\expr}[1]{{\ensuremath{\prg{e{#1}}}}}
%\newcommand{\fld}[1]{{\ensuremath{\prg{f{#1}}}}}
%\newcommand{\param}{{\ensuremath{\prg{x}}}}
%
%\newcommand{\clss}[1]{\ensuremath{\prg{c}{\prg{#1}}}}
%\newcommand{\clssD}[1]{\ensuremath{\prg{d}{{#1}}}}
%\newcommand{\type}[1]{\ensuremath{\prg{t{#1}}}}
%
%\newcommand{\val}[1]{{\ensuremath{\prg{v}{\prgCol{#1}}}}}
%\newcommand{\res}[1]{{\ensuremath{\prg{dv}{#1}}}}
%\newcommand{\valOrDev}[1]{{\ensuremath{\prg{r}{#1}}}}


%\newcommand{\expandexp}[1]{}
%
%\newcommand{\oo}{object-oriented}
%\newcommand{\mExtS}{\ensuremath{\Downarrow}}
%
%% re-classification expression
%\newcommand{\cm}[1]{\this{\prgCol{\ensuremath{\mExtS}}}\prg{#1}}






% ------------------------------------------------------------------
%                                             identifiers in the examples
%                     ---------------------
%                                      Stack
 %                     ---------------------
%                                   Employee
\newcommand{\Empl}{\prg {Empl}}
\newcommand{\Pers}{\prg {Pers}}
\newcommand{\E}{{_\prg {e}}}
\newcommand{\Man}{\prg {Boss}}
\newcommand{\Stud}{\prg {Stdt}}
\newcommand{\Scholar}{\prg {Scholar}}
\newcommand{\sal}{\prg {sal}}
\newcommand{\bYear}{\prg {bYear}}
\newcommand{\frnd}{\prg {frnd}}
% \newcommand{\marks}{\prg {marks}}
\newcommand{\pay}{\prg {fee}}
\newcommand{\setP}{\prg {set}}
\newcommand{\String}{\prg {String}}
\newcommand{\hobby}{\prg {hobby}}
\newcommand{\intg}{\kw {int}}
\newcommand{\boolg}{\kw {bool}}
\newcommand{\ii}{\prg {i}}
\newcommand{\prom}{\prg {promote}}
\newcommand{\mkStud}{\prg {study}}
\newcommand{\dem}{\prg {demote}}
\newcommand{\grad}{\prg {employ}}
\newcommand{\assist}{\prg {assist}}
\newcommand{\amount}{\prg {amount}}

\newcommand{\Phil}{\prg {Phil}}
\newcommand{\Book}{\prg {Book}}
\newcommand{\Person}{\prg {Person}}
\newcommand{\fav}{\prg {favourite}}

\newcommand{\aMan}{\prg {billy}}
\newcommand{\anotherMan}{\prg {bob}}
\newcommand{\aMutMan}{\prg {bea}}
\newcommand{\employees}{employees}
\newcommand{\aStud}{\prg {steve}}
\newcommand{\aPers}{\prg {peter}}
\newcommand{\aStudEmpl}{\prg {mary}}
\newcommand{\anoStudEmpl}{\prg {judy}}
\newcommand{\yetAnoStudEmpl}{\prg {betty}}
\newcommand{\yetYetAnoStudEmpl}{\prg {joe}}

\newcommand{\exprog}{_{\prg{ei}}}



% structuring macros
\newcommand{\EndDefLemma}{\noindent $\bigtriangleup$}



\newcommand{\para}[1]{{\ensuremath{\prg{x}{#1}}}}


\newcommand{\methAndBodyNew}{\ensuremath{
\prg{\type{}~m \lp \type{_1}~\param{}\rp\   \ \lb
~\expr{}~\rb} }}
\newcommand{\methAndBodyNewOne}{\ensuremath{
\prg{\type{}~m \lp \type{_1}~\param{}\rp\   \ \lb
~\expr{'}~\rb} }}
\newcommand{\methAndBodyNewTwo}{\ensuremath{
\prg{\type{}~m \lp \type{_1}~\param{}\rp\   \ \lb
~\expr{''}~\rb} }}
\newcommand{\methAndBodyNewThree}{\ensuremath{
\prg{\type{''}~m \lp \type{_1''}~\param{}\rp\   \ \lb
~\expr{''}~\rb} }}
\newcommand{\methAndBodyNewPrime}{\ensuremath{
\prg{\type{}~m \lp  \type{_1}~\para{} \rp\   \
\lb~\expr{'}~\rb} }}
\newcommand{\methAndBodyNewPrimeAll}{\ensuremath{
\prg{\type{'}~m \lp  \type{_1'}~\para{} \rp\   \
\lb~\expr{'}~\rb} }}
\newcommand{\methAndBodyNewP}{\ensuremath{ % slight diff in Primes from above
\prg{\type{}~m \lp  \type{_1}~\para{} \rp\   \
\lb~\expr{'}~\rb} }}
\newcommand{\methAndBodyFromOneSig}{\ensuremath{
\prg{\type{}~m \lp  \type{_1}~\para{} \rp\ \eff{} \lb~ ... ~\rb}
}}

%-----------------

\newcommand{\Body}[2]{\ensuremath{\mathcal{B}ody(#1,\prg{#2})}}

\newcommand{\T}[1]{{\ensuremath{\type{}{#1}}}}

% find the declaration of an identifier
\newcommand{\LookUp}[2]{\ensuremath{ {#1}({\prg{#2}}) }}
\newcommand{\LookUpEnv}[2]{{ \Env{#1}({\prg{#2}}) }}
\newcommand{\LookUpState}[2]{{ \ensuremath{\sigma{#1}}({\prg{#2}}) }}
\newcommand{\LookUpHeap}[2]{{ \ensuremath{\heap{#1}}({\prg{#2}}) }}
\newcommand{\LookUpBody}[2]{{ {\M{#1}}({\prg{#2}}) }}
%\newcommand{\LookUpClass}[2]
%        { \ensuremath{{\mathcal C}%{\mathcal D}%{\it ef}
%        (}\M{#1},\prg{#2}\ensuremath{)}}
\newcommand{\LookUpField}[3]
        {\ensuremath{{\mathcal F}{\mathcal D}% {\it f}
                (}\M{#1},\prg{#2},\prg{#3}\ensuremath{)}}
\newcommand{\LookUpMethod}[3]
        {\ensuremath{{\mathcal M}{\mathcal D}% {\it f}
         (}\M{#1},\prg{#2},\prg{#3}\ensuremath{)}}
\newcommand{\Undef}{{\ensuremath{\mathcal U\!}{\it df}}}
\newcommand{\ClassOf}[2] { \ensuremath{{\mathcal C}{\mathit{lass}}(#1)_{#2}}}

%-------------------Part Three: Types ...
% Notation for types ( open and close types).


\newcommand{\subclass}{\typCol{\sqsubseteq}}
\newcommand{\sclass}{\subclass}
\newcommand{\widen}{\typCol{\ensuremath {\le}}}

%-------------------- implies, and, or, iff, etc -----------------
\newcommand{\inset}[3]{\prg{#1}\!\in\!\prg{#2},\ldots,\prg{#3}}
% do not change above, it is also used with \forall, \exists
%\newcommand{\IMPLIES}{{\SP \Longrightarrow \SP}}
 \newcommand{\AND}{{\SPsmall {\mbox{and}} \SPsmall}}
\newcommand{\WITH}{{\SPsmall {\mbox{with}} \SPsmall}}

%\newcommand{\IFF}{{\SPsmall {\mbox{iff}} \SPsmall}}
\newcommand{\IFF}{{\SP {\mbox{ iff }} \SP}}

\newcommand{\OR}{{\SPsmall {\mbox{or}} \SPsmall}}
\renewcommand{\implies}{{\ensuremath{\longrightarrow}}}
\newcommand{\upd}{{\mapsto}}

%----------------------- Conformance --------------------
\newcommand{\stateConf}[3]
  {\M{#1},\Env{#2}
  \rtjCol{\ensuremath{\vdash}}\,\ensuremath{\state{#3}}\,\rtjCol{\ensuremath{\DDiamond}}  }
\newcommand{\stackFrameHeapConf}[4]
  {\M{#1},\Env{#2}
  \rtjCol{\ensuremath{\vdash}}\,\ensuremath{\stackFrame{#3},\heap{#4}}\,\rtjCol{\ensuremath{\DDiamond}}  }
\newcommand{\stackFrameHeapConfS}[4]
  {\M{#1},\Env{#2}
  \rtjCol{\ensuremath{\vdash}}\,\ensuremath{ {#3},\heap{#4}}\,\rtjCol{\ensuremath{\DDiamond}}  }

\newcommand{\stackFrameHeapConfNOT}[4]
  {\M{#1},\Env{#2}
  \rtjCol{\ensuremath{\not\vdash}}\,\ensuremath{\stackFrame{#3},\heap{#4}}\,\rtjCol{\ensuremath{\DDiamond}}  }

\newcommand{\stackFrameHeapConfLong}[4]
  {\M{#1}, {#2}
  \rtjCol{\ensuremath{\vdash}}\,\ensuremath{\stackFrame{#3},\heap{#4}}\,\rtjCol{\ensuremath{\DDiamond}}  }
\newcommand{\stackFrameHeapConfLongNOT}[4]
  {\M{#1}, {#2}
  \rtjCol{\ensuremath{\not\vdash}}\,\ensuremath{\stackFrame{#3},\heap{#4}}\,\rtjCol{\ensuremath{\DDiamond}}  }


\newcommand{\stateConfEnv}[3]
  {\M{#1}, {#2}
  \rtjCol{\ensuremath{\vdash}}\,\ensuremath{\state{#3}}\,\rtjCol{\ensuremath{\DDiamond}}  }
  \newcommand{\stateConfLong}[3]
  { {#1}, {#2}
        \rtjCol{\ensuremath{\vdash}}\,\ensuremath{\state{#3}}\,\rtjCol{\ensuremath{\DDiamond} } }
\newcommand{\stateConfLongNot}[3]
  { {#1}, {#2}
        \rtjCol{\ensuremath{\vdash}\!\!\!\not}\ \ensuremath{\state{#3}}\,
         \rtjCol{\ensuremath{\DDiamond} } }
\newcommand{\conf}[4]
  {\M{#1},\ensuremath{\heap{#2}}
        \rtjCol{\ensuremath{\vdash}} \ensuremath{\prg{#3}} \rtjCol{\ensuremath{\lhd}}
        \ensuremath{\prg{#4}}}
\newcommand{\confW}[4]
  {\M{#1},\ensuremath{\heap{#2}}
        \rtjCol{\ensuremath{\vdash}} \ensuremath{\prg{#3}} \rtjCol{<:}
        \ensuremath{\prg{#4}}}
% states conforming to each other:


\newcommand{\confState}[5]
    % Prog, eff, thsiMut, state, state
  {\confStateL{\M{#1}}{\eff{#2}}{\clss{#3}}{\state{#4}}{\state{#5}}}
  \newcommand{\confStateP}[4]
    % Prog, eff, state, state
  {\ensuremath{\M{#1},{\eff{#2}}\vdash{\state{#3}}\lhd{\state{#4}}}}
\newcommand{\confStateL}[5]
  {\ensuremath{#1,{#2},{#3}\vdash{#4}\lhd{#5}}}
\newcommand{\confStateLP}[5]
  {\ensuremath{#1,{#2}\vdash{#3}\lhd{#4}}}




% ------------------------------------------- Type assertions ----------------
\newcommand{\assert}[3]{\prg{#1} \vdash \prg{#2}~:~\prg{#3}}
% \newcommand{\assertT}[5] {\M{#1}, \Env{#2} \   \ensuremath{\vdash} \ \prg{#4}~
% \ensuremath{:~\prg{#5}}}
\newcommand{\EFFSEP}{\typCol{[\!]}}

\newcommand{\assertTAll}[6]
    % program, env, expr, type, mut, eff
{\M{#1}, \Env{#2} \   \typCol{\ensuremath{\vdash}} \ \prg{#3}~
\ensuremath{\typCol{:~} \prg{#4} } }% ~\EFFSEP~ {#5}~\EFFSEP~{#6} }}
\newcommand{\assertTAllS}[6]
% with subsumption
    % program, env, expr, type, mut, eff
{\M{#1}, \Env{#2} \   \typCol{\ensuremath{\vdash_s}} \
\prg{#3}~
\ensuremath{\typCol{:~} \prg{#4} } }% ~\EFFSEP~ {#5}~\EFFSEP~{#6} }}
\newcommand{\assertTAllR}[6]
% with subsumption
    % program, env, expr, type, mut, eff
{\M{#1}, \Env{#2} \   \typCol{\ensuremath{\vdash_r}} \
\prg{#3}~
\ensuremath{\typCol{:~} \prg{#4} } }% ~\EFFSEP~ {#5}~\EFFSEP~{#6} }}
\newcommand{\assertTAllL}[6]
    % program, env, expr, type, mut, eff
{{#1}, {#2} \   \typCol{\ensuremath{\vdash}} \ \prg{#3}~
\ensuremath{\typCol{:~}} \prg{#4} }%~\EFFSEP~ {#5}~\EFFSEP~{#6} }}
\newcommand{\assertTFour}[4]
    % for the explanations, ignore P and \eff
{ {#1} \   \typCol{\ensuremath{\vdash}} \ \prg{#2}~
\ensuremath{\typCol{:~} \prg{#3} ~\EFFSEP~ {#4}  }}
% same as above, but in 2 lines!
% do not remove! it is difficult to program, so
% leave it in, even if temporarily not used!
\newcommand{\assertTAllTwoL}[6]
    % program, env, expr, type, mut, eff
{\M{#1}, \Env{#2} \   \ensuremath{\vdash} \ \prg{#3}~
\ensuremath{:~\\ \SP\SP\SP\SP\SP
   \prg{#4} }}
\newcommand{\assertTAllEnvS}[6]
    %as before, but the environment is given in full
{\M{#1}, {#2} \   \typCol{\ensuremath{\vdash}_s} \ \prg{#3}~
\ensuremath{\typCol{:~}}\prg{#4}  }
\newcommand{\assertTAllEnvR}[6]
    %as before, but the environment is given in full
{\M{#1}, {#2} \   \typCol{\ensuremath{\vdash}_r} \ \prg{#3}~
\ensuremath{\typCol{:~}}\prg{#4}  }
\newcommand{\assertTAllEnv}[6]
    %as before, but the environment is given in full
{\M{#1}, {#2} \   \typCol{\ensuremath{\vdash}} \ \prg{#3}~
\ensuremath{\typCol{:~}}\prg{#4}  }




 % Effects


% receiver mutations

\newcommand{\mutOr}[1] % takes as parameter the program
        {\ensuremath{\sqcup\!{_{{#1}}}}} % binary on muts

\newcommand{\assertc}[3]{\ensuremath{\M{#1}
 \typCol{\vdash}\, \prg{#2}\, \subclass\, \prg{#3}}}

 \newcommand{\assertw}[3]{\ensuremath{\M{#1} \typCol{\vdash}
\prg{#2}\widen\prg{#3}}}
\newcommand{\assertion}[3]{\ensuremath{#1 \typCol{\vdash} \prg{#2}~#3}}
\newcommand{\notStateConfEnv}[3]
  {\M{#1},\prg{#2}\,
  \rtjCol{\ensuremath{\not\vdash}}\,\ensuremath{\state{#3}}\,\rtjCol{\ensuremath{\DDiamond}}  }
 \newcommand{\notConf}[4]
  {\M{#1},\ensuremath{\heap{#2}}
        \rtjCol{\ensuremath{\not\vdash}} \ensuremath{\prg{#3}} \rtjCol{\ensuremath{\lhd}}
        \ensuremath{\prg{#4}}}







%Macros for inference rules
\newcommand{\inferencerule}[2]{
\begin{array}{l} #1 \\ \hline #2 \end{array}
}

\newcommand{\inferenceruleN}[3]
{
\begin{array}{l}
% \SP\SP\SP\SP\SP\SP\SP\SP
% \SP\SP\SP\SP\SP\SP\SP\SP
\SP\SP\SP\SP\SP\SP\SP\SP
\SP\SP\SP\SP\SP\SP  {\sf #1}
\\ #2  \\ \hline   #3
  \end{array}
}

\newcommand{\inferenceruleNN}[3]
{
\begin{array}{l}
\SP\SP\SP\SP\SP\SP\SP\SP
\SP\SP\SP\SP\SP\SP\SP\SP
\SP\SP\SP\SP\SP\SP\SP\SP
\SP\SP\SP\SP\SP\SP\SP\SP

   {\sf #1}
\\ #2  \\ \hline   #3
  \end{array}
}

%===========================================================================
%  Definition-Lemma-Theorem-Proof
%
% Adaptation of LaTeX's theorem environment; can be used as a command
% (eg just \Lemma not \begin{Lemma}) and no italicisation; also works
% with ptmac; result numbering is uniform within subsections and can be
% suppressed.
%
\newif\ifNumberResults\NumberResultstrue
\def\@@opargbegintheorem#1#2#3{\@@@@begintheorem{\bf\@@thmname{#1}{#2}(#3)}}
\def\@@begintheorem#1#2{\@@@@begintheorem{\bf\@@thmname{#1}{#2}}}
\def\@@@@begintheorem#1{\par\removelastskip\smallskip\noindent{#1}}
\def\@@thmname#1#2{#1\ \ifNumberResults#2\ \fi}

% similarly \Proof or \begin{Proof}...\end{Proof}
% prefer proofs with statements if possible - hence \penalty700
%\let\qedsymbol\S% make it \square or \blacksquare if you like for kb
\let\qedsymbol \Box
\def\qed{\hfill{$\qedsymbol$}}
\def\Proof{\par\removelastskip\smallskip\penalty700\noindent{\bf Proof}\enskip}
\def\endProof{\qed\penalty-700 \smallskip}
\let\endproof\endProof

%   The actual words

%\newtheorem{theo}{Theorem} this is necessay if we want eg \newtheorem{definition}[theo]{Definition} to work 
\newtheorem{definition}{Definition} %  \newtheorem{definition}[theo]{Definition}
\newtheorem{example}{Example} %  \newtheorem{example}[theo]{Example}
%\newtheorem{lemma}{Lemma} % \newtheorem{mylemma}[theo]{Lemma}
%\renewtheorem{mylemma}{Lemma} % \newtheorem{mylemma}[theo]{Lemma}
\newtheorem{conjecture}{Conjecture} %\newtheorem{conjecture}[theo]{Conjecture}
\newtheorem{theorem}{Theorem} %\newtheorem{theo}{Theorem}
\newtheorem{note}{Note} % \newtheorem{note}[theo]{Note}
\newtheorem{observation}{Observation} %  \newtheorem{observation}[theo]{Observation}


%--------------------------------- the ones that Susan introduced
\newcommand{\z}{{\prg z}}

\newcommand{\Fields}[3]{\ensuremath{{\mathcal F}(}\\Mg{#1},\prg{#2},
\prg{#3}\ensuremath{)} }
\newcommand{\FieldIds}[2]{\ensuremath{{\mathcal F}{\it {s}}(\M{#1},\prg{#2})}}
\newcommand{\Meths}[3]{\ensuremath{{\mathcal M}(}\M{#1},\prg{#2},
\prg{#3}\ensuremath{)} }







 


\newcommand{\rewriteLong}[1]
{\rtCol{\ensuremath{\ \leadsto\!\!\!\!\!\!_{_{#1}\,\ \ }}}}
\newcommand{\rewrite}[1]
{\rtCol{\ensuremath{\ \leadsto\!\!\!\!\!\!_{_{\M{#1}}\,\ \ }}}}
\newcommand{\rewriteP}
    {{\ensuremath{\ \ \leadsto\!\!_{_{\prg{P}}}\,\,\,}}}
 % {\rtCol{\ensuremath{\ \ \leadsto \!\!\!\!_{{\M{}}\,\ \ }}}}
% \newcommand{\Conf}[2]% configurations: < #1, #2 >
% inside the mathmode
% { \langle \prg{#1}, {#2} \rangle }
% {  \prg{#1}\ensuremath{,}{#2}  }
%\newcommand{\greenComma}{\rtCol{\ensuremath{,}}}
%\newcommand{\ConfL}[3]
%{  \prg{#1}\greenComma\stackFrame{#2}\greenComma\heap{#3}  }
%\newcommand{\ConfLLong}[3]
%{  \prg{#1}\greenComma {#2}\greenComma\heap{#3}  }
%\newcommand{\ConfR}[2]
%{  \prg{#1}\greenComma\heap{#2}  }
%

% ------------------------------------- Well formed, unique acyclic ---------------
 
%{\ensuremath{\M{#1} \vdash {\prg{#2}}\s \DDiamond_{\it {eff}}  }}

  \newcommand{\nullPEC}{\lit {nullPntrExc}}



%\newcommand{\Obj}{\prg{Object}}
%\newcommand{\void}{\kw{void}}
%\newcommand{\newK}{{\kw {new}}~} % {{\kw {new}}}% no ~ around it
%\newcommand{\newKW}{~{\kw {new}}~}
%




\newcommand{\back}{{$\!\!\!\!\!\!\!$}}


\newcommand{\WideFig}[3]
{
\begin{figure*}[t]
\begin{center}
\noindent
\fbox{
\begin{minipage}{4.7 in}
{#1} % the contents
\end{minipage}
}
\caption{#2}
\label{#3}
\end{center}
\end{figure*}
}


\newcommand{\WideFigWhere}[4] % you can specify where it should appear!
{
\begin{figure*}[{#4}]
\begin{center}
\noindent
\fbox{
\begin{minipage}{5. in}
{#1} % the contents
\end{minipage}
}
\caption{#2}
\label{#3}
\end{center}
\end{figure*}
}

\newcommand{\BigWideFigWhere}[4] % you can specify where it should appear!
{
\begin{figure*}[{#4}]
\begin{center}
\noindent
{\normalsize
\hrule
\begin{minipage}{5. in}
{#1} % the contents
\end{minipage}
\hrule
}
\caption{#2}
\label{#3}
\end{center}
\end{figure*}
}

\newcommand{\NotTooWideFigWhere}[4] % you can specify where it should appear!
{
\begin{figure*}[{#4}]
\begin{center}
\noindent
\fbox{
\begin{minipage}{4.3 in}
{#1} % the contents
\end{minipage}
}
\caption{#2}
\label{#3}
\end{center}
\end{figure*}
}


\newcommand{\opsemExprFig}
{\BigWideFigWhere {\opsemExpr} {Execution of expressions\MD}
{opsemTrad} {htbp} }



\newcommand{\mlc}{ }%{\heartsuit}
%\newcommand{\mcl}{ }%{\heartsuit}
\newcommand{\mc}{ }%{\heartsuit}

\newcommand{\BigNotTooWideFigWhere}[4] % you can specify where it should appear!
{
\begin{figure*}[{#4}]
\begin{center}
\noindent
{\normalsize
\hrule
\begin{minipage}{4.3 in}
{#1} % the contents
\end{minipage}
\hrule
}
\caption{#2}
\label{#3}
\end{center}
\end{figure*}
}

 

%]})
%}
, otherwise won't compile

%\makeatletter
%\RequirePackage[bookmarks,unicode,colorlinks=true]{hyperref}%
%   \def\@citecolor{blue}%
%   \def\@urlcolor{blue}%
%   \def\@linkcolor{blue}%
%\def\UrlFont{\rmfamily}%
%\def\orcidID#1{\smash{\href{http://orcid.org/#1}{\protect\raisebox{-1.25pt}{\protect\includegraphics{orcid_color.eps}}}}}
%\makeatother


 \newcommand{\ttt}{\prg{true}}
\newcommand{\ff}{\prg{false}}
\newcommand{\unkn}{\prg{b???}}
\newcommand{\bv}{\prg{bval}}


\newcommand{\prg}[1]{{\mbox{\tt{#1}}}}
 \newcommand{\prgCol}[1]{#1}

 \newcommand{\forget}[1]{}
\newcommand{\etc}{{\it etc.}}
\newcommand{\eg}{{\it e.g.\,}}
\newcommand{\ie}{{\it i.e.\,}}

\newcommand{\Future}[1]{\lozenge\, #1}% {\bullet #1}% {{{\mathcal F}}(#1)} % {{{\mathcal B}}(#1)}
\newcommand{\Using}[2]{#1\,\kw{in}\, #2} %{{{\mathcal U}}(#1,#2)}
\newcommand{\SigmaUsing}[2]{#1\@ #2} %{{{\mathcal U}}(#1,#2)}
\newcommand{\Past}[1] {\nabla #1} %{\lozenge\!\!\!\!\-\!\!-\,#1}
%{\lozenge\!\!\!\!\!\circ  #1} % {\lozenge\!\!\!\!\-\!\!- #1} %{\upupsilon #1}  %{\nabla #1} %{\circ #1}%  {{{\mathcal P}}(#1)}
\newcommand{\Initial}[1] {{{\mathcal I}\!nitial}(#1)}

\newcommand{\Pol}[1] {{\ensuremath{\prg{Pol}\_{\prg{#1}}}}}
%\newcommand{\MOne} {{\ensuremath{\prg{M}_{\prg{BA}}}}}
%\newcommand{\MTwo} {{\ensuremath{\prg{M}_{\prg{BA}'}}}}

\newcommand{\strongImplies}{\leqq} %{{ \,^\sqsubset\!\!\!_{\sim}\, }}
\newcommand{\weakImplies}{\lessapprox} %{{ \,^\sqsubset\!\!\!_{\sim}\, }}
\newcommand{\frames}{~\kw{frames}~}

\newcommand{\appref}[1]{see App.~\ref{#1}}

%\newcommand{\sE}{{\prg{e}}}

\newcommand{\LangOO} {\ensuremath{{\mathcal L}ang{_{\tt oo}}}}

% ------------------------------------------------------------------
%                                             positions, separations
\newcommand{\cf}{{\it c.f.~}}
%\newcommand{\HYPHENA}{{\em-- }}
%\newcommand{\HYPHENB}{{\em-- }}
%\newcommand{\SP}{{\hspace{.1in}}}
%\newcommand{\s}{{\hspace{.01in}}}

%\newcommand{\obeys}{\,\textbf{\textrm{obeys}}\,}
%\newcommand{\StrongDom}{\ensuremath{\mathcal{S}\textrm{\textit{trong}}{\mathcal{D}}\textrm{\textit{om}}}}
%\newcommand{\Dom}{\ensuremath{\mathcal{D}}\textrm{\textit{om}}}

\newcommand{\Changes}[1]{\ensuremath{\mathcal{C}\textrm{\textit{hanges}}(#1)}}
\newcommand{\VisibleLit}{\ensuremath{\mathcal{V}\textrm{\textit{isible}}}}

\newcommand{\Gives}{\ensuremath{\mathcal{G}\textrm{\textit{ives}}}}
\newcommand{\MayCall}{\ensuremath{\mathcal{M}\textrm{\textit{ay}}{\mathcal{C}}\textrm{\textit{all}}}}
%\newcommand{\Dom}{\ensuremath{\mathcal{D}\textrm{\textit{om}}}}
\newcommand{\MayRead}{\ensuremath{\mathcal{M}\textrm{\textit{ay}}{\mathcal{R}}\textrm{\textit{ead}}}}
\newcommand{\MayAccess}{\ensuremath{\mathcal{M}\textrm{\textit{ay}}{\mathcal{A}}\textrm{\textit{ccess}}}}
\newcommand{\CanAccess}[2]{\ensuremath{{\mathcal{A}}\textrm{\textit{ccess}}}(#1,#2)}
\newcommand{\Calls}[1]{\ensuremath{{\mathcal{C}}\textrm{\textit{alls}}}(\prg{#1})}
\newcommand{\Caller}{\ensuremath{{\mathcal{C}}\textrm{\textit{aller}}}}
%{\ensuremath{\mathcal{C}\textrm{\textit{an}}{\mathcal{A}}\textrm{\textit{ccess}}}(#1,#2)}
\newcommand{\WillAccessThrough}{\ensuremath{\mathcal{W}\textrm{\textit{ill}}{\mathcal{A}}\textrm{\textit{ccess}}{\mathcal{T}}\!\!\textrm{\textit{hrough}}}}
\newcommand{\modelsWithO}{\models\!\!\!\!{_{_{_{\tiny{\mathcal O}}}}}}
\newcommand{\A}{\ensuremath{A}}
\newcommand{\SA}{\ensuremath{{^{\small{\prg{s}}}\! A}}}
\newcommand{\SE}{\ensuremath{{^{\small{\prg{s}}}\!e}}}
\newcommand{\SEOne}{\ensuremath{{^{\small{\prg{s}}}\!e}}}
\newcommand{\SETwo}{\ensuremath{{^{\small{\prg{s}}}\!e'}}}
\newcommand{\B}{\ensuremath{B}}
\newcommand{\Arising}{{\mathcal{A}}\textrm{\textit{rising}}}

 %------------------------ syntax tables

\newcommand{\syntax}[1]{\prg{{\it #1}}}
\newcommand{\BBC}{$::=$} %in syntactic definitions
\newcommand{\SOR}{\ensuremath{\ \mid\ }} % BNF or
\newcommand{\MID}{{\SPsmall ~ \mid ~ \SPsmall }} % in sets


\newcommand{\pre}{\ensuremath{_{{pre}}}}   %kjx no \sc  in math mode
\newcommand{\post}{\ensuremath{_{{post}}}} %kjx no \sc  in math mode
\newcommand{\PRE}{\pre}
\newcommand{\POST}{\post}

%\newcommand{\eval}[2]{{\ensuremath{\langle{ {#1}}\rangle_{#2}}}}
\newcommand{\interp}[2]{{\ensuremath{\lfloor{ {#1}}\rfloor_{#2}}}}
%\newcommand{\interpBL}[1]{{\lceil   {#1}  \rfloor}}
%  { \langle \!\langle { {#1} \rangle \!\rangle}\! }
% { \langle   { {#1}  \rangle} }
% ------------------------------------------------------------------
%                                             English abbreviations

% ------------------------------------------------------------------
%                                              keywords, program text
\newcommand{\kw}[1]{{\bf{\sf {#1}}}}
%{\mbox{\prgCol{\rm{\bf {#1}}}}}
\newcommand{\lit}[1]{{\prg {#1}\xspace}}
\newcommand{\com}{\ensuremath{\prg{//}}}
%\newcommand{\cnf}{\ensuremath{\kappa}}

 \newcommand{\code}[1]{{\mbox{\tt{#1}}}}
\newcommand{\M}{\ensuremath{\prg{M}}}
%\newcommand{\C}{\ensuremath{\prg{C}}}

%\newcommand{\ext}{\mbox{\,\,{\kw {extends}}\,\,}}
%\newcommand{\extends}{\mbox{\,\,{\kw {extends}}\,\,}}
%\newcommand{\ass}{\mbox{{\kw {:=}}\,}}
\newcommand{\semi}{\mbox{{\kw {;}}\ }}
%\newcommand{\comma}{\mbox{{\kw {,}}\,}}
%\newcommand{\dotK}{\mbox{{\kw {.}}}}
%\newcommand{\class}{\mbox{{\kw {class}}\,\,}}
%\newcommand{\stat}{\mbox{{\kw {state}}}}
%\newcommand{\trans}{\mbox{{\kw {root}}}}
%            % {\mbox{{\kw {root state}}}}
%            % was {\mbox{{\kw {abs-state}}}}
\newcommand{\lb}{\prgCol{\mbox{\tt{\bf{\{ }}}}}
\newcommand{\rb}{\prgCol{\mbox{\tt{\bf{\} }}}}}
\newcommand{\lp}{\prgCol{\mbox{\tt{\bf{( }}}}}
\newcommand{\rp}{\prgCol{\mbox{\tt{\bf{) }}}}}
 





\newcommand{\assertTC}[2]{{\M{#1} \typCol{\vdash} {\prg{#2}}\s
    \typCol{\DDiamond_{r} }}}
\newcommand{\assertSC}[2]{{\M{#1} \typCol{\vdash} {\prg{#2}}\s
    \typCol{\DDiamond_{s}} }}




% \newcommand{\M}[1]  {{\ensuremath{\prg{M}{{\prg{#1}}}}}}
    % {\prg{P}}
%\newcommand{\Env}[1]{\envCol{\ensuremath{\Gamma{#1}}}}
% \newcommand{\state}[1]{\stCol{\ensuremath{\sigma???{#1}}}}
%\newcommand{\stackFrame}[1]{\stCol{\ensuremath{\phi{#1}}}}
%\newcommand{\heap}[1]{\stCol{\ensuremath{\chi{#1}}}}

%\newcommand{\expr}[1]{{\ensuremath{\prg{e{#1}}}}}
%\newcommand{\fld}[1]{{\ensuremath{\prg{f{#1}}}}}
%\newcommand{\param}{{\ensuremath{\prg{x}}}}
%
%\newcommand{\clss}[1]{\ensuremath{\prg{c}{\prg{#1}}}}
%\newcommand{\clssD}[1]{\ensuremath{\prg{d}{{#1}}}}
%\newcommand{\type}[1]{\ensuremath{\prg{t{#1}}}}
%
%\newcommand{\val}[1]{{\ensuremath{\prg{v}{\prgCol{#1}}}}}
%\newcommand{\res}[1]{{\ensuremath{\prg{dv}{#1}}}}
%\newcommand{\valOrDev}[1]{{\ensuremath{\prg{r}{#1}}}}


%\newcommand{\expandexp}[1]{}
%
%\newcommand{\oo}{object-oriented}
%\newcommand{\mExtS}{\ensuremath{\Downarrow}}
%
%% re-classification expression
%\newcommand{\cm}[1]{\this{\prgCol{\ensuremath{\mExtS}}}\prg{#1}}






% ------------------------------------------------------------------
%                                             identifiers in the examples
%                     ---------------------
%                                      Stack
 %                     ---------------------
%                                   Employee
\newcommand{\Empl}{\prg {Empl}}
\newcommand{\Pers}{\prg {Pers}}
\newcommand{\E}{{_\prg {e}}}
\newcommand{\Man}{\prg {Boss}}
\newcommand{\Stud}{\prg {Stdt}}
\newcommand{\Scholar}{\prg {Scholar}}
\newcommand{\sal}{\prg {sal}}
\newcommand{\bYear}{\prg {bYear}}
\newcommand{\frnd}{\prg {frnd}}
% \newcommand{\marks}{\prg {marks}}
\newcommand{\pay}{\prg {fee}}
\newcommand{\setP}{\prg {set}}
\newcommand{\String}{\prg {String}}
\newcommand{\hobby}{\prg {hobby}}
\newcommand{\intg}{\kw {int}}
\newcommand{\boolg}{\kw {bool}}
\newcommand{\ii}{\prg {i}}
\newcommand{\prom}{\prg {promote}}
\newcommand{\mkStud}{\prg {study}}
\newcommand{\dem}{\prg {demote}}
\newcommand{\grad}{\prg {employ}}
\newcommand{\assist}{\prg {assist}}
\newcommand{\amount}{\prg {amount}}

\newcommand{\Phil}{\prg {Phil}}
\newcommand{\Book}{\prg {Book}}
\newcommand{\Person}{\prg {Person}}
\newcommand{\fav}{\prg {favourite}}

\newcommand{\aMan}{\prg {billy}}
\newcommand{\anotherMan}{\prg {bob}}
\newcommand{\aMutMan}{\prg {bea}}
\newcommand{\employees}{employees}
\newcommand{\aStud}{\prg {steve}}
\newcommand{\aPers}{\prg {peter}}
\newcommand{\aStudEmpl}{\prg {mary}}
\newcommand{\anoStudEmpl}{\prg {judy}}
\newcommand{\yetAnoStudEmpl}{\prg {betty}}
\newcommand{\yetYetAnoStudEmpl}{\prg {joe}}

\newcommand{\exprog}{_{\prg{ei}}}



% structuring macros
\newcommand{\EndDefLemma}{\noindent $\bigtriangleup$}



\newcommand{\para}[1]{{\ensuremath{\prg{x}{#1}}}}


\newcommand{\methAndBodyNew}{\ensuremath{
\prg{\type{}~m \lp \type{_1}~\param{}\rp\   \ \lb
~\expr{}~\rb} }}
\newcommand{\methAndBodyNewOne}{\ensuremath{
\prg{\type{}~m \lp \type{_1}~\param{}\rp\   \ \lb
~\expr{'}~\rb} }}
\newcommand{\methAndBodyNewTwo}{\ensuremath{
\prg{\type{}~m \lp \type{_1}~\param{}\rp\   \ \lb
~\expr{''}~\rb} }}
\newcommand{\methAndBodyNewThree}{\ensuremath{
\prg{\type{''}~m \lp \type{_1''}~\param{}\rp\   \ \lb
~\expr{''}~\rb} }}
\newcommand{\methAndBodyNewPrime}{\ensuremath{
\prg{\type{}~m \lp  \type{_1}~\para{} \rp\   \
\lb~\expr{'}~\rb} }}
\newcommand{\methAndBodyNewPrimeAll}{\ensuremath{
\prg{\type{'}~m \lp  \type{_1'}~\para{} \rp\   \
\lb~\expr{'}~\rb} }}
\newcommand{\methAndBodyNewP}{\ensuremath{ % slight diff in Primes from above
\prg{\type{}~m \lp  \type{_1}~\para{} \rp\   \
\lb~\expr{'}~\rb} }}
\newcommand{\methAndBodyFromOneSig}{\ensuremath{
\prg{\type{}~m \lp  \type{_1}~\para{} \rp\ \eff{} \lb~ ... ~\rb}
}}

%-----------------

\newcommand{\Body}[2]{\ensuremath{\mathcal{B}ody(#1,\prg{#2})}}

\newcommand{\T}[1]{{\ensuremath{\type{}{#1}}}}

% find the declaration of an identifier
\newcommand{\LookUp}[2]{\ensuremath{ {#1}({\prg{#2}}) }}
\newcommand{\LookUpEnv}[2]{{ \Env{#1}({\prg{#2}}) }}
\newcommand{\LookUpState}[2]{{ \ensuremath{\sigma{#1}}({\prg{#2}}) }}
\newcommand{\LookUpHeap}[2]{{ \ensuremath{\heap{#1}}({\prg{#2}}) }}
\newcommand{\LookUpBody}[2]{{ {\M{#1}}({\prg{#2}}) }}
%\newcommand{\LookUpClass}[2]
%        { \ensuremath{{\mathcal C}%{\mathcal D}%{\it ef}
%        (}\M{#1},\prg{#2}\ensuremath{)}}
\newcommand{\LookUpField}[3]
        {\ensuremath{{\mathcal F}{\mathcal D}% {\it f}
                (}\M{#1},\prg{#2},\prg{#3}\ensuremath{)}}
\newcommand{\LookUpMethod}[3]
        {\ensuremath{{\mathcal M}{\mathcal D}% {\it f}
         (}\M{#1},\prg{#2},\prg{#3}\ensuremath{)}}
\newcommand{\Undef}{{\ensuremath{\mathcal U\!}{\it df}}}
\newcommand{\ClassOf}[2] { \ensuremath{{\mathcal C}{\mathit{lass}}(#1)_{#2}}}

%-------------------Part Three: Types ...
% Notation for types ( open and close types).


\newcommand{\subclass}{\typCol{\sqsubseteq}}
\newcommand{\sclass}{\subclass}
\newcommand{\widen}{\typCol{\ensuremath {\le}}}

%-------------------- implies, and, or, iff, etc -----------------
\newcommand{\inset}[3]{\prg{#1}\!\in\!\prg{#2},\ldots,\prg{#3}}
% do not change above, it is also used with \forall, \exists
%\newcommand{\IMPLIES}{{\SP \Longrightarrow \SP}}
 \newcommand{\AND}{{\SPsmall {\mbox{and}} \SPsmall}}
\newcommand{\WITH}{{\SPsmall {\mbox{with}} \SPsmall}}

%\newcommand{\IFF}{{\SPsmall {\mbox{iff}} \SPsmall}}
\newcommand{\IFF}{{\SP {\mbox{ iff }} \SP}}

\newcommand{\OR}{{\SPsmall {\mbox{or}} \SPsmall}}
\renewcommand{\implies}{{\ensuremath{\longrightarrow}}}
\newcommand{\upd}{{\mapsto}}

%----------------------- Conformance --------------------
\newcommand{\stateConf}[3]
  {\M{#1},\Env{#2}
  \rtjCol{\ensuremath{\vdash}}\,\ensuremath{\state{#3}}\,\rtjCol{\ensuremath{\DDiamond}}  }
\newcommand{\stackFrameHeapConf}[4]
  {\M{#1},\Env{#2}
  \rtjCol{\ensuremath{\vdash}}\,\ensuremath{\stackFrame{#3},\heap{#4}}\,\rtjCol{\ensuremath{\DDiamond}}  }
\newcommand{\stackFrameHeapConfS}[4]
  {\M{#1},\Env{#2}
  \rtjCol{\ensuremath{\vdash}}\,\ensuremath{ {#3},\heap{#4}}\,\rtjCol{\ensuremath{\DDiamond}}  }

\newcommand{\stackFrameHeapConfNOT}[4]
  {\M{#1},\Env{#2}
  \rtjCol{\ensuremath{\not\vdash}}\,\ensuremath{\stackFrame{#3},\heap{#4}}\,\rtjCol{\ensuremath{\DDiamond}}  }

\newcommand{\stackFrameHeapConfLong}[4]
  {\M{#1}, {#2}
  \rtjCol{\ensuremath{\vdash}}\,\ensuremath{\stackFrame{#3},\heap{#4}}\,\rtjCol{\ensuremath{\DDiamond}}  }
\newcommand{\stackFrameHeapConfLongNOT}[4]
  {\M{#1}, {#2}
  \rtjCol{\ensuremath{\not\vdash}}\,\ensuremath{\stackFrame{#3},\heap{#4}}\,\rtjCol{\ensuremath{\DDiamond}}  }


\newcommand{\stateConfEnv}[3]
  {\M{#1}, {#2}
  \rtjCol{\ensuremath{\vdash}}\,\ensuremath{\state{#3}}\,\rtjCol{\ensuremath{\DDiamond}}  }
  \newcommand{\stateConfLong}[3]
  { {#1}, {#2}
        \rtjCol{\ensuremath{\vdash}}\,\ensuremath{\state{#3}}\,\rtjCol{\ensuremath{\DDiamond} } }
\newcommand{\stateConfLongNot}[3]
  { {#1}, {#2}
        \rtjCol{\ensuremath{\vdash}\!\!\!\not}\ \ensuremath{\state{#3}}\,
         \rtjCol{\ensuremath{\DDiamond} } }
\newcommand{\conf}[4]
  {\M{#1},\ensuremath{\heap{#2}}
        \rtjCol{\ensuremath{\vdash}} \ensuremath{\prg{#3}} \rtjCol{\ensuremath{\lhd}}
        \ensuremath{\prg{#4}}}
\newcommand{\confW}[4]
  {\M{#1},\ensuremath{\heap{#2}}
        \rtjCol{\ensuremath{\vdash}} \ensuremath{\prg{#3}} \rtjCol{<:}
        \ensuremath{\prg{#4}}}
% states conforming to each other:


\newcommand{\confState}[5]
    % Prog, eff, thsiMut, state, state
  {\confStateL{\M{#1}}{\eff{#2}}{\clss{#3}}{\state{#4}}{\state{#5}}}
  \newcommand{\confStateP}[4]
    % Prog, eff, state, state
  {\ensuremath{\M{#1},{\eff{#2}}\vdash{\state{#3}}\lhd{\state{#4}}}}
\newcommand{\confStateL}[5]
  {\ensuremath{#1,{#2},{#3}\vdash{#4}\lhd{#5}}}
\newcommand{\confStateLP}[5]
  {\ensuremath{#1,{#2}\vdash{#3}\lhd{#4}}}




% ------------------------------------------- Type assertions ----------------
\newcommand{\assert}[3]{\prg{#1} \vdash \prg{#2}~:~\prg{#3}}
% \newcommand{\assertT}[5] {\M{#1}, \Env{#2} \   \ensuremath{\vdash} \ \prg{#4}~
% \ensuremath{:~\prg{#5}}}
\newcommand{\EFFSEP}{\typCol{[\!]}}

\newcommand{\assertTAll}[6]
    % program, env, expr, type, mut, eff
{\M{#1}, \Env{#2} \   \typCol{\ensuremath{\vdash}} \ \prg{#3}~
\ensuremath{\typCol{:~} \prg{#4} } }% ~\EFFSEP~ {#5}~\EFFSEP~{#6} }}
\newcommand{\assertTAllS}[6]
% with subsumption
    % program, env, expr, type, mut, eff
{\M{#1}, \Env{#2} \   \typCol{\ensuremath{\vdash_s}} \
\prg{#3}~
\ensuremath{\typCol{:~} \prg{#4} } }% ~\EFFSEP~ {#5}~\EFFSEP~{#6} }}
\newcommand{\assertTAllR}[6]
% with subsumption
    % program, env, expr, type, mut, eff
{\M{#1}, \Env{#2} \   \typCol{\ensuremath{\vdash_r}} \
\prg{#3}~
\ensuremath{\typCol{:~} \prg{#4} } }% ~\EFFSEP~ {#5}~\EFFSEP~{#6} }}
\newcommand{\assertTAllL}[6]
    % program, env, expr, type, mut, eff
{{#1}, {#2} \   \typCol{\ensuremath{\vdash}} \ \prg{#3}~
\ensuremath{\typCol{:~}} \prg{#4} }%~\EFFSEP~ {#5}~\EFFSEP~{#6} }}
\newcommand{\assertTFour}[4]
    % for the explanations, ignore P and \eff
{ {#1} \   \typCol{\ensuremath{\vdash}} \ \prg{#2}~
\ensuremath{\typCol{:~} \prg{#3} ~\EFFSEP~ {#4}  }}
% same as above, but in 2 lines!
% do not remove! it is difficult to program, so
% leave it in, even if temporarily not used!
\newcommand{\assertTAllTwoL}[6]
    % program, env, expr, type, mut, eff
{\M{#1}, \Env{#2} \   \ensuremath{\vdash} \ \prg{#3}~
\ensuremath{:~\\ \SP\SP\SP\SP\SP
   \prg{#4} }}
\newcommand{\assertTAllEnvS}[6]
    %as before, but the environment is given in full
{\M{#1}, {#2} \   \typCol{\ensuremath{\vdash}_s} \ \prg{#3}~
\ensuremath{\typCol{:~}}\prg{#4}  }
\newcommand{\assertTAllEnvR}[6]
    %as before, but the environment is given in full
{\M{#1}, {#2} \   \typCol{\ensuremath{\vdash}_r} \ \prg{#3}~
\ensuremath{\typCol{:~}}\prg{#4}  }
\newcommand{\assertTAllEnv}[6]
    %as before, but the environment is given in full
{\M{#1}, {#2} \   \typCol{\ensuremath{\vdash}} \ \prg{#3}~
\ensuremath{\typCol{:~}}\prg{#4}  }




 % Effects


% receiver mutations

\newcommand{\mutOr}[1] % takes as parameter the program
        {\ensuremath{\sqcup\!{_{{#1}}}}} % binary on muts

\newcommand{\assertc}[3]{\ensuremath{\M{#1}
 \typCol{\vdash}\, \prg{#2}\, \subclass\, \prg{#3}}}

 \newcommand{\assertw}[3]{\ensuremath{\M{#1} \typCol{\vdash}
\prg{#2}\widen\prg{#3}}}
\newcommand{\assertion}[3]{\ensuremath{#1 \typCol{\vdash} \prg{#2}~#3}}
\newcommand{\notStateConfEnv}[3]
  {\M{#1},\prg{#2}\,
  \rtjCol{\ensuremath{\not\vdash}}\,\ensuremath{\state{#3}}\,\rtjCol{\ensuremath{\DDiamond}}  }
 \newcommand{\notConf}[4]
  {\M{#1},\ensuremath{\heap{#2}}
        \rtjCol{\ensuremath{\not\vdash}} \ensuremath{\prg{#3}} \rtjCol{\ensuremath{\lhd}}
        \ensuremath{\prg{#4}}}







%Macros for inference rules
\newcommand{\inferencerule}[2]{
\begin{array}{l} #1 \\ \hline #2 \end{array}
}

\newcommand{\inferenceruleN}[3]
{
\begin{array}{l}
% \SP\SP\SP\SP\SP\SP\SP\SP
% \SP\SP\SP\SP\SP\SP\SP\SP
\SP\SP\SP\SP\SP\SP\SP\SP
\SP\SP\SP\SP\SP\SP  {\sf #1}
\\ #2  \\ \hline   #3
  \end{array}
}

\newcommand{\inferenceruleNN}[3]
{
\begin{array}{l}
\SP\SP\SP\SP\SP\SP\SP\SP
\SP\SP\SP\SP\SP\SP\SP\SP
\SP\SP\SP\SP\SP\SP\SP\SP
\SP\SP\SP\SP\SP\SP\SP\SP

   {\sf #1}
\\ #2  \\ \hline   #3
  \end{array}
}

%===========================================================================
%  Definition-Lemma-Theorem-Proof
%
% Adaptation of LaTeX's theorem environment; can be used as a command
% (eg just \Lemma not \begin{Lemma}) and no italicisation; also works
% with ptmac; result numbering is uniform within subsections and can be
% suppressed.
%
\newif\ifNumberResults\NumberResultstrue
\def\@@opargbegintheorem#1#2#3{\@@@@begintheorem{\bf\@@thmname{#1}{#2}(#3)}}
\def\@@begintheorem#1#2{\@@@@begintheorem{\bf\@@thmname{#1}{#2}}}
\def\@@@@begintheorem#1{\par\removelastskip\smallskip\noindent{#1}}
\def\@@thmname#1#2{#1\ \ifNumberResults#2\ \fi}

% similarly \Proof or \begin{Proof}...\end{Proof}
% prefer proofs with statements if possible - hence \penalty700
%\let\qedsymbol\S% make it \square or \blacksquare if you like for kb
\let\qedsymbol \Box
\def\qed{\hfill{$\qedsymbol$}}
\def\Proof{\par\removelastskip\smallskip\penalty700\noindent{\bf Proof}\enskip}
\def\endProof{\qed\penalty-700 \smallskip}
\let\endproof\endProof

%   The actual words

%\newtheorem{theo}{Theorem} this is necessay if we want eg \newtheorem{definition}[theo]{Definition} to work 
\newtheorem{definition}{Definition} %  \newtheorem{definition}[theo]{Definition}
\newtheorem{example}{Example} %  \newtheorem{example}[theo]{Example}
%\newtheorem{lemma}{Lemma} % \newtheorem{mylemma}[theo]{Lemma}
%\renewtheorem{mylemma}{Lemma} % \newtheorem{mylemma}[theo]{Lemma}
\newtheorem{conjecture}{Conjecture} %\newtheorem{conjecture}[theo]{Conjecture}
\newtheorem{theorem}{Theorem} %\newtheorem{theo}{Theorem}
\newtheorem{note}{Note} % \newtheorem{note}[theo]{Note}
\newtheorem{observation}{Observation} %  \newtheorem{observation}[theo]{Observation}


%--------------------------------- the ones that Susan introduced
\newcommand{\z}{{\prg z}}

\newcommand{\Fields}[3]{\ensuremath{{\mathcal F}(}\\Mg{#1},\prg{#2},
\prg{#3}\ensuremath{)} }
\newcommand{\FieldIds}[2]{\ensuremath{{\mathcal F}{\it {s}}(\M{#1},\prg{#2})}}
\newcommand{\Meths}[3]{\ensuremath{{\mathcal M}(}\M{#1},\prg{#2},
\prg{#3}\ensuremath{)} }







 


\newcommand{\rewriteLong}[1]
{\rtCol{\ensuremath{\ \leadsto\!\!\!\!\!\!_{_{#1}\,\ \ }}}}
\newcommand{\rewrite}[1]
{\rtCol{\ensuremath{\ \leadsto\!\!\!\!\!\!_{_{\M{#1}}\,\ \ }}}}
\newcommand{\rewriteP}
    {{\ensuremath{\ \ \leadsto\!\!_{_{\prg{P}}}\,\,\,}}}
 % {\rtCol{\ensuremath{\ \ \leadsto \!\!\!\!_{{\M{}}\,\ \ }}}}
% \newcommand{\Conf}[2]% configurations: < #1, #2 >
% inside the mathmode
% { \langle \prg{#1}, {#2} \rangle }
% {  \prg{#1}\ensuremath{,}{#2}  }
%\newcommand{\greenComma}{\rtCol{\ensuremath{,}}}
%\newcommand{\ConfL}[3]
%{  \prg{#1}\greenComma\stackFrame{#2}\greenComma\heap{#3}  }
%\newcommand{\ConfLLong}[3]
%{  \prg{#1}\greenComma {#2}\greenComma\heap{#3}  }
%\newcommand{\ConfR}[2]
%{  \prg{#1}\greenComma\heap{#2}  }
%

% ------------------------------------- Well formed, unique acyclic ---------------
 
%{\ensuremath{\M{#1} \vdash {\prg{#2}}\s \DDiamond_{\it {eff}}  }}

  \newcommand{\nullPEC}{\lit {nullPntrExc}}



%\newcommand{\Obj}{\prg{Object}}
%\newcommand{\void}{\kw{void}}
%\newcommand{\newK}{{\kw {new}}~} % {{\kw {new}}}% no ~ around it
%\newcommand{\newKW}{~{\kw {new}}~}
%




\newcommand{\back}{{$\!\!\!\!\!\!\!$}}


\newcommand{\WideFig}[3]
{
\begin{figure*}[t]
\begin{center}
\noindent
\fbox{
\begin{minipage}{4.7 in}
{#1} % the contents
\end{minipage}
}
\caption{#2}
\label{#3}
\end{center}
\end{figure*}
}


\newcommand{\WideFigWhere}[4] % you can specify where it should appear!
{
\begin{figure*}[{#4}]
\begin{center}
\noindent
\fbox{
\begin{minipage}{5. in}
{#1} % the contents
\end{minipage}
}
\caption{#2}
\label{#3}
\end{center}
\end{figure*}
}

\newcommand{\BigWideFigWhere}[4] % you can specify where it should appear!
{
\begin{figure*}[{#4}]
\begin{center}
\noindent
{\normalsize
\hrule
\begin{minipage}{5. in}
{#1} % the contents
\end{minipage}
\hrule
}
\caption{#2}
\label{#3}
\end{center}
\end{figure*}
}

\newcommand{\NotTooWideFigWhere}[4] % you can specify where it should appear!
{
\begin{figure*}[{#4}]
\begin{center}
\noindent
\fbox{
\begin{minipage}{4.3 in}
{#1} % the contents
\end{minipage}
}
\caption{#2}
\label{#3}
\end{center}
\end{figure*}
}


\newcommand{\opsemExprFig}
{\BigWideFigWhere {\opsemExpr} {Execution of expressions\MD}
{opsemTrad} {htbp} }



\newcommand{\mlc}{ }%{\heartsuit}
%\newcommand{\mcl}{ }%{\heartsuit}
\newcommand{\mc}{ }%{\heartsuit}

\newcommand{\BigNotTooWideFigWhere}[4] % you can specify where it should appear!
{
\begin{figure*}[{#4}]
\begin{center}
\noindent
{\normalsize
\hrule
\begin{minipage}{4.3 in}
{#1} % the contents
\end{minipage}
\hrule
}
\caption{#2}
\label{#3}
\end{center}
\end{figure*}
}

 

%]})
%}


\usepackage{latexsym}
\usepackage{listings}
\definecolor{dkgreen}{rgb}{0,0.6,0}
\definecolor{gray}{rgb}{0.5,0.5,0.5}
\definecolor{mauve}{rgb}{0.58,0,0.82}

\lstset{ %
  language=Java,                % the language of the code
  mathescape=true,
  basicstyle=\footnotesize\tt,           % the size of the fonts that are used for the code
%  numbers=left,                   % where to put the line-numbers
%  numberstyle=\tiny\color{dkgreen},  % the style that is used for the line-numbers
%  stepnumber=1,                   % the step between two line-numbers. If it's 1, each line
                                  % will be numbered
%  numbersep=5pt,                  % how far the line-numbers are from the code
  backgroundcolor=\color{white},      % choose the background color. You must add \usepackage{color}
  showspaces=false,               % show spaces adding particular underscores
  showstringspaces=false,         % underline spaces within strings
  showtabs=false,                 % show tabs within strings adding particular underscores
%  frame=single,                   % adds a frame around the code
  rulecolor=\color{black},        % if not set, the frame-color may be changed on line-breaks within not-black text (e.g. commens (green here))
  tabsize=2,                      % sets default tabsize to 2 spaces
  captionpos=b,                   % sets the caption-position to bottom
  breaklines=true,                % sets automatic line breaking
  breakatwhitespace=false,        % sets if automatic breaks should only happen at whitespace
  title=\lstname,                   % show the filename of files included with \lstinputlisting;
                                  % also try caption instead of title
  keywordstyle=\color{blue},          % keyword style
  commentstyle=\color{gray},       % comment style
  stringstyle=\color{mauve},         % string literal style
 % escapeinside={\%*}{*)},            % if you want to add LaTeX within your code
  morekeywords={PRE,POST,result,assume,method,mthd,function,fresh,assert,private,then,elseif,public,final,this,throw,new,||,to,def,any,fun,fld,abstract,policy,specification,ghost,field,func}        }  
         % if you want to add more keywords to the set

\newcommand{\kjx}[1]{{\color{orange}{KJX: #1}}}
\newcommand{\scd}[1]{{\color{dkgreen}{SD: #1}}}
\newcommand{\sdN}[1]{{\color{dkgreen}{#1}}}
%\newcommand{\jm}[1]{{\color{magenta}{JM: #1}}}
\newcommand{\sdcomment}[1]{{\ensuremath{\blacksquare}}\footnote{\color{dkgreen}{SD: #1}}}
\newcommand{\secomment}[1]{{\ensuremath{\blacksquare}}\footnote{\se{#1}}}
\newcommand{\jncomment}[1]{{\ensuremath{\blacksquare}}\footnote{\kjx{#1}}}

 \newcommand{\sd}[1]{#1} % {{\color{dkgreen}{#1}}}% {#1} %
\newcommand{\tobyM}[1]{#1} %[1]{{\color{purple}{Toby: #1}}}
\newcommand{\se}[1]{#1} %{{\color{blue}{#1}}}


\newcommand{\ponders}[3]{\marginpar{\tiny\itshape\raggedright\textcolor{#2}{\textbf{#1:} #3}}\ignorespaces}
\marginparwidth=1.6cm \marginparsep=0cm
\newcommand{\TODO}[1]{{\color{red}#1}}
\newcommand{\sophia}[1]{#1} % {{\color{red}#1}}
\newcommand{\susan}[1]{#1} % [1]{{\color{blue}#1}} 
\newcommand{\toby}[1]{} % {\ponders{Toby}{purple}{#1}}
%\newcommand{\susan}[1]{} %{\ponders{Susan}{blue}{#1}}
\newcommand{\james}[1]{} % {\ponders{James}{orange}{#1}}
\newcommand{\jm}[2][]{\ponders{Julian}{magenta}{#1} \textcolor{magenta}{#2}\xspace}
\newcommand{\mrr}[2][]{\ponders{Matthew Ross}{teal}{#2} \textcolor{teal}{#1}\xspace}
\newcommand{\mrrz}[1]{\textcolor{teal}{{#1}}\xspace}
\newcommand{\mrrx}[2][]{\ponders{Matthew Ross}{teal}{{#1}} \textcolor{teal}{{#2}}\xspace}

\begin{document}

\acmJournal{TOSEM}

\title{Holistic Specifications for Robust Programs}
% use title[shorttitle]{...} if neccessary --mrr

\author{Sophia Drossopoulou}
\orcid{0000-0002-1993-1142}
\affiliation{
	\institution{Imperial College London}
	\department{Department of Computing}
	\streetaddress{180 Queen's Gate}
	\city{London}
	\country{United Kingdom}}
\email{scd@imperial.ac.uk}

\author{James Noble}
\orcid{0000-0001-9036-5692}
\affiliation{
	\institution{Victoria University of Wellington}
	\department{School of Engineering and Computer Science}
	\streetaddress{Cotton Building, Gate 7, Kelburn Parade}
	\city{Wellington}
	\country{New Zealand}}
\email{kjx@ecs.vuw.ac.nz}

\author{Julian Mackay}
\orcid{0000-0003-3098-3901}
\affiliation{
	\institution{Victoria University of Wellington}
	\department{School of Engineering and Computer Science}
	\streetaddress{Cotton Building, Gate 7, Kelburn Parade}
	\city{Wellington}
	\country{New Zealand}}
\email{julian.mackay@ecs.vuw.ac.nz}

\author{Matthew Ross Rachar}
\affiliation{
	\institution{Imperial College London}
	\department{Department of Computing}
	\streetaddress{180 Queen's Gate}
	\city{London}
	\country{United Kingdom}}

\author{Susan Eisenbach}
\orcid{0000-0001-9072-6689}
\affiliation{
	\institution{Imperial College London}
	\department{Department of Computing}
	\streetaddress{180 Queen's Gate}
	\city{London}
	\country{United Kingdom}}
\email{susan@imperial.ac.uk}

\begin{abstract}
Functional specifications describe what
program components \emph{can} do: the \emph{sufficient} conditions to
invoke components' operations.
They allow us to reason about the  use  of  
components in  a \emph{closed world} setting, where
components interact with known client code, and 
where the client code must establish the appropriate pre-conditions 
before calling into a component.

%don't start sentences with However!
Sufficient conditions are not enough to reason about 
the use of components in an \emph{open world} setting, where
components interact with external code, possibly of unknown
provenance, 
and where components may evolve over time. 
%
In this   open world  setting,
%%
%% we must also consider the
ensuring that your component is robust even when executing
 with buggy or malicious external code is critical.
 \emph{Holistic specifications}
--- as their name implies --- 
are concerned  with the \emph{overall} behaviour of a component,  in all possible 
interleavings of calls to the component's operations with those of the external code.
Thus, holistic specifications are concerned with \emph{sufficient} conditions, \ie
what is enough to \emph{cause} some effect, as well as with
\emph{necessary} conditions, \ie
what are the conditions without which an effect will \emph{not} happen. 
%

% New para. The previous was the problem, the current is what we do
In this paper we
%argue for the need for such holistic specifications,
 propose the 
 \Chainmail specification language for writing \emph{holistic} specifications that
 focus on necessary conditions (as well as sufficient conditions). We
 give a formal semantics for \Chainmail, and discuss several examples.
 The core of \Chainmail has been
  mechanised in the Coq proof assistant.
%% By adopting holistic specification techniques like \Chainmail,
%% programmers can explicitly define what their components should not do,
%% making it easier to write
%% robust and reliable programs.
\end{abstract}

\maketitle

\renewcommand{\shortauthors}{S. Drossopoulou, J. Noble, et al.}

%% TODO: \\
%% Add equivalences of assertions, eg Wll(A) == Will(Will(A))\\
%% WIll(A) = A or Next(A or Will(A))

\section{Introduction}
%Software guards our secrets, our money, our intellectual property,
our reputation \cite{covern}.  We entrust personal and
corporate information to software which works in an \emph{open} world, 
where  it interacts with 
third party software of unknown provenance, possibly buggy and potentially malicious.

This means we need our software to be \emph{robust}:
to behave correctly even if  used 
by erroneous or malicious third parties.
We expect that our bank will only make payments 
from our account if instructed by us, or by somebody we have authorised, 
that space on a web given to an advertiser will not be used
to obtain access to our bank details \cite{cwe}, or that a
concert hall will not book the same seat more than once.


%% The importance of robustness has led to the design of many programming
%% language mechanisms to help developers write robust programs:
%% constant fields, private methods, ownership \cite{ownalias}
%% as well as the object capability paradigm \cite{MillerPhD},
%% and its adoption in  web systems
%% \cite{CapJavaHayesAPLAS17,CapNetSocc17Eide,DOCaT14}, and programming languages such as Newspeak
%% \cite{newspeak17}, Dart \cite{dart15},
%% Grace \cite{grace,graceClasses}, and Wyvern \cite{wyverncapabilities}.

%% While such programming language mechanisms
While language mechanisms such as constants, invariants, 
object capabilities \cite{MillerPhD}, and 
ownership \cite{ownalias} 
make it \textit{possible} to write robust
programs, they cannot \textit{ensure} that programs are robust.
Ensuring robustness is difficult because it means 
different things for different systems: perhaps
that critical operations should only be invoked with the requisite authority;
perhaps that sensitive personal information should not be leaked; 
or perhaps that a resource belonging to one user should not be consumed by another.g
%
To ensure robustness, we need ways to specify what robustness means for a 
particular program, and ways to demonstrate that the particular program 
adheres to its specific robustness requirements.

 \begin{figure}[htb]
 \begin{tabular}{lll}
\begin{minipage}{0.39\textwidth}
\begin{lstlisting}
class Wallet{
  fld balance 
  fld secret 
  mthd pay(who,amt,scr){
    if ( (secret==scr)
       &amnt>0 ) then 
      balance-=amnt
      who.balance+=amt   }
 }
\end{lstlisting}
\end{minipage}
  & 
\begin{minipage}{0.29\textwidth}
\begin{lstlisting}
class Wallet{
  fld balance  
  fld secret  
  mthd pay(...){
    $\mathit{... as\, version\,1 ...}$ }
  mthd set(secr){
      secret=secr }
}
\end{lstlisting}
\end{minipage} &  
\begin{minipage}{0.32\textwidth}
\begin{lstlisting}
class Wallet{
  fld balance  
  fld secret  
  fld owner  
  mthd pay(...){
    $\mathit{... as\, version\,1 ...}$ }
  mthd send( ){
    owner.take(secret) }  
}
\end{lstlisting}
  \end{minipage}
 \end{tabular}
  \vspace*{-0.95cm}
  \caption{Three Versions of the class \prg{Wallet}}
 \label{fig:ExampleWallet}
 \end{figure}

Consider the code snippets from Figure \ref{fig:ExampleWallet}. Oobjects of
 class \prg{Wallet}  hold a \prg{balance} and a \prg{secret}, and  only the holder of the secret can use the wallet to
 make payments
 -- for the sake of simplicity, we allow   balances to grow negative.
 \sd{We assume that fields are private in the sense of C++, ie only methods of that class may read or write these fields.}
 We show the code in three versions; each of these versions has the same method \prg{pay}, and the two last versions
 have an additional method \prg{sendSecret}.
  We use a Java-like syntax, and assume an untyped language (as we are in the open world setting).
 Thus, the classical Hoare triple describing the behaviour of \prg{pay} would be:
 
% \begin{figure}[htbp] 
\begin{lstlisting}
   method pay(who,amt,scr)
   PRE:  this,who:Wallet $\wedge$  this$\neq$who $\wedge$ amt:$\mathbb{N}$  $\wedge$  scr=secret   
   POST: this.balance=this.balance$\pre$-amt $\wedge$ who.balance=who.balance$\pre$+amt 
 \end{lstlisting}
%^\end{figure} 
\vspace{-.2in}
The above specification shows that knowledge of the \prg{secret} is a \emph{sufficient} condition to make payments. 
But it does not show that it is a \emph{necessary} condition. To make the specification  
  more ``robust'' we can also describe the behaviour of    \prg{pay} when the pre-condition is not satisfied:

%\begin{figure}[htbp] 
\begin{lstlisting}
   method pay(who,amt,scr)
   PRE:  this:Wallet $\wedge$  $\neg$ (this$\neq$who $\wedge$ amt:$\mathbb{N}$  $\wedge$  this.scr=secret  )
   POST: $\forall$w: Wallet.$ \,$w.balance=w.balance$\pre$-amt 
 \end{lstlisting}
%\end{figure} 
\vspace{-.2in}
The specification from above mandates that the method \prg{pay} cannot make a payment unless the secret is
provided. But it cannot preclude that \prg{Wallet} -- or some other class, for that matter -- contains more methods 
which might make it possible to affect a reduction in the balance   without knowledge of the
secret. To preclude this, we introduce \emph{holistic specifications}, and require that:

\vspace{.02in}
(Spec1)\ \  $\triangleq$\ \ $\forall \prg{w},\prg{m}.$\\
$\strut \hspace{.6in} [\ \ \prg{w}:\prg{Wallet} \wedge \prg{w.balance}=m\ \wedge \Future{\prg{w.balance}<m} \ \    
    \longrightarrow \hfill $ \\
  $ \strut \hspace{6.3cm} 
  \exists \prg{o}. [\  \External{\prg{o}} \wedge  \CanAccess{\prg{o}}{\prg{w.secret}}\ ]  \  \ ] \hfill $
\vspace{.02in}

(Spec1) mandates that for any wallet \prg{w} defined in the current configuration, if some time in the future the balance of
\prg{w} were to decrease, then at least one external object (\ie an object whose class is not \prg{Wallet}) in the current configuration
has direct access to the secret. This external object need not have caused the change in $\prg{w.balance}$ but it would  have (transitively) passed access to the secret which ultimately did cause the change in the balance.

The class \prg{Wallet} from Figure \ref{fig:Example}.version 1, satisfies (Spec1), but \prg{Wallet} from Figure \ref{fig:Example}.version 2, does not.
It is possible to overwrite the \prg{secret} of the \prg{Wallet} and then to effect a \prg{pay}ment. Neither does \prg{Wallet} from Figure \ref{fig:Example}.version 3,  sarisfy (Spec1), since it is possible for the \prg{owner} not to know the \prg{secret} and the secret to be 
communicated to them. Insteed, the class saisfies (Spec2) from below
 

\vspace{.02in}
(Spec2)\ \  $\triangleq$\ \ $\forall \prg{w},\prg{m}.$\\
$\strut \hspace{.6in} [ \ \ \prg{w}:\prg{Wallet} \wedge \prg{w.balance}=m\ \wedge \Future{\prg{w.balance}<m} \ \    
    \longrightarrow \ \    \hfill$ \\
  $\strut   \hspace{3.4cm} 
  \exists \prg{o}. [\  \External{\prg{o}}\,  \wedge\, (\,  \CanAccess{\prg{o}}{\prg{w.secret}}\ \vee \prg{o}=\prg{w}.\prg{owner}\, )   \ \ ] \hfill $
\vspace{.02in}

 
 (Spec2) mandates that for any wallet \prg{w} defined in the current configuration, if some time in the future the balance of
\prg{w} were to decrease, then at least one external object   in the current configuration
has direct access to the secret, or is the \prg{owner} of the \prg{Wallet}. 
The class \prg{Wallet} %from Figure \ref{fig:ExampleWallet}
from .version 1 and version 3 satisfy (Spec2), but \prg{Wallet} %from Figure \ref{fig:ExampleWallet}.
version 2, does not.

In this paper we propose \Chainmail, a specification language to
express holistic specifications.
The design of \Chainmail was guided by the study of a sequence of
examples from the object-capability literature and the smart contracts world: the
membrane \cite{membranesJavascript}, the DOM \cite{dd,ddd}, the Mint/Purse \cite{MillerPhD}, the Escrow \cite{proxiesECOOP2013}, the DAO \cite{Dao,DaoBug} and
ERC20 \cite{ERC20}.  As we worked through the
examples, we found a small set of language constructs that let us
write holistic specifications across a range of different contexts.
%
In particular, \Chainmail extends 
traditional program specification languages \cite{Leavens-etal07,Meyer92} with features which talk about:
%
\begin{description}
\item[Permission: ] 
%\ \ \textbullet \ \emph{Permission}, \ie
Which objects may have access to which other objects; 
this is central since access to an object usually also grants access to the functions it provides.
%
\item[Control: ]
%\ \ \textbullet  \ \emph{Control}, \ie
Which objects called functions on other objects; this
 is useful in identifying the causes of certain effects - eg 
funds can only be reduced if the owner called a payment function.
%
%$\bullet$ \ \\emph{Authority}, \ie which objects' state or properties may change; this is useful in describing effects, such as reduction of funds.
%
\item[Time: ]
%\ \ \textbullet \ \emph{Time}:\ \ie
What holds some time in  the past, the future, and what changes with time,
\item[Space: ]
%\ \ \textbullet \ \emph{Space}:\ \ie
Which parts of the heap are considered when establishing some property, or when 
performing program execution; a concept
related to, but different from, memory footprints and separation logics,
\item[Viewpoint: ]
%\ \ \textbullet \ \emph{Viewpoints}:\ \ie
%a distinction between the objects internal to our component, and those external to it;
Which objects and which configurations are internal to our component, and which  are
external to it;
a concept related to the open world setting.
\end{description}

%% Holistic assertions usually employ several of these features. They often have the form  of a guarantee
%% that only if some property holds now will a certain effect occur in the future, or that
%% certain effects can only be caused if another property held earlier.
%% For example, if within a certain heap (space) some change is possible in the future (time), then this particular heap 
%% (space again) contains 
%% at least one object which has access (permission) to a specific other, privileged object.
%% %\james{moved around --- not sure we need this para}
%% %\susan{I think we don't so there is a paragraph I have commented out.}
%% \forget{Often, holistic assertions typically have the form of a guarantee
%% that if some property ever holds in the future, then some other property holds now.
%% For example, if within a certain heap some change is possible in the future, then this particular heap contains 
%% at least one object which has access to a specific other, privileged object.}
%% A module satisfies a holistic assertion if  
%%  the assertion is satisfied  in all runtime configurations reachable through execution of the combined code of our module and any other module.
%%   This reflects the open-world view.

While many individual features of \Chainmail can be found in other work, 
their power and novelty for specifying open systems lies in their careful combination.
The contributions of this paper are:
\begin{itemize}
\item the design of the holistic specification language \Chainmail,
\item the semantics of \Chainmail, and a Coq mechanisation of its core,
\item the application of \Chainmail  to a sequence of examples.
%\item a further validation of \Chainmail through informal proofs of adherence of code to some of these specifications.
\end{itemize}  
  
  
The rest of the paper is organised as follows:
Section
~\ref{sect:motivate:Bank} 
\sd{gives an example from the literature} which we will use 
to elucidate key points of \Chainmail.
%motivates our work via an example, and then section
~\ref{sect:chainmail} presents the \Chainmail\ specification
language.  Section~\ref{sect:formal} introduces the formal model
underlying \Chainmail, and then section~\ref{sect:assertions} defines
the 
semantics of \Chainmail's assertions.
% SD the below is NOT ture
%full details are relegated toappendices.   
Section~\ref{sect:example} shows how key points of 
exemplar problems can be specified in \Chainmail,
section~\ref{sect:discussion}
discusses our design, \ref{sect:related} considers related
work, and section~\ref{sect:conclusion} concludes.
We relegate various details to appendices.

Software guards our secrets, our money, our intellectual property,
our reputation \cite{covern}.  We entrust personal and
corporate information to software which works in an \emph{open} world, 
where  it interacts with 
third party software of unknown provenance, possibly buggy and potentially malicious.

This means we need our software to be \emph{robust}:
to behave correctly even if  used 
by erroneous or malicious third parties.
We expect that our bank will only make payments 
from our account if instructed by us, or by somebody we have authorised; 
that space on a website given to an advertiser will not be used
to obtain access to our bank details \cite{cwe}; or that a
concert hall will not book the same seat more than once.


%% The importance of robustness has led to the design of many programming
%% language mechanisms to help developers write robust programs:
%% constant fields, private methods, ownership \cite{ownalias}
%% as well as the object capability paradigm \cite{MillerPhD},
%% and its adoption in  web systems
%% \cite{CapJavaHayesAPLAS17,CapNetSocc17Eide,DOCaT14}, and programming languages such as Newspeak
%% \cite{newspeak17}, Dart \cite{dart15},
%% Grace \cite{grace,graceClasses}, and Wyvern \cite{wyverncapabilities}.

%% While such programming language mechanisms
While language mechanisms such as private members,  constants, invariants, 
object capabilities \cite{MillerPhD}, and 
ownership \cite{ownalias} are
\textit{indispensable} to writing robust
programs, they cannot \textit{ensure} that programs are robust.
Ensuring robustness is difficult because it means 
different things for different systems: perhaps
that critical operations should only be invoked with the requisite authority;
perhaps that sensitive personal information should not be leaked; 
or perhaps that a resource belonging to one user should not be consumed by another. 
%
To ensure robustness, we need ways to specify what robustness means for a 
particular program, and ways to demonstrate that the particular program 
adheres to its specific robustness requirements.

\jm[removed myBank fields]{}
  \vspace*{-0.2cm}
 \begin{figure}[htb]
 \begin{tabular}{lll} % {lll}
\begin{minipage}{0.45\textwidth}
\begin{lstlisting}
class Account {
    field balance
    field myBank
    
    method deposit(src,amt){
       if (amt>=0 && src.MyBank = this.myBank && src.balance>=amt) then {
           this.balance = this.balance + amt
           src.balance = src.balance - amt
   }   }
}
\end{lstlisting}
\end{minipage}
  &\ \ \  \ \ \ \ \  &
\begin{minipage}{0.45\textwidth}
\begin{lstlisting}
class Account {
    field balance
    field myBank
    
    method deposit(src, amt){
       $\mathit{... as\, version\,1 ...}$          
    }  }     
    method freeMoney(){
       this.balance = this.balance + 1000000
    }
}
\end{lstlisting}
\end{minipage} 
 \end{tabular}
  \vspace*{-0.95cm}
  \caption{Two versions of the class \prg{Account}}
  \Description{Two classes beside each other, the one on the left containing a simple account, with a balance and myBank field, and a deposit method. The right class is the same, except an additional freeMoney method which adds 1 million to the account's balance.}
 \label{fig:ExampleBank}
 \end{figure}
 
% \begin{figure}[htb]
% \begin{tabular}{lll} % {lll}
%\begin{minipage}{0.45\textwidth}
%\begin{lstlisting}
%class Wallet{
%  fld balance 
%  fld secret 
%  mthd pay(who,amt,scr){
%    if ( (secret==scr)
%       &amnt>0 ) then 
%      balance-=amnt
%      who.balance+=amt   }
% }
%\end{lstlisting}
%\end{minipage}
%  & 
%\begin{minipage}{0.45\textwidth}
%\begin{lstlisting}
%class Wallet{
%  fld balance  
%  fld secret  
%  mthd pay(...){
%    $\mathit{... as\, version\,1 ...}$ }
%  mthd set(secr){
%      secret=secr }
%}
%\end{lstlisting}
%\end{minipage} 
%%&  
%%\begin{minipage}{0.32\textwidth}
%%\begin{lstlisting}
%%class Wallet{
%%  fld balance  
%%  fld secret  
%%  fld owner  
%%  mthd pay(...){
%%    $\mathit{... as\, version\,1 ...}$ }
%%  mthd send( ){
%%    owner.take(secret) }  
%%}
%%\end{lstlisting}
%%  \end{minipage}
% \end{tabular}
%  \vspace*{-0.95cm}
%  \caption{Three Versions of the class \prg{Wallet}}
% \label{fig:ExampleWallet}
% \end{figure}

Consider the code snippets from fig.~\ref{fig:ExampleBank}. Objects of
 class \prg{Account} hold \jm{money in the field \prg{balance}}, and only the holders of the references to accounts can move money between them, given the accounts are at the same bank.

 We show the code in two versions; both have the same method \prg{deposit}, and the \jm{right} version 
 has an additional method \prg{freeMoney}.
  We use a Java-like syntax, and assume an untyped language (as we are in the open world setting).
 \sd{We assume that fields are private in the sense of C++ or Java, \ie only methods of that class may read or write these fields,
 and that addresses are unforgeable, so there is no way to guess an account.}
 Thus, the classical Hoare triple describing the behaviour of \prg{deposit} would be:
 
 \vspace{0.1in}
 \jm[removed myBank field]{}
(ClassicalSpec)\ \  $\triangleq$\\ 
\vspace{-0.22in}
\begin{lstlisting}
   method deposit(src, amt)
   PRE:  this,src:Account $\wedge$  this$\neq$src $\wedge$  this.myBank = src.myBank
   			 amt:$\mathbb{N}$  $\wedge$   src.balance$\geq$amt
   POST: src.balance=src.balance$\pre$-amt $\wedge$ this.balance=this.balance$\pre$+amt
\end{lstlisting}
\vspace{-0.1in}
%SD changed function to method, to fir what comes later.

Our (ClassicSpec) expresses that knowledge of the \prg{src} account, and \jm{that} the accounts \jm{share} the same bank, is \emph{sufficient}, %condition 
and that  % from the safe.
%make payments. 
%But it does not show that it is a \emph{necessary} condition. 
%To make the specification  
%  more ``robust'' we  also describe the behaviour of    \prg{take} when the pre-condition is not satisfied:
%
%%\begin{figure}[htbp] 
%\vspace{-.1in}
%\begin{lstlisting}
%   method take(scr)
%   PRE:  this:Safe $\wedge$  
%   POST: $\forall$s: Safe.$ \,$s.treasure=s.treasure$\pre$ 
% \end{lstlisting}
%%\end{figure} 
%\vspace{-.1in}
%The specification from above mandates 
 \prg{deposit} must maintain the same total balance for the two accounts.\footnote{We take that there must be no effects if (ClassicSpec)'s precoditions aren't satisfied. This is simple to state in a holistic specification, but has been omitted for brevity.}
%
But it cannot preclude that \prg{Account} -- or some other class, for that matter -- contains more methods 
which might make it possible to create "free" money, as shown with \prg{freeMoney} on the right in fig.~\ref{fig:ExampleBank}, a situation any sensible bank implementation should avoid. In fact it would be impossible to express such a restriction using a classical specification. 
\jm{Thus, in addition to classical specifications}, we propose \emph{holistic specifications}, and require that:
 
  \vspace{.01in}
(HolisticSpec)\ \  $\triangleq$\\ 
$\forall \prg{a}.[\ \ \prg{a}:\prg{Account} \wedge \Changes{\prg{a.balance}}  \ \    
    \longrightarrow \ \    \hfill$ \\
  $\strut \hspace{2.3cm} 
% TODO explain:
% we no longer need Past here, as we are ion visible states 
  \exists \prg{o}. [\    \Calls{\prg{o}}{\prg{deposit}}{\prg{a}}{\_,\_} \vee\  \Calls{\prg{o}}{\prg{deposit}}{\_}{\prg{a},\_}\  \ ] \ \ \ \ ] \hfill $
\vspace{.05in}



Our (HolisticSpec) mandates that for any change in the value of an account, the deposit method must have been called on it, or it must have been the source to another accounts deposit. This prevents the existence of the \prg{freeMoney} method above, where \prg{a.balance} changes, but the deposit method isn't called. Further holistic propositions are discussed in section~\ref{sect:motivate:Bank} to provide other protections, such as against the leaking of accounts.
%Neither does \prg{Safe} from Figure \ref{fig:Example}.version 3,  sarisfy (Spec1), since it is possible for the \prg{owner} not to know the \prg{secret} and the secret to be 
%communicated to them. Insteed, the class saisfies (Spec2) from below
% 
%
\vspace{.02in}
%(Spec2)\ \  $\triangleq$\ \ $\forall \prg{w},\prg{m}.$\\
%$\strut \hspace{.6in} [ \ \ \prg{w}:\prg{Safe} \wedge \prg{w.balance}=m\ \wedge \Future{\prg{w.balance}<m} \ \    
%    \longrightarrow \ \    \hfill$ \\
%  $\strut   \hspace{3.4cm} 
%  \exists \prg{o}. [\  \External{\prg{o}}\,  \wedge\, (\,  \CanAccess{\prg{o}}{\prg{w.secret}}\ \vee \prg{o}=\prg{w}.\prg{owner}\, )   \ \ ] \hfill $
%\vspace{.02in}
%
% 
% (Spec2) mandates that for any Safe \prg{w} defined in the current configuration, if some time in the future the balance of
%\prg{w} were to decrease, then at least one external object   in the current configuration
%has direct access to the secret, or is the \prg{owner} of the \prg{Safe}. 
%The class \prg{Safe} %from Figure \ref{fig:ExampleSafe}
%from .version 1 and version 3 satisfy (Spec2), but \prg{Safe} %from Figure \ref{fig:ExampleSafe}.
%version 2, does not.

In this paper we discuss \Chainmail, a specification language to
express holistic specifications.
The design of \Chainmail was guided by the study of a sequence of
examples from the object-capability literature and the smart contracts world: the
membrane \cite{membranesJavascript}, the DOM \cite{dd,ddd}, the Mint/Purse \cite{MillerPhD}, the Escrow \cite{proxiesECOOP2013}, the DAO \cite{Dao,DaoBug} and
ERC20 \cite{ERC20}.  As we worked through the
examples, we found a small set of language constructs that let us
write holistic specifications across a range of different contexts.
%
In particular, \Chainmail extends 
traditional program specification languages \cite{Leavens-etal07,Meyer92} with features which talk about:
%
\begin{description}
\item[Permission: ] 
%\ \ \textbullet \ \emph{Permission}, \ie
Which objects may have access to which other objects; 
this is central since access to an object usually also grants access to the functions it provides.
%
\item[Control: ]
%\ \ \textbullet  \ \emph{Control}, \ie
Which objects called functions on other objects; this
 is useful in identifying the causes of certain effects - eg 
funds can only be reduced if the owner called a payment function.
%
%$\bullet$ \ \\emph{Authority}, \ie which objects' state or properties may change; this is useful in describing effects, such as reduction of funds.
%
\item[Time: ]
%\ \ \textbullet \ \emph{Time}:\ \ie
What holds some time in  the past, the future, and what changes with time,
\item[Space: ]
%\ \ \textbullet \ \emph{Space}:\ \ie
Which parts of the heap are considered when establishing some property, or when 
performing program execution; a concept
related to, but different from, memory footprints and separation logics,
\item[Viewpoint: ]
%\ \ \textbullet \ \emph{Viewpoints}:\ \ie
%a distinction between the objects internal to our component, and those external to it;
Which objects and which configurations are internal to our component, and which  are
external to it;
a concept related to the open world setting.
\end{description}

%% Holistic assertions usually employ several of these features. They often have the form  of a guarantee
%% that only if some property holds now will a certain effect occur in the future, or that
%% certain effects can only be caused if another property held earlier.
%% For example, if within a certain heap (space) some change is possible in the future (time), then this particular heap 
%% (space again) contains 
%% at least one object which has access (permission) to a specific other, privileged object.
%% %\james{moved around --- not sure we need this para}
%% %\susan{I think we don't so there is a paragraph I have commented out.}
%% \forget{Often, holistic assertions typically have the form of a guarantee
%% that if some property ever holds in the future, then some other property holds now.
%% For example, if within a certain heap some change is possible in the future, then this particular heap contains 
%% at least one object which has access to a specific other, privileged object.}
%% A module satisfies a holistic assertion if  
%%  the assertion is satisfied  in all runtime configurations reachable through execution of the combined code of our module and any other module.
%%   This reflects the open-world view.
 
  
The rest of the paper is organised as follows:
Section~\ref{sect:motivate:Bank} 
\sd{gives an example from the literature} which we will use 
to elucidate key points of \Chainmail.
%motivates our work via an example, and then section
Section~\ref{sect:chainmail} presents the \Chainmail\ specification
language.  Section~\ref{sect:formal} introduces the formal model
underlying \Chainmail, and then section~\ref{sect:assertions} defines
the 
semantics of \Chainmail's assertions.
% SD the below is NOT ture
%full details are relegated toappendices.   
Section~\ref{sect:discussion}
discusses our design, section~\ref{sect:problemdriven} shows how key points of 
exemplar problems can be specified in \Chainmail. 
\jm{The semantics of \Chainmail, along with the underlying language \LangOO, have been formalised in Coq. Further, we have used Coq to prove that \Chainmail assertions are classical, and satisfaction of specific examples can be proven.} Section~\ref{sect:model} discusses the Coq model, \jm[``section'' -> ``Section'']{Section}~\ref{sect:related} considers related
work, and \jm[do we want to capitalize Section? This would be in line with capitalisation of ``Fig'']{Section}~\ref{sect:conclusion} concludes.
We relegate various details to appendices. 

An earlier version of this work appeared at FASE 2020 \cite{FASE}. The main extensions of this paper are linearising the logic for temporal connectives, an expanded specification of \LangOO, and simplifications to name-binding with respect to quantification and time. \jm{The current} paper also introduces an expanded implementation of the mechanised model, and a deeper understanding of many of the properties of \Chainmail. Differences between the \jm{current work and that of FASE 2020~\cite{FASE}} are discussed where relevant throughout the paper.
%Further study discovered intricacies to this problem, which made it unnecessarily complex for a simple introduction.

\section{Motivating Example: The Bank}
\label{sect:motivate:Bank}

\label{section:motivationBank}


Traditional functional specifications describe what components are
guaranteed to do. So long as a method is called in a state satisfying
its preconditions, the method will complete its work and establish a
state satisfying its postconditions.   Method specifications
are \textit{sufficient} conditions under which methods can be called,
that is. sufficient conditions under which their method's behaviour
can be invoked.


Consider the specification of a trivial Bank component in
fig.~\ref{fig:BankSpec}.  The bank is essentially a wrapper for a map
from account objects to account balances: given an instance of a Bank
component, calling \prg{newAccount} returns a new account object and
then calling \prg{initialise} sets up initial accounts and balances.
(GRRR: not really happy about this. should have a boolean setup ghost
field so initialise can onlyt be called once. which the functions have
as precondtion :ARGH).  Given an account, calling \prg{balance} with an
account returns the account balance, and calling \prg{deposit} with
two accounts deposits funds from the source to the destination account.


\begin{figure}[tbp]
\begin{lstlisting}
  specification Bank {

    ghost field ledger : Map[Account, Number]

    function newAccount {
      a \in Account, b \in Bank
        { a := b.newAccount }
      FRESH(a)
    }

    function balance {
      a \in Account, b \in Bank, n \in Number
        { n := b.balance(a) }
      n == ledger.at( a )
    }

    function deposit {
      src, dst \in Account, b \in Bank, n \in Number
      sb == ledger.at(src), db == ledger.at(dst), n > 0
        { b.deposit(dst, src, n) }
      ledger.at(src) == sb - n
      ledger.at(dst) == db + n
    }

    function initialise {
      b \in Bank, m in Map[Account, Number]
        { b.initialise( m ) }
      ledger == m
    }

  }
\end{lstlisting}
\caption{Functional specification of a Bank}
\label{fig:BankSpec}
\end{figure}

The specification in fig.~\ref{fig:BankSpec} is enough to let us
calculate the result of operations on the bank and the accounts ---
for example it is straightforward to determine that the code in
fig.~\ref{fig:rego} satisfies its assertions: given that the
\prg{acm} object has a balance of 10,000 before an author is
registered then afterwards it will have a balance of 11,000 while the
\prg{author} now has a balance of 500 from a starting balance of 1,500
(barely enough to buy a round of drinks at the conference hotel bar).


\begin{figure}[tbp]
\begin{lstlisting}
  assume b.balance(acm) == 10000
  assume b.balance(author) == 1500

  b.deposit(acm, author, 1000)

  assert b.balance(acm) == 11000
  assert b.balance(author) == 500
\end{lstlisting}
\caption{Registering at a Conference}
\label{fig:rego}
\end{figure}


This reasoning is fine in a closed world, where we only have to
consider complete programs, where all the code in our programs (or any
other systems with which they interact) is under our control.   In an
open world, however, things are more complex: our systems will be made
up of a range of components, many of which we do not control; and
furthermore will have to interact with external systems which we
certainly do not control.  Returning to our author, say some time
after regisering by executing the code in fig.~\ref{fig:rego}, they
attempt to pay for a round at the bar.  Under what circumstances can
they be sure they have enough funds in their account?

To see the problem, consider the additional function specified in
fig.~\ref{fig:steal}. This method says the bank additional provides a
\prg{steal} method that empties out every account in the bank and puts
all their funds into the thief's account. If this method exists, and
if it is somehow called between registering at the conference and
going to the bar, the author (actually everyone using the same bank)
will find all their accounts empty (except the thief, of course).

\begin{figure}[tbp]
\begin{lstlisting}
  specification Theft {

    function steal {
      b \in Bank, thief in Account, m in Map[Account, Number]
      m == b.ledger
        { b.steal(thief) }
      forall a in dom(m) :
      ledger.at(a) =
        if (a == thief)
          then sum(codom(m))
          else 0
    }
  }
\end{lstlisting}
\caption{Functional specification of taxation}
\label{fig:steal}
\end{figure}

The critical problem is that a bank implementation including a \prg{steal}
method would meet the functional specifications of the bank friom
fig.~\ref{fig:BankSpec}, so long as its \prg{newAccount},
\prg{balance}, \prg{deposit}, and \prg{initialise} methods do meet
that specification.

One obvious solution would be to return to a closed-world
interpretation of specifications: we interpret specifications such as
fig.~\ref{fig:BankSpec} as \emph{exact} in the sense that only
implementations that meet the functional specification exactly,
\emph{with no extra methods or behaivour}, are considered as suitable
implementations of the functional specification. The problem is that
this solution is far too strong: it would for example rule out a bank
that simply counted the number of deposits that had taken place,
i.e. met fig.~\ref{fig:count} as well as fig.~\ref{fig:BankSpec}.


\begin{figure}[tbp]
\begin{lstlisting}
  specification CountDeposits {

    ghost field count : Number = 0

    function deposit {
      c : Number = count
        { b.deposit(dst, src, n) }
      count == c + 1
    }

    function count {
      b : Bank
        { c = b.countDeposits }
      c == b.count
    }
  }
\end{lstlisting}
\caption{Functional specification counting the number of deposits}
\label{fig:count}
\end{figure}


What we need is some way to permit bank implementations that meet
fig.~\ref{fig:count} but to forbit implementatons that meet
fig.\ref{fig:steal}.  The key here is to capture the (implicit)
assumptionms underlying fig.\ref{fig:BankSpec}, and to provide
additional specifications that capture those assumptions.  There are
at least two assumptions that can prevent methods like \prg{steal}:

\begin{enumerate}
\item after initialisation, the \emph{only} way an account's
  balance can be changed is if a client calls the \prg{deposit} method
\item an account's balance can \emph{only} be changed if a client has
  that particular account object.
\end{enumerate}

Compared with the functional specification we have seen so far, these
assunmptions capture \emph{necessary} conditions rather than
\emph{sufficient} conditions. It is necessary that the \prg{deposit}
method is called to change an account's balance, and it is necessary
that the particular account object can be passed as a parameter to
that method. The fig.~\ref{fig:steal} specification is not consistent
with these assumptions, while the.~\ref{fig:count} specification is
consistent with these assunmptions.


The contribution of this paper is a specification langauge and
semantics that can be used to specify necessary specifications, and a
semantics for those specifications that can determine whether some
functional (sufficient) specifications are consistent (or not) with
the necessary specifications. Fig.~\ref{fig:nec} shows how we
can express these two informal assumptions using our specification
language Chainmail II.  Rather than specifying \prg{functions} to
describe the behaviour of particular methods when they are called, we
write \prg{policies} that range across the whole behvaiour of the
component.


An overall ``holistic'' specification for the bank account, then,
would be our original sufficient functional specification from
fig.~\ref{fig:BankSpec} plus the necessary security policy
specification in fig.~\ref{fig:nec}.  This holistic specification
permits an implememntation of the bank that also meets the \prg{count}
specification from fig.~\ref{fig:count}, but does not permit an
implementation that also meets the \prg{steal} specification from
fig.~\ref{fig:steal}.


\begin{figure}[htbp]
%\begin{definition}
%\label{def:pol2}

    (1)\ \  $\triangleq$\ \ $\forall \prg{a}.\forall \prg{S}.\ [ \ \  \prg{a}:\prg{Account}\   \wedge\   \prg{this}\neq\prg{a} \ \wedge\ \Using{(\Future\Changes{\prg{a.balance}})}{\prg{S}}\ \ \   \
    \longrightarrow$ \\
 $\strut \hspace{3.9cm} \hfill \exists \prg{o}.\ [\, \prg{o}\in \prg{S}\ \wedge \ \Calls{\prg{deposit}}\ \wedge  \ \prg{o} \notin\prg{Internal}(\prg{a}) \ ] \ \ \ \ ]$

\vspace{.1cm}

    (2)\ \  $\triangleq$\ \ $\forall \prg{a}.\forall \prg{S}.\ [ \ \  \prg{a}:\prg{Account}\   \wedge\   \prg{this}\neq\prg{a} \ \wedge\ \Using{(\Future\Changes{\prg{a.balance}})}{\prg{S}}\ \ \   \
    \longrightarrow$ \\
 $\strut \hspace{3.9cm} \hfill \exists \prg{o}.\ [\, \prg{o}\in \prg{S}\ \wedge \ \CanAccess{\prg{o}}{\prg{a}}\ \wedge  \ \prg{o} \notin\prg{Internal}(\prg{a}) \ ] \ \ \ \ ]$

%\end{definition}

\caption{Necessary specifications for \prg{deposit}}
\label{fig:nec}
\end{figure}


Policy (1) in fig.~\ref{fig:nec} says that if --- in any future state
($\Future$\ldots) an account's balance is changed
($\Changes{\prg{a.balance}}$)
then there must be some client object $o$ which is outside the
bank and its associated accounts ($\prg{o} \notin\prg{Internal}(\prg{a})$)t
that calls the \prg{deposit} method: ($\Calls{\prg{deposit}}$).
Policy (2) similarly constraints any possible code that may change an
account's balance, but requires that the client object making the call
has direct access to the account object
($\CanAccess{\prg{o}}{\prg{a}}$).

We can then prove that e.g.\ the \prg{steal} method from
fig.~\ref{fig:steal} is inconsistent with both of these policies.
First, the \prg{steal} method clearly changes the balance of
every account in the bank, but policy (1) requires that any method
that changes the balance of any account must be called \prg{deposit}.
Second, the \prg{steal} method changes the balance of every account in
the system, and will do so without the called having a reference to
most of those accounts, which breaches policy (2).   Note
that \prg{steal} putting all the funds into the thief's account
does not breach policy (2) with respect to the thief's own account,
because that account is passed in as a parameter to the \prg{steal}
method, and so the called of the \prg{steal} must have access to that
account.


\paragraph{random minor point}

These necessary specification policies
can be defined and interpreted independently of any particular
implementation of a specification --- rather our policies constrain
implementations, in just the same way as traditional functional
specifications.  This is in contrast to e.g.\ class invariants, which
establish invariants across the implementation of an abstract, or
abstraction functions, which link an abstract model to a concrete
implementation of that model. 



\section{\Chainmail\ Overview}
\label{sect:chainmail}
%\section{\Chainmail\ Overview}
%\label{sect:chainmail}

In this section we give a brief, informal, overview of the most salient features of  \Chainmail --
a full exposition appears in Section \ref{sect:assertions}.
We revisit the \prg{Bank} example from the previous section. 
To explain the meaning of our holistic assertions, we will use specific configurations $\sigma_1$ and $\sigma_2$
depicted in the left and right diagrams in Figure \ref{fig:BakAccountDiagrams}.
Configuration  $\sigma_1$ may arise from execution using a module $M_{BA1}$, where the objects of
class \prg{Account} have a field pointing to their \prg{Bank}, an another field containing their \prg{balance}
-- the code can be found in \TODO. 
Configuration  $\sigma_2$ may arise from execution using a module $M_{BA2}$, where the objects of
class \prg{Bank} have \prg{ledger} implemented though a sequence of \prg{Node}s, each of which has a 
 field pointing to the \prg{Account}, an another field containing that account's \prg{balance}; also
 \prg{Account}s have a field pointing to the \prg{Bank}.

In our diagrams. 
We use the transparent green rectangle to show which objects belong to  $\M_{BA1}$ or  $\M_{BA2}$ respectively.
We have a  \prg{Bank} object at \prg{1},
 three \prg{Account} objects at \prg{2}, \prg{3} and \prg{4}, and some further  objects of the 
   module $\M_{BA}$  %\prg{Bank}/\prg{Account} module are depicted in 
at \prg{7}, \prg{8} \etc, all in green. 
The objects
outside the module are depicted in grey; here objects \prg{100}, \prg{101}, \prg{200}, and \prg{201}.

 
\begin{figure}[htbp]
\begin{tabular}{lr}
 \begin{minipage}{0.45\textwidth}
 \includegraphics[width=\linewidth, trim=55  330 300 60,clip]{diagrams/BankAccount_version_1.pdf}
 \end{minipage}
 &
 \begin{minipage}{0.45\textwidth}
 \includegraphics[width=\linewidth, trim=55  330 300 60,clip]{diagrams/BankAccount_version_2.pdf}
 \end{minipage}
\end{tabular}
\caption{Two runtime confugurations for the \prg{Bank}/\prg{Account} example.}
\label{fig:BakAccountDiagrams}
\end{figure}

% X Y Z W
% Z: top
% LEFT

  
%%Unlike JML or Dafny \cite{Leavens-etal07,dafny} \Chainmail specifications are not
%%tied tightly to the systems they are specifying.
%%\sdcomment
%%\sophia{Not sure what this means.}
%% Rather, a system may
%%be described by a set of \Chainmail\ specifications, each consisting of
%%a number of named \lstinline+policy+ clauses.
%%%\sdcomment
%%\sophia{We do not have that 
%%syntax any more. Do we need it?}
%\Chainmail\ specifications and policies overlap and are interlinked to
%provide strong protection against attacks --- just like the links in
%physical chainmail.
%%\sdcomment
%\sophia{What does overlap mean?}
%Each policy should aim to capture one very
%specific concern in the design of a system. 
%%\sdcomment
\sophia{I thibk this para is great -- not sure it belongs here.}
While traditional policies are
expressed as Hoare triples --- often describing a single method
invocation on an instance of the class being specified (as in
XXXXX), R%Figure~\ref{BANK-OR-DOM}).  
holistic policies are expressed as 
temporal or spatial invariants that a module that
conforms to the specification must maintain even though any other code
may be executed (as in the three policies in section~\ref{DOM-OR-BANK}).

%% \Chainmail\ policies (and specifications) can cross cut both each
%% other and the various modules and objects in the system being
%% specified.  The validity of a specification is the conjunction of its
%% policies; a module or an object must satisfy all the specification's
%% policies for us to consider that the object meets the specification.
%% Policies and specifications are not tied to any specific module or
%% class: rather, any implementing module that satisfies the
%% specification's policies obeys the specification.


\Chainmail can express necessary specifications because its invariant
language includes \kjx{several carefully-chosen} \textit{holistic}
assertion constructs, along with traditional state-based assertions.
%\jncomment
\james{if we say things like that, need to discuss alternative
  constructs in the discussion section. do we have alternatives?
  transitive Access. two-state assertions. what else? \sd{that would be good, but 
  more work to do. Can some body do it? Is it very important to have?}}.
That is, as well as supporting assertions on the contents of local
variables and object fields (\eg $\prg{x}.\f1 > \prg{this}.\f2$),
\Chainmail\ incorporates assertions which talk about
%
\textit{access}
--- objects being accessible from other objects (\eg
$\CanAccess{\x}{\y}$);
%
\textit{control} --- the
next method to be invoked ($\Calls {\x} {\y} {\m} {\z}$);
%
\textit{authority} --- about the change of some
property (\eg $\Changes{\x.\f}$);
%
\textit{space} --- some property being observable within a subset of
the current state ( $\Using{\A}{S}$);
%
and
%
\textit{time} --- about some property
holding in the future or in the past (\eg $\Future \A$ or $\Past \A$).
%
.
%\scd
  \sophia{While
many  individual
features of \Chainmail can  be found also in other work, we
%claim that their  combination as well as
%their application in the specification of open systems are novel.
argue that their power and novelty for specifying open systems lies in their careful
  combination}
  %\kjx
  \james{Hmm. a delicate, subtle argument \ldots }
%
%
These assertions draw from some concepts from object capabilities
($\CanAccess{\_}{\_}$  for  permission and $\Changes{\_}$ for
authority) 
as well as temporal logic ($\Future \A$, $\Past \A$ and friends), and the relation of
our spatial connective ($\Using{\A}{S}$)  with ownership and effect
systems.
%\sdcomment
\sophia{TODO: add references here}. 
%\jncomment
\james{don't forget the whole
  AOP mionitoring/specs stuff, just for fun.  (HELM contracts!!! (vs
  Meyer Contracts :-)}  
%
%\jncomment
\james{somewhere, should we say something like: the goal is
  to allow wholistic specifations with as extra little machinery as
  possible over a basic Hoare language}.

\paragraph{Permission: Access}

Our first holistic assertion $\CanAccess{\x}{\y}$ asserts that one
object $\x$ has a direct reference to another object $\y$: that a
reference to $\y$ is the value of at least one of $\x$'s fields.
This assertion can be used to make assertions about heap structures (and thus
object values), for example if a \prg{cacheValid} field is true,
then that object can access to a cached value:

\begin{lstlisting}
  a.cacheValid == true `$\weakImplies \CanAccess(a)(cacheValue)$` 
\end{lstlisting}

%\kjx
\james{do we really not also need reachable (transitive closure of access)}
\susan{we don't seem to need transitivity in any of the examples}


\paragraph{Control: Calls}

The  $\Calls {\x} {\y} {\m} {\z}$
assertion is more-or-less the control flow analogue of
the access assertion, and is true 
in program states where a method on object 
${\x}$ makes a method call ${\y}.{\m}({\z})$ --- that is it calls method 
{\m} on object {\y} with arguments {\z}.


\begin{lstlisting}
   another illustrative example, presumably in the context of the
   running example from the intro. which we need to pick first.
\end{lstlisting}


\paragraph{Authority Changes}

The $\Changes{\x.\f}$  assertion is true when the value of \prg{x.f}
in the next state is different to the value in the current state.
For example, we 

\begin{lstlisting}
   another illustrative example, presumably in the context of the
   running example from the intro. which we need to pick first.
\end{lstlisting}

\paragraph{Space: With}

The space assertionn $\Using{\A}{S}$ states the some assertion $\A$ is
true when the heap used by that assertion is restrcited to the
footprint $\S$  
%\kjx
\james{footprint F?}.

%\kjx
\james{OK someone needs to explain what that's for and when/why it is sound to
do it}

\begin{lstlisting}
   another illustrative example, presumably in the context of the
   running example from the intro. which we need to pick first.
\end{lstlisting}


\paragraph{Time: Next, Will, Prev, Was}

\Chainmail supports several temporal operators familiar from temporal
logic ($\Future \A$ or $\Past \A$ or $\Next \A$ or $\Prev \$A$). 
We support birectional temporal
assertions, constraining either future ($\Future \A$. $\Next \A$)
or past behaviour ($\Past \A$.  $\Prev \$A$) either considering only
the  immediate next or immediate previous step ($\Next \A$,
$\Prev \$A$) or for conditions that become eventually true in some
distant future, or were true once in some distant past  
($\Future \A$, $\Past \A$). We have bidirectional pairs of operators
to give expressiveness in writing assertions: this does not offer any
additional reasoning power. 

For example, a part of the observer pattern is that when
a subject is notified of a change, then the observer must
be told to update itself.    We can write this from the
subject's perspective, looking forwards:

\begin{lstlisting}
  Call(_,subject,notify,_) --> Will(Call(subject,observer,update,_))
\end{lstlisting}

\noindent meaning that once notify is called on a subejct, then its
observer will be updated sometime in the future.  We can write a very
similar specification for an observer, looking backwards.

\begin{lstlisting}
  Call(subject,observer,update,_) --> Was(Call(_,subject,notify,_))
\end{lstlisting}

\noindent meaning that if a subject updates an observer, that subject
have been notified sometime previously. We could tighten each
specifaction, so that the update must immediately follow the
notification, by replacing $\Future \A$ or $\Past \A$ with 
$\Next \A$ or $\Prev \A$.

%\kjx
\james{bother. should probably do time earlier, because most of the
   example assertions I can think of a things like
   \lstinline+(change(x.f) ->  past(call(_,xm,_)+ which needs the temporal operator}


%\jncomment
\james{OK really do need the example to be chosen to do more work here for this}



\section{Overview of the Formal foundations}
\label{sect:formal}
We now give an overview of the formal model for \Chainmail. In section \ref{sect:overviewmodel} we 
introduce  the shape of the judgments used to give semantics to \Chainmail, while in section \ref{sect:PL} 
we describe the most salient aspects of an underlying programming language used in \Chainmail.

\subsection{\Chainmail\ judgments}
\label{sect:overviewmodel}
%\section{Overview of the \Chainmail\  formal model}
% \subsection{The Open World}

\kjx{Can we have a better title? is this style of modelling the adversary claimed contribution of the paper that we should claim explicitly?  IF so the title should highlight that claim.}

Having outlined the ingredients of our holistic specification
language, the next question to ask is: When does a module $\M$ satisfy
an holistic assertion $\A$?  More formally: when does
%
%\sd{So, the question about modules satisfying assertions put formally is,} when does  \\
$~ \strut  \hspace{1.3in}\ \M \models \A$ \\
hold? \kjx{do we really want M modles A on a line on its own?}

  
Our answer has to reflect the fact that we are dealing with an 
\emph{open  world},  where  $\M$, our module, may be
linked with \textit{arbitrary untrusted code}.
Modules are repositories of code, so we adopt the common formalisation 
of modules as  mappings from
class identifiers to class definitions.
%
%% \sd{Note that we use the term \emph{module} to talk about repositories of code; in this work modules are mappings from
%% class identifiers to class definitions.}
%
%
To model the open world, we consider
 pairs of modules, 
$\M \mkpair {\M'}$,  where $\M$ is the module 
whose code is supposed to satisfy the assertion,
and $\M'$  is  another % wused to say \textit{any}  -- but why?
 module which exercises
the functionality of $\M$. We call our module $\M$ the {\em internal} module, and
 $\M'$ the {\em external} module, which represents potential
 attackers or adversaries.
%or {\em potentially adversarial} module. 
    
We can now answer the question: $\M \models \A$ 
 holds if for all further, {\em potentially adversarial}, modules $\M'$ and in  all runtime configurations $\sigma$ which may be observed as arising from the  execution of the code of $\M$ combined with that of $\M'$, the assertion $\A$ is satisfied. More formally, we define:\\
$~ \strut  \hspace{1.3in} \M \models \A \ \ \  \ \ \ \ \ \mbox{
if               } \ \ \  \ \ \  \  \forall \M'.\forall \sigma\in\Arising
{\M \mkpair  {\M'}}. [\ \M \mkpair  {\M'},\sigma \models \A\ ]$.  \\
Module $\M'$ represents all possible clients of {\M}; and as it is arbitrarily chosen, it reflects the open world nature of our specifications.%\sdcomment

%% \sophia{Is is sentence superfluous now?.}
%% \sophia{In contrast to what we said on Friday's conf call we do not need to put any restrictions
%% on $\M'$ -- not even disjointness is required.}
%% \kjx{OK so in the \textbf{next iteration} we can just replace M;M' with a ' operator applied to any module\ldots}

The judgement $\M \mkpair  {\M'},\sigma \models \A$ means that  
assertion $\A$ is satisfied by  $\M \mkpair  {\M'}$ and $\sigma$.  
As in traditional specification languages \cite{Leavens-etal07,Meyer92}, satisfaction is judged 
in the context of  runtime configuration $\sigma$; but in addition, it is judged in the context of the internal and external modules.
We need the modules, because temporal assertions may
talk about future configurations: the modules contain the code necessary to reach those configurations.

Our distinction between internal and external modules offers two other advantages.
\kjx{is this distinction a claimed contribution?}
First, 
\Chainmail\ includes the ``$\External{\prg{o}}$'' assertion to require
that an object belongs to the external module, as in the Bank
Account's assertion (2)and (3) in
section~\ref{sect:motivate:Bank}. Second, we adopt a version of
visible states semantics \cite{MuellerPoetzsch-HeffterLeavens06,other-visible-state-semantics}, treating all
executions within a module as atomic.
We only record runtime configurations which are {\em external}
 to module $\M$, \ie those where the
 executing object (\ie the current receiver) comes from module $\M'$.
 Program execution is
 a judgment of the form\\
 $~ \strut  \hspace{1.3in}    \M \mkpair  {\M'},\sigma \leadsto \sigma'$\\  
 where we ignore all intermediate steps
 whose receivers are internal to $\M$. 
Similarly, when considering $\Arising {\M \mkpair  {\M'}}$, \ie the configurations arising from 
executions in $\M \mkpair  {\M'}$, we can take method bodies defined in $\M$ or in $\M'$, but we will only consider the runtime 
configurations which are external to $\M$.

%\sd{Therefore, the pair $\M \mkpair  {\M'}$ is different than the concatenation of the two modules}
%In  that sense, our approach is similar to that of visible states semantics, without, however the need to consider issues
%around different objects of the same class or re-entrancy.

\kjx{if this is about the internal/external distinction, I'd move the
following to section 6?, someoone else should check}

\sd{As a notational convenience, we keep the code to be executed as a component of the runtime configuration.
Thus, $\sigma$ consists of a stack of frames and a heap, and each frame consists of a variable map and a continuation.
The variable map is a mapping from variables to addresses or to set of addresses -- the latter are needed to
deal with assertions which quantify over footprints, as \eg (1) and (2) from section \ref{sect:motivate:Bank}.}


\sd{To give meaning to assertions with footprint restrictions such as \eg $\Using {\A} {\prg{S}}$ ,we define  restrictions on the
configuration. Thus $\sigma\!\!\downarrow_{\sigma(\prg{S})}$ is the same as $\sigma$ but with the domain of the heap restricted
to the addresses from $\sigma(\prg{S})$. And then we define\\
  $~ \strut  \hspace{1.3in} \M \mkpair  {\M'},\sigma  \models\Using {\A} {\prg{S}}$\ \ \ \ \ if \ \ \ \  \  $\M \mkpair  {\M'},\sigma\!\!\downarrow_{\sigma(\prg{S})}  \models {\A}$}
  
\sd{The meaning of assertions therefore may depend on the variable map, eg \prg{x} may be pointing to a different 
object in .... TODO
The treatment of time in combination with the fact that the meaing od assertions TODO}




\subsection{An underlying programming language, \LangOO}
\label{sect:PL}
 %\newcommand{\appref}[1]{, c.f. App. Def.\ref{#1}}
 
 
As was have already seen, \Chainmail assertions not only talk about the contents of the current state (stack frame and heap),
but they also talk about future and past states, Therefore, the meaning to \Chainmail assertions depends on the underlying programming
language. In this section, we outline a minimal such language, which we call  \LangOO. Full definitions appear in Appendix \ref{app:LangOO}. \sophia{TODO-say which features need to be in such a language and say that \LangOO is an example of such one.}

Central to our work is the concept of \emph{module}, which is a repository of code. Modules map class identifiers to class definitions, \appref{defONE}, and class definitions consist of method declarations and field declarations, \appref{def:syntax:classes}

\section{Assertions}
\label{sect:assertions}
%\section{ Assertions}
%\label{sect:assertions}

\subsection{The syntax of Expressions and Assertions}

%\secomment
\susan{if you like this I will write macros so as not to have parameter lists with the keywords, alternatively this list could just have the identifiers and no descriptions\sd{not clear what is meant}}
In section~\ref{sect:chainmail} we introduced our assertion language \Chainmail with keywords 
$\CanAccess{}{}$ to check whether one object can call another, $\Calls{}$ for the current function call, 
$\Changes{}$ to check whether the next configuration will affect validity of some assertion, and 
 $\Next {}$ or $\Future {}$  for expressing an assertion will hold at
the immediate successor execution point or at some future point, and
$\Prev{}$ or $\Past{}$ to express  that an assertion held at the immediately previous or
some point in the past, and  $\Using{}{}$, 
for expressing that an assertion holds in
the sub-configuration determined by a witness.

The keywords enable \Chainmail assertions to support 
reflection over various aspects of the current 
runtime configurations, reflection over past or future configurations, and 
reflection over sub-configurations.
Assertions can contain logical operators and interestingly, the existential and universal quantifiers may quantify over object addresses, as well as 
over sets of addresses, numbers, and sequences of field identifiers of a given length.


%\secomment
\susan{if you prefer this paragraph then link it back to section 4. \sd{Yes, it may now be superfluous.}}


Assertions, $\A$, support standard logical operators, 
reflection over various aspects of the current 
runtime configurations, reflection over past or future configurations, and 
reflection over sub-configurations.
The standard logical operators are, unsurprisingly,
 $\wedge$, $\vee$, $\rightarrow$, $\neg$, $\exists$ and $\forall$.
Interestingly, the existential and universal quantifiers may quantify over object addresses, but also 
over sets of addresses, numbers, and sequences of field identifiers of a given length.
When reflecting over the current state, we can reflect over the class and contents of objects
(\eg \x:\prg{ClassId} or \x.\f=\y.\f'), whether an
object has direct access to (and thus may call on) another object $\CanAccess{\_}{\_}$,
and the current function call $\Calls{\prg{\_},\prg{\_},\prg{\_},\prg{\_}}$.
We can also talk about whether the next configuration will affect the 
validity of some assertion $\Changes{\_}$
\footnote{Note that $\Changes{\_}$ may be encoded; do we keep it?
The reason to keep it is that we can then talk of "permission" and "authority" }.  
We also support {\em temporal} modifiers, where $\Next \A$ or $\Future \A$  express  that $\A$ will hold at
the immediate successor execution point or at some future point, while
$\Prev \A$ or $\Past \A$ express  that $\A$ held at the immediately previous or
some point in the past.
Finally, we support a {\em spatial modifier}, $\Using{\A}{S}$, 
which expresses that assertion $\A$ holds in
the sub-configuration determined by the witness \prg{S}.


\begin{definition}[Assertions] The syntax of simple expressions $\SE$) and assertions ($\A$) is:
\label{def:assertions}

 $\begin{array}{lcl}
  \SE & ::= &  \prg{true}  \ \mid\ \prg{false}  \    \mid\ \prg{null}  \ \mid \ \x  \ \mid \ \SE.\f    \ \mid \ \SE.\f^n \   \ \mid\  \ \\
 ~ \\
\A &\ ::=\  &   \SE  \ \mid \ \SE > \SE \ \mid \  \SE=\SE  \ \mid \ \SE \equiv \SE\ \mid \   \SE:\prg{ClassId}  \ \mid \
    \SE\in\prg{S}   \ \mid  \ \A \rightarrow \A  \ \mid\  \  \\
 &   &  \exists \x.\A  \ \mid \  \exists \prg{S}:SET.\A  \ \mid \  \exists fs:FLD^k.\A
 \ \mid \  \exists k:\mathbb{N}.\A  \ \mid\  \
\\
 &    & \CanAccess x y \ \mid\  \ \Changes e \ \mid\  \Calls{\prg{x},\prg{y},\prg{m},\prg{z}} \ \mid\  \\  
 &    &  \Next \A  \ \mid \   \Future \A \ \mid \  \Prev \A    \ \mid \  \Past \A \ \mid \ \Using \A \prg{S }  \ \mid\  \
% \\
 \\
  &   &  \A \wedge \A  \ \mid\  \ \A \vee \A  \ \mid\  \ \neg A   \ \mid\  \ \forall \x.\A  \ \mid \  \forall \prg{S}:SET.\A  \ \mid \  \forall fs:FLD^k.\A
 \ \mid \  \forall k:\mathbb{N}.\A
\end{array}$


\end{definition}

Note that the operators $\wedge$, $\vee$,  $\neg$ and $\forall$  could have been defined  through the usual shorthands, \eg, $\neg \A$ is short for
$\A \rightarrow \ff$ \etc, but here we give full definitions instead.
 Validity of assertions has the format $\M\mkpair \M', \sigma \models \A$, where  $\M$ is the internal module, whose internal workings
 are opaque to the external, client module $\M'$.

\subsection{Configuration adaptation and configuration restrictions}
In order to define whether a runtime configuration satisfies an assertion we need two auxiliary concepts:
the adaptation of a runtime configuration to another one, and the restriction of a runtime configuration to only the set of objects from a
given set.

We need adaptation to deal with time, and the corresponding changes of scope. For example, the assertion
$\Future {\x.\f=\prg{3}}$, is satisfied if in some {\em future} configuration, the field  \f\, of the object that is pointed at by \x\, in the {\em current} configuration has the value \prg{3}; note that in the future  configuration, \x\, may be pointing to a different object, or may
even no longer be in scope (\eg if a nested call or a nesting call is executed).
Therefore, we introduce the operator \  $\adapt\;$,  \ \ which combines runtime configurations: $\sigma \adapt \sigma'$ adapts the second configuration to the top frame's view of the former: it returns a new configuration whose stack has  the top frame as taken from $\sigma$ and where the \prg{contn} has been consistently renamed, while the heap is taken from $\sigma'$. This allows us to interpret expressions  in the newer (or older) configuration $\sigma'$ but with the variables bound according to the top frame from $\sigma$; \eg we can obtain that value of \prg{x}.\prg{f} in configuration  $\sigma'$ even if \prg{x} was out of scope. The consistent renaming of the code allows the correct modelling of execution (as needed,   for the semantics of  nested time assertions, as \eg in $\Future {\x.\f=\prg{3} \wedge \Future {\x.\f=\prg{5}}}$


 \begin{definition}[Adaptation on Runtime Configurations]  The operator $\adapt$\ \  is a binary operator on runtime configurations.
 \label{def:config:adapt}
 $~ $

\begin{itemize}
\item
$\sigma \adapt \sigma' \triangleq (\phi''\cdot\psi',\chi')$  \IFF $\sigma=(\phi\cdot\_,\_)$, and $\sigma'= (\phi'\cdot\psi',\chi')$, and
 \\
$\ \strut \ \ \hspace{1.45in} $
$\phi$=$(\prg{contn},varMap)$, and $\phi'$=$(\prg{contn}',varMap')$, and
 \\
$\ \strut \ \ \hspace{1.45in} $     % $\phi''$ such that
  $\phi''=(\, \prg{contn}'[\prg{zs}/\prg{zs}' ],\,varMap'[\prg{zs}'\mapsto varMap(\prg{zs})]\, ) $, where
 \\
$\ \strut \ \ \hspace{1.45in} $
\prg{zs}=$dom(varMap)$, and
 \\
$\ \strut \ \ \hspace{1.45in} $      $\prg{zs}'$ is a set  of variables with  the  same cardinality as \prg{zs}, and
 \\
$\ \strut \ \ \hspace{1.45in} $   all variables in
$\prg{zs}'$  are fresh in $varMap$ and in $varMap'$.


\end{itemize}

\end{definition}

 On the other hand, an assertion of the form $\Using{A}{S}$ promises that $\A$ holds in subconfiguration, whose heap is restricted to the objects from \prg{S}.

 \begin{definition}[Restriction on Runtime Configurations]  The restriction operator~$\;\restrct{} {} $ applied to a runtime configuration $\sigma$ and a set $R$ is defined as follows:
 \label{def:config:restrct}
 $~ $

\begin{itemize}
\item
$\restrct {\sigma}{\prg{S}} \ \triangleq \ (\phi, \chi')$, \IFF  $\sigma$=$(\phi,\chi)$, \ and  \  $dom(\chi')=\interp {\prg{S}} {\sigma}$, and  \\
$\ \strut \ \ \hspace{1.2in} $
 $\forall \alpha\!\in\!dom(\chi').[ \ClassOf {\alpha} {\chi'} =  \ClassOf {\alpha} {\chi}\ \wedge \ \forall \f.  \chi'(\alpha,\f)=\chi(\alpha,\f)]$.
\end{itemize}
\end{definition}

\subsection{Satisfaction of assertions}



\begin{definition}[Interpretations for simple expressions]

For any runtime configuration, $\sigma$, and any $k\in \mathbb{N}$, and any simple expression, $\SE$, we define its interpretation as follows:

\begin{itemize}
     \item
  $\interp {\prg{true}}{\sigma}$ $ \triangleq$   \prg{true}, \ and \ \    $\interp {\prg{false}}{\sigma}$ $ \triangleq$ \prg{false}, \ and \ \
   $\interp {\prg{null}}{\sigma}$ $ \triangleq$  \prg{null}
  \item
  $\interp {\x}{\sigma}$ $ \triangleq$ $\phi(\x)$  \ \ if \ \ $\sigma$=$(\phi\cdot\_,\_)$
  \item
  $\interp {\SE.\prg{f}}{\sigma}$ $ \triangleq$ $\chi({\interp {\SE}{\sigma}}, \prg{f})$  \ \ if \ \ $\sigma$=$(\_,\chi)$
   \item
     $\interp {\SE.\prg{f}^0}{\sigma}$ $ \triangleq$  $\interp {\SE}{\sigma}$, \ \ \ and \ \ \ $\interp {\SE.\prg{f}^{k+1}}{\sigma}$ $ \triangleq$   $\chi({\interp {\SE.\prg{f}^k}{\sigma}})(\prg{f})$, where $\sigma$=$(\_,\chi)$.
   \end{itemize}
\end{definition}

\begin{lemma}[Interpretation corresponds to execution]
For any simple expression $\SE$, runtime configuration $\sigma$, and value $v$:

\begin{itemize}
     \item
  $\interp \SE {\sigma}$ = $v$\ \     if and only if \ \ $\M_\emptyset, \sigma[\prg{contn}\mapsto \SE] \leadsto v$,\\
  where $\M_\emptyset$ stands for the empty module.
  \item
   $\interp \SE {\sigma}$ = $v$\ \     if and only if \ \ $\M, \sigma[\prg{contn}\mapsto \SE] \leadsto v$ \ \ \ for any module $\M$ .
   \end{itemize}
   \end{lemma}

   \begin{proof} The  first guarantee is proven structural induction  over the definition of $\SE$.
   The second guarantee  is a corollary of the first guarantee  and of lemma \ref{lemma:linking:properties}.\end{proof}


\begin{definition}[Satisfaction of  Assertions] We define below when a configuration satisfies an assertions. We first extend the definition of interpretation
to simple expressions.
\label{def:valid:assertion}

We first consider simpler assertions which only involve expressions:

\begin{itemize}
\item
$\M\mkpair \M', \sigma \models\SE$ \IFF  $\interp{\SE}{\sigma}$ = \prg{true}.
\item
$\M\mkpair \M', \sigma \models\SE>\SEPrime$ \IFF $\interp{\SE}{\sigma}$ > $\interp{\SEPrime}{\sigma}$.
\item
$\M\mkpair \M', \sigma \models\SE=\SEPrime$ \IFF $\interp{\SE}{\sigma}$ = $\interp{\SEPrime}{\sigma}$.
\item
$\M\mkpair \M', \sigma \models\SE\equiv\SEPrime$ \IFF $\SE$ and $\SEPrime$ are textually identical.
\item
$\M\mkpair \M', \sigma \models \SE:\prg{ClassId}$ \IFF $\ClassOf {\interp{\SE}{\sigma}} {\sigma}$ = $\prg{ClassId}$.
\item
$\M\mkpair \M', \sigma \models \SE\in \prg{S}$ \IFF $\interp{\SE}{\sigma}\in \interp{\prg{S}}{\sigma}$.
\end{itemize}

Next, we consider assertions involving existential quantifiers over program variables, field sequences, sets and numbers.

\begin{itemize}
\item
$\M\mkpair \M', \sigma \models \exists x.\A$ \IFF
$\M\mkpair \M', \sigma[\prg{z}\mapsto \alpha] \models  \A[\prg{x}/\prg{z}]$ \ for some  $\alpha\in dom(\sigma)$, and   \prg{z} free in $\sigma$ and $\A$.\item
$\M\mkpair \M', \sigma \models \exists \prg{S}:\prg{SET}\!.\,\A$ \IFF  $\M\mkpair \M', \sigma[\prg{Q}\mapsto R] \models  \A[\prg{S}/\prg{Q}]$ \\
$\strut ~ \hspace{1.4in} $ for some set of addresses $R\subseteq dom(\sigma)$, and   \prg{Q} free in $\sigma$ and $\A$.

\item
$\M\mkpair \M', \sigma \models  \exists \prg{fs}:\prg{FLD}^k\!.\,\A$ \IFF
$\M\mkpair \M', \sigma \models  \A[\prg{fs}/\prg{f}_1.\f_2.\,...\,\prg{f}_k]$\  for  $k$ field identifiers $\prg{f}_1$,..,$\prg{f}_k$.
\item
$\M\mkpair \M', \sigma \models  \exists \prg{n}:\prg{Nat}.\A$ \IFF  $\M\mkpair \M', \sigma \models \A[\prg{n}/k]$\ \ for some $k\in\mathbb{N}$.

\end{itemize}

And now, we consider the assertions which involve space, time or control:

\begin{itemize}
\item
$\M\mkpair \M', \sigma \models  \CanAccess{\prg{x}}{\prg{y}}$   \IFF  \begin{itemize}
\item
$\interp {\x} {\sigma}$=$\interp {\y} {\sigma}$, or
\item
$\interp {\x.\f} {\sigma}$=$\interp {\y} {\sigma}$  for some field \prg{f},  or
\item
$\interp {\x} {\sigma}$=$\interp {\this} {\sigma}$ and
  $\interp {\y} {\sigma}$=$\interp {\z} {\sigma}$,
\
and \z\ appears in  $\sigma$.\prg{contn}.\footnote{
That is, $\CanAccess{\prg{x}}{\prg{y}}$ expresses that \x has a {\em direct} path to \y.
In more detail, in the current frame,
either \x and \y\, are  aliases, or \x points to an object which has a field
whose value is the same as \y, or \x is the currently executing object and \y is
 a local variable or formal parameter which appears in the code in the continuation.
 %That means, that variables which were introduced into the variable map in order to give meaning to existentially quantified assertions are not considered.
 }
 \end{itemize}
 \item
 $\M\mkpair \M', \sigma \models   \Changes{\prg{e}}$  \IFF
 $\exists \sigma'.\, [\ \ \M\mkpair \M',\sigma \leadsto \sigma' \ \wedge \interp{e}{\sigma} \neq \interp{e}{\sigma\triangleleft \sigma'} \ \  ]$.
   \item
$\M\mkpair \M', \sigma \models  \Calls{\prg{x},\prg{y},\prg{m},\prg{z}}$ \IFF \
 $\sigma.\prg{contn}$=\prg{u.m(v);\_}\ \ for some variables \prg{u} and \prg{v},  \ and \
\\ $\strut ~ \hspace{1.4in} $
$\interp{\prg{this}}{\sigma}$=$\interp{\prg{x}}{\sigma}$, \ and\ $\interp{\prg{y}}{\sigma}$=$\interp{\prg{u}}{\sigma}$,
 \ and\ $\interp{\prg{z}}{\sigma}$=$\interp{\prg{v}}{\sigma}$.
 \item
  $\M\mkpair \M', \sigma \models  \Next \A $
  \IFF
  $\exists \sigma'.\, [\ \ \M\mkpair \M',\phi \leadsto  \sigma' \ \wedge \M\mkpair \M',\sigma\adapt\sigma' \models \A \ \  ]$,
 \\
$\strut ~ \hspace{1.4in} $  and where $\phi$ is
so that $\sigma$=$(\phi\cdot\_,\_)$.\footnote{$\M\mkpair \M', \sigma \models  \Future \A $ holds if
$\A$ holds in some configuration $\sigma'$ which arises from execution of $\phi$, where $\phi$ is the top frame of $\sigma$. By requiring that $\phi \leadsto^* \sigma' $ rather than
$\sigma \leadsto^* \sigma' $ we are restricting the set of possible future configurations to
just those that are caused by the top frame.
Namely, we do not want to also consider the effect of  enclosing function calls.
This allows us to write more natural specifications
when giving necessary conditions for some future effect.
}
\item
  $\M\mkpair \M', \sigma \models  \Future \A $
  \IFF
  $\exists \sigma'.\, [\ \ \M\mkpair \M',\phi \leadsto^* \sigma' \ \wedge \M\mkpair \M',\sigma\adapt\sigma' \models \A \ \  ]$,
 \\
$\strut ~ \hspace{1.4in} $  and where $\phi$ is
so that $\sigma$=$(\phi\cdot\_,\_)$.  
  \item
 $\M\mkpair \M', \sigma \models  \Prev \A $ \IFF
 $\forall \sigma_1, \sigma_2. [\ \ \Initial{\sigma_1}\ \wedge \   \M\mkpair \M', \sigma  \leadsto^*  \sigma_2 \ \wedge \   \M\mkpair \M', \sigma_2  \leadsto   \sigma  
$
 \\
$\strut ~ \hspace{1.9in} $  $ \longrightarrow \ \ \   
 \M\mkpair \M', \sigma\adapt\sigma_2  \models \A\ \
 ]$\footnote{past includes the present, perhaps change this}
 \item
 $\M\mkpair \M', \sigma \models  \Past \A $ \IFF
 $\forall \sigma_1, ... \sigma_n. [\ \ \Initial{\sigma_1}\ \wedge \  \sigma_n=\sigma 
  \ \wedge \ \forall i\in[1..n). \M\mkpair \M', \sigma_{i} \leadsto  \sigma_{i+1}
$
 \\
$\strut ~ \hspace{1.9in} $  $ \longrightarrow \ \ \  \exists j\in [1..n-1).
 \M\mkpair \M', \sigma\adapt\sigma_j  \models \A\ \
 ]$\footnote{past includes the present, perhaps change this}
 \item
 $\M\mkpair \M', \sigma \models \Using {\A} {\prg{S}}$
 \IFF
 $\M\mkpair \M', \restrct \sigma {\prg{S}} \models  \A  $.
 \item
  \sd{$\M\mkpair \M', \sigma \models \External {\prg{e}}$}
  \IFF
$\sd{{\interp{\SE}{\sigma}} {\sigma}\notin dom(\M)}$
\end{itemize}

The remaining assertions introduce the remaining logical operators (\ie $\wedge$, $\vee$, $\neg$ and the universal quantifiers). These could be encoded in terms of the preceding operators, but we nevertheless give their meaning explicitly here.

\begin{itemize}
\item
$\M\mkpair \M', \sigma \models \A \rightarrow \A' $ \IFF  $\M\mkpair \M', \sigma \models \A $ implies $\M\mkpair \M', \sigma \models \A' $
\item
$\M\mkpair \M', \sigma \models  \A \wedge \A'$   \IFF  $\M\mkpair \M', \sigma \models  \A $
and $\M\mkpair \M', \sigma \models  \A'$.
\item
$\M\mkpair \M', \sigma \models  \A \vee \A'$   \IFF  $\M\mkpair \M', \sigma \models  \A $
or $\M\mkpair \M', \sigma \models  \A'$.
\item
$\M\mkpair \M', \sigma \models  \neg\A$   \IFF  $\M\mkpair \M', \sigma \models  \A $
does not hold.
\item
$\M\mkpair \M', \sigma \models \forall x.\A$ \IFF
$\sigma[\prg{z}\mapsto \alpha] \models  \A[\prg{x}/\prg{z}]$ \ for all  $\alpha\in dom(\sigma)$, and   all \prg{z} free in $\sigma$ and $\A$.\item
$\M\mkpair \M', \sigma \models \forall \prg{S}\!\!:\!\!\prg{SET}.\A$ \IFF  $\M\mkpair \M', \sigma[\prg{Q}\mapsto R] \models  \A[\prg{S}/\prg{Q}]$ \\
$\strut ~ \hspace{1.4in} $ for all sets of addresses $R\subseteq dom(\sigma)$, and  all \prg{Q} free in $\sigma$ and $\A$.

\item
$\M\mkpair \M', \sigma \models \forall \prg{fs}\!\!:\!\!\prg{FLD}^k.\A$ \IFF
$\M\mkpair \M', \sigma \models  \A[\prg{fs}/\prg{f}_1.\f_2.\,...\,\prg{f}_k]$\  for  all  field identifiers $\prg{f}_1$,..,$\prg{f}_k$.
\item
$\M\mkpair \M', \sigma \models  \forall \prg{n}:\prg{Nat}.\A$ \IFF  $\M\mkpair \M', \sigma \A[\prg{n}/k]$\ \ for all $k\in\mathbb{N}$.
\end{itemize}\end{definition}
 
We define equivalence of   assertions in the usual sense: two assertions are equivalent if they are satisfied  in
the context of the same configurations.
Similarly, an assertion entails another assertion, iff all configurations 
which satisfy the former also satisfy the latter.  

\begin{definition}[Equivalence and entailments of assertions]
$ ~ $

\begin{itemize}
\item
$\A \equiv \A'\  \IFF\    \forall \sigma.\, \forall \M, \M'. \ [\ \ \M\mkpair \M', \sigma \models \A\ \mbox{ if and only if }\ \M\mkpair \M', \sigma \models \A'\ \ ].$
\item
$\A \subseteqq \A'\  \IFF\    \forall \sigma.\, \forall \M, \M'. \ [\ \ \M\mkpair \M', \sigma \models \A\ \mbox{ implies }\ \M\mkpair \M', \sigma \models \A'\ \ ].$
\end{itemize}
\end{definition}



\begin{lemma}[Assertions are classical-1]
For all runtime configurations $\sigma$,    assertions $\A$ and $\A'$, and modules $\M$  and $\M'$, we have
\begin{enumerate}
\item
$\M\mkpair \M', \sigma \models \A$\ or\ $\M\mkpair \M', \sigma \models \neg\A$
\item
$\M\mkpair \M', \sigma  \models \A \wedge \A'$ \SP if and only if \SP $\M\mkpair \M', \sigma \models \A$ and $\M\mkpair \M', \sigma  \models \A'$
\item
$\M\mkpair \M', \sigma  \models \A \vee \A'$ \SP if and only if \SP $\M\mkpair \M', \sigma  \models \A$ or  $\sigma \models \A'$
\item
$\M\mkpair \M', \sigma  \models \A \wedge \neg\A$ never holds.
\item
$\M\mkpair \M', \sigma  \models \A$ and  $\M\mkpair \M', \sigma  \models \A \rightarrow \A'$  implies
$\M\mkpair \M', \sigma  \models \A '$.
\end{enumerate}
\end{lemma}
\begin{proof} By application of the corresponding definitions from \ref{def:valid:assertion}.\end{proof}.

\begin{lemma}[Assertions are classical-2]
For     assertions $\A$, $\A'$, and $\A''$ the following equivalences hold
\label{lemma:basic_assertions_classical}
\begin{enumerate}
\item
$ \A \wedge\neg \A \ \equiv \  \prg{false}$
\item
$ \A \vee \neg\A   \ \equiv \  \prg{true}$
\item
$ \A \wedge \A'  \ \equiv \  \A' \wedge \A$
\item
$ \A \vee \A'  \ \equiv \  \A' \vee \A$
\item
$(\A \vee \A') \vee \A'' \ \equiv \  \A \vee (\A' \vee\A'')$
\item
$(\A \vee \A') \wedge \A'' \ \equiv \  (\A \wedge \A')\, \vee\, (\A \wedge \A'')$
\item
$(\A \wedge \A') \vee \A'' \ \equiv \  (\A \vee \A')\, \wedge\, (\A \vee \A'')$
\item
$\neg (\A \wedge \A') \  \ \equiv \  \neg  \A   \vee\, \neg \A''$
\item
$\neg (\A \vee \A') \  \ \equiv \  \neg  \A   \wedge\, \neg \A''$
\item
$\neg (\exists \prg{x}.\A )  \  \ \equiv \  \forall \prg{x}.(\neg  \A)$
\item
$\neg (\exists k:\mathbb{N}.\A )  \  \ \equiv \  \forall  k:\mathbb{N}.(\neg  \A)$
\item
$\neg (\exists \prg{fs}:FLD^k.\A )  \  \ \equiv \  \forall \prg{fs}:FLD^k.(\neg  \A)$
\item
$\neg (\forall \prg{x}. \A)  \  \ \equiv \  \  \exists \prg{x}.\neg(\A )$
\item
$\neg (\forall k:\mathbb{N}. \A)  \  \ \equiv \  \  \exists k:\mathbb{N}.\neg(\A )$
\item
$\neg (\forall \prg{fs}:FLD^k. \A)  \  \ \equiv \  \  \exists \prg{fs}:FLD^k.\neg(\A )$
\end{enumerate}
\end{lemma}
\begin{proof}
All points follow by application of the corresponding definitions from \ref{def:valid:assertion}.
 \end{proof}

Notice that satisfaction is not preserved with growing configurations; for example, the assertion $\forall \x. [\ \x : \prg{Purse} \rightarrow \x.\prg{balance}>100\ ]$ may hold in a smaller configuration, but not hold in an extended configuration. Nor is it preserved with configuratio s getting smaller; consider \eg $\exists \x. [\ \x : \prg{Purse} \wedge \x.\prg{balance}>100\ ]$

% \begin{definition}
%We say that $\sigma \vdash \A$ if for any  \x\, is free in $\A$ and any
%  any term $\x.\f_1...\f_n$ appearing in $\A$,
% the interpretation $\interp{\x.\f_1...\f_n} \sigma$ is defined.
%\end{definition}
%
%Note that if we take $n=0$ in the definition above we obtain as corollary that   all variables that appear free in $\A$ they  are in the domain of the top frame in $\sigma$.
%
%\begin{lemma}[Preservation of satisfaction] $ $
%\label{lemma:preserve:valid}
%\begin{itemize}
%\item
%If  $\sigma \vdash \A$ and $\M\mkpair \M',  \sigma \vdash \A$ and   $\sigma' \subconf \sigma$, \  then  \ $\M\mkpair \M',  \sigma' \models \A$.
%\end{itemize}
%\end{lemma}

Finally, we define satisfaction of assertions by modules: A module $\M$ satisfies an assertion $\A$ if for all modules $\M'$, in all configurations arising from executions of $\M\mkpair\M'$, the assertion $\A$ holds.

\begin{definition}
\label{def:module_satisfies}
For any module $\M$, and  assertion $\A$, we define:
\begin{itemize}
\item
$\M \models \A$ \IFF  $\forall \M'.\, \forall \sigma\!\in\!\Arising{\M\mkpair\M'}.\   \M\mkpair\M', \sigma \models \A$
\end{itemize}
\end{definition}

The next sections contain full details. Section~\ref{sect:LangOO} defines a small oo language,  \LangOO, in terms of its  syntax, the structure of its runtime configurations $\sigma$, and its operational semantics in terms of a judgment with   the form $\M \mkpair \M', \sigma \leadsto \sigma'$. Section \ref{sect:assertions} gives the full definition of assertions $\A$,   when assertions are valid in given runtime configurations in terms of a judgment with   the form $\M \mkpair \M', \sigma \models \A$, and finally defines modules' adherence to assertions  in terms of a judgment with   the form $\M \models \A$.




\section{Discussion}
\label{sect:discussion} 

\paragraph{Design choices}

For our underlying language, we have chosen a class based language; we
use classes, because we concentrate on class-based, object-oriented
programming: we could extend our work to prototype-based programming
by creating an (anonynous) class to reify each
prototype \cite{graceClasses}. We expect the ideas will also be
applicable to other kinds of languages (object-oriented or
otherwise).

We have chosen to use a dynamically langauge because many of the
problems we hope to address are written in dynamically typed
languages: web apps and mashups in Javascript; backends in Ruby or
PHP.  We expect that supporting types would make the problem easier,
not harder, but at the cost of significantly increasing the complexity
of the trusted computing base that we assume will run our programs. In
an open world, without some level of assurance (e.g. proof-carrying
code) about the trustworthiness of type information, unfounded
assumptions about types can give rise to new vulnerabilities that
attackers can exploit \cite{pickles}.

Finally, we don't address inheritance. As a specification language,
individual \Chainmail assertions can be combined or reused without any
inheritance mechanism: the semantics are simply that all
the \Chainmail\ assertions are expected to hold at all the points of
execution that they constrain.  \LangOO\ does not contain inheritance
simply because it is not necessary to demonstrate specifications of
robustness: whether an \LangOO class is defined in one place, or
whether it is split into many multiply-inherited superclasses, traits,
default methods in interfaces or protocols, etc.\ is irrelevant,
provided we can model the resulting (flattened) behaviour of such a
composition as a single logical \LangOO\ class.

\paragraph{Contracts and Preconditions}

Traditional specification lanaguges based on pre- and post-conditions
are generally based on design-by-contract assumptions: ``if the
precondition is not satisfied, the routine is not bound to do
anything'' \cite{meyer92dbc} --- that is, the routine can do
anything. Ensuring preconditions is the responsibility of the caller,
and if the caller fails to meet that responsibility it is not the
responsibility of the invoked function to fix the problem \cite{Mey88}.
Underlying this approach, however, is the assumption of a close
system: that all the modules in a system are equally trusted, and so
inserting redundant tests would just make the system more complex and
more buggy. Indeed, \citet{meyer92dbc} states that ``This principle is
the exact opposite of the idea of defensive programming.''

In an open world, however, we do not have the luxury of trusting the
other components with which we interact --- indeed we barely have the
luxury of trusting ourselves.  We must assume that our methods may be
invoked at any time, with any combination of arguments the underlying
platform will support, irrespective of the overall state of the
system, or of the object that recives the method request: in some
sense, this is the very definition of an open system in an open world.
Since methods cannot control when they are invokved, we must work
as if all (potentially) externally-visible methods just have the precondition
\prg{true} --- and we had better be very careful about any assumptions
we make about which method or objects are in fact externally visible
and which are not.  \kjx{likes this sentence: Holistic specifications
directly support robust programming by making those kinds of
assumptions explicit, giving the necessary conditions under which
objects should be accessesd or their methods invoked, and then can
help programmers ensure their programs maintain those conditions.}

%\footnote{Shall we add this, or is it too technical? 
%It is hypothetical, because it holds only under the hypothesis that all functions have been
%specified with \prg{TRUE} as their preconditions. Namely,  a function is at liberty to behave in any way it
%likes when its precondition does not hold; therefore a function with a precondition different than \prg{TRUE}
%could have a behaviour out
%mean: a function is obligated to deliver its postcondtion if its precondtion holds, but is under no obligation whatsoever when the
%precondtion does not hold. Thus, its beahviour would fall outside the ovals.  }


%%% it's a point James would like made somewhre? but where?



\paragraph{Necessary v.s.\ sufficient condtiions.}

%%One might ask whether the necessary conditions are different from
%%the complement of all the sufficient conditions.
%%
%%one might ask indeed. but then one would sound like jacob rees-mogg
Are the necessary conditions the same as the complement of all the
sufficient conditions?  The possible behaviour of a component is the
union of all possible behaviours of each individual function on that
component.  This is shown in the left hand side of
Figure \ref{fig:NecessaryAndSuff}. The all theoretically possible
behaviours are the points in the rectangle.  Each function is a
coloured oval and its possible behaviours are the points in the area
of that oval.  Then, presumably, the necessary conditions are all the
points outside the ovals.

This view is mathematically sound but it is impractical, brittle with
regards to software maintenance, and weak with regards to reasoning in
the open world.


Treating necessary conditions as the complement of the sufficient
conditions is impractical, because it suggests that when interested in
a necessity guarantee a programmer would need to read the
specifications of all the functions in that module.  In view of the
number of these functions, and also the number of behaviours emerging
from their combination, this can be a very large undertaking.  What if
the bank did indeed enforce that only the account owner may withdraw
funds, but had another function which allowed the manager to appoint
an account supervisor, and another which allowed the account
supervisor to assign owners?

This approach is also brittle when it comes to software maintenance,
because it gives no guidance to the team maintaining a piece of
software.  If the necessary conditions are only implicit in the
sufficient conditions, then developers' intentions about those
conditiions are not represented anywhere: there can be no distinction
bewteen a condition that is accidental (if logging is not
implemnented, then it cannot be permitted) and once that is essential
(money can only be transferred by account owners).
Subsequent developers may inadvertenty add functions which break these
intentions, without knowning they've done so.

Finally, this implicit approach is weak when it comes to reasoning
about programs in an open world, because if does not give any
guarantees about objects when they are passed as arguments to calls
into unknown code. For example, what guarantees can we make about the
top of the DOM tree when we pass a wrapper pointing to lower parts of
the tree to an unknown advertisement?


\section{Problem--Driven Design}
\label{sect:problemdriven} 
%proposed replacement for discussion

%\kjx{could be two subsetions or one section}

\subsection{Examplars}

The design of \Chainmail was guided by the study of a sequence of
exemplars taken from the object-capability literature and the smart
contracts world:

\begin{enumerate}
\item \textbf{Bank} \cite{arnd18} - Bank and Account as in
Section~\ref{sect:motivate:Bank} with two different implementations.
\item
\textbf{ERC20} \cite{ERC20} - Ethereum-based token contract.
\item
\textbf{DAO} \cite{Dao,DaoBug} - Ethereum contract for Decentralised Autonomous
Organisation.
\item
\textbf{DOM} \cite{dd,ddd} - Restricting access to browser Domain Object Model\\
\end{enumerate}

Space does not permit us to include any more detail in this paper,
however we present these exemplars as 
appendices \cite{examples}. Our design was also driven by work on other
examples such as the membrane \cite{membranesJavascript},
the Mint/Purse \cite{MillerPhD}, and the Escrow \cite{proxiesECOOP2013,swapsies}.

%% \begin{table}
%%   \begin{tabular}{|l|l|}
%%     \hline
%% \textbf{Bank} \cite{arnd18} & Bank and Account as in
%% Section~\ref{sect:motivate:Bank} with two different implementations.\\
%% \hline
%% \textbf{ERC20} \cite{ERC20} &   Ethereum-based token contract.\\
%% \hline
%% \textbf{DAO} \cite{Dao,DaoBug} & Ethereum contract for Decentralised Autonomous
%% Organisation.\\
%% \hline
%% \textbf{DOM} \cite{dd,ddd} & Restricting access to browser Domain Object Model\\
%%     \hline
%%   \end{tabular}

%%   ~\\
  
%%   \caption{Examplar specifications \cite{examples}}
%%   \label{table}
%% \end{table}





\subsection{Model}


We have constructed a Coq model \cite{examples} of the core of the Chainmail
specification language, along with the underlying \LangOO language.
Our formalism is organised as follows:
\begin{enumerate}
\item
The \LangOO Language: a class based, object oriented language with mutable references.
\item
Chainmail: The full assertion syntax and semantics defined in Definitions \ref{def:execution:internal:external}, \ref{def:arise}, \ref{def:valid:assertion:access}, \ref{def:valid:assertion:control}, \ref{def:valid:assertion:view}, \ref{def:restrict}, \ref{def:valid:assertion:space}, \ref{def:config:adapt}, \ref{def:valid:assertion:time} and \ref{def:module_satisfies}.
\item
\LangOO Properties: Secondary properties of the loo language that aid in reasoning about its semantics.
\item
Chainmail Properties: The core properties defined on the semantics of Chainmail.
\end{enumerate}

We also formalise several of the properties defined in this paper. In the associated appendix (see Appendix \ref{sect:coq}) we list and present the properties of Chainmail we have formalised in Coq.
We have proven that Chainmail obeys much of the properties of classical logic. While we formalise most of the underlying semantics, we make several assumptions in our Coq formalism: (i) the law of the excluded middle,  a property that is well known to be unprovable in constructive logics, and (ii) the equality of variable maps and heaps down to renaming. Coq formalisms often require fairly verbose definitions and proofs of properties involving variable substitution and renaming, and assuming equality down to renaming saves much effort.

More details of the formal foundations of \Chainmail, and the model,
are also in appendices \cite{examples}.



\section{Related Work}
\label{sect:related}
\paragraph{Behavioural Specification Languages} 

Hatcliff et al.\ \cite{behavSurvey2012} provide an excellent survey of
contemporary specification approaches.  With a lineage back to Hoare
logic \cite{Hoare69}, Meyer's Design by Contract \cite{Meyer97} was the
first popular attempt to bring verification techniques to
object-oriented programs as a ``whole cloth'' language design in
Eiffel.  Several more recent specification languages are now making
their way into practical and educational use, including JML
\cite{Leavens-etal07}, Spec$\sharp$ \cite{BarLeiSch05}, Dafny
\cite{dafny} and Whiley \cite{whiley15}. Our approach builds upon
these fundamentals, particularly Leino \& Shulte's
%\kjx{and Naumann's} 
formulation of
two-state invariants \cite{usingHistory}, and Summers and
Drossopoulou's Considerate Reasoning \cite{Considerate}.
%
In general, these approaches assume a closed system, where modules
can be trusted to co{\"o}perate. In this paper we aim to
% illustrate the kinds of techniques required
work
in an open system where modules'
invariants must be protected irrespective of the behaviour of the rest
of the system.

%% \sd{\Chainmail assertions are} guarantees upheld throughout program execution. 
%% Other systems which give such ``permanent'' guarantees are  type systems, 
%% which ensure that well-formed programs  always produce well-formed runtime
%% configurations, or information flow control systems \cite{infoflow}, which ensure that values 
%% classified as high  will not be passed into contexts classified as low. 
%% Such  guarantees %made by types or information flow control
%%  are  practical to check, but   too coarse grained
%% for the purpose of fine-grained,  module-specific specifications. 


%% \Chainmail\ specifications can cross-cut the code they are
%% specifying; \sd{therefore,} they are related to
%% aspect-oriented specification
%% languages such as AspectJML \cite{AspectJML} and AspectLTL
%% \cite{AspectLTL}.
%% %
%% AspectJML is an aspect-oriented extension to JML;
%%  in much the same way that AspectJ is an aspect-oriented extension to
%% Java \cite{AspectJ}.  AspectJML offers AspectJ-style pointcuts 
%% that allow the definition of crosscutting specifications, such as 
%% shared pre- or post-conditions for a range of method calls. 
%% % SD removed the below, as I do not understand it.
%% % These crosscutting specifications can be checked dynamically along with
%% % traditional object-oriented JML assertions. In contrast, \Chainmail\
%% %specifications naturally cross-cut implementation and specification
%% %modules without any special notation, although, lacking wildcards,
%% %\Chainmail\ is not as flexible as AspectJML. 
%% % % SD removed the below, because I do not think it is important
%% %To our knowledge, the
%% %semantics of AspectJML have yet to be defined formally, although
%% %earlier work by Molderez and Janssens describes the formal core of a
%% %similar language \cite{DbCAspectJ}.

%% AspectLTL \cite{AspectLTL} is a specification language based on Linear
%% Temporal Logic (LTL). \sd{It} %AspectLTL 
%% adds cross-cutting aspects to more
%% traditional LTL module specifications: these aspects can further
%% constrain specifications in modules. In that sense, AspectLTL and
%% \Chainmail\ %both 
%% \sd{use} similar implicit join point models, rather than
%% importing AspectJ style explicit pointcuts as in AspectJML.
%% %% % SD removed the below, because I do not think it is important
%% %  AspectLTL
%% %has a formal definition, as does \Chainmail; unlike \Chainmail,
%% %AspectLTL has support for automated reasoning with an efficient
%% %synthesis algorithm.

%% % \paragraph{Concurrent Reasoning} Deny-Guarantee \cite{DenyGuarantee}
%% % distinguishes between assertions guaranteed by a thread, and actions
%% % denied to all other threads. Deny properties correspond to our
%% % requirements that certain properties be preserved by all code linked
%% % to the current module. Compared with our work, deny-guarantee assumes
%% % co{\"o}peration: composition is legal only if  threads adhere  to
%% % their deny properties. In our work, a module has to be robust  and
%% % ensure that these properties cannot be affected by  other code. 


%% %Finally, 
%% \sd{Our} work is also related to the causal obligations in Helm et
%% al.'s behavioural contracts \cite{helm90}. Causal obligations allow
%% programmers to specify e.g.\ that whenever one object receives a
%% message (such as a subject in the Observer pattern having its value
%% changed) that object must send particular messages off to other objects
%% (e.g.\ the subject must notify its observers). \Chainmail's control
%% %SD: not "control flow"
%%  operator % (`$\Calls{\_} {\_} {\_} {\_} $) 
%%  %allows  programmers to make
%%  \sd{supports}  similar specifications, (e.g. 
%%  ${\Calls{\_}  {\prg{setValue}} {\prg{s}} {\prg{v}}}  \rightarrow \Future{\Calls{\prg{s}}{\prg{notify}}{\prg{s.observer}}{\prg{v}}}$ --- when a subject receives a \prg{setValue} method,
%%   it must ``forward'' those messages to the observer.

\paragraph{Defensive Consistency}

%cute but wrong.
%To misparaphrase Tolstoy, secure systems are all alike;
%every insecure system is insecure in its own way
%\cite{WikipediaAnnaKareninaPrinciple}.

In an open world, we cannot rely on the kindness of strangers: rather
we have to ensure our code is correct regardless of whether it
interacts with friends or foes.
Attackers 
\textit{``only have to be lucky once''} while secure systems 
\textit{``have to be lucky always''} \cite{IRAThatcher}.
% SD 
Miller \cite{miller-esop2013,MillerPhD} defines the necessary approach
as \textbf{defensive consistency}: \textit{``An object is defensively
  consistent when it can defend its own invariants and provide correct
  service to its well behaved clients, despite arbitrary or malicious
  misbehaviour by its other clients.''}  Defensively consistent
modules are particularly hard to design, to write, to understand, and
to verify: but
% they have the great advantage that
they make it much
easier to make guarantees about systems composed of multiple components
\cite{Murray10dphil}.


\paragraph{Object Capabilities and Sandboxes.}
{{\em Capabilities} as a means to support the development of concurrent and distributed system  were developed in the 60's
by Dennis and Van Horn \cite{Dennis66}, and were adapted to the
programming languages setting in the 70's \cite{JamesMorris}. 
{\em Object capabilities} were first introduced~\cite{MillerPhD} in the early 2000s},
 and many recent % work attempts to manage
studies manage
to verify  safety or correctness of object capability programs.
Google's Caja \cite{Caja} applies   sandboxes, proxies, and wrappers
 to limit components'
access to \textit{ambient} authority.
% --- that is, capabilities that
%can be obtained from the wider environment, rather than being granted
%to a component explicitly.
Sandboxing has been validated
formally: Maffeis et al.\ \cite{mmt-oakland10} develop a model of
JavaScript, demonstrate that it obeys two principles of
object capability systems
%  (``connectivity begets connectivity'' and
%``no authority amplification''), and then % uses these principles to
and show  how untrusted applications can be prevented from interfering with
the rest of the system. 
Recent programming languages % and web systems
\cite{CapJavaHayesAPLAS17,CapNetSocc17Eide,DOCaT14} including Newspeak
\cite{newspeak17}, Dart \cite{dart15}, Grace \cite{grace,graceClasses}
and Wyvern \cite{wyverncapabilities} have adopted the object
capability model.

%% \paragraph{Verification of Dynamic Languages}
%% A few formal verification frameworks  address JavaScript's highly
%% dynamic, prototype-based semantics. Gardner et al.\ \cite{Gardner12}
%%  developed a formalisation of JavaScript based on separation logic
%% % that they have used
%% and verified   examples. Xiong and Qin et
%% al.\ \cite{XiongPhd,Qin11}  worked on similar lines.
%% % More substantially,
%% Swamy et al.\ \cite{JSDijkstraMonad}  recently
%% developed a mechanised verification technique for JavaScript based on
%% the Dijkstra Monad in the F* programming language.  Finally, Jang et
%% al.\ \cite{Quark} % have %  managed to provide
%% developed a machine-checked proof of
%% five important properties of a web browser --- again similar to our
%% % \prg{any\_code} 
%% invariants --- such as
%% % \textit{``no tab may interfere with
%% %  another tab''} and 
%% \textit{``cookies may not be shared across
%%   domains''} by writing the minimal kernel of the browser in Haskell.
  
%%   \paragraph{JavaScript analyses.}
%% More practically, 
%% Karim et al. apply static analysis on
%% Mozilla's JavaScript Jetpack extension framework \cite{adsafe}, including
%%  pointer analyses. % In a different direction,
%% Bhargavan et al.\ \cite{DefJS}
%% extend language-based sandboxing techniques to support defensive
%% components that can execute successfully  in otherwise untrusted
%% environments.   Politz et
%% al.\ \cite{ADsafety} use a JavaScript type checker to check
%% properties such as
%% % \textit{``widgets cannot obtain direct references
%%  % to DOM nodes''} and
%%  \textit{``multiple widgets on the same page
%%   cannot communicate.''}
%% % --- somewhat similar in spirit to our \textbf{Pol\_4}.
%% Lerner et al.\ extend this system to ensure browser
%% extensions observe \textit{``private mode''} browsing conventions,
%% such as that \textit{``no private browsing history retained''}
%% \cite{Lerner2013b}.  Dimoulas et al.\ \cite{DPCC14} generalise the
%% language and type checker based approach to enforce explicit policies,
%% % although the policies  are restricted to
%% that  describe  which components  may
%% access, or may influence the use of, particular capabilities.
%% Alternatively, Taly et al.\ \cite{secureJS}
%% model  JavaScript APIs in Datalog, and then
%% carry out a Datalog search for an ``attacker'' from the set of all
%% valid API calls. 



\paragraph{Verification of Object Capability Programs}
Murray made the first attempt to formalise defensive consistency and
correctness~\cite{Murray10dphil}.  Murray's model was rooted in
counterfactual causation~\cite{Lewis_73}: an object is defensively
consistent when the addition of untrustworthy clients cannot cause
well-behaved clients to be given incorrect service.  Murray formalised
defensive consistency very abstractly, over models of (concurrent)
object-capability systems in the process algebra CSP~\cite{Hoare:CSP},
without a specification language for describing effects, such as what
it means for an object to provide incorrect service.  Both Miller and
Murray's definitions are intensional, describing what it means for an
object to be defensively consistent.


Dro\-sso\-pou\-lou and Noble \cite{capeFTfJP,capeFTfJP14} have
analysed Miller's Mint and Purse example \cite{MillerPhD} 
% SD Chope details by
% expressing it in Joe-E 
% a Java subset without reflection and static
%fields, 
%and in Grace \cite{capeFTfJP14}, 
and discussed the six
capability policies 
% that characterise the correct behaviour of the
% program, 
as proposed in \cite{MillerPhD}.
%We argued that these policies require a novel
%approach to specification, and showed some first ideas on how to use
%temporal logic.
In %  an unpublished technical report
\cite{WAS-OOPSLA14-TR}, {they} % Drossopoulou and Noble
sketched a  specification language,  \sd{used}  it to  
specify the six policies from \cite{MillerPhD}, % however,
%{their} partial formalisation showed that % they allowed
\sd{showed} that several possible interpretations were possible, %.  They also 
\sd{and} uncovered
the need for another four further policies.
%  and formalised them as well, showing how different implementations of the underlying Mint and Purse
% systems coexist with different policies \cite{capeIFM14},
They also
  sketched how 
a trust-sensitive 
example (the escrow exchange) could be verified in an open world
\cite{swapsies}. 
% In contrast, our work focuses on the semantics of the  \Chainmail\ specification
% language and how it can be used to provide holistic specifications for
% robust programs.
\sd{Their work does not support the concepts of control, time, or space, as in \Chainmail,
but it offers a primitive expressing trust.}
 
Devriese et al.\ \cite{dd}  have deployed
   \sd{powerful} %rather more complex
  theoretical techniques to address similar problems:  % Devrise et al.\ 
  \sd{They} show how step-indexing, Kripke worlds, and representing objects
as state machines with public and private transitions can be used to
reason about % object-oriented programs in general.
\sd{object capabilities}.
Devriese have demonstrated solutions to a range of exemplar problems,
including the DOM wrapper (replicated in our
Section~\ref{sect:example:DOM}) and a mashup application.
% Although the formal techniques are much more sophisticated than we
%apply here, and consequently 
% not true can e.g.\ reason about recursion where we
%cannot, there are some similarities, e.g.\ with the 
\sd{Their} distinction
between public and private transitions %being related 
\sd{is similar} to the
distinction between internal and external objects.

More recently, Swasey et al.\ \cite{ddd}  designed OCPL, a logic
for object capability patterns, that supports specifications and
proofs for object-oriented systems in an open world.  
% The key idea here is to 
\sd{They} % say it simpler
draw on verification techniques for security and
information flow: separating internal implementations (``high values''
which must not be exposed to attacking code) from interface objects
(``low values'' which may be exposed).  OCPL supports defensive
consistency % (Swasey et al.\ use 
(\sd{they} use the term ``robust safety'' from the
security community \cite{Bengtson}) via a proof system that ensures
low values can never leak high values to external attackers. 
%\susan{How does this imply that high values can be exposed?}
%\james{typo fixed: it's LOW values that can be exposed}
This means that low values \textit{can} be exposed to external code,
and the behaviour of the system is described by considering attacks only
on low values.  %OCPL is a program logic, and Swasey
\sd{They} use that logic to
prove a number of object-capability patterns, including
sealer/unsealer pairs, the caretaker, and a general membrane.

Schaefer et al.\ \cite{schaeferCbC} have recently
% taken a similar approach to Swasey,
% adding support for
\sd{added}  support for information-flow security % in a setting 
\sd{using} refinement to ensure correctness (in this case confidentiality) by
construction. 
% Although designed to support
% confidentialty, it seems likely that the information-flow guarantees
% could also be used to ensure robustness.  
By enforcing encapsulation, \sd{all} % used to say both
these approaches share similarity with techniques such as
ownership types \cite{ownalias,NobPotVitECOOP98}, which also
protect internal implementation objects from accesses that cross
encapsulation boundaries.  Banerjee and Naumann demonstrated that by
ensuring confinement, ownership
systems can enforce representation independence (a property close to
``robust safety'') some time ago \cite{Banerjee:2005}.

 
\Chainmail\ differs from Swasey, Schaefer's, and Devriese's work in a number of ways:
% \citet{ddd} and \citet{schaeferCbC} 
\sd{They} are primarily concerned \sd{with} %about
mechanisms that ensure encapsulation (aka 
confinement) while we abstract away from any mechanism via the
$\External{}$ predicate. 
\sd{They use powerful mathematical techniques
% , such as Kripke worlds and step-indexing 
which  the users need  to understand in order to write their specifications,
while \Chainmail users only need  to understand  first order logic and 
the holistic operators presented in this paper.}
% While \Chainmail's $\Using{}{}$ is related to Banerjee
% and Naumann's region sets, the assertion languages here are mostly
% traditional (extensions of) Hoare logics --- Swasey et al.\ build on a
%concurrent separation logic. 
\sd{ Finally, none of these systems offer the kinds of
holistic assertions addressing control flow, change, or temporal
operations that are at the core of \Chainmail's approach.
}

Scilla \cite{scillaOOPSLA19} is a minimalistic typed functional
language for writing smart contracts that compiles to the Ethereum
bytecode. Scilla's semantic model is restricted, assuming actor based
communication and restricting recursion,  thus facilitating static
analysis of Scilla contracts and ensuring termination.
Scilla is able to demonstrate that a number of popular Ethereum
contracts avoid type errors, out-of-gas resource failures, and
preservation of virtual currency. 
Scilla's semantics are defined formally, but have not yet been represented in a
mechanised model.

%% \kjx{NPChecker \cite{NPcheckerOOPSLA19} analyses Ethereum smart
%% contracts to detect bugs related to nondeterministic
%% execution. NPChecker undertakes an information flow
%% analysis to detect potential read-write hazards
%% particularly reentrancy and ordering dependencies.
%% \textbf{We don't do concurrency. Do we need this one? I don't think so}
%% }


Finally, the recent VerX tool is able to verify a range of
specifications for solidity contracts automatically \cite{VerX}.
Similar to \Chainmail, VerX has a specification language based on
temporal logic.  VerX offers three temporal operators (always, once,
prev) but only within a past modality, while \Chainmail\ has two
temporal operators, both existential, but with both past and future
modalities.   VerX specifications can also include predicates th	at
model the current invocation on a contract (similar to \Chainmail's
``calls''), can access variables, and compute sums (only) over
collections. \Chainmail\ is strictly more expressive as a
specification language, including quantification over objects and sets
(so can compute arbitrary reductions on collections) and of course
specifications for permission (``access''), space (``in'') and
viewpoint (``external'') which have no analogues in VerX. 
Unlike \Chainmail, VerX includes a practical tool that has
been used to verify   a hundred properties across case studies of
twelve Solidity contracts.
%\textbf{(ideally also say something about proof status)}}

\jm{
\paragraph{Incorrectness Logic.} O'Hearn~\cite{IncorrectnessLogic} defined a Hoare
Logic for modelling program incorrectness. O'Hearn's Incorrectness Logic
is based on a Reverse Hoare Logic \cite{reverseHoare}, and empowers programmers to 
specify preconditions under which specific errors and program states may result. 
Incorrectness Logic provides a sound and compositional way to reason about 
the presence of bugs rather than the absence of bugs. 
As with Hoare logic, Incorrectness Logic provides a system
for reasoning about sufficent conditions for post-conditions to hold.
However, where Hoare logic specifies the shape of the result of execution 
of all program states that satisfy the precondition, Incorrectness Logic
specifies that all states that satisfy the postcondition are reachable
from those that satisfy the precondition. This suits the specification
of program errors, as it allows for the exclusion of false negatives.
In comparison, \Chainmail, as with Hoare Logic, is concerned with correctness
(as seen in the exemplars of Section \ref{sect:problemdriven}). 
Extending the comparison, \Chainmail differs from both Hoare Logic and Incorrectness,
in the ability to specify, not just sufficient conditions, but necessary conditions for 
reaching certain program states. Neither Incorrectness Logic, 
nor Hoare Logic allows for such specifications.
}

% SD chopped as did not like
%As with Swasey et al.\ this work does not provide a holistic
%assertion language like \Chainmail.
% SD Chopped, as it sounds as if their is not real code, which is debatable
% and what is an extensional framework? they would say that theirs is too.
%In contrast, \Chainmail\ is
%meant for describing and reasoning about real code, and we provide an
%expressive, extensional framework for evaluating defensive consistency
%in actual open systems.
%


%%%%%%%%%%%%%%%%%%%%%%%%%%%%%%%%%%%%%%%%%%%%%%%%%%%%%%%%%%%%
%%NOTES:
%% the other thing this section needs to do, particularly with Devrise, is to lay out precisely the way our work is more limited than theirs:
%% (Swasey, I'm more and more convinced, is just ownership-via-a-proof-system) 
%% we don't step-index, don't have logical relations, etc: what do we lose by NOT having those things
%% (or what do we gain by having those things...

%% The "deep" comparison with Swasey and with Devirese (and also
%% information flow control and temporal logics) needs to say why these
%% works are not as good (expressive? easy to understand?) as ours.
%% Currently the Related work just mentions them, but does not answer the
%% question as to why our work is important when theirs already has been
%% published.




%% *Difference between Spec Languages and Chainmail*  One way to tackle
%%  this would be to enumerate which elements of Chainmail appear at
%%  other works, which do not, and claim that Chainmail’s novelty is the
%%  good combination of these elements


%% Eg: reflection about contents of stack and heap (in classical Hoare
%% Logics), two state assertions (JML etc), invariants (Hoare and Meyer),
%% internal/external (Liskov?, Noble et al,modules in Neumann and also
%% O’Hearn), time (temporal logic, but they do not have the other stuff),
%% Control (none?), Space (in Sep. logic, and in effects, buyt the
%% meaning is different), Permissions (our earlier work, and less
%% flexible approaches such as owenrship types and perhaps also
%% oinformation flow control), Authority (effect systems and modifies
%% clauses, and perhaps also Bierman&Parkison abstract predicates, but
%% there it is tied to pre-post conditions.


%% Also, point out difference between our invariants and Hoare
%% triples. Subtle and needs thinking







%%%%%%%%%%%%%%%%%%%%%%%%%%%%%%%%%%%%%%%%%%%%%%%%%%%%%%%%%%%%

%% Neither effort addresses specification languages for security and
%% robustness, provides Hoare logics to reason about object-capability
%% programs.

%% , model protocols that dynamically ascribe trust
%% \cite{swapsies,lefthand} or quantify the damage an untrustworthy
%% object can do.






% \kjx{History-Based Specification and Verification of Scalable
%  Concurrent and Distributed Systems --- ICFEM15}


% \paragraph{Specifying Design Patterns}

% Techniques for specifying Design Patterns go back at least to 
% Helm's contracts \cite{Helm92}.

% more importantly: work on formalisation of design patterns.
% (again look at JC grant, even if refs are 5 years old)
% let's be shameless here...



% This search is similar to the quantification over
% potential code snippets in our model.
% The problem posed by the Escrow example is that it establishes a two-way
% dependency between trusted and untrusted systems --- precisely the
% kind of dependencies these techniques prevent.

% %These approaches are all based on static analyses.
%  The WebSand
% \cite{flowcaps11,sabelfeld-inlining2012} and Jeeves \cite{jeeves2012}
% projects use dynamic techniques to monitor safe execution of information flow policies.
%  Richards et al.\ \cite{FlacJS}   extended this approach by
% incorporating explicit dynamic ownership of objects (and thus of
% capabilities) and policies that may examine the history of objects'
% computations. While these dynamic techniques can restrict or terminate
% the execution of a component that breaches its security policies, they
% cannot guarantee in advance that such violations can never happen.
% While information flow policies are concerned with the flow of objects (and thus also capabilities)
% across the program code, our work is more concerned with the identification of the objects which protect
% the services.

%Compared with all these approaches, our work   focuses on
%\textit{general} techniques for specifying (and ultimately verifying)
%capability policies, whereas these systems are generally much more
%\textit{specific}: focusing on one (or a small number) of actual
%policies. % This seems to be because contemporary object capability
%programming is primarily carried out in JavaScript, but
% There are few

 
% \paragraph{Relational models of trust.}
% Artz and Gil \cite{artz-trust-survey-2007} survey various
% types of trust in computer science generally, although trust has also
% been studied in specific settings, 
% %
% ranging from peer-to-peer systems \cite{aberer-trust-p2p-2001} and
% cloud computing \cite{habib-trust-cloud-2011} 
% to mobile ad-hoc networks \cite{cho-trust-survey-adhocnets-2011}, 
% the internet of things \cite{lize-trust-IoT-2014}, 
% online dating \cite{norcie-trust-online-dating},
% and as a component of a wider socio-technical system
% \cite{cho-trust-sustainable-2013,walter-trust-cloud-ecis2013}. 
% %
% Considering trust (and risk) in systems design, Cahill et al.'s overview
% of the \textsc{Secure} project \cite{cahill-trust-pervasive-2003}
% gives a good introduction to both theoretical and practical issues of
% risk and trust, including a qualitative analysis of an e-purse
% example. This project builds on Carbone's trust model
% \cite{carbone-formal-trust-2003} which offers a core semantic model of
% trust based on intervals to capture both trust and uncertainty in that
% trust. Earlier Abdul-Rahman proposed using separate relations for
% trust and recommendation in distributed systems
% \cite{abdul-rahman-distributed-trust-1998}, more recently Huang and
% Nicol preset a first-order formalisation that makes the same
% distinction \cite{huang-formal-semantics-trust-calculus-2010}.
% Solhaug and St{\o}len \cite{solhaug-trust-uncertainty-2011} 
% consider how risk and trust are related to uncertainties over
% actual outcomes versus knowledge of outcomes.
% Compared with our work, these approaches produce models of trust
% relationships between high-level system components 
% (typically treating risk as uncertainty in trust) 
% but do not link those relations to the system's code. 



% \paragraph{Logical models of trust.}
% \sd{A detailed study of how web-users decide whether to trust appears in \cite{GilArtz}.}
% \sd{Starting with \cite{Lampson92},} various different logics have been used to measure trust in different
% kinds of systems.
% Murray and Lowe \cite{murray10-infoflow} model object capability
% programs in CSP, and use a model checker to ensure program executions
% do not leak authority.
% Carbone et al.\ \cite{carbone-formal-trust-2011}
% use linear temporal logic to model specific trust relationships in service
% oriented architectures.
% Ries et al.\ \cite{habib-trust-CertainLogic-2011} evaluate trust under
% uncertainty by evaluating Boolean expressions in terms of real values
% for average rating, certainty, and initial expectation.
% % Perhaps closer to our work, Aldini
% Aldini \cite{aldini-calculus-trust-IFIPTM2014} describes a temporal logic for
% trust that supports model checking to verify some trust properties.
% Primiero and Taddeo \cite{primiero-modal-theory-trust-2011} have
% developed a modal type theory that treats trust as a second-order
% relation over base relations between
% counterparties. Merro and Sibilio
% \cite{merro-calculus-trust-adhoc-facs2011} developed a trust model for
% a process calculus based on labelled transition systems.
% Compared with our proposal, these approaches use
% process calculi or other abstract logical models of systems, rather
% than engaging directly with the system's code.






%%%% %%%% %%%% %%%% %%%% %%%% %%%% %%%% %%%% %%%% %%%% %%%% %%%% %%%% 
%%%% %%%% %%%% %%%% %%%% %%%% %%%% %%%% %%%% %%%% %%%% %%%% %%%% %%%% 







\section{Conclusions}
\label{sect:conclusion}
%Susan:Please read first bit as I have just written it
%\se{When you write a module that is to be used with other code, the last thing you want to happen is that some other code uses it to cause effects that you never intended. Our specification language \Chainmail has been designed, so that developers whose modules are going to be used in the wild, have the language to constrain the usage of their code. In addition to classical function by function specification techniques, we have shown that a holistic or whole program approach is needed to make open world code robust. We have shown} 
% going to the old one, as running out of space.
% also, the new one brings new words, and I think all th words in concluson should have appeared earlier
In this paper we have motivated the need for holistic specifications,
presented the \Chainmail specification language for writing such
specifications, and shown 
how \Chainmail can be used to give holistic
specifications of key exemplar problems: the bank account,  the
wrapped DOM, the ERC20, and and the DAO.

To focus on the key attributes of a holistic specification language,
% we have tried to keep the
\sd{we have kept  \Chainmail simple, only requiring an understanding of first order logic.}
\sd{We believe that the holistic features (permission, control, time, space and viewpoint),
are intuitive concepts %for ptogrammers. 
when reasoning informally, and were pleased to have been able to provide their
formal semantics in what  we  argue is a simple manner.}
% below not true, we do have recusrions  
%do not even support recursive procedures to avoid circularities in the
%metatheory, let alone concurrency, exceptions, distribution,
%networking, etc. 

\sd{The development of the semantics of \Chainmail assertions posed several interesting 
challenges, \eg the treatment of the open world requires two-module execution
and the concept of external objects,
recursion is confined to ghostfields and assertions require termination of included expressions,
space required the concept of restricting runtime configurations,
and time required adaptation operators which apply bindings from one configuration to another.}  

\sd{\Chainmail is powerful enough to express many key examples from the
literature; nevertheless, it lacks several important features: It provides 
recursion  only in a restricted form, it has a rather inflexible notion of
module and does not support hierarchies of modules, and knows nothing about
concurrency or distribution.  We plan to remove these restrictions by applying
techniques such as step-indexing \cite{stepindex}, but hope to keep any mathematical 
sophitsication in the
model of \Chainmail without exposing it to the person who writes the specification.  We are also
 interested in extending \Chainmail\ to situations
where internal modules are typed, but the external modules are untyped.
%
We also plan to extend \Chainmail to support reasoning about
conditional trust in programs, and to quantify the risks involved in
interacting with untrustworthy software \cite{swapsies}.
}

\sd{To make these kinds of specifications
practically useful,  we plan to develop logics for proving adherence of module's code to holistic specs, as well
as logics for using holistic specs in the proof of open programs. We want to develop 
dynamic monitoring  and model checking techniques for our specifications. 
And finally, we plan to automate reasoning with these logics.}


\section{Acknowledgments}

\sophia{This work is based on a long-standing collaboration with Mark
  S.\ Miller and Toby Murray.
We have received invaluable feedback from Alex Summers, Bart Jacobs,  Chris Hawblitzel,
Michael Jackson, Lucius G. Meredith,
Mike Stay, Shuh Peng Loh,  Emil Klasan, members of WG 2.3, 
and the FASE 2020  reviewers.
The work has been supported by the 
Royal Society of New Zealand (Te Ap\={a}rangi) Marsden Fund (Te P\={u}tea Rangahau a Marsden)
grants VUW-1318 and VUW-1815, and research gifts from Agoric\susan{, the Ethereum Foundation,} and
Facebook.}

\bibliographystyle{ACM-Reference-Format}
\bibliography{Case,more}




\vfill

{\small\medskip\noindent{\bf Open Access} This chapter is licensed under the terms of the Creative Commons\break Attribution 4.0 International License (\url{http://creativecommons.org/licenses/by/4.0/}), which permits use, sharing, adaptation, distribution and reproduction in any medium or format, as long as you give appropriate credit to the original author(s) and the source, provide a link to the Creative Commons license and indicate if changes were made.}

{\small \spaceskip .28em plus .1em minus .1em The images or other third party material in this chapter are included in the chapter's Creative Commons license, unless indicated otherwise in a credit line to the material.~If material is not included in the chapter's Creative Commons license and your intended\break use is not permitted by statutory regulation or exceeds the permitted use, you will need to obtain permission directly from the copyright holder.}

\medskip\noindent\includegraphics{cc_by_4-0.eps}

\newpage
\appendix

\section{Programming Language Formalism}
\label{app:LangOO}
%\section{The language \LangOO}
%\label{sect:LangOO}
\subsection{Modules and Classes}
\label{secONE}

\LangOO programs consist of modules, which are repositories of code. Since we study class based oo languages,
in this work, code is represented as classes, and  modules are mappings from  identifiers to class  descriptions.

\begin{definition}[Modules]
\label{defONE}
We define $\syntax{Module}$ as  the set of mappings from identifiers to class descriptions (the latter defined in Definition \ref{def:syntax:classes}):\\  % to force line break

\begin{tabular}  {@{}l@{\,}c@{\,}ll}
\syntax{Module} \ \  &  \   $\triangleq $  \ &
   $ \{ \ \ \M \ \ \mid \ \  \M: \ \prg{Identifier} \   \longrightarrow \
  \  \syntax{ClassDescr}     \  \    \}$
 \end{tabular}
\end{definition}
 
Classes, as defined   below,
consist of field, method definitions and ghost field declarations.
 \LangOO is untyped, and therefore fields are declared without types, 
 method signatures and ghost field signatures consist of  sequences of parameters without types, and no return type.
 Method bodies consist of sequences of statements;
these can be field read or field assignments, object creation, method calls, and return statements.
All else, \eg booleans, conditionals, loops,  can be encoded.
Field read or write is only allowed \sd{if the object whose field is being read 
belongs to the same class as the current method. This is enforced by the operational semantics, \cf
Fig.  \ref{fig:Execution}.}
\sd{Ghost fields  are defined as implicit, side-effect-free functions with zero or more parameters. They are ghost information, \ie 
they are not directly stored in the objects, and are not read/written during execution. When such a ghostfield is
mentioned in an assertion, the corresponding function is evaluated. More in XXX. \sophia{TODO}.
Note that the expressions that make up the bodies of ghostfield declarations (\prg{e}) are more complex than the terms that 
appear in individual statements.}

From now on we expect that the set of field and the set of ghostfields defined in a class are disjoint.  

\label{sec:syntax:classes}


\begin{definition}[Classes]
\label{def:syntax:classes}
Class descriptions consist of field declarations, method declarations, and ghostfield  declarations.
 
\begin{tabular}{lcll}
 \syntax{ClassDescr}   &   \BBC  &     \kwN{class}  \syntax{ClassId}    \lb\,  $($\ \ \syntax{FieldDecl} $)^*$ \
 $($  \syntax{MethDecl}\ $)^*$   \   $($   \syntax{GhosDecl}\ $)^*$ \ \ \rb
\\
\syntax{FieldDecl} &\BBC& \kwN{field} \f \\
\syntax{MethDecl} &\BBC&
     \kwN{method}\    \m\lp \x$^*$\rp     \lb\, \syntax{Stmts}  \,
    \rb
 \\
 \syntax{Stmts}  &\BBC&  \syntax{Stmt}     ~\SOR~  \syntax{Stmt} \semi \syntax{Stmts} \\
\syntax{Stmt}    &\BBC&
      \x.\f {\kw{:=}} \x   ~\SOR~  \x{\kw{:=}}  \x.\f    ~\SOR~        \x  {\kw{:=}} \x.\m\lp \x$^*$\rp     ~\SOR~     \x  {\kw{:=}}     \newKW\, \c\,\lp \x$^*$\rp   ~\SOR~
   \returnKW \,  \x   \\
  \syntax{GhostDecl} &\BBC&  \kwN{ghost} \f\lp \ \x$^*$\ \rp \lb \  \SE\ \rb\\
 \SE  &\BBC&    \kwN{true}   ~\SOR~  \kwN{false}   ~\SOR~  \kwN{null}  ~\SOR~  \x  \   ~\SOR~  
     \   \SE=\SE    ~\SOR~ \kwN{if}\, \SE\,   \kwN{then}\,  \SE\,    \kwN{else}\, \SE    ~\SOR~  \SE.\f\lp\ \SE$^*$ \ \rp\\
 \x, \f, \m &\BBC&  \prg{Identifier} 
 \end{tabular}

  \vspace{.03in}
  \noindent
 where we use metavariables as follows:
 $\x \in  \syntax{VarId} \ \ \  \f \in  \syntax{FldId} \ \ \  \m \in  \syntax{MethId} \ \ \  \c \in  \syntax{ClassId}$, and  \x\ includes \this
\end{definition}


We define a method lookup function, $\mathcal{M}$ which returns the corresponding method definition given a class \c\ and a method identifier \m, and similarly a ghostfield lookup function, $\mathcal{G}$ 

 \begin{definition}[Lookup] For a class identifier \prg{C}  and a method identifier \prg{m}$:$  $ ~ $ \\
\label{def:lookup}
\noindent
$
\Meths {} {\prg{C}} {m}       \triangleq  \ \left\{
\begin{array}{l}
                        \m\, \lp \p_1, ... \p_n \rp \lb Stmts   \rb\\
\hspace{0.5in} \mbox{if}\  \M(\prg{C}) =   \kwN{class}\, \prg{C}\, \  \lb ...   \kwN{method}\, \m\, \lp \p_1, ... \p_n \rp \lb Stmts  \rb  ... \ \rb.
\\
\mbox{undefined},  \ \ \ \mbox{otherwise.}
\end{array}
                    \right.$
\\
$
{\mathcal G} (\M, {\prg{C}}, {\f})    \ \   \triangleq  \ \left\{
\begin{array}{l}
                        \f\, \lp \p_1, ... \p_n \rp \lb \prg{e}  \rb\\
\hspace{0.5in} \mbox{if}\  \M(\prg{C}) =   \kwN{class}\, \prg{C}\, \  \lb ...   \kwN{ghost}\,  \m\, \lp \p_1, ... \p_n \rp \lb \prg{e}  \rb  ... \ \rb.
\\
\mbox{undefined},  \ \ \ \mbox{otherwise.}
\end{array}
                    \right.$
  \end{definition}

\subsection{The Operational Semantics of \LangOO}
\label{formal:semantics}

We will now define execution of \LangOO code.
We start by  defining the  runtime entities, and runtime configurations, $\sigma$, which consist of heaps and stacks of frames.
 The frames are pairs consisting of a continuation, and a mapping from identifiers to values.
The continuation represents the code to be executed next, and the mapping gives meaning
to the formal and local parameters.

\begin{definition}[Runtime Entities]
We define  addresses, values, frames, stacks, heaps and runtime configurations.

\begin{itemize}
\item
We take addresses to be an  enumerable set,  \prg{Addr}, and use the identifier $\alpha\in \prg{Addr}$ to indicate an address.
\item
Values, $v$, are either addresses, or sets of addresses or null:\\
 $~ ~ ~ \ v \in \{ \prg{null} \} \cup \prg{Addr}\cup {\mathcal P}( \prg{Addr})$.
\item
  Continuations are either   statements  (as defined in Definition~\ref{def:syntax:classes}) or a marker, \x {\kw{:=}} $\bullet$, for a nested call followed by
  statements to be executed
  once the call returns.


\begin{tabular}{lcll}
\syntax{Continuation} &\BBC&   \syntax{Stmts} ~\SOR~   \x {\kw{:=}} $\bullet$ \semi\ \syntax{Stmts} \\
 \end{tabular}

\item
Frames, $\phi$, consist of a code stub  and a  mapping from identifiers to values:\\  $~ ~ ~ \ \phi \ \in\ \syntax{CodeStub} \times \prg{Ident} \rightarrow Value$,
\item
Stacks,  $\psi$, are sequences of frames, $\psi\ ::=   \phi \ | \ \phi\cdot\psi$.
\item
Objects consist of a class identifier, and a partial mapping from field identifier to values: \\  \ $~ ~ ~ \ Object\ = \ \prg{ClassID} \times (\prg{FieldId} \rightarrow Value)$.
\item
Heaps, $\chi$, are mappings from addresses to objects:\  \  $\chi\ \in\ \prg{Addr} \rightarrow Object$.
\item
Runtime configurations, $\sigma$, are pairs of stacks and heaps, $\sigma\ ::=\ (\ \psi, \chi\ )$.
\end{itemize}

\end{definition}


Note that values may be sets of addresses. Such values are never part of the execution of 
\LangOO, but are used to give semantics to assertions -- we shall see that in Definition \ref{def:valid:assertion}.



Next, we define the interpretation of variables (\x) and   field look up  (\x.\f)
in the context of frames,
heaps and runtime configurations; these interpretations are used to define the operational semantics and  also  the
validity of assertions, later on in Definition \ref{def:valid:assertion}:

\begin{definition}[Interpretations]
\label{def:interp}
We first define lookup of fields and classes, where $\alpha$ is an address, and \f\, is a field identifier:
\begin{itemize}
\item
$\chi ({\alpha},{\f})$ $\triangleq$  $\fldMap({\alpha},{\f})$\ \ \ if \ \ $\chi(\alpha)=(\_, \fldMap)$.
\item
$\ClassOf {\alpha} {\chi} $ $\triangleq$ $\c$\  \ \ if \ \ $\chi(\alpha)=(\c,\_)$
\end{itemize}

\noindent
We now define interpretations  as follows:

\begin{itemize}
\item
$\interp {\x}{\phi} $ $\triangleq$ $\phi(\x)$
\item
$\interp {\x.\f}{(\phi,\chi)} $ $\triangleq$ $v$, \ \ \ if \ \ $\chi(\phi(\x))=(\_, \fldMap)$ and $\fldMap(\f)$=$v$

\end{itemize}

\noindent
For ease of notation, we also use the shorthands below:
\begin{itemize}
\item
$\interp {\x}{(\phi\cdot\psi,\chi)} $ $\triangleq$ $\interp {\x}{\phi} $
\item
$\interp {\x.\f}{(\phi\cdot\psi,\chi)} $ $\triangleq$ $\interp  {\x.\f}{(\phi,\chi)} $
\item
$\ClassOf {\alpha} {(\psi,\chi)} $ $\triangleq$ $\ClassOf {\alpha} {\chi} $
\item
$\ClassOf {\x} {\sigma} $ $\triangleq$ $\ClassOf {\interp {\x}{\sigma}} {\sigma} $
\end{itemize}

\end{definition}

In the definition of the operational semantics of \LangOO we use the following notations for lookup and updates of runtime entities :

\begin{definition}[Lookup and update of runtime configurations]
We define convenient shorthands for looking up in  runtime entities.
\begin{itemize}
\item
Assuming that $\phi$ is the tuple  $(\prg{stub}, varMap)$, we use the notation  $\phi.\prg{contn}$ to obtain \prg{stub}.
\item
Assuming a value v, and that $\phi$ is the tuple  $(\prg{stub}, varMap)$, we define $\phi[\prg{contn}\mapsto\prg{stub'}]$ for updating the stub, \ie
$(\prg{stub'}, varMap)$.   We use  $\phi[\x \mapsto v]$  for updating the variable map, \ie  $(\prg{stub}, varMap[\x \mapsto v])$.
\item
Assuming a heap $\chi$, a value $v$, and   that $\chi(\alpha)=(\c, fieldMap)$,
we use $\chi[\alpha,\f \mapsto v]$ as a shorthand for updating the object, \ie $\chi[\alpha \mapsto (\c, fieldMap[\f \mapsto v]]$.
\end{itemize}

\end{definition}



\begin{figure*}
$\begin{array}{l}
\inferenceruleNN {methCall\_OS} {
\\
\phi.\prg{contn}\ =\ \x {\kw{:= }} \x_0.\m \lp \x_1, ... \x_n \rp \semi \prg{Stmts}
\\
\interp{\x_0}{\phi} = \alpha
\\
\Meths {} {\ClassOf {\alpha} {\chi}} {\m} \  =  \ \m\lp \p_1, \ldots \p_n \rp \lb \prg{Stmts}_1   \rb
  \\
 \phi''\ =\  (\  \prg{Stmts}_1,\ \ (\ \this \mapsto \alpha,
  \p_1 \mapsto  \interp{\x_1}{\phi}, \ldots \p_n \mapsto  \interp{\x_n}{\phi}\ ) \ )
}
{
 \M,\, (\ \phi\cdot\psi,\ \chi\ )\ \ \leadsto\  \ (\ \phi''\cdot\phi[\prg{contn}\mapsto\x  \kw{:=} \bullet \semi \prg{Stmts}] \cdot\psi,\ \chi\ )
}

\\ \\
\inferenceruleNN {varAssgn\_OS} {
 \phi.\prg{contn} \ = \ \x  {\kw{:= }}  \y.\f \ \semi \prg{Stmts}\ \hspace{2cm} \ClassOf {\y} {\sigma} =\ClassOf {\this} {\sigma}
}
{
 \M,\,  (\ \phi\cdot\psi, \chi\ )\ \ \leadsto\  \ (\ \phi[ \prg{contn} \mapsto \prg{Stmts}, \x\mapsto \interp{\y.\f}{\phi,\chi}] \cdot\psi,\ \chi\  )
}
\\
\\
\inferenceruleNN{fieldAssgn\_OS} {
 \phi.\prg{contn}\ =\  \x.\f  \kw{:=} \y  \semi \prg{Stmts} \hspace{2cm} \ClassOf {\x} {\sigma} =\ClassOf {\this} {\sigma}
}
{
 \M,\,  (\ \phi\cdot\psi, \chi\  )\ \ \leadsto\  \ (\ \phi[\prg{contn}\mapsto  \prg{Stmts} ] \cdot\psi, \chi[\interp{\x}{\phi},\f \mapsto \interp{\y}{\phi,\chi}]\  )
}
\\
\\
\inferenceruleNN {objCreate\_OS} {
 \phi.\prg{contn}\ =\  \x  \kw{:=} \kwN{new }\, \c \lp \x_1, ... \x_n \rp  \semi \prg{Stmts}
 \\
 \alpha\ \mbox{new in}\ \chi
 \\
\f_1, .. \f_n\ \mbox{are the fields declared in } \M(\c)
}
{
 \M,\,  (\ \phi\cdot\psi, \chi\ )\ \ \leadsto\  \ (\ \phi[\prg{contn}\mapsto  \prg{Stmts},\x \mapsto \alpha\ ] \cdot\psi, \ \chi[\alpha \mapsto (\c,\f_1 \mapsto \interp{\x_1}{\phi},  ... \f_n \mapsto \interp{\x_n}{\phi}  ) ]\ )
}
\\
\\
\inferenceruleNN {return\_OS} {
 \phi.\prg{contn}\ =\   {\kwN{return }}\, \x  \semi \prg{Stmts}\ \  \ or\  \ \  \phi.\prg{contn}\ =\   {\kwN{return}}\, \x
 \\
\phi'.\prg{contn}\ =\  \x' \kw{:=} \bullet  \semi \prg{Stmts}'
}
{
 \M,\,  (\ \phi\cdot\phi'\cdot\psi, \chi\ )\ \ \leadsto\  \ (\ \phi'[\prg{contn}\mapsto  \prg{Stmts'},\x' \mapsto \interp{\x}{\phi}] \cdot\psi, \ \chi \ )
}
\end{array}
$
\caption{Operational Semantics}
\label{fig:Execution}
\end{figure*}

Execution of a statement has the form $\M, \sigma \leadsto \sigma'$, and is defined in figure \ref{fig:Execution}.

\begin{definition}[Execution] of one or more steps is defined as follows:

\begin{itemize}
     \item
   The relation $\M, \sigma \leadsto \sigma'$, it is defined in Figure~\ref{fig:Execution}.

   \item
   $\M, \sigma \leadsto^* \sigma'$ holds, if i) $\sigma$=$\sigma'$, or ii) there exists a $\sigma''$ such that
   $\M, \sigma \leadsto^* \sigma''$ and $\M, \sigma'' \leadsto \sigma'$.
 \end{itemize}

\end{definition}
 
%\begin{figure*}
%{$\begin{array}{l}
%\begin{array}{llll}
%\inferenceruleN {True\_Val} {
%}
%{
% \M,\, \sigma, \kwN{true} \ \hookrightarrow_\emptyset\  \kwN{true}
%}
%& 
%\inferenceruleN {False\_Val} {
%}
%{
% \M,\, \sigma, \kwN{false} \ \hookrightarrow_\emptyset\  \kwN{false}
%}
%&
%\inferenceruleN  {Null\_Val} {
%}
%{
% \M,\, \sigma, \kwN{null} \ \hookrightarrow_\emptyset\  \kwN{null}
%}
%&
%\inferenceruleN {Var\_Val} {
%}
%{
% \M,\,  \sigma, \x \ \hookrightarrow_\emptyset\   \sigma({\x})
%}
%\end{array}
%\\ \\
%\begin{array}{lll}
%\inferenceruleNM{Field\_Heap\_Val} {
%~ \\
%~ \\
%~ \\
%~ \\
% \M,\,  \sigma, \SE \ \hookrightarrow_\SAF\   \alpha \\
% \sigma(\alpha,\f)=v
%}
%{
% \M,\, \sigma, \SE.\f \lp \rp \ \hookrightarrow_\SAF\   v
%}
%& &
%\inferenceruleNM{Field\_Ghost\_Val}
%{
% \M,\, \sigma, \SE_0   \ \hookrightarrow_{\SAF_0}\  \alpha
%\\
% \M,\, \sigma, \SE_i  \ \hookrightarrow_{\SAF_i}\   v_i\ \ \ \ i\in\{1..n\}
% \\
%{\mathcal{G}}
%(\M, {\ClassOf {\alpha} {\sigma}}, {\f}) \  =  
%\ \f\lp \p_1, \ldots \p_n \rp \lb\ \SE \ \rb
%  \\
%  \M,\, \sigma[\p_1\mapsto v_1, .... \p_n\mapsto v_n], \SE    \hookrightarrow_{\SAF}\   v 
%  \\
%  \SAF,\ \SAF_i,\ \{ \, (\alpha, \f)\,  \}\  \ \mbox{are all disjoint for} \ i\in\{1..n\}
%  \\
%  \SAF' =  \SAF\cup\SAF_1 \cup \ldots \cup \SAF_n \cup \{ \alpha, \f \}
%}
%{
% \M,\,  \sigma, \ \SE_0.\f \lp \SE_1,....\SE_n\rp \hookrightarrow_{\SAF'}  \ v
%}
%\\ \\
%\inferenceruleNM{If\_True\_Val} 
%{
% \M,\,  \sigma, \SE \ \hookrightarrow_\SAF\   \prg{true}  \\
%   \M,\,  \sigma, \SE_1 \ \hookrightarrow_{\SAF_1}\   v \\
%   \SAF, \SAF_1 \ \mbox{are  disjoint}  
%}
%{
% \M,\, \sigma, \kwN{if}\ \SE\  \kwN{then} \ \SE_1 \ \kwN{else} \ \SE_2\  \hookrightarrow_{ \SAF\cup \SAF_1} \ v
%}
%& &
%\inferenceruleNM {If\_False\_Val} 
%{
% \M,\,  \sigma, \SE \ \hookrightarrow_\SAF\   \prg{false}  \\
%   \M,\,  \sigma, \SE_2 \ \hookrightarrow_{\SAF_2}\   v \\
%    \SAF, \SAF_2\   \mbox{are  disjoint}  
%}
%{
% \M,\, \sigma, \kwN{if}\ \SE\  \kwN{then} \ \SE_1 \ \kwN{else} \ \SE_2\  \hookrightarrow_{ \SAF\cup \SAF_2}\  v
%}
%\\ \\ 
%\inferenceruleNM {Equals\_True\_Val} 
%{
% \M,\,  \sigma, \SE_1 \ \hookrightarrow_{\SAF_1}\    v \\
%   \M,\,  \sigma, \SE_2 \ \hookrightarrow_{\SAF_2}\     v\\
%    \SAF_1, \SAF_2\   \mbox{are  disjoint}  
%}
%{
% \M,\, \sigma, \SE_1 =  \SE_2 \hookrightarrow_{{\SAF_1}\cup{\SAF_2}} \prg{true}
%}
%& &
%\inferenceruleNM {Equals\_False\_Val} 
%{
% \M,\,  \sigma, \SE_1 \ \hookrightarrow_{\SAF_1}\    v \\
%   \M,\,  \sigma, \SE_2 \ \hookrightarrow_{\SAF_2}\     v'\\
%       \SAF_1, \SAF_2\   \mbox{are  disjoint}  \hspace{2cm}
%    v\neq v'
%}
%{
% \M,\, \sigma, \SE_1 =  \SE_2 \hookrightarrow_{{\SAF_1}\cup{\SAF_2}}\ \prg{false}
%}
%\end{array}
%\end{array}
%$}
%\caption{The Value of Simple Expressions, where we take $\SE.\f$  is a shorthand for $\SE.\f\lp\ \rp$}
%\label{fig:ValueSimpleExpressions}
%\end{figure*}

\subsection{Definedness of execution, and extending configurations}

Note that interpretations and executions need not always be defined.
For example, in a configuration whose top frame does not contain \x\,  in its domain, $\interp {\x}{\phi} $ is undefined. We define the relation $\sigma \subconf \sigma'$ to express that   $\sigma$ has more information than $\sigma'$, and then prove that more defined configurations preserve interpretations:

\begin{definition}[Extending runtime configurations]
The relation $\subconf$   is defined on runtime configurations as follows. Take arbitrary
configurations $\sigma$, $\sigma'$, $\sigma''$, frame $\phi$, stacks $\psi$, $\psi'$,  heap $\chi$, address $\alpha$ free in $\chi$, value $v$ and object $o$, and define $\sigma  \subconf \sigma'$ as the smallest relation such that:

\begin{itemize}
\item
$\sigma  \subconf \sigma$
\item
$(\phi[\x \mapsto v]\cdot \psi, \chi) \subconf  (\phi\cdot \psi, \chi)$
\item
$(\phi\cdot\psi\cdot\psi', \chi) \subconf  (\phi\cdot \psi, \chi)$
\item
$(\phi, \chi[\alpha \mapsto o) \subconf  (\phi\cdot \psi, \chi)$
\item
$\sigma'  \subconf \sigma''$ and $\sigma''  \subconf \sigma$ imply $\sigma'  \subconf \sigma$
\end{itemize}
\end{definition}



\begin{lemma}[Preservation of interpretations and executions]
If $\sigma'  \subconf \sigma$, then

\begin{itemize}
\item
If $\interp {\x}{\sigma}$ is defined,\ \  then $\interp {\x}{\sigma'}$=$\interp {\x}{\sigma}$.
\item
If $\interp {\this.\f}{\sigma}$ is defined,\ \  then $\interp {\this.\f}{\sigma'}$=$\interp {\this.\f}{\sigma}$.
\item
If $\ClassOf {\alpha} {\sigma} $  is defined, \ \ then  $\ClassOf {\alpha} {\sigma'} $  = $\ClassOf {\alpha} {\sigma} $.
\item
If $\M, \sigma \, \leadsto^*\, \sigma''$, \ \  then     \ \ there exists a $\sigma''$, so that\ $\M, \sigma'\, \leadsto^*\, \sigma'''$
and $\sigma''' \subconf \sigma''$.
\end{itemize}
\end{lemma}




\subsection{Module linking}

When studying validity of assertions in the open world we are concerned with whether   the  module
under consideration makes  a certain guarantee when executed in conjunction with other modules. To answer this, we
 need the concept of linking other modules to the module  under consideration.
 Linking, $\link$ ,  is an operation that takes two modules, and creates a module which corresponds  to the union of the two.
 %We use the concept of module linking in order to model the open world, where our module $\M$ whose code we know, will be executed together with further modules whose code we do not know.
We place some conditions for module linking to be defined: We require that the two modules do not contain implementations for the same class identifiers,

%SD removed the below as I think it is settled.
%\susan{where does the aux come from? I think what you said in the fragment calculus about disjointedness is neater} 
%\sophia{aux is defined in last line of Def. below. In the Frag Calculus the modules were not mappings, so we did not need something like aux; any idea how to avoid?}


\begin{definition}[Module Linking]
\label{def:link}
The linking operator\  \ $\link:\  \syntax{Module} \times  \syntax{Module} \longrightarrow \syntax{Module}$ is defined as follows:

$
\M \link \M{'}  \ \triangleq  \ \ \left\{
\begin{array}{l}
                        \M\ \link\!_{aux}\ \M{'},\ \ \   \hbox{if}\  \ dom(\M)\!\cap\!dom(\M')\!=\!\emptyset\\
\mbox{undefined}  \ \ \ \mbox{otherwise.}
\end{array}
                    \right.$

and where,
\begin{itemize}
     \item
   For all  $\prg{C}$: \ \
   $(\M\ \link\!_{aux}\ \M')(\prg{C})\  \triangleq  \ \M(\prg{C})$  if  $\prg{C}\in dom(\M)$, and  $\M'(\prg{C})$ otherwise.
 \end{itemize}
\end{definition}

The lemma below says  that linking is associative and commutative, and preserves execution.

\begin{lemma}[Properties of linking] -- this is the same as \ref{lemma:linking:properties} in the main text --
 For any modules $\M$,   $\M'$ and $\M''$, and runtime configurations $\sigma$, and $\sigma'$ we have$:$
 \label{lemma:linking:properties}

 \begin{itemize}
     \item
     $(\M \link \M')\link \M''$ = $\M \link (\M' \link \M'')$.
    \item
      $\M \link \M'$  = $\M' \link\M$.
      \item
      $\M, \sigma \leadsto \sigma'$, and $\M\link \M'$ is defined, \  \  implies\ \   $\M\link \M', \sigma \leadsto \sigma'$
   \end{itemize}

 \end{lemma}

 \subsection{Module pairs and visible states semantics}

A module $\M$ adheres to an invariant assertion  $\A$, if it satisfies
$\A$ in all runtime configurations that  can be reached through execution of the code of $\M$ when linked to that
of {\em any other} module $\M'$, and
which are {\em external} to $\M$. We call external to $\M$ those
configurations which are currently executing code which does not come from $\M$. This allows the code in $\M$ to break
the invariant internally and temporarily, provided that the invariant is observed across the states visible to the external client $\M'$.

Therefore, we define execution in terms of an internal module $\M$ and an external module $\M'$, through the judgment $\M \mkpair \M', \sigma \leadsto \sigma'$, which mandates that $\sigma$ and $\sigma'$ are external to $\M$, and that there exists an execution which leads from $\sigma$ to $\sigma'$ which leads through intermediate configurations
$\sigma_2$, ...  $\sigma_{n+1}$ which are all internal to $\M$, and thus unobservable from the client.
In a sense, we "pretend" that all calls to functions from $\M$ are executed atomically, even if they involve several intermediate,
internal steps.


\begin{definition} [Repeating definition \ref{def:execution:internal:external}]
Given runtime configurations $\sigma$,  $\sigma'$,  and a module-pair $\M \mkpair \M'$ we define
execution where $\M$ is the internal, and $\M'$ is the external module as below:

\begin{itemize}
\item
$\M \mkpair \M', \sigma \leadsto \sigma'$ \IFF
there exist  $n\geq 2$ and runtime configurations $\sigma_1$,  ...
$\sigma_n$, such that
\begin{itemize}
\item
$\sigma$=$\sigma_1$,\ \  \ \ and\ \ \ \ $\sigma_n=\sigma'$.
\item
$\M \link \M', \sigma_i \leadsto \sigma_{i+1}'$,\  \  for $1\leq i \leq n\!-\!1$
\item
$\ClassOf{\interp {\this} {\sigma}} {\sigma}\not\in dom({\M})$,  \ \  \ \ and\ \ \ \
$\ClassOf{\interp {\this} {\sigma'}} {\sigma'} \not\in dom({\M})$,
\item
 $\ClassOf{\interp {\this} {\sigma_i}} {\sigma_i} \in dom({\M})$,\ \ \ \ for $2\leq i \leq n\!-\!2$
\end{itemize}
\end{itemize}

\end{definition}

In the definition above $n$ is allowed to have the value $2$. In this case the final bullet is trivial and  there exists a direct, external transition from $\sigma$ to $\sigma'$.  Our definition is related to the concept of visible states semantics, but differs in that visible states semantics select the configurations at which an invariant is expected to hold, while we select the states which are considered for executions which are expected to satisfy an invariant. Our assertions can talk about several states (through the use of the $\Future {\_}$ and $\Past{\_}$ connectives), and thus, the intention of ignoring some intermediate configurations can only be achieved if we refine the concept of execution. 

The following lemma states that linking external modules preserves execution

\begin{lemma}[Linking modules preserves execution]
\label{lemma:module_pair_execution}
For any modules $\M$, $\M'$, and $\M''$, whose domains are pairwise disjoint, and runtime configurations $\sigma$, $\sigma'$,

\begin{itemize}
\item
 $\M \mkpair \M', \sigma \leadsto \sigma'$  implies $\M \mkpair (\M'\link\M'') ,\sigma \leadsto \sigma'$.  
\item
  $\M \mkpair \M', \sigma \leadsto \sigma'$  implies
$(\M\link\M'') \mkpair \M' , \sigma \leadsto \sigma'$.

\end{itemize}
\end{lemma}

\begin{proof} For the second guarantee  we use the fact that   $\M \mkpair \M', \sigma \leadsto \sigma'$ implies that all
intermediate configurations are internal to $\M$ and thus also to $\M\link\M''$.
\end{proof}

We can now answer the question as to which runtime configurations are pertinent when judging a module's
adherence to an assertion.
First, where does execution start? We define {\em initial} configurations to be those which may contain arbitrary code stubs, but which contain no objects. Objects will be created, and further methods will be called through execution of the code in $\phi.\prg{contn}$. From such initial configurations, executions of code from $\M \mkpair \M'$ creates a set of {\em arising} configurations, which, as we will see in Definition \ref{def:module_satisfies}, are pertinent when judging $\M$'s  adherence to assertions.

\begin{definition}[Initial and arising Configurations -- repeating Definition \ref{def:arise}] are defined as follows: \label{defn:iniial-and-arising}

\begin{itemize}
     \item
   $\Initial {(\psi,\chi)}$, \ \ if \ \ $\psi$ consists of a single frame $\phi$ with $dom(\phi)=\{ \this \}$, and there exists  some address $\alpha$, such that \ \ \    $\interp {\this}{\phi}$=$\alpha$, and \ $dom(\chi)$=$\alpha$,\  and\  
    $\chi(\alpha)=(\prg{Object},\emptyset)$.
 \item
 $\Arising  {\M\mkpair\M'} \ = \ \{ \ \sigma \ \mid \ \exists \sigma_0. \ [\  \Initial{\sigma_0} \  \ \wedge\ \  \M\mkpair\M', \sigma_0 \leadsto^* \sigma \ \ ] \ \ \} $
 \end{itemize}

\end{definition}

Note that there are infinitely many different initial configurations, they will be differing in the code stored in the continuation of the unioque frame.



\clearpage

\section{Properties of Assertions}
\label{app:assertions}
We now define the syntax and semantics of expressions and holistic assertions.
\sd{The novel, holistic, features of \Chainmail (permission, control, time, space, and viewpoint),
as well as our wish to support some form of recursion while keeping the logic of assertions classical,  introduced 
challenges, which we discuss in this section.}

 \subsection{Syntax of Assertions}
 


\begin{definition}[Assertions]  \sd{Assertions consist of (pure) expressions \e, classical assertions about the contents of heap/stack, the usual logical  connectives, as well as our holistic concepts.}
\label{def:assertions}


 $\begin{array}{lcl}
  ~  \\
 \SE  &\BBC&    \kwN{true}   ~\SOR~  \kwN{false}   ~\SOR~  \kwN{null}  ~\SOR~  \x  \   ~\SOR~  
     \   \SE=\SE    ~\SOR~ \kwN{if}\, \SE\,   \kwN{then}\,  \SE\,    \kwN{else}\, \SE    ~\SOR~  \SE.\f\lp\ \SE^* \ \rp\\
     \\
 \A &\ \BBC   &   \SE \   \mid \  \SE=\SE  \mid \   \SE:\prg{ClassId}  \ \mid \
    \SE\in\prg{S}   \mid  \  \\
    &
  &  \A \rightarrow \A  \ \mid\  \     \A \wedge \A  \ \mid\  \  \A \vee \A  \ \mid\  \ \neg A   \ \mid \ \\
  & &  \forall \x.\A  \ \mid \  \forall \prg{S}:SET.\A  \ \mid  \  \exists \x.\A  \ \mid \  \exists \prg{S}:SET.\A  \  \ \mid\   \\
 &    &  \CanAccess x y %\ \mid\  \ \Changes e 
           \ \mid\  \Calls {\prg{x}}  {\prg{m}} {\prg{x}}  {\prg{x}^*}\\          
% &    &  \kjx{\CanAccess x y \ \mid\  \ \Changes e 
%           \ \mid\  \Calls {\prg{x}}  {\prg{m}} {\prg{x}}  {\prg{x}^*} }\\
&    &  \Next \A  \ \mid \   \Future \A \ \mid \  \Prev \A   \ \mid \ \Past \A \ \mid \\  
 &    &        \Using \SF  \A  \ \mid \  \External \x     \\
% &    &   \kjx{\Using \SF  \A  \ \mid \  \External \x \ \mid \ \prg{x} \obeys \prg{S} }  \\
 \\
 \x, \f, \m &\BBC&  \prg{Identifier}  ~ \\
\end{array}$
\end{definition}
%% \footnote{
%% The operators $\wedge$, $\vee$,  $\neg$ and $\forall$  could have been
%% defined  through the usual shorthands, \eg, $\neg \A$ is short for
%% $\A \rightarrow \ff$ \etc, but here we give full definitions
%% instead\kjx{can we just cut this please?}}
 
 \sd{Expressions support calls with parameters  ($\e.\f(\e^*)$); these are calls to ghostfield
functions. This  supports recursion at the level of expressions; therefore, the value of  an expression  may be
undefined (either because of infinite recursion, or because the expression accessed undefined fields or variables). 
Assertions of the form   $\e$=$\e'$ are satisfied only if both $\e$ and $\e'$ are defined. Because we do not support 
recursion at the level of assertions, assertions from a classical logic (\eg $\A \vee \neg\A$ is a tautology). }
 
We will discuss evaluation of expressions in section \ref{sect:expressions}, standard assertions about heap/stack and logical
 connectives in \ref{sect:standard}. 
 \sophia{We have discussed  the treatment of  permission, control, space, and viewpoint  in 
the main text in  Definitions 3-7, % \ref{sect:pcsv} %HARD
the treatment of time in  Definitions 8,9 in the main text. % HARD, \ref{sect:time},
We will discuss properties of assertions in Lemmas \ref{lemma:classic}-\ref{lemma:classic:two}.}
 \sd{The judgement $\M\mkpair \M', \sigma  \models \A$ expresses that $\A$ holds in  $\M\mkpair \M'$ and $\sigma$, and 
while $\M\mkpair \M', \sigma  \not\models \A$  expresses that $\A$ does not hold  in  $\M\mkpair \M'$ and $\sigma$.} 
 

\subsection{Values of Expressions}
\label{sect:expressions}

The value  of  an expression  is described through judgment $ \M,\, \sigma, \SE \ \hookrightarrow\  v$,
defined in  Figure \ref{fig:ValueSimpleExpressions}.
We use the configuration, $\sigma$, to read the contents of the top stack frame
% value of variables defined in the stack frame
(rule ${\sf {Var\_Val}}$) or the contents of the heap (rule
${\sf {Field\_Heap\_Val}}$). We use the module, \M, to find the  ghost field declaration corresponding to the
ghost field being used. 



The treatment of fields and ghost fields is described in rules ${\sf {Field\_Heap\_Val}}$,\\  ${\sf {Field\_Ghost\_Val}}$ and 
${\sf {Field\_Ghost\_Val2}}$.  If the field \f~ exists in the heap, then its value is returned (${\sf {Field\_Heap\_Val}}$). 
Ghost field reads, on the other hand, have the form $\e_0.\f(\e_1,...\e_n)$, and their value is
described in rule ${\sf {Field\_Ghost\_Val}}$:
%
The lookup function $\mathcal G$  (defined in the obvious way in the Appendix, Def.\ref{def:lookup})
returns the expression constituting the body for that ghost field, as defined in the class of $\e_0$.
We return  that expression
evaluated in a configuration where the formal parameters have been substituted by the values of the actual
parameters.


Ghost fields support recursive definitions. For example, imagine a module $\M_0$ with
a class \prg{Node} which has a field called \prg{next}, and which 
had a ghost field \prg{last}, which finds  the last \prg{Node} in a sequence
and is defined recursively as \\
$~ \strut \hspace{.1cm}$ \ \ \ \prg{if}\ \ \prg{this.next}=\prg{null}\  \prg{then} \ \prg{this} \ \prg{else} \ \prg{this.next.last},\\
and another ghost field \prg{acyclic}, which expresses that a sequence is acyclic,
defined recursively as \\
$~ \strut \hspace{.1cm}$ \ \ \ \prg{if}\ \ \prg{this.next}=\prg{null}\  \prg{then} \ \prg{true} \ \prg{else} \ \prg{this.next.acyclic}.\\



The relation $ \hookrightarrow$ is partial. 
For example, assume   a configuration
$\sigma_0$ where
\prg{acyc} points to a \prg{Node} whose field \prg{next} has value \prg{null}, and   
\prg{cyc} points to a \prg{Node} whose field \prg{next} has the same value as \prg{cyc}. Then,   
$\M_0,\sigma_0,\,\prg{acyc.acyclic}  \ \hookrightarrow\  \prg{true}$, but we would have no value for 
$\M_0,\sigma_0,\, \prg{cyc.last}  \ \hookrightarrow\  ...$, nor for
$\M_0,\sigma_0,\, \prg{cyc.acyclic}  \ \hookrightarrow\  ...$.

Notice also that for an expression of the form  
\prg{\e.\f}, both ${\sf {Field\_Heap\_Val}}$ and ${\sf {Field\_Ghost\_Val2}}$ could be applicable: rule ${\sf {Field\_Heap\_Val}}$
will be applied if \prg{f} is a field of the object at \prg{e}, while rule ${\sf {Field\_Ghost\_Val}}$
will be applied if \prg{f} is a ghost field of the object at \prg{e}. We expect the set of fields and ghost fields in a 
given class to be disjoint.
This allows a specification to be agnostic over whether a field is a physical field or just ghost information.
For example, assertions (1) and (2) from \sophia{section 2 in the main text} % HARD
%  \ref{sect:motivate:Bank}
 talk about the \prg{balance} of an \prg{Account}. 
In module $\M_{BA1}$ (Appendix~\ref{Bank:appendix}), where we keep the balances in the account objects, this is a physical field. 
In $\M_{BA2}$ (also in Appendix~\ref{Bank:appendix}), where we keep the
balances in a ledger, this is ghost information.  
 


\begin{figure*}
{$\begin{array}{l}
\begin{array}{llll}
\inferenceruleN {True\_Val} {
}
{
 \M,\, \sigma, \kwN{true} \ \hookrightarrow\  \kwN{true}
}
& 
\inferenceruleN {False\_Val} {
}
{
 \M,\, \sigma, \kwN{false} \ \hookrightarrow\  \kwN{false}
}
&
\inferenceruleN  {Null\_Val} {
}
{
 \M,\, \sigma, \kwN{null} \ \hookrightarrow\  \kwN{null}
}
&
\inferenceruleN {Var\_Val} {
}
{
 \M,\,  \sigma, \x \ \hookrightarrow\   \sigma({\x})
}
\end{array}
\\ \\
\begin{array}{lll}
\begin{array}{l}
\inferenceruleNM{Field\_Heap\_Val} {
 \M,\,  \sigma, \SE \ \hookrightarrow\   \alpha \hspace{1.5cm} 
 \sigma(\alpha,\f)=v
}
{
 \M,\, \sigma, \SE.\f  \ \hookrightarrow\   v
}
\\
\\
\inferenceruleNM{Field\_Ghost\_Val2} {
 \M,\, \sigma, \SE.\f \lp \rp \ \hookrightarrow\   v
}
{
 \M,\, \sigma, \SE.\f   \ \hookrightarrow\   v
}
\end{array}
& &
\inferenceruleNM{Field\_Ghost\_Val}
{
~ \\
 \M,\, \sigma, \SE_0   \ \hookrightarrow\  \alpha
\\
 \M,\, \sigma, \SE_i  \ \hookrightarrow\   v_i\ \ \ \ i\in\{1..n\}
 \\
{\mathcal{G}}
(\M, {\ClassOf {\alpha} {\sigma}}, {\f}) \  =  
\ \f\lp \p_1, \ldots \p_n \rp \lb\ \SE \ \rb
  \\
  \M,\,\sigma[\p_1\mapsto v_1, .... \p_n\mapsto v_n], \SE    \hookrightarrow_{\SAF}\   v 
 }
{
 \M,\,  \sigma, \ \SE_0.\f \lp \SE_1,....\SE_n\rp \hookrightarrow   \ v
}
\\ \\
\inferenceruleNM{If\_True\_Val} 
{
 \M,\,  \sigma, \SE \ \hookrightarrow\   \prg{true}  \\
   \M,\,  \sigma, \SE_1 \ \hookrightarrow\   v  
}
{
 \M,\, \sigma, \kwN{if}\ \SE\  \kwN{then} \ \SE_1 \ \kwN{else} \ \SE_2\  \hookrightarrow  \ v
}
& &
\inferenceruleNM {If\_False\_Val} 
{
 \M,\,  \sigma, \SE \ \hookrightarrow\   \prg{false}  \\
   \M,\,  \sigma, \SE_2 \ \hookrightarrow\   v  }
{
 \M,\, \sigma, \kwN{if}\ \SE\  \kwN{then} \ \SE_1 \ \kwN{else} \ \SE_2\  \hookrightarrow\  v
}
\\ \\ 
\inferenceruleNM {Equals\_True\_Val} 
{
 \M,\,  \sigma, \SE_1 \ \hookrightarrow\    v \\
   \M,\,  \sigma, \SE_2 \ \hookrightarrow\     v 
}
{
 \M,\, \sigma, \SE_1 =  \SE_2 \hookrightarrow \prg{true}
}
& &
\inferenceruleNM {Equals\_False\_Val} 
{
 \M,\,  \sigma, \SE_1 \ \hookrightarrow\    v \\
   \M,\,  \sigma, \SE_2 \ \hookrightarrow\     v' \hspace{2cm}  v\neq v'
}
{
 \M,\, \sigma, \SE_1 =  \SE_2 \hookrightarrow \ \prg{false}
}
\end{array}
\end{array}
$}
\caption{Value of  Expressions}
\label{fig:ValueSimpleExpressions}
\end{figure*}

\subsection{Satisfaction of Assertions - standard}
\label{sect:standard}
\sd{
We now define the semantics of assertions involving expressions, the heap/stack, and logical connectives.
The semantics are unsurprising, except, perhaps the relation between validity of assertions and the values of
expressions.
}


 \begin{definition}[Interpretations for simple expressions]

For a runtime configuration, $\sigma$,    variables $\x$ or \SF, we define its interpretation as follows:

\begin{itemize}
  \item
  $\interp {\x}{\sigma}$ $ \triangleq$ $\phi(\x)$  \ \ if \ \ $\sigma$=$(\phi\cdot\_,\_)$
  \item
  $\interp {\SF}{\sigma}$ $ \triangleq$ $\phi(\SF)$  \ \ if \ \ $\sigma$=$(\phi\cdot\_,\_)$
  \item
    $\interp {\x.\f}{\sigma}$ $ \triangleq$ $\chi(\interp {\x}{\sigma},\f)$  \ \ if \ \ $\sigma$=$(\_,\chi)$
   \end{itemize}
\end{definition}   

 
\begin{definition}[ Basic Assertions] For modules $\M$, $\M'$,  configuration $\sigma$,  we define$:$
%validity of basic assertions: 
\label{def:valid:assertion:basic}
\begin{itemize}
\item
$\M\mkpair \M', \sigma \models\SE$ \IFF   $ \M,\,  \sigma, \SE \ \hookrightarrow\   \prg{true}$ 
\item
$\M\mkpair \M', \sigma \models\SE=\SEPrime$ \IFF there exists a value $v$ such that  $\M,\,  \sigma, \SE \ \hookrightarrow\   v$  and $ \M,\,  \sigma, \SEPrime \ \hookrightarrow\   v$.
           \item
$\M\mkpair \M', \sigma \models\SE:\prg{ClassId}$ \IFF there exists an address $\alpha$ such that \\
$\strut ~ $ \hspace{2in} \hfill   
 $ \M,\,  \sigma, \SE \ \hookrightarrow\   \alpha$, and $\ClassOf{\alpha}{\sigma}$ = \prg{ClassId}.
\item
$\M\mkpair \M', \sigma \models \SE\in \prg{S}$ \IFF there exists a value $v$ such that 
 $ \M,\,  \sigma, \SE \ \hookrightarrow\   v$, and $v \in \interp{\prg{S}}{\sigma}$.
\end{itemize}
\end{definition}

Satisfaction of assertions which contain expressions is predicated on termination of these expressions.
Continuing our earlier example,  
$\M_0\mkpair \M', \sigma_0 \models \prg{acyc.acyclic}$ holds for any $\M'$, while $\M_0\mkpair \M', \sigma_0 \models \prg{cyc.acyclic}$
does not hold, and $\M_0\mkpair \M', \sigma_0 \models \prg{cyc.acyclic}=\prg{false}$ does not hold either.
In general, when $\M\mkpair \M', \sigma  \models \prg{e}$ holds,  then $\M\mkpair \M', \sigma  \models \prg{e}=\prg{true}$ holds too.
But when $\M\mkpair \M', \sigma  \models \prg{e}$ does not hold, this does \emph{not} imply that $\M\mkpair \M', \sigma  \models \prg{e}=\prg{false}$ holds.
Finally, an assertion of the form $\e_0=\e_0$ does not always hold; for example,   $\M_0\mkpair \M', \sigma_0 \models \prg{cyc.last}=\prg{cyc.last}$ does not hold.

% \subsubsection{Logical connectives, quantifiers, space and control} 
We now define satisfaction of assertions which involve logical connectives and existential or universal quantifiers, in the standard way:

\begin{definition}[Assertions with logical connectives and quantifiers]  
%We now consider 
\label{def:valid:assertion:logical}
For modules $\M$, $\M'$, assertions $\A$, $\A'$, variables \prg{x}, \prg{y}, \prg{S},  and configuration $\sigma$, we define$:$
\begin{itemize}
\item
$\M\mkpair \M', \sigma \models \forall \prg{S}:\prg{SET}.\A$ \IFF  $\M\mkpair \M', \sigma[\prg{Q}\mapsto R] \models  \A[\prg{S}/\prg{Q}]$ \\
$\strut ~ $ \hfill for all sets of addresses $R\subseteq dom(\sigma)$, and  all \prg{Q} free in $\sigma$ and $\A$.
\item
$\M\mkpair \M', \sigma \models \exists \prg{S}:\prg{SET}\!.\,\A$ \IFF  $\M\mkpair \M', \sigma[\prg{Q}\mapsto R] \models  \A[\prg{S}/\prg{Q}]$ \\
 $\strut ~ $ \hfill  for some set of addresses $R\subseteq dom(\sigma)$, and   \prg{Q} free in $\sigma$ and $\A$.
\item
$\M\mkpair \M', \sigma \models \forall \prg{x}.\A$ \IFF
$\sigma[\prg{z}\mapsto \alpha] \models  \A[\prg{x}/\prg{z}]$ \ for all  $\alpha\in dom(\sigma)$, and  some \prg{z} free in $\sigma$ and $\A$.
\item
$\M\mkpair \M', \sigma \models \exists \prg{x}.\A$ \IFF
$\M\mkpair \M', \sigma[\prg{z}\mapsto \alpha] \models  \A[\prg{x}/\prg{z}]$\\
$\strut ~ $ \hfill for some  $\alpha\in dom(\sigma)$, and   \prg{z} free in $\sigma$ and $\A$.
\item
$\M\mkpair \M', \sigma \models \A \rightarrow \A' $ \IFF  $\M\mkpair \M', \sigma \models \A $ implies $\M\mkpair \M', \sigma \models \A' $
\item
$\M\mkpair \M', \sigma \models  \A \wedge \A'$   \IFF  $\M\mkpair \M', \sigma \models  \A $
and $\M\mkpair \M', \sigma \models  \A'$.
\item
$\M\mkpair \M', \sigma \models  \A \vee \A'$   \IFF  $\M\mkpair \M', \sigma \models  \A $
or $\M\mkpair \M', \sigma \models  \A'$.
\item
$\M\mkpair \M', \sigma \models  \neg\A$   \IFF  $\M\mkpair \M', \sigma \models  \A $
does not hold.
\end{itemize}
\end{definition}

Satisfaction is not preserved with growing configurations; for example, the assertion $\forall \x. [\ \x : \prg{Account} \rightarrow \x.\prg{balance}>100\ ]$ 
may hold in a smaller configuration, but not hold in an extended configuration. 
Nor is it preserved with configurations getting smaller; consider \eg $\exists \x. [\ \x : \prg{Account} \wedge \x.\prg{balance}>100\ ]$.

\noindent
Again, with our earlier example,  
$\M_0\mkpair \M', \sigma_0 \models \neg (\prg{cyc.acyclic}=\prg{true})$    and  
$\M_0\mkpair \M', \sigma_0 \models  \neg (\prg{cyc.acyclic}=\prg{false})$, 
and also 
$\M_0\mkpair \M', \sigma_0 \models  \neg (\prg{cyc.last}=\prg{cyc.last})$
hold.

\label{sect:pl} 
We define equivalence of  assertions in the usual sense: two assertions are equivalent if they are satisfied  in
the context of the same configurations.
Similarly, an assertion entails another assertion, iff all configurations 
which satisfy the former also satisfy the latter.  

\begin{definition}[Equivalence and entailments of assertions]
$ ~ $

\begin{itemize}
\item
$\A \subseteqq \A'\  \IFF\    \forall \sigma.\, \forall \M, \M'. \ [\ \ \M\mkpair \M', \sigma \models \A\ \mbox{ implies }\ \M\mkpair \M', \sigma \models \A'\ \ ].$
\item
$\A \equiv \A'\  \IFF\     \A \subseteqq \A' \mbox{ and }  \A' \subseteqq \A.$
\end{itemize}
\end{definition}



\begin{lemma}[Assertions are classical-1]
\label{lemma:classic}
For all runtime configurations $\sigma$,    assertions $\A$ and $\A'$, and modules $\M$  and $\M'$, we have
\begin{enumerate}
\item
$\M\mkpair \M', \sigma \models \A$\ or\ $\M\mkpair \M', \sigma \models \neg\A$
\item
$\M\mkpair \M', \sigma  \models \A \wedge \A'$ \SP if and only if \SP $\M\mkpair \M', \sigma \models \A$ and $\M\mkpair \M', \sigma  \models \A'$
\item
$\M\mkpair \M', \sigma  \models \A \vee \A'$ \SP if and only if \SP $\M\mkpair \M', \sigma  \models \A$ or  $\sigma \models \A'$
\item
$\M\mkpair \M', \sigma  \models \A \wedge \neg\A$ never holds.
\item
$\M\mkpair \M', \sigma  \models \A$ and  $\M\mkpair \M', \sigma  \models \A \rightarrow \A'$  implies
$\M\mkpair \M', \sigma  \models \A '$.
\end{enumerate}
\end{lemma}
\begin{proof} \sd{The proof of part (1) requires to first prove that for all \emph{basic assertions} \A, \\
\strut \hspace{1.1cm} (*) \ \ \ either $\M\mkpair \M', \sigma  \models \A$
or $\M\mkpair \M', \sigma  \not\models \A$.\\
We prove this using Definition \ref{def:valid:assertion:basic}.
 Then, we prove (*) for all
possible assertions, by induction of the structure of \A, and the Definitions 
 \ref{def:valid:assertion:logical},
 \sophia{and also Definition 3-9 from the main text} %HARD
 % \ref{def:valid:assertion:access}, \ref{def:valid:assertion:control}, \ref{def:valid:assertion:view},  
 %\ref{def:valid:assertion:space}, and \ref{def:valid:assertion:time}.
Using the definition of $\M\mkpair \M', \sigma \models \neg\A$ from Definition  \ref{def:valid:assertion:logical} we conclude the proof of (1).

For parts  (2)-(5) the proof goes by application of the corresponding definitions from \ref{def:valid:assertion:logical}.
 }
  \end{proof}.
 
 \begin{lemma}[Assertions are classical-2]
 \label{lemma:classic:two}
For     assertions $\A$, $\A'$, and $\A''$ the following equivalences hold
\label{lemma:basic_assertions_classical}
\begin{enumerate}
\item
$ \A \wedge\neg \A \ \equiv \  \prg{false}$
\item
$ \A \vee \neg\A   \ \equiv \  \prg{true}$
\item
$ \A \wedge \A'  \ \equiv \  \A' \wedge \A$
\item
$ \A \vee \A'  \ \equiv \  \A' \vee \A$
\item
$(\A \vee \A') \vee \A'' \ \equiv \  \A \vee (\A' \vee\A'')$
\item
$(\A \vee \A') \wedge \A'' \ \equiv \  (\A \wedge \A')\, \vee\, (\A \wedge \A'')$
\item
$(\A \wedge \A') \vee \A'' \ \equiv \  (\A \vee \A')\, \wedge\, (\A \vee \A'')$
\item
$\neg (\A \wedge \A') \  \ \equiv \  \neg  \A   \vee\, \neg \A''$
\item
$\neg (\A \vee \A') \  \ \equiv \  \neg  \A   \wedge\, \neg \A'$
\item
$\neg (\exists \prg{x}.\A )  \  \ \equiv \  \forall \prg{x}.(\neg  \A)$
\item
$\neg (\exists \prg{S}:\prg{SET}.\A )  \  \ \equiv \  \forall \prg{S}:\prg{SET}.(\neg  \A)$
%\item
% $\neg (\exists k:\mathbb{N}.\A )  \  \ \equiv \  \forall  k:\mathbb{N}.(\neg  \A)$
%\item
%$\neg (\exists \prg{fs}:FLD^k.\A )  \  \ \equiv \  \forall \prg{fs}:FLD^k.(\neg  \A)$
\item
$\neg (\forall \prg{x}. \A)  \  \ \equiv \  \  \exists \prg{x}.\neg(\A )$
\item
$\neg (\forall \prg{S}:\prg{SET}. \A)  \  \ \equiv \  \  \exists \prg{S}:\prg{SET}.\neg(\A )$
%\item
%$\neg (\forall k:\mathbb{N}. \A)  \  \ \equiv \  \  \exists k:\mathbb{N}.\neg(\A )$
%\item
%$\neg (\forall \prg{fs}:FLD^k. \A)  \  \ \equiv \  \  \exists \prg{fs}:FLD^k.\neg(\A )$
\end{enumerate}
\end{lemma}
\begin{proof}
All points follow by application of the corresponding definitions from \ref{def:valid:assertion:logical}. % and lemma 
 \end{proof}

 

% \begin{definition}
%We say that $\sigma \vdash \A$ if for any  \x\, is free in $\A$ and any
%  any term $\x.\f_1...\f_n$ appearing in $\A$,
% the interpretation $\interp{\x.\f_1...\f_n} \sigma$ is defined.
%\end{definition}
%
%Note that if we take $n=0$ in the definition above we obtain as corollary that   all variables that appear free in $\A$ they  are in the domain of the top frame in $\sigma$.
%
%\begin{lemma}[Preservation of satisfaction] $ $
%\label{lemma:preserve:valid}
%\begin{itemize}
%\item
%If  $\sigma \vdash \A$ and $\M\mkpair \M',  \sigma \vdash \A$ and   $\sigma' \subconf \sigma$, \  then  \ $\M\mkpair \M',  \sigma' \models \A$.
%\end{itemize}
%\end{lemma}


\clearpage

\section{Bank Account Implementations}
\label{Bank:appendix}
\begin{figure}[thb]
\begin{lstlisting}
class Bank{

   method newAccount(amt){
        if (amt>=0) then{
            return new Account(this,amt)
   }   }
}

class Account{

    field balance
    field myBank
    
    method deposit(src,amt){
       if (amt>=0 && src.myBank=this.myBank && src.balance>=amt) then{
           this.balance = this.balance+amt
           src.balance = src.balance-amt
   }   }
   method makeNewAccount(amt){
       if (amt>=0 && this.balance>=amt) then{
           this.balance = this.balance - amt;
           return new Account(this.myBank,amt)
   }    }
}
\end{lstlisting}
 \vspace*{-7mm}
\caption{$M_{BA1}$: Implementation of \prg{Bank} and \prg{Account} -- version 1}
\label{fig:BanAccImplV1}
\end{figure}

In this section we revisit the \prg{Bank}/\prg{Account} example from
 section  \ref{sect:motivate:Bank}, and show two different
 implementations, derived from Noble et al.\ \cite{arnd18} . Both implementations  satisfy the three functional specifications and the holistic assertions
 (1), (2) and (3)  shown in section \ref{sect:motivate:Bank}.
 The first version gives rise to runtime configurations as $\sigma_1$, 
 shown on the left side of Fig. \ref{fig:BakAccountDiagrams}, while the
 second version gives rise to runtime configurations as $\sigma_2$,
 shown on the right side of Fig. \ref{fig:BakAccountDiagrams}. 

 In this code, we use more syntax than the minimal syntax defined for \LangOO in Def. \ref{defONE}, as we use conditionals, and we allow nesting of expressions, e.g.\ a field read to be the receiver of a method call. Such extension can easily be encoded in the base syntax.

$\M_{BA1}$, the fist version is shown Fig. \ref{fig:BanAccImplV1}. It keeps all the information in the \prg{Account} object: namely,
the \prg{Account} contains the pointer to the bank, and the balance, while the \prg{Bank} is a pure capability, which contains
no state but is necessary for the creation of new \prg{Account}s.
%\sophia{James, you have a nice, high level description of that.\kjx{somewhere}}
In this version we have no ghost fields.

\begin{figure}[htb]
\begin{lstlisting}
class Bank{
   field ledger // a Node
   
    method deposit(dest,src,amt){
       destNd = this.ledger.find(dest)
       srcNd = this.ledger.find(src)
       srcBalance = srcNd.getBalance()
       if ( destNd =/=null && srcNd=/=null && srcBalance>=amt && amt >=0 ) then
           destNd.addToBalance(amt)
           srcNc.addToBalance(-amt)           
    }  }     
    method newAccount(amt){
      if (amt>=0)  then{
           newAcc = new Account(this);
           this.ledger = new Node(amt,this.ledger,newAcc)
           return newAcc 
    } }
   
   ghost  balance(acc){ this.ledger.balance(acc)  } 
}
\end{lstlisting}
 \vspace*{-7mm}
\caption{$M_{BA2}$: Implementation of \prg{Bank}   -- version 2}
\label{fig:BanAccImplV2a}
\end{figure}

 
\begin{figure}[htb]
\begin{lstlisting}
class Node{
   field balance
   field next   
   field myAccount
   
   method addToBalance(amt){
       this.balance = this.balance + amt
   }   
   method find(acc){
      if this.myAccount == acc then{
          return this
     } else { 
          if this.next==null then{
              return null
          } else {
              return this.next.find(acc)
    }  } } 
    method getBalance(){ return balance }
    
    ghost balance(acc){
        if (this.myAccount == acc)  then  this.balance
                                                    else  ( if this.next==null then -1 else this.next.find(acc) )
    }
}          
\end{lstlisting}
 \vspace*{-7mm}
\caption{$M_{BA2}$: Implementation of \prg{Node}   -- version 2}
\label{fig:BanAccImplV2a}
\end{figure}

\begin{figure}[htb]
\begin{lstlisting}
class Account{

    field myBank
    
    method deposit{src,amt){
             this.myBank.deposit(this,src,amt)
    }   }    
    method makeNewAccount(amt){
      if (amt>=0 && this.balance>=amt) then{
           newAcc = this.myBank.makeNewAccount(0)
           newAcc.deposit(this,amt)
           return newAcc
    }    }     

   ghost balance(){ this.myBank.balance(this) }   
}
\end{lstlisting}
 \vspace*{-7mm}
\caption{$M_{BA2}$: Implementation of  \prg{Account} -- version 2}
\label{fig:BanAccImplV2b}
\end{figure}
 


% Notice that even though anybody can create a new Account, by calling the constructor from the class \prg{Account} 
$\M_{BA1}$, the second version is shown Fig. \ref{fig:BanAccImplV2a} and \ref{fig:BanAccImplV2b}. It keeps all the information 
in the \prg{ledger}: each \prg{Node} points to an \prg{Account} 
and contains the balance for this particular \prg{Account}. Here \prg{balance} is a
ghost field of \prg{Account}; the body of that declaration calls the ghost field function \prg{balanceOf} of the \prg{Bank} which in its
turn calls the ghost field function \prg{balanceOf} of the \prg{Node}. Note that the latter is recursively defined.

Note also that \prg{Node} exposes the function \prg{addToBalance(...)}; a call to this function   modifies the \prg{balance} of an \prg{Account} without requiring that the caller has access to the \prg{Account}. This might look as if it contradicted assertions (1) and  (2)
  from section \ref{sect:motivate:Bank}. However, upon closer inspection, we see that the assertion is satisfied. Remember that we employ a two-module semantics, where any change in the balance of an account is observed from one external state, to another external state. By definition, a configuration is external if its receiver is external.  However, no external object will ever have access to a \prg{Node}, and therefore no external object will ever be able to call the method \prg{addToBalance(...)}. In fact, we can add another assertion, (4), which promises that any internal object which is externally accessible is either a \prg{Bank} or an \prg{Account}.

(4)\ \  $\triangleq$\ \ $\forall \prg{o},\forall \prg{o}'.[\ \ \External{ \prg{o}}\  \wedge\ \neg (\External{ \prg{o}'}) \ \wedge\  \CanAccess{\prg{o}}{\prg{o}'}     
    \longrightarrow \ \    \hfill$ \\
  $\strut \hspace{8.3cm} 
% TODO explain:
% we no longer need Past here, as we are ion visible states 
  [\    \prg{o}:\prg{Account}\ \vee \ \prg{o}':\prg{Bank}\  \ ]\ \ \  \ \ \ \ ] \hfill $




\clearpage

\section{Coq Formalism}
\label{sect:coq}
We present the properties of Chainmail that have been formalised in the Coq model. Table \ref{Coq} refers to proofs that can be found in the associated Coq formalism \cite{coq}.

\begin{table}
  \begin{tabular}{|l|l|l|l|}
    \hline

\textbf{Lemma \ref{lemma:linking} (and 3)} &
Properties of Linking
        & 
\parbox{.45\textwidth}{\scriptsize\begin{enumerate}[label={(\arabic*)}]
            \item \texttt{moduleLinking\_associative}
            \item \texttt{moduleLinking\_commutative\_1}
            \item \texttt{moduleLinking\_commutative\_2}
            \item \texttt{linking\_preserves\_reduction}
        \end{enumerate}}
        \\
\hline
\textbf{Lemma \ref{lemma:classic}} &   
\parbox{.45\textwidth}{\scriptsize\begin{enumerate}[label={(\arabic*)}]
            \item $A \wedge \neg A \equiv \texttt{false}$
            \item $A \vee \neg A \equiv \texttt{true}$
            \item $A \vee A' \equiv A' \wedge A$
            \item $A \wedge A' \equiv A' \wedge A$
            \item $(A \vee A') \vee A'' \equiv A \vee (A' \vee A'')$
        \end{enumerate}}
        & 
\parbox{.45\textwidth}{\scriptsize\begin{enumerate}[label={(\arabic*)}]
            \item \texttt{sat\_and\_nsat\_equiv\_false}
            \item -
            \item \texttt{and\_commutative}
            \item \texttt{or\_commutative}
            \item \texttt{or\_associative}
        \end{enumerate}}
        \\
\hline
\textbf{Lemma \ref{lemma:basic_assertions_classical}} &   
\parbox{.45\textwidth}{\scriptsize\begin{enumerate}[label={(\arabic*)}]
            \item $A \wedge \neg A \equiv \texttt{false}$
            \item $A \vee \neg A \equiv \texttt{true}$
            \item $A \vee A' \equiv A' \wedge A$
            \item $A \wedge A' \equiv A' \wedge A$
            \item $(A \vee A') \vee A'' \equiv A \vee (A' \vee A'')$
            \item $(A \vee A') \wedge A'' \equiv (A \vee A'') \wedge (A' \vee A'')$
            \item $(A \wedge A') \vee A'' \equiv (A \wedge A'') \vee (A' \wedge A'')$
            \item $\neg (A \wedge A') \equiv (\neg A \vee \neg A')$
            \item $\neg (A \vee A') \equiv (\neg A \wedge \neg A')$
            \item $\neg (\exists x.A) \equiv \forall x. (\neg A)$
            \item $\neg (\exists S.A) \equiv \forall S. (\neg A)$
            \item $\neg (\forall x.A) \equiv \exists x. (\neg A)$
            \item $\neg (\forall S.A) \equiv \exists S. (\neg A)$
        \end{enumerate}}
        & 
\parbox{.45\textwidth}{\scriptsize\begin{enumerate}[label={(\arabic*)}]
            \item \texttt{sat\_and\_nsat\_equiv\_false}
            \item -
            \item \texttt{and\_commutative}
            \item \texttt{or\_commutative}
            \item \texttt{or\_associative}
            \item \texttt{and\_distributive}
            \item \texttt{or\_distributive}
            \item \texttt{neg\_distributive\_and}
            \item \texttt{neg\_distributive\_or}
            \item \texttt{not\_ex\_x\_all\_not}
            \item \texttt{not\_ex\_$\Sigma$\_all\_not}
            \item \texttt{not\_all\_x\_ex\_not}
            \item \texttt{not\_all\_$\Sigma$\_ex\_not}
        \end{enumerate}}
\\
\hline
  \end{tabular}
  \caption{Chainmail Properties Formalised in Coq}
  \label{Coq}
\end{table}

% \newpage
% replaced for ACM-ization
% % \small{
%   \bibliographystyle{plain}
% \onecolumn{
%  \bibliography{Case,more}
% }

\end{document}

%%%%%%%%%%%%%%%%%%%%%%%%%%%%%%%%%%%%%%%%%%%%%%%%%%%%%%%
%%%%%%%%%%%%%%%%%%%%%%%%%%%%%%%%%%%%%%%%%%%%%%%%%%%%%%%%%
\forget{
\input{motivateDOM}

\section{Wrappers -- traditional Specification}

The figure needs some beautification

\begin{figure}[htb]
\begin{tabular}{llll}
\begin{lstlisting}
class Wrapper(nd,hgt){

  fld node=nd;
  fld height=hgt;

  func setPropety(i,prp)
  // PRE: i
  // POST:
  //     i>this.height -> modifies nothing
  //     &&
  //     i<=this.height -->
  //           modifies = { this.node.parent^i.property }
  //          &&
  //          this.node,parent^i.property == prp
  {
    if (i>height){ 
       return 
    } 
    else  
    {  nd=node;  
       while (i>0){
          nd=nd.getParent();
          i--;    };
        nd.setProperty(prp); }
  }    
  func getChild(i)
  
  { 
    Wrapper(node.getChild(i),
                    height+1); 
 }                           
}
\end{lstlisting}
\end{minipage}
\end{tabular}
 \vspace*{-7mm}
\caption{\prg{Node}  and \prg{Wrapper} }
\label{fig:DOM}
\end{figure}

In Figure \ref{fig:WrapperUse} we show an example of the use of  \prg{Wrapper} objects to attenuate the use of \prg{Node}s\footnote{SD: Is that a good sentence?}  The function \prg{usingWrappers} takes as parameter an object of unknown provenance, here called \prg{unknwn}. On lines 2-7 it creates a tree, consisting of nodes \prg{n1}, \prg{n2}, ... \prg{n6}, depicted as blue circles on the   right-hand-side of the Figure. On line 8 it creates a wrapper of \prg{n5} with height \prg{1}. This means that the wrapper \prg{w} may be used to modify \prg{n3}, \prg{n5} and \prg{n6} (\ie the objects in the green triangle), while it cannot be used to modify \prg{n1}, \prg{n2}, and \prg{4} (\ie the objects within the blue triangle). 
On line 8 we call   function \prg{untrusted} on the \prg{unknown} object, passing \prg{w} as   argument. 

\begin{figure}[htb]
\begin{tabular}{llll}
\ \ &
\begin{minipage}{0.35\textwidth}
\begin{lstlisting}
func usingWrappers(unknwn){
   n1=Node(null,"fixed"); 
   n2=Node(n1,"robust"); 
   n3=Node(n2,"volatile"); 
   n4=Node(n2,"const");
   n5=Node(n3,"variable");
   n6=Node(n3,"ethereal");
   w=Wrapper(n5,1);
   
   unknwn.untrusted(w);
   
   assert n2.property=="robust" 
   ...
}
\end{lstlisting}
\end{minipage}
& & 
\begin{minipage}{0.75\textwidth}
\includegraphics[width=\linewidth, trim=145  320 60 105,clip]{diagrams/DOM.pdf}
% x y z w
% y seems to eat up the bollom
% y=320 is good
% x eats space from left, if you increase it the diagram decreases from left
% w eats space from top, if you increase it the diagram decreases from top
% w=100 is good
%\includegraphics[page=3, width=\linewidth, trim=150  270 40 150, clip]{diagrams/snmalloc.pdf}\sdcomment{I think we need to change the diagram so that it says small slab.}
\strut \\
\strut \\

\end{minipage}
\end{tabular}
 \vspace*{-2.5mm}
\caption{\prg{Wrapper}s protecting \prg{Node}s }
\label{fig:WrapperUse}
\end{figure}

However, even though we know nothing about \prg{unknown} or \prg{untrusted}, and even though the call gives to \prg{unknown}  access to \prg{w}, which in turn has transitively access to all  \prg{Node}-s in the tree, 
we know that the call to the \prg{untrusted} function is guaranteed not to affect the   \prg{property} fields of the nodes \prg{n1}, \prg{n2}, and \prg{n4}. 
Thus, the assertion on line 10 is guaranteed to succeed. 
The question is how do we specify \prg{Wrapper}, so as to be able to prove this assertion.

We give a  specification of the class \prg{Wrapper}  in the appendix in the traditional specification style \cite{Leavens-etal07}.
 




\section{Example -- DAO}
The DAO ~\cite{DAO}  is a famous Ethereum contract  aiming  to support
collective management of funds,  and to place power directly in the hands of owners the DAO 
rather than delegate to directors. 
Nevertheless, a re-entrancy bug   exploited in June 2016, lead  to a loss of   \$50M, and
a hard-fork in the  chain ~\cite{DaoBug}. 
 
%In a similar style as that  of the ERC20 spec earlier, 
%We can give a \Chainmail~specification 
Holistic specifications allow us to write specifications which avoid the bug, and guarantee that under all circumstances,
all owners will be able to be reapid their money.
Namely, we
requiring that  % holds as much ether as the sum of
% the   its clients' balances, and
 the owners'  balances may only be affected by clients joining or leaving, and
projects being approved or repaying.
We also require,
as below, that % Consider a  simplified version of the DAO~\cite{DAO}:
% It keeps the moneys of a set of clients, and will refund them when they call the function \prg{repay}. 
%
% The \RoSpec~policy   from below says:  
if  \prg{p}  has a balance of \prg{m} at a \prg{DAO} contract \prg{d},
and if \prg{p} calls \prg{repay} on \prg{d}, then  
 \prg{d} is required to hold at least \prg{m} ether at the time of that call ($\prg{d}.\prg{ether}\geq \prg{m}$), and will eventually send \prg{m} back to \prg{p} (expressed as $\Future{\Calls{\prg{d.send(p)},m}}$:

%Formally: 
\vspace{.07cm}

\noindent  
% \prg{Pol\_DAO\_withdraw} \ $\equiv$ \\ 
\strut \hspace{0.5cm} $\forall \prg{d}:\prg{DAO}.\forall \prg{p}:\prg{Any}.\forall\prg{m}:\prg{Nat}.$\\
\strut \hspace{0.5cm} $[\ \  \Calls(\prg{p},\prg{d.repay(),\_})\, \wedge\, \prg{d.Balance(p)}=\prg{m} $\\ 
\strut \hspace{0.5cm} \ \ \ $\longrightarrow$\\
\strut \hspace{0.5cm} \ \ \ $\prg{d}.\prg{ether}\geq \prg{m}\ \wedge$ $\ \Future(\Calls(\prg{d.send(p)},\prg{m}))\ \ ] $ 

\noindent
The requirement that \prg{d} holds at least \prg{m} ether precludes the DAO bug,
in the sense that  any contract satisfying that spec cannot exhibit  the  bug:   a contract
which satisfies the spec  is guaranteed to always have enough money to satisfy all \prg{repay} requests.
This guarantee  holds, regardless of how many functions there are in the DAO.
In contrast, to preclude the DAO  bug with a classical spec, one would need to write a spec for each of the 
DAO functions (currently 19) and then study their emergent  behaviour. 


\vspace{.005cm}

We can now define  what it means for \prg{p} to have a  \prg{Balance} at  \prg{d}. The \prg{Balance}  is \prg{0} if the previous call was
a repayment; it is \prg{m} if  the previous call was \prg{p} joining \prg{d} and paying in \prg{m}. More cases are needed to reflect the financing and repayments of proposals, but they can be expressed with the concepts described so far.

\noindent
$\strut \hspace{0.5cm} \forall \prg{d}:\prg{DAO}.\forall \prg{p}.\forall:\prg{m}:\prg{Nat}.$\\
$\strut \hspace{0.5cm} [ \ \ \  $\\
$\strut \hspace{0.5cm} \ \  \ \  \prg{d.Balance(p)}=\prg{m}$\ $\longrightarrow$ \ $ \left\{
                            \begin{array}{ll}
                             \prg{0}, & \hbox{if}\ Prev(Call(\prg{p},\prg{d.repay(),\_})    \\
                             \vee
                             \\
                             \prg{m},  & \hbox{if}\  Prev(Call(\prg{p},\prg{d.join(),m}))   \\
                             ..., & ... 
                           \end{array} 
                         \right.    $\\
$\strut \hspace{0.5cm} ] $                         

\section{Example -- Purse and Mint}
take from our earlier works

In two versions: one where there is a ledger inside the Mint, and one where the Mint has no path to the Purses. This will serve to demonstrate how \prg{internal} is supposed to work.

\section{Example -- Membrane}

TODO - take from ShuPeng's thesis
}

\include{rest}

\subsection*{LATEX mysteries and terminology}
 \begin{enumerate}
 \item
 How can we make the references refer to the Definitions, Lemmas etc
 rather than the section where these appear?
 \begin{quotation}
   \color{orange} KJX:   Not sure what the problem is. I've put labels
   in the definitions and I can use refs to get definition
   numbers~\ref{defONE} and~\ref{def:syntax:classes} ---
   not ~\ref{secONE} and ~\ref{sec:syntax:classes}, the section
   numbers containing those definions.

   Alternatively there is the ``cleverer'' package
   \url{http://tug.ctan.org/tex-archive/macros/latex/contrib/cleveref/cleveref.pdf}
   where a ``\verb+\cref{foo}+'' can generate both the type and the
   number e.g.  ``Definition 3''.
 \end{quotation}

 \item
 Need a nice metavariable for set of addresses, currently it is $R$. Perhaps instead use an enumeration, as eg $\{ \ \alpha_1,...\alpha_n\ \} $
 or $\kappa$?

 \begin{quotation}
 \color{orange} KJX: Hmm, the enumeration is fine. Otherwise $A$?
 $\mathcal{A}$?  Or we could call that set a ``footprint'' and so go
 with $F$ or $\mathcal{F}$\ldots
 \end{quotation}
\item
Find a nice term  to refer to module pairs  (internal, external), and a term for
our version visible states semantics.
 \begin{quotation}
   \color{orange} KJX: ``modules'' and ``modular state semantics''.
   Going to ``modules'' only makes sense with my answer below.
   Other permutations of
   ``visible module/modular state semantics'' work also work:
     modular visible state; visible modular state; etc\ldots
 \end{quotation}


\item
Better symbols for module linking (currently a $\M\link\M'$), and
for module pairing (currently a $\M\mkpair \M'$) -- perhaps there should not be such an operator, as
it does not create a new module, it is only used in execution ($\M\mkpair \M', \sigma \leadsto \sigma'$)
and in satisfaction of assertions ($\M\mkpair \M', \sigma\models \A$).
\footnote{\toby{TM: I like~$\M \mkpair \M'$ as it suggests the asymmetry of the visible
    state semantics wrt~$\M$ and~$\M'$.}}

 \begin{quotation}
   \color{orange} KJX: I'm so used to $\M\link\M'$ that I can't think
   of an alternative --- or do I recall we used $\M * \M'$ as a
   separating conjunction?

   So I really liked $\M\mkpair \M'$ --- except then I though that I
   couldn't remember which was the module (inside) and which the
   anti-module (the outside).  For some reason I thought outside would
   go first.  Then I realised, it's easy, cos $\M$ is always the
   module, and $\M'$ is the antimodule.

   At which point I though: OK so let's just write $\M$ as the module,
   and given any $\M$, then $\M'$ (or $\overline{\M}$ or I guess
   \textbf{out}($\M$)) for the antimodule.

   The only thing I think this loses is that the $\M\mkpair \M'$
   syntax, also like a separating conjunction, is sort of
   self-framing: $\M\mkpair \M'$ encompasses the universe of modules.
   Whereas the other way around, we'd need a (implicit) universe of
   all modules $\mathcal{U}$, and then define $\M' \triangleq
   \mathcal{U} - \M$  If we went with $\M'$ then Sophia couldn't use
   $\M''$ and friends --- have to write \texttt{N} and \texttt{O} for other
   modules?

   I think the only change I could see in the whole document was that
   lemma~\ref{lemma:module_pair_execution} is subsumed into
   lemma~\ref{lemma:linking:properties}.
 \end{quotation}


 \end{enumerate}
