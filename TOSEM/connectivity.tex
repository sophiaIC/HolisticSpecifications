\section{Connectivity}
\label{sect:connectivity}
%Many programs are only allowed to pass on objects under certain conditions. This is to keep these objects safe from code that shouldn't have access to them, such as malicious external modules, or even other internal objects that were not expecting them, and whose actions may have unintended consequences. \mrr{As shown in assertion (3) in section~\ref{sect:motivate:Bank}, in} \Chainmail restricting access is expressible: assertions can ensure that objects which will need access have a means to gain it. Alternatively, they can ensure that when an object's state changes, some other object could have had access to cause that change.
\sdf{As we already saw in Section~\ref{sect:motivate:Bank}, 
access to objects is a crucial capability, without which certain effects 
(\eg reducing an account's balance) cannot take place. Therefore, to make code robust, we need to be able to 
restrict access to objects in the current, as well as in future configurations. This restriction requires
that the code itself does not leak access to sensitive objects. But it also requires that the underlying
programming language supports notions of confinement, and forbids ambient authority, as well as 
objects reading the state of other objects, following the principle\\
$\strut \quad \quad \quad  \quad \quad \quad  \quad \quad \quad  \quad \quad \quad $ \emph{Only connectivity begets connectivity}, \\
as 
formulated in  
\cite{MillerPhDx}.
}\mrr[We don't seem to have the citation for this?]{}

\sdf{Indeed, \LangOO supports the connectivity principle:}
% However, to prove these assertions, it is necessary to ensure code only has access to objects it is directly passed. This allows us to check only connections created explicitly to check these assertions. To check code only has access to what it is passed, we can ensure 
any method call---the barrier between \Mrr{the private state of different objects}---only \emph{connects} to what its parameters \Mrr{can obtain} access to. This \sdf{is stated in the following lemma, 
which mandates} that if two objects  are connected after completion of a method call, 
% This states that for any method call, if two objects are connected in the configuration when the call has completed, 
then either they were originally connected, or \mrr{for each object a parameter or the receiver could obtain access to the object}. %\mrr[Should we mention how \Chainmail doesn't need this guarantee]
%{This holds due to the lack of ambient authority in \LangOO, preventing methods from implicitly expanding their connectivity}.
%can be expressed formally as:

\begin{lemma}[Method call connectivity]
\label{lemma:method_call_connectivity}
\begin{align*}
\forall M . \forall \sigma, \sigma'. &\forall \alpha, \alpha' \in \chi. [\\
&\sigma\text{.cont} = \x_0.\m(\x_1,\dots\x_n)\ \  \wedge\ \  \M,\sigma\leadsto^\text{call} \sigma'\ \ \wedge\ \ \text{access}_{\sigma'}\langle \alpha,\alpha'\rangle 
\longrightarrow\\
&\text{access}_\sigma\langle\alpha,\alpha'\rangle\ \quad \vee
\\&\Mrrz{
\exists \alpha'' ,\alpha'''. (\exists k. (\x_k = \alpha'') \wedge \exists k. (\x_k = \alpha''')\ \  \wedge \ \ 
\text{access}^*_\sigma\langle \alpha'',\alpha\rangle\ \ \wedge \ \ 
\text{access}^*_\sigma\langle \alpha''',\alpha'\rangle)}\\
&]
\end{align*}
\end{lemma}

\sdf{The lemma above uses relation $\leadsto^\text{call}$ to describe 
execution of method to completion, and
%the connective  -- it is not a connective
 $\text{access}_\sigma\langle\alpha,\alpha'\rangle$ to describe objects accessing each other. 
We shall define these relations below, but first discuss  the use of the lemma:}


\mrr{Lemma~\ref{lemma:method_call_connectivity} %renders it
\sdf{makes} it sufficient to check the values passed at the boundaries---method call parameters, and returns---to prove that a given program doesn't leak information.}
\mrr{%An example of where the lemma is useful is for 
For example, assertion (3) from %the Bank example from 
Section~\ref{sect:motivate:Bank}   states
that  if someone external will have access to an account, then someone external must have had access to it originally.  By checking the methods of the \prg{Account} and \prg{Bank} classes, we need only ensure they do not pass out account references through \prg{return}s or method calls to external objects. From an internal method, an external call can only access the objects it is given, so if there is no \prg{Account} passed it will be unable to access one. From an external method, an internal call will either pass an \prg{Account} at a boundary by providing or returning the \prg{Account} reference, or, as per the disjunction in lemma~\ref{lemma:method_call_connectivity}, the external object must have originally had access before the the internal code began executing. This final case must hold if the internal code doesn't pass an \prg{Account} reference, in which case there is no leak since the external object already held a reference to the \prg{Account}.  
% SD thinks we can omit that
%Whilst checking the boundaries may be the expected requirement for proving the validity of assertion (3), lemma~\ref{lemma:method_call_connectivity} is important to prove to ensure the soundness of this approach, as it proves any method is unable to expand its connectivity beyond that which it is given.
}


\sdf{We now define }%To evaluate the entire call, we define a new reduction operator, 
 $\leadsto^\text{call}$, which executes function calls in one step. It does this by skipping any intermediate reduction steps with the additional frames for the method call and its child calls in the stack.

\begin{definition}[Call-as-step execution]
\label{def:call_execution} 
Given runtime configurations $\sigma$,  $\sigma'$,  and a module $\M$ we define call-as-step execution as follows:
 
\begin{itemize}
\item\Mrr[This definition has been updated]{
$\M, \sigma \leadsto^\text{call} \sigma'$ \IFF
there exist $n\geq 2$, runtime configurations $\sigma_1$ ... $\sigma_n$,
frames~$\phi$,~$\phi'$, and heaps~$\chi$,~$\chi'$, such that}
\begin{itemize}
\item
$\sigma = (\phi \cdot \psi, \chi)$, \ \ \ \ and\ \ \ \
$\sigma' = (\phi' \cdot \psi, \chi')$, for some stack $\psi$
\item
$\sigma_1=(\phi, \chi)$,\ \ \ \ \ \ \ \ and\ \ \ \ $\sigma_n=(\phi', \chi')$.
\item
$\M, \sigma_i \leadsto \sigma_{i+1}$,\ \ \ \ \ \ \ for $1\leq i \leq n\!-\!1$,
\item
 $\sigma_i = (\psi'\cdot\phi'', \_)$,\ \ \ \ for $2\leq i \leq n\!-\!1$, some frame $\phi''$, and some non-empty stack $\psi'$ 
\end{itemize}
\end{itemize}
\end{definition}

\sdf{Next, we} define the relation $\text{access}\langle\_,\_\rangle$,
similar to $\CanAccess{\_}{\_}$ in \Chainmail, to \sdf{express} that %encode when 
 two objects are connected,
 %However, the receiver and parameters can gain access to any object they initially only have indirect access to. If one of their fields has access to an object, this object could be passed up, either directly if they are of the same class, or otherwise by a method invocation. This could happen to any depth, and so any $n^\text{th}$ child field must be considered. We therefore define a transitive access predicate, $\text{access}^*\langle\_,\_\rangle$, to encode this property.
 \sdf{%The accessibility relation holds between an object and another object, 
either  through a field access, if the first object   
is the receiver and the second object the argument, or %if it can do that  
 through the transitive closure of the $\text{access}$ relation.}

\begin{definition}[\Mrrz{Access relations}]
\label{def:access_rel} \Mrr[Defintion now given]{Given runtime configuration $\sigma$, and addresses $\alpha$ and $\alpha'$, we define access and transitive access as:}

\begin{itemize}
\item $\text{access}_{\sigma}\langle\alpha,\alpha'\rangle \triangleq \alpha = \alpha' \vee \exists f. (\sigma(\alpha, \f) = \alpha')\ \  \vee\ \  \alpha = \interp{\prg{this}}{\sigma} \wedge \exists x. (\alpha' = \interp{\x}{\sigma})$
\item \Mrr[I have written up the formal definition for this, but left it out as it's obvious and takes up space]{$\text{access}^*_{\sigma}\langle\alpha,\alpha'\rangle$ is the transitive closure of $\text{access}_\sigma\langle\alpha,\alpha'\rangle$}
%\item $\text{access}^*_{\sigma}\langle\alpha,\alpha'\rangle \triangleq \exists k (\text{access}^k_\sigma\langle\alpha,\alpha'\rangle)$ \IFF $\text{access}^n_\sigma\langle\alpha, \alpha'\rangle$, $n \ge 1$ such that
%\begin{itemize}
%\item $\text{access}^n_\sigma\langle\alpha,\alpha'\rangle = \begin{cases}
%\text{access}_\sigma\langle\alpha,\alpha'\rangle & n = 1\\
%\exists\alpha''.(\text{access}_\sigma^{(n-1)}\langle\alpha,\alpha''\rangle \wedge \text{access}\langle\alpha'',\alpha'\rangle) & n \ge 1
%\end{cases}$
%\end{itemize}
\end{itemize}
\end{definition}

\vspace{.01in}
\sdf{The remainder of this section outlines the proof of lemma \ref{lemma:method_call_connectivity},}
%To prove lemma~\ref{lemma:method_call_connectivity}, 
for which we use a series of auxiliary lemmas. The first, lemma~\ref{lemma:local_variable_connectivity}, 
%concerns local variables, and 
says \sdf{that} \Mrr{under constrained} \jm{execution}, %that 
if an object is referred to by a local variable in any frame,
 then the initial receiver, \prg{this}, was transitively connected to it.

\begin{lemma}[Local variable connectivity]
\label{lemma:local_variable_connectivity}
\begin{align*}
\forall M. \forall \chi \forall \phi. \forall \psi,\psi'.\forall \x. \forall \alpha \in \chi. &\ [ \ \ \M,(\psi,\chi) \lceil\leadsto^*\rceil (\psi' \cdot \psi, \_) \wedge \phi \in \psi' \wedge \interp{\x}{\phi} = \alpha \longrightarrow \\
&\quad \text{access}^*_{(\psi,\chi)}\langle \interp{\prg{this}}{(\psi,\chi)}, \alpha\rangle\ \ \ ]
\end{align*}
\end{lemma}

%The strengthening of this lemma 
\sdf{To prove this lemma we strengthen it to all frames of the stack instead of simply the top frame, 
and then apply induction.}
% enables its inductive proof to handle the \prg{return} case. 
\sdf{For the} \prg{return} case we need to prove that the initial receiver can access the variables in the new top frame, which was previously the second top frame.
% Knowing \Mrr{the receiver can access local variables in all frames inductively renders this step} straightforward.  
\Mrr{We use constrained \jm[was ``reduction'']{execution} to prevent the computation from returning from the initial method, which would render the initial receiver irrelevant.}

We then prove that if two objects are connected by the final configuration, either the initial \prg{this} had access to both, or they were initially connected.
\Mrr{
\begin{lemma}[Receiver connectivity]
\label{lemma:receiver_connectivity}
\begin{align*}
\forall \M. \forall \sigma, \sigma' \forall \alpha, \alpha' \in \sigma. &\ [ \ \ \M,\sigma \lceil\leadsto^*\rceil \sigma' \wedge \text{access}_{\sigma'}\langle \alpha, \alpha'\rangle\ \  \longrightarrow\\
&\quad \text{access}^*_\sigma \langle \interp{\prg{this}}{\sigma}, \alpha\rangle \vee \text{access}_\sigma\langle{\alpha, \alpha'}\rangle\ \ \ ]
\end{align*}
\end{lemma}
}

This follows % intuitively
 from lemma~\ref{lemma:local_variable_connectivity}, since any connections made must occur through local variables by the grammar of \LangOO.
 \sdf{We can now prove lemma~\ref{lemma:method_call_connectivity} 
%  From  \sdf{lemma \ref{lemma:receiver_connectivity}}  we   prove the % original lemma~\ref{lemma:method_call_connectivity} 
 by noticing that in \sdf{$\sigma'$} % the next step,
  the receiver % will have 
  \sdf{has} access to the parameters, and by applying lemma~\ref{lemma:receiver_connectivity}.}
