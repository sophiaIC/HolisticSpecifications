%Susan:Please read first bit as I have just written it
%\se{When you write a module that is to be used with other code, the last thing you want to happen is that some other code uses it to cause effects that you never intended. Our specification language \Chainmail has been designed, so that developers whose modules are going to be used in the wild, have the language to constrain the usage of their code. In addition to classical function by function specification techniques, we have shown that a holistic or whole program approach is needed to make open world code robust. We have shown} 
% going to the old one, as running out of space.
% also, the new one brings new words, and I think all th words in concluson should have appeared earlier
In this paper we have motivated the need for holistic specifications,
presented the specification language \Chainmail for writing such
specifications, and outlined the formal foundations of the language.
%% , and shown 
%% how it can be used to give holistic
%% specifications of key exemplar problems: the bank account,  the
%% wrapped DOM, the ERC20, and and the DAO.
%
To focus on the key attributes of a holistic specification language,
% we have tried to keep the
\sd{we have kept  \Chainmail simple, only requiring an understanding of first order logic.}
\sd{We believe that the holistic features (permission, control, time, space and viewpoint)
are intuitive concepts %for ptogrammers. 
when reasoning informally, and were pleased to have been able to provide their
formal semantics in what  we  argue is a simple manner.}
We have built upon and improved the foundations of \Chainmail presented in prior work~\cite{FASE}.

\jm{The variant of \Chainmail presented in this paper is simpler than that of 
earlier formulations~\cite{FASE}. Prior definitions of \Chainmail allowed programmers to 
write assertions using local variables, introducing significant overhead in the 
way of variable renaming. The \Chainmail described in this paper allows assertions to be written using addresses of objects on the heap. Since addresses are immutable during execution 
there are no required rewrites. This modification primarily simplifies the metatheory 
of the temporal logic, and especially aids in the Coq formulation of that metatheory.}

\jm{
We have build upon prior work on \Chainmail, in particular 
this paper represents a
 simplified version of \Chainmail.
 We have designed a linear temporal logic for the 
semantics of assertion satisfaction for program configurations. 
Satisfaction now takes the new form $\M\mkpair \M', \sigma_0 , \sigma \models  \A$.
Thus, satisfaction is defined for a specific 
program execution path (starting from ($\sigma_0$, with current program configuration $\sigma$)).
Branching of programs is still considered, but is reserved to module satisfaction which is defined as satisfaction of all paths from all possible starting program configurations.
Linearity in \Chainmail's temporal logic has three primary advantages: 
\begin{enumerate}
\item
\textbf{Distributivity of Disjunctions}: 
Disjunctions are distributive over temporal operators, 
allowing for a degree of normalization of assertions.
\item
\textbf{Change of Satisfaction}: 
Change of satisfaction over time can be identified as 
being caused by a specific program step. This property 
is key in proving satisfaction several of \Chainmail's 
key motivating examples.
\item
\textbf{Simplification of metatheory}:
Pushing the topic of branching in the temporal logic
up to the level of module satisfaction greatly simplifies 
the metatheory, and in particular the encoding of that 
metatheory in Coq. It is considerably easier to prove 
properties of a specific program execution path than
of all possible program execution paths when reasoning 
about metatheory at the level of the semantics of satisfaction.
\end{enumerate}}

\jm{
A crucial concept to ensure secure and robust programs
is that of connectivity of different parts of the heap.
Section \ref{sect:connectivity} develops the metatheory 
of connectivity as it pertains to \Chainmail and the 
underlying language \LangOO.
}

\jm{
In Section \ref{sect:discussion} have detail our work on how recursion is 
included as part of \Chainmail assertions. Limiting recursion to ghost fields, 
and thus to expressions, ensures that \Chainmail is able to include recursion
while retaining classical properties that might otherwise be lost in the presence 
of infinite recurstion.
}

\jm{
We have further encoded our simplifications to \Chainmail
in a Coq model that consists of the syntax and semantics
of \Chainmail, along with proofs of satisfaction of 
the Bank Account example (see Section \ref{sect:motivate:Bank}),
and a variant of the Safe example (see Section \ref{sect:exampleSafe}).
}
% below not true, we do have recusrions  
%do not even support recursive procedures to avoid circularities in the
%metatheory, let alone concurrency, exceptions, distribution,
%networking, etc. 

%% \sd{The development of the semantics of \Chainmail assertions posed several interesting 
%% challenges, \eg the treatment of the open world requires two-module execution
%% and the concept of external objects,
%% recursion is confined to ghostfields and assertions require termination of included expressions,
%% space required the concept of restricting runtime configurations,
%% and time required adaptation operators which apply bindings from one configuration to another.}  

%% \sd{\Chainmail is powerful enough to express many key examples from the
%% literature; nevertheless, it lacks several important features: It provides 
%% recursion  only in a restricted form, it has a rather inflexible notion of
%% module and does not support hierarchies of modules, and knows nothing about
%% concurrency or distribution.  We plan to remove these restrictions by applying
%% techniques such as step-indexing \cite{stepindex}, but hope to keep any mathematical 
%% sophitsication in the
%% model of \Chainmail without exposing it to the person who writes the specification.  We are also
%%  interested in extending \Chainmail\ to situations
%% where internal modules are typed, but the external modules are untyped.
%% %
%% We also plan to extend \Chainmail to support reasoning about
%% conditional trust in programs, and to quantify the risks involved in
%% interacting with untrustworthy software \cite{swapsies}.
%% }

%% \sd{To make these kinds of specifications
%% practically useful,  we plan to develop logics for proving adherence of module's code to holistic specs, as well
%% as logics for using holistic specs in the proof of open programs. We want to develop 
%% dynamic monitoring  and model checking techniques for our specifications. 
%% And finally, we plan to automate reasoning with these logics.}
