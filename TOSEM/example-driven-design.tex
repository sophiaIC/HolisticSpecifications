%proposed replacement for discussion

%\kjx{could be two subsetions or one section}

%\subsection{Examplars}

The design of \Chainmail was guided by the study of a sequence of
exemplars taken from the object-capability literature and the smart
contracts world:

\begin{enumerate}
\item \textbf{Bank} \cite{arnd18} - Bank and accounts as described in
Section~\ref{sect:motivate:Bank}, with two different implementations given in appendix C
\item
\textbf{ERC20} \cite{ERC20} - Ethereum-based token contract
\item
\textbf{DAO} \cite{Dao,DaoBug} - Ethereum contract for Decentralised Autonomous
Organisation
\item
\textbf{DOM} \cite{dd,ddd} - Restricting access to browser Domain Object Model\\
\end{enumerate}\vspace{-1em}

\subsection{Authorising ERC20}
\label{sect:example:ERC20}
{ 
 ERC20~\cite{ERC20} is a widely used token standard which describes the 
 basic functionality expected by any    Ethereum-based token contract. 
 It issues and keeps track of participants' tokens, and supports the  transfer
 of tokens between participants. 


An important question, therefore, is to identify the precise circumstances under which a transfer of tokens may take place.
The answer is that transfer of tokens 
 can only take place only provided that  there were sufficient tokens in the
 owner's account, and that
 the transfer was instigated by the owner, or by somebody authorized
 by the owner.

We can use holistic specifications to express our understanding of
these circumstances, by specifying the necessary conditions for a
decrease in a participant's balance, and the necessary conditions for
authorization.

A decrease in  a participant's balance 
(\ie  $\prg{e}.\prg{balance}=...\, \wedge\, \Next{\prg{e}.\prg{balance}=...}$)
can only be caused by a transfer instigated by the 
account holder themselves (\ie $\Calls{\prg{p}, ...}$), or by
a transfer instigated by another participant $\prg{p}''$  (\ie $\Calls{\prg{p}''...}$) who 
has authority for more than the tokens spent(\ie  $\prg{e}.\prg{allowed}(\prg{p},\prg{p}'')\geq \prg{m}'$)
This is described by the following policy, which
binds any \prg{e} which is an \prg{ERC20}  contract:

\vspace{.15cm}
\noindent
% \strut \hspace{0.3cm} 
$\forall \prg{e}:\prg{ERC20}.\forall \prg{p}:\prg{Object}.\forall \prg{m},\prg{m}':\prg{Nat}.$\\
\strut \hspace{0.3cm} $[\ \ \prg{e}.\prg{balance}=\prg{m}+\prg{m'}\ \wedge \ \Next{\prg{e}.\prg{balance}=\prg{m}'}$ \\ %.\forall\prg{m}:\prg{Nat}.$\\
\strut \hspace{0.4cm} \ \ \ $\longrightarrow$\\
\strut \hspace{0.4cm} \ \ \ $\exists \prg{p}',\prg{p}'':\prg{Object}.$ \\
\strut \hspace{0.4cm} \ \ \  $[\ \  \Calls{\prg{p},\prg{e.transfer(p',m)},\_} \  \  \ \vee\, $\\
\strut \hspace{0.4cm} \ \ \   $\ \ \ \ \prg{e}.\prg{allowed}(\prg{p},\prg{p}'')\geq \prg{m} \ \wedge \ \Calls{\prg{p}'',\prg{e.transferFrom(p,p',m),\_}}\       \  ]$\\
\strut \hspace{0.3cm} $] $
\vspace{.15cm}

\noindent
That is to say: if next configuration witnesses a decrease of the balance by
 $\prg{m}'$, then the current configuration was a call of \prg{transfer} instigated by
 the owner, or it was a call of \prg{transferFrom} instigated by somebody authorized.
Note the term \prg{e}.\prg{allowed}(\prg{p},\prg{p}''), which means that the
ERC20 variable \prg{e} holds a field called \prg{allowed} of   mapping type, which maps pairs of participants to numbers; such
mappings are supported in Solidity\cite{Solidity}, but could also be
 understood as ghost fields or predicates\footnote{cite ghost
 field\kjx{actually, be clear about the context, \Chainmail seems to
 have its own ghost fields, so we just use them without apology}}.
Also, % allow for fields which are mappings, as supported
we use an underscore ($\_$) to indicate some value or variable of no importance; thus the term $\Calls{\prg{p}'',\prg{e.transferFrom(p,cl',m),\_}}$ is a convenient shorthand for 
$\exists \prg{m}''.\Calls{\prg{p}'',\prg{e.transferFrom(p,cl',m),m''}}$. \kjx{thjis
 shorthand should go into the overview of chainmail, if we want to
 keep it. add a ``notational conventions'' section or something.}

We now define what it means for $\prg{p}$ to have authorized $\prg{p}'$ to  spend 
up to \prg{m} tokens on the behalf of $\prg{p}$: At some point in the
past,  \prg{p} gave authority to $\prg{p}'$  to spend   \prg{m}
plus the sum of  tokens
spent so far by $\prg{p}' $ on the behalf of \prg{p}. 

 
\vspace{.15cm}
\noindent
 $\forall \prg{e}:\prg{ERC20}.\forall \prg{p},\prg{p'}:\prg{Object}.\forall \prg{m}:\prg{Nat}.$\\
\strut \hspace{0.3cm} $[\ \ \prg{e}.\prg{allowed}(\prg{p},\prg{p}')=\prg{m} $\\
\strut \hspace{0.4cm} \ \ \ $\longrightarrow$\\
\strut \hspace{0.4cm} \ \ \  
     $\PrevId\langle\ \  \Calls{\prg{p},\prg{e}.\prg{approve}(\prg{p}',\prg{m})} $\\
      \strut \hspace{1.7cm} \ $\vee $\\
\strut \hspace{1.7cm} \  
     $    \prg{e}.\prg{allowed}(\prg{p},\prg{p}')=\prg{m}   
        \  \wedge\ $\\
\strut \hspace{1.5cm} \ \ \ \ \          $  \neg   (\, \Calls{\prg{p}',\prg{e.transferFrom(p,\_,\_)},\_ }\, \vee \, \Calls{\prg{p},\prg{e}.\prg{approve}(\prg{p}',\_)\ }\, ) $\\
      \strut \hspace{1.7cm}\  $\vee $\\
\strut \hspace{1.7cm}   \  $ \exists \prg{p}'':\prg{Object}.\exists\prg{m'}:\prg{Nat}.$\\
 \strut \hspace{1.7cm}\  $[\   
  \prg{e}.\prg{allowed}(\prg{p},\prg{p}')=\prg{m}+\prg{m}'  \, \wedge\,   \Calls{\prg{p}',\prg{e.transferFrom(p,p'',\prg{m}')},\_ }\  ]$\\
\strut \hspace{0.4cm} \ \ \  \ \ \  \ \ \ \ \ $\rangle $\\
\strut \hspace{0.3cm} $]$
\vspace{.15cm}
 
In more detail, $\prg{p}'$ is allowed to spend 
up to \prg{m} tokens on their behalf of $\prg{p}$, if in the immediately previous step either a)
 \prg{p} made the call \prg{approve} on \prg{e} 
with arguments $\prg{p}'$ and \prg{m}, or b)  
$\prg{p}'$ was allowed to spend  up to \prg{m} tokens for $\prg{p}$
and did not transfer any of \prg{p}'s tokens, nor did \prg{p} issue a fresh authorization,
or c) \prg{p} was authorized for $\prg{m}+\prg{m}'$ and spent $\prg{m}'$ 
\kjx{This seems to be MUCH NICER than the previous versions. Still
stepwise, but a better style of definition}.  
 

 Thus, the holistic specification gives to account holders an
 "authorization-guarantee": their balance cannot decrease unless they
 themselves, or somebody they had authorized, instigates a transfer of
 tokens. Moreover, authorization is {\em not} transitive: only the
 account holder can authorise some other party to transfer funds from
 their account: authorisation to spend from an account does not confer
 the ability to authorise yet more others to spend also.
 
\paragraph{Comparison with Traditional Specifications}
 
 Traditional  specifications %for the ERC20 example would consist of
 describe the behaviour of each function separately.
 They  consist of pre- and post- conditions for each function; the pre-condition is 
 a {\em sufficient} condition for the effect of the particular function.
 Usually, for each function, we need one specification for the case 
 where the function's pre-condition holds, another for the case where it does not.\footnote{ARGH! 
 Here we use pre-condition with two different meanings! What to do?}
 \kjx{HUh? why?  we support/need this, but traditional contracts,
 Hoare, Meyer \& co, don't!}  
With traditional  specifications, to obtain the "authorization-guarantee", 
one would need to inspect the pre- and post- conditions of {\em all} the functions
in the contract, and determine which of the functions decrease balances, and which of the functions 
 affect authorizations.
 In the case of the \prg{ERC20}, one would have to inspect all eight such specifications, 
 where only five are relevant to the question at hand.
 In other cases, \eg the DAO, the number of the functions which are unrelated
 to the question at hand can be very large.
 Moreover, with the traditional specification, the "authorization-guarantee" can only be 
 obtained if  all the functions have pre- and post-conditions which we
 trust while this is not necessaty for the
 holistic case.
 
More importantly, with traditional  specifications, nothing stops the next release of the contract to add, 
\eg, a method which allows participants to share their authority, and thus
violate the "authorization-guarantee", or even a super-user from skimming 0.1\% from each of the accounts.

In Figure \ref{fig:classicalERC20} we outline a traditional specification for the \prg{ERC20}.
We give two speficiations for \prg{transfer}, another two for \prg{tranferFrom}, and one for all 
the remaining functions. The  first specification says, \eg, that if  
 \prg{p} has sufficient tokens, and it calls \prg{transfer}, then the transfer will take place.  
The second specification says that  if \prg{p} has insufficient tokens, then 
the transfer will not take place (we assume that in this
specification language, any entities not mentioned in the pre- or post-condition 
are not affected).\footnote{cite framing -- but anyway, our subject is
not the classical spec.} \kjx{again why. classical specs just say WTF
happens if preconditions are met, and intentionally do not say what
happens if they are not met.  Two specs is one of our innovations,
isn't it. One we dont want to talk about here, but...}
 
 Similarly, we would have to give another two specifications to define the behaviour of 
if \prg{p''} is authorized and executes \prg{transferFrom}, then   the balance decreases. 
But they are {\em implicit} about the overall behaviour and the   {\em necessary} conditions,
e.g., what are all the possible actions that can cause a decrease of balance?


 
\begin{figure}   
\fbox{
$
\begin{array}{c}
 \prg{e}:\prg{ERC20}\ \wedge\  \prg{p},\prg{p''}:\prg{Object} 
  \wedge\ \prg{m},\prg{m}',\prg{m}'':\prg{Nat}\ \wedge\   \\
 \prg{e}.\prg{balance(p)} = \prg{m}+\prg{m}'\ \ \wedge\ \ \prg{e}.\prg{balance(p'')} = \prg{m}''\ \ \wedge\ \ \prg{this}=\prg{p} \\
   \{ \ \ \prg{e.transfer(p'',m')} \ \ \}   \\
    \prg{e}.\prg{balance(p)} = \prg{m}\ \ \wedge\ \ \prg{e}.\prg{balance(p'')} = \prg{m}''+\prg{m}'
\ \ \\
\ \ \\
  \prg{e}:\prg{ERC20}\ \wedge\  \prg{p},\prg{p'}:\prg{Object}  \wedge\ \prg{m},\prg{m}',\prg{m}'':\prg{Nat}\ \wedge\     \prg{e}.\prg{balance(p)} = \prg{m} \ \ \wedge \prg{m} <  \prg{m}'  \\
   \{ \ \ \prg{e.transfer(p',m')} \ \ \}   \\
  \prg{e}.\prg{balance(p)} = \prg{m}  
  \\
  \\
\prg{e}:\prg{ERC20}\ \wedge\  \prg{p},\prg{p'},\prg{p}'':\prg{Object} 
  \wedge\ \prg{m},\prg{m}',\prg{m}'',\prg{m}''':\prg{Nat}\ \wedge\   \\
 \prg{e}.\prg{balance(p)} = \prg{m}+\prg{m}'\ \ \wedge\ \ \prg{e}.\prg{allowed(p,p')}=\prg{m}'''+\prg{m}' \ \wedge\\
  \prg{e}.\prg{balance(p'')} = \prg{m}''\ \ \wedge\ \ \prg{this}=\prg{p'} \\
   \{ \ \ \prg{e.transferFrom(p',p'',m')} \ \ \}   \\
    \prg{e}.\prg{balance(p)} = \prg{m}\ \ \wedge\ \ \prg{e}.\prg{balance(p'')} = \prg{m}''+\prg{m}'
     \ \wedge\ \ \prg{e}.\prg{allowed(p,p')}=\prg{m}'''
\ \ \\
\ \ \\
  \prg{e}:\prg{ERC20}\ \wedge\  \prg{p},\prg{p'}:\prg{Object}  \wedge\ \prg{m},\prg{m}',\prg{m}'':\prg{Nat}\ \wedge\ \prg{this}=\prg{p}' \ \wedge \\
      ( \ \prg{e}.\prg{balance(p)} =\prg{m} \wedge \prg{m} <  \prg{m}''\  \vee \ 
  \prg{e}.\prg{allowed(p,p')}=\prg{m'} \wedge \prg{m'} < \prg{m}'' \ ) \\
   \{ \ \ \prg{e.transferFrom(p,p'',m'')} \ \ \}   \\
  \prg{e}.\prg{balance(p)} = \prg{m} \wedge  \prg{e}.\prg{allowed(p,p')}=\prg{m'}
  \\
  \\
  \prg{e}:\prg{ERC20}\ \wedge\  \prg{p},\prg{p'}:\prg{Object}  \wedge\ \prg{m}:\prg{Nat}\ \wedge\ \prg{this}=\prg{p}  \\
   \{ \ \ \prg{e.approve(p',m')} \ \ \}   \\
  \prg{e}.\prg{allowed(p,p')} = \prg{m} 
  \\
  \\
   \prg{e}:\prg{ERC20}\ \wedge\ \prg{m}:\prg{Nat}\ \wedge\    \prg{p}.\prg{balance}=\prg{m}    \\
   \{ \ \ \prg{k}=\prg{e.balanceOf(p)} \ \ \}   \\
  \prg{k}=\prg{m} \ \wedge \ \prg{e.balanceOf(p)} = \prg{m}  
  \\
  \\
   \prg{e}:\prg{ERC20}\ \wedge\ \prg{m}:\prg{Nat}\ \wedge\    \prg{e}.\prg{allowed(p,p')}=\prg{m}    \\
   \{ \ \ \prg{k}=\prg{e.allowance(p,p')} \ \ \}   \\
  \prg{k}=\prg{m} \ \wedge \ \prg{e}.\prg{allowed(p,p')}=\prg{m} 
  \\
  \\
   \prg{e}:\prg{ERC20}\ \wedge\ \prg{m}:\prg{Nat}\ \wedge\     \sum_{\prg{p}\in dom(\prg{e}.\prg{balance})}^{}{\prg{e}.\prg{balance}(\prg{p})}=\prg{m}    \\
   \{ \ \ \prg{k}=\prg{e.totalSupply()} \ \ \}   \\
  \prg{k}=\prg{m}   
\end{array}
$
}
\caption{Classical specification for the \prg{ERC20}}
\label{fig:classicalERC20}
\end{figure}
}

\subsubsection{ERC20, the traditional specification}
\label{ERC20:appendix}
 
\mrr[We compare the holistic and the traditional specification of ERC20]{The following can be an appendix, but I've provisionally moved it up to try and avoid using appendices as asked to}

As we said earlier,  the holistic specification gives to account holders an
 "authorisation-guarantee": their balance cannot decrease unless they
 themselves, or somebody they had authorised, instigates a transfer of
 tokens. Moreover, authorisation is {\em not} transitive: only the
 account holder can authorise some other party to transfer funds from
 their account: authorisation to spend from an account does not confer
 the ability to authorise yet more others to spend also.
 
 With traditional  specifications, to obtain the "authorisation-guarantee", 
one would need to inspect the pre- and post-conditions of {\em all} the functions
in the contract, and determine which of the functions decrease balances, and which of the functions 
 affect authorisations.
In Figure \ref{fig:classicalERC20} we outline a traditional specification for the \prg{ERC20}.
We give two specifications for \prg{transfer}, another two for \prg{tranferFrom}, and one for all 
the remaining functions. The  first specification says, \eg, that if  
 \prg{p} has sufficient tokens, and it calls \prg{transfer}, then the transfer will take place.  
The second specification says that  if \prg{p} has insufficient tokens, then 
the transfer will not take place (we assume that in this
specification language, any entities not mentioned in the pre- or post-condition 
are not affected).
 
 Similarly, we would have to give another two specifications to define the behaviour of 
if \prg{p''} is authorised and executes \prg{transferFrom}, then   the balance decreases. 
But they are {\em implicit} about the overall behaviour and the   {\em necessary} conditions,
e.g., what are all the possible actions that can cause a decrease of balance?


 
\begin{figure}   
\fbox{
$
\begin{array}{c}
 \prg{e}:\prg{ERC20}\ \wedge\  \prg{p},\prg{p''}:\prg{Object} 
  \wedge\ \prg{m},\prg{m}',\prg{m}'':\prg{Nat}\ \wedge\   \\
 \prg{e}.\prg{balance(p)} = \prg{m}+\prg{m}'\ \ \wedge\ \ \prg{e}.\prg{balance(p'')} = \prg{m}''\ \ \wedge\ \ \prg{this}=\prg{p} \\
   \{ \ \ \prg{e.transfer(p'',m')} \ \ \}   \\
    \prg{e}.\prg{balance(p)} = \prg{m}\ \ \wedge\ \ \prg{e}.\prg{balance(p'')} = \prg{m}''+\prg{m}'
\ \ \\
\ \ \\
  \prg{e}:\prg{ERC20}\ \wedge\  \prg{p},\prg{p'}:\prg{Object}  \wedge\ \prg{m},\prg{m}',\prg{m}'':\prg{Nat}\ \wedge\     \prg{e}.\prg{balance(p)} = \prg{m} \ \ \wedge \prg{m} <  \prg{m}'  \\
   \{ \ \ \prg{e.transfer(p',m')} \ \ \}   \\
  \prg{e}.\prg{balance(p)} = \prg{m}  
  \\
  \\
\prg{e}:\prg{ERC20}\ \wedge\  \prg{p},\prg{p'},\prg{p}'':\prg{Object} 
  \wedge\ \prg{m},\prg{m}',\prg{m}'',\prg{m}''':\prg{Nat}\ \wedge\   \\
 \prg{e}.\prg{balance(p)} = \prg{m}+\prg{m}'\ \ \wedge\ \ \prg{e}.\prg{allowed(p,p')}=\prg{m}'''+\prg{m}' \ \wedge\\
  \prg{e}.\prg{balance(p'')} = \prg{m}''\ \ \wedge\ \ \prg{this}=\prg{p'} \\
   \{ \ \ \prg{e.transferFrom(p',p'',m')} \ \ \}   \\
    \prg{e}.\prg{balance(p)} = \prg{m}\ \ \wedge\ \ \prg{e}.\prg{balance(p'')} = \prg{m}''+\prg{m}'
     \ \wedge\ \ \prg{e}.\prg{allowed(p,p')}=\prg{m}'''
\ \ \\
\ \ \\
  \prg{e}:\prg{ERC20}\ \wedge\  \prg{p},\prg{p'}:\prg{Object}  \wedge\ \prg{m},\prg{m}',\prg{m}'':\prg{Nat}\ \wedge\ \prg{this}=\prg{p}' \ \wedge \\
      ( \ \prg{e}.\prg{balance(p)} =\prg{m} \wedge \prg{m} <  \prg{m}''\  \vee \ 
  \prg{e}.\prg{allowed(p,p')}=\prg{m'} \wedge \prg{m'} < \prg{m}'' \ ) \\
   \{ \ \ \prg{e.transferFrom(p,p'',m'')} \ \ \}   \\
  \prg{e}.\prg{balance(p)} = \prg{m} \wedge  \prg{e}.\prg{allowed(p,p')}=\prg{m'}
  \\
  \\
  \prg{e}:\prg{ERC20}\ \wedge\  \prg{p},\prg{p'}:\prg{Object}  \wedge\ \prg{m}:\prg{Nat}\ \wedge\ \prg{this}=\prg{p}  \\
   \{ \ \ \prg{e.approve(p',m')} \ \ \}   \\
  \prg{e}.\prg{allowed(p,p')} = \prg{m} 
  \\
  \\
   \prg{e}:\prg{ERC20}\ \wedge\ \prg{m}:\prg{Nat}\ \wedge\    \prg{p}.\prg{balance}=\prg{m}    \\
   \{ \ \ \prg{k}=\prg{e.balanceOf(p)} \ \ \}   \\
  \prg{k}=\prg{m} \ \wedge \ \prg{e.balanceOf(p)} = \prg{m}  
  \\
  \\
   \prg{e}:\prg{ERC20}\ \wedge\ \prg{m}:\prg{Nat}\ \wedge\    \prg{e}.\prg{allowed(p,p')}=\prg{m}    \\
   \{ \ \ \prg{k}=\prg{e.allowance(p,p')} \ \ \}   \\
  \prg{k}=\prg{m} \ \wedge \ \prg{e}.\prg{allowed(p,p')}=\prg{m} 
  \\
  \\
   \prg{e}:\prg{ERC20}\ \wedge\ \prg{m}:\prg{Nat}\ \wedge\     \sum_{\prg{p}\in dom(\prg{e}.\prg{balance})}^{}{\prg{e}.\prg{balance}(\prg{p})}=\prg{m}    \\
   \{ \ \ \prg{k}=\prg{e.totalSupply()} \ \ \}   \\
  \prg{k}=\prg{m}   
\end{array}
$
}
\caption{Classical specification for the \prg{ERC20}}
\label{fig:classicalERC20}
\end{figure}


\subsection{Defending the DAO}
\label{Dao:appendix}
The DAO ~\cite{DAO}  is a famous Ethereum contract  aiming  to support
collective management of funds,  and to place power directly in the hands of owners the DAO 
rather than delegate to directors. 
Nevertheless, a re-entrancy bug   exploited in June 2016, lead  to a loss of   \$50M, and
a hard-fork in the  chain ~\cite{DaoBug}. 
 
%In a similar style as that  of the ERC20 spec earlier, 
%We can give a \Chainmail~specification 
Holistic specifications allow us to write specifications which avoid the bug, and guarantee that under all circumstances,
all owners will be able to be reapid their money.
Namely, we
requiring that  % holds as much ether as the sum of
% the   its clients' balances, and
 the owners'  balances may only be affected by clients joining or leaving, and
projects being approved or repaying.
We also require,
as below, that % Consider a  simplified version of the DAO~\cite{DAO}:
% It keeps the moneys of a set of clients, and will refund them when they call the function \prg{repay}. 
%
% The \RoSpec~policy   from below says:  
if  \prg{p}  has a balance of \prg{m} at a \prg{DAO} contract \prg{d},
and if \prg{p} calls \prg{repay} on \prg{d}, then  
 \prg{d} is required to hold at least \prg{m} ether at the time of that call ($\prg{d}.\prg{ether}\geq \prg{m}$), and will eventually send \prg{m} back to \prg{p} (expressed as $\Future{\Calls{\prg{d.send(p)},m}}$:

%Formally: 
\vspace{.07cm}

\noindent  
% \prg{Pol\_DAO\_withdraw} \ $\equiv$ \\ 
\strut \hspace{0.5cm} $\forall \prg{d}:\prg{DAO}.\forall \prg{p}:\prg{Any}.\forall\prg{m}:\prg{Nat}.$\\
\strut \hspace{0.5cm} $[\ \  \Calls(\prg{p},\prg{d.repay(),\_})\, \wedge\, \prg{d.Balance(p)}=\prg{m} $\\ 
\strut \hspace{0.5cm} \ \ \ $\longrightarrow$\\
\strut \hspace{0.5cm} \ \ \ $\prg{d}.\prg{ether}\geq \prg{m}\ \wedge$ $\ \Future(\Calls(\prg{d.send(p)},\prg{m}))\ \ ] $ 

\noindent
The requirement that \prg{d} holds at least \prg{m} ether precludes the DAO bug,
in the sense that  any contract satisfying that spec cannot exhibit  the  bug:   a contract
which satisfies the spec  is guaranteed to always have enough money to satisfy all \prg{repay} requests.
This guarantee  holds, regardless of how many functions there are in the DAO.
In contrast, to preclude the DAO  bug with a classical spec, one would need to write a spec for each of the 
DAO functions (currently 19) and then study their emergent  behaviour. 


\vspace{.005cm}

We can now define  what it means for \prg{p} to have a  \prg{Balance} at  \prg{d}. The \prg{Balance}  is \prg{0} if the previous call was
a repayment; it is \prg{m} if  the previous call was \prg{p} joining \prg{d} and paying in \prg{m}. More cases are needed to reflect the financing and repayments of proposals, but they can be expressed with the concepts described so far.

\noindent
$\strut \hspace{0.5cm} \forall \prg{d}:\prg{DAO}.\forall \prg{p}.\forall:\prg{m}:\prg{Nat}.$\\
$\strut \hspace{0.5cm} [ \ \ \  $\\
$\strut \hspace{0.5cm} \ \  \ \  \prg{d.Balance(p)}=\prg{m}$\ $\longrightarrow$ \ $ \left\{
                            \begin{array}{ll}
                             \prg{0}, & \hbox{if}\ Prev(Call(\prg{p},\prg{d.repay(),\_})    \\
                             \vee
                             \\
                             \prg{m},  & \hbox{if}\  Prev(Call(\prg{p},\prg{d.join(),m}))   \\
                             ..., & ... 
                           \end{array} 
                         \right.    $\\
$\strut \hspace{0.5cm} ] $                         

\subsection{Attenuating the DOM}
\label{sect:example:DOM}
%We will now discuss  an example demonstrating the need to describe attenuation in specifications.
\emph{Attenuation} is the ability to give restricted access to an object's functionality. Thus is usually done thorugh
introduction of an intermediate object. While it is a common programming practice the term was coined, and the practice 
was studied in detail in the object capabilities literature \cite{millerPhD}. 

The example of attenuating the DOM was proposed in \cite{dd}. 
In this section we revisit that example, and use it to motivate the need for holistic specifications, to give an informal introduction
to  our language for such holistic specifications. We also argue that compared with the soecifcation in \cite{dd}, our specification xxxx.

This example deals with a tree of DOM nodes: Access to a DOM node gives access to all its parent and children nodes, and the ability to modify the node's properties. However, as the top nodes of the tree usually contain privileged information, while the lower nodes contain less crucial information, such as xxxx, we want to be able to limit  the access that we give  to third parties to only part of the DOM tree. We do this through \prg{Wrapper}, which has a field \prg{node} pointing to a \prg{Node}, and a field \prg{height} which restricts the range of \prg{Node}s which may be modified through the use of the particular \prg{Wrapper}. Namely, when you hold a \prg{Wrapper}  you can modify the \prg{property} of all the descendants    and also all the \prg{height}-th ancestors of the \prg{node} of that particular \prg{Wtrapper}.

The corresponding code appears in Figure \ref{fig:DOM}. 
We write our code in a small fictitious class-based, dynamically typed, imperative programming language. All functions are public, \ie may be called by any other object, and all fields are private, \ie may be read or written only by the object itself. 

\begin{figure}[htb]
\begin{tabular}{llll}
&
\begin{minipage}{0.40\textwidth}
\begin{lstlisting}
class Node(par,prop){

  fld parent = par;
  fld property = prop;
  fld children = new array[10];
  fld maxChild=0;
  children[maxChild++]=this
  
  func getParent(){
       parent 
  }  
  func getChild(i){
       children[i] 
  }
  func setProperty(prp){
       property = prp 
  }  
  func setChild(nd){
       children[i] = nd
  }  
}
\end{lstlisting}
\end{minipage}
& & 
\begin{minipage}{0.40\textwidth}
\begin{lstlisting}
class Wrapper(nd,hgt){

  fld node=nd;
  fld height=hgt;

  func setPropety(i,prp){
    if (i>height){ 
       return 
    } 
    else  
    {  nd=node;  
       while (i>0){
          nd=nd.getParent();
          i--;    };
        nd.setProperty(prp); }
  }    
  func getChild(i){ 
    Wrapper(node.getChild(i),
                    height+1); 
 }                           
}
\end{lstlisting}
\end{minipage}
\end{tabular}
 \vspace*{-7mm}
\caption{DOM - \prg{Node}  and \prg{Wrapper} }
\label{fig:DOM}
\end{figure}

In Figure we show an example of the use of  \prg{Wrapper} objects to attenuate the use of \prg{Node}s\footnote{SD: Is that a good sentence?}  The function \prg{usingWrappers} takes as parameter an object of unknown provenance, here called \prg{unknwn}. On lines 2-7 it creates a tree, consisting of nodes \prg{n1}, \prg{n2}, ... \prg{n6}, depicted as blue circles on the   right-hand-side of the Figure. On line 8 it creates a wrapper of \prg{n5} with height \prg{1}. This means that the wrapper \prg{w} may be used to modify \prg{n3}, \prg{n5} and \prg{n6} (\ie the objects in the green triangle), while it cannot be used to modify \prg{n1}, \prg{n2}, and \prg{4} (\ie the objects within the blue triangle). 

On line 8 we call the function \prg{untrusted} on the \prg{unknown} object, and pass \prg{w} as an argument. This means that \prg{unknown} obtains access to \prg{w}, which in turn has transitively access to all  \prg{Node}-s in the tree. Nevertheless, we know that the call to the \prg{untrusted} function is guaranteed no to affect the values of the \prg{property} field of the nodes \prg{n1}, \prg{n2}, and \prg{4}. Thus, the assertion on line 10 is guaranteed to succeed.

\begin{figure}[htb]
\begin{tabular}{llll}
&
\begin{minipage}{0.50\textwidth}
\begin{lstlisting}
func usingWrappers(unknwn){
   n1=Node(null,"fixed"); 
   n2=Node(n1,"robust"); 
   n3=Node(n2,"volatile"); 
   n4=Node(n2,"const");
   n5=Node(n3,"variable");
   n6=Node(n3,"ethereal");
   w=Wrapper(n5,1);
   unknwn.untrusted(w);
   assert  n2.property == "robust" 
   ...
}
\end{lstlisting}
\end{minipage}
& & 
\begin{minipage}{0.75\textwidth}
\includegraphics[width=\linewidth, trim=145  320 60 105,clip]{diagrams/DOM.pdf}
% x y z w
% y seems to eat up the bollom
% y=320 is good
% x eats space from left, if you increase it the diagram decreases from left
% w eats space from top, if you increase it the diagram decreases from top
% w=100 is good
%\includegraphics[page=3, width=\linewidth, trim=150  270 40 150, clip]{diagrams/snmalloc.pdf}\sdcomment{I think we need to change the diagram so that it says small slab.}
\end{minipage}
\end{tabular}
 \vspace*{-1mm}
\caption{\prg{Wrapper}s protecting \prg{Node}s }
\label{fig:WrapperUse}
\end{figure}
 



%\subsection{Model}
\section{Model}
\label{sect:model}

We have constructed a Coq model\footnote{A current model can be found at: https://github.com/sophiaIC/HolisticSpecifications} \cite{coq} of the core of the Chainmail
specification language, along with the underlying \LangOO language.
Our formalism is organised as follows:
\begin{enumerate}
\item
The \LangOO Language: a class based, object oriented language with mutable references.
\item
Chainmail: The full assertion syntax and semantics defined in Definitions \ref{def:execution:internal:external}, \ref{def:arise}, \ref{def:valid:assertion:access}, \ref{def:valid:assertion:control}, \ref{def:valid:assertion:view}, \ref{def:restrict}, \ref{def:valid:assertion:space}, \ref{def:config:adapt}, \ref{def:valid:assertion:time} and \ref{def:module_satisfies}.
\item
\LangOO Properties: Secondary properties of the loo language that aid in reasoning about its semantics.
\item
Chainmail Properties: The core properties defined on the semantics of Chainmail.
\end{enumerate}

%We also formalise several of the properties defined in this paper. 
\sophia{In the associated appendix} (see Appendix \ref{sect:coq}) we list and present the properties of Chainmail we have formalised in Coq.
We have proven that Chainmail obeys much of the properties of classical logic. While we formalise most of the underlying semantics, we make several assumptions in our Coq formalism: (i) the law of the excluded middle,  a property that is well known to be unprovable in constructive logics, and (ii) the equality of variable maps and heaps down to renaming. Coq formalisms often require fairly verbose definitions and proofs of properties involving variable substitution and renaming, and assuming equality down to renaming saves much effort.

More details of the formal foundations of \Chainmail, and the model,
are also in appendices \cite{examples}.

