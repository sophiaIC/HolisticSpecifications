
% for the section on Spec^X

The first form says that $A$ always holds, \textit{i.e.} it is a invariant; 
it consists of assertions,  $A$, supporting the usual expressions about program state,
%%(e.g. \prg{x.f > 3}),
  logical connectives and quantifiers, 
%%(e.g. $\wedge$, $\forall$), 
 and additional predicates
 to capture \textit{provenance} and \textit{access}.
 %(whether an object $o$'s class comes 
%$\internal{\texttt{o}}$ or $\external{\texttt{o}}$) to the current
%module, and \textit{permission} \cite{miller-esop2013} (whether an
%object $o$ has direct access to another object $o'$:
%$\access{\texttt{o}}{\texttt{o'}}$).
The second form says that  a change from a current state satisfying $A_{curr}$ to a future
state satisfying $A_{fut}$ %(i.e.\ the transition  $A_{curr}$ to $A_{fut}$ being an effect) 
is possible only if the necessary condition
$A_{nec}$ holds in the current state.
%
%(i.e.\ $(A_1 \wedge \Diamond A_2) \longrightarrow A_3$).
The third form says that a change from a current state satisfying $A_{curr}$ to a future
state satisfying $A_{fut}$  is possible only if the necessary condition
$A_{nec}$ holds in some intermediate state.
 Note, that unlike \citeauthor{VerX} or \citeauthor{FASE}
 the necessity operators $\onlyIf {\_} {\_} {\_}$  and $\onlyThrough {\_} {\_} {\_}$
 are second class; they may nor appear in the assertions $A$.
 
 In section on proofs
  Thus, 
   a method's sufficient conditions are used to infer a method's and effect's necessary conditions.
 
 In section on compar with Verx
 \susan[logic sentence breaks flow and I would omit]{}$(A_1 \wedge \Diamond A_2) \longrightarrow A_3$ says that  $A_3$ is a necessary condition to
 transition from $A_1$ to $A_2$.