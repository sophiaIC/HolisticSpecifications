\documentclass[11pt]{amsart}
\usepackage{geometry}                % See geometry.pdf to learn the layout options. There are lots.
\geometry{letterpaper}                   % ... or a4paper or a5paper or ... 
%\geometry{landscape}                % Activate for for rotated page geometry
%\usepackage[parfill]{parskip}    % Activate to begin paragraphs with an empty line rather than an indent
\usepackage{graphicx}
\usepackage{amssymb}
\usepackage{epstopdf}

\newcommand{\rev}[1]{\emph #1}
\newcommand{\us}[1]{\bf #1}
\newcommand{\prg}[1]{\texttt{#1}}
\DeclareGraphicsRule{.tif}{png}{.png}{`convert #1 `dirname #1`/`basename #1 .tif`.png}

\usepackage{filecontents}
\begin{filecontents}{Response1.bib}
@article{Grossman,
author = {Grossman, Shelly and Abraham, Ittai and Golan-Gueta, Guy and Michalevsky, Yan and Rinetzky, Noam and Sagiv, Mooly and Zohar, Yoni},
title = {Online Detection of Effectively Callback Free Objects with Applications to Smart Contracts},
year = {2017},
issue_date = {January 2018},
publisher = {Association for Computing Machinery},
address = {New York, NY, USA},
volume = {2},
number = {POPL},
url = {https://doi.org/10.1145/3158136},
doi = {10.1145/3158136},
journal = {Proc. ACM Program. Lang.},
month = {dec},
articleno = {48},
numpages = {28},
keywords = {Program analysis, Modular reasoning, Smart contracts}
}
@article{Albert,
author = {Albert, Elvira and Grossman, Shelly and Rinetzky, Noam and Rodr\'{\i}guez-N\'{u}\~{n}ez, Clara and Rubio, Albert and Sagiv, Mooly},
title = {Taming Callbacks for Smart Contract Modularity},
year = {2020},
issue_date = {November 2020},
publisher = {Association for Computing Machinery},
address = {New York, NY, USA},
volume = {4},
number = {OOPSLA},
url = {https://doi.org/10.1145/3428277},
doi = {10.1145/3428277},
journal = {Proc. ACM Program. Lang.},
month = {nov},
articleno = {209},
numpages = {30},
keywords = {blockchain, program verification, program analysis, logic and verification, smart contracts, invariants}
}

@article{Permenev,
  title={VerX: Safety Verification of Smart Contracts},
  author={Anton Permenev and Dimitar I. Dimitrov and Petar Tsankov and Dana Drachsler-Cohen and Martin T. Vechev},
  journal={2020 IEEE Symposium on Security and Privacy (SP)},
  year={2020},
  pages={1661-1677}
  }
  
  @inproceedings{Drossopoulou,
author="Drossopoulou, Sophia and Noble, James and Mackay, Julian and Eisenbach, Susan",
editor="Wehrheim, Heike and Cabot, Jordi",
title="Holistic Specifications for Robust Programs",
booktitle="Fundamental Approaches to Software Engineering",
year="2020",
publisher="Springer International Publishing",
address="Cham",
pages="420--440",
}

@misc{Chlipala,
  author = "Adam Chlipala",
  title = "Certified Programming with Dependent Types",
 url = "http://adam.chlipala.net/cpdt/",
 year = 2019,
 date = "2021-12-02",
}

%@misc{Chlipala,
%  author = "Adam Chlipala",
%  title = "frap: Formal Reasoning about Programs",
% url = "https://github.com/achlipala/frap",
% year = 2019,
% date = "2021-12-02",
%}

\end{filecontents}


\usepackage{natbib}

\title{Response for Paper 30: Necessity Specifications are Necessary for Robustness }
%\author{The Author}
%\date{3 December 2021}                                           % Activate to display a given date or no date

\begin{document}
\maketitle
\section{Overview}
We sincerely thank the reviewers for detailed and thoughtful comments, and for the opportunity this gives us to explain our work better. 



\subsection*{The fundamental argument}

Our fundamental argument is that \prg{Mod1} and \prg{Mod3} are "fit for purpose", while \prg{Mod2} is not -- hope all reviewers agree on that.
Moreover, that (S2), i.e. an account's balance does not decrease unless transfer was called with the correct password, is satisfied by all three modules; namely, (S3) does not preclude \prg{Mod2}, which allows the password to be overwritten. On the other hand, (S3), ie that the balance of an account does not decrease ever in the future unless some external object has access to the account's current password, is satisfied by the "good" modules \prg{Mod1} and \prg{Mod3} and not by the "bad" module \prg{Mod2}. This is, we argue, sufficient motivation to study Necessity specifications. 

Reviewer D argues further that (S3) is not strong enough when we operate in an environment where an external module has access to the password. Indeed, in such a case, all bets are off. However, (S3) does guarantee that if the account is passed to a context where no external object has access to the password, then the balance will not be modified. Namely, the "active" external objects  in the context of the callee are a subset of the "active" external objects  in the context of the caller.  In further work, we plan to extend Hoare logics, so as to make this guarantee formal.

There are some subtle distinctions between Necessity specifications, which we summarized in section 3.4.1, but should explain in more detail. 

In particular, \prg{NecessityBankSpec}$_c$, defined as 
\texttt{from a:Account $\wedge$ a.balance==bal to a.balance < bal
 onlyIf $\exists$ o.[$\langle$o external$\rangle$ $\wedge$ $\langle$o calls a.transfer(\_, \_, \_)$\rangle$]}
, mandates that if the balance should decrease after any number of execution steps, then the \emph{first} step was a call  to \prg{transfer}. This is clearly not satisfied by \emph{any} of the three modules; consider, as a counterexample, the code \prg{b=new Account; c=new Accoiunt; x=x+1; a.transfer(...); c=new Account}. This code \emph{does} call \prg{transfer}, but not in the first step.

On the other hand,   \prg{NecessityBankSpec}$_d$, defined as 
\texttt{
from a:Account $\wedge$ a.balance==bal to a.balance < bal
onlyThrough $\exists$ o.[$\langle$o external$\rangle$ $\wedge$ $\langle$o calls a.transfer(\_, \_, \_)$\rangle$]
}, is slightly weaker, in that it mandates that if the balance should decrease after \emph{any} number of execution steps, then the \emph{some} step was a call  to \prg{transfer}.  

Finally, we could add a further variation, \prg{NecessityBankSpec}$_e$, defined as 
\texttt{from a:Account $\wedge$ a.balance==bal next a.balance < bal
 onlyIf $\exists$ o.[$\langle$o external$\rangle$ $\wedge$ $\langle$o calls a.transfer(\_, \_, \_)$\rangle$]}
  PLEASE write this out, which would mandate  that if the balance should decrease after \emph{one} execution step, then that step was a call  to \prg{transfer} and the right password was passed. This is satisfied by all three modules, and is, in fact, the formal description of $(S2)$.

We feel the concerns fall into three areas. The technical concern is that Necessity does not yet support calls of external methods from within internal modules. There is a level of contribution concern as to how different it is from Chainmail and VerX. Finally, there are several areas where the presentation of our ideas should be made clearer. The changes we describe in this response could comfortably be made before the February deadline.

\subsection*{External calls}

Necessity does not --yet-- support calls of external methods from within internal modules. This is, indeed, a limitation, but it is not uncommon in the related literature. For example, VerX~\citep{Permenev} work on effectively call-back free contracts, while~\citep{Grossman} and ~\citep{Albert} on drastically restricting the effect of a callback on a contract. Therefore, we argue that a treatment of  external calls in Necessity would bring some further complexity, and would  detract from the main focus of our paper.  

\subsection*{Novelty}

Our Necessity operators are novel and we do not believe that our {\tt{onlyThrough}} operator can be encoded into either VerX or Chainmail, or using traditional temporal logic operators. Neither VerX, Chainmail, or temporal logic operators, are able to refer to two program states in an execution, and then another third program state that lies between them. This is a core component of necessary preconditions, describing necessary intermediate operations or program states required to achieve specific outcomes.

Whereas both VerX and Necessity deal with protecting code from unknown code, VerX is Smart Contracts specific whereas Necessity is not domain specific.  The technology used is also different: VerX is a model checker whereas Necessity programs are proven using Coq.

It is true that some of the Necessity definitions, and their encodings, are inspired by Chainmail~\citep{Drossopoulou}, 
%and as such there may be some similarities between the two formalisms, but these similarities do not extend to 
but the contributions of our paper go much farther.
The Necessity language and proof system, 
the soundness result, and the proven examples have no equivalents in Chainmail.
% of the proof system. 
%Where there are similarities, these either have been explicitly stated in the paper (eg. permission, provenance, and control), or are common coq idioms (eg. variable substitution).
For the Coq code both Chainmail and Necessity use Chlipala's CpdtTactics library~\citep{Chlipala}, but do not share any other code.

\subsection*{Presentation}

\paragraph{\textbf{Adaptation}} is indeed an important auxiliary concept   used in Def. 3.10 and 4.2, but is not a central contribution. 
We motivated the adaptation operator $\triangleleft$  in lines 413-424, but we should have given more detail:
The specification on lines 416-417 talks about 
three different program states. That specification only makes sense if variable $a$ denotes
the \emph{same} object in all three states, even though it is possible that in these states the contents of the variable 
$a$ has changed (eg through an assignment of the form $a:=...$, or even because $a$ got out of scope).
%We propose the following amended explanation to clarify both it's importance, and it's meaning:
`%`We deal with this via an \emph{adaptation} operator~\cite{Drossopoulou}. We write $\adapt {\sigma'} {\sigma}$   to view a future state $\sigma'$ from the perspective of a current  (or past) state $\sigma$.

Thus, $\triangleleft$ is a variable renaming operator; it ensures that variable names 
used at one point in the execution refer to the same object at a future point in the execution. 
%This is necessary as, within assertions, objects are referenced using variable names, and 
%not by addresses (unique,  immutable identifiers).
% During program execution, variables may be either overwritten or lost from scope, and as such using the same variable name at 
%different points in execution has little meaning unless there is a way to rewrite variables so that their intended meaning can 
%be preserved. Adapting one program state with a second allows variables in the second state to point to the same object that was in the first.
For example, for any variable $x \in dom(\sigma)$ we have that $\lfloor x \rfloor_\sigma$ = $\lfloor x \rfloor_{\sigma' \triangleleft \sigma}$, even though it is possible that $\lfloor x \rfloor_\sigma \neq \lfloor x \rfloor_{\sigma'}$, and that
$\lfloor x.f \rfloor_\sigma \neq \lfloor x.f \rfloor_{\sigma' \triangleleft \sigma}$.

%SD START please leave that in the text. I thought I could finish it, but now I wonder 
%Reviewer D proposes the use of auxiliary logical variables ..
%
%> To connect the "from" and "to" conditions, it uses a rather complex physical mechanism instead of using logical auxiliary variables.
%
%Indeed, we could, instead, use auxiliary variables, and do the renaming in the assertions. Thus, we would define \prg{from} ... \prg{to} ... as follows
%
%$M \models  \prg{from} A_1 \prg{to} A_2 \prg{onlyIf} A_3$ if .... 
%
%\begin{itemize}
%\item
%M, \sigma \models A_1$
%\item
%M, \sigma' \models A_2[\overline{z]/\overline{u}
% SD end 

%Def.\ref{d:adapt} shows how $\adapt{\sigma'}{\sigma}$, constructs a new
%state, taking the heap and most of the stack from the future state 
%$\sigma'$. We replace the top frame's variable map 
%with the variable map from the top frame of the past state $\sigma$,
%avoiding name clashes by renaming the 
%variables in the top frame of $\sigma'$ with fresh variables
%($\overline{v}$) and renaming free variables in the continuation similarly.''

\paragraph{\textbf{Access}}  Access is not deep, and only refers to objects that an object has direct access to via a field or within the context of the current scope. 
A transitive definition of access would not be useful in specifying safe and robust software. The restricted form of access used in Necessity
specifically captures a crucial property of robust programs in the open world: \emph{access to an object does not imply access to that object's internal data}.
For example, an object may have access to an account \texttt{a}, but a \emph{safe} implementation of the account
would never allow that object to leverage that access to gain direct access to \texttt{a.password}.
Necessity is thus concerned with if and how objects are able to gain direct access to an object, and not deep, transitive access.
Indeed, if access were defined transitively, then many objects would be defined as having access to objects that they could not gain a direct reference to, 
and as such render \texttt{<x access y>} as almost meaningless, and any safety specifications written using access to be prohibitively restrictive.

\paragraph{\textbf{Assertion encapsulation}} captures a property that is essential to proofs of safety in the open world: certain operations may only occur as the result of internal module method calls, thus, the satisfaction of properties that depend on such operations may only change as a result of internal module code. The simplest such operation in a Java-like language would be the mutation of a field of an object of an internal class. Satisfaction of assertions about the value of such a field may only change as a result of internal code being executed. Assertions are therefore not ``encapsulated'' by some arbitrary code, but rather by the internal module, and thus only programs that contain method calls to the internal module are able to invalidate those assertions. In the reviewer's example where C' = C; x := x, if C contains internal method calls that invalidate some assertion A, then so does C'.

(TODO: Sophia)

\paragraph{\textbf{Emergent Behaviour}} 
%The reviewer correctly identifies that we did not sufficiently indicate what we meant by ``emergent behavior''. We propose the follow change 
%to Section 1, where ``emergent behavior'' is first mentioned:
Reviewer A had difficulty understanding what we meant by emergent behaviour but surmised that it meant that  ``no single procedure call is capable of breaking the necessity specification, but a sequence of calls can very well be''. That is correct.

``(S2) does not take account of the module's \emph{emergent behavior}. That is, (S2) does not consider the behaviour that emerges from the interaction between the 
\texttt{transfer} method, and the other methods of the bank module. What if the module leaks the password?''

(Julian: Move the above text to the propoesed changes)

\paragraph{\textbf{Bank Account example}} The formal proof of the bank account example is very heavy weight given how straightforward the introductory example is. 
This is because we wished to show off more sophisticated features of Necessity:
\begin{itemize}
\item
Proofs involving complex properties expressed using ghost fields, and not just proofs on field values.
\item
Proofs of a more complex module and data structures, where emergent behavior arises across multiple 
classes. The ability to write proofs of an entire module with many interoperating classes is an important strength of Necessity.
\end{itemize}

 \section{Change List}
  We will make all the minor changes suggested by the reviewers.
 \subsection*{External calls}
 We cannot promise a full treatment of external calls by the end of February, but we can share out current thinking:  As a first approach, we will require that the arguments to external calls do not include internal objects, except for the receiver and parameters (thus ensuring that external accessibility of internal methods does not increase); we would rely on the classical pre- and post- conditions of the internal methods -- as we currently do. As a more advanced approach, we will develop extensions to classical Hoare Logics, which would allow us to reason about points in the code where external calls are being made. This would be the first time we could be inspecting the code in the bodies of the functions.
 \subsection*{Novelty}
 We will strengthen our statements about VerX and Chainmail in line with what we said above.
 
 \subsection*{Presentation}
 
 For adaption, access, and encapsulation we will amend the explanations as stated above. Susan: or do you want to discuss Julian's cleaner definition for adaption he sent yesterday???
 
 For emergent behaviour we will include the reviewer's statement and also say that ``(S2) does not take account of the module's \emph{emergent behaviour}. That is, (S2) does not consider the behavior that emerges from the interaction between the 
\texttt{transfer} method, and the other methods of the bank module. What if the module leaks the password?''
 
 We will replace the current Bank Account proof with a simpler Coq proof that matches the straightforward introductory example. We will put the current example in an appendix so that we can 
show reasoning about ghost fields and more complex data structures. 

We will move the clarifying examples to Section 2.

The largest piece of work is the proof and that shouldn't take more than a week so we believe that we can make substantial improvements in presentation before mid January.


\subsubsection*{Reviewer A}
\begin{itemize}
\item Streamline Introduction: we will move the related work to the related work section
\item Rework Section 2 to be clearer: we will make the outline of the proof structure in Section 2 clearer, at a higher level, and more concise.
\item We will change the order of $M$ and $M'$ in the definition of Arising.
\item We will make clear which state is the original state.
\item We will clarify Def. 3.9, provide a clearer description and definition.
\item Ensure consistent usage of Section vs. section.
\end{itemize}

\subsubsection*{Reviewer B}
\begin{itemize}
\item Fix the flow of the paper. Present a ``consistent high-level story``
\item Clarify the differences between Necessity and Chainmail and VerX.
\item Be more explicit about the reasons and justifications for restricting external method calls
\item Provide better names for Mod1, Mod2, Mod3, etc
\item Emphasize the separation of Necessity from the inspection of code.
\item Clarify why shallow access is necessary
\item Restate NecessityBankSpec in 3.4.1
\item Provide better justification, explanation, and intuition for the example specifications in 3.4
\item Include a brief description of the expressiveness earlier in the paper than 3.4.3
\item Explain why the restriction on return values is sufficient in If1-Inside
\end{itemize}

\subsubsection*{Reviewer C}
\begin{itemize}
\item Rephrase the liveness and safety verification in the Section 1.
\item Rewrite Section 2.4. (Julian: Reviewer B appreciated this, but neither C nor A did)
\item Replace Section 5 with the simpler, original bank account example, and move the current one to the appendix.
\end{itemize}


  
\section{Response} We provide a reviewer-by-reviewer list of answers to questions. We are planning to implement as requested answers any questions that have been omitted.
  
  \subsection*{Reviewer 30A}
  
{\rev {{In 36-55, the introduction suddenly becomes/mixes with a part on related work. To my opinion, this completely distracts from the description of the problem that this paper is attacking. I'd rather that this part is moved to the dedicated related work section.}}}
{\us{{We will be re-organising the introduction along with Section 2 to more directly describe the problem tackled by the paper. Discussion of the related work will be moved to the related work section.}}}

{\rev {{ 68. At this point (and also many points later), it was completely unclear to me what "emergent behaviour" is supposed to mean. Part of the reason is certainly that I am not a native English speaker, but when I translate emergent to my mother tongue, the translation does not make sense to me. The closest explanation I can find in the English wiktionary says for emergent "Having properties as a whole that are more complex than the properties contributed by each of the components individually." But this is so vague that I am having trouble to imagine what emergent means in this context.}}}

{\rev {{    The way I understood what you mean by (unwanted) emergent behaviour is that no single procedure call is capable of breaking the necessity specification, but a sequence of calls can very well be. But this is not what I would have guessed from "emergent" or "Having properties as a whole that are more complex than the properties contributed by each of the components individually." }}}
{\us{The reviewer is correct in the meaning of the word emergent. We have proposed a refined and more descriptive way to introduce ``emergent behavior''. 
(Julian: do we need to mention this? We have already discussed it earlier.)}}

{\rev {{102: I have two problems with the notion of being  `encapsulated': }}}

{\rev {{ Maybe a minor comment, actually, but again - I'm sorry - I do not understand the name "encapsulated". The fact that the concept >>only by executing a pice of code C one can invalidate a logical assertion A<< is called >>C encapsulates A<< does not make much sense to me because "encapsulating" - to me - suggests rather that A is somehow part of C or that A is somehow wrapped/surrounded by C. }}}

{\rev {{ More severely, I still cannot get my head around the notion that only by executing C one can invalidate A. Imagine $C' = C; x:=x$. Obviously $C' \neq C$, but if $C$ can invalidate A, so can $C'$. How can there be a piece of code $C$, so that only $C$ but not $C; x:=x$ can invalidate A? }}}
{\us{We have expanded on the description of ``assertion encapsulation''. Hopefully this clears up any misunderstanding.}}

{\rev {{167 Here, one slowly understands that emergent = putting multiple calls in sequence. }}}

{\rev {{176-181: Is what you describe here really inherent to your necessity logic? Isn't it rather a consequence of - where - you check the validity of your specifications? }}}
{\us{{(Julian: I'm not entirely sure what to put here.)}}}

{\rev {{182: I find the term "not monotonic" misplaced here. Isn't it obvious that adding more behavior can invalidate more stuff? }}}
{\us{Perhaps this is obvious, although it is an important point, and one that is not true of many existing specification languages where specifications are restricted to a single 
function. Proofs in Necessity take a holistic view of a module.}}

{\rev {{192: One could draw a comparison to loop invariants: While pushing a loop invariant through a loop body, the invariant can (and likely will) also break intermediately. The point is just that it holds before and after the execution of the loop body, just like you specifications hold before and after the call. }}}

{\rev {{199: "the executing object (this) is always external". Without your formal semantics presented later, this sentence (snippet) is very confusing. Intuitively, "this" is always an internal thing because it is the name that an objects gives to itself, internally. }}}
{\us{Yes, \texttt{this} is always internal to the executing object, however, in this paper, when we use the words ``internal'' and ``external'', we are generally referring ``internal'' or ``external'' to 
a specific module. We will amend the paper to make this clearer.}}

{\rev {{ Section 2.4, I must say, I find extremely tedious and very difficult to follow. I firmly believe that this can be streamlined to that one must not mentally follow through 9 steps (a - i). Many notions are also only explained later. E.g. it is not clear (and also does not become apparent from the explanations in Sec. 2.4) why one needs to construct from 'per-method conditions' the 'single-step conditions'. And is this really important in order to get an overview of the approach that the paper is taking? }}}
{\us{Julian: thoughts?}}

 {\rev {{  In 206, it is again stated that an assertion A, i.e. a logical formula A, can be encapsulated by a module. Later, in l.211, it is said that 'balance' is encapsulated. But balance is a term, not a formula. What does it mean to invalidate a term? I don't get it. }}}
 {\us{As in other specification languages, expression (including both fields and ghost fields) are themselves assertions. While it is true that it usually would 
 not make sense to say that the field \texttt{balance}, it's equality to some value could be invalidated.}}

{\rev {{I214.5: "Per-method conditions are necessary conditions for given effect and given single, specified, method call." I do not understand what this sentence is telling me. In particular, I really don't understand what an effect is here. What exactly is the difference between call, step, effect, ...? }}}

{\rev {{I243 says "Note that our proofs of necessity do not inspect method bodies" This makes absolutely no sense to me. How can I infer - or more: 'prove' - anything about an object C (the code) without looking at C? This needs an explanation. In the explanation that follows you mention pre and postconditions of methods, but how can I prove pre- and postconditions of methods, if I cannot look at the methods? }}}
{\us{Necessity proofs do not inspect method bodies. Necessity does rely on traditional pre- and post- conditions which, as you say, does rely on inspecting code. Pre- and post- conditions are an area that is well researched, and thus such specifications can be outsourced to existing logics and tools. In fact Necessity is parametric with such a system, and does not impose further code inspection.}}

{\rev {{253.5: What is an "unsurprising" language?}}}
{\us{TooL is a very simple imperative, class based, object oriented language, whose semantics should not surprise anyone.}}

{\rev {{I263 following: At this point I was wondering, which of Mod1, Mod2, Mod3 is internal, what is external? It would be good to refer back to that example and point out to the reader what is supposed to be internal and what external. }}}
{\us{Necessity is a specification for programs in the open world. In this paper we characterize the ``internal'' as the module we are specifying, and the ``external'' as the unknown open world. 
Under this characterization, \texttt{Mod1}, \texttt{Mod2}, and \texttt{Mod3} all represent different implementation of a bank account being specified, and thus
are all ``internal'' with respect to the open world.}}

 {\rev {{IDef. 3.2: Why would one write Arising(M, Y, sigma) iff ... Y; M, sigma-0 ... Why flip the order of M and Y? Does that not cause unnecessary confusion? Or is there a good reason to flip the order? }}}
 {\us{This is a good point. We will make this change.}}

{\rev {{I 349-350 why should x be fresh in sigma? }}}
{\us{x is fresh in sigma because it is used to refer to the object being quantified. It is alpha (the location in the heap) that should not be fresh in sigma.}}

{\rev {{I393-394: "If an arising state ... then the original state must have ..." What is the "original state"? The arising one? Or the initial state that arises from the definition of arising states? If it is the former, I'd suggest "If an arising state $\sigma$ ... then $\sigma$ must have...". }}}
{\us{Yes, the original state is arising. We will make this change and clarify this in the text.}}

{\rev {{Def. 3.8: It would be good to ostentatiously clarify that necessity specifications cannot nest, i.e. the nonterminal S does not appear on the right-hand side of the grammar. Only nonterminals A and those come from the language Assert, I suppose. }}}
{\us{Yes, that is correct. We will make this change to the text.}}

{\rev {{417: It is undefined what  - means. There exists a value? For any value? It depends on the formula? }}}
{\us{Yes, \_ means for any value.}}

{\rev {{ Def. 3.9: Again, to me this is absolutely central. It definitely needs - more- explanations and a lot of intuition. See also above. Technically, I am not 100\% sure what the colon operator does in l.431 and 432. Is that a concatenation of lists? If yes, please say so. While being very nitpicky: should {local ..., contn...} not be a tuple instead of a set? Less nitpicky: You primed all names from $\sigma'$ except for $\psi$. It would be much less distracting $\psi$ was called $\psi'$. }}}

{\rev {{442-456: It is totally unclear how/why the $\triangleleft$-operator does the trick for you necessity modalities. Part of the reason is that the definition $\triangleleft$ is described/explained in not enough detail (for me). But I also firmly believe that the definition of the semantics of the necessity modalities deserves to be provided some intuition. In particular an intuition of how $ \triangleleft$ defines these semantics. }}}
{\us{We refer the reviewer to G for our explanation of adaptation.}}

{\rev {{482: I believe that it deserves an explenation why no module satisfies NecessityBankSpec-c. It is not obvious to me. }}}
{\us{For an account's balance to change after some number of execution steps, \texttt{a.transfer} must be called in some intermediate state.
It is not necessary that that call to \texttt{a.transfer} must occur in the first program state of that chain (as implied by OnlyIf).}}

{\rev {{Sec. 3.4.2: This example is completely inaccessible and handwavy to me. }}}

{\rev {{ Definition 4.1: I have (almost) the same problem as with l.442-456 here. Why almost? Because here there is the half-sentence "we have to interpret one assertion in two different states" that provides a little bit of intuition on how $\triangleleft$ helps in this definition. But it is not enough intuition for me. }}}
{\us{We refer the reviewer to G for our explanation of adaptation.}}

   
  \subsection*{Reviewer 30B}
  
   68: the focus on emergent behavior is good. This is why I like the motivation for this work - emergent behavior is very important in real software systems, but a lot of verification/PL work does not reason about it explicitly (or at all).

90: this is the first example I noticed of a weird transition. The authors go straight from saying that there are three new operators to a lot of detail on the first of those. The paper here would flow better if the authors gave an informal description of all three operators (to give the reader an intuition for what’s coming next) rather than going straight into gory detail.
{\us{We will rework the Introduction to be less technical, and better introduce the concepts within the paper.}}

99: the authors state that necessity operators are second-class, but don’t really justify this choice or explore its consequences, and it is never returned to. I’m not sure if the necessity operators being second-class is actually the right choice, and especially at this point in the paper, where I as a reader don’t yet fully understand them, this statement throws me off.
{\us{Julian: thoughts?}}

200: when you make an assumption like this one, please justify it to the reader rather than simply saying “Note ...“ Especially for such an important assumption, as this one, it is unsatisfying as a reader to be left wondering why you have done this.
{\us{We refer the reviewer to G for our explanation of our choice around external method calls.}}

205: you refer to Mod1, Mod2, Mod3, etc. many times, but their names are not descriptive at all. This was the point where I got mildly annoyed at having to go back and check which implementation Mod3 was. I suggest renaming the modules to something that describes their properties, e.g. Mod3 could become “SafePwdSet” or something like that.
{\us{We will make this change.}}

243: you might consider emphasizing this point more: it is a strength of your approach that it doesn’t make many assumptions about how these annotations are checked, and so is compatible with lots of existing work.
{\us{Yes, we will highlight this more when we rework the presentation of the paper.}}

335: thanks for defining your shorthands rather than just using them and making us guess what they mean :)

351: I have a serious concern about definition 8 (x access y) here. It seems to me that it is saying that if y’s value is exactly the value of one of x’s fields (f), then x can access y. That seems reasonable, but I don’t see any notion of multi-step access paths here, which seems suspicious to me. After all, intuitively I would expect that (x access y) would be true if for example the value of the expression x.f.g is the same as the value of y (and so on for any arbitrary number of field accesses). Did I miss something about your setup/language that forbids fields themselves having fields? Or is it the case that (x access y) is defined to be false in a case like the one I mentioned above? If so, that seems to me like a “soundness” problem in the sense that it violates what I as a specification writer would expect that specification to mean, and therefore might result in incorrectly-specified code.
{\us{We refer the reviewer to G for our explanation of access.}}

382: inside is a very useful concept. I might consider mentioning it earlier, perhaps by working it into one of the examples in section 2.

459: this is another case where the flow of the text is jarring. There is no explanation for why now is the appropriate time to consider more examples of specifications. This might make more sense as a separate section rather than as a subsection, with a short justification saying something like “we now present some examples to give the reader an idea of the expressiveness of our approach”; you didn’t explain that that was the goal of the example section in this draft until the very end of the section!
{\us{Thank you for pointing this out. We will amend this in our presentation rework.}}

463: I suggest re-stating the original NecessityBankSpec here so that it’s easier to compare with the a-d variations.
{\us{We will make this change.}}

486: please explain what you want the reader to take away from these examples! Why is it important that e.g. both Mod2 and Mod3 satisfy b and d, but neither satisfy c? Explain to the reader why the changes that you made from the original cause these changes in what can satisfy the specification, rather than just stating it as fact. Do not make the reader guess what you want them to understand.

571: I’m not convinced by your claim here that all you need to do is ensure that objects of a confined type only must never be returned from method bodies. What if a confined type is accessible from the field of a returned object? Maybe your type system forbids this (I expect that it does!), but the presentation of it here is a bit confusing, perhaps just because you’ve given so few details.
{\us{Indeed, the type system does forbid it. We will add clarity here.}}

619: can you comment on how hard it would be to modify your system to support this? It seems like it is important if you want to use this system to validate a realistic program.
{\us{It is hard to say how difficult it would be to add this, but it would likely require some work developing aspects of a more traditional specification language to detect 
and specify certain kinds of external method calls. It is notable that other work in the area makes the assumption that methods are ``effectively call-back free''~\citep{Permenev}. 
This would likely guide our approach as a starting point.}}

819: it is not easy to see this unless you introduce another data structure, like a list, to hold the arbitrarily-many accounts. Since you don’t allow internal code to call into external libraries, you’d have to define the list yourself, so this would actually be quite the project.
{\us{Technically the restriction does not allow calls to unknown, and thus unspecifiable, code. If the List library code was known, 
and specifiable then we could extend the example. Even if we did allow for external method calls, it would not be possible to write
specifications about those calls as the effects of such a call would be unknown.}}

1036-1042: there seems to be a lot of overlap between the present work and Chainmail. You might want to spend another sentence or two here discussing the differences.
{\us{We will add this to our revision.}}
  
 
  \subsection*{Reviewer 30C}
  

Section 1: I'm not sure if paraphrasing liveness and safety is a good
idea because liveness/safety verification in the traditional sense is
also reasoning about sufficient conditions for good things to
eventually happen or for bad things to never happen.  The point
here is the distinction between sufficient and necessary conditions
about the behavior of a program.

Section 2.4 doesn't work very well (at least for a first reading)
because it is written in a bottom-up manner.  I had no idea why
assertion encapsulation is the first step because I didn't have a big
picture how necessity specification might be verified.  Explaining backwards from Part 4 to 1
may work better (but I'm not sure...).

It seem to me that it can be hard to show assertion encapsulation
because of universal quantification over all external modules and
states.  The existence of a type system and a proof system for
assertion encapusulation, which is discussed in 4.1.3, is nice and
plausible but I find the discussion on Enc(A) handwaving and the
derivation in Section 5.1 is hard to follow.  I think you should
explicitly state that the bank account example is typed by the assumed
simple type system for confinement.

I'm not sure why Section 5 extends the simple bank account example so
much.  Is the extension needed to demonstrate the verification
framework, or you thought the examples in Section 2 would be too simple? 
Maybe they are simple but I think it is a good exercise to show verification of the first examples. 
It was hard for me to follow the extended example as a first verification exercise.

 \subsection*{Reviewer 30D}
 
The direction of this research is important. However, the advances that this paper makes do not seem strong enough and the work looks rather premature to me for the following reasons.

First, the way to connect the "from" and "to" conditions seems to involve unnecessary complications, which may restrict the applicability of the work to other languages. Updating physical states using fresh variables sounds too restrictive. For example, in assembly, one cannot create fresh registers. Here it is unclear to me why the authors cannot use logical auxiliary variables as done in modern separation logics.

Second, the allowed control flow between the external and internal modules is restricted (ie, the internal module cannot invoke external modules). Since reasoning about the internal module should be essentially independent of external modules (because external modules are completely arbitrary), similar reasoning as presented in this paper should be possible even when the internal module invokes functions of external modules.

Finally, more importantly, although the paper claims the (S3) condition instead of (S2) as an advance over Chainmail, the condition (S3) is not so convincing. There are two concerns. First, (S3) does not say anything useful in the presence of an external module that has access to the password, which is the case in general. In other words, (S3) cannot distinguish between external modules that have access to the password and those that do not. On the other hand, (S2) implies that external modules without knowing the correct password cannot change the balance. Second, (S3) seems to relying on the setting that the password is an unforgeable object instead of a string, which is rather artificial. On the other hand, (S2) seems to work even when the password is a string.


\bibliographystyle{plainnat}
\bibliography{Response1} 


\end{document}  
%
% For emergent behaviour we will include the reviewer's statement and also say that ``(S2) does not take account of the module's \emph{emergent behaviour}. That is, (S2) does not consider the behavior that emerges from the interaction between the 
%\texttt{transfer} method, and the other methods of the bank module. What if the module leaks the password?''
% 
% We will replace the current Bank Account proof with a simpler Coq proof that matches the straightforward introductory example. We will put the current example in an appendix so that we can 
%show reasoning about ghost fields and more complex data structures. 
%
%We will move the clarifying examples to Section 2.
%
%The largest piece of work is the proof and that shouldn't take more than a week so we believe that we can make substantial improvements in presentation before mid January.
%
%
%
% 
%
%
% 
% %We propose the following amended explanation to clarify both it's importance, and it's meaning:
%
% 
% A list of the changes that you plan to make in
%  response to the reviews and the timeline for those changes.
%  
% 
%  
%\section{Response} A reviewer-by-reviewer list of answers to questions
%  with context extracted from the reviews. Use markdown syntax.
%
%\bibliographystyle{plainnat}
%\bibliography{Response1} 
%
%
%\end{document}  