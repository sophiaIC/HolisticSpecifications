
\section{\Loo}
\label{app:loo}


We introduce \Loo, a simple, typed, class-based, object-oriented language that underlies the \Nec specifications
introduced in this paper. \Loo includes ghost fields, recursive definitions that may only be
used in the specification language.
%
\kjx{
To reduce the complexity of our formal models, \Loo lacks many
common languages features, omitting static fields and methods, interfaces,
inheritance, subsumption, exceptions, and control flow.  These features are
well-understood: their presence (or absence) would not chanage the
results we claim nor the structures of the proofs of those results.
Similarly, while Loo is typed, we don't present or mechanise
its type system. 
Our results and proofs rely only upon type
soundness --- in fact, we only need that an expression of
type $T$ (where $T$ is a class $C$ declared in module $M$)
will evaluate to an instance of some class from $M$,
with the same confinement status as $C$.
Featherweight Java extended with modules and assignment
will more than suffice \cite{IgaPieWadTOPLAS01}.
% well-understood that it is too boring to present here or to mechanise anew --- 
%
}


\subsection{Syntax}
The syntax of \Loo is given in Fig. \ref{f:loo-syntax}.
\Loo modules ($M$) map class names ($C$) to class definitions ($\textit{ClassDef}$).
A class definition consists of \jm[]{an optional annotation \enclosed},
a list of field definitions, ghost field definitions, and method definitions.
\jm[]{Fields, ghost fields, and methods all have types: \kjx{types are
    classes}. Ghost fields may be optionally 
annotated as \texttt{intrnl}, requiring the argument to have an internal type, and the 
body of the ghost field to only contain references to internal objects. This is enforced by the
limited type system of \Loo.}
A program state ($\sigma$) is represented as a heap ($\chi$), stack ($\psi$) pair, 
where a heap is a map from addresses ($\alpha$) to objects ($o$), and a stack is a non-empty list of frames ($\phi$). A frame consists of a local variable
map and a continuation ($c$) that represents the statements that are yet to be executed ($s$),
or a hole waiting to be filled by a method return in the frame above ($x := \bullet; s$).
A statement is either a field read ($x := y.f$), a field write ($x.f := y$), a method call
($x := y.m(\overline{z})$), a constructor call ($\prg{new}\ C(\overline{x})$), a method return statement
($\prg{return}\ x$), or a sequence of statements ($s;\ s$).

\Loo also includes syntax for expressions $e$ that may %only
be used in writing
specifications or the definition of ghost fields.


\begin{figure}[t]
\footnotesize
\[
\begin{syntax}
\syntaxID{x, y, z}{Variable}
\syntaxID{C, D}{Class Id.}
\syntaxElement{T}{Type}
		{
		\syntaxline
%				{\_}
				{C}
		\endsyntaxline
		}
\endSyntaxElement\\
\syntaxID{f}{Field Id.}
\syntaxID{g}{Ghost Field Id.}
\syntaxID{m}{Method Id.}
\syntaxID{\alpha}{Address Id.}
\syntaxInSet{i}{\IntSet}{Integer}
\syntaxElement{v}{Value}
		{
		\syntaxline
				{\alpha}
				{i}
				{\true}
				{\false}
				{\nul}
		\endsyntaxline
		}
\endSyntaxElement\\
\syntaxElement{e}{Expression}
		{
		\syntaxline
				{x}
				{v}
				{e + e}
				{e = e}
				{e < e}
		\endsyntaxline
		}
		{
		\syntaxline
				{\prg{if}\ e\ \prg{then}\ e\ \prg{else}\ e}
				{e.f}
				{e.g(e)}
		\endsyntaxline
		}
\endSyntaxElement\\
\syntaxElement{o}{Object}
		{\{\prg{class}:=C;\ \prg{flds}:=\overline{f \mapsto v} \}}
\endSyntaxElement\\
\syntaxElement{s}{Statement}
		{
		\syntaxline
				{x:=y.f}
				{x.f:=y}
				{x:=y.m(\overline{z})}
		\endsyntaxline
		}
		{
		\syntaxline
				{\new{C}{\overline{x}}}
				{\return{x}}
				{s;\ s}
		\endsyntaxline
		}
\endSyntaxElement\\
\syntaxElement{c}{Continuation}
		{
		\syntaxline
				{s}
				{x:=\bullet; s}
		\endsyntaxline
		}
\endSyntaxElement\\
\syntaxElement{\chi}{Heap}
		{\overline{\alpha \mapsto o}}
\endSyntaxElement\\
\syntaxElement{\phi}{Frame}
		{\{\prg{local}:=\overline{x\mapsto v};\ \prg{contn}:=c\}}
\endSyntaxElement\\
\syntaxElement{\psi}{Stack}
		{\syntaxline{\phi}{\phi : \psi}\endsyntaxline}
\endSyntaxElement\\
\syntaxElement{\sigma}{Program Config.}
		{(\prg{heap}:=\chi,\prg{stack}:=\psi)}
\endSyntaxElement\\
\syntaxElement{mth}{Method Def.}
		{
		\prg{method}\ m\ (\overline{x : T})\{\ s\ \}
		}
\endSyntaxElement\\
\syntaxElement{fld}{Field Def.}
		{\syntaxline
			{\prg{field}\ f\ :\ T}
		\endsyntaxline}
\endSyntaxElement\\
\syntaxElement{gfld}{Ghost Field Def.}
		{\syntaxline
			{\prg{ghost}\ g\ (x : T)\{\ e\ \} : T}
			{\prg{ghost}\ \prg{intrnl}\ g\ (x : T)\{\ e\ \} : T}
		\endsyntaxline}
\endSyntaxElement\\
\syntaxElement{An}{Class Annotation}
		{\enclosed}
\endSyntaxElement\\
\syntaxElement{CDef}{Class Def.}
		{
		[An]\ \prg{class}\ C\ \{\ \prg{constr}:= (\overline{x : T})\{s\};\ \prg{flds}:=\overline{fld};\ \prg{gflds}:=\overline{gfld};\ \prg{mths}:=\overline{mth}\ \}
		}
\endSyntaxElement\\
\syntaxElement{Mdl}{Module Def.}
		{
		\syntaxline{\overline{C\ \mapsto\ ClassDef}}\endsyntaxline
		}
\endSyntaxElement\\
\end{syntax}
\]
\caption{\Loo Syntax}
\label{f:loo-syntax}
\end{figure}

\subsection{Semantics}
\Loo is a simple object oriented language, and the operational semantics 
(given in Fig. \ref{f:loo-semantics} and discussed later)
do not introduce any novel or surprising features. The operational 
semantics make use of several helper definitions that we 
define here.

We provide a definition of reference interpretation in Definition \ref{def:interpret}
\begin{definition}
\label{def:interpret}
For a program state $\sigma = (\chi, \phi : \psi)$, we provide the following function definitions:
\begin{itemize}
\item
$\interpret{\sigma}{x}\ \triangleq\ \phi.(\prg{local})(x)$
\item
$\interpret{\sigma}{\alpha.f}\ \triangleq\ \chi(\alpha).(\prg{flds})(f)$
\item
$\interpret{\sigma}{x.f}\ \triangleq\ \interpret{\sigma}{\alpha.f}$ where $\interpret{\sigma}{x}=\alpha$
\end{itemize}
\end{definition}
That is, a variable $x$, or a field access on a variable $x.f$ 
has an interpretation within a program state of value $v$
if $x$ maps to $v$ in the local variable map, or the field
$f$ of the object identified by $x$ points to $v$.

Definition \ref{def:class-lookup} defines the class lookup function an object 
identified by variable $x$.
\begin{definition}[Class Lookup]
\label{def:class-lookup}
For program state $\sigma = (\chi, \phi : \psi)$, class lookup is defined as 
$$\class{\sigma}{x}\ \triangleq\ \chi(\interpret{\sigma}{x}).(\prg{class})$$
\end{definition}

Definition \ref{def:meth-lookup} defines the method lookup function for a method
call $m$ on an object of class $C$.
\begin{definition}[Method Lookup]
\label{def:meth-lookup}
For module $M$, class $C$, and method name $m$, method lookup is defined as 
$$\meth{M}{C}{m}\ \triangleq\ M(C).\prg{mths}(m)$$
\end{definition}

Fig. \ref{f:loo-semantics} gives the operational semantics of \Loo. 
Program state $\sigma_1$ reduces to $\sigma_2$ in the context of
module $M$ if $\exec{M}{\sigma_1}{\sigma_2}$. The semantics in Fig. \ref{f:loo-semantics}
are unsurprising, but it is notable that reads (\textsc{Read}) and writes (\textsc{Write})
are restricted to the class that the field belongs to.
\begin{figure}[t]
\begin{minipage}{\textwidth}
\begin{minipage}{\textwidth}
\footnotesize
\begin{mathpar}
\infer
	{
	\sigma_1 = (\chi, \phi_1 : \psi)\\
	\sigma_2 = (\chi, \phi_2 : \phi_1' : \psi)\\
	\phi_1.(\prg{contn}) = (x := y.m(\overline{z}); s)\\
	\phi_1' = \phi_1[\prg{contn} := (x := \bullet; s)]\\
	\meth{M}{\class{\sigma_1}{x}}{m} = m(\overline{p : T})\{body\}\\
	\phi_2 = \{\prg{local}:= ([\prg{this}\ \mapsto\ \interpret{\sigma_1}{x}]\overline{[p_i\ \mapsto\ \interpret{\sigma_1}{z_i}]}), \prg{contn}:=body\}
	}
	{\exec{M}{\sigma_1}{\sigma_2}}
	\quad(\textsc{Call})
	\and
\infer
	{
	\sigma_1 = (\chi, \phi_1 : \psi) \\
	\sigma_2 = (\chi, \phi_2 : \psi) \\
	\phi_1.(\prg{contn}) = (x := y.f; s)\\
	\interpret{\sigma_1}{x.f} = v \\
	\phi_2 = \{\prg{local}:=\phi_1.(\prg{local})[x\ \mapsto\ v],\ \prg{contn}:=s\}\\
	\class{\sigma_1}{\prg{this}} = \class{\sigma_1}{y}
	}
	{\exec{M}{\sigma_1}{\sigma_2}}
	\quad(\textsc{Read})
	\and
\infer
	{
	\sigma_1 = (\chi_1, \phi_1 : \psi) \\
	\sigma_2 = (\chi_2, \phi_2 : \psi) \\
	\phi_1.(\prg{contn}) = (x.f := y; s)\\
	\interpret{\sigma_1}{y} = v \\
	\phi_2 = \{\prg{local}:=\phi_1.(\prg{local}),\ \prg{contn}:=s\}\\
	\chi_2 = \chi_1[\interpret{\sigma_1}{x}.f \mapsto\ v]\\
	\class{\sigma_1}{\prg{this}} = \class{\sigma_2}{x}
	}
	{\exec{M}{\sigma_1}{\sigma_2}}
	{}
	\quad(\textsc{Write})
	\and
\infer
	{
	\sigma_1 = (\chi, \phi : \psi) \\
	\phi.(\prg{contn}) = (x := \prg{new}\ C(\overline{z}); s)\\
	M(C).(\prg{constr}) = (\overline{p : T})\{ s' \} \\
	\phi' = \{\prg{local}:=[\prg{this} \mapsto \alpha],\overline{[p_i \mapsto \lfloor z_i \rfloor_{\sigma_1}}], \prg{contn} := s'\}\\
	\sigma_2 = (\chi[\alpha\ \mapsto\ \{\prg{class}:=C, \prg{flds}:=\overline{f\ \mapsto\ \nul}], \phi' : \phi[\prg{contn}\ :=\ (x := \bullet; s)] : \psi)
	}
	{\exec{M}{\sigma_1}{\sigma_2}}
	\quad(\textsc{New})
	\and
\infer
	{
	\sigma_1 = (\chi, \phi_1 : \phi_2 : \psi) \\
	\phi_1.(\prg{contn}) = (\prg{return}\ x; s)\ \textit{or}\ \phi_1.(\prg{contn}) = (\prg{return}\ x)\\
	\phi_2.(\prg{contn}) = (y := \bullet; s)\\
	\sigma_2 = (\chi, \phi_2[y\ \mapsto\ \interpret{\sigma_1}{x}] : \psi)
	}
	{\exec{M}{\sigma_1}{\sigma_2}}
	{}
	\quad(\textsc{Return})
\end{mathpar}
\caption{\Loo operational Semantics}
\label{f:loo-semantics}
\end{minipage}
\begin{minipage}{\textwidth}
\footnotesize
\begin{mathpar}
\infer
		{}
		{\eval{M}{\sigma}{v}{v}}
		\quad(\textsc{E-Val})
		\and
\infer
		{}
		{\eval{M}{\sigma}{x}{\interpret{\sigma}{x}}}
		\quad(\textsc{E-Var})
		\and
\infer
		{
		\eval{M}{\sigma}{e_1}{i_1}\\
		\eval{M}{\sigma}{e_2}{i_2}\\
		i_1 + i_2 = i
		}
		{
		\eval{M}{\sigma}{e_1 + e_2}{i}
		}
		\quad(\textsc{E-Add})
		\and
\infer
		{
		\eval{M}{\sigma}{e_1}{v}\\
		\eval{M}{\sigma}{e_2}{v}
		}
		{
		\eval{M}{\sigma}{e_1 = e_2}{\true}
		}
		\quad(\textsc{E-Eq}_1)
		\and
\infer
		{
		\eval{M}{\sigma}{e_1}{v_1}\\
		\eval{M}{\sigma}{e_2}{v_2}\\
		v_1 \neq\ v_2
		}
		{
		\eval{M}{\sigma}{e_1 = e_2}{\false}
		}
		\quad(\textsc{E-Eq}_2)
		\and
\infer
		{
		\eval{M}{\sigma}{e}{\true}\\
		\eval{M}{\sigma}{e_1}{v}
		}
		{
%		\eval{M}{\sigma}{\ifthenelse{e}{e_1}{e_2}}{v}
		\eval{M}{\sigma}{e}{v}
		}
		\quad(\textsc{E-If}_1)
		\and
\infer
		{
		\eval{M}{\sigma}{e}{\false}\\
		\eval{M}{\sigma}{e_2}{v}
		}
		{
%		\eval{M}{\sigma}{\ifthenelse{e}{e_1}{e_2}}{v}
		\eval{M}{\sigma}{e}{v}
		}
		\quad(\textsc{E-If}_2)
		\and
\infer
		{
		\eval{M}{\sigma}{e}{\alpha}
		}
		{
		\eval{M}{\sigma}{e.f}{\interpret{\sigma}{\alpha.f}}
		}
		\quad(\textsc{E-Field})
		\and
\infer
		{
		\eval{M}{\sigma}{e_1}{\alpha}\\
		\eval{M}{\sigma}{e_2}{v'}\\
		\prg{ghost}\ g(x : T)\{e\} : T'\ \in\ M(\class{\sigma}{\alpha}).(\prg{gflds})\\
		\eval{M}{\sigma}{[v'/x]e}{v}
		}
		{
		\eval{M}{\sigma}{e_1.g(e_2)}{v}
		}
		\quad(\textsc{E-Ghost})
\end{mathpar}
\caption{\Loo expression evaluation}
\label{f:evaluation}
\end{minipage}
\end{minipage}
\end{figure}

While the small-step operational semantics of \Loo is given in Fig. \ref{f:loo-semantics},
specification satisfaction is defined over an abstracted notion of 
the operational semantics that models the open world, called \jm[]{\emph{external states semantics}}. 
That is, execution occurs in the context of not just an internal, trusted module, but 
an external, untrusted module. We borrow the definition of external \jm[]{states} semantics 
from \citeauthor{FASE}, along with the related definition of module linking, given in Definition \ref{def:linking}.
\begin{definition}
\label{def:linking}
For all modules $M$ and $M'$, if the domains of $M$ and $M'$ are disjoint, 
we define the module linking function as $M\ \circ\ M'\ \triangleq\ M\ \cup\ M'$.
\end{definition}
That is, given an internal, module $M$, and an external module $M'$, 
we take their linking as the union of the two if their domains are disjoint.



An \emph{Initial} program state contains a single frame 
with a single local variable \prg{this} pointing to a single object 
in the heap of class \prg{Object}, and a continuation.
\begin{definition}[Initial Program State]
\label{def:initial}
A program state $\sigma$ is said to be an initial state ($\initial{\sigma}$)
if and only if
\begin{itemize}
\item
$\sigma.\prg{heap} = [\alpha\ \mapsto\ \{\prg{class}:=\prg{Object};\ \prg{flds}:=\emptyset\}]$ and
\item
$\sigma.\prg{stack} = \{\prg{local}:=[\prg{this}\ \mapsto\ \alpha];\ \prg{contn}:= s\}$
\end{itemize} 
for some address $\alpha$ and some statement $s$.
\end{definition}


%We give the semantics of module pair execution in Definition \ref{def:pair-reduce}
%\begin{definition}[External State Semantics]
%\label{def:pair-reduce-appendix}
%For all internal modules $M_1$, external modules $M_2$, and program configurations $\sigma$ and $\sigma'$, 
%we say that $\reduction{M_1}{M_2}{\sigma}{\sigma'}$ if and only if
%\begin{itemize}
%\item
%$\class{\sigma}{\sigma.(\prg{this})}\ \in\ M_2$ and
%\item
%$\class{\sigma'}{\sigma'.(\prg{this})}\ \in\ M_2$ and 
%\end{itemize} 
%and
%\begin{itemize}
%\item
%$\exec{M_1\ \circ\ M_2}{\sigma}{\sigma'}$ or
%\item
%$M_1 \circ M_2, \sigma \leadsto \sigma_1 \leadsto \ldots \sigma_n \leadsto \sigma'$ and $\class{\sigma_i}{\sigma_i.(\prg{this})} \in M_1$ for all $1 \leq i \leq\ n$
%\end{itemize}
%\end{definition}

Finally, we provide a semantics for expression evaluation is given in Fig. \ref{f:evaluation}. 
That is, given a module $M$ and a program state $\sigma$, expression $e$ evaluates to $v$
if $\eval{M}{\sigma}{e}{v}$. Note, the evaluation of expressions is separate from the operational
semantics of \Loo, and thus there is no restriction on field access.



\newpage
