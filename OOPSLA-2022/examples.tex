\section{Proving that \ModC satisifes \SrobustB}
\label{s:examples}
We now revisit our example from  \S  \ref{s:intro} and \S \ref{s:outline},
and outline a proof that \ModC satisfies \SrobustB. 
A {summary} of this proof has already been discussed in \S \ref{s:approach}.
 A more complex variant of this example that employs   can be found in Appendix \ref{app:examples}.
 \sdN{It demonstrates dealing with modules consisting of several classes some of which are confined, and
 which use ghost fields
 defined through functions; it also demonstrates proofs of assertion encapsulation of assertions 
 which involve reading the values of 
 several fields.} \julian{Mechanised versions of the proofs in both this Section, and Appendix \ref{app:examples}
 can be found in the associated Coq artifact in \prg{simple\_bank\_account.v} and \prg{bank\_account.v} respectively.}
 
 \jm[todo: add reference so \prg{simple\_bank\_account.v} \& \prg{bank\_account.v}]{}
 
Recall that an \prg{Account} includes %a \prg{balance} field, and a \prg{transfer} method, along with any other fields and methods. 
 at least a  field  (or ghost field)  called \prg{balance}, and a method called \prg{transfer}. 

%
 We first rephrase 
\SrobustB to use the $\wrapped{\_}$ predicate.
\begin{lstlisting}[language=Chainmail, mathescape=true, frame=lines]
$\SrobustB$ $\triangleq$ from a:Account $\wedge$ a.balance=bal 
          to a.balance < bal onlyIf $\neg\wrapped{\prg{a.pwd}}$
\end{lstlisting}

We next revisit the   \funcSpec from \S \ref{s:bank} and derive the following 
\prg{PRE}- and \prg{POST}-conditions. The first two pairs of \prg{PRE}-, \prg{POST}-conditions correspond to the first two \prg{ENSURES}
clauses from \S \ref{s:bank}, while the next two pairs correspond to the \prg{MODIFIES}-clause. The current expression in terms
of \prg{PRE}- and \prg{POST}-conditions is weaker than the one in \S \ref{s:bank}, and is not modular, but is
sufficient for proving adherence to  \SrobustB.

\begin{lstlisting}[mathescape=true, frame=lines, language=Chainmail]
$\SclassicP$  $\triangleq$
   method transfer(dest:Account, pwd':Password) -> void  
      (PRE: this.balance$=$bal1 $\wedge$ dest.balance$=$bal2 $\wedge$ this.pwd$=$pwd' $\wedge$ this$\neq$dest
       POST: this.balance=bal1-100 $\wedge$ dest.balance=bal2+100)
      (PRE: this.balance$=$bal1 $\wedge$ dest.balance$=$bal2 $\wedge$ (this.pwd$\neq$pwd' $\vee$ this$=$dest)
       POST: this.balance=bal1 $\wedge$ dest.balance=bal2)
      (PRE: a:Account $\wedge$ a.balance$=$bal $\wedge$ a$\neq$this $\wedge$ a$\neq$dest 
       POST: a.balance=bal)          
      (PRE: a:Account $\wedge$ a.pwd$=$pwd1  
       POST: a.pwd=pwd1)         
\end{lstlisting}


%\begin{figure}[t]
%\begin{lstlisting}[mathescape=true, frame=lines]
%module $\ModC$
%  class Password{}
%  class Account
%    field balance:int
%    field pwd:Password
%    method set(pwd':Password, pwd'':Password):void
%      {if(this.pwd==pwd') 
%         this.pwd := pwd'}
%	method transfer(pwd':Password, destAcc:Account)
%	  {if(this.pwd==pwd' && this.balance > 100)
%	     this.balance := this.balance - 100
%	     destAcc.balance := destAcc.balance + 100}
%\end{lstlisting}
%\caption{Bank Account Module}
%\label{f:ex-bank-short1}
%\end{figure}

%Susan: I changed the label because somewhere (I think in an appendix) there is a f:ex-bank-short and the ref picks it up rather than this one

\subsection{Part 1: Assertion Encapsulation}
\label{s:BA-encap}
The first part of the proof demonstrates that the \prg{balance}, \prg{pwd}, and external accessibility to the password are 
encapsulated properties. That is, for the \prg{balance} to change (i.e. for \prg{a.balance = bal} to be invalidated), or for 
or for the encapsulation of \prg{a.pwd} to be broken (ie for a transition from ${\wrapped{\prg{a,pwd}}}$ to $\neg {\wrapped{\prg{a.pwd}}}$),
internal computation is required. 

We use \sdNr[used to say "a conservative approach to an encapsulation system" but "conservative" here means sound]{a simple encapsulation system}, detailed in \jm[]{Appendix} \ref{s:encap-proof}, 
and provide the proof steps below.
\textbf{\prg{aEnc}} and \textbf{\prg{balanceEnc}} state that 
\jm[]{\prg{a} and \prg{a.balance}} satisfy \sdN{the \textsc{Enc$_e$} predicate. That is, if any objects' contents are to be
looked up during execution of these expressions, then these objects are internal.}
\textsc{Enc}$_e$(\prg{a}) holds because no object's contents is looked up,
while \textsc{Enc}$_e$(\prg{a.balance}) holds because \prg{balance} is a field of \prg{a}, and \prg{a} is
internal.
\\
\begin{figure}[h]
\begin{proofexample}
\proofsteps{\prg{BalEncaps}}
	{\begin{proofexample}
		\proofsteps{\prg{aEnc}}
			{\proofstepwithrule
			{$\proves{\ModC}{\givenA{\prg{a:Account $\wedge$ a.balance=bal}}{\intrnl{\prg{a}}}}$}
				{by \textsc{Enc$_e$-Obj}}
		}
		\endproofsteps
	\end{proofexample}
		}
	{\begin{proofexample}
		\proofsteps{\prg{balanceEnc}}
			{\proofstepwithrule
			{$\proves{\ModC}{\givenA{\prg{a:Account $\wedge$ a.balance=bal}}{\intrnl{\prg{a.balance}}}}$}
				{by \prg{aEnc} and \textsc{Enc-Field}}
		}
		\endproofsteps
	\end{proofexample}
		}
	{\begin{proofexample}
		\proofsteps{\prg{balEnc}}
			{\proofstepwithrule
			{$\proves{\ModC}{\givenA{\prg{a:Account $\wedge$ a.balance=bal}}{\intrnl{\prg{bal}}}}$}
				{by \textsc{Enc$_e$-Int}}
		}
		\endproofsteps
	\end{proofexample}
		}
		{\proofstepwithrule
			{
			$\proves{\ModC}{\givenA{\prg{a:Account $\wedge$ a.balance=bal}}{\encaps{\prg{a.balance=bal}}}}$
			}{by \prg{balanceEnc}, \prg{balEnc}, \textsc{Enc-Eq}, and \textsc{Enc-=}}}
\endproofsteps
\end{proofexample}
\end{figure}


\sdN{Moreover},  \textbf{\prg{balEnc}} states that \prg{bal} satisfies \sdN{the \textsc{Enc}$_e$ predicate}
-- it is an integer, and no object look-up is involved in its calculation.
% used to say
% since bal is an integer, {its} value is constant and may not change, and thus is encapsulated.
\textbf{\prg{balanceEnc}} and \textbf{\prg{balEnc}} combine to prove that the assertion \prg{a.balance = bal} is encapsulated --
%Note: it may seem odd to say that the integer \prg{bal}  { or the address
%\prg{a} is encapsulated};  {but remember that their value cannot change, and thus they
%vacuously sarisfy the definition of encapsulation.}
\sdN{only internal object lookups are involved in the validity of that assertion, and therefore only 
internal computation may cause it to be invalidated.}


\sdN{Using similar reasoning, we} prove that \prg{a.pwd} {is encapsulated} (\textbf{\prg{PwdEncaps}}), and
that \wrapped{\prg{a.pwd}} {is encapsulated} (\textbf{\prg{PwdInsideEncaps}}). 
\sdNr[I chopped the below "That is, if only internal objects have access
to an account's \prg{pwd}, then only internal computation may grant  access to \prg{pwd} 
 to an external object." as it only repeats the definition.]{}

\begin{figure}[h]
\begin{proofexample}
\proofsteps{\prg{PwdEncaps}}
		{\proofstepwithrule
			{
			$\proves{\ModC}{\givenA{\prg{a:Account}}{\encaps{\prg{a.pwd=p}}}}$
			}{by \textsc{Enc$_e$-Obj}, \textsc{Enc-Field}, and \textsc{Enc-Eq}}}
\endproofsteps
\end{proofexample}
\\\begin{proofexample}
\proofsteps{\prg{PwdInsideEncaps}}
		{\proofstepwithrule
			{
			$\proves{\ModC}{\givenA{\prg{a:Account}}{\encaps{\wrapped{\prg{a.balance}}}}}$
			}{by \textsc{Enc-Inside}}}
\endproofsteps
\end{proofexample}
\end{figure}

\sdNr[removing references to Hoare logic.]{}

\subsection{Part 2: Per-Method \Nec Specifications}
\label{s:BA-classical}
Part 2 proves necessary preconditions for each method in 
the module interface. 
%We employ the observation that
%we can build necessary preconditions on top of classical 
\sdN{We employ the rules from  \S \ref{s:classical-proof} which describe how to derive 
necessary preconditions from \funcSpecs.
}
%Hoare logic (\S \ref{s:classical-proof}).

\textbf{\prg{SetBalChange}} uses a 
 \funcSpec and a rule of consequence
% classical Hoare logic
to prove that the \prg{set} method in \prg{Account}
never modifies the \prg{balance}. We then use \textsc{If1-Classical}
and our \Nec logic to prove that if it ever did change (a logical absurdity),
then \prg{transfer} must have been called.
\begin{figure}[htb]
{
	\begin{proofexample}
		\proofsteps{SetBalChange}
			{\proofstepwithrule
				{\hoareEx
						{a, a$^\prime$:Account $\wedge$ a$^\prime$.balance=bal}
						{a.set(\_, \_)}
						{a$^\prime$.balance = bal}
						}
					{by \funcSpec}
			}
			{\proofstepwithrule
				{\hoareEx
						{a, a$^\prime$:Account $\wedge$ a$^\prime$.balance = bal $\wedge$ $\neg$ false}
						{a.set(\_, \_)}
						{$\neg$ a$^\prime$.balance < bal }
						}
					{by 
					\sdN{rule of consequence}} %Hoare logic}
			}
			{\proofstepwithrule
				{\onlyIfSingleExAlt
						{a, a$^\prime$:Account $\wedge$ a$^\prime$.balance=bal $\wedge$ $\calls{\_}{\prg{a}}{\prg{set}}{\prg{\_, \_}}$}
						{a$^\prime$.balance < bal}
						{false}
						}
					{by \textsc{If1-Classical}}
			}
			{\proofstepwithrule
				{\onlyIfSingleExAlt
						{a, a$^\prime$:Account $\wedge$ a$^\prime$.balance=bal $\wedge$ $\calls{\_}{\prg{a}}{\prg{set}}{\prg{\_, \_}}$}
						{a$^\prime$.balance < bal}
						{$\calls{\_}{\prg{a}^\prime}{\prg{transfer}}{\prg{\_, a$^\prime$.pwd}}$}
						}
					{by \textsc{Absurd} and \textsc{If1-}$\longrightarrow$}
			}
		\endproofsteps
	\end{proofexample}
}
\end{figure}

\sdN{Similarly,} in \textbf{\prg{SetPwdLeak}}  we employ \funcSpecs % classical Hoare logic
to prove that a method does not leak access to some data (in this case the \prg{pwd}).
Using \textsc{If1-Inside}, we reason that since the return value of \prg{set} is
\prg{void}, and \prg{set} is prohibited from making external method calls,
no call to \prg{set} can result in an object (external or otherwise) gaining access to the \prg{pwd}.
\begin{figure}[htb]
{
	\begin{proofexample}
		\proofsteps{SetPwdLeak}
			{\proofstepwithrule
				{\hoareEx
						{a:Account $\wedge$ a$^\prime$:Account $\wedge$ a.pwd == p}
						{\prg{res}=a$^\prime$.set(\_, \_)}
						{res != pwd}
						}
					{by \funcSpec}
			}
			{\proofstepwithrule
				{\hoareEx
						{a:Account $\wedge$ a$^\prime$:Account $\wedge$ a.pwd == p $\wedge$ $\neg$ false}
						{\prg{res}=a$^\prime$.set(\_, \_)}
						{res != p}
						}
					{by \sdN{rule of consequence}} % Hoare logic}
			}
			{\proofstepwithrule
				{\onlyIfSingleExAlt
						{$\wrapped{\prg{pwd}}$ $\wedge$ a, a$^\prime$:Account $\wedge$ a.pwd=p $\wedge$ $\calls{\_}{\prg{a}^\prime}{\prg{set}}{\_, \_}$}
						{$\neg \wrapped{\_}$}
						{false}
						}
					{by \textsc{If1-Inside}}
			}
		\endproofsteps
	\end{proofexample}
	}
	\end{figure}
	
In the same manner as \textbf{\prg{SetBalChange}} and \textbf{\prg{SetPwdLeak}}, we also prove
\textbf{\prg{SetPwdChange}}, \textbf{\prg{TransferBalChange}}, \textbf{\prg{TransferPwdLeak}}, and \textbf{\prg{TransferPwdChange}}. We provide their 
statements, but omit their proofs.
\begin{figure}[htb]
{
	\begin{proofexample}
		\proofsteps{SetPwdChange}
			{\proofstepwithrule
				{\onlyIfSingleExAlt
						{a, a$^\prime$:Account $\wedge$ a$^\prime$.pwd=p $\wedge$ $\calls{\_}{\prg{a}}{\prg{set}}{\prg{\_, \_}}$}
						{$\neg$ a.pwd = p}
						{$\calls{\_}{\prg{a}^\prime}{\prg{set}}{\prg{a$^\prime$.pwd, \_}}$}
						}
					{by \textsc{If1-Classical}}
			}
		\endproofsteps
	\end{proofexample}
}
{
	\begin{proofexample}
		\proofsteps{TransferBalChange}
			{\proofstepwithrule
				{\onlyIfSingleExAlt
						{a, a$^\prime$:Account $\wedge$ a$^\prime$.balance=bal $\wedge$ $\calls{\_}{\prg{a}}{\prg{transfer}}{\prg{\_, \_}}$}
						{a$^\prime$.balance < bal}
						{$\calls{\_}{\prg{a}^\prime}{\prg{transfer}}{\prg{\_, a$^\prime$.pwd}}$}
						}
					{by \textsc{If1-Classical}}
			}
		\endproofsteps
	\end{proofexample}
}
{
	\begin{proofexample}
		\proofsteps{TransferPwdLeak}
			{\proofstepwithrule
				{\onlyIfSingleExAlt
						{$\wrapped{\prg{pwd}}$ $\wedge$ a, a$^\prime$:Account $\wedge$ a.pwd=p $\wedge$ $\calls{\_}{\prg{a}^\prime}{\prg{transfer}}{\_, \_}$}
						{$\neg \wrapped{\_}$}
						{false}
						}
					{by \textsc{If1-Inside}}
			}
		\endproofsteps
	\end{proofexample}
	}
	{
	\begin{proofexample}
		\proofsteps{TransferPwdChange}
			{\proofstepwithrule
				{\onlyIfSingleExAlt
						{a, a$^\prime$:Account $\wedge$ a$^\prime$.pwd=p $\wedge$ $\calls{\_}{\prg{a}}{\prg{transfer}}{\prg{\_, \_}}$}
						{$\neg$ a.pwd = p}
						{$\calls{\_}{\prg{a}^\prime}{\prg{set}}{\prg{a$^\prime$.pwd, \_}}$}
						}
					{by \textsc{If1-Classical}}
			}
		\endproofsteps
	\end{proofexample}
	}
\end{figure}

\subsection{Part 3: Per-Step \Nec Specifications}
Part 3 builds upon the proofs of Parts 1 and 2 to 
construct proofs of necessary preconditions, not for single method execution, 
but \sdN{for} any single execution step. That is, a proof that for
\emph{any} single step in program execution, \sdNr[removed "potentially dangerous", since such proof
steps apply whether the changes are potentially dangerous or not]{} changes
in program state require specific preconditions.
\begin{figure}[htb]
{
	\begin{proofexample}
		\proofsteps{BalanceChange}
			{\proofstepwithrule
				{\onlyIfSingleExAlt
						{a:Account $\wedge$ a.balance=bal}
						{a.balance < bal}
						{$\calls{\_}{\prg{a}}{\prg{transfer}}{\prg{\_, a.pwd}}$}
						}
					{by \textbf{\prg{BalEncaps}}, \textbf{\prg{SetBalChange}}, \textbf{TransferBalChange}, and \textsc{If1-Internal}}
			}
		\endproofsteps
	\end{proofexample}
	}{
	\begin{proofexample}
		\proofsteps{PasswordChange}
			{\proofstepwithrule
				{\onlyIfSingleExAlt
						{a:Account $\wedge$ a.pwd=p}
						{$\neg$ a.pwd = bal}
						{$\calls{\_}{\prg{a}}{\prg{set}}{\prg{a.pwd, \_}}$}
						}
					{by \textbf{\prg{PwdEncaps}}, \textbf{\prg{SetPwdChange}}, \textbf{TransferPwdChange}, and \textsc{If1-Internal}}
			}
		\endproofsteps
	\end{proofexample}
	}{
	\begin{proofexample}
		\proofsteps{PasswordLeak}
			{\proofstepwithrule
				{\onlyIfSingleExAlt
						{a:Account $\wedge$ a.pwd=p $\wedge$ $\wrapped{\prg{p}}$}
						{$\neg$ $\wrapped{\prg{p}}$}
						{false}
						}
					{by \textbf{\prg{PwdInsideEncaps}}, \textbf{\prg{SetPwdLeak}}, \textbf{TransferPwdLeak}, and \textsc{If1-Internal}}
			}
		\endproofsteps
	\end{proofexample}
}
\end{figure}

	
\subsection{Part 4: Emergent \Nec Specifications}
Part 4 raises necessary preconditions for single execution steps proven in Part 3 to 
the level of an arbitrary number of execution steps in order to prove specifications of emergent behaviour.
The proof of \SrobustB takes the following form:
\begin{description}
\item [(1)]
If the balance of an account decreases, then
by \prg{BalanceChange} there must have been a call
to \prg{transfer} in \jm[]{\prg{Account}} with the correct password.
\item [(2)]
If there was a call where the \prg{Account}'s password 
was used, then there must have been an intermediate program state
when some external object had access to the password.
\item [(3)]
Either that password was the same password as in the starting 
program state, or it was different:
\begin{description}
\item [(Case A)]
If it is the same as the initial password, then since by \prg{PasswordLeak}
it is impossible to leak the password, it follows that some external object 
must have had access to the password initially.
\item [(Case B)]
If the password is different from the initial password, 
then there must have been an intermediate program state when it 
changed. By \prg{PasswordChange} we know that this must have occurred
by a call to \prg{set} with the correct password. Thus,
there must be a some intermediate program state where the initial
password is known. From here we proceed by the same reasoning 
as \textbf{(Case A)}.
\end{description}
\end{description}
\begin{figure}[htb]
\begin{proofexample}
\proofsteps{\SrobustB}
	{\proofstepwithrule{\onlyThroughExAlt
				{a:Account $\wedge$ a.balance=bal}
				{a.balance < bal}
				{$\calls{\_}{\prg{a}}{\prg{transfer}}{\_, \prg{a.pwd}}$}
				}
			{by \textsc{Changes} and \prg{BalanceChange}}}
	{\proofstepwithrule{\onlyThroughExAlt
				{a:Account $\wedge$ a.balance=bal}
				{b.balance(a) < bal}
				{$\neg$$\wrapped{\prg{a.pwd}}$}
				}
			{by $\longrightarrow$, \textsc{Caller-Ext}, and \textsc{Calls-Args}}}
%	{\proofstepwithrule{\onlyThroughExAlt
%				{a:Account $\wedge$ b:Bank $\wedge$ a.balance=bal $\wedge$ a.password=pwd}
%				{a.balance < bal}
%				{$\neg$$\wrapped{\prg{a.password}}$}
%				}
%			{by $\longrightarrow$}}
	{\proofstepwithrule{\onlyThroughEx
				{a:Account $\wedge$ a.balance=bal $\wedge$ a.pwd=p}
				{a.balance < bal}
				{$\neg$$\wrapped{\prg{a.pwd}}$ $\wedge$ (a.pwd=p $\vee$ a.pwd != p)}
				}
			{by $\longrightarrow$ and \textsc{Excluded Middle}}}
	{\proofstepwithrule{\onlyThroughEx
				{a:Account $\wedge$ a.balance=bal $\wedge$ a.pwd=p}
				{a.balance < bal}
				{($\neg$$\wrapped{\prg{a.pwd}}$ $\wedge$ a.pwd=p) $\vee$\\
				($\neg$$\wrapped{\prg{a.pwd}}$ $\wedge$ a.pwd != p)}
				}
			{by $\longrightarrow$}}
	{\proofstepwithrule{\onlyThroughExAlt
				{a:Account $\wedge$ a.balance=bal $\wedge$ a.pwd=p}
				{a.balance < bal}
				{$\neg$$\wrapped{\prg{p}}$ $\vee$
				a.pwd != p}
				}
			{by $\longrightarrow$}}
	{
	\begin{proofexample}
	\proofsteps{Case A ($\neg\wrapped{\prg{p}}$)}
			{\proofstepwithrule
				{\onlyIfExAlt
					{a:Account $\wedge$ a.balance=bal $\wedge$ a.pwd=p}
					{$\neg$$\wrapped{\prg{p}}$}
					{$\wrapped{\prg{p}}\ \vee \neg\wrapped{\prg{p}}$}
					}
				{by \textsc{If-}$\longrightarrow$ and \textsc{Excluded Middle}}}
			{\proofstepwithrule{\onlyIfExAlt
					{a:Account $\wedge$ b:Bank $\wedge$ b.balance(a)=bal $\wedge$ a.password=pwd}
					{$\neg$$\wrapped{\prg{p}}$}
					{$\neg\wrapped{\prg{p}}$}
					}
				{by $\vee$E and \prg{PasswordLeak}}}
	\endproofsteps
	\end{proofexample}
	}
	{
	\begin{proofexample}
	\proofsteps{Case B (\prg{a.pwd != p})}
		{\proofstepwithrule{\onlyThroughExAlt
					{a:Account $\wedge$ b:Bank $\wedge$ b.balance(a)=bal $\wedge$ a.password=pwd}
					{a.pwd != p}
					{$\calls{\_}{\prg{a}}{\prg{set}}{\prg{p}, \_}$}
					}
				{by \textsc{Changes} and \textsc{PasswordChange}}}
		{\proofstepwithrule{\onlyThroughExAlt
					{a:Account $\wedge$ a.balance=bal $\wedge$ a.pwd=p}
					{a.pwd != p}
					{$\neg\wrapped{\prg{p}}$}
					}
				{by $\vee$E and \prg{PasswordLeak}}}
		{\proofstepwithrule{\onlyIfExAlt
					{a:Account $\wedge$ a.balance=bal $\wedge$ a.pwd=p}
					{a.pwd != p}
					{$\neg\wrapped{\prg{p}}$}
					}
				{by \textbf{Case A} and \textsc{Trans}}}
	\endproofsteps
	\end{proofexample}
	}
	{\proofstepwithrule{\onlyIfExAlt
				{a:Account $\wedge$ a.balance=bal $\wedge$ a.pwd=p}
				{b.balance(a) < bal}
				{$\neg\wrapped{\prg{p}}$}
				}
			{by \textbf{Case A}, \textbf{Case B}, \textsc{If-}$\vee$I$_2$, and \textsc{If-}$\longrightarrow$}}
\endproofsteps
\end{proofexample}
\end{figure}
