Dear reviewers 

Thank you for the guidance on the resubmission of our work. We look forward to our
writing the 2nd version and  to the feedback we will then receive from you. 
In the mean time, we would be grateful for clarification on the following two points:

**1. Completion of Coq proofs**

> The Coq proof should be completed: currently it has so many admits.

We believe that at most one proof should be completed. 

Namely, the Coq proof has 39  admissions, which fall into the following 
3 categories:
 - *1st Category*: 1 admission in a file that does not form part of the proof 
- *2nd Category*: 37 admissions of properties of the assumed proof system 
    of specifications of the form M |- A
- *3rd Category*: 1 admission relating to variable rewriting in assertions

The 2nd Category is about expected properties of the assumed logic, relating to 
proofs of consequence  (eg. `A1 /\ A2 -> A1`), and properties of expressions 
(eg. `x < y -> x != y`). In the paper (line 731) we are assuming a sound such logic,
and outline it in Appendix E. The proof of such properties would be standard.

These properties are not required in the proof  of our main Theorem (4.4) --
in fact the Theorem holds for any sound proof system   `M |- A`. Thus, there
properties  have no impact on the main result of our paper.
But these properties were used in the proof of the Bank Account example --
we explain that at the start of file `inference_soundness.v`. This is why we used
the Coq keyword `admit`,  but we can instead use the `parameter` keyword.   

The 3rd Category was initially admitted as it is not very interesting, and proofs 
relating to variable rewrites are well known to be vastly more complex than they 
are interesting. 

*Our proposal:*

`P1`:  We will address the 1st Category  by deleting the file containing it.

`P2`:  We will explain in the paper why we simply admit the properties from
 the 2nd Category.

`P3`: We will take the reviewers' guidance as to whether we should complete
the proof for the 3rd Category, or explain the reasons for omitting it.

`P4`: We will add to the Coq proofs a file describing the contents of each *.v file, 
 what properties we admitted and why, and what other properties we were
parametric over and why, with references back to the paper.

**2. Unrealistic motivating example**

> The motivating password example is still not realistic. The example does not 
provide any interface to allow someone to get the correct password, or at 
least initialize the password. ... example should be extended in such a way, 
which may require a more powerful logic..*

This is a good point, and we thank the reviewers for pointing out that we should
have expanded:

We are not clear what is meant by "an interface to allow someone to get the 
correct password". Such a function (called, say, `getPassword`) is expressible
in the language, and we could add it to our example. But if the module allowed 
a client to read an account's password without prior knowledge of the password, 
then the module would no longer be robust, because anybody with access to
the account would be able to take the money out of the account. 

On the other hand, Mod_3 does support initialization of the password, through
execution of the following code

	p1=new Object; a=new Account(); a.setPassword(null,p1)

Namely, object creation, (`new Account()`) initializes the account's password
to `null`, and thus the call `a.setPassword(null,p1)` will succesfully set the password
to `p1`. 

*Our proposal* 

`P5`: We will add explanations as per above, showing how object intialization works 
in our approach. 

------
**Appendix**

More details about point 2, for the interested reviewer.

If we assumed some "unknown" method `m_unknown`, and "untrusted" 
object `o_untrust`, and some account `a0` with some money in it, and with password
`p0`, then, the following code

         `a=new Account(); a0.transfer(p0,100); o_untrust.m_unknown(a)`

gives _no_ guarantee about the call `o_untrust.m_unknown(a)` not removing the money in the account `a`, while

        `p1=new Object; a=new Account();  a.setPassword(null,p1);`
        `a0.transfer(p0,100);  o_untrust.m_unknown(a)`

_does_ guarantee that no money will be removed during the call `o_untrust.m_unknown(a)`
-- because no external objects involved in that call have access to `p1`.  

One may now ask how we would initialize the account so that it contains some money, 
without adding methods to class `Account` which make it possible to just create money
out of thin air. We can do this (initialize the money) in the language we have
provided: add a function, which pays money into the account only while the password 
is still `null`. Or, even better, we could follow the approach  from [Mark Samuel Miller, 
Chip Morningstar, and Bill Frantz. 2000. Capability-based Financial Instruments: 
From Object to Capabilities] and have a `Bank` object, with `Account`s belonging 
to that `Bank`. The `Account`s may transfer moneys across 
each other only if they belong to the same `Bank`, and only `Bank`s may create `Account`s 
belonging to them. We could give the code for this, but it would require another page, 
discussing issues that are not central to the contribution of this work. Therefore,
we propose not to do that, but are happy to take the reviewers' guidance.

------
We are looking forward to your answer whether you agree with our proposals `P1`,`P2`,`P3`, `P4` and `P5`, and thank you, again, for your interest.
 
