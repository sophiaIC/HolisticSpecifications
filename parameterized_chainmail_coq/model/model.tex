\title{Model Outline}	

\documentclass[12pt]{article}

\usepackage{mathpartir}
\usepackage{amsmath}
\usepackage{amssymb}
\usepackage{color,soul}
\usepackage{oz}
\usepackage{listings}
\usepackage{color}

\renewcommand\lstlistingname{Quelltext} % Change language of section name

\lstset{ % General setup for the package
	language=Java,
	basicstyle=\small\sffamily,
	numbers=left,
 	numberstyle=\tiny,
	frame=tb,
	tabsize=4,
	columns=fixed,
	showstringspaces=false,
	showtabs=false,
	keepspaces,
	morekeywords={field, method},
	commentstyle=\color{red},
	keywordstyle=\color{blue}
}

\date{}

\begin{document}
\maketitle

\section{Syntax}
\begin{figure}[h]
$$
\begin{array}{llr}
s & ::= &\\ 
| & \texttt{skip} &\\
| & \texttt{call}\ \alpha.\texttt{m}(\overline{\texttt{x} \mapsto v}) &\\ 
| & \texttt{acc}\ \texttt{x}\ v &\\
| & \texttt{drop}\ \texttt{x} &\\
| & \texttt{mut}\ \alpha.\texttt{f} := v &\\
| & \texttt{new}	\ \texttt{C}(\overline{\texttt{f} \mapsto v}) &\\
| & \texttt{return}\ v
\end{array}
\begin{array}{llr}
p & ::= &\\ 
| & \texttt{true} &\\
| & \texttt{false} &\\
| & i & \textit{where}\ i \in \mathbb{Z}
\end{array}
\begin{array}{llr}
v & ::= &\\ 
| & p &\\
| & \alpha\\
| & \texttt{null}
\end{array}
\begin{array}{llr}
b & ::= &\\ 
| & \texttt{return}\ v &\\
| & \texttt{x} := \bigcirc\ ; b&\\
| & s\ ; b\\
\end{array}
$$
\[
\phi\ ::=\ (\texttt{this:=}\alpha\; ;\texttt{local:=}\overline{v}\; ; \texttt{contn:=}\{b\})\\
\psi\ ::=\ \phi\ |\ \phi : \psi\\
o\ ::=\ (\texttt{class:=C}\; ; \texttt{fields:=}\overline{\texttt{f}\mapsto v})\\
\chi\ ::=\ \overline{\alpha \mapsto o}\\
\sigma\ ::=\ (\chi, \psi)\\
\mathcal{B} : \overline{\texttt{x} \mapsto v}\ \rightarrow\ b \\
\texttt{CDef}_\texttt{E}\ ::=\ \texttt{class}\ \texttt{C} \{ 
	\overline{\texttt{f}}\; ; 
	\overline{\texttt{m} \mapsto \{ \mathcal{B}\}}
\}\\
\mathcal{M} : \sigma\ \rightarrow\ \sigma \\
\mathcal{H} : \{P\}\ \_\ \{Q\}  \\
\texttt{CDef}_\texttt{I}\ ::=\ \texttt{class}\ \texttt{C} \{ 
	\overline{\texttt{f}}\; ; 
	\overline{\texttt{m} \mapsto \overline{\mathcal{H}};\ \{\mathcal{M}\}}
\}\\
M_{\texttt{E}}\ ::=\ \overline{\texttt{C} \mapsto \texttt{CDef}_\texttt{E}}\\
M_{\texttt{I}}\ ::=\ \overline{\texttt{C} \mapsto \texttt{CDef}_\texttt{I}}
\]
\caption{Syntax}
\label{f:syntax}
\end{figure}


\section{Operational Semantics}

\begin{figure}[h]
\hfill \fbox{$M_\texttt{I}\ \fcmp\ M_\texttt{E}\ \bullet\ \sigma_1\ \leadsto\ \sigma_2$}
\begin{mathpar}
\infer
	{
	\alpha.\texttt{f},\ v\ \in\ visible((\chi, \psi), (\chi, \psi).\texttt{this}) \\
	(\chi, \psi).\texttt{contn:=}\ \{\texttt{mut}\ \alpha.\texttt{f}\ \texttt{:=}\ v\ ; \ b\}
	}
	{
	M_\texttt{I}\ \fcmp\ M_\texttt{E}\ \bullet\ 
		(\chi, \psi)\ 
		\leadsto\ 
		(\chi[\alpha.\texttt{f} \mapsto v], \psi[\texttt{contn:=\{b\}}])
	}
	\quad (\textsc{Mut})
	\and
\infer
	{
	\overline{v}\ \in\ visible((\chi, \psi), (\chi, \psi).\texttt{this}) \\
	\phi.\texttt{contn:=}\ \{\texttt{x}\ :=\ \texttt{new}\ \texttt{C}(\overline{f\ \mapsto\ v})\ ; \ b\} \\
	\phi' = \phi[\texttt{local:=}[x \mapsto \alpha]\phi.\texttt{local}\ ;\ \texttt{contn:=}\{b\}] \\
	\alpha\ \textit{fresh in}\ \chi \\
	o = (\texttt{class:=C}\ ;\ \texttt{fields:=}\overline{\texttt{f}\ \mapsto\ v})
	}
	{
	M_1\ \fcmp\ M_2\ \bullet\ 
		(\chi, \phi:\psi)\ 
		\leadsto\ 
		((\chi)[\alpha \mapsto o], \phi\ :\ \psi)
	}
	\quad (\textsc{New})
	\and
\infer
	{
	\alpha_\texttt{this} = (\chi, \psi).\texttt{this} \\
	\alpha, \overline{v}\ \in\ visible((\chi, \psi), \alpha_\texttt{this}) \\
	classOf((\chi, \psi), \alpha) = \texttt{C} \\
	M_2(C).m = \{b\}\\
	(\chi, \psi)\texttt{.contn} = \{\texttt{x}\ :=\ \texttt{call}\ \alpha.\texttt{m}(\overline{v})\}
	}
	{
	M_1\ \fcmp\ M_2\ \bullet\ 
		(\chi, \psi)\ 
		\leadsto\ 
		(\chi, (\texttt{this:=}\alpha_\texttt{this}\ ;\ \texttt{local:=}\overline{v}\ ;\ \texttt{contn:=}\{b\})\ :\ \psi)
	}
	\quad (\textsc{Call})
	\and
\infer
	{ 
	\sigma_1 = (\chi, \phi_1\ :\ \phi_2 \ :\ \psi) \\
	v\ \in\ visible(\sigma_1, \sigma_1.\texttt{this}) \\
	\phi_1.\texttt{contn:=}\ \{\texttt{return}\ v\ ; \ b\} \\
	\phi = \phi_2[\texttt{local:=}\phi_2.\texttt{local}\cup\{v\}] 
	}
	{
	M_1\ \fcmp\ M_2\ \bullet\ 
		\sigma_1\ 
		\leadsto\ 
		(\chi, \phi\ :\ \psi)
	}
	\quad (\textsc{Rtrn$^1$})
	\and
\infer
	{ 
	v\ \in\ visible((\chi, \phi_1\ :\ \phi_2 \ :\ \psi) , \phi_1.\texttt{this}) \\
	\phi_1.\texttt{contn:=}\ \{\texttt{return}\ v\} \\
	\phi = \phi_2[\texttt{local:=}[\texttt{x}\ \mapsto\ v]\phi_2.\texttt{local}] 
	}
	{
	M_1\ \fcmp\ M_2\ \bullet\ 
		(\chi, \phi_1\ :\ \phi_2 \ :\ \psi) \ 
		\leadsto\ 
		(\chi, \phi\ :\ \psi)
	}
	\quad (\textsc{Rtrn$^2$})
\end{mathpar}
\caption{Pair Reduction.}
\label{f:pair_reduction}
\end{figure}


\section{External and Internal Steps}

Pair reduction consists of a series of either external or internal steps. That is any pair reduction step of the form
$$M_1\ \fcmp\ M_2\ \bullet\ \sigma_1\ \leadsto\ \sigma_2$$
is either
\begin{enumerate}
\item
An external step: i.e. a single reduction step purely between external frames. Only external objects can be modified in an external step.
\item
an internal step: a series of internal steps between two external frames that takes one of the following forms:
\begin{enumerate}
\item \texttt{call-call}: a method call on an internal object that in its execution makes an external method call. The resulting stack introduces a continuous block of internal frames  to the top of the stack.
\item \texttt{call-return}: a method call on an internal object that returns without making any external method calls. 
The resulting stack is more or less equal to to the initial stack (not including modifications to the local environment and continuation of the top frame).
\item \texttt{return-call}: a return from a prior internal to external method call, followed by another method call. The resulting stack is  a single external frame on top of a continuous block of internal frames.
\item \texttt{return-return}: a return from a prior internal to external method call, followed by a return from a prior external to internal method call. The resulting stack removes a continuous block of internal frames from the stack.
\end{enumerate}
\end{enumerate}

The distinction between internal and external steps is important because it determines which objects in the heap might be modified during a particular pair reduction step, and what kind of internal reduction step is important because it is required to determine to which possible internal execution path might be 


\section{Changes}

In proving satisfaction of different specifications, it is useful to investigate when different portions of the heap is modified.
Many specifications assert that certain portions of the heap can only be modified under certain conditions, and this is often 
done using temporal operators. As a simple example, the bank account example from the paper:
 
  \vspace{.01in}
(BankSpec)\ \  $\triangleq$\\ 
$\forall \texttt{a}.[\ \ \texttt{a}:\texttt{Account} \wedge \texttt{changes}\langle{\texttt{a.balance}}\rangle  \ \    
    \longrightarrow \ \    \hfill$ \\
  $\strut \hspace{2.3cm} 
% TODO explain:
% we no longer need Past here, as we are ion visible states 
  \exists \texttt{o}. [\    \texttt{o}\ \texttt{calls}\ {\texttt{a.}}{\texttt{deposit}}({\_,\_}) \vee\  \texttt{o}\ \texttt{calls}\ {\_.}{\texttt{deposit}}({\texttt{a},\_})\rangle\  \ ] \ \ \ \ ] \hfill $
\vspace{.05in}

Currently there is no \texttt{changes} assertion in  chainmail, however one might encode \texttt{changes} using \texttt{next}:
$$\texttt{changes}\langle \texttt{x.f} \rangle = \exists \texttt{i}. [\texttt{x.f} = i\ \wedge\ \texttt{next}\langle \texttt{x.f}\neq \texttt{i} \rangle]$$
The above encoding does not generally capture \texttt{changes} because \texttt{next} has no concept of the future if 
the current configuration is a method return statement. This does not technically matter for \texttt{changes} because 
method return does not modify fields. This arises more problematically if we need to assert that a reduction step 
changes \texttt{access}, as this *can* be changed by method return (i.e. via the local enviroment).

Of the 4 kinds of internal steps, a version of \texttt{changes} that is defined using \texttt{next} can only 
capture internal steps of the form of (a) and (b), and not internal steps of the form of (c) and (d).

\subsection{Hoare Logic}
Define a Hoare triple as 
$$M_1\ \fcmp\ M_	2, \sigma_0 \vDash \{A_1\}\ \sigma\ \{ A_2\}$$
That is, for modules $M_1$ and $M_2$, initial configuration $\sigma_0$, current configuration $\sigma$, and assertions $A_1$ and $A_2$, 
if $A_1$ is satisfied before reduction of $\sigma$ then $A_2$ will be satisfied afterward.

Using such a hoare triple, we can define \texttt{changes} as such:
$$M_1\ \fcmp\ M_	2, \sigma_0, \sigma \vDash \texttt{changes}\langle A \rangle\ \equiv\ M_1\ \fcmp\ M_2, \sigma_0 \vDash \{A_1\}\ \sigma\ \{ A_2\}$$
In most cases (I think, ..., probably, ..., not sure) we are concerned with changes that require an internal step, and not an external one. For this purpose, 
it is helpful if we assume that the internal module $M_1$ is fully specified with classical specs. That is, for every possible pass through internal code 
(of the forms listed (a), (b), (c), and (d) above) that internal step is fully specified with regards to modifications to the heap. In the bank account example
this means that we are able to identify all possible internal methods that might modify the balance of an account, and thus determine that the only 
possible step that might result in such a change is a call to \texttt{deposit}.








\bibliographystyle{abbrv}
\bibliography{main}

\end{document}