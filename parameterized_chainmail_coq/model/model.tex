	\title{Model Outline}	
	
	\documentclass[12pt]{article}
	
	\usepackage{mathpartir}
	\usepackage{amsmath}
	\usepackage{amsthm}
	\usepackage{amssymb}
	\usepackage{color,soul}
	\usepackage{oz}
	\usepackage{listings}
	\usepackage{color}
	
	\newtheorem{definition}{Definition}
	
	\renewcommand\lstlistingname{Quelltext} % Change language of section name
	
	\lstset{ % General setup for the package
		language=Java,
		basicstyle=\small\sffamily,
		numbers=left,
	 	numberstyle=\tiny,
		frame=tb,
		tabsize=4,
		columns=fixed,
		showstringspaces=false,
		showtabs=false,
		keepspaces,
		morekeywords={field, method},
		commentstyle=\color{red},
		keywordstyle=\color{blue}
	}
	
	\date{}
	
	\begin{document}
	\maketitle
	
	\section{Syntax}
	\begin{figure}[h]
	$$
	\begin{array}{llr}
	s & ::= &\\ 
	| & \texttt{skip} &\\
	| & \texttt{call}\ \alpha.\texttt{m}(\overline{\texttt{x} \mapsto v}) &\\ 
	| & \texttt{acc}\ \texttt{x}\ v &\\
	| & \texttt{drop}\ \texttt{x} &\\
	| & \texttt{mut}\ \alpha.\texttt{f} := v &\\
	| & \texttt{new}	\ \texttt{C}(\overline{\texttt{f} \mapsto v}) &\\
	| & \texttt{return}\ v
	\end{array}
	\begin{array}{llr}
	p & ::= &\\ 
	| & \texttt{true} &\\
	| & \texttt{false} &\\
	| & i & \textit{where}\ i \in \mathbb{Z}
	\end{array}
	\begin{array}{llr}
	v & ::= &\\ 
	| & p &\\
	| & \alpha\\
	| & \texttt{null}
	\end{array}
	\begin{array}{llr}
	b & ::= &\\ 
	| & \texttt{return}\ v &\\
	| & \texttt{x} := \bigcirc\ ; b&\\
	| & s\ ; b\\
	\end{array}
	$$
	\[
	\phi\ ::=\ (\texttt{this:=}\alpha\; ;\texttt{local:=}\overline{v}\; ; \texttt{contn:=}\{b\})\\
	\psi\ ::=\ \phi\ |\ \phi : \psi\\
	o\ ::=\ (\texttt{class:=C}\; ; \texttt{fields:=}\overline{\texttt{f}\mapsto v})\\
	\chi\ ::=\ \overline{\alpha \mapsto o}\\
	\sigma\ ::=\ (\chi, \psi)\\
	\mathcal{B} : \overline{\texttt{x} \mapsto v}\ \rightarrow\ b \\
	\texttt{CDef}_\texttt{E}\ ::=\ \texttt{class}\ \texttt{C} \{ 
		\overline{\texttt{f}}\; ; 
		\overline{\texttt{m} \mapsto \{ \mathcal{B}\}}
	\}\\
	\mathcal{M} : \sigma\ \rightarrow\ \sigma \\
	\mathcal{H} : \{P\}\ \_\ \{Q\}  \\
	\texttt{CDef}_\texttt{I}\ ::=\ \texttt{class}\ \texttt{C} \{ 
		\overline{\texttt{f}}\; ; 
		\overline{\texttt{m} \mapsto \overline{\mathcal{H}};\ \{\mathcal{M}\}}
	\}\\
	M_{\texttt{E}}\ ::=\ \overline{\texttt{C} \mapsto \texttt{CDef}_\texttt{E}}\\
	M_{\texttt{I}}\ ::=\ \overline{\texttt{C} \mapsto \texttt{CDef}_\texttt{I}}
	\]
	\caption{Syntax}
	\label{f:syntax}
	\end{figure}
	
	
	\section{Operational Semantics}
	
	\begin{figure}[h]
	\hfill \fbox{$M_\texttt{I}\ \fcmp\ M_\texttt{E}\ \bullet\ \sigma_1\ \leadsto\ \sigma_2$}
	\begin{mathpar}
	\infer
		{
		\alpha.\texttt{f},\ v\ \in\ visible((\chi, \psi), (\chi, \psi).\texttt{this}) \\
		(\chi, \psi).\texttt{contn:=}\ \{\texttt{mut}\ \alpha.\texttt{f}\ \texttt{:=}\ v\ ; \ b\}
		}
		{
		M_\texttt{I}\ \fcmp\ M_\texttt{E}\ \bullet\ 
			(\chi, \psi)\ 
			\leadsto\ 
			(\chi[\alpha.\texttt{f} \mapsto v], \psi[\texttt{contn:=\{b\}}])
		}
		\quad (\textsc{Mut})
		\and
	\infer
		{
		\overline{v}\ \in\ visible((\chi, \psi), (\chi, \psi).\texttt{this}) \\
		\phi.\texttt{contn:=}\ \{\texttt{x}\ :=\ \texttt{new}\ \texttt{C}(\overline{f\ \mapsto\ v})\ ; \ b\} \\
		\phi' = \phi[\texttt{local:=}[x \mapsto \alpha]\phi.\texttt{local}\ ;\ \texttt{contn:=}\{b\}] \\
		\alpha\ \textit{fresh in}\ \chi \\
		o = (\texttt{class:=C}\ ;\ \texttt{fields:=}\overline{\texttt{f}\ \mapsto\ v})
		}
		{
		M_1\ \fcmp\ M_2\ \bullet\ 
			(\chi, \phi:\psi)\ 
			\leadsto\ 
			((\chi)[\alpha \mapsto o], \phi\ :\ \psi)
		}
		\quad (\textsc{New})
		\and
	\infer
		{
		\alpha_\texttt{this} = (\chi, \psi).\texttt{this} \\
		\alpha, \overline{v}\ \in\ visible((\chi, \psi), \alpha_\texttt{this}) \\
		classOf((\chi, \psi), \alpha) = \texttt{C} \\
		M_2(C).m = \{b\}\\
		(\chi, \psi)\texttt{.contn} = \{\texttt{x}\ :=\ \texttt{call}\ \alpha.\texttt{m}(\overline{v})\}
		}
		{
		M_1\ \fcmp\ M_2\ \bullet\ 
			(\chi, \psi)\ 
			\leadsto\ 
			(\chi, (\texttt{this:=}\alpha_\texttt{this}\ ;\ \texttt{local:=}\overline{v}\ ;\ \texttt{contn:=}\{b\})\ :\ \psi)
		}
		\quad (\textsc{Call})
		\and
	\infer
		{ 
		\sigma_1 = (\chi, \phi_1\ :\ \phi_2 \ :\ \psi) \\
		v\ \in\ visible(\sigma_1, \sigma_1.\texttt{this}) \\
		\phi_1.\texttt{contn:=}\ \{\texttt{return}\ v\ ; \ b\} \\
		\phi = \phi_2[\texttt{local:=}\phi_2.\texttt{local}\cup\{v\}] 
		}
		{
		M_1\ \fcmp\ M_2\ \bullet\ 
			\sigma_1\ 
			\leadsto\ 
			(\chi, \phi\ :\ \psi)
		}
		\quad (\textsc{Rtrn$^1$})
		\and
	\infer
		{ 
		v\ \in\ visible((\chi, \phi_1\ :\ \phi_2 \ :\ \psi) , \phi_1.\texttt{this}) \\
		\phi_1.\texttt{contn:=}\ \{\texttt{return}\ v\} \\
		\phi = \phi_2[\texttt{local:=}[\texttt{x}\ \mapsto\ v]\phi_2.\texttt{local}] 
		}
		{
		M_1\ \fcmp\ M_2\ \bullet\ 
			(\chi, \phi_1\ :\ \phi_2 \ :\ \psi) \ 
			\leadsto\ 
			(\chi, \phi\ :\ \psi)
		}
		\quad (\textsc{Rtrn$^2$})
	\end{mathpar}
	\caption{Pair Reduction.}
	\label{f:pair_reduction}
	\end{figure}
	
	
	\section{External and Internal Steps}
	
	Pair reduction consists of a series of either external or internal steps. That is any pair reduction step of the form
	$$M_1\ \fcmp\ M_2\ \bullet\ \sigma_1\ \leadsto\ \sigma_2$$
	is either
	\begin{enumerate}
	\item
	An external step: i.e. a single reduction step purely between external frames. Only external objects can be modified in an external step.
	\item
	an internal step: a series of internal steps between two external frames that takes one of the following forms:
	\begin{enumerate}
	\item \texttt{call-call}: a method call on an internal object that in its execution makes an external method call. The resulting stack introduces a continuous block of internal frames  to the top of the stack.
	\item \texttt{call-return}: a method call on an internal object that returns without making any external method calls. 
	The resulting stack is more or less equal to to the initial stack (not including modifications to the local environment and continuation of the top frame).
	\item \texttt{return-call}: a return from a prior internal to external method call, followed by another method call. The resulting stack is  a single external frame on top of a continuous block of internal frames.
	\item \texttt{return-return}: a return from a prior internal to external method call, followed by a return from a prior external to internal method call. The resulting stack removes a continuous block of internal frames from the stack.
	\end{enumerate}
	\end{enumerate}
	
	The distinction between internal and external steps is important because it determines which objects in the heap might be modified during a particular pair reduction step, and what kind of internal reduction step is important because it is required to determine to which possible internal execution path might be 
	
	
	\section{Changes}
	
	In proving satisfaction of different specifications, it is useful to investigate when different portions of the heap is modified.
	Many specifications assert that certain portions of the heap can only be modified under certain conditions, and this is often 
	done using temporal operators. As a simple example, the bank account example from the paper:
	 
	  \vspace{.01in}
	(BankSpec)\ \  $\triangleq$\\ 
	$\forall \texttt{a}.[\ \ \texttt{a}:\texttt{Account} \wedge \texttt{changes}\langle{\texttt{a.balance}}\rangle  \ \    
	    \longrightarrow \ \    \hfill$ \\
	  $\strut \hspace{2.3cm} 
	% TODO explain:
	% we no longer need Past here, as we are ion visible states 
	  \exists \texttt{o}. [\    \texttt{o}\ \texttt{calls}\ {\texttt{a.}}{\texttt{deposit}}({\_,\_}) \vee\  \texttt{o}\ \texttt{calls}\ {\_.}{\texttt{deposit}}({\texttt{a},\_})\rangle\  \ ] \ \ \ \ ] \hfill $
	\vspace{.05in}
	
	Currently there is no \texttt{changes} assertion in  chainmail, however one might encode \texttt{changes} using \texttt{next}:
	$$\texttt{changes}\langle \texttt{x.f} \rangle = \exists \texttt{i}. [\texttt{x.f} = i\ \wedge\ \texttt{next}\langle \texttt{x.f}\neq \texttt{i} \rangle]$$
	The above encoding does not generally capture \texttt{changes} because \texttt{next} has no concept of the future if 
	the current configuration is a method return statement. This does not technically matter for \texttt{changes} because 
	method return does not modify fields. This arises more problematically if we need to assert that a reduction step 
	changes \texttt{access}, as this *can* be changed by method return (i.e. via the local enviroment).
	
	Of the 4 kinds of internal steps, a version of \texttt{changes} that is defined using \texttt{next} can only 
	capture internal steps of the form of (a) and (b), and not internal steps of the form of (c) and (d).
	
	\subsection{Hoare Logic}
	Define a Hoare triple as 
	$$M_1\ \fcmp\ M_	2, \sigma_0 \vDash \{A_1\}\ \sigma\ \{ A_2\}$$
	That is, for modules $M_1$ and $M_2$, initial configuration $\sigma_0$, current configuration $\sigma$, and assertions $A_1$ and $A_2$, 
	if $A_1$ is satisfied before reduction of $\sigma$ then $A_2$ will be satisfied afterward.
	
	Using such a hoare triple, we can define \texttt{changes} as such:
	$$M_1\ \fcmp\ M_	2, \sigma_0, \sigma \vDash \texttt{changes}\langle A \rangle\ \equiv\ M_1\ \fcmp\ M_2, \sigma_0 \vDash \{A\}\ \sigma\ \{ \neg A\}$$
	In most cases (I think, ..., probably, ..., not sure) we are concerned with changes that require an internal step, and not an external one. For this purpose, 
	it is helpful if we assume that the internal module $M_1$ is fully specified with classical specs. That is, for every possible pass through internal code 
	(of the forms listed (a), (b), (c), and (d) above) that internal step is fully specified with regards to modifications to the heap. In the bank account example
	this means that we are able to identify all possible internal methods that might modify the balance of an account, and thus determine that the only 
	possible step that might result in such a change is a call to \texttt{deposit}.
	
	\section{Examples}
	
	We classify pair reduction steps in the following way.
	\begin{definition}
	\begin{mathpar}
	\infer
		{(M_1\ \circ\ M_2)\ \bullet\ \sigma\ \leadsto\ \sigma_1\ \leadsto\ \ldots\ 
		\leadsto\ \sigma_n\ \leadsto\ \sigma'\\
		(\forall \sigma_i.\ M_1\ \fcmp\ M_2, \sigma_0, \sigma_i\ \vDash \texttt{internal}\langle \sigma_i.(\texttt{self}) \rangle)\\
		M_2, \sigma_0, \sigma\ \vDash \texttt{external}\langle \sigma.(\texttt{self}) \rangle\\
		M_2, \sigma_0, \sigma'\ \vDash \texttt{external}\langle \sigma'.(\texttt{self}) \rangle}
		{M_1\ \fcmp\ M_2\ \bullet\ \sigma\ \leadsto^\texttt{int}\ \sigma'}
		\quad(\textsc{Int})
		\and
	\infer
		{(M_1\ \circ\ M_2)\ \bullet\ \sigma\ \leadsto\ \sigma'\\
		M_2, \sigma_0, \sigma\ \vDash \texttt{external}\langle \sigma.(\texttt{self}) \rangle\\
		M_2, \sigma_0, \sigma'\ \vDash \texttt{external}\langle \sigma'.(\texttt{self}) \rangle}
		{M_1\ \fcmp\ M_2\ \bullet\ \sigma\ \leadsto^\texttt{ext}\ \sigma'}
		\quad(\textsc{Ext})
	\end{mathpar}
	\end{definition}
	\begin{definition}
	$$M_1\ \fcmp\ M_2\ \bullet\ \sigma\ \leadsto\ \sigma'$$
	if and only if
	\begin{itemize}
	\item
	$M_1\ \fcmp\ M_2\ \bullet\ \sigma\ \leadsto^\texttt{int}\ \sigma'$ or
	\item
	$M_1\ \fcmp\ M_2\ \bullet\ \sigma\ \leadsto^\texttt{ext}\ \sigma'$
	\end{itemize}
	\end{definition}
	
	Next we define some useful properties.
	
	$$\texttt{changes}\langle A \rangle\ \equiv\ A\ \wedge\ \texttt{next}\langle \neg A \rangle$$
	$$\texttt{changed}\langle A \rangle\ \equiv\ \neg A\ \wedge\ \texttt{prev}\langle A \rangle$$
	
	\begin{mathpar}
	\infer
		{
		\sigma\ \vDash\ \texttt{changes}\langle A \rangle
		}
		{
		\sigma\ \vDash\ \texttt{next}\langle\texttt{changed}\langle A \rangle \rangle
		}
		\and
	\infer
		{
		\sigma\ \vDash\ A \\
		\sigma\ \vDash\ \texttt{will}\langle \neg A \rangle
		}
		{
		\sigma\ \vDash\ \texttt{will}\langle\texttt{changed}\langle A \rangle\rangle
		}
		\and
	\infer
		{
		\sigma\ \vDash\ A_1\ \longrightarrow\ A_2 \\
		\sigma\ \vDash\ \texttt{will}\langle A_1 \rangle
		}
		{
		\sigma\ \vDash\ \texttt{will}\langle A_2 \rangle
		}
		\and
	\infer
		{
		\sigma\ \vDash\ \texttt{changed}\langle A \rangle
		}
		{
		\sigma\ \vDash\ \texttt{prev}\langle A \rangle
		}
		\and
	\infer
		{
		\sigma\ \vDash\ \texttt{changed}\langle A \rangle
		}
		{
		\sigma\ \vDash\ \neg A
		}
	\end{mathpar}
	
	
	\begin{definition}
	For an initial configuration $\sigma_0$, internal module $M_1$, and external module $M_2$, we define the following:
	\begin{itemize}
	\item
	$\Sigma(\sigma_0) = \{\sigma_0\} \cup\ \{\sigma | M_1\ \fcmp\ M_2\ \bullet\ \sigma_0 \leadsto^*\ \sigma\}$
	\item
	$\mathcal{L}(\sigma) = \{p | p\ \text{is primitive}\ \wedge\ \sigma\ \in\ \Sigma(\sigma_0)\ \wedge\ M_1\ \fcmp\ M_2\ \bullet\ \sigma_0, \sigma \vDash p\}$
	\item
	$\mathcal{H} = \{\ \{P\}\ \_\ \{Q\}\ |\ \sigma\ \sigma'\ \in\ \Sigma(\sigma_0)\ \wedge\ M_1\ \fcmp\ M_2\ \bullet\ \sigma\ \leadsto^\texttt{int}\ \sigma'\ \longrightarrow\ P(\sigma)\ \wedge\ Q(\sigma') \}$
	\end{itemize}
	\end{definition}
	
	\begin{definition}
	We define the set of internally changeable assertions as
	\begin{mathpar}
	\infer
		{}
		{v\ \in\ \mathcal{P}(\sigma)}
		\and
	\infer
		{e_1, e_2\ \in\ \mathcal{P}(\sigma)}
		{e_1\ =\ e_2\ \in\ \mathcal{P}(\sigma)}
		\and
	\infer
		{\sigma\ \vDash\texttt{internal}\langle \alpha \rangle}
		{\alpha.f \in\ \mathcal{P}(\sigma)}
		\and
	\infer
		{\sigma\ \vDash\texttt{internal}\langle \alpha \rangle}
		{\texttt{access}\langle \alpha, \_ \rangle \in\ \mathcal{P}(\sigma)}
		\and
	\infer
		{A\ \in\ \mathcal{P}(\sigma)}
		{\neg A \in\ \mathcal{P}(\sigma)}
		\and
	\infer
		{A_1, A_2\ \in\ \mathcal{P}(\sigma)}
		{A_1\ \wedge\ A_2 \in\ \mathcal{P}(\sigma)}
		\and
	\infer
		{A_1, A_2\ \in\ \mathcal{P}(\sigma)}
		{A_1\ \vee\ A_2 \in\ \mathcal{P}(\sigma)}
	\end{mathpar}
	\end{definition}
	Finally we state the following properties.
	\begin{mathpar}
	\infer
		{
		\sigma\ \vDash\ A \\ 
		\sigma\ \leadsto^\texttt{ex}\ \sigma' \\
		A\ \in\ \mathcal{P}(\sigma)
		}
		{
		\sigma'\ \vDash\ A
		}
		\and
	\infer
		{
		\sigma\ \vDash\ A \\
		\sigma'\ \vDash\ \neg A \\
		A\ \in\ \mathcal{P}(\sigma)
		}
		{
		\sigma\ \leadsto^\texttt{int}\ \sigma'
		}
		\and
	\infer
		{
		\sigma\ \leadsto^\texttt{int}\ \sigma'
		}
		{
		\exists\ P\ Q. [\{P\}\_\{Q\}\in\ \mathcal{H}\ \wedge\ P (\sigma)\ \wedge\ Q(\sigma')]
		}
	\end{mathpar}
	
	\subsection{Bank Account Example}
	
	\begin{lstlisting}[mathescape=true]
	class Account{
		field balance
		method deposit(from, amt){
			...
		}Pre:  $\exists$ b$_1$, this.balance = b$_1$
		       $\exists$ b$_2$, from.balance = b$_2$
		 Post: this.balance = b$_1$ + amt
		       from.balance = b$_2$ + amt
	}
	\end{lstlisting}
	 
	  \vspace{.01in}
	(BankSpec)\ \  $\triangleq$\\ 
	$\forall \texttt{a}.[\ \ \texttt{a}:\texttt{Account} \wedge \texttt{changes}\langle{\texttt{a.balance}}\rangle  \ \    
	    \longrightarrow \ \    \hfill$ \\
	  $\strut \hspace{2.3cm} 
	% TODO explain:
	% we no longer need Past here, as we are ion visible states 
	  \exists \texttt{o}. [\    \texttt{o}\ \texttt{calls}\ {\texttt{a.}}{\texttt{deposit}}({\_,\_}) \vee\  \texttt{o}\ \texttt{calls}\ {\_.}{\texttt{deposit}}({\texttt{a},\_})\rangle\  \ ] \ \ \ \ ] \hfill $
	\vspace{.05in}
	
	\noindent
	let the current configuration be $\sigma$. Thus we have
	$$(\texttt{a.balance}\ = \ \texttt{b})\ \in\ \mathcal{P}(\sigma)$$ and 
	$$\sigma \vDash \texttt{changes}\langle\texttt{a.balance}\ = \ \texttt{b})\rangle$$
	It follows that $\sigma$ must be an internal step, and that there must be a specified 
	internal path that modifies \texttt{a.balance}.
	It follows that the only possible path is a \texttt{deposit} call to either \texttt{a}
	or some other \texttt{Account} with \texttt{a} as the first argument.  
	
	%\subsection{Safe Example}
	%In the example below we assume that \texttt{secret} cannot be internal to another safe.
	% 
	%  \vspace{.01in}
	%(SafeSpec)\ \  $\triangleq$\\ 
	%$\forall \texttt{s}.[\ \ \texttt{s}:\texttt{Safe} \wedge\ \neg \texttt{s.treasure = null}\ \wedge\ \texttt{will}\langle \texttt{s.treasure = null} \rangle  \ \    
	%    \longrightarrow \ \    \hfill$ \\
	%  $\strut \hspace{2.3cm} 
	%  \exists \texttt{o}. [\    \texttt{access}\langle \texttt{o}, \texttt{s.secret} \rangle\  \ ] \ \ \ \ ] \hfill $
	%\vspace{.05in}
	%
	%$$\neg \texttt{s.treasure = null}$$
	%$$\texttt{will}\langle\neg \neg \texttt{s.treasure = null}\rangle$$
	%$$\texttt{will}\langle \texttt{changed}\langle \neg \texttt{s.treasure = null} \rangle \rangle$$
	%$$\texttt{will}\langle \texttt{changed}\langle \neg \texttt{s.treasure = null} \rangle \rangle$$
	
	
	
	\newpage

	
	\section{LTL Model of Chainmail}
	
	Following the section on LTL of Michael Huth and Mark Ryan, I am attempting to build
	a Chainmail model for checking satisfaction of Chainmail assertions. This diverges some what 
	from that book, especially in two ways:
	\begin{enumerate}
	\item
		The model is not generally finite as there may be an infinite number of program configurations. However, in proving satisfaction 
		we should only need to consider a finite number (is this true?).
	\item
		Assertions in Chainmail are not just about the current program configurations, but also about the transitions, i.e. method calls/returns etc.
		In fact I am trying to explicitly capture a way of modeling these transformations, and thus developing a standard way of proving satisfaction
		for code that includes external calls (i.e. something like the DAO).
	\end{enumerate}
	
	Firstly, state in the model is captured by program configurations, and transitions between configurations 
	is captured by pair reduction steps. That is, either a transition is a single reduction between external program
	configurations, or a series of steps through internal code.
	
	
	To model internal steps in the underlying language, we define transitions through internal code as a set of 
	functions mapping configurations to configurations:
	
	\begin{definition}[Module Transition functions]
	For a module $M$, all potential execution paths that consist of only module code
	can be represented by a finite set of transformations $\pi_M$, such that for any 
	external program configuration $\sigma$ of external module $M^\prime$ that either calls a method defined in $M$,
	or returns a value to a method in $M$, there exists $p\ \in\ \pi_M$ such that
	$$M\ \fcmp\ M^\prime\ \bullet\ \sigma\ \leadsto\ p(\sigma)$$
	\end{definition}
	
	It is in fact simple to check whether any particular program configuration makes a specific 
	module transition: i.e. if the configuration makes an internal method call, or if the 
	configuration is a method return, and the underlying frame is an internal frame.
	For this reason, we extend the above definition of a module transition path to include a predicate $R$
	that specifies how $p$ is called (or returned to), along with pre- and post- conditions.
	$$R\ :\ \{P\}\ p \{Q\}$$
	
	In the case of a method call, $R$ would resemble $x\ \texttt{calls}\ \texttt{y}.\texttt{m}(\ldots) $.
	In the case of a method return, $R$ would look something like $x\ \texttt{returns}\ v\ \texttt{to}\ y$.
	(should internal frames just have transition functions instead of a normal continuation?)
	
	\begin{definition}[Chainmail Model]
	A transition system $\mathcal{M} = (M_1, M_2, \sigma_0, \Sigma, L)$ 
	is a module pair $M_1$ and $M_2$,
	an initial program configuration $\sigma_0$,
	a set of program configurations $\Sigma$ such that
	$$\Sigma\ =\ \{ \sigma \vert\ M_1\ \fcmp\ M_2\ \bullet\ \sigma_0\ \leadsto^*\ \sigma \}\ \cup\ \{\sigma_0\} $$
	and a labelling function
	$L : \Sigma \mapsto P(Atoms)$.
	\end{definition}
	
	
	
	
	\begin{figure}[h]
	\[
	\begin{array}{lcr}
	\begin{array}{llr}
	a & ::= &\textit{Atoms}\\ 
	| & e & \\
	| & e\ :\ C & \\
	| & e\ \in\ S & \\
	| & \langle x\ \texttt{access}\ y \rangle &\\
	| & \texttt{internal}\langle x \rangle &\\
	| & \texttt{external}\langle y \rangle &\\
	| & \langle x\ \texttt{makes step}\ p \rangle & \\
	\end{array}&
	~~~~~~~~~~~~~~~ &
	\begin{array}{llr}
	A & ::= & \textit{Assertions}\\ 
	| & a & \\
	| & \neg A & \\
	| & A\ \wedge\ A & \\
	| & A\ \vee\ A & \\
	| & A\ \longrightarrow\ A & \\
	| & \forall\ x.\ A & \\
	| & \exists\ x.\ A & \\
	| & \forall\ S.\ A & \\
	| & \exists\ S.\ A & \\
	| & \texttt{next}\langle  A \rangle& \\
	| & \texttt{will}\langle  A \rangle& \\
	| & \texttt{prev}\langle  A \rangle& \\
	| & \texttt{was}\langle  A \rangle& 
	\end{array}
	\end{array}
	\]
	\caption{Assertions}
	\label{f:assertions}
	\end{figure}
	
	Fig. \ref{f:assertions} gives a syntax for Chainmail assertions.
	The main difference between the assertions of Fig. \ref{f:assertions}, and those of FASE is the 
	replacement of $\langle\texttt{x calls y.m(...)}\rangle$ with $\langle x\ \texttt{makes step}\ p\rangle$.
	
	\begin{definition}[Satisfaction of internal path step]
	$\sigma$ satisfies $\langle x\ \texttt{makes step}\ p \rangle$ if and only if there exists $R$, $P$, and $Q$ such that $R\ :\ \{P\}p\ \{Q\} \in\ \pi_M$ and
	$(R\ \sigma)$ holds.
	\end{definition}
	
	The second difference is the distinction between atomic assertions and general assertions.
	Atomic assertions are simple irreducible assertions.
	Satisfaction of atomic assertions can be determined in the absence of an initial configuration as they do not include temporal operators,
	thus we introduce an alternate form for satisfaction:
	$$M_1 \fcmp M_2,\ \sigma_1\ \vDash\ a$$
	Transitions between program configurations are defined using pair reduction:
	
	\begin{mathpar}
	\infer
		{
		(M_1 \circ M_2)\ \bullet\ \sigma_1\ \leadsto\ \sigma_2 \\
		classOf\ \sigma_1.(\texttt{this})\ \in M_2 \\
		classOf\ \sigma_2.(\texttt{this})\ \in M_2
		}
		{
		M_1 \fcmp M_2\ \bullet\ \sigma_1\ \leadsto\ \sigma_2
		}
		\and
	\infer
		{
		M_1 \fcmp M_2,\ \sigma_1\ \vDash\ \langle x\ \texttt{makes step}\ p\rangle
		}
		{
		M_1 \fcmp M_2\ \bullet\ \sigma\ \leadsto\ p(\sigma)
		}
	\end{mathpar}
	
	\subsection{Module Code without External Calls}
	
	In cases were modules do not include external calls, 
	module transition paths are relatively simple, and
	are just method calls. In this case, the syntax of Fig. \ref{f:assertions}
	becomes that of FASE.
	
	\subsection{Module Code with External Calls}

	
	\subsection{Changes Requiring Internal Reduction Steps}
	
	The main reason for broadening the chainmail expression of 
	an internal step to include not only method calls, but also  
	method returns is to be able to refer these boundaries inside of
	chainmail, and not just as part of some meta-reasoning. Here I
	connect the $\langle x \texttt{makes step}\ p\rangle$ to the 
	earlier reasoning about  \texttt{changed}.
	There are two categories of changes that require internal reduction steps:
	
	\begin{enumerate}
	\item Internal Mutation:
	\begin{mathpar}
	\infer
		{
		\exists \ x\ v. [\texttt{changed}\langle x.f\ =\ v \rangle\ \wedge\ 
		\langle x\ \texttt{internal} \rangle
		}
		{\exists\ y\ p, \langle y\ \texttt{makes step}\ p\rangle}
	\end{mathpar}
	\item Internal Encapsulation:
	\begin{mathpar}
	\infer
		{\exists \ x\ y. [\texttt{changed}\langle \neg \langle x\ \texttt{access}\ y\rangle \rangle\ \wedge\ \texttt{prev}\langle 
		\forall z. [\neg \langle z\ \texttt{access}\ y \rangle\ \vee\ \langle \texttt{internal}\ z \rangle] 
		\rangle ]}
		{\exists\ x'\ p, \langle x'\ \texttt{makes step}\ p\rangle}
	\end{mathpar}
	\end{enumerate}
	
	All other changes could occur as part of external client code.
	
	
	\subsubsection{Safe}
	
	As an example, consider a simplified version of the Safe specification:
	\begin{mathpar}
	\infer
		{\forall\ \texttt{s}, [\texttt{s}\ :\ \texttt{Safe}\ \wedge\ \texttt{will}\langle \texttt{change}\langle \texttt{s.treasure} \rangle\rangle]\ \longrightarrow\ 
		\texttt{will}\langle \exists o. [\langle o\ \texttt{access}\ \texttt{s.secret} \rangle] \rangle}
		{}
	\end{mathpar}
	\begin{mathpar}
	\infer
		{\forall\ \texttt{s}, [\texttt{s}\ :\ \texttt{Safe}\ \wedge\ \texttt{will}\langle \texttt{changed}\langle \texttt{s.treasure} \rangle\rangle]\ \longrightarrow\ 
		\texttt{will}\langle \exists o. [\langle o\ \texttt{access}\ \texttt{s.secret} \rangle] \rangle}
		{}
	\end{mathpar}
	\begin{mathpar}
	\infer
		{\forall\ \texttt{s}, [\texttt{s}\ :\ \texttt{Safe}\ \wedge\ \texttt{will}\langle \texttt{changed}\langle \texttt{s.treasure} \rangle\ \wedge\ \langle \texttt{s}\ \texttt{internal}\rangle\rangle]\ \longrightarrow}
		{}
	\end{mathpar}
	\begin{mathpar}
	\infer
		{\texttt{will}\langle \exists o. [\langle o\ \texttt{access}\ \texttt{s.secret} \rangle] \rangle}
		{}
	\end{mathpar}
	\begin{mathpar}
	\infer
		{\forall\ \texttt{s}, [\texttt{s}\ :\ \texttt{Safe}\ \wedge\ \texttt{will}\langle \texttt{changed}\langle \texttt{s.treasure} \rangle\ \wedge\ \langle \texttt{s}\ \texttt{internal}\rangle\rangle]\ \longrightarrow}
		{}
	\end{mathpar}
	\begin{mathpar}
	\infer
		{\texttt{will}\langle \exists o. [\langle o\ \texttt{access}\ \texttt{s.secret} \rangle] \rangle}
		{}
	\end{mathpar}
	
	


	
	
	
	
	
	
	
	
	
	\bibliographystyle{abbrv}
	\bibliography{main}
	
	\end{document}