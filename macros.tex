 \newcommand{\ttt}{\prg{true}}
\newcommand{\ff}{\prg{false}}
\newcommand{\unkn}{\prg{b???}}
\newcommand{\bv}{\prg{bval}}

 \newcommand{\m}{\prg{m}}
 \newcommand{\f}{\prg{f}}
 \renewcommand{\c}{\prg{C}}
 \renewcommand{\v}{\prg{v}}
  \newcommand{\this}{\kw{this}}
  \newcommand{\caller}{\kw{caller}}
\newcommand{\addr}{\ensuremath{\alpha}}

\newcommand{\prg}[1]{{\mbox{\tt{#1}}}}
 \newcommand{\prgCol}[1]{#1}

 \newcommand{\forget}[1]{}
\newcommand{\etc}{{\it etc.}}
\newcommand{\eg}{{\it e.g.\,}}
\newcommand{\ie}{{\it i.e.\,}}

 



\newcommand{\Pol}[1] {{\ensuremath{\prg{Pol}\_{\prg{#1}}}}}
%\newcommand{\MOne} {{\ensuremath{\prg{M}_{\prg{BA}}}}}
%\newcommand{\MTwo} {{\ensuremath{\prg{M}_{\prg{BA}'}}}}

\newcommand{\strongImplies}{\leqq} %{{ \,^\sqsubset\!\!\!_{\sim}\, }}
\newcommand{\weakImplies}{\lessapprox} %{{ \,^\sqsubset\!\!\!_{\sim}\, }}
\newcommand{\frames}{~\kw{frames}~}

\newcommand{\appref}[1]{see App.~\ref{#1}}
 
\newcommand{\LangOO} {\ensuremath{{\mathcal L}ang{_{\tt oo}}}}

% ------------------------------------------------------------------
%                                             positions, separations
\newcommand{\cf}{{\it c.f.~}}

\newcommand{\Arising}[1]{{{\mathcal{A}}\textrm{\textit{rising}}(#1)}}
\newcommand{\Initial} {{{\mathcal I}\!nitial}}

\newcommand{\kwa}[1]{\mbox{\bf{#1}}}
%\newcommand{\Gives}{\ensuremath{\mathcal{G}\textrm{\textit{ives}}}}

\newcommand{\CanAccess}[2]{\kwa{access}(#1,#2 )} 
\newcommand{\Calls}[1] {\kwa{calls}(#1)} 
\newcommand{\Future}[1]{\kwa{will}(#1)}
\newcommand{\Using}[2]{#1\,\kwa{in}\, #2} %{{{\mathcal U}}(#1,#2)}
\newcommand{\Past}[1]{\kwa{was}(#1)}% {\nabla #1} %{\lozenge\!\!\!\!\-\!\!-\,#1}
\newcommand{\Changes}[1]{\kwa{change}(#1)}


\newcommand{\A}{\ensuremath{A}}
\newcommand{\B}{\ensuremath{B}}

\newcommand{\SA}{\ensuremath{{^{\small{\prg{s}}}\! A}}}
\newcommand{\SAPrime}{\ensuremath{{^{\small{\prg{s}}}\! A'}}}

\newcommand{\SE}{\ensuremath{{^{\small{\prg{s}}}\!e}}}
\newcommand{\SEPrime}{\ensuremath{{^{\small{\prg{s}}}\!e'}}}
\newcommand{\SEOne}{\ensuremath{{^{\small{\prg{s}}}\!e_1}}}
\newcommand{\SETwo}{\ensuremath{{^{\small{\prg{s}}}\!e_2}}}

 %------------------------ syntax tables

\newcommand{\syntax}[1]{\prg{{\it #1}}}
\newcommand{\BBC}{$::=$} %in syntactic definitions
\newcommand{\SOR}{\ensuremath{\ \mid\ }} % BNF or
\newcommand{\MID}{{\SPsmall ~ \mid ~ \SPsmall }} % in sets


\newcommand{\pre}{\ensuremath{_{{pre}}}}   %kjx no \sc  in math mode
\newcommand{\post}{\ensuremath{_{{post}}}} %kjx no \sc  in math mode
\newcommand{\PRE}{\pre}
\newcommand{\POST}{\post}

%\newcommand{\eval}[2]{{\ensuremath{\langle{ {#1}}\rangle_{#2}}}}
\newcommand{\interp}[2]{{\ensuremath{\lfloor{ {#1}}\rfloor_{#2}}}}
%\newcommand{\interpBL}[1]{{\lceil   {#1}  \rfloor}}
%  { \langle \!\langle { {#1} \rangle \!\rangle}\! }
% { \langle   { {#1}  \rangle} }
% ------------------------------------------------------------------
%                                             English abbreviations

% ------------------------------------------------------------------
%                                              keywords, program text
\newcommand{\kw}[1]{{\bf{{#1}}}}  %{\sf {#1}}}}
%{\mbox{\prgCol{\rm{\bf {#1}}}}}
\newcommand{\lit}[1]{{\prg {#1}\xspace}}
\newcommand{\com}{\ensuremath{\prg{//}}}
%\newcommand{\cnf}{\ensuremath{\kappa}}

 \newcommand{\code}[1]{{\mbox{\tt{#1}}}}
\newcommand{\M}{\ensuremath{\prg{M}}}
%\newcommand{\C}{\ensuremath{\prg{C}}}

\newcommand{\semi}{\mbox{{\kw {;}}\ }}
\newcommand{\lb}{\prgCol{\mbox{\tt{\bf{\{ }}}}}
\newcommand{\rb}{\prgCol{\mbox{\tt{\bf{\} }}}}}
\newcommand{\lp}{\prgCol{\mbox{\tt{\bf{( }}}}}
\newcommand{\rp}{\prgCol{\mbox{\tt{\bf{) }}}}}
  
 
\newcommand{\mkpair}{\fatsemi}
\newcommand{\link}{\!\circ\!}


% structuring macros
\newcommand{\EndDefLemma}{\noindent $\bigtriangleup$}



 
%-----------------

\newcommand{\Body}[2]{\ensuremath{\mathcal{B}ody(#1,\prg{#2})}}
\newcommand{\LookUp}[2]{\ensuremath{ {#1}({\prg{#2}}) }}
\newcommand{\Undef}{{\ensuremath{\mathcal U\!}{\it df}}}
\newcommand{\ClassOf}[2] { \ensuremath{{\mathcal C}{\mathit{lass}}(#1)_{#2}}}



\newcommand{\inset}[3]{\prg{#1}\!\in\!\prg{#2},\ldots,\prg{#3}}
 \newcommand{\AND}{{\SPsmall {\mbox{and}} \SPsmall}}
\newcommand{\WITH}{{\SPsmall {\mbox{with}} \SPsmall}}

\newcommand{\IFF}{{\SP {\mbox{ iff }} \SP}}
\newcommand{\SP}{\strut \ \ \ \ }


\newcommand{\OR}{{\SPsmall {\mbox{or}} \SPsmall}}
\renewcommand{\implies}{{\ensuremath{\longrightarrow}}}
\newcommand{\upd}{{\mapsto}}






% ------------------------------------------- Infereence Rules and Tables ---------------- 

%Macros for inference rules
\newcommand{\inferencerule}[2]{
\begin{array}{l} #1 \\ \hline #2 \end{array}
}

\newcommand{\inferenceruleN}[3]
{
\begin{array}{l}
% \SP\SP\SP\SP\SP\SP\SP\SP
% \SP\SP\SP\SP\SP\SP\SP\SP
\SP\SP\SP\SP\SP\SP\SP\SP
\SP\SP\SP\SP\SP\SP  {\sf #1}
\\ #2  \\ \hline   #3
  \end{array}
}

\newcommand{\inferenceruleNN}[3]
{
\begin{array}{l}
\SP\SP\SP\SP\SP\SP\SP\SP
\SP\SP\SP\SP\SP\SP\SP\SP
\SP\SP\SP\SP\SP\SP\SP\SP
\SP\SP\SP\SP\SP\SP\SP\SP

   {\sf #1}
\\ #2  \\ \hline   #3
  \end{array}
}

%===========================================================================
%  Definition-Lemma-Theorem-Proof

%
\newif\ifNumberResults\NumberResultstrue
\def\@@opargbegintheorem#1#2#3{\@@@@begintheorem{\bf\@@thmname{#1}{#2}(#3)}}
\def\@@begintheorem#1#2{\@@@@begintheorem{\bf\@@thmname{#1}{#2}}}
\def\@@@@begintheorem#1{\par\removelastskip\smallskip\noindent{#1}}
\def\@@thmname#1#2{#1\ \ifNumberResults#2\ \fi}

% similarly \Proof or \begin{Proof}...\end{Proof}
% prefer proofs with statements if possible - hence \penalty700
%\let\qedsymbol\S% make it \square or \blacksquare if you like for kb
\let\qedsymbol \Box
\def\qed{\hfill{$\qedsymbol$}}
\def\Proof{\par\removelastskip\smallskip\penalty700\noindent{\bf Proof}\enskip}
\def\endProof{\qed\penalty-700 \smallskip}
\let\endproof\endProof

 
%\newtheorem{theo}{Theorem} this is necessay if we want eg \newtheorem{definition}[theo]{Definition} to work 
\newtheorem{definition}{Definition} %  \newtheorem{definition}[theo]{Definition}
\newtheorem{example}{Example} %  \newtheorem{example}[theo]{Example}
%\newtheorem{lemma}{Lemma} % \newtheorem{mylemma}[theo]{Lemma}
%\renewtheorem{mylemma}{Lemma} % \newtheorem{mylemma}[theo]{Lemma}
\newtheorem{conjecture}{Conjecture} %\newtheorem{conjecture}[theo]{Conjecture}
\newtheorem{theorem}{Theorem} %\newtheorem{theo}{Theorem}
\newtheorem{note}{Note} % \newtheorem{note}[theo]{Note}
\newtheorem{observation}{Observation} %  \newtheorem{observation}[theo]{Observation}


%--------------------------------- the ones that Susan introduced
\newcommand{\z}{{\prg z}}

\newcommand{\Fields}[3]{\ensuremath{{\mathcal F}(}\\Mg{#1},\prg{#2},
\prg{#3}\ensuremath{)} }
\newcommand{\FieldIds}[2]{\ensuremath{{\mathcal F}{\it {s}}(\M{#1},\prg{#2})}}
\newcommand{\Meths}[3]{\ensuremath{{\mathcal M}(}\M{#1},\prg{#2},
\prg{#3}\ensuremath{)} }







 


\newcommand{\rewriteLong}[1]
{\rtCol{\ensuremath{\ \leadsto\!\!\!\!\!\!_{_{#1}\,\ \ }}}}
\newcommand{\rewrite}[1]
{\rtCol{\ensuremath{\ \leadsto\!\!\!\!\!\!_{_{\M{#1}}\,\ \ }}}}
\newcommand{\rewriteP}
    {{\ensuremath{\ \ \leadsto\!\!_{_{\prg{P}}}\,\,\,}}}
  
  \newcommand{\nullPEC}{\lit {nullPntrExc}}

 


\newcommand{\back}{{$\!\!\!\!\!\!\!$}}


\newcommand{\WideFig}[3]
{
\begin{figure*}[t]
\begin{center}
\noindent
\fbox{
\begin{minipage}{4.7 in}
{#1} % the contents
\end{minipage}
}
\caption{#2}
\label{#3}
\end{center}
\end{figure*}
}


\newcommand{\WideFigWhere}[4] % you can specify where it should appear!
{
\begin{figure*}[{#4}]
\begin{center}
\noindent
\fbox{
\begin{minipage}{5. in}
{#1} % the contents
\end{minipage}
}
\caption{#2}
\label{#3}
\end{center}
\end{figure*}
}

\newcommand{\BigWideFigWhere}[4] % you can specify where it should appear!
{
\begin{figure*}[{#4}]
\begin{center}
\noindent
{\normalsize
\hrule
\begin{minipage}{5. in}
{#1} % the contents
\end{minipage}
\hrule
}
\caption{#2}
\label{#3}
\end{center}
\end{figure*}
}

\newcommand{\NotTooWideFigWhere}[4] % you can specify where it should appear!
{
\begin{figure*}[{#4}]
\begin{center}
\noindent
\fbox{
\begin{minipage}{4.3 in}
{#1} % the contents
\end{minipage}
}
\caption{#2}
\label{#3}
\end{center}
\end{figure*}
}


\newcommand{\opsemExprFig}
{\BigWideFigWhere {\opsemExpr} {Execution of expressions\MD}
{opsemTrad} {htbp} }


 
\newcommand{\BigNotTooWideFigWhere}[4] % you can specify where it should appear!
{
\begin{figure*}[{#4}]
\begin{center}
\noindent
{\normalsize
\hrule
\begin{minipage}{4.3 in}
{#1} % the contents
\end{minipage}
\hrule
}
\caption{#2}
\label{#3}
\end{center}
\end{figure*}
}

 

%]})
%}
