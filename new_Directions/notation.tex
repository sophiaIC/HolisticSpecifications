\label{sect:notation}



\subsection{Specifications} Proposed new notation for specifications 

\noindent 
For $from\ A1\ to\  A2\ onlyIf A3$, is: \ \ \ \ \ \ \  \  \ \ \ $\onlyIf{A1} {A2} {A3}$. \\
\ \ \ \ For $from\ A1\ next\  A2\ onlyIf A3$, it is: \ \ \ \ \  $\onlyIfSingle{A1} {A2} {A3}$.
\\
\ \ \ \  For  $from\ A1\ to\  A2\ onlyThrough A3$, is:  \ \ \ $\onlyThrough{A1} {A2} {A3}$.

\vspace{.3in}
\noindent
\textbf{Proposal} Should we  perhaps only write necessity specs which will never happen, eg rather than 
$\onlyIf{A_1}{A_2}{A_3}$ have something like 
$Never({(A_1 \wedge \neg A_3) \diamond A_2})$. In terms of temporal logic, this would have the meaning: {$\square( \neg ( {(A_1 \wedge \neg A_3)  \wedge  \diamond A_2}))$}.  
Or, we could go one step further, and specify in the terms of $\square( ( A_1 \wedge \neg A_3) \diamond \neg A_2)$. This would turn our specs into binary operators. 

\vspace{.1in}
\noindent
We could have something like \\
$(*)\ \  \strut \ \ \ \ \ \ \ \ \ \ \ \ \ \ \ \   \TwoState{A} {A'}$  \\
which would be like a shorthand for  the temporal  $\square ( A \rightarrow \diamond A')$, which means 
\\
$(**)\ \  \strut \ \ \ \ \ \ \ \ \ \ \ \ \ \ \ \ \forall \sigma, \sigma',\prg{cmd}.\ [ \ \sigma \models A  \ \wedge\ \sigma, \prg{cmd} \leadsto^* \sigma' \    \   \longrightarrow \ \ \sigma' \models A'\ ].$

\vspace{.1in}
\noindent
Then, our Bank spec could have the form 

$\TwoState {\prg{a:Account}\wedge \inside{\prg{a.passwd}} \wedge \prg{a.balance}=b}  {\prg{a.balance}\geq b}$ 

\vspace{.1in}
\noindent
\textbf{Discussion}  Having only binary operators would be good, but would open two problems a) how to describe "only-through", b) the "narrative" would be different. We would not have "necessary preconditions" any more. Our spec would be more like 2-state invariants.

\paragraph{Specification Implication}
\label{lab:def:imply}

\begin{definition}
A specification $\SpecB{'}$ is \emph{stronger} than a specification $\SpecB{}$:

$\begin{array}{lll}
 \SpecB{'}\prec \SpecB{} & \triangleq &  
 \forall  \Mod . [\ \  \Mod \models \SpecB{'} \ \ \longrightarrow\ \   \Mod  \models \SpecB{}\ \ ]
  \end{array}
$

\end{definition}\footnote{ We should explore some $\prec$ relations, in the future. And relate with similar in temporal logic. But perhaps all the $\prec$ are already covered in the inference rules that Julian had created? Check whether Julian's rules are complete?}


\subsection{Assertions}

Proposed new notation for access: $\access{y} {x}$, and for external: $\external{p}$. Therefore, we will have nothing like "words" or "keywords" in the assertions, or the specifications/

Based on the above, we define "outside", as something that is accessible from something external:

\begin{definition}

We define the predicates $\outside{\_}$, and $\inside{\_}$ as follows:

$\begin{array}{lll}
   \Mod, \sigma \models \outside{x} & \triangleq & \ \Mod, \sigma \models \exists y.[ \ \access{y} {x} \wedge \external {y}  \ ] 
  ~ \\
   \\
      \Mod, \sigma \models \inside{x} & \triangleq & \ \neg \ (\  \Mod, \sigma \models \outside{x}  \ )
  \end{array}
$

\end{definition}



\paragraph{Implication}

Implication in assertions must be $\longrightarrow$. But what would it be in the metalanguage, eg in Def. \ref{lab:def:imply}.