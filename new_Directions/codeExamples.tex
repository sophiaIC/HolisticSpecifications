
Here we will write examples of the codes. We denote though \prg{untrust.unkn(...)} the call to an untrusted (external) object.


\subsection{Something}

These  method s have no "story"; they serve  as warnings about things we should not take for grated.

\begin{lstlisting}[language=chainmail, mathescape=true, frame=lines]
class Account

     void warning( ) {
          p := new Password
          p' := new Password
          a := new Accoount.
          a.setPassword(null,p);
          untrust.unkn(a)
          this.password := p  /*  and in variant\_1 */     this.password := p' 
          return p    
     }
     
     void dare_you(){
          this.status := frozen    //  to transfer we need the account to be unfrozen 
          untrust.unkn(this.password)
          this.password:=new Password
          this.frozen := false
     }   // SD: hmhhh,  essentially we de-activated the account! We need 2.3.1!!!
     
...
\end{lstlisting}


\subsection*{Deactivate and leak old}

This method writes a new \prg{Passowrd} into the account, and thus essentially disables it. And it returns the old password.

\begin{lstlisting}[language=chainmail, mathescape=true, frame=lines]
class Account

     Password deacrtivate_and_leak_old( ) {
          p := this.password
          this.password := new Password
          return p    
     }
     
...
\end{lstlisting}

Julian will write how this function can be used in the client to xxxx.

\subsection{Transfer with external}

This method  does half the job of  \prg{transfer}, then calls "outside", and then does the rest. Was meant to demo proof of preservation of currency. Proof ot fre- and post in the presence of unknown.

\begin{lstlisting}[language=chainmail, mathescape=true, frame=lines]
class Account

     void transfer_2 (Password p, Account to, int amt) {
          if (p==this.password){
                  this.balance -= 2
                  to.balance += 2
                  untrust.unkn();
                  this.balance -= amt -2
                  to.balance += amt-2
                     
     }
     
...
\end{lstlisting}

\subsection{Bank may leak unused Passwords}
\label{s:deactivate_leak_old}

Bank keeps all unused passwords in a list.  Sophia will expand it once "deactive and leak old" is done.

\begin{lstlisting}[language=chainmail, mathescape=true, frame=lines]

class Bank

     Password makePassword( ) {
          p := new Password
          this.listUnusedPwds.enter(p) 
          return p;                    
     }
     
     Password getMeAPassword(){
          p := this.listUnusedPwds.popAndTop();
          return p;
     }
     
...
\end{lstlisting}

\subsection{Safe External Call}
\label{s:safe:extern}

Here we overwrite the password before making the external calls

\begin{lstlisting}[language=chainmail, mathescape=true, frame=lines]
class MrBean
  method transfer(from, to, pwd)
    from.transfer(to, pwd)
    p = new Password()
    from.setP(pwd, p)
    log.logger(from)
    from.setP(p, pwd)
\end{lstlisting}


\subsection{BlackAdder's Box}
\label{s:blackadder_box}

Below is a variant of the example of 2.3.1 where the Bank Account has a box field 
that may contain the password. If it contains the password, then it will allow anyone to change the balance.

This example has "ambient authority", but the authority is not explicit. INTERESTING!

\begin{lstlisting}[language=chainmail, mathescape=true, frame=lines]
module BankAccount
  class BankAccount
    field password
    field balance
    field box
    method transfer(pwd, to)
      if pwd == password || box.open() == password
        this.balance -= 10
        to.balance += 10
    method setBox(b)
      this.box = b
    method setPassword(pwd, newP)
      if this.password == pwd
        this.password = newP
      
module Ext
  class Box
    field thing
    method open()
      return thing
    method setThing(t)
      this.thing = t
    
module Client
  class BlackAdder
    method doSomething()
      a = new Account()
      p = new Password()
      b = new Box()
      b.setThing(p)
      a.setPassword(null, p)
      a.setBox(b)
      u = new Unknown()
      u.stuff(a)   
\end{lstlisting}

On Friday, we had thought that RULE-2 from Sect.  \ref{S:ruleTwo} would be unsound if applied on line 34 (the call \prg{u.stuff(a)}).
Namely, we can see that $\prg{BankAccount} \models \Spec{{pwd\_prot\_bal}}$.  
Also, in line 34, the account is no longer protected, as its box contains the password. On Friday we thought that RULE-2 would 
erroneously guarantee that the balance would not decrease.

Here we overwrite the password before making the external calls. 
We can prove it with this week's work.
\SP But SD now argues that Rule-2 would make no such guarantee, Namely, in line 34, we are passing \prg{a} as an argument, and \prg{a} has access to its box (which is external), and which has access to the password. 
Therefore, in we consider only the objects that are (transitively) accessible from the frame that gets created 
in line 34, then $\inside{a.pwd}$ would not hold. And therefore $\Spec{{pwd\_prot\_bal}}$ would not be applicable. 



\subsection{DAO}
\label{s:DAO}

\SP Sophia maintains that the DAO breaks the invariant even if we consider "shallow semantics"
Here the DAO problem paraphrased: 

\begin{lstlisting}[language=chainmail, mathescape=true, frame=lines]
private class Bank
    field money: int
    
    method pay(u: Unknown, amt: int)
    {
         money:- ant
         ...
     }

class DAOAccount
	field balance: int
	field myBank: Bank
	
	method payOut(u: Unknown){
		if balance>0 then
			myBank.pay(u,this.balance);
			u.notify("made the payment")
	}
}
\end{lstlisting}

On line 17 we are making an external call. At this point, $ \Spec{curr\_consistnt}$ from Sect. \ref{bank:curr:consist} mandates that we have to establish that ${\prg{myBank}.\prg{currcy}= \prg{myBank}.\prg{money}}$. However, ${\prg{myBank}.\prg{currcy}= \prg{myBank}.\prg{money}}$ does not hold. Note that for this argument we make no special use of NL, and it makes no difference wether we have shallow or deep semantics.
