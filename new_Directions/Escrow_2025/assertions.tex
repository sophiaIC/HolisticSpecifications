

\label{sub:SpecO}

We assume a sequence of specification identifiers, $S_1$, .... $S_m$, and a sequence of module identifiers, $M_1$,... $M_n$, where $n\geq m$. We also assume lookup functions, $\Spec$ and $\Mod$, which map the specification identifiers to specifications, and the module identifiers to modules. 


Our assertions are    standard  (\eg properties of the values of expressions,  connectives, quantification \etc),   about protection (\ie ${\protectedFrom{{\re}} {{\re}}} $ and  $ {\inside {{\re}}} $), and about adherence to Specifications ($ {\obeys {\re} {S}}$). Moreover, in order to be able to give a well-founded definition to the meaning of assertions, we introduce stratification to  assertions: an assertion at level $n$, denoted by $A^n$, may contain  $ {\obeys {\re} {S_i}}$ for $i<n$ in any positions, and may contain $ {\obeys {\re} {S_n}}$ in positive positions only.

 

\begin{definition}[Staified Assertions]
\label{def:assert:syntax}

Given a sequence of specification identifiers,  $S_1$, .... $S_m$, for any number $n\in[0..m)$, we define $A^n$, the syntax of assertions at level $n$, as follows:

$
\begin{syntax}
\syntaxElement{A^c}{}
		{
		\syntaxline
				{{\re}}
				{{\re} : C}
				{\external{{\re}}}
 				{\protectedFrom{{\re}} {{\re}}} 
				 {\inside {{\re}}} 
				  {\neg A^c}
				{A^c \wedge A^c}	
		\endsyntaxline
		}
\endSyntaxElement
\\
\syntaxElement{A^0}{}
		{
		\syntaxline
				{{\re}}
				{{\re} : C}
				{\external{{\re}}}
 				{\protectedFrom{{\re}} {{\re}}} 
				 {\inside {{\re}}} 
				  {\neg A^0}
				{A^0 \wedge A^0\ \ \ }
				 {\all{x:C}{A^0}}			
		\endsyntaxline
		}
\endSyntaxElement
 \\
\syntaxElement{A^{n+1}}{}
		{
		\syntaxline
				{{\re}}
				{{\re} : C}
				{\external{{\re}}}
 				{\protectedFrom{{\re}} {{\re}}} 
				 {\inside {{\re}}} 
				  {\neg  A^n}
				 { A^{n+1}  \wedge A^{n+1}}
				 {\all{x:C}{A^{n+1}}}	
                  \endsyntaxline
                }
                {
		\ \ 
		\syntaxline
				 {\obeys {\re} {S_{n+1}}}		
		\endsyntaxline
		}
\endSyntaxElement
\end{syntax}
$

 
\end{definition}

We write $A$ for any $A\in A^m$ for some $m\in \mathbb{N}$.
Concrete assertions, $A^c$ will be used when modules give meaning to abstract predicates.

Like assertions, specifications are stratified too:

\begin{definition}[Specifications]
A specification consists of a name, $S_n$, and a set of stratified policies, $Pol^n$, defined as:

$
\begin{syntax}
\syntaxElement{Spec^m}{}
		{
		\syntaxline
				{ {\textbf{specification}}\ \  S_n \ \{ \ \ { (\ {\textbf{abstr}}\ P ( x^* ) \ )^*} \ \  (\ Pol^n\ )^* \ \ \} }		
		\endsyntaxline
		}
\endSyntaxElement
\\
 \syntaxElement{Pol^{1}}{}
		{
		\syntaxline				
 				{ \{ \ A^0 \ \} \ \  \prg{this.m(y}^*\prg{)} \ \ 	     \{ \ A^1\ \} \ \ \	}
% 				{ \{ \ \wf A \ \} \ \ \prg{any\_code} \ \   \{ \ \wf A \ \}	}	
				{ {\textbf{invr}}\ A^1 \ }	
		\endsyntaxline
		}
\endSyntaxElement
\\
\syntaxElement{Pol^{n+1}}{}
		{
		\syntaxline				
 				{ \{ \ A^n \ \} \ \  \prg{this.m(y}^*\prg{)} \ \ 	     \{ \  A^{n+1} \ \}	}
% 				{ \{ \ \wf A \ \} \ \ \prg{any\_code} \ \   \{ \ \wf A \ \}	}	
				{ {\textbf{invr}}\ A^{n+1} \  }	
		\endsyntaxline
		}
\endSyntaxElement
\end{syntax}
$
\footnote{@James: you were right that "obeys" in pre-conditions is as.if it appeared in a negative position.}

 \end{definition} 
\footnote{ AUTHORS:  
 Note that $\re$ are ghostfields, and therefore can also be abstract predicates whose bodies are defined in the modules -- same as in ghostfields. 
 Perhaps we do not need to clarify this.
 }
 
%\label{def:chainmail-semantics-all}
%\label{dup:def:chainmail-semantics}
%\noindent
We now define satisfaction  of assertions $A$, in the context of a modules $\Mod$, specifications $\Spec$, and a state $\sigma$.
The judgment $\satM   \sigma m A$ is defined only if $A\in A^n$.
The judgement $\satMN \sigma m n A$ is used to define $\satM \sigma m A$.

@authors: In contract with OOPSLA-25, we now use \emph{all} the modules. We need that for Def ???

@authors:  What is external? Is it external to a module-id? or a spec-id?


\begin{definition}[Stratified Satisfaction of Assertions] 
\label{def:chainmail-semantics}
\label{def:chainmail-protection-from}
\label{sect:semantics:assert:prtFrom}
 \label{def:chainmail-protection}
% We define satisfaction of basic assertions, $\ba A$, by a % program  state $\sigma$ and 
% module $M$ as:\footnote{AUTHORS: This is essentially OOPSLA 2025}
is defined below, where $m$ and $n$ are arbitrary out of $\mathbb N$ -- unless explicitly quantified.
%We assume that the number of specifications in $\Spec$ is ${\mathcal s}$.

\begin{enumerate}
\item
\label{sat}
$\sat \sigma  A \ \ \ \triangleq \ \ \   \satM  \sigma {\mathcal s}  A $,\ \ \  where ${\mathcal s}$=number of specifications in $\Spec$.\footnote{But what about externals?}
\item
\label{satM}
$\satM \sigma m A \ \ \ \triangleq \ \ \  \forall n\in \mathbb{N}. [\ \satMN \sigma m n A\ ]$
\item
\label{cExpr}
$\satMN \sigma m n {{\re}}\ \ \ \triangleq \ \ \   \eval{\Mod}{\sigma}{{\re}}{\true}$  
\item
\label{cClass}
$\satMN \sigma m n {\re : C}\ \ \ \triangleq \ \ \   \exists \alpha.[\ \eval{\Mod}{\sigma}{{\re}}{\alpha}\   \wedge \ \class{\alpha} {\sigma}= C\ ]$  
\item
\label{cExternal}
$\satMN \sigma m n {\external \re}\ \ \ \triangleq \ \ \  \exists \alpha.[\ \eval{\Mod}{\sigma} {\re} {\alpha}\   \wedge \ \class{\alpha} {\sigma}= C  \ \wedge \ C \notin ???? \ ]$ 
 \item
 \label{cProtected}
$\satMN \sigma m n {\protectedFrom \re  {\re_o}}$ $\ \  \triangleq\ \ $ 
$\exists \alpha, \alpha_{o}. [$ 
  \begin{enumerate}
 \item
 $ \eval{\Mod}{\sigma}{{\re}}{\alpha}\ \ \ \wedge\ \ \eval{\Mod}{\sigma}{{\re_o}}{\alpha_o}$,
  \item
$\alpha\neq \alpha_0$,
 \item
$\forall \alpha'.\forall f.[\ \alpha' \in {\Relevant {\alpha_o} {\sigma}} \wedge\   \satMN \sigma m n {\external {\alpha'}} 
\ \ \Longrightarrow \ \  
  \interpret {\sigma} {\alpha'.f} \neq \alpha     \ ] $  $  \strut \hspace{.25cm} ] $.
\end{enumerate}

\item
 \label{sect:semantics:assert:prt}
$\satMN \sigma m n {\inside \re} $ $\ \  \triangleq\ \  $ 
$\exists \alpha. [$  
\begin{enumerate}
\item
$ \eval{\Mod}{\sigma}{{\re}}{\alpha}$,
 \item 
 $\satMN {\sigma} m n {\extThis}\ \ \Longrightarrow\ \ \forall x\!\in\! \sigma. \satMN {\sigma} m n {x\neq \alpha}$,
 \item
$\forall \alpha'.\forall f.[\ \alpha' \in {\LRelevantO {\sigma}} \wedge\    \satMN \sigma m n {\external {\alpha'}} 
\ \ \Longrightarrow \ \  
  \interpret {\sigma} {\alpha'.f} \neq \alpha     \ ]$ 
 \strut \hspace{.25cm} ].    \end{enumerate} 
 
  \item
 $\satMN \sigma m n {\neg \ A}\ \ \ \triangleq \ \ \   \notsatM   {\sigma} {m-1}  {A}$
\item
$\satMN \sigma m n {A_1\ \wedge\ A_2}\ \ \ \triangleq \ \ \   \satMN \sigma m n {A_1} \   \wedge \  \satMN \sigma m n {A_2}$
\item
\label{quant1}
$\satMN \sigma m n  {\all{x:C}{A}} \ \ \ \triangleq \ \ \   
\forall \alpha.[\   \satMN \sigma m n {\alpha:C}  \ \Longrightarrow   \ \satMN \sigma m n  {A[\alpha/x]} \ ] $

\item
\label{satObeys}
$\satM   \sigma m    {\obeys  \re {S_i}} \ \ \ \triangleq \ \ \   \forall n\in \mathbb{N}.[ \satMN  \sigma m n   {\obeys \re {S_i} }$.

\item
\label{satObeysMNa}
$\satMN  \sigma m n   {\obeys  \re {S_i}} \ \ \ \triangleq \ \ \   \satM  \sigma i    {\obeys  \re {S_i}} $\ \ \ \ if $i < m$.

\item
\label{satObeysMNb}
$\satMN  \sigma m 0   {\obeys  \re {S_m}} \ \ \ \triangleq \ \ \  \exists \alpha. \eval{\Mod}{\sigma}{{\re}}{\alpha}$

\item
\label{satObeysMNc}
$\satMN  \sigma m {i+1}   {\obeys  \re {S_m}} \ \ \ \triangleq \ \  $

\begin{enumerate}
\item
$ \satMN  \sigma m {i}   {\obeys  \re {S_m}}$
\item 
$\forall k\leq i.\forall \sigma'\forall pol.[ \ pol\in\Spec(S_m) \ \wedge \Mod,\sigma \leadsto_k \sigma' \ 
\Longrightarrow \ \satMN  \sigma m {i+1}   {\sats  \re {pol}} $
\end{enumerate}

\item
$ \satMN  \sigma m {i+1}   {\sats  \re {\{ \ A  \ \} \   \prg{this.m(y}^*\prg{)} \  \{ \  A'  \ \} } }
 \ \ \ \triangleq \ \ \
 \forall \sigma'. $\\
$ [ \  \  \ \satM   {\sigma} {m-1}  {A[\re/\prg{this}]}\ \ \wedge\ \  \sigma.cont=(\prg{this}.m(\overline y))[\re/\prg{this}] \ \ \wedge\ \
\Mod, \sigma\leadsto^*_{fin} \sigma'  $\\
$\strut \hspace{3cm} \hfill \ \   \ \  \  \Longrightarrow\ \ \ \  \satMN  {\sigma'} m {i}  {A'[\re/\prg{this}] }\ \ \ ]$

\end{enumerate}

 \end{definition} 

 
% We now define satisfaction of well-formed assertions, For this, we will be using step-indexing. 
% We use $\satisfiesAK M \sigma k A$, where $k\in \mathbb{N}$ to express that $A$ is satisfied at level $k$.
% 
% 
% \begin{definition}[Satisfaction of Well-fomedAssertions] 
%For any module $M$, state $\sigma$, assertion $A$ and $k\!\in\! \mathbb{N}$ we define
%\begin{enumerate}
%\item
%\label{cExpr}
%$\satisfiesAK{M}{\sigma} k {\re}  \ \ \triangleq \ \ \   \satisfiesAB{M}{\sigma}{{\re}}$
%\item
%\label{cClass}
%$\satisfiesAK{M}{\sigma} k {\re  : C}\ \ \ \triangleq \ \ \    \satisfiesAB{M}{\sigma} {\re  : C}$
%\item
%\label{cExternal}
%$\satisfiesAK{M}{\sigma} k {\external{{\re}}} \ \ \ \triangleq \ \ \satisfiesAB{M}{\sigma}{\external{{\re}}} $
% 
% \item
% \label{cProtected}
%$\satisfiesAK{M}{\sigma} k {\protectedFrom{{\re}} {{\re_{o}}}}$ $\ \  \triangleq\ \ $
%$\satisfiesAB{M}{\sigma}{\protectedFrom{{\re}} {{\re_{o}}}}$ 
%
%\item
% \label{sect:semantics:assert:prt}
%$\satisfiesAK{M}{\sigma} k {\inside{{\re}}} $ $\ \  \triangleq\ \  $ 
%$\satisfiesAB{M}{\sigma}{\inside{{\re}}}$ 
% 
% 
%%  \item
%%$\satisfiesAK{M}{\sigma} k {\neg \ba A}\ \ \ \triangleq \ \ \   \satisfiesAB{M}{\sigma}{\neg \ba A}$
%%\item
%%$\satisfiesAK{M}{\sigma} k {\wf A_1\ \wedge\ \wf A_2}\ \ \ \triangleq \ \ \   \satisfiesAK{M}{\sigma} k {\wf A_1}\ \wedge\ \satisfiesAK{M}{\sigma} k {\wf A_2}$
%%\item
%%\label{quant1}
%%$\satisfiesAK{M}   {\sigma} k {\all{x:C}{\wf A}} \ \ \ \triangleq \ \ \   
%%\forall \alpha.[\   \satisfiesAB {M}{\sigma} {\alpha:C}  \ \Longrightarrow   \ \satisfiesAK{M}{\sigma}  k {\wf A[\alpha/x]} \ ] $
%%
%%\item
%%$\satisfiesAK{M} \sigma k {\obeys \re S}$ \ \ is defined by cases on $k$:
%%\begin{itemize}
%%\item
%%$\satisfiesAK{M} \sigma  0 {\obeys \re S}  \ \ \ \ \ \triangleq \ \ \   \exists \alpha. \eval{M}{\sigma}{{\re}}{\alpha}$
%%\item
%%$\satisfiesAK{M} \sigma  {k+1} {\obeys {\re} {S}}  \ \ \ \triangleq \ \ \   \satisfiesAK{M} \sigma {k} {\obeys {\re} {S}} \wedge \forall pol\! \in\! S. \lsatisfiesAK{M}\sigma  {k} {\obeys {\re} {pol}}$
%%\end{itemize}
%%
%%
%%\end{itemize}
%\end{enumerate}
%
% \end{definition} 
 
 
 
 
