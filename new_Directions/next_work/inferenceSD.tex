\newcommand{\extract}[1]{\ensuremath{{\mathcal I}\!nvs(#1)}}
\newcommand{\specs}[1]{\ensuremath{Specs(#1)}}%{\ensuremath{{\mathcal S}\!pecs(#1)}}

\section{Correct previous}
Below we define 
$\pushSymbolAA$, which as we will see translates an assertion from the viewpoint of the callee, to that of the caller:
it applies to assertions and sequences of variables:  $\PushAS y A$   guarantees that $A$ will hold when the values of $y$ have been pushed onto a new frame:
 thus, $\PushAS y A$ is \emph{hypothetical}.
 % : if a state satisfies $\PushAS y A$, then after pushing
% onto that state a frame which contains the values  of $\overline y$, assertion $A$ will hold. }

% Lemma \ref{lemma:vars:to:addresses} says that: (1)Validity of an assertion $A$ in the context of a state $\sigma$ implies  validity of the assertion resulting  from substituting free variables according to the top frame ($A[\sigma]$)}.
%%(2) A bounded execution step (thus not returning from current call)
%in an external state preserves absolute protection.
%(3) An objects which is protected from  the receiver and arguments of a method call, is protected   after the corresponding frame has been pushed onto the stack.


\begin{definition}
%\begin{figure}[hbt]
$
\begin{array}{c}
\begin{array}{l}
\begin{array}{rclcrcl}
  \PushAS y {(\inside \re)} & \triangleq &  \protectedFrom \re {\overline {y} }
  & \ \ \  \ &
  \PushAS y   {(A_1  \wedge  A_2)} & \triangleq &  (\PushAS y  { A_1})  \wedge  ( \PushAS y  {A_2} )  
\\ 
 \PushAS y {(\protectedFrom \re {\overline {u}})} &  \triangleq& \protectedFrom \re {\overline {u}} 
  & &
 \PushAS y  {(\forall x:C.A)} & \triangleq & \forall x:C.({\PushAS y A} )  
  \\  
  \PushAS y  {(\internal \re)} &  \triangleq & {\internal \re}  %   \PushAS y  {(\external \re)} & \triangleq &   {\external \re}
  & & 
  \PushAS y  {(\neg A)} &  \triangleq & \neg( {\PushAS y A} )  
    \\
     \PushAS y  {\re} &  \triangleq&   \re %    \PushAS y  {(\internal \re)} &  \triangleq & {\internal \re}
    & &
    \PushAS y  {(\re:C)} &  \triangleq&   \re:C 
 \end{array}
\end{array}
\end{array}
$
% \caption{The $\pushSymbolAA$  operator  } 
\label{f:Push}
%\end{figure}
\end{definition}

%$\pushSymbolAA$  operator is  defined in Fig. \ref{f:Push}. 
Only the first equation is interesting, i.e.  $\PushAS y {(\inside x)}$: For 
$x$ to be protected from the viewpoint of the callee, it should be protected from all the call's arguments,
\ie  $\protectedFrom x {\overline {y}}$. 
In all other cases,   $\pushSymbolAA$  leaves simple $\re$'s unmodified %(i.e. the second to sixth equation), 
 or is applied to the sub-assertions. % (i.e. the seventh to eleventh equation).


\newcommand{\sigmas}{\widetilde \sigma}



\vspace{.1cm}

 Lemma \ref{lemma:push:ass:state}  states that $\pushSymbolAA$ is the counterpart to the states' operator $\pushSymbol$:
 %states the guarantees given by $\pushSymbolAA$: \  
 (1) If \ %the current state 
$\sigma$ satisfies  $\PushAS y A$, then   { the callee state (after pushing a frame with the values of $\overline y$) 
will satisfy $A$}. %after pushing a frame with the values of $\overline y$ (here $\PushS {y} {\sigma}$)
(2) is the opposite: {If the callee state (in which we pushed a frame with the values for $\overline y, \overline z$) satisfies $A$, then the
caller state will satisfy 
% (here , conversely,   if a state satisfies  $A$ after a top frame containing the  values of $\overline y$  and some other variables has been pushed % (here \PushSLong {(\overline y, \overline z)} \sigma), 
$\PushAS {y} {A}$. } % will hold {in the caller state}. % after popping that frame.
%, and mapping the free variables of $A$ to their values  in the state  before.

\section{Specs}
We  now discuss   syntax and semantics of \SpecLang specifications, and illustrate through examples.

\subsection{\SpecLang syntax }

Our specifications language, % \SpecLang,  
 supports the one state invariants, two state invariants,  method specifications, and the conjunction of such specifications:
 
\begin{definition} [\SpecLang Specifications]  { } %  We define the \emph{syntax} of \SpecLang and well-frmedness:
\noindent

\begin{itemize}
\item
\label{f:holistic-syntax}
\[
\begin{syntax}
\syntaxElement{S}{ }
		  {\syntaxline
                                  {\ \ \ \ \  \OneStateQ {\overline {x:C}} {A} }	
				{\TwoStatesN {\overline {x:C}} {A}  }
				  {\mprepost{A}{C}{m}{x}{C}{A} }
		 \endsyntaxline
 		}
  		{
 		\syntaxline
 				{\mprepostN{A}{C}{m}{x}{C}{A} {A}}%{A[\!\![ \,C\!::\!m(\overline{x:C})\,]\!\!] A }}
				{S\, \wedge \, S}
		 \endsyntaxline
 		}
\endSyntaxElement\\
\end{syntax}
\]

\item
{\emph{Well-formedness},  $\vdash S$,  is   defined by cases on $S$:\\
 $\strut \ \  \vdash {\OneStateQ {\overline {x:C}} {A}}$  if  $fv(A)\subseteq\{  \overline x \}$; \ \ \ \ \ \ \ \ \ \ \ 
 $\vdash {\TwoStatesQ {\overline {x:C}} {A} {A'}}$  if    $fv(A), fv(A')\subseteq \{ \overline x \}$; \\
 $\strut \ \  \vdash {\mprepost{A}{C}{m}{x}{C}{A'} }$   if      $fv(A')\subseteq  fv(A)\cup \{\prg{res}, \prg{this}, \overline x\}$; \\
 $\strut \ \  \vdash {\mprepostN{A}{C}{m}{x}{C}{A'} {A'}}$   if      $fv(A')\subseteq  fv(A)\cup \{\prg{res}, \prg{this}, \overline x\}$; \\
 $\strut \ \  \vdash S\, \wedge \, S'$  if  $\vdash S$   and   $\vdash S'$.  
}
\end{itemize} 
\end{definition}

\begin{definition}% [Semantics of \SpecLang Specifications]

We define $\satisfies{M}{{S}}$ by cases over the five possible syntactic forms.

\label{def:necessity-semantics}

\begin{enumerate}
\item
$
\satisfies{M}{ \OneStateQ {\overline {x:C}} {A} } \  \ \ \ \ \ \ \ \ \ \ \ \ \ \mbox{iff}  \     
    \begin{cases}
     \mbox{for all }   \Mtwo  \mbox{ and }  \sigma:\\
   \ \ \ \  [\ \ \   \arising{\sigma}{M\madd\Mtwo}  \ \  \wedge\ \ M, \sigma  \models {\external{\prg{this}}} \\
    \ \ \ \ \ \   \ \ \ \Longrightarrow\\
    \ \ \ \ \ \  \  \satisfiesA{M}{\sigma}{\forall \overline{x:C}.A}\ \  \  ] 
    \end{cases} 
 $
 \item
 $\satisfies{M}{\TwoStatesN {\overline {x:C}} {A}} \  \ \ \ \ \ \ \ \ \ \ \ \ \ \mbox{iff}  \   \
    \begin{cases}
     \mbox{for all }   \Mtwo,  \sigma, , \sigma', \mbox{ and }  \overline{\alpha}    \\
 \ \ \ \  [ \ \ { \arising{\sigma}{M\madd \Mtwo }\   \  \wedge\ \ \GRelevant {\overline \alpha}  \sigma}\ \ \ \wedge  \\
  \ \ \ \  \ \ \  \satisfiesA {M}   {\sigma}  {(\overline {\alpha :C} \ \wedge\  {A[\overline{\alpha/x}]}\ \wedge \  {\external{\prg{this}}}) } \ \ \ \wedge   \\
 \ \ \ \   \ \ \ {\leadstoBoundedStar {M\madd \Mtwo}{\sigma}  {\sigma'} } \ \ \wedge\ \  M, \sigma' \models {\external{\prg{this}}}  \\
 \ \ \ \   \ \ \ \ \  \ \Longrightarrow   \\
 \ \ \ \   \ \ \  {\satisfiesA{M}{\sigma'}{ A[\overline{\alpha/x}]}  }  \ \ \  ]  
     \end{cases} 
 $
 \item
{ $\satisfies{M}{\mprepost{A}{C}{m}{x}{C}{A'} }     \ \, \ \ \  \mbox{iff}  \   \ 
    \begin{cases}
     \mbox{for all }   \Mtwo,  \sigma, \sigma', \alpha_0, \mbox{ and }  \overline{\alpha}    \\
 \ \ \ \  [ \ \ { \arising{\sigma}{M\madd \Mtwo }\   \  \wedge\ \ \GRelevant {\overline \alpha}  \sigma}\ \ \ \wedge  \\
  \ \ \ \  \ \ \  \wedge \ \  \sigma.\prg{cont}\txteq \mbox{body of $C::m$} \ \ \wedge \\
  \ \ \ \  \ \ \  \satisfiesA {M}   {\sigma}  {(\, \alpha_0:C \, \wedge\, \overline {\alpha:C} \, \wedge\,  {A[\alpha_0/\prg{this},\overline{\alpha/x}]}\, ) } \ \ \ \wedge   \\
 \ \ \ \   \ \ \ {\leadstoBoundedStar {M\madd \Mtwo}{\sigma}  {\sigma'} }    \\
 \ \ \ \   \ \ \ \ \  \ \Longrightarrow   \\
 \ \ \ \   \ \ \  {\satisfiesA{M}{\sigma'}{ A'[\alpha_0/\prg{this},\overline{\alpha/x}]}  }  \ \ \  ]  
    \end{cases} 
    $
 }
  \item
{$ \begin{array}{l}
 \satisfies{M}{\mprepostNTwo{A}{C}{m}{x}{C}{A'} {A''}} 
 \end{array}  \ \,  \mbox{iff}  \   \ 
    \begin{cases}
     \mbox{for all }   \Mtwo,  \sigma, \sigma',\sigma'', \alpha_0, \mbox{ and }  \overline{\alpha}    \\
   \ \ \ \  [ \ \ { \arising{\sigma}{M\madd \Mtwo }\   \  \wedge\ \ \GRelevant {\overline \alpha}  \sigma}\ \ \ \wedge  \\
  \ \ \ \  \ \ \  \wedge \ \  \sigma.\prg{cont}\txteq \mbox{body of $C::m$} \ \ \wedge  \\
  \ \ \ \  \ \ \  \satisfiesA {M}   {\sigma}  {(\, \alpha_0:C \, \wedge\, \overline {\alpha:C} \, \wedge\,  {A[\alpha_0/\prg{this},\overline{\alpha/x}]}\, ) } \ \ \ \wedge   \\
   \ \ \   \ \ \  \ {\leadstoBoundedStarFin {M\madd \Mtwo}{\sigma}  {\sigma'} }  \, \wedge  \, {\leadstoBoundedStar  {M\madd \Mtwo}{\sigma}  {\sigma''} } \\
 \ \ \ \   \ \ \ \ \  \ \Longrightarrow   \\
 \ \ \ \   \ \ \  {\satisfiesA{M}{\sigma'}{ A'[\alpha_0/\prg{this},\overline{\alpha/x}]}  }\ \wedge\\
 \ \ \ \   \ \ \      {\satisfiesA{M}{\sigma''}{ ( \external{\prg{this}}\ \rightarrow\ A''[\alpha_0/\prg{this},\overline{\alpha/x}})}\ \  \ ]
    \end{cases} 
    $
}
 \item
 $\satisfies{M}{S\, \wedge\, S'}$\ \ \ \ \ \  \ \ \ \ \ \ \ \ \ \ \ \ \ \ \ \ \ \ \  \ \ \  iff  \  \ \  \   $\satisfies{M}{S}\ \wedge \ \satisfies{M}{S'}$
\end{enumerate}
\end{definition}

\section{Proving Adherence to \SpecLang Specifications}
\label{sect:proofSystem}

We will now develop an inference system to prove that a module is well-formed $\vdash M$, \ie that a module satisfies its specification.
For this, we will   develop an extended  Hoare logic with quadruples of the  form  ${\hprovesN{M}  {A} {\ s\ }{A'} {A''} }$. 
These promise, as usual, that (*) execution of statement $s$ in a state satisfying $A$ will lead to a state satisfying $A'$.
In addition, they promise that (**) any intermediate external states reachable during execution of that statement will satisfy the invariant  $A'''$. 
These quadruples are developed in two phases:
% Our logic extends any underlying classical Hoare logic judgment of the form ${M \vdash_{ul}  \{A\} {\ s\ } \{A'\} }$.

In the first phase we develop a logic of triples ${\hproves{M}  {A} {\ s\ }{A'} }$, which have the expected meaning (*).
These tuples only apply to statements $s$ that  do not contain method calls -- namely method calls may contain calls to external methods, and therefore can only be described through quadruples.
Our triples extend an underlying Hoare logic  (${M \vdash_{ul}  \{A\} {\ s\ } \{A'\} }$) and  introduce new judgements  which talk about protection.

In the second phase we develop a logic of quadruples ${\hprovesN{M}  {A} {\ s\ }{A'} {A''}}$, which promise (*) and (**).  We incorporate all triples from the first phase, %${\hproves{M}  {A} {\ s\ }{A'} }$,      
introducing invariants, give the usual substructural rules, and deal with method calls using the module's specification.

\vspace{.1cm}
Before describing the two phases, we introduce some preliminaries. 
  

\subsection{Preliminaries: Specification Lookup and  Assertion Encapsulation}

We will be using the notation $\promises M S$ to indicate that $S$ is part of the module's specification

\begin{definition}
 $\promises M S\ \ \ \triangleq\ \ \ \SpecOf M \txteq  S \wedge \_  \ \vee\  \SpecOf M \txteq \_ \wedge S   \ \vee\ \SpecOf M \txteq \_ \wedge S $
 \end{definition} 

\subsubsection {Assertion Encapsulation}
\label{s:encaps-proof}

{
Proofs of adherence to {\SpecLang specifications  hinge on the expectation that some, 
specific, assertions cannot be invalidated unless some 
} internal (and thus known) computation took place. 
{We call such assertions   \emph{encapsulatied},}
}
formally $M\ \models  \encaps{A}$.
 

% \subsubsection{Semantics of Assertion Encapsulation}

{An assertion $A$  is  encapsulated by a module $M$  if in all possible states which arise from execution of module $M$ with any other  module $\Mtwo$, the validity of $A$}  {can only be changed via computations internal to that module}.
 

\begin{definition}[An assertion $A$ is \emph{encapsulated} by module $M$ ] $~$ \\
\label{def:encapsulation}
$ % \begin{equation}
    M\ \models \encaps{A}\ \   \mbox{iff}  \ \   
    \begin{cases}
     \forall \Mtwo, \sigma, \sigma', \overline{x} \mbox{ with } \overline{x}=Free(A):\\
   \ \ \ \  [\ \ \  \arising{M\madd\Mtwo}{\sigma}\ \wedge \  \satisfiesA{M}{\sigma}{(A \ \wedge\ \external {\prg{this})}}\  \wedge\  \leadstoOrig  {M\madd\Mtwo}  {\sigma}{\sigma'}\\
    \ \ \ \ \ \   \ \ \ \Longrightarrow\\
    \ \ \ \ \ \  \  {M},{\sigma'}\models{A[{\overline {{\interpret \sigma x}/x}}]} \ \  ]
    \end{cases}
 $% \end{equation}
  \end{definition}
 \footnote{\textbf{Aside} In the definition above we used the unbounded execution,  $\leadstoOrig  {M\madd\Mtwo}  {\sigma}{\sigma'}$;  however, the extension of the definition would have been the same if we had used bounded execution, $\leadstoBounded {M\madd\Mtwo}  {\sigma}{\sigma'}$ instead. TODO: write a lemma and also write a lemma for multiple steps}
 
\renewcommand{\ModD}{\ModParam{unprt}}
\newcommand{\ModE}{\ModParam{prtct}}
\begin{example}
Consider assertion $A_{bal}\  \triangleq\ a:\prg{Account}\wedge a.\prg{balance}=b$, then\ \ and modules \ModB and  \ModC  from \S~\ref{s:outline}. Then, \ \ \ $\ModB\ \models\ \encaps{ A_{bal} }$, \ \ \ and \ \ \ $\ModB\ \models\ \encaps{ A_{bal} }$.

Assume two further modules, $\ModD$ and $\ModE$, both of which  use ledgers which map  accounts to their balances, and export functions that update of this map. In  $\ModD$ the ledger is \emph{not} protected, while in $\ModE$ it \emph{is}. 
Then  \ \ $\ModD \ \not\models\encaps{ A_{bal}} $, \ \  and \ \ $\ModE  \models \encaps{ A_{bal}} $.
\noindent
Note that in $\ModParam{up\_ldg}$ and $\ModParam{pr\_ldg}$, the term \prg{a.balance} is a ghost field. 
\end{example}

In any module, relative protection % (a variable is protected from another one) 
is not encapsulated, but  absolute protection is.
 Note  that  encapsulation of an assertion does not imply encapsulation of its negation; 
 for example,  ${\inside{x}}$ is encapsulated (as per below), but $\neg  {\inside{x}}$ is not.


\begin{lemma}
For any module $M$, and variables $x$ and $y$:
\begin{enumerate} 
\item $M \models \encaps{\inside{x}}$
\item $M \not\models {\encaps{\neg\inside{x}}}$
\item $M \not\models {\encaps{\protectedFrom{x}{y}}}$
\item  $M \models  \encaps{A}$ and $M \models  \encaps{A'}$ implies 
$M \models  \encaps{A \wedge A'}$ and $M \models  \encaps{A \vee A'}$
\end{enumerate}
\end{lemma}

On the other hand,  $M \models A \rightarrow A'$ and $ M \models  \encaps{{A'}}$ does not  imply \ $M \models  \encaps{A}$, nor does
$M \models A \rightarrow A'$ and 
 $ M \models  \encaps{{A}}$ imply \ $M \models  \encaps{A'}$.

%\begin{proof} Sketches:
%
%(1) because if $y$ is protected, and since the fields are all private ... the only way .. 
%\\
%(2) Take a state where $\neg\inside{x}$ and that there is only external object that has access to $x$, and that this object becomes no longer accessible -- eg through field override. That means that we now have  $\inside{x}$.
%\\
%(3) it is always possible that another external object $z'$ has unprotected access to $x$; if $z'$ has access to $z$, then it can give it access to $x$ without invoking am internal method. (3) as a counterexample, 
%\\
%(4) Use that $M \models A \rightarrow A'$ implies $M \models \neg A'  \rightarrow \neg  A$. BUT CAREFUL, TODO! with the potential of non-termination on assertions!!
%\end{proof}
%%The key consequence of soundness is that -- SD dropped; it is   not a consequence of soundness!

\subsubsection{Proving  Assertion Encapsulation}

{In general},  code that does not contain 
calls to a {given} module is guaranteed not to invalidate any assertions encapsulated by that module.
 Assertion encapsulation has been used in proof systems to {address}   the  {frame} problem
 \cite{objInvars,encaps}. 

We  do not mandate how this property should be derived -- instead, we rely on a judgment 
$M\ \models  \encaps{A}$ provided by some external system. \footnoteSD{This is simpler than the oopsla-33 setting}
Thus, \SpecLang is parametric over the derivation of the encapsulation
     judgment; in fact, several ways to do that are possible \cite{TAME2003,ownEncaps,objInvars}. 
     % For example,  the appendices of  \cite{necessityFull} present a    rudimentary system that is sufficient to support our example  proof.  


%As we have already stated at the beginning of this section,
%encapsulation is a deep topic that is well studied in the literature, 
%and is not the focus of this paper. For now, we simply assume the existence 
%of a proof system for encapsulation as it is secondary to the central topic 
%of this paper. We need only assert that such an algorithmic proof system 
%must be sound (Definition \ref{lem:encap-soundness}).
%% \susan[I commented out what was there as I thought it was repetious]
%% {We are assuming the existence of a proof system for encapsulation and only need to assert that such an algorithmic proof system nust be sound.}
%% The construction of the algorithmic system is not central to our work,
%% because, as we shall see in later sections, our logic 
%% does not rely on the specifics of an encapsulation algorithm, only its soundness.

Like OOPSLA 22 Our logic does not {deal with, nor} rely on, the specifics of  how   encapsulation
{is derived}.
  % model, 
{Instead, it relies} on an encapsulation judgment and expects it to be sound:

\begin{definition}[Encapsulation Soundness]
\label{lem:encap-soundness}
A judgement of the form $\proves{M}{\encaps{A}}$  is\  \emph{sound}, \ if 
for all modules $M$, and assertions $A$:\\

$\strut \hspace{1.5cm} \proves{M}{\encaps{A}} \ \ \ \ $ implies $\ \ \ \ \satisfies{M}{\encaps{A}}$.
\end{definition}

\subsection{First Phase: Triples}

In  Fig. \ref{f:underly} we introduce our triples, of the form ${   \hproves{M}  {A} s {A'}}$. 
These promise, as expected, that any execution of $s$ in a state that satisfies $A$ leads to a state that satisfies $A'$.
These tuples only apply to statements that do not contain method calls.

 
\subsubsection{Extending an underlying Hoare Logic}


\begin{figure}[tht]
$
\begin{array}{c}
% \begin{array}{lcl}
\inferrule[\sc{extend}]
	{ M \vdash_{ul} \{ \ A\ \} {\ s\ }\{\ A'\ \} \ \ \  $s$ \ \mbox{contains no method call} }
	{\hproves{M}  {A} {\ s\ }{A'} }
%	& &
\\
\\
\inferrule[\sc{types-1}]
	{  $s$ \ \mbox{contains no method call} }
	{\hproves{M}  {x:C} {\ s\ }{x:C} }
% \end{array}
\\
	{\inferruleSD{\hspace{5.3cm} [\sc{prot-1}]}
	{ }
	{\hproves{M} 
						{\  \inside{x}  \wedge\ (\, \internal {y} \vee y'.f \neq x\, ) \ }
						{\ y =y'.f\ }
						{\  \inside{x}\ }
	}
	}
\\
        {\inferruleSD{\hspace{5.3cm} [\sc{prot-2}]}
	{ }
	{\hproves{M} 
						{\ \protectedFrom{x}{z} \ }
						{\ y =y'.f\ }
						{\ \protectedFrom{x}{z}\ }
	}
}
\\
       {\inferruleSD{\hspace{5.3cm} [\sc{prot-3}]}
	{ }
	{\hproves{M} 
						{\  \inside{x}\ \wedge \   \protectedFrom{x}{y'} \ }
						{\ y.f=y'\ }
						{\  \inside{x}\ }
	}
}	
\\
        {\inferruleSD{\hspace{5.3cm} [\sc{prot-4}]}
	{ }
	{\hproves{M} 
						{\ \protectedFrom{x}{z} \ \wedge \   \protectedFrom{x}{y'} }
						{\ y.f=y'\ }
						{\ \protectedFrom{x}{z} \ }
	}
}	 
\end{array}
 $
\caption{Hoare triples -- incorporating underlying logic, types and protection}
\label{f:underly}
\label{f:protection}
\end{figure}

We assume  an underlying Hoare logic with judgements of the form $M \vdash_{ul} \{ A \} s \{ A' \}$.
% On top this, we define our extension, with shape ${   \hproves{M}  {A} s {A'}}$. 
With rule {{\sc{extend}} in Fig. \ref{f:underly},  any assertion $M \vdash_{ul} \{ A \} s \{ A' \}$  whose statement does not contain a method call, and which 
can be proven in the underlying Hoare logic, to also be proven in our logic.

% The rules for this extension appear in Figures \ref{f:protection}, \ref{f:calls} and \ref{f:substructural}.

Moreover, rule {\sc{types-1}} promises that types of local variables do not change. The restriction to statements that do not contain method calls will be discussed with rule {\sc{types-2}} in the next section.

 

 \subsubsection{Reasoning about protection}

{
In Fig. \ref{f:protection} we also give rules about the preservation of protection.
In \textsc{Prot-1}, if $y$ is internal, then no new external object becomes locally reachable, and therefore $ \inside{x}$ is preserved;
similarly, if the rhs of the assignment ($y'.f$) is not an alias of $x$, then no new unprotected route to $x$ is introduced, and therefore  $ \inside{x}$ is preserved.
In \textsc{Prot-2}, the assignment $y=y'.f$ does not create a new route from $z$ to $x$, and therefore  $\protectedFrom{x}{z}$ is preserved.
In \textsc{Prot-3}, if $x$ is protected from $y'$, then the assignment $y.f=y'$ does not create a new unprotected route   to $x$, and therefore  $\inside {x}$ is preserved.
Similarly, in \textsc{Prot-4}, if $x$ is protected from $y'$, then the assignment $y.f=y'$ does not create a new unprotected route  to $x$, and therefore   $\protectedFrom{x}{z}$ is preserved.
}

Moreover, "protection" of an object can decrease is if we call an eternal method, and pass it an internal object as argument. This is then covered by the rule in Fig. \ref{f:external:calls}.

\begin{lemma}
\label{l:no:meth:calls}
If ${\hproves{M}  {A} {\ s\ }{A'} }$, then $s$ contains no method calls.
\end{lemma}

\subsection{Second Phase: Quadruples}

\subsubsection{Introducing Invariants, and substructural rules}
Fig. \ref{f:substructural} introduces quadruple rules of the from ${\hprovesN{M}  {A} {\ s\ }{A'} {A''} }$ which do not deal with method calls.

Rule {\sc{inv}} embeds  triples  ${\hproves{M}  {A} {\ s\ }{A'} }$  into quadruples ${\hprovesN{M}  {A} {\ s\ }{A'} {A''} }$; this is sound, because $s$ is guaranteed not to contain method calls (by lemma \ref{l:non:meth:calss})\footnote{The restriction to statements which do not contain method calls in rule {\sc{types-1}} was imposed so as to make this lemma valid}, and therefore its execution is guaranteed not to reach any external states.

 
The remaining rules in Fig. \ref{f:substructural} apply to \emph{any} statements -- even those containing method calls.
 Rule {\sc{types-2}} generalizes {\sc{types-1}} to any statement, provided that  there already exists a triple for that statement.
Rule {\sc{combine}} combines two quadruples for the same statement into one.
Rules  {\sc{sequ}} and  {\sc{consequ}} are the usual rules for statement sequences, and for consequence, adapted to quadruples.


\begin{figure}[htb]
$
\begin{array}{c}
% \begin{array}{lcl}
\inferrule[\sc{inv}]
	{\hproves{M}  {A} {\ s\ }{A'} }
	{\hprovesN{M}  {A} {\ s\ }{A'} {A''} }
\\
\\
% THIS RULE IS NOT SOUND -- we need to adapt it
%{
\inferrule[\sc{types-2}]
	{ \hprovesN{M}  {A} {\ s\ }  {A'} {A''}  }
	{\hprovesN{M}  {x:C \wedge A} {\ s\ }  {x:C\wedge A'} {A''}}
%}
% \end{array}
% \\
% \\
 \\
 \\
\inferrule[\sc{combine}]
	{  \hprovesN{M}  {A_1} {\ s\ } {A_2}  {A} \hspace{1.4cm}  \hprovesN{M}  {A_3} {\ s\ } {A_4} {A}}
	{ \hprovesN{M}  {A_1 \wedge A_3 }{\ s\ } {A_2 \wedge A_4} {A} }
\\
\\
\inferrule[\sc{sequ}]
	{  \hprovesN{M}  {A_1} {\ s_1\ } {A_2}  {A}  \hspace{1.4cm} \hproves{M}  {A_3} {\ s_2\ } {A_4} {A}}
	{   \hprovesN{M}  {A_1 \wedge A_3 }{\ s\ } {A_2 \wedge A_4} {A} }
\\ \\
\inferrule[\sc{consequ}]
	{
%	\begin{array}{l}
	 { \hprovesN  {M}  {A_2} {\, s\, } { A_3 } {A_4}  }
	 \\
	 M \vdash A_1 \rightarrow A_2 
	 \hspace{1.4cm} 
	{ M \vdash A_3 \rightarrow A_5  }
	 \hspace{1.4cm}   
	{  M \vdash A_4 \rightarrow A_6 }
%	\end{array}
	}
	{   \hprovesN{M}  {A_2 }{\ s\ } {A_5 } {A_6} }
  \end{array}
 $
\caption{Hoare Quadruples - incorporating Hoare triples and subsctructural rules }
\label{f:substructural}
\end{figure}
 
  
\subsubsection{Reasoning about internal calls}
is described in Fig. \ref{f:internal:calls}. The first two rules are about calling private methods, and the next four rules are about public methods.
% We now consider the logic for internal calls.
% Rule {\sc{IntCall}} uses the 
%We use the callee's specification, ${\mprepostN{A_1}{C}{m}{y}{C}{A_2}{A_3}}$ (

% In all six cases, ; 
For private methods, we start by  looking up (one of) the method's specifications: $ \promises  M {\mprepostN{A_1}{\prg{private} C}{m}{y}{C}{A_2} {A_3}}  $  indicates that if $m$ starts in a state that satisfies $A_1$, then  its execution will lead to a state that sariafies $A_2$, and
that during its execution, $A_3$  any reachable scoped external state will satisfy $A_3$.
This is reflected  in rule {\sc{Call\_Priv\_A}};  notice that in the post condition ($A_2$) we replace the $result$ by the left hand side of the assignment ($u$).
In rule  {\sc{Call\_Priv\_B}} we require a weaker pre-condition ($  \PushAS {y}{A_1}$  rather than $A_1$) and also ensure a weaker post-condition ($ PushAS {y}{A_2}$, rather than $A_2$).
Remember that $  \PushAS {y}{A_1}$ guarantees that $A_1$ held before popping the callee's frame (ie $  \PushAS {y}{A_1}$). and thartxxxx

%Looking more closely, we notice  that {\sc{IntCall}} does not require the method's precondition ($A_1$) to hold before the call, but instead requires that it will hold after the method call has been pushed on the stack, (ie $  \PushAS {y}{A_1}$).

%Similarly, after the call it does not guarantee the method's post condition ($A_2$), but instead guarantees that it held before popping the calllee's frame (ie $  \PushAS {y}{A_1}$).
%Remember that the assertion $\PushAS y A$ is \emph{hypothetical}: if a state satisfies $\PushAS y A$, then after pushing
%onto that state a frame which contained the values  of $\overline y$, assertion $A$ will hold, and conversely,   if a state satisfies  $A$ with a top frame containing the  values of $\overline y, \overline z$, then  after popping that frame, the state  satisfies assertion  $\PushAS y A$ 
(cf.  lemma \ref{lemma:push:ass:state}).

%
% We use  

expalin explainexplain explain expalin explainexplain explain expalin explainexplain explain expalin explainexplain explain
expalin explainexplain explain expalin explainexplain explain expalin explainexplain explain expalin explainexplain explain
expalin explainexplain explain expalin explainexplain explain

expalin explainexplain explain expalin explainexplain explain expalin explainexplain explain expalin explainexplain explain
expalin explainexplain explain expalin explainexplain explain expalin explainexplain explain expalin explainexplain explain
expalin explainexplain explain expalin explainexplain explain

% {\small{ 
\begin{figure}[thb]
$\begin{array}{c}
%\begin{array}{ll}
\inferruleSD{\hspace{4.7cm} [\sc{Call\_Priv\_A}]}
 	{
	   	
	   	\begin{array}{l}
		\promises  M  { \mprepostN {A_1} {\prg{private}\ C} {m} {y} {C} {A_2} {A_3} }
%		\\
%		A_3 \txteq \overline {x:C} \wedge A \hspace{1.4cm}
%		\promises {M} {\TwoStatesN  {\overline {x:C}} {A}}
          	\end{array}
          		}
	{  \hprovesN {M} 
						{ \  y_0:C  \ \wedge\ \overline {y:C}\ \wedge \ \red{A_1}\   \  }
						{ \ u:=y_0.m(y_1,.. y_n)\    }
						{ \ \red{ A_2[ u/result] }  \ }
						{\red {A_3}}	
 }
 \\
 \\
 \inferruleSD{\hspace{4.7cm} [\sc{Call\_Priv\_B}]}
	{
	   	\begin{array}{l}
		\promises  M {\mprepostN{A_1}{\prg{private}\ C}{m}{y}{C}{A_2} {A_3}}  
%		\\
%		A_3 \txteq \overline {x:C} \wedge A \hspace{1.4cm}
%		\promises {M} {\TwoStatesN  {\overline {x:C}} {A}}
          	\end{array}
		}
	{  \hprovesN {M} 
						{ \  y_0:C  \ \wedge\ \overline {y:C}\ \wedge \ \red{\PushAS {y}{A_1}}  \ } %\wedge A_3)}}}
						{ \ u:=y_0.m(y_1,.. y_n)\    }
						{ \ \red{ \PushAS {y}{(A_2[ u/result]}}  \ }
						{ \red {A_3} }	
}

 \\
 \\
 \inferruleSD{\hspace{4.7cm} [\sc{Call\_Pub\_A}]}
	{
	   	\begin{array}{l}
		\promises  M {\mprepost{A_1}{\prg{public}\ C}{m}{y}{C}{A_2} } \\
%		 \promises M   {\TwoStatesN {\overline {x:C}} {A}} 
%		\\
%		A_3 \txteq \overline {x:C} \wedge A \hspace{1.4cm}
%		\promises {M} {\TwoStatesN  {\overline {x:C}} {A}}
          	\end{array}
		}
	{  \hprovesN {M} 
						{ \  y_0:C  \, \wedge\ ,\overline {y:C}\, \wedge \, \red{\PushAS {y}{A_1} } \ }
						%\wedge A_3)}}}
						{ \ u:=y_0.m(y_1,.. y_n)\    }
						{ \ \red{ \PushAS {y}{(A_2[ u/result]}}  \ }
						{ \red {true} }	
}
 \\
 \\
 \inferruleSD{\hspace{4.7cm} [\sc{Call\_Pub\_A}]}
	{
	   	\begin{array}{l}
		\promises  M {\mprepost{A_1}{\prg{public}\ C}{m}{y}{C}{A_2} } \\
%		 \promises M   {\TwoStatesN {\overline {x:C}} {A}} 
%		\\
%		A_3 \txteq \overline {x:C} \wedge A \hspace{1.4cm}
%		\promises {M} {\TwoStatesN  {\overline {x:C}} {A}}
          	\end{array}
		}
	{  \hprovesN {M} 
						{ \  y_0:C   \wedge \overline {y:C}  \wedge  A_1    \ }
						%\wedge A_3)}}}
						{ \ u:=y_0.m(y_1,.. y_n)\    }
						{ \ \red{ A_2[ u/result]} \ }
						{ \red {true} }
}
\\
\\
 \inferruleSD{\hspace{4.7cm} [\sc{Call\_Pub\_C}]}
	{
	   	\begin{array}{l}
		\promises  M {\mprepost{A_1}{\prg{public}\ C}{m}{y}{C}{A_2} } \\
		 \promises M   {\TwoStatesN {\overline {x:D}} {A}} 
          	\end{array}
		}
	{  \hprovesN {M} 
						{ \  y_0:C  \, \wedge\ ,\overline {y:C}\, \wedge \,  \overline{x:D}  \wedge\, \red{\PushAS {y}{A}}\ }
						%\wedge A_3)}}}
						{ \ u:=y_0.m(y_1,.. y_n)\    }
						{    \red{\PushAS {y}{A}} \ }
						{ \red {\  \overline{x:D} \, \wedge \,  A\ }} 	
}
 \\
 \\
  \inferruleSD{\hspace{4.7cm} [\sc{Call\_Pub\_D}]}
	{
	   	\begin{array}{l}
		\promises  M {\mprepost{A_1}{\prg{public}\ C}{m}{y}{C}{A_2} } \\
		 \promises M   {\TwoStatesN {\overline {x:D}} {A}} 
          	\end{array}
		}
	{  \hprovesN {M} 
						{ \  y_0:C  \, \wedge\ ,\overline {y:C}\, \wedge \,  \overline{x:D}  \wedge\, \red{A}\ }
						%\wedge A_3)}}}
						{ \ u:=y_0.m(y_1,.. y_n)\   }
						{   \ \red{A} \ }
						{\  \red { \overline{x:D} \, \wedge \,  A}\  }	
}
 \\
\\
 \\ 
  \inferruleSD{\hspace{4.7cm} [\sc{Ext\_Cal\_A}]}
	{ 
   	 \promises M   {\TwoStatesN {\overline {x:D}} {A}} 
        }
	{   \hprovesN{M} 
						{ \    \sdN{ \external{y_0}} \,     \wedge \,  \overline{x:D}\  \wedge\ \sdN{\PushAS {y}{A}}\ }  
						{ \ u:=y_0.m(y_1,.. y_n)\    }
						{ \   \sdN{\PushAS {y}{A}}  \ }
						{ \overline{x:D} \wedge A }	
}
\\
\\
 \inferruleSD{\hspace{4.7cm} [\sc{Ext\_Call\_B}]}
	{ 
   	 \promises M   {\TwoStatesN {\overline {x:D}} {A}} 
        }
	{   \hprovesN{M} 
						{ \    \sdN{ \external{y_0}} \,     \wedge \,  \overline{x:D}\  \wedge\ \sdN{A} \ } 
						{ \ u:=y_0.m(y_1,.. y_n)\    }
						{ \   \sdN{A}  \ }
						{ \overline{x:D} \wedge A }	
}
\end{array}
$
\caption{Hoare Quadruples for Internal and External Calls}
\label{f:internal:calls}
\end{figure}
% }}

\footnoteSD{\red{TO DISCUSS: What is the underlying Hoare logic also has rules for internal calls? SD thinks that this is Ok, because we require the logic to be sound. This could mean that the underluying Hoare logic does not support arguments about protection.} 
}


% \subsubsection{External Calls}



\subsubsection{Reasoning about external calls} 
TODO: look at explanations

expalin explainexplain explain expalin explainexplain explain expalin explainexplain explain expalin explainexplain explain
expalin explainexplain explain expalin explainexplain explain expalin explainexplain explain expalin explainexplain explain
expalin explainexplain explain expalin explainexplain explain

expalin explainexplain explain expalin explainexplain explain expalin explainexplain explain expalin explainexplain explain
expalin explainexplain explain expalin explainexplain explain expalin explainexplain explain expalin explainexplain explain
expalin explainexplain explain expalin explainexplain explain

expalin explainexplain explain expalin explainexplain explain expalin explainexplain explain expalin explainexplain explain
expalin explainexplain explain expalin explainexplain explain expalin explainexplain explain expalin explainexplain explain
expalin explainexplain explain expalin explainexplain explain

expalin explainexplain explain expalin explainexplain explain expalin explainexplain explain expalin explainexplain explain
expalin explainexplain explain expalin explainexplain explain expalin explainexplain explain expalin explainexplain explain
expalin explainexplain explain expalin explainexplain explain

expalin explainexplain explain expalin explainexplain explain expalin explainexplain explain expalin explainexplain explain
expalin explainexplain explain expalin explainexplain explain expalin explainexplain explain expalin explainexplain explain
expalin explainexplain explain expalin explainexplain explain expalin explainexplain explain expalin explainexplain explain expalin explainexplain explain expalin explainexplain explain
expalin explainexplain explain expalin explainexplain explain expalin explainexplain explain expalin explainexplain explain
expalin explainexplain explain expalin explainexplain explain

expalin explainexplain explain expalin explainexplain explain expalin explainexplain explain expalin explainexplain explain
expalin explainexplain explain expalin explainexplain explain expalin explainexplain explain expalin explainexplain explain
expalin explainexplain explain expalin explainexplain explain
expalin explainexplain explain expalin explainexplain explain expalin explainexplain explain expalin explainexplain explain
expalin explainexplain explain expalin explainexplain explain expalin explainexplain explain expalin explainexplain explain
expalin explainexplain explain expalin explainexplain explain

expalin explainexplain explain expalin explainexplain explain expalin explainexplain explain expalin explainexplain explain
expalin explainexplain explain expalin explainexplain explain expalin explainexplain explain expalin explainexplain explain
expalin explainexplain explain expalin explainexplain explain

expalin explainexplain explain expalin explainexplain explain expalin explainexplain explain expalin explainexplain explain
expalin explainexplain explain expalin explainexplain explain expalin explainexplain explain expalin explainexplain explain
expalin explainexplain explain expalin explainexplain explain


%now move to the discussion of  external calls. For all external calls we require that the module's invariants hold before the external call.
%This is enforced through the requirement ${\external{y_0}}  \rightarrow \  \PushAS {y}{\extract{M}}$ in rules
%{\sc{CallAndAlias}} and {\sc{CallNonAlias}} in  Fig. \ref{f:internal:alias:calls}, and the precondition part  
% $\PushAS {y}{\extract{M}}$ in all rules in Fig.  \ref{f:external:calls}. }
%
%
%
% 
% {In rule {\sc{ExtCall}} we   ensure that the module's invariants are preserved by the external call. } {Notice that while  {\sc{CallAndAlias}}, {\sc{CallNonAlias}}, {\sc{ExtCall\_WithSpec\_Weak}} and {\sc{ExtCall\_WithSpec\_Weak}}    include 
% $\PushAS y {A_1}$ in their precondition,  they do not include it in the postcondition. A variation where was  $\PushAS y {A_1}$ in the postcondition is admissible, as it would be the outcome of the combination of these rules with  {\sc{ExtCall}} through the application of {\sc{Consequ}}.}
% 
%{The precondition of {\sc{ExtCall\_WithSpec\_Weak}} is stronger than that of {\sc{ExtCall}}. It makes use of one of the  module's  promises: Here $  {\TwoStatesQ {\overline {x:C}} {A_1}{A_2}}$, i.e. that  $\overline {x:C} \ \wedge\ A_1$ can only lead to $A_2$. 
%% Moreover, the call's precondition requires  $\overline {x:C} \wedge \PushAS y {A_1}$. 
% Therefore, by lemma \ref{lemma:push:ass:state}.\ref{lemma:push:ass:state:one}, right after pushing a frame with $\overline y$, ie right after  entering the external call, $\overline {x:C}$ and  $A_1$ hold. `Therefore, by the semantics of $  {\TwoStatesQ {\overline {x:C}} {A_1}{A_2}}$, at the end of the external call,  $\overline {x:C} \wedge A_2$ will hold. This, together with  lemma \ref{lemma:push:ass:state}.\ref{lemma:push:ass:state:two} gives that after popping the frame, ie after  exiting the external call, $\PushAS y {A_1}$ will hold.}
%
%
%{The precondition of {\sc{ExtCall\_WithSpec\_Strong}} is stronger than that of {\sc{ExtCall\_WithSpec\_Weak}}: it also requires $A_1$. Therefore, the module's promise, $  {\TwoStatesQ {\overline {x:C}} {A_1}{A_2}}$, guarantees that after exiting the external call, $A_2$ will hold.}
 





%%\small{
%\begin{figure}[hbt]
%$\begin{array}{c}
%\end{array}
%$
%\caption{Logic for External Calls}
%\label{f:external:calls}
%\end{figure}
%% }

  \subsubsection{Aliasing and Calls}
TODO: revisit that section

%{Rules  {\sc{CallAndAlias}}  and  {\sc{CallNonAlias}} say that calls preserve aliasing, resp. non-aliasing, between variables, ie they preserve $x=x$ resp. $x\neq x$. These two rules apply to internal as well as external calls. When the callee's receiver is external, they make the extra requirement that  $\PushAS {y}{\extract{M}}$ -- we  will discuss this requirement together with the discussion of Fig. \ref{f:external:calls}.     Note that $x=x'$ expresses that $x$ and $x'$ are aliases, while  $u\txteq x$ expresses that $u$ and $x$ are textually the same --
%the latter is stronger, i.e.   $x\txteq u$ implies $x=u$. 
%% It is possible that variables are aliases, without being textually the same, i.e. it is possible to have $x=x'$ while $u\not\equiv x'$. 
% As $...\equiv ...$ is a textual assertion, and thus  state-independent,  it is a side-condition of the rules and is  not part of the Hoare triple's precondition.
%}
%
%
%\small{
%\begin{figure}[hbt]
%$\begin{array}{c}
%\inferruleSD{\hspace{4.7cm}  [{\sc{CallAndAlias}}}
%	{ 
%	{   x \txtneq u\txtneq x'  }
%	}
%	{   \hproves{M}  { \ x=x'\   \wedge \ ({\external{y_0}}  \rightarrow \  \PushAS {y}{\extract{M}})\  }	{ \ u:=y_0.m(y_1,.. y_n)\  } { \  x=x'\ }	 }
%
%\\ \\ 
%\inferruleSD{\hspace{4.7cm} [{\sc{CallNonAlias}}]}	
%{ 
%		{ \ x \txtneq u \txtneq x'\   }
%	}
%	{   \hproves{M}   { \ x\neq x'\   \wedge \ ({\external{y_0}}  \rightarrow \  \PushAS {y}{\extract{M}})\  } { \ u:=y_0.m(y_1,.. y_n))\  } { \  x\neq x'\ }	 }
%\\
%\\
%\end{array}
%$
%\caption{Logic for Aliasing around Calls }
%\label{f:internal:alias:calls}
%\end{figure}
%}



\subsection{Proving \SpecLang Specifications}
\label{sect:wf}

We now define the judgment $\vdash M$, which says that module $M$ has been proven to satisfy its specification.

expalin explainexplain explain expalin explainexplain explain expalin explainexplain explain expalin explainexplain explain
expalin explainexplain explain expalin explainexplain explain expalin explainexplain explain expalin explainexplain explain
expalin explainexplain explain expalin explainexplain explain

expalin explainexplain explain expalin explainexplain explain expalin explainexplain explain expalin explainexplain explain
expalin explainexplain explain expalin explainexplain explain expalin explainexplain explain expalin explainexplain explain
expalin explainexplain explain expalin explainexplain explain

expalin explainexplain explain expalin explainexplain explain expalin explainexplain explain expalin explainexplain explain
expalin explainexplain explain expalin explainexplain explain expalin explainexplain explain expalin explainexplain explain
expalin explainexplain explain expalin explainexplain explain

expalin explainexplain explain expalin explainexplain explain expalin explainexplain explain expalin explainexplain explain
expalin explainexplain explain expalin explainexplain explain expalin explainexplain explain expalin explainexplain explain
expalin explainexplain explain expalin explainexplain explain

expalin explainexplain explain expalin explainexplain explain expalin explainexplain explain expalin explainexplain explain
expalin explainexplain explain expalin explainexplain explain expalin explainexplain explain expalin explainexplain explain
expalin explainexplain explain expalin explainexplain explain

expalin explainexplain explain expalin explainexplain explain expalin explainexplain explain expalin explainexplain explain
expalin explainexplain explain expalin explainexplain explain expalin explainexplain explain expalin explainexplain explain
expalin explainexplain explain expalin explainexplain explain

expalin explainexplain explain expalin explainexplain explain expalin explainexplain explain expalin explainexplain explain
expalin explainexplain explain expalin explainexplain explain expalin explainexplain explain expalin explainexplain explain
expalin explainexplain explain expalin explainexplain explain


$
\begin{array}{l}
\inferruleSD 
{~ \strut \hspace{6.5cm} {\sc{public\_method}}}
{  
 \prg{mBody}(m,D,M)=\prg{public} \ \overline{x:C}\{\  stmt \ \}      
    \\
  {\hprovesN{M} { \ \prg{this}:\prg{D}\, \wedge\, \overline{x:C}\, \wedge\, A_1\  } {\ stmt\ } {\ A_2 \ }   {true} } 
}
{
M \vdash {\mprepost {A_1}{D}{m}{x}{C}{A_2} }
}
\\ \\
\inferruleSD 
{~ \strut \hspace{6.5cm} {\sc{private\_method}}}
{  
  \prg{mBody}(m,D,M)=\prg{private} \ \overline{x:C}\{\  stmt \ \}      
    \\
  {\hprovesN{M} { \ \prg{this}:\prg{D}\, \wedge\, \overline{x:C}\, \wedge\, A_1\  } {\ stmt\ } {\ A_2 \ }   {\ A_3\ } } 
}
{
M \vdash {\mprepostN {A_1}{D}{m}{x}{C}{A_2} {A_3} }
}
\\
\\
\inferruleSD 
{~ \strut \hspace{6.5cm} {\sc{invariants}}}
{  
\forall  D\in M,  m:\ \ \  \ \  \prg{mBody}(m,D,M)=\prg{public} \ (\overline{y:D})\{\  stmt \ \}      \ \ \Longrightarrow  
    \\
    \begin{array}{l}
   ~ \strut \hspace{1.7cm}[ \ \ \ \  \ \ A_{inv} \triangleq \overline{x:C}\, \wedge\, A
  \\
  ~ \strut \hspace{1.7cm} \ \ \ \ \  \ \ \  {\hprovesN{M} { \ \prg{this}:\prg{D}\, \wedge\, \overline{y:D}\, \wedge\,  A_{inv} } {\ stmt\ } {\ A_{inv}\ } {\ A_{inv} \ }  } \ \ \ \ \  \ \ ]
\end{array}
}
{
M \vdash \TwoStatesN{ \overline{x:C}} {A}
}
\\
\\
\\
\begin{array}{lcl}
\inferruleSD 
{~ \strut   {\sc{Comb\_Spec}}}
{  
M \vdash S_1 \hspace{1cm}  M \vdash S_2
}
{
M \vdash S_1 \wedge S_2
}
& &
\inferruleSD 
{~ \strut  {\sc{WellFrm\_Mod}}}
{  M \vdash \SpecOf {M}
}
{
\vdash M  
}
\end{array}
\end{array}
$
 
\footnoteSD{\red{QUESTIONS: }

\begin{enumerate}
\item 
Can we express what Fig \ref{f:calls} does, but more streamlined? -- SD has no ideas here  :-(
\item
In the rule for $\models M$,   I wonder whether we should be using ${\extract{M}}$ rather than $\PushAS {y} {\extract{M}}$. -- SD thinks this is only a matter of more thinking
\item
Do we want to forbid private methods to make external calls -- see Sect \ref{s:privateMs}? If we do that, then we need to extend the oper semantics and the inference rules in trivial, and boring ways...
\item
In  Fig \ref{f:calls} I wonder whether there are also stronger versions, where we require ${\extract{M}}$ rather than $\PushAS {y} {\extract{M}}$, pretty much as in the difference between {\sc{ExtCall\_WithSpec\_Weak}}  and the other rule, {\sc{ExtCall\_WithSpec\_Strong}}. But this would blow up the rules even more! 
\item
 Does the consequence rule require that the assertions are encapsulated? And if an assertion is encapsulated, is its consequence also encapsulated? TODO - EASY
\item
All rules require that the variables in the quantifiers do not appear in the meathod bodies, and are disjoint from the parameters.
TODO explain -EASY. 
\item  We need to add some dynamic type checking to the language, ie the public method call crashes if the actual params do not fit the formal types. Thisi is only for convenience; we cloud type them all as \prg{Object}, and crash by hand. TODO - EASY
\item
Do we need "modifies" or "does not modify" clauses too? Or can we assume that the underlying HL does this implicitly? For example, does\\
$\strut \ \ \  \hproves{M}  {a:Account \wedge p:Password \wedge a.passwd=p} {stmts}  {a.passwd=p}$\\ (with $a$ and $p$  free in $stmts$),  implicitly promise that no account's password has been modified? Shall we require the underlying logic to support such judgments -- say that if it has modifies clauses, or SL, it could do that?\\
Such an approach might solve the "late binding" issues mentioned below.
\end{enumerate}
 }





 