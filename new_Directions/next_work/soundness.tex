

We now outline some interesting aspects when proving soundness of the logic from \S \ref{sect:proofSystem}.

 
\paragraph{\Scoped Satisfaction} 
\label{s:scoped:valid}


Remember that an assertion which held at the end of a method execution, need not hold upon return from it -- \cf Ex. \ref{ex:pop:does:not:preserve}, and   \ref{ex:motivate:scopes}. To address this, we introduce \emph{scoped satisfaction}: % of assertions: 
 $ \satDAssertFrom M  \sigma k   A$   says that $\sigma$ satisfies $A$ from $k$ onwards, if it satisfies it in $k$-th frame,  and all the frames above it. 
% That is   $ \satDAssertFrom M  \sigma k   A\ \triengleq\   
\ie $\forall j. [\  k\leq j \leq \DepthSt \sigma \ \Rightarrow \ M, \RestictTo \sigma j \models A \ ]$.
%And define  and $k\leq n$ and $\forall j. [\  k\leq j \leq n \ \Rightarrow \ M, ((\phi_1\cdot ... \phi_j), \chi) \models A'\ ]$ where $A'$ is $A$ whose free variables have been substituted according to $\phi_n$ -- \cf Def. \ref{def:restrict}.
We also introduce \emph{scoped quadruples},  $\satisfiesD {M} {\quadruple  {A} }   {\sigma}   {A'} {A''}$, which promise for all $k\leq \DepthSt \sigma$,  if $\sigma$ satisfies $A$ from $k$ onwards, and executes its continuation to termination, then the final state will satisfy $A'$ from $k$ onwards, and that  all intermediate external states will satisfy $A''$ from $k$ onwards - \cf Def \ref{def:restrict}.
More in \A\ \ref{s:scoped:valid}.
% We define   $\satisfiesD {M} {\quadruple  {A} }   {stmt}   {A'} {A''} $ and  $\satisfiesD {M} {S}$ accordingly.
\Scoped satisfaction is stronger than shallow:   
 
\begin{lemma}[\Scoped   vs Shallow Satisfaction]
For all $M$, $A$, $A'$, $A''$, $\sigma$, $stmt$:  
\begin{itemize}
\item
 $\satisfiesD {M} {\quadruple  {A} }   {\sigma}   {A'} {A''}   \ \ \ \Longrightarrow \ \ \   \satisfies {M} {\quadruple  {A} }   {\sigma}   {A'} {A''}$

%\item
% $\satisfiesD {M} {\quadruple  {A} }   {stmt}   {A'} {A''}   \ \ \ \Longrightarrow \ \ \   \satisfies {M} {\quadruple  {A} }   {stmt}   {A'} {A''}$::q

%\item 
%$\satisfiesD {M} {S}  \ \ \ \Longrightarrow \ \ \ \satisfies {M} {S}$
\end{itemize}
\end{lemma}

  %%%%%%%%

\paragraph{Soundness of the Hoare Triples Logic}
\label{sect:prove:triples:sound}

We require the assertion logic,  $M\vdash A$, and  the    underlying Hoare logic,  $M\ \vdash_{ul}\  \triple A {stmt} {A'}$,   to be be  sound. %, \cf Axiom \ref{ax:ul:sound}.
We   prove  properties of protection, and 
 soundness of the inference system for triples $M \vdash  \triple A {stmt} {A'} $ -- \cf \A\ \ref{s:sound:app:triples}.

 

\begin{Theorem}
\label{l:triples:sound}
For module  $M$   such that  $\vdash M$, and for any assertions $A$,  $A'$, $A''$ and statement  $stmt$:
\begin{center}
$M\ \vdash\  \triple A {stmt} {A'}  \ \ \ \  \Longrightarrow  \ \ \ \ \satisfiesD {M} {\quadruple {A} {stmt} {A'} {A''}}$
\end{center}
\end{Theorem}
 

\paragraph{Summarised Execution}
\label{s:summaized}

%Soundness of   rule {\sc{Call\_Ext}} raises the challenge that 
Execution of an external call may consist of any number of external
transitions, interleaved with calls to public internal methods, which in
turn may make any number of further internal calls (public or private),  % chopped "whether" so as to be more succinct
and these, again may call external methods.
For the   proof of soundness,  internal and external transitions use different arguments.
 For  external transitions we consider small steps  and  argue in terms of  preservationon of  encapsulated properties,
while for internal calls, we use large steps, and appeal to the method's specification.
Therefore, we define  \emph{sumarized} executions, where  internal calls are collapsed into one. large step, \eg below:
  

\label{sect:termExecs}


% \vspace{.1cm}

\begin{tabular}{lll}
\begin{minipage}{.41\textwidth}
%Original execution %, \textbf{three}  internal \& {four} external calls. 
% 
 \resizebox{4.1cm}{!}{
\includegraphics[width=\linewidth]{diagrams/summaryA.png}
} 
 \end{minipage}
&  \begin{minipage}{.14\textwidth}
summarized\\
$\strut \ \ \ \ \ $ to 
\end{minipage}
 &
\begin{minipage}{.41\textwidth}
~ \\
~ \\
\resizebox{4.1cm}{!}
{
\includegraphics[width=\linewidth]{diagrams/summaryB.png}
} \end{minipage}
\end{tabular}
 
 


\vspace{.1cm}

% We now revisit external executions interleaved with public method calls:   
%In the appendix, we prove 
Lemma \ref{lemma:external_breakdown:term} % from  the  Appendix   
says that any terminating execution 
 starting in an external state  consists of a  sequence of  external states interleaved with terminating executions
  of public methods.
Lemma  \ref{lemma:external_exec_preserves_more} says that such an execution preserves an encapsulated assertion $A$  
provided that all these finalising internal executions  %(the public methods called at $\sigma_1$, ... $\sigma_n$) 
also preserve $A$.
% It is used to prove  soundness of the rule {\sc{ExtCall}}\footnoteSD{perhaps also {\sc{ExtCall\_WithSpec\_Strong}}}
% 
%In the appendix we prove lemmas \ref{lemma:external_breakdown:term} and \ref{lemma:external_exec_preserves_more} 
%which guarantee  that if $\sigma$ is external and $ \leadstoBoundedStarFin {\Mtwo}  {\sigma}  {\sigma'}$, 
%then there exist $\sigma_1$, ... $\sigma_n$, such that
% $\WithExtPub {\Mtwo\cdot M} {\sigma\bd}  {\sigma}  {\sigma'} {\sigma_1...\sigma_n}$.
% Conversely, if $A$ is encapsulated, 
% and $\WithExtPub {\Mtwo\cdot M} {\sigma\bd}  {\sigma}  {\sigma'} {\sigma_1...\sigma_n}$, and 
% the calls at  $\sigma_1$, ... $\sigma_n$ preserve $A$,  then $A$ is preserved from $\sigma$ to $\sigma'$.
  


  %%%%%%%%%%%%%%%%%%%%%%%%%%%%%%%%%%%%%%%%%%%%%%%%%%%

\paragraph{ Soundness of the Hoare Quadruples Logic}

Proving soundness of our quadruples in  some cases  requires  induction on the execution while in other cases  requires induction on the derivation of the quadruples.  We address this   through  a well-founded ordering that combines both, \cf 
\label{sect:prove:wellfounded}
\label{sect:prove:sound:quadruples}
  Def.  \ref{def:measure}  and  lemma \ref{lemma:normal:two}. 
  Finally, in \ref{s:app:proof:sketch;quadruples}, we prove soundness:
 

\begin{theorem}
\label{t:quadruple:sound}
\label{thm:soundness}
For module  $M$,   assertions $A$, $A'$, $A''$,   state  $\sigma$, and specification $S$:

\begin{enumerate}[(A)]
\item
 $:\strut \   \vdash M  \ \ \ \wedge \ \ \  M\ \vdash\  \quadruple {A} {stmt} {A'} {A''}  \ \ \ \ \ \ \ \Longrightarrow \ \ \ \ \ \  \ \ \  M\ \modelsD\  \quadruple {A} {stmt} {A'} {A''}$
 \item
  $:\strut \  \  \proves{M}{S}\ \ \ \ \ \ \Longrightarrow\ \ \ \ \ \  \ \ \ {M} \modelsD {S}$
 
\end{enumerate}

\end{theorem}

%The proofs make use of summarized executions, well-founded orderings, and various assertion preservation properties.  %
%\begin{theorem}[Soundness]
%\label{thm:soundness}
%%Assume an \SpecO proof system, $\proves{M}{A}$, 
%%an encapsulation inference system, $\proves{M}{\encaps{A}}$,
%%% Axiom xx, and 
%% and  that on top of these systems we built
%% the \SpecLang logic according to zzzz,  then, for    all modules $M$, and all \SpecLang specifications  $S$:
% For any module $M$, and specification $S$:
% 
% $$\proves{M}{S}\ \ \ \ \ \ \ \mbox{implies}\ \ \ \ \ \  \ \ \ {M} \modelsD {S}$$
%\end{theorem}
%
%We now prove soundness of the inference system $M \vdash  \quadruple A {stmt} {A'} {A''}$, using summarized executions from the earlier section, and the ordering $_ \ll \_$:   Proof outlines for these theorems can be found in \A \ref{s:app:proof:sketch;quadruples} and \ref{s:app:proof:sketch;overall}. 

%
%Theorem. \ref{thm:soundness} demonstrates 
% that the   \SpecLang logic is sound with respect to the semantics of \SpecLang specifications.
% The \SpecLang logic parametric wrt to the algorithms for proving validity of assertions
% $\proves{M}{A}$, and 
% assertion encapsulation ($\proves{M}{\encaps{A}}$), and is sound
% provided that these two proof systems are sound.
