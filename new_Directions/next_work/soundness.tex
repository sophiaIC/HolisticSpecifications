


\section{Soundness of our proof system}

In this section we will prove that the proof system presented in section \ref{sect:proofSystem} is sound. We give the usual meaning to the Hoare triples (Sect \ref{sect:HLmeans}), we give several properties of terminating executions (Sect \ref{sect:termExecs}), and finally, we define a well-founded ordering which allows if to prove soundness (Sect \ref{sect:prove:sound}).

We start by requiring  that the supporting proof system for assertions, and for encapsulation are sound.
\begin{axiom}
We assume a judgment of the form $M \vdash A$,  a judgment of the form $M \vdash \encaps{A}$, which have the property that
\begin{center}
$M \vdash A $ \ \ \ \ implies \ \ \ \ $M \vDash A$.\\
 $M \vdash \encaps{A} $ \ \ \ \ implies \ \ \ \ $M \vDash \encaps{A}$.
 \end{center}
\end{axiom}

\subsection{Meaning of Hoare triples}
\label{sect:HLmeans}

We  define the {\emph {meaning}} of  our Hoare triples, $M\ \vdash\  \{\, A \,  \}\ stmt\  \{\, A' \, \}$,  in the usual way, \ie that execution of $stmt$ in a state that satisfies $A$ leads to a state which satisfies $A'$. 

 
\begin{definition}[Semantics of Hoare triples]

 
For modules $M$, and assertions $A$, $A'$   we define:
%  the semantics of Hoare-triples,   $M\ \models\  \{\, A \,  \}\ stmt\  \{\, A' \, \}$ as follows:
\begin{itemize}
\item
$M\ \models\  \{\, A \,  \}\ stmt\  \{\, A' \, \}$ \\
iff\\
 for   all $\Mtwo$, for all $\sigma$, $\sigma'$ such that {$\arising \sigma {\Mtwo\cdot M}$}\\
$\strut ~ ~ ~ ~ [ \ \ M,\sigma \ \models \ A \ \wedge\  
 \sigma.cont = stmt  \ \wedge\     \leadstoBoundedStarFin {\Mtwo\cdot M}  {\sigma}  {\sigma'}    \ \ \Longrightarrow\ \ M,\sigma' \ \models \ A'\ \ ]$
\end{itemize}
\end{definition}
 

  
 
\subsection{Auxilairy properties of terminating execution} 
\label{sect:termExecs}
% In order to prove the soundness of our proof rules (\ie that $M \models  \{\, A \,  \}\ stmt\  \{\, A' \, \}$ implies that $M \vdash  \{\, A \,  \}\ stmt\  \{\, A' \, \}$,
We also need some auxiliary properties of the operational semantics.
 
Lemma \ref{lemma:encl:tem} guarantees that any state reachable from a state with a terminating execution has itself a terminating execution, and that this execution is enclosed in the original one:
 
 \begin{auxLemma}[Enclosed Terminating Executions]\footnote{TODO find better name for the aux lemma}
 \label{lemma:encl:tem}
 For any modules $\Mtwo$,   and states $\sigma$, $\sigma'$, $\sigma_1$:
\begin{itemize}
\item
$  \leadstoBoundedStarFin {\Mtwo}  {\sigma}  {\sigma'} \  \wedge \  \leadstoBoundedStar  {\Mtwo}  {\sigma}  {\sigma_1} 
% $\\ $
\ \ \  \Longrightarrow\ \ \  % $\\ $  
 \exists \sigma_2.[\ \ \leadstoBoundedStarFin {\Mtwo} {\sigma_1}  {\sigma_2}  
\ \wedge\ 
\leadstoBoundedStarThree  {\Mtwo}  {\sigma_2}  {\sigma}   {\sigma'} \ \ ]$
\end{itemize}

\end{auxLemma} 
 
Llemma \ref{lemma:subexp} makes the usual guarantee about terminating execution of statement sequences.
  
\begin{auxLemma}[Execution of sequences]
\label{lemma:subexp}
For any modules $\Mtwo$, statements $s_1$, $s_2$, and states $\sigma$, $\sigma'$:
\begin{itemize}
\item
$  \leadstoBoundedStarFin {\Mtwo}  {\sigma}  {\sigma'} \   \wedge\  \sigma.\texttt{cont} = s_1; s_2$\\
$  \Longrightarrow$\\
$   \exists \sigma''.[\ \ \leadstoBoundedStarFin {\Mtwo} {\sigma[\texttt{cont}\mapsto s_1]}  {\sigma''}  
\ \wedge\ 
\leadstoBoundedStarFin {\Mtwo} {\sigma''[\texttt{cont}\mapsto s_2]}   {\sigma'} \ \ ]$
\end{itemize}
\end{auxLemma}
 

Lemma \ref{lemma:external_breakdown} says that a terminating execution,  $ \leadstoBoundedStarFin {\Mtwo}  {\sigma}  {\sigma'}$ starting in an external state  consists of a sequence of  external states interleaved with terminating executions in internal states. 
Its meaning %of Auxialiry lemma \ref{lemma:external_breakdown} 
is illustrated through an example in Fig. \ref{fig:summaries}.

 
\begin{auxLemma}
\label{lemma:external_breakdown}[Summarised External Terminating Execution]
For any module $M$, modules $\Mtwo$, states $\sigma$ and $\sigma'$:
\\
If \\
$\strut \ \ \ \ \ \leadstoBoundedStarFin {M\cdot \Mtwo}  {\sigma}  {\sigma'} \ \ \wedge  \ \ M,\sigma \models \external {\texttt{this}}$
\\
then
\\
$\strut \ \ \ \ \ \exists p,q\in \mathbb{N},  \sigma_1, ... \sigma_q \in States, m_1,...m_p, n_1, ... n_p \in \mathbb{N}.$\\  
$\strut \ \ \ \ \ \ \ \ \ \ [  $ \\
$\strut \ \ \ \ \ \ \ \ \ \  \ \ \ \ \ \ \ \ \sigma_1=\sigma \ \ \wedge\ \  \sigma_p=\sigma'\ \  \wedge\ \ \forall i\in [1..q).[\   \leadstoOrig {M\cdot \Mtwo}  {\sigma_i}  {\sigma_{i+1}}\  ] \  \ \ \wedge$\\
$\strut \ \ \ \ \ \ \ \ \ \  \ \ \ \ \ \ \ \ m_1=1 \ \wedge\ n_p=q+1 \  \wedge\ \forall i\in[1..p).[  \  m_i < n_i < m_{i+1}  ] \ \wedge \ m_p<n_p\ \ \wedge  $\\
$\strut \ \ \ \ \ \ \ \ \ \  \ \ \ \ \ \ \ \ \forall i\in[1..p].\forall j\in [m_i..n_i)[\   M,\sigma_j \models \external {\texttt{this}} \ ] \  \ \ \wedge$\\
$\strut \ \ \ \ \ \ \ \ \ \  \ \ \ \ \ \ \ \ \forall i\in[1..p). [ \ M,\sigma_{n_i} \models \internal {\texttt{this}}   \ \wedge \ \ 
 \leadstoBoundedStarFin {M\cdot \Mtwo}  {\sigma_{n_{i}}}  {\sigma_{m_{i+1}-1}} \ ] $ \\
$\strut \ \ \ \ \ \ \ \ \ \ ]$
\end{auxLemma}



\begin{figure}[htb]
\begin{tabular}{c}
\hline \\
the original execution:
\\
~ \\
\resizebox{9cm}{!}
{
\includegraphics[width=\linewidth]{diagrams/summaryA.png}
} 
\\
\hline \\
the summarised execution:
\\
~ \\
\resizebox{9cm}{!}
{
\includegraphics[width=\linewidth]{diagrams/summaryB.png}
} 
\\
\hline \hline
\end{tabular}
   \caption{Summaries. 
   }
   \label{fig:summaries}
 \end{figure}

The lemma \ref{lemma:external_exec_preserves} describes how an encapsulated assertion $A$ is preserved during an execution, provided that all finalizing internal executions preserved it. 
It will be used in the proof of soundness of the rule {\sc{ExtCall\_WithSpec\_Weak}}\footnoteSD{perhaps also {\sc{ExtCall\_WithSpec\_Strong}}}

\begin{auxLemma}
\label{lemma:external_exec_preserves}[Preservation of Encapsulated Assertions]
For any module $M$, modules $\Mtwo$, numbers $p$ and $q$, states$\sigma_1$, .... $\sigma_p$,  numbers $m_1,...m_p, n_1, ... n_p \in \mathbb{N}$, and assertions $A$:
\\
If \\
$\strut \ \ \ \  M, \sigma_1 \models  A   \  \ \wedge \ \ M \models \encaps A \ \ \ \  \ \ \wedge$\\
$\strut \ \ \ \  \forall i\in [1..q).[\   \leadstoOrig {M\cdot \Mtwo}  {\sigma_i}  {\sigma_{i+1}}\  ] \  \ \ \wedge$\\
$\strut \ \ \ \  m_1=1 \ \wedge\ n_p=q+1 \  \wedge\ \forall i\in[1..p).[  \  m_i < n_i < m_{i+1}  ] \ \wedge \ m_p<n_p\ \ \wedge  $\\
$\strut \ \ \ \  \forall i\in[1..p].\forall j\in [m_i..n_i)[\   M,\sigma_j \models \external {\texttt{this}} \ ] \  \ \ \wedge$\\
$\strut \ \ \ \ \forall i\in[1..p). [ \ M,\sigma_{n_i} \models A   \ \Rightarrow \ \ 
M, {\sigma_{m_{i+1}-1}} \models A  \ ] $ \\
then\\
$\strut \ \ \ \  \ M, \sigma_q \models  A$
\end{auxLemma}



\subsection{Proving Soundness of the Hoare Logic}
\label{sect:prove:sound}

We will prove soundness by well-founded induction. \footnote{This kind of induction is described in \texttt{https://en.wikipedia.org/wiki/Well-founded\_relation.}}
We start by designing a well--founded ordering. 

\begin{definition}
For a module $M$, modules $\Mtwo$, assertions $A_1$, $A_2$, $A_3$, $A_4$, states $\sigma_1$, $\sigma_2$, and   $P$ a set of proof trees for triples of the form 
$M \vdash \{ A \} stmt  \{ A' \} $., we define
\begin{itemize}
\item
$(A_1,\sigma_1,A_2) \ll_{M,\Mtwo,P}  (A_2,\sigma_2,A_4)$ iff
\\
$\exists \sigma_4.[\ \leadstoBoundedStarFin {M\cdot \Mtwo} {\sigma_2}    {\sigma_4}\ ]$, and 
\begin{enumerate}
\item
$P$ has a proof for $M \vdash \{ A_2 \} \ \sigma_2.\texttt{cont}\  \{ A_4 \} $, and one of the direct predecessors of that proof in $P$ is $M \vdash \{ A_1 \}\ \sigma_1.\texttt{cont}  \{ A_3 \} $ \\
or
\item
 $\leadstoBoundedStar {M\cdot \Mtwo} {\sigma_2}    {\sigma_1}$, and $M \vdash \{ A_1 \}\ \sigma_1.\texttt{cont}\  \{ A_3 \} $ is proven in $P$
\end{enumerate}
\end{itemize}
\end{definition}

We need to show that $\_ \ll_{\_} \_$ is well-founded. 
To do this, we   "normalize" the set of proofs $P$, so that  
the proof trees or subtrees of any given triple within P are all identical.
% any proof tree (or subtree) of the triple  $M \vdash \{ A \}\ strms\  \{ A' \} $
% appearing in $P$, is identical to any other proof  tree (or subtree) of that triple in $P$.
% We take two modules $M$ and $M'$ arbitrary. We define the following property of states:

\begin{auxLemma}
\label{lemma:normal}
For a module $M$, modules $\Mtwo$,   and  a $P$ normalized  set of proof trees, the relation $\_ \ll_{M,\Mtwo,P}  \_$
is well-founded.
Moreover, for any set of proof trees, $P$, there exists a normalized set of proof trees $P'$ which proves the same set of triples.
\end{auxLemma}

We now prove soundness of the inference system $M \vdash \{ A \}\ strms\  \{ A' \} $:

\begin{theorem}
Take modules $M$ and $\Mtwo$, such that $\vdash M$. Then, for any assertions $A$ and $A'$, and statement  $stmt$, we have

\begin{center}
$M\ \vdash\  \{\, A \,  \}\ stmt\  \{\, A' \, \}$ \ \ \ \ implies \ \ \ \ $M\ \models\  \{\, A \,  \}\ stmt\  \{\, A' \, \}$
\end{center}

\end{theorem}


%$Q(\sigma)\ \ \ \triangleq\  \ \ \forall A, A'. $\\
%$\strut \hspace{2cm} [ \  \ Arising(\sigma,M*M') \ \  \wedge\ \ 
%% $\\$\strut \hspace{2cm}  \ \ \  \ 
%M \vdash \{A \} \sigma.\texttt{cont} \{A' \} \ \ \wedge\ \  M, \sigma \models  A \ \ 
%\wedge\ \  \leadstoFin  {M*M'}{\sigma}  {\sigma'}$\\
%$\strut \hspace{2cm}   \ \  \Longrightarrow$\\
%$\strut \hspace{2cm} \ \   \ \ M, \sigma' \models  A'  \  \ ]$
%
%\noindent
%We assume that: \\
%$\strut \ \ \ (*) \ \ M \models HS(M)$
%where $HS$ also includes Hoare triples for methods.\\
%We will prove that \\
%$\strut \ \ \ (**) \ \ \forall \sigma.[  \ \ Q(\sigma)\ \ ]$
%\\
%The proof is by induction on $\sigma$ using the ordering $\ll_{M*M'}$. 

\noindent
\vspace{.2cm}
  {\textbf{Proof Sketch}} 

Take a set of proof trees $P$ so that $P$ proves that $\vdash M$, and also $M\ \vdash\  \{\, A \,  \}\ stmt\  \{\, A' \, \}$, which is normalized  (this exists by lemma \ref{lemma:normal})..
Then, the relation $\_ \ll_{M,\Mtwo,P}  \_$ is well-founded  (lemma \ref{lemma:normal}).

Assume that  $\ \ (*) \ \ M\ \vdash\  \{\, A \,  \}\ stmt\  \{\, A' \, \}\ \ $ is proven in $P$. We want to show that $M\ \models\  \{\, A \,  \}\ stmt\  \{\, A' \, \}$. 

We proceed by case analysis on the last step in $P$'s proof that $M\ \vdash\  \{\, A \,  \}\ stmt\  \{\, A' \, \}$.


\begin{description} 

\item[${\sc{extend}}$] 

By soundness of the underlying Hoare logic

\item[${\sc{combine}}$] 

Therefore, $A \txteq A_1 \wedge A_2$,  and $A' \txteq A_1' \wedge A_2'$, and the direct predecessors of the proof of $(*)$ in $P$ are $M\ \vdash\  \{\, A_1 \,  \}\ stmt\  \{\, A_1' \, \}$ and $M\ \vdash\  \{\, A_2 \,  \}\ stmt\  \{\, A_2' \, \}$. 

Therefore, $(A_1,stmt,A_1') \ll_{M,\Mtwo,P} (A,stmt,A')$ and also $(A_2,stmt,A_2') \ll_{M,\Mtwo,P} (A,stmt,A')$. 

By  induction hypothesis we obtain   $M\ \models\  \{\, A_1 \,  \}\ stmt\  \{\, A_1' \, \}$ and $M\ \models\  \{\, A_2 \,  \}\ stmt\  \{\, A_2' \, \}$. 

This gives that $M\ \models\  \{\, A \,  \}\ stmt\  \{\, A' \, \}$.

%
% \item[$e\ \equiv\ y.m(\overline y)$]  
%and $y$ is internal. 
%
%We construct the new frame for the call, $\sigma''$. We obtain that $\sigma'' \ll_{M*M'} \sigma$. We also have that  $ \leadstoFin  {M*M'}{\sigma''}  {\sigma'}$, and $\sigma''$ "returns to $\sigma'$.
%We apply the IH to $\sigma''$ and knowledge from Hoare loci rules, and are done.  
%
%\item[$e\ \equiv\ y.m(\overline y)$]  and $y$ is external. 
%\\
%We apply the rule for external calls, and obtain that the argument hinges on a requirement that there is an invariant property $A''$, which is lifted to the first external call, and will we need to demonstrate is preserved by al enclosed call internal or external, and this $A''$ must be valid at the last external frame, and is then lowered to the current frame.
%\\
%We apply lemma \ref{lemma:external_breakdown}. In terms of the example in Fig. \ref{fig:summaries} we are now working with the summarised execution, \ie the execution of the methods in $\sigma_8$ and 
%$\sigma_{22}$ is summarised into one large step.
%Regardless of example, obtain that the first external calls ($\sigma_2$, ... $\sigma_{j_2}$) all preserve $A''$ because it is encapsulated. 
%Therefore $A''$ also holds at $\sigma_{j_2}+1$. 
%Since we have (*), we know that $A''$ has been proven as an invariant of all external methods, and therefore, by applying the (H), we obtain that $A''$ holds at $\sigma_{j_2} + 3$  (In terms of our example in Fig. \ref{fig:summaries}  $A''$ would hold in $\sigma_{24}).$ 
%We continue in the same vain, until we reach $\sigma_{n-1}$ (that is $\sigma_{28}$ in our example), and then we apply the Hoare rule and what we know about lowering.
% \item[$e\ \equiv e_1; e_2$ ]
% By lemma \ref{lemma:subexp} we can break execution to two parts. For the second part we can apply the IH. But for the first part is looks as if we are repeating the work from earlier cases. OR... should we extend the def of $\\ll_M$ so that $\sigma[\texttt{cont}\mapsto e_1] \ll_M \sigma$.
\end{description}
\noindent
\vspace{.1cm}
  {\textbf{End Proof Sketch}} 

\vspace{1cm}

Finally, we can now prove soundness of the overall system

\begin{theorem}[Soundness]
\label{thm:soundness}
Assume a a sound \SpecO proof system, $\proves{M}{A}$, 
a sound encapsulation inference system, $\proves{M}{\encaps{A}}$,
 and  that on top of these systems we built
 the \SpecLang logic according to the rules in Figures \ref{f:classical->singlestep},  \ref{f:only-if-single}, 
 \ref{f:only-through},  and \ref{f:only-if},   then, for    all modules $M$, and all \SpecLang specifications  $S$:
 
 $$\proves{M}{S}\ \ \ \ \ \ \ \mbox{implies}\ \ \ \ \ \  \ \ \ \satisfies{M}{S}$$
\end{theorem}

\begin{proof}
follows from earlier theorem
% by induction on the derivation of $\proves{M}{S}$.
\end{proof}
 


Theorem. \ref{thm:soundness} demonstrates 
 that the   \SpecLang logic is sound with respect to the semantics of \SpecLang specifications.
 The \SpecLang logic parametric wrt to the algorithms for proving validity of assertions
 $\proves{M}{A}$, and 
 assertion encapsulation ($\proves{M}{\encaps{A}}$), and is sound
 provided that these two proof systems are sound.

