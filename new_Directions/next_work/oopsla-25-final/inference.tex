We   develop  an inference system for adherence to our specifications.
 We distinguish three phases:
 
\vspace{.05cm}
%In the first phase, 
\textit{\textbf{First Phase:}} We assume  an underlying Hoare logic, ${M \vdash_{ul}  \{A\} {\ stmt\ } \{A'\} }$, and extend it to a logic  % of triples
  ${\hproves{M}  {A} {\ stmt\ }{A'} }$  with % which have 
the expected meaning, \ie 
(*) execution of statement $stmt$ in a state satisfying %{the \emph{precondition}}  
$A$ will lead to a state satisfying  % {the \emph{postcondition}}  
$A'$.
These triples only apply to   $stmt$'s  that  do not contain method calls  (even internal) -- this is so, because method calls may make further calls to   external methods.
%, and therefore can only be described through quadruples.
In our extension we  introduce   judgements   which 
talk about protection.

\vspace{.05cm}

\textit{\textbf{Second Phase:}} We develop a logic of quadruples ${\hprovesN{M}  {A} {\ stmt\ }{A'} {A''}}$. These promise  (*) and  
in addition,   that (**) any intermediate external states reachable during execution of that $stmt$
 satisfy the \emph{mid-condition}  $A''$.  
% {We call $A''$ the \midcond.}
 We incorporate the triples from the first phase,       
introduce  mid-conditions, give the usual substructural rules, and deal with method calls. 
For internal   calls we use the methods' specs. %, and in addition, for public methods we use the fact that they preserve the module's invariants. 
For external   calls, we   use %can only use  the fact that they preserve
 the module's invariants. 
 
 \vspace{.05cm}
 
\textit{\textbf{Third Phase:} } We prove adherence  to  our specifications. 
For method specifications we require that the body maps the precondition to the postcondition and preserves the method's  mid-condition. 
For module invariants we require that they  are preserved by all public methods of the module.

 \vspace{.1cm}

\noindent
\textit{Preliminaries:}  
 The judgement    ${\promises M S}$ expresses that $S$ is part of $M$'s specification.  
In particular, it allows   \emph{safe  renamings}. 
These renamings are   a convenience, akin to the Barendregt convention, and  allow simpler Hoare rules  -- \cf Sect. \ref{sect:wf},
Def. \aref{F.1}{\ref{d:promises}}, and Ex. \aref{F.2}{\ref{e:rename}}. 
We also require an underlying Hoare logic with judgements $M \vdash_{ul} \{ A \} stmt \{ A' \}$ 
-- \cf Ax. \aref{F.3}{\ref{ax:ul}}.



\subsection{First Phase: Triples}
\label{s:hoare:first}

In  Fig. \ref{f:underly}  we introduce our triples, of the form ${   \hproves{M}  {A} {stmt}  {A'}}$. 
These promise, as expected, that any execution of $stmt$ in a state   satisfying $A$ leads to a state satisfying $A'$.
% chopped; we say it later in the lemma
%These triples only apply to statements that do not contain method calls.

{\small{
\begin{figure}[tht]
$
\begin{array}{c}
\begin{array}{lcl}
\inferruleSD{\hspace{0.5cm} [\sc{embed\_ul}]}
	{stmt  \ \mbox{contains no method call}   
        \\
        \Stable{A}  \ \  \Stable{A'}  \ \ \ \ \  \ \ { M \vdash_{ul} \{ \ A\ \} {\ stmt\ }\{\ A'\ \}}
	 	 }
	{\hproves{M}  {A} {\ stmt\ }{A'} } 
& &
\inferruleSD{\hspace{0.5cm} [\sc{prot-new}]}
	{
	\sdN{u \txtneq   x	 }
	}
	{	 
 	\hproves{M} 
 						{ \sdN{true} }  
 					{\  u = \prg{new}\ C \ }
 						  {  \inside{u}\  \wedge  \protectedFrom{u}{x}} 
 	} 
\\ \\
	{\inferruleSD{\hspace{0.5cm} [\sc{prot-1}]}
	{   stmt \ \mbox{ is free of  meth. cals, or assignment to $z$}
	\\
	{\hproves{M}  {{A \wedge} \re\!=\!z} {\ stmt\ }{ \re\!=\!z} }
	}
	{\hproves{M} 
						{\  {A \wedge} \inside{\re}  \ }
						{\  stmt \ }
						{\  \inside{\re}\ }
	}
}
& &
      {\inferruleSD{\hspace{0.5cm} [\sc{prot-2}]}
	{ stmt \mbox{ is either $x:=y$ or $x:=y.f$}\ \ \ \ \    z,z' \txtneq x 
		\\
	{\hproves{M} 
						{  {A \wedge} z\!=\!\re \wedge z'\!\!=\!\! \re' }
						{\ stmt }
						{   z\!=\! \re \wedge z'\!\!=\!\! \re' }
	}
	}
	{
	\hproves{M} 
						{\ {A \wedge} \protectedFrom{\re}{\re'}\ }
						{\ stmt }
						{\ \protectedFrom{\re}{\re'}\ }
	}
}
\\ \\
      {\inferruleSD{\hspace{0.5cm} [\sc{prot-3}]}
	{\sdN{x \txtneq   z	 } }
	{\hproves{M} 
						{\ \protectedFrom{y.f}{z}\ }
						{\ x =y.f\ }
						{\ \protectedFrom{x}{z}\ }
	}
}
& &
        {\inferruleSD{\hspace{0.5cm} [\sc{prot-4}]}
	{ \ A\txteq\protectedFrom{x}{z}  \wedge   \protectedFrom{x}{y'} \ \ \ \  \ A'\txteq\protectedFrom{x}{z}\ }
	{\hproves{M} 
						{  \ A \  }
						{\ y.f=y'\ }
						{  \ A'\   }
	}
}
\end{array}
\end{array}
 $
\caption{Embedding the Underlying Hoare Logic, and Protection}
\label{f:protection}
\label{f:underly}
\end{figure}
}}


With rule {\sc{embed\_{ul}}} in Fig. \ref{f:underly},  any triple $ \{ A \} stmt \{ A' \} $  whose statement does not contain a method call, and which 
can be proven in the underlying Hoare logic, can also be proven in our logic.  
In \textsc{Prot-1}, we see that  protection of an object $o$ is preserved by internal code which does not call any methods: namely any heap modifications will
ony affect internal objects, and this will not expose new, unmitigated external access to $o$.
 \textsc{Prot-2}, \textsc{Prot-3} and \textsc{Prot-4} describe the preservation of relative protection.
Proofs of soundness for these rules can be found in App.\aref{G.5}{\ref{s:app:protect:lemmas}}. 
    
 
\subsection{Second Phase: Quadruples}
\label{s:hoare:second}

\subsubsection{Introducing  mid-conditions, and substructural rules}
We now introduce  quadruple rules.  
Rule {\sc{mid}} embeds  triples  ${\hproves{M}  {A} {\ s\ }{A'} }$  into quadruples ${\hprovesN{M}  {A} {\ s\ }{A'} {A''} }$.
This is sound, because $stmt$ is guaranteed not to contain method calls (by lemma \aref{F.5}{ \ref{l:no:meth:calls}}).

\begin{center}
$
\begin{array}{c}
% \begin{array}{lcl}
\inferruleSD{[\sc{Mid}]}
	{\hproves{M}  {A} {\ stmt\ }{A'} }
	{\hprovesN{M}  {A} {\ stmt\ }{A'} {A''} }
  \end{array}
 $
 \end{center}
 
Substructural quadruple rules appear in  Fig.\aref{15}{ \ref{f:substructural:app}}, and are as expected: 
Rules   {\sc{sequ}} and {\sc{consequ}} are  the usual rules for statement sequences and consequence, adapted to quadruples.
Rule {\sc{combine}} combines two quadruples for the same statement into one.
Rule  {\sc{Absurd}} allows us to deduce anything out of \prg{false} precondition, and  {\sc{Cases}} allows for case analysis.
These  rules  apply to \emph{any} statements -- even those containing method calls.

\subsubsection{Adaptation}

  \label{s:viewAndProtect}
 
In the outline of the Hoare proof of the external call in  \S  \ref{sec:howThird},  we saw that an assertion of the form $\protectedFrom x {\overline {y}}$ at the call site may imply $\inside{x}$ at entry to the call.
More generally,  the $\pushSymbolInText$ operator  adapts an assertion from the view of the callee to that of the caller,
 and is used in the Hoare logic for method calls. It is defined below. 
 
 
\begin{definition}
\label{def:push}
[The $\pushSymbolInText$  operator]  

$
\begin{array}{c}
\begin{array}{l}
\begin{array}{rclcrcl}
  \PushAS y {\inside \re} & \triangleq &  \protectedFrom \re {\overline {y} }
  & \ \ \  \ &
  \PushAS y   {(A_1  \wedge  A_2)} & \triangleq &  (\PushAS y  { A_1})  \wedge  ( \PushAS y  {A_2} )  
\\ 
 \PushAS y {(\protectedFrom \re {\overline {u}})} &  \triangleq& \protectedFrom \re {\overline {u}} 
  & &
 \PushAS y  {(\forall x:C.A)} & \triangleq & \forall x:C.({\PushAS y A} )  
  \\  
  \PushAS y  {(\external \re)} &  \triangleq & {\external \re}  %   \PushAS y  {(\external \re)} & \triangleq &   {\external \re}
  & & 
  \PushAS y  {(\neg A)} &  \triangleq & \neg( {\PushAS y A} )  
    \\
     \PushAS y  {\re} &  \triangleq&   \re %    \PushAS y  {(\internal \re)} &  \triangleq & {\internal \re}
    & &
    \PushAS y  {(\re:C)} &  \triangleq&   \re:C 
 \end{array}
\end{array}
\end{array}
$
\label{f:Push}
\end{definition}
 

Only the first equation in  Def.  \ref{def:push}  is interesting: for $\re$ to be {protected at  a} callee with arguments $\overline y$, it should be protected from   % the call's 
these arguments -- thus
  $\PushAS y {\inside \re} = \protectedFrom {\re} {\overline {y}}$. 
The notation $\protectedFrom {\re} {\overline {y}}$   stands for $\protectedFrom \re {y_0}\  \wedge\  ...  \wedge \protectedFrom \re {y_n}$, assuming that $\overline y$=${y_0, ... y_n}$.

Lemma \ref{lemma:push:ass:state}  states that   
indeed, $\pushSymbolInText$ adapts assertions from the callee to the caller, and is the counterpart to the  
%operator 
$\pushSymbol$.
{In particular:\ \ 
 \ (\ref{l:push:stbl}):\    $\pushSymbolInText$ turns an assertion into a stable assertion.
\ (\ref{lemma:push:ass:state:one}):\ If the caller,   $\sigma$,  satisfies  $\PushSLong  {Rng(\phi)} {A}$, then  the callee,   $\PushSLong {\phi} {\sigma}$, satisfies $A$.
\ \ (\ref{lemma:push:ass:state:two}): \ When returning from external states,  an assertion implies its adapted version.
 \ \ (\ref{lemma:push:ass:state:three}): \ When calling from external states, an assertion implies its adapted version. % (Lemma \ref{lemma:internal:adapted}).}
}
  

{
\begin{lemma} 
\label{lemma:push:ass:state}
For  states  $\sigma$, assertions $A$, %, $A'$  $Stb^+(A)$, and 
so that $Stb^+(A)$ and $\fv(A)=\emptyset$,  % with $\fv(A)=\emptyset$,  %variables   $\overline y$, 
frame $\phi$,  variables $y_0$, $\overline y$: % with $Range (\phi)=\overline {\interpret \sigma y}$:

\begin{enumerate}
 \item
\label{l:push:stbl}
$\Stable{\,  \PushASLong {(y_0,\overline y)} A\, }$
\item
 \label{lemma:push:ass:state:one}
$M, \sigma \models \PushASLong  {Rng(\phi)} {A}\ \  \ \ \ \ \  \ \ \    \Longrightarrow  \ \ \ \ M,  \PushSLong {\phi} {\sigma}   \models A$
\item
 \label{lemma:push:ass:state:two}
$M,  \PushSLong {\phi} {\sigma}   \models  A  \wedge \extThis    \ \  \ \ \  \  \Longrightarrow  \ \ \ \ M, \sigma \models \PushASLong  {Rng(\phi)} {A}$
 \item
 \label{lemma:push:ass:state:three}
$M, \sigma  \models  A  \wedge \extThis  \ \wedge \ M\cdot\Mtwo \models \PushSLong {\phi} {\sigma}   \ \  \ \ \  \  \Longrightarrow  \ \ \ \ M, \PushSLong {\phi} {\sigma} \models \PushASLong  {Rng(\phi)} {A}$
\end{enumerate}
\end{lemma}
}

{
Proofs %of the lemma can be found
 in \A\ \aref{F.5}{\ref{appendix:adaptation}}. Example \ref{push:does:not:imply}
 demonstrates the need for    \prg{extl}  %  \prg{intl} and  
  requirement in  % (\ref{lemma:internal:adapted}) and 
  (\ref{lemma:push:ass:state:two}). %  above.  -- shorter
}



\begin{example}[When returning from   internal states, $A$ does not imply $\PushASLong {Rng(\phi)} {A}$]  
\label{push:does:not:imply}   
In  Fig. \ref{fig:ProtectedBoth} we have
 $\sigma_2= \PushSLong {\phi_2} {\sigma_1}$, and    $\sigma_2 \models \inside{o_1}$,  and $o_1\!\in \!Rng(\phi_2)$,
 but $\sigma_1 \not\models \protectedFrom {o_1} {o_1}$. % (by Def. and because $o_1=o_1$).  
% not needed
%--   (\ref{lemma:push:ass:state:two}): \textit{When from popping internal states, $A$ does not imply $\PushASLong {Rng(\phi} {A}$}:\  We have $\sigma_2 \models \inside{o_1}$, 
%$o_1\in Rng(\phi_1)$But, since $o_1=o_1$, we do not have 
%$\sigma_2 \not\models \protectedFrom {o_1} {o_1}.$.
%
% needs checking
%--  \textit{$\PushASLong {Rng(\phi} {A}$ does not imply $A$}:\  In  $\sigma_2$ from Fig. we have $\sigma_1 \models \inside{o_1}$, and
%$o_1\in Rng(\phi_1)$. But, since $o_1=o_1$, we do not have 
%$\sigma_1 \not\models \protectedFrom {o_1} {o_1}$.  
%  where $\interpret {\sigma_1} {\prg{this}}$ = $o_1$, and $o_1$ is {external},  and there is no other object. Then, we have
%$\_,\sigma_1 \models \inside {\prg{this}}$ and $\_,\sigma_1 \not\models \protectedFrom {\prg{this}} {\prg{this}}$.
%\notesep Nor does  $\PushAS {y} {A}$  imply $A$.\  \Eg  take a $\sigma_2$ where $\interpret {\sigma_2} {\prg{this}}$ = $o_1$,
% $\interpret {\sigma_2} {x}$ = $o_2$ , and  $\interpret {\sigma_2} {x.f}$ = $o_3$, and $o_2$ is external, and there are no other objects or fields.
% Then $\_,\sigma_2 \models   \protectedFrom {x.f} {\prg{this}}$ but  $\_,\sigma_2 \not\models \inside {x.f}$.
% 
\end{example}






\subsubsection{Reasoning about   calls}
\label{s:calls}
is described in Fig. \ref{f:internal:calls}. {\sc{Call\_Int}}  %and {\sc{Call\_Int\_Adapt}}  
 for internal methods, whether public or private; \ % \  {\sc{Call\_Ext}} 
and {\sc{Call\_Ext\_Adapt}} and {\sc{Call\_Ext}\_Adapt\_Strong} for  external methods.




\begin{figure}[htb]
{\small{
$\begin{array}{c}
 \inferruleSD{\hspace{4.7cm} [\sc{Call\_Int}]}
	{
	   	\begin{array}{c}
		\promises  M {\mprepostN{A_1}{p\ C}{m}{x}{C}{A_2} {A_3}}  \\
		{A_1' = A_1[y_0,\overline y/\prg{this}, \overline x]  \ \ \ \ \ \ \ \ \  A_2' = A_2[y_0,\overline y,u/\prg{this}, \overline x,\prg{res}]}  
		          	\end{array}
		}
	{  \hprovesN {M} 
						{ \  y_0:C,\overline {y:C} \wedge  {A_1'}\ }  
						 { \ u:=y_0.m(y_1,.. y_n)\    }
					         { \ {A_2'} \ } 
						{  \ A_3 \ }  
}
% \\
%  \\
% removed, as not sound.The example in \refl{push:does:not:imply}   shows why not.
%{ \inferruleSD{\hspace{4.7cm} [\sc{Call\_Int\_Adapt}]}
%	{
%	   	\begin{array}{c}
%		\promises  M {\mprepostN{A_1}{p\ C}{m}{x}{C}{A_2} {A_3}}   \\
%				{A_1' = A_1[y_0,\overline y/\prg{this}, \overline x]  \ \ \ \ \ \ \ \ \  A_2' = A_2[y_0,\overline y,u/\prg{this}, \overline x,\prg{res}] }  
%
%          	\end{array}
%		}
%	{  \hprovesN {M} 
%						{ \,  y_0\!:\!C, \overline {y\!:\!C} \wedge {\PushASLong {{(y_0,\overline {y})}} {A_1'}}\ }
%												{ \, u\!:=\!y_0.m(y_1,.. y_n)\,    }
%						{ \  { \PushASLong  {{(y_0,\overline {y})}} {A_2'}  } \ }  
%						{  \  A_3\  }  
%}
%}
\\
 \\ 
 \inferruleSD{\hspace{4.7cm} [\sc{Call\_Ext}\_Adapt]}
 	{ 
	 \promises M   {\TwoStatesN {\overline {x:C}} {A}} 
        }
	{   \hprovesN{M} 
						{ \    { \external{y_0}} \,     \wedge \,  \overline{x:C}\  \wedge\ {\PushASLong {{(y_0,\overline {y})}}  {A}}  \ } 
						{ \ u:=y_0.m(y_1,.. y_n)\    }
						{ \   {\PushASLong {{(y_0,\overline {y})}}  A}  \ }
						{\  A \   }
	}	
\\
 \\ 
{
 \inferruleSD{\hspace{4.7cm} [\sc{Call\_Ext}\_Adapt\_Strong]}
 	{ 
	 \promises M   {\TwoStatesN {\overline {x:C}} {A}} 
        }
	{   \hprovesN{M} 
						{ \    { \external{y_0}} \,     \wedge \,  \overline{x:C}\ \wedge  A   \wedge\ {\PushASLong {{(y_0,\overline {y})}}  {A} }\  }   
						{ \ u:=y_0.m(y_1,.. y_n)\    }
						{ \   A \wedge {\PushASLong {{(y_0,\overline {y})}}  {A} } \  }  
						{\  A \   }	
}
}

\end{array}
$
}}
\caption{Hoare Quadruples for Internal and External Calls -- here $\overline y$ stands for $y_1, ... y_n$}
\label{f:internal:calls}
\label{f:external:calls}
\label{f:calls}
\end{figure}

For  internal calls, we  start, as usual,  by looking up the method's specification, and 
{substituting the formal  by the actual parameters parameters ($\prg{this}, \overline{x}$  by $y_0,\overline{y}$).} % in the method's pre- and post-condition.
  {\sc{Call\_Int}} is as expected:   we  require the precondition, and guarantee the postcondition and mid-condition.
%For {\sc{Call\_Int\_Adapt}} we require the adapted pre-condition ($  \PushASLong {(y_0, \overline y)}{A_1'}$  rather than $A_1'$) and also ensure the adapted post-condition ($ \PushASLong {(y_0, \overline y)}{A_2'}$  rather than $A_2'$).
%Remember that $ \PushASLong {(y_0, \overline y)}{A_1}$ at the caller's side guarantees that $A_1$ holds at the start of the call  (after pushing the frame with   $y_0, \overline y$), while  
%$A_2$ at the end of the call  guarantees that  $ \PushASLong {(y_0, \overline y)}{A_2}$  holds when returning to the caller's side  (after popping the callee's frame)
%-- cf.  lemma \ref{lemma:push:ass:state}.
 {\sc{Call\_Int}} %and {\sc{Call\_Int\_Adapt}}  are 
 {is} applicable whether the method is public or private.


For external calls, % {\sc{Call\_Ext\_Adapt}}, 
we consider the module's invariants. 
If the module promises to preserve $A$, \ie if  $\promises M   {\TwoStatesN {\overline {x:D}} {A}}$, {and  if its adapted version, $ \PushASLong {(y_0, \overline y)}{A}$},  holds before the call, then it also holds after  the call \ ({\sc{Call\_Ext\_Adapt}}).
{If, in addition, the un-adapted version also holds before the call, then it also holds after the call  ({\sc{Call\_Ext\_Adapt\_Strong}}).}
%\sdNew{We require the adapted invariant ($  \PushASLong {(y_0, \overline y)}{A}$  rather than $A$) and  ensure invariant (again $  \PushASLong {(y_0, \overline y)}{A}$  rather than $A$).
%Remember that $  \PushASLong {(y_0, \overline y)}{A}$ at the caller's side guarantees that $A$ holds at the start of the call  (after pushing the frame with   $y_0, \overline y$), and conversely, 
%$A\wedge \extThis$ at the end of the call  guarantees that  $ \PushASLong {(y_0, \overline y)}{A}$  holds when returning to the caller's side  (after popping the callee's frame)
%-- cf.  lemma \ref{lemma:push:ass:state}.
%}

% SHORTER
% Moreover, $A$ is also a \midcond of the call.


\vspace{.1cm}

Notice that   internal calls, {\sc{Call\_Int}},   require   the \emph{un-adapted} %version of the the 
 method  precondition (\ie $A_1'$), while   external calls, both {\sc{Call\_Ext\_Adapt}} and {\sc{Call\_Ext\_Adapt\_Strong}},  require the 
 \emph{adapted} % version of the 
 invariant (\ie $ \PushASLong {(y_0, \overline y)}{A}$). 
{This is sound, because  internal callees preserve  %variable-free, 
 $\Pos{\_}$-assertions} % are preserved against pushing of frames 
 -- \cf Lemma \ref{l:preserve:asrt}. 
On the other hand, %when the callee is external, then
% $\Pos{\_}$-assertions are not necessarily preserved against pushing of frames
 {external callees do not necessarily preserve  $\Pos{\_}$-assertions} -- \cf Ex. \ref{push:does:not:preserve}. 
Therefore, in order to guarantee that $A$ holds upon entry to the callee, we need to know that $ \PushASLong {(y_0, \overline y)}{A}$ held at the caller site -- \cf Lemma \ref{lemma:push:ass:state}.

% SD removed for time being:
%Remember also, that $A$ does not imply $ \PushASLong {(y_0, \overline y)} {A}$, nor does $ \PushASLong {(y_0, \overline y)}{A}$  imply $A$ -- \cf example \ref{push:does:not:imply}.


Remember that {popping frames does not necessarily preserve}
%SD droped below, as we say it also about 15 lines earlier 
% while $\Pos{\_}$,-assertions are preserved against pushing of internal frames, 
% above they 
 $\Pos{\_}$  assertions % are not necessarily preserved against popping of such frames 
-- \cf Ex. \ref{ex:pop:does:not:preserve}.
Nevertheless, {\sc{Call\_Int}} guarantees the unadapted version, $A$,  upon return from the call. 
This is sound, because of our % use  of the concept of
 \emph{\strong satisfaction} of assertions -- more in Sect.  \ref{s:scoped:valid}.
 

%{Rules  {\sc{CallAndAlias}}  and  {\sc{CallNonAlias}} say that calls preserve aliasing, resp. non-aliasing, between variables, ie they preserve $x=x$ resp. $x\neq x$. These two rules apply to internal as well as external calls. When the callee's receiver is external, they make the extra requirement that  $\PushAS {y}{\extract{M}}$ -- we  will discuss this requirement together with the discussion of Fig. \ref{f:external:calls}.     Note that $x=x'$ expresses that $x$ and $x'$ are aliases, while  $u\txteq x$ expresses that $u$ and $x$ are textually the same --
%the latter is stronger, i.e.   $x\txteq u$ implies $x=u$. 
%% It is possible that variables are aliases, without being textually the same, i.e. it is possible to have $x=x'$ while $u\not\equiv x'$. 
% As $...\ equiv ...$ is a textual assertion, and thus  state-independent,  it is a side-condition of the rules and is  not part of the Hoare triple's precondition.
%}
%
%
%\small{
%\begin{figure}[hbt]
%$\begin{array}{c}
%\inferruleSD{\hspace{4.7cm}  [{\sc{CallAndAlias}}}
%	{ 
%	{   x \txtneq u\txtneq x'  }
%	}
%	{   \hproves{M}  { \ x=x'\   \wedge \ ({\external{y_0}}  \rightarrow \  \PushAS {y}{\extract{M}})\  }	{ \ u:=y_0.m(y_1,.. y_n)\  } { \  x=x'\ }	 }
%
%\\ \\ 
%\inferruleSD{\hspace{4.7cm} [{\sc{CallNonAlias}}]}	
%{ 
%		{ \ x \txtneq u \txtneq x'\   }
%	}
%	{   \hproves{M}   { \ x\neq x'\   \wedge \ ({\external{y_0}}  \rightarrow \  \PushAS {y}{\extract{M}})\  } { \ u:=y_0.m(y_1,.. y_n))\  } { \  x\neq x'\ }	 }
%\\
%\\
%\end{array}
%$
%\caption{Logic for Aliasing around Calls }
%\label{f:internal:alias:calls}
%\end{figure}
%}

\paragraphsd{%Discussion: 
Polymorphic Calls.} Our rules do  not \emph{directly} address a scenario where  the receiver may be
   internal or external, and where  the choice about this is made at runtime. 
However, such scenaria  are \emph{indirectly} supported, through our rules of consequence and  case-split.
More   in  \A\ \aref{H.6}{\ref{app:polymorphic}}.

\begin{example}[Proving external calls]
We continue our discussion from \S \ref{sec:howThird} on how to establish the Hoare triple    \textbf{(1)} :

 \vspace{.05cm}
  \begin{minipage}{.05\textwidth}
   \textbf{(1?)}\ \ 
\end{minipage}
\hfill
\begin{minipage}{.95\textwidth}
\begin{flushleft}
$\{\  \   \external{\prg{buyer}} \ \wedge\ 
% red here is meanrt to make the contrast with the  next one 
   {\protectedFrom {\prg{this.\myAccount.key}}  {\prg{buyer} } }
 \ \wedge\ \prg{this.\myAccount.\balance}= b  \ \  \}$\\
$\ \ \ \ \ \ \ \ \ \ \ \ {\ \prg{buyer.pay(this.accnt,price)}   \ } $\\
$  \{\  \ \  {\prg{this.\myAccount.\balance}} \geq  b \  \  \} \ \ ||\ \  \{\ \inside{\prg{a.\password}}\wedge  \prg{a.\balance}\!\geq\!{\prg{b}}   \ \}  $ 
\end{flushleft}
\end{minipage}
 
\vspace{.03cm}
\noindent
We use $S_3$, which says that $\TwoStatesN{ \prg{a}:\prg{Account},\prg{b}:\prg{int} } {\inside{\prg{a.key}} \wedge \prg{a.\balance} \geq \prg{b} }$. 
We can apply rule {\sc{Call\_Ext\_Adapt}}, by taking  $y_0 \triangleq \prg{buyer}$,  and $\overline {x : D}\triangleq \prg{a}:\prg{Account},\prg{b}:\prg{int}$, 
and $A \triangleq  \inside{\prg{a.\pwd}}\wedge \prg{a.\balance}\geq \prg{b}$, \ \ 
and $m \triangleq \prg{pay}$,\ and $\overline {y} \triangleq \prg{this.accnt},\prg{price}$,
and provided that we can establish that\\
\strut \ \ \   \textbf{(2?)}  $ {\small{\strut \ \ \ \protectedFrom {\prg{this.\myAccount.key}} {(\prg{buyer},\prg{this.\myAccount},\prg{price})}}}$\\
holds. Using type information, we obtain that all fields transitively accessible from \prg{this.\myAccount.key}, or \prg{price} are internal or scalar. This implies\\
\strut \ \ \   \textbf{(3)}  $ {\small{\strut \ \ \ \protectedFrom {\prg{this.\myAccount.key}} {\prg{this.\myAccount}} \wedge  \protectedFrom {\prg{this.\myAccount.key}} {\prg{price}}}} $\\
Using then    Def. \ref{def:push},  we can indeed establish that\\
\strut \ \ \   \textbf{(4)} $ {\small{\strut \ \ \ \protectedFrom {\prg{this.\myAccount.key}} {(\prg{buyer},\prg{this.\myAccount},\prg{price})}  \ = \   \protectedFrom {\prg{this.\myAccount.key}} {\prg{buyer}}}}$\
Then, by application of the rule of consequence, \textbf{(4)}, and the rule {\sc{Call\_Ext\_Adapt}}, we can establish \textbf{(1)}.\ \ \ 
More details in \S  \aref{H.3}{\ref{l:buy:sat}}.
\end{example}

\subsection{Third phase: Proving adherence to Module Specifications}
\label{sect:wf}

In Fig. \ref{f:wf} we  define the judgment $\vdash M$, which says that  
$M$ has been proven to be well formed. 
 

\begin{figure}[thb]
$
\begin{array}{l}
\begin{array}{lcl}
\inferruleSDNarrow 
{~ \strut  {\sc{WellFrm\_Mod}}}
{  \vdash \SpecOf {M}
  \hspace{1.2cm}  M \vdash \SpecOf {M}
}
{
\vdash M  
}
& \hspace{0.7cm} &
\inferruleSDNarrow 
{~ \strut   {\sc{Comb\_Spec}}}
{  
M \vdash S_1 \hspace{1.2cm}  M \vdash S_2
}
{
M \vdash S_1 \wedge S_2
}
\end{array}
\\
\inferruleSD 
{~ \strut \hspace{6.5cm} {\sc{method}}}
{  
 \prg{mBody}(m,D,M)=p \ (\overline{y:D})\{\  stmt \ \}       
    \\
  {\hprovesN{M} { \ \prg{this}:\prg{D}, \overline{y:D}\, {\wedge\, A_1}\  } %\, \sdN{\wedge\, \PushASLong {(\prg{this},\overline y)}  {A_1}}\, } 
  {\ stmt\ } {\ A_2\, \sdN{\wedge\, \PushASLong {\prg{res}} {A_2}} \ }   {A_3} } 
}
{
M \vdash {\mprepostN {A_1}{p\ D}{m}{y}{D}{A_2} {A_3} }
}
\\
\inferruleSD 
{~ \strut \hspace{6.5cm} {\sc{invariant}}}
{
\begin{array}{l}
\forall  D,  m.[\  \prg{mBody}(m,D,M)=\prg{public} \ (\overline{y:D})\{\  stmt \ \}      \ \ \Longrightarrow  
  \\
%   ~ \strut \hspace{0.3cm}  \ \ \ 
 \begin{array}{l}
   \hprovesN {M}  
{ \ \prg{this}:\prg{D}, \overline{y:D},\,   \overline{x:C} \, \wedge\,  A \, \sdN{\wedge\, \PushASLong {(\prg{this},\overline y)} {A}}\, }  
  	{\ stmt\ }   
	 {\  A\, \wedge\, \PushASLong {\prg{res}} {A} \ }  
{\ A \ } \ ]
 \end{array}
 \end{array}
}
{
M \vdash \TwoStatesN{ \overline{x:C}} {A}
}
\end{array}
$
\caption{Methods' and Modules' Adherence to Specification}
\label{f:wf}
\end{figure}

\footnoteSD{Julian's reflections on the invariant and method call rules:

 I think it somewhat makes sense for the Invariant rule to be asymmetric since
calls to internal code are asymmetric from the perspective of protection. There is no change in specific protection when entering 
internal code, but there may be when returning to external code from internal code. 

I'd have to think about it more, but I don't believe there is any decrease in general protection when entering internal code,
but there may be when exiting internal code. The adaptation in the post-condition is there to address the potential
for protection to be lost on return.

Magic wand adaptation (i.e. A -* x) expresses what must hold before crossing the boundary from internal to external code
(i.e. before an internal method return or call to an external method) for the adapted assertion to still be satisfied after crossing 
that boundary. I think there are a few things that are useful to consider here:
it is good to note that if you remove protection from the assertion language, the magic wand would be the identity
(i.e. A -* x = A) since only protection is modified when passing values between objects via method calls
more importantly, the way we define protection is based around fields and external/internal objects, and passing a value
to an internal object (via a method call) modifies neither a field nor grants any external object access to any other object
According to my understanding of the intended semantics of the magic wand, entering internal code from frame should
not affect any sort of protection.
My impression is that part of the intent of the magic wand is to be able to express ``what needs to hold before a call
so that <x>  holds within the new frame (whether that frame is internal or external)?''. I still need to think about how that
fits in with my above description of magic wand adaptation.
}

{\sc{WellFrm\_Mod}} and {\sc{Comb\_Spec}} say that $M$ is well-formed  if its specification is well-formed (according to Def. \ref{f:holistic-wff}), and if $M$ satisfies all conjuncts of the specification.
{\sc{Method}} says that  
a module satisfies a method specification if the % choopped method body for space
body satisfies the corresponding pre-, post- and \midcond. 
% The following sentence is wrong!
%\sdN{Moreover, the precondition is strengthened by $\PushASLong {(\prg{this},\overline y)} {A}$ -- this is sound because the state is internal, and by Lemma  \ref{lemma:internal:adapted}.
 In  the postcondition we also ask that $\PushASLong {\prg{res}}  {A}$, so that \prg{res} does not leak any of the values that $A$ promises will be protected. 
{\sc{Invariant}} says that  a module satisfies {a} %an invariant 
specification $\TwoStatesN{ \overline{x:C}} {A}$,  if the method body of each public method
 has $A$ as its  pre-, post- and \midcond. 
{Moreover, the precondition is strengthened by $\PushASLong {(\prg{this},\overline y)} {A}$ -- this is sound because
 the caller is external, and by Lemma   \ref{lemma:push:ass:state}, part (\ref{lemma:push:ass:state:three}).}
 
 
\paragraphsd{Barendregt} In  {\sc{method}} we \sdN{implicitly} require   the free variables  in a method's precondition  not to overlap with variables in its body, unless they are the receiver or one of the parameters ($\sdN{\vs(stmt) \cap \fv(A_1) \subseteq   \{ \prg{this}, y_1, ... y_n \} }$).  And in {\sc{invariant}} we require   the free variables in $A$ (which are a subset of  $\overline x$) not to overlap with the variable  in $stmt$ ($ \sdN{ \vs(stmt)\,  \cap\, \overline x\, =\, \emptyset}$).
This can easily be achieved through renamings, \cf Def. \aref{H.3}{\ref{d:promises}}.

\newcommand{\sdsp}{\strut \ \ \ \ \ }

\begin{example}[Proving a public method] Consider the proof that \prg{Account::set} from $M_{fine}$ satisfies $S_2$. 
Applying rule {\sc{invariant}}, we need to establish:\\
\label{e:public}
{\small{ \vspace{.05cm}
  \begin{minipage}{.05\textwidth}
  \textbf{(5?)}\ \ 
\end{minipage}
\hfill
\begin{minipage}{.95\textwidth}
\begin{flushleft}
$\{ \  \   ...\  \prg{a}:\prg{Account}\ \wedge\,  {\inside{\prg{a.key}}} \, \wedge \, \protectedFrom {\prg{a.key}} { (\, \prg{key'},\prg{key''}\, )} \  \} \ $\\
$\ \ \ \ \ \ \ \ \ \ \  \prg{body\_of\_set\_in\_Account\_in}\_ M_{fine}\   $\\
$  \{\  \    {\inside{\prg{a.key}}}\ \wedge\ {\PushASLong {\prg{res}\ } {\ \inside{\prg{a.key}}}} \ \   \} \ \ \  || \ \ \ 
	\{\ \    {\inside{\prg{a.key}}}\ \  \}  $ 
\end{flushleft}
\end{minipage}
}}
\vspace{.03cm}
\noindent
Given the conditional statement in \prg{set}, and with the obvious treatment of conditionals (\cf Fig. \aref{15}{\ref{f:substructural:app}}), among other things, we  need to prove for the \prg{true}-branch that:\\
\vspace{.01cm}
{\small{  \begin{minipage}{.05\textwidth}
  \textbf{(6?)}\ \ 
\end{minipage}
\hfill
\begin{minipage}{.95\textwidth}
\begin{flushleft}
$\{ \  \   ... \   {\inside{\prg{a.key}}} \, \wedge \, \protectedFrom {\prg{a.key}} { (\, \prg{key'},\prg{key''}\, )}  \wedge  \,  \prg{this.key}=\prg{key'}\  \} \ $\\
$\ \ \ \ \ \ \ \ \ \ \  \prg{this.key := key''}\   $\\
$  \{\  \  {\inside{\prg{a.key}}}\   \   \} \ \ \  || \ \ \ \{  \ \  {\inside{\prg{a.key}}}\  \  \}  $ 
\end{flushleft}
\end{minipage}
}}
\\
 \vspace{.03cm}
\noindent
We can apply case-split  (\cf Fig. \aref{15}{\ref{f:substructural:app}}) on whether \prg{this}=\prg{a}, and thus a proof of \textbf{(7?)} and \textbf{(8?)}, would give us a proof of \textbf{(6?)}:\\
 \vspace{.03cm}
{\small{  \begin{minipage}{.05\textwidth}
   \textbf{(7?)}\ \ 
\end{minipage}
\hfill
\begin{minipage}{.95\textwidth}
\begin{flushleft}
$\{ \  \   ... \   {\inside{\prg{a.key}}} \, \wedge \, \protectedFrom {\prg{a.key}} { (\, \prg{key'},\prg{key''}\, )} \wedge  \,  \prg{this.key}\!=\!\prg{key'}\ \wedge\ \prg{this}\!=\!\prg{a} \ \  \} \ $\\
$\ \ \ \ \ \ \ \ \ \ \   \prg{this.key := key''}\    $\\
$  \{\  \      {\inside{\prg{a.key}}} \   \   \} \ \ \  || \ \ \ \{  \ \  {\inside{\prg{a.key}}} \ \  \}  $ 
\end{flushleft}
\end{minipage}
}}
\\
and also
\\
 \vspace{.03cm}
{\small{  \begin{minipage}{.05\textwidth}
   \textbf{(8?)}\ \ 
\end{minipage}
\hfill
\begin{minipage}{.95\textwidth}
\begin{flushleft}
$\{ \  \   ...   {\inside{\prg{a.key}}} \, \wedge \, \protectedFrom {\prg{a.key}} { (\, \prg{key'},\prg{key''}\, )}  \wedge  \,  \prg{this.key}\!=\!\prg{key'}\ \wedge\ \prg{this}\!\neq\!\prg{a} \ \ \} \ $\\
$\ \ \ \ \ \ \ \ \ \ \  \prg{this.key := key''}\   $\\
$  \{\  \   {\inside{\prg{a.key}}}  \  \  \}\ \ \  || \ \ \ \{  \ \  {\inside{\prg{a.key}}} \ \  \}  $ 
\end{flushleft}
\end{minipage}
}}
 
 \vspace{.03cm}
\noindent
If  $ \prg{this.key}\!\!=\!\!\prg{key'} \wedge\ \prg{this}\!\!=\!\!\prg{a}$, then $\prg{a.key}\!\!=\!\!\prg{key'}$. But   $\protectedFrom {\prg{a.key}} {\prg{key'}}$ and   % rule 
 {\sc{Prot-Neq}} from Fig. \aref{16}{\ref{f:protection:conseq:ext}} give $\prg{a.key}\!\neq\!\prg{key'}$. So,  by contradiction (\cf Fig. \aref{15}{\ref{f:substructural:app}}), we can prove    \textbf{(7?)}.
If  $\prg{this}\!\neq \!\prg{a}$, then we  obtain from the underlying Hoare logic that the value of \prg{a.key} did not change. Thus, by rule {\sc{Prot\_1}}, we obtain  \textbf{(8?)}. \ \ \ More details in \S \aref{H.5}{\ref{l:set:sat}}.
 
 \vspace{.045cm}
\noindent
On the other hand, \prg{set} from $M_{bad}$ cannot be proven to satisfy $S_2$, because it % \prg{set} from $M_bad$ 
% does not have the condition \prg{this.key}=\prg{key'}, and 
requires  proving\\
\vspace{.03cm}
{\small{  \begin{minipage}{.05\textwidth}
   \textbf{(???)}\ \ 
\end{minipage}
\hfill
\begin{minipage}{.95\textwidth}
\begin{flushleft}
$\{ \  \   ...  \  {\inside{\prg{a.key}}} \, \wedge \, \protectedFrom {\prg{a.key}} { (\, \prg{key'},\prg{key''}\, )}  \   \} \ $\\
$\ \ \ \ \ \ \ \ \ \ \  \prg{this.key := key''}\   $\\
$  \{\  \    \{ {\inside{\prg{a.key}}}\   \   \}\ \ \  || \ \ \ \{  \ \  {\inside{\prg{a.key}}} \ \  \}  $ 
\end{flushleft}
\end{minipage}
}}
\\
and without the condition \prg{this.key}=\prg{key'} there is no way we can prove \textbf{(???)}.

\subsection{Our Example Proven} % of our proof system}
\label{sect:example:proof:short}
Using our Hoare logic, we have developed a mechanised proof in Coq, that, indeed, $M_{good} \vdash S_2 \wedge S_3$.
This proof is  part of the accompanying artifact. 
 
 Our proof models  \LangOO, the assertion language, the specification language, and the Hoare logic from \S \ref{s:hoare:first},  \S   \ref{s:hoare:second},  \S  \ref{sect:wf},  \S \aref{F}{\ref{app:hoare}} and Def. \ref{def:push}.
In keeping with   the start of  \S \ref{sect:proofSystem}, our proof assumes the existence of an underlying Hoare logic,  
and several, standard, properties of that underlying logic, the assertions logic (\eg equality of objects implies equality of field accesses) and of type systems
(\eg  fields of objects of different types cannot be aliases of one another).
All assumptions  are clearly indicated in the associated artifact.
 Appendix \aref{H}{\ref{s:app:example}}  %included in the auxiliary material, 
outlines   that proof. 

% Finally, we discuss why  $M_{good} \not\vdash S_2 $

\end{example}

 
