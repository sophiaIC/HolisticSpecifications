
This paper advances the state-of-the-art in verifying the protection
of component invariants against massive
attacks \cite{massive-attack-protection} in the untrusted environment
of the open world \cite{baresi2006toward,swapsies}.
%
We demonstrate how classical software verification techniques
can be extended to verify properties of software that must
interact with unverified, untrusted, unknown, or even antagonistic
contexts; and where control and data must flow from untrusted to
trusted components, from trusted to untrusted components, and then
back into the trusted components.
%
The core of our contribution is a scoped viewpoint based notion of
protection for object capabilities, which ensures that attackers
cannot subvert verified components to gain unmediated access to
protected capabilities --- but without needing information about how (or
indeed if) those capabilities are used in the rest of the program, or
otherwise constraining their use.  We work in a framework of Hoare
logics, enabling a systematic approach to
verification, rather than constructing specialised proofs \textit{ab
initio} for each problem. 


Directions for future work include applying these techniques to
languages that rely on lexical nesting for access
control such as Javascript \cite{ooToSecurity},
rather than public/private annotations;
languages that support ownership types that can be leveraged for
verification
\cite{leveragingRust-oopsla2019,RustHornBelt-pldi2022,verus-oopsla2023},
and languages from the
functional tradition such as OCAML, which is gaining imperative
feature such as ownership and uniqueness \cite{funk-ownership-oopsla2024,ocaml-ownership-icfp2024}. 
%
Similarly there is potential in applying this approach within
proof systems based e.g.\ on separation logic \cite{cerise-jacm2024},
and post-Hoare logics like incorrectness logic \cite{IncorrectnessLogic} or
hyperlogics \cite{compositional-hypersafety-oopsla2022,hyper-hoare-pldi2024},
and to verifying (and ``disentangling'') concurrent programs \cite{seplog-disentanglelment-popl2024}.
%
We expect our techniques can be incorporated into existing program
verification tools \cite{Cok2022}, especially those attempting
gradual verification \cite{gradual-verification-popl2024},
thus paving the way towards practical verification for
the open world.






%% current apple watch 2024 
%% 64G storage
%% 18000Mhz  - 1.8Ghz 
%% 1G RAM
%% 2 main 64bit cores 
%% 100Mbits  netowrk
%%
%% first apple watch  2015
%% 512M dram
%% 8G storage
%% 500MhZ clock
%%
%% first iPhone 2007
%% 128 M Dram
%% 4/8/6G 
%% 400MhZ clock
%% 1 main core 32 bit
%% 100 kbits netowrk
