\section{Proof of Guarantee of Safety in \S\ref{sec:how}}
\label{app:safety}

In this section we provide a proof sketch that \SrobustB ensures our balance
does not decrease in contexts with no access to our password. This 
property is expressed in \S\ref{sec:how}, and the example is repeated below.

\begin{lstlisting}[mathescape=true, language=chainmail, frame=lines]
module $\ModParam{1}$
     ...
    method cautious(untrusted:Object)
        a = new Account
        p = new Password
        a.set(null,p)
        ...
        untrusted.make_payment(a)
        ...
\end{lstlisting}
\jm[]{
The guarantee for the above code snippet is that  as long as 
\prg{untrusted} does not have external access (whether transitive or direct)
to \sdN{\prg{a.pwd}} before the call on line 7, then \prg{a.balance} will not decrease during the 
execution of line 8. This property is expressed and proven in Theorem \ref{thm:safety}.
}
%\begin{itemize}
%\item
%suppose we rewrite \prg{Mod$\_1$} such that we duplicate all methods used in line 7. In the duplicated version of the methods
%we add a tuple as an argument that contains the account, and the current password. In every duplicate method, whenever the
%method \prg{setPassword} is called, we check if the receiver is \prg{a}, and if so we update the password in the tuple to the 
%new password. In this manner by the end of line 7, we still know what the password of the account is.
%\item
%We then insert the following code at the end of line 7:\\
%\prg{a.setPassword(t.snd(), new Password())}
%\item
%We are now in an arising program state where there is no external knowledge of the password, and we are able to apply \SrobustB
%\end{itemize}

%\begin{lemma}[Out of Scope heap locations are not modified by execution]
%Let $\sigma_1 = (\chi_1, \phi_1 : \psi)$ and $\sigma_2 = (\chi_2, \phi_2 : \psi)$ be program states such that
%$\reductions{M}{M'}{\sigma_1}{\sigma_2}$ for some modules $M$ and $M'$.
%Let $\chi = \chi_A \cup \chi_B$, where $\chi_A$ is the portion of the heap that is not
%transitively accessible by $\phi_1.(\prg{this})$ (i.e. the transitive closure of $\access{\_}{\_}$), then $\chi_A \subset \chi_2$.
%\end{lemma}
%\begin{proof}
%We proceed simply by contradiction. Suppose that $\chi_A \not\subset \chi_2$, that means that there must 
%have been some single execution step where $\chi_A$ was modified. Such a step requires access to some object $o$
%in $\chi_A$. Thus, there must have been some series of execution steps that provided access to $o$ to an object $o'$
%that is transitively accessible from $\phi_1.(\prg{this})$. This contradicts our premise.
%\end{proof}

%\begin{lemma}[Execution is unaffected by out of scope heap locations]
%\label{lemma:scoped-execution}
%Let $\sigma_1 = (\chi_1, \phi_1 : \psi)$ and $\sigma_2 = (\chi_2, \phi_2 : \psi)$ be program states such that
%$\reductions{M}{M'}{\sigma_1}{\sigma_2}$ for some modules $M$ and $M'$.
%Let $\chi = \chi_A \cup \chi_B$, where $\chi_A$ is the portion of the heap that is not
%transitively accessible by $\phi_1.(\prg{this})$ (i.e. the transitive closure of $\access{\_}{\_}$), then 
%\begin{enumerate}
%\item
%there exists $\chi_C$ such that $\chi_2 = \chi_A \cup \chi_C$ and 
%\item
%for any $\chi_A'$, we have $\reductions{M}{M'}{(\chi_A' \cup \chi_B, \phi_1 : \psi)}{(\chi_A' \cup \chi_C, \psi_2)}$
%\end{enumerate}
%\end{lemma}
%\begin{proof}
%In both instances we proceed simply by contradiction. 
%
%\begin{enumerate}
%\item
%Suppose that $\chi_A \not\subset \chi_2$, that means that there must 
%have been some single execution step where $\chi_A$ was modified. Such a step requires access to some object $o$
%in $\chi_A$. Thus, there must have been some series of execution steps that provided access to $o$ to an object $o'$
%that is transitively accessible from $\phi_1.(\prg{this})$. Such a series of execution steps requires transitive
%access from $\sigma.(\prg{this})$ (whether by method calls or field accesses) since heap locations in \Loo are unforgeable.
%Such a case would imply transitive access from $\sigma.(\prg{this})$ to $o$. This contradicts our premise.
%
%\item
%For execution to differ in the modified program state $(\chi_A' \cup \chi_B, \phi_1 : \psi)$,
%that would imply that there must have been at least one execution step that differed in outcome due to the presence of $\chi_A'$ over $\chi_A$.
%By case analysis on the semantics of \Loo detailed in Fig. \ref{f:loo-semantics}, we can see that in each instance for 
%a result from execution to differ as a result a modification to the heap (when all else remains that same), would require transitive access 
%to the modified portion of the heap. Again, as before this implies a contradiction to our original premise.
%\end{enumerate}
%\end{proof}
%
%\begin{lemma}[Constructing an Effectively Equivalent Heap]
%\label{lemma:garbage}
%Let $\sigma = (\chi_A \cup \chi_B, \phi : \psi)$ be a program state arising (in the context of internal module $M_I$, and external module $M_E$)
%and let $o$ be an object in $\chi_B$, where $\chi_A$ is the set of all objects that are 
%not transitively accessible by $o$. Let $p$ be some internal object in $\chi_B$ of class $C$. There exists some 
%$M_E'$ and $\chi_A'$ such that $(\chi_A' \cup \chi_B, \phi : \psi)$ is \textit{Arising}
%in modules $M_I$ and $M_E'$ and for all $o'$ in $\chi_A'$, $\satisfies{M_I; (\chi_A', \chi_B, \phi : \psi)}{\neg \access{o}{o'}}$
%\end{lemma}
%\begin{proof}
%The solution is to start by creating an $M_E'$ that results in such a $\chi_A'$. To attain such a module we
%\begin{enumerate}
%\item
%add a \prg{nullify} method to all classes in $M_E$, that when called replaces all references to objects of type $C$ with \prg{null}
%\item
%for each instance in the execution from the initial program state until $\sigma$, whenever an external object with 
%access to an object of type $C$ is about to leave scope for the remainder of that execution (whether by overwriting a reference, or by making a method call),
%a duplicate of that method is made, and in the duplicate we insert a call to \prg{nullify} on that object. The current call 
%is then replaced with a call to the duplicate of that method
%\end{enumerate}
%In this way we are effectively manually ``garbage collecting'' objects of type $C$. The result of this $M_E'$ will be a $\chi_A'$ that does not have
%any references to object $p$. Further, since the process only works when objects leave scope, none of the modified objects will be in $\chi_B$.
%\end{proof}

\begin{theorem}[\SrobustB Guarantees Account Safety]
\label{thm:safety}
Let \prg{BankMdl} be some module that satisfies \SrobustB, let 
$M$ be any external module, and $\sigma_1 = (\chi_1, \phi_1 : \psi_1)$ be some \textit{Arising} program state,
\sdN{$\arising{M}{\prg{BankMdl}}{\sigma_1}$}.
\\
If
\begin{enumerate}
\item
the continuation of $\sdN{\phi_1}$ is
\begin{verbatim}
    a = new Account; 
    p = new Password; 
    a.set(null,p); 
    s; 
    untrusted.make_payment(a, z1, ..., zn); ...
\end{verbatim}
\item
$\sigma_2 = (\chi_2, \phi_2 : \psi_2)$ is the program state immediately preceding the execution of \prg{s}
\item
$\sigma_3 = (\chi_3, \phi_3 : \psi_3)$ is the program state immediately following the execution of \prg{s}
% where $\satisfies{\prg{BankMdl};\sigma_3}{\prg{a.password} \neq \prg{zk}}$ for all $i \leq \prg{k} \leq n$
\item
$\sigma_4 = (\chi_4, \phi_4 : \psi_4)$ is the program state immediately following the execution of\\ \prg{untrusted}\prg{.make\_payment}\prg{(a, z1, ..., zn)} 
\item
for all objects $o \in \chi_3$ which are transitively accessible (i.e. the transitive closure of $\access{\_}{\_}$) from \prg{untrusted}
\sdN{or from \prg{z1},...\prg{zn}}:\\
 $\strut \ \ \ \  \ \ \satisfies{\prg{BankMdl}; \sigma_3}{\access{o}{\sdN{\prg{a.pwd}}}}$, \ \ \
implies \ \ \  $\satisfies{\prg{BankMdl}; \sigma_3}{\internal{o}}$, 
\item
$\satisfies{\prg{BankMdl}; \sigma_3}{\prg{a.balance} = b}$
\end{enumerate}
then 
\begin{itemize}
\item
$\satisfies{\prg{BankMdl};\sigma_4}{\prg{a.balance} \geq b}$.
\end{itemize}
\end{theorem}

\vspace{.1in}

\noindent
{\bf{Proof Idea}}

\noindent
\sdS{We would like to apply \SrobustB in state $\sigma_3$, and argue that since by (5) no external object transitively accessible from \prg{a},  
\prg{untrusted}, \prg{z1}, ... \prg{zn} has access to the password, the balance in $\sigma_4$ will not decrease over what it was in $\sigma_3$.}
\sdS{However, the challenge is that the premise of \SrobustB is stronger than what we have in (5). Namely the premise of \SrobustB requires
that no external object has (direct) access to the password, but this
 requirement might not hold in  $\sigma_3$: depending on the contents of the code in \prg{s},} there may exist  external 
objects that have access to \prg{a.password}.  For example, if \prg{s} 
is the empty code, then $\sigma_1(\prg{this})$ has access to \prg{a}. 

\sdN{To address this challenge}, we will   create 
a program state, say \sdN{$\sigma_3'$}. In the new program state $\sigma_3'$ there will be no external access to \prg{a.password}. 
Also, $\sigma_3'$  must be similar enough to 
$\sigma_3$ so  that the execution of \prg{untrusted.make\_payment(a, z1, ..., zn)} \sdS{starting from state $\sigma_3$ is effectively
 equivalent to the execution of \prg{untrusted.make\_payment(a, z1, ..., zn)}  when starting from  $\sigma_3'$.}
 \sdN{Moreover,}  \sdN{$\sigma_3'$}, must also be \textit{Arising} for us to apply the \Nec specification 
\SrobustB to it.

 \sdN{This throws up a new challenge: $\sigma_3'$ is not necessarily \textit{Arising} in \prg{BankMdl} and $M$. We address the latter challenge by creating a new module, $M'$, such that  $\arising{M'}{\prg{BankMdl}}{\sigma_3'}$.}

\vspace{.1in}

\noindent
{\bf{Proof Sketch}}

\noindent
We construct $M'$ from $M$ by 1) modifying  all methods in all 
classes in $M$  so that all methods are duplicated: a) the original version, and b) 
a version almost identical  to that in $M$ with the addition that it keeps track of all the objects which contain fields pointing to
any objects of the \prg{Password} class, 
2) We add to all classes in $M$ a method called \prg{nullify}
that compares the contents of each of its fields with the method's argument, and if they are
equal overwrites the field with \prg{null}, 
3) all method calls  are replaced by those in part 1a, except of the body of \prg{make\_payment},
4) we modify the code in \prg{s} (and any methods called from it) so that it also keeps track of the current value of
\prg{a.pwd}, 
5) after \prg{s} and before the call \prg{untrusted.make\_payment(a, z1, ..., zn)} we insert  code which
runs through the list created in part 1, and calls \prg{nullify} with the current value of \prg{a.pwd} by \prg{null} as its argument.

By staring with the same initial configuration which reached $\sigma_3$,
 but now using $M'$ as the external module, we reach $\sigma_3'$, 
that is, $\arising{M'}{\prg{BankMdl}}{\sigma_3'}$. Moreover,  $\sigma_3'$ 
satisfies the premise of  \SrobustB. 
We execute $\prg{untrusted}\prg{.make\_payment}\prg{(a, z1, ..., zn)}$ in the context of   $\sigma_3'$ 
and reach $\sigma_4'$. We apply  \SrobustB, and obtain that $\satisfies{\prg{BankMdl};\sigma_4'}{\prg{a.balance} \geq b}$.

We use the latter fact, to conclude that $\satisfies{\prg{BankMdl};\sigma_4}{\prg{a.balance} \geq b}$.
Namely, $\sigma_3$ and $\sigma_3'$ are equivalent -- up to renaming of addresses  -- for all all the objects which are
reachable from \prg{o}, \prg{z1}, ... \prg{zn}, and for all objects from 
$\prg{BankMdl}$. Therefore, the execution of \prg{make\_payment} in $M;\prg{BankMdl}$ and $\sigma_3$
will be "equivalent" to that in $M';\prg{BankMdl}$ and $\sigma_3'$. Therefore, $\sigma_4$ and $\sigma_4'$ are equivalent -- up to renaming of addresses  -- for all all the objects which are
reachable from \prg{o}, \prg{z1}, ... \prg{zn} and for all objects from 
$\prg{BankMdl}$.
This gives us that  $\satisfies{\prg{BankMdl};\sigma_4}{\prg{a.balance} \geq b}$.

%\prg{candidates}.
% We also modify the code in \prg{s} in two ways: 1) We keep track of the current value of \prg{a.pwd}, and 2) at the end  of 
%
%
% the statements \prg{s}, so that the code in \prg{s} keeps track of the current 
%value of \prg{a.pwd}, and keeps a list of all external objects which have a field pointing to \prg{a.pwd} -- this requires that methods
%in $M'$ are modified to accommodate this. Moreover, at the end of these modified statements, we replace all such 
%By Lemma \ref{lemma:garbage}, we know that there exists such a program state, and that we can construct it. By Lemma \ref{lemma:scoped-execution} this will 
%not alter the execution of \prg{untrusted.make\_payment(a, z1, ..., zn)}.
%Thus, it follows that \prg{a.balance} in $\chi_3'$ is equal to \prg{a.balance} in $\chi_3$.
%Since \SrobustB is applicable to $\sigma_3'$, we can conclude that \prg{a.balance} in $\chi_4'$ (and thus $\chi_4$)
% 
 



