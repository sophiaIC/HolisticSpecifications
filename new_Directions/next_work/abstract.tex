 \begin{abstract}
 
 
 In today's complex software, internal, trusted, code  is tightly intertwined  
with external, untrusted, code: external code may call  methods 
from the  internal  code, and internal code may call method from the external code.
% In particular, trusted  code may be called by external, unverified, untrusted code; moreover, it may call such external, unverified, untrusted code
Thus, \emph{external calls}, \ie internal code calling methods from external code, may, in principle, have an unlimited range of effects.

Nevertheless, the effects of external calls, can be \emph{\tamed}, \ie reduced, if the internal module %makes use of encapsulation and
  is programmed so as to prevent some effects from happening.
\Tamed effects allow us to reason that external calls will preserve some \se{why only some?} assertions.
%
\se{Given that this is the abstract I would skip the next two sentences.}
To \tame effects, one needs to use encapsulation (\eg private fields, no address forging, \etc). 
One may also employ  the OCAP (object capabilities) model, whereby 
% capabilities are transferable rights to perform one or more operations on a given object;
access to the capability is  necessary for some  effect to  happen.
%Capabilities are unforgeable, and can only be acquired through explicit passing from caller to callee, or at object creation.
 % As a result, sharing of capabilities has to be carefully managed.
%

Our work addresses the specification and verification of internal code which uses encapsulation and OCAP to \tame effects.
\se{Are they assertions and specifications or assertion constructs and specification language?}
We propose new assertions talking about access to capabilities,
new specifications describing \tamed effects,
a Hoare logic that can verify external calls, and a Hoare logic that can verify that a module satisfies its \tamed effects specification.
We prove soundness of the Hoare logic.
We illustrate the approach though a running example. 
%% -- something not yet sufficiently  tackled by the OCAP  literature.
%{\textbf {For the specification:}} To describe
% not being ``entitled'' to some effect, we give a means to guarantee  that lack of (eventual) access to certain capabilities ensures
%that certain effects will not take place.
% To reflect that the guarantees
%should hold in the presence of internal code calling and being called by external code, we 
%% are about eventual} effects we 
%propose {\emph{scoped invariants}}, where the notion of the future is  {constrained} by the currently executing method. 
%%We propose means to specify the preservation of such properties, and the meaning of being ``entitled'' to some effect. 
%To define the   management of sharing of capabilities, we propose the concept of  protected capability which guarantees  that access by external objects
% is controlled by the internal objects.
%  
%%  
%%  Our specification language can express that capabilities are necessary (rather than sufficient) conditions for certain effects
%%(thus going further than traditional object capabilities).
%% We propose 
%% With these ingredients, internal code may call external code in the knowledge that ...
%{\textbf {For the verification: }} We  propose Hoare logic inference rules   {to prove} that calls to external code are safe. 
%We also propose inference rules   {to prove}    that  {our} modules satisfy our specifications.
%We then  {demonstrate}  that a sound   Hoare logic augmented by the proposed rules is also sound. 
%We use a motivating example  and   prove its properties using our logic.


\end{abstract}

