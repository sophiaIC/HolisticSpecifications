 \begin{abstract}
In today's complex software, internal, trusted, code % needs to cooperate with external code  of unknown provenance and trustworthiness.
tightly cooperates with external, untrusted, code: external code may call into internal  code, and internal code may call out to external code.
% In particular, trusted  code may be called by external, unverified, untrusted code; moreover, it may call such external, unverified, untrusted code
This tight cooperation introduces a high degree of uncertainty which developers strive to mitigate:
they %want to preserve   important properties of their data, and  
 want to prevent the external code from causing certain effects unless it is ``entitled'' to.

The OCAP (object capabilities) model has proposed means to mitigate the uncertainty:
Capabilities are transferable rights to perform one or more operations on a given object.
Capabilities are unforgeable, and can only be acquired through explicit passing from caller to callee, or at object creation.
\susan{Previous sentence should be deleted as it breaks the flow and I don't think is needed in the abstract. If you do that then the next sentence needs to be slightly reworded}
As a result, sharing of capabilities has to be carefully managed.

The topic of this work is  the specification and verification of code written using the OCAP model -- something not yet tackled by the OCAP model itself. 
To define what it means to be ``entitled'' to some effect, our specifications can express that lack of (eventual) access to certain capabilities ensures
that certain effects will not take place.
%We propose means to specify the preservation of such properties, and the meaning of being ``entitled'' to some effect. 
  To define the   management of sharing of capabilities, we propose the concept of  "protected capability" which guarantees  that access of external code is controlled by the internal code.
%  
%  Our specification language can express that capabilities are necessary (rather than sufficient) conditions for certain effects
%(thus going further than traditional object capabilities).
% We propose 
% With these ingredients, internal code may call external code in the knowledge that ...
We  propose an extension of a Hoare logic that can prove that modules satisfy such specifications, we prove the soundness of this extension, and also prove one specific example.
 

\end{abstract}

