 \begin{abstract}
In today's complex software, internal, trusted, code % needs to cooperate with external code  of unknown provenance and trustworthiness.
tightly cooperates with external, untrusted, code: external code may call into internal  code, and internal code may call out to external code.
% In particular, trusted  code may be called by external, unverified, untrusted code; moreover, it may call such external, unverified, untrusted code
This tight cooperation introduces a high degree of uncertainty which developers strive to mitigate:
they %want to preserve   important properties of their data, and  
 want to prevent the external code, \susan{when calling into the internal code,} 
 from 
 causing certain effects unless it is ``entitled'' to.

The OCAP (object capabilities) model has proposed means to mitigate the uncertainty:
Capabilities are transferable rights to perform one or more operations on a given object.
\sdN{Thus, capabilities \emph{enable} effects. }
More strongly than that:  
Capabilities \emph{guard} effects: callers can make the effect   happen, only if they have access to the corresponding capability.
%Capabilities are unforgeable, and can only be acquired through explicit passing from caller to callee, or at object creation.
  As a result, sharing of capabilities has to be carefully managed.

Our work addresses the specification and verification of code \sdN{relying on capabilities as guards}.
% -- something not yet sufficiently  tackled by the OCAP  literature.
To describe being ``entitled'' to some effect, our specifications can guaratee that lack of (eventual) access to certain capabilities ensures
that certain effects will not take place.
\sdN{To reflect that the guarantees
should hold in the presence of internal code calling and being called by external code, we 
% are about \sdN{eventual} effects we 
propose \emph{bounded two-state invariants}. where the notion of the  future is bounded by the currently executing method.}
%We propose means to specify the preservation of such properties, and the meaning of being ``entitled'' to some effect. 
To define the   management of sharing of capabilities, we propose the concept of  \sdN{protected capability} which guarantees  that access by external objects
 is controlled by the internal objects.
  
%  
%  Our specification language can express that capabilities are necessary (rather than sufficient) conditions for certain effects
%(thus going further than traditional object capabilities).
% We propose 
% With these ingredients, internal code may call external code in the knowledge that ...
\sdN{We  propose an extension of a Hoare logic that can prove that calls to external code are safe, and that \susan{our} modules satisfy our specifications.
We prove the soundness of the extended Hoare logic.
We use a motivating example  and   prove its properties using our logic.
}

\end{abstract}

