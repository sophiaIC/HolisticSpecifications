
\section{\LangOO - full defintion}
\label{app:loo}


We introduce \LangOO, a simple, typed, class-based, object-oriented language.
To reduce the complexity of our formal models, \LangOO lacks many
common languages features, omitting static fields and methods, interfaces,
inheritance, subsumption, exceptions, and control flow.  
 \LangOO includes ghost fields,  that may only be used in the specification language.
and which may be defined recursively.
%
%\kjx{
%These features are
%well-understood: their presence (or absence) would not chanage the
%results we claim nor the structures of the proofs of those results.
%Similarly, while Loo is typed, we don't present or mechanise
%its type system. 
%Our results and proofs rely only upon type
%soundness --- in fact, we only need that an expression of
%type $T$ (where $T$ is a class $C$ declared in module $M$)
%will evaluate to an instance of some class from $M$,
%with the same confinement status as $C$.
%Featherweight Java extended with modules and assignment
%will more than suffice \cite{IgaPieWadTOPLAS01}.
%% well-understood that it is too boring to present here or to mechanise anew --- 
%%
%}


\subsection{Syntax}
The syntax of \LangOO is given in Fig. \ref{f:loo-syntax}.
\LangOO modules ($M$) map class names ($C$) to class definitions ($\textit{ClassDef}$).
A class definition consists of % \jm[]{an optional annotation \enclosed},
a list of field definitions, ghost field definitions, and method definitions.
{Fields, ghost fields, and methods all have types, $C$; {types are
    classes}.
    Ghost fields may be optionally 
 annotated as \texttt{intrnl}, requiring the argument to have an internal type, and the 
body of the ghost field to only contain references to internal objects. This is enforced by the
limited type system of \LangOO.}
A program state ($\sigma$) is a pair of of a stack and a heap.
The stack is a a stack is a non-empty list of frames ($\phi$), and the heal ($\chi$)
is a map from addresses ($\alpha$) to objects ($o$). A frame consists of a local variable
map and a continuation (\prg{cont}) that represents the statements that are yet to be executed ($s$).
% or a hole waiting to be filled by a method return in the frame above ($x := \bullet; s$).
A statement is either a field read ($x := y.f$), a field write ($x.f := y$), a method call
($u :=y_0.m(\overline{y})$), a constructor call ($\prg{new}\ C$), 
% a method return statement ($\prg{return}\ x$), 
  a sequence of statements ($s;\ s$),
  or empty ($\epsilon$).


\LangOO also includes syntax for ghost terms $gt$ that may %only
be used in writing
specifications or the definition of ghost fields.


\begin{figure}[t]
\footnotesize
\[
\begin{syntax}
\syntaxID{x, y, z}{Variable}
\syntaxID{C, D}{Class Id.}
\syntaxID{f}{Field Id.}
\syntaxID{g}{Ghost Field Id.}
\syntaxID{m}{Method Id.}
\syntaxID{\alpha}{Address Id.}
\syntaxInSet{i}{\IntSet}{Integer}
\syntaxElement{v}{Value}
		{
		\syntaxline
				{\alpha}
				{i}
				{\true}
				{\false}
				{\nul}
		\endsyntaxline
		}
\endSyntaxElement\\
\\
\syntaxElement{Mdl}{Module Def.}
		{
		\syntaxline{\overline{C\ \mapsto\ CDef}}\endsyntaxline
		}
\endSyntaxElement\\
\syntaxElement{CDef}{Class Def.}
		{
		% [An]\
		 \prg{class}\ C\ 
		\{\  \overline{fld}; \overline{mth};\  \overline{gfld};\  \}		
		}
\endSyntaxElement\\
%\syntaxElement{An}{Class Annotation}
%		{\enclosed}
%\endSyntaxElement\\
\syntaxElement{T}{Type}
		{
		\syntaxline
%				{\_}
				{C}
		\endsyntaxline
		}
\endSyntaxElement
\\
\syntaxElement{s}{Statement}
		{
				\syntaxline
				{\sdN{x:=y}}
				{\sdN{x:=v}}
				{x:=y.f}
				{x.f:=y}
				{x:=y_0.m(\overline{y})}
				{\prg{new} \ {C} }
				{ s;\ s }
				  { \epsilon }
			       \endsyntaxline
		}
\endSyntaxElement
\\
%\syntaxElement{c}{Continuation}
%		{
%		\syntaxline
%				{\sdN{s; \ x}}
%				{\sdN{x}}
%		\endsyntaxline
%		}
%\endSyntaxElement\\
\syntaxElement{mth}{Method Def.}
		{
		\syntaxline
		{ \sdN{pr}\  \prg{method}\ m\ (\overline{x : T})\sdN{:T}\{\ s\ \} }
		\endsyntaxline
		}
\endSyntaxElement\\
\syntaxElement{pr}{Privacy}
		{
		\syntaxline
		{\prg{private}}
		{\prg{protected}}
		\endsyntaxline
		}
\endSyntaxElement\\
\syntaxElement{fld}{Field Def.}
		{\syntaxline
			{\prg{field}\ f\ :\ T}
		\endsyntaxline}
\endSyntaxElement\\
\\
\syntaxElement{gfld}{Ghost Field Def.}
		{\syntaxline
			{\prg{ghost}\ g\ (\overline{x : T})\{\ gt\ \} : T}
			{\prg{ghost}\ \prg{intrnl}\ g\  (\overline{x : T})\{\ gt\ \} : T}
		\endsyntaxline}
\endSyntaxElement\\
\syntaxElement{\sdN{gt}}{\sdN{Ghost Term}}
		{
		\syntaxline
				{x}
				{v}
				{gt\! +\! gt}
				{gt\! =\! gt}
				{gt \!< \!gt}
				{\prg{if}\ gt\ \prg{then}\ gt\ \prg{else}\ gt}
		\endsyntaxline
		}
		{
		\syntaxline
				{gt.f}
				{gt.g(\overline {gt})}
		\endsyntaxline
		}
\endSyntaxElement
\\
\\
\syntaxElement{\sigma}{Program \sdN{State}}
		{
		\syntaxline
		{( \overline \phi, \chi )}
		\endsyntaxline
		}
\endSyntaxElement 
\\
% SD Stack is implictt now
%\syntaxElement{\psi}{Stack}
%		{\syntaxline{\phi}{\phi \sdN{\cdot} \psi}\endsyntaxline}
%\endSyntaxElement\\
\syntaxElement{\phi}{Frame}
		{
		\syntaxline
		{  (\  \overline{x\mapsto v};\ s \ ) }
		\endsyntaxline
		}
\endSyntaxElement\\
\syntaxElement{\chi}{Heap}
		{(\  \overline{\alpha \mapsto o}\ )}
\endSyntaxElement\\
\syntaxElement{o}{Object}
		{(\ C;\  \overline{f \mapsto v} \ )}
\endSyntaxElement\\
\end{syntax}
\]
\caption{\LangOO Syntax}
\label{f:loo-syntax}
\end{figure}

\footnoteSD{\red{JULIAN: Do we need the selectors? SD}}

\subsection{Semantics}
\LangOO is a simple object oriented language, and the operational semantics 
(given in Fig. \ref{f:loo-semantics} and discussed later)
do not introduce any novel or surprising features. The operational 
semantics make use of several helper definitions that we 
define here.

\sdN{
We provide a definition of reference interpretation in Definition \ref{def:interpret}
\begin{definition}
\label{def:interpret}
For a frame $\phi= (\overline {x \mapsto v}, s)$, and a program state $\sigma = (\overline \phi \cdot \phi,, \chi)$, we   define:
\begin{itemize}
\item
$\interpret{\phi}{x}\ \triangleq\ v_i$\ \ \ if \ \ \ $x=x_i$
\item
 $\interpret{\sigma}{x}\ \triangleq\  \interpret{\phi}{x}$
\item
$\interpret{\sigma}{\alpha.f}\ \triangleq\ v_i $ \ \ if \ \ $\chi(\alpha)=(\_; \  \overline {f \mapsto v})$, and $f_i=f$
\item
$\interpret{\sigma}{x.f}\ \triangleq\ \interpret{\sigma}{\alpha.f}$ where $\interpret{\sigma}{x}=\alpha$
\item
$\phi(\prg{contn}) \ \triangleq\ s$ 
\item
$\sigma(\prg{contn}) \ \triangleq\ \phi(\prg{contn})$\
\item
$\phi[\prg{contn}\mapsto s'] \ \triangleq\ (\overline {x \mapsto v}, s')$
\item
$\sigma[\prg{contn}\mapsto s'] \ \triangleq \ (\ {\overline \phi}\cdot \phi[\prg{contn}\mapsto s'],\  \chi\ )$ 
\item
$\phi[\prg{x'}\mapsto v'] \ \triangleq\ ( \ (\overline {x \mapsto v})[\prg{x'}\mapsto v'],\ s \ )$
\item
$\sigma[\prg{x'}\mapsto v'] \ \triangleq\ (\ (\overline {\phi} \cdot (\phi[\prg{x'}\mapsto v']), \ \chi)$ 
\item
$\sigma [\alpha \mapsto o ] \ \triangleq\ (\ (\overline {\phi} \cdot \phi), \ \chi [\alpha \mapsto o ]\ )$ 
\item
$\sigma [\alpha.f' \mapsto v' ] \ \triangleq\ \sigma [\alpha \mapsto o ] $\ \ \  if \ \  
$\chi(\alpha)=(C, {\overline {f \mapsto v}})$, and $o=(\ C;  ({\overline {f \mapsto v}})[f' \mapsto v' ]\ )$ 
\end{itemize}
\end{definition}
}
That is, a variable $x$, or a field access on a variable $x.f$ 
has an interpretation within a program state of value $v$
if $x$ maps to $v$ in the local variable map, or the field
$f$ of the object identified by $x$ points to $v$.

Definition \ref{def:class-lookup} defines the class lookup function an object 
identified by variable $x$.
\begin{definition}[Class Lookup]
\label{def:class-lookup}
For program state $\sigma = ({\overline {\phi}}\cdot\phi, \chi)$, class lookup is defined as 
$$\class{\sigma}{x}\ \triangleq\ C \ \ \ \ \ \mbox{if} \ \ \  \chi(\interpret{\sigma}{x})=(C,\_ )$$
\end{definition}

Module linking is defined for modules with disjoint definitions:

\begin{definition}
\label{def:linking}
For all modules $\Mtwo$ and $M$, if the domains of $\Mtwo$ and $M$ are disjoint, 
we define the module linking function as $M\cdot \Mtwo\ \triangleq\ M\ \cup\ M'$.
\end{definition}
That is,  their linking is the union of the two if their domains are disjoint.

Definition \ref{def:meth-lookup} defines the method lookup function for a method
call $m$ on an object of class $C$.
\begin{definition}[Method Lookup]
\label{def:meth-lookup}
For module $\Mtwo$, class $C$, and method name $m$, method lookup is defined as 
$$\meth{\Mtwo}{C}{m}\ \triangleq\ { pr}\  \prg{method}\ m\ (\overline{x : T}) {:T}\{\ s\ \}  $$
if there exists an $M$ in $\Mtwo$, so that $M(C)$ contains the definition ${ pr}\  \prg{method}\ m\ (\overline{x : T}) {:T}\{\ s\ \} $
\end{definition}



\newcommand{\Same}[4]{{SameModule(#1,#2,#3,#4)}}

Finally, we define what it means for two objects to come from the same module
\begin{definition}[Same Module]
\label{def:class-lookup}
For program state $\sigma$,  modules $\Mtwo$, and variables $x$ and $y$, we defone
$$\Same {x} {y} {\sigma}{\Mtwo}\ \triangleq\ \exists C, C', M[ \ M\in \Mtwo \wedge C, C'\in M \wedge  \class{\sigma}{x}=C \wedge \class{\sigma}{y} =C'\ ]$$
\end{definition}


Fig. \ref{f:loo-semantics} gives the operational semantics of \LangOO. 
Program state $\sigma_1$ reduces to $\sigma_2$ in the context of
modules$\Mtwo$ if $\exec{\Mtwo}{\sigma_1}{\sigma_2}$. The semantics in Fig. \ref{f:loo-semantics}
are unsurprising, but it is notable that reads (\textsc{Read}) and writes (\textsc{Write})
are restricted to the class that the field belongs to,
\sdN{and methods  may only be called if public, or from same module as current receiver.}
\begin{figure}[t]
\begin{minipage}{\textwidth}
\begin{minipage}{\textwidth}
\footnotesize
\begin{mathpar}
\infer
	{
	   \sdN{\phi_n}(\prg{contn})  \txteq   u := y_0.m(\overline{y}); s \\
    \meth{\Mtwo}{\class{(\phi,\chi)}{y}}{m} = p \ C::m(\overline{x : T})\red{:T}\{s'\}\\
        	{\sdN{p=\prg{public} \ \vee \ \Same{\prg{this}} {y_0} {\sigma_1}{\Mtwo} }} \\
	\phi'_{n+1}= \sdN{  (\  \prg{this}\, \mapsto\, \interpret{\phi}{y_0},\overline{x\, \mapsto\, \interpret{\phi}{y}}; \ s' \ ) } 
	% \\   \sigma_2 = \sdN{(\overline{\phi}\cdot\phi\cdot\phi'},\chi)\\
	}
	{\exec{\Mtwo}{ ( \sdN{\overline{\phi}\cdot\phi_n},  \chi) }{\sdN{(\overline{\phi}\cdot\phi_n\cdot\phi_{n+1}},\chi)}}
	\quad(\textsc{Call})
	\and
\infer
	{
		\phi_{n+1}(\prg{contn}) \txteq  \red{\epsilon} \\ % )\ \textit{or}\ \phi_1(\prg{contn}) = (\red{z})\\
	\phi_n(\prg{contn})   \txteq  \red{x := y_0.m(\overline y)}; s \\
	\phi'_n= \phi[x \mapsto \interpret{\phi_{n+1}}{\prg{res}}][\prg{contn} \mapsto \red{s} ]\\ 
		}
	{\exec{\Mtwo}{(\sdN{\overline{\phi} \cdot \phi_n \cdot \phi_{n+1}}, \chi) }{(\sdN{\overline{\phi}\cdot  \phi'_n}, \chi)}}
	\quad(\textsc{Return})
	\and
\infer
	{
%	\sigma_1 = (\chi, \phi_1 \cdot\psi) \\
%	\sigma_2 = (\chi, \phi_2 \cdot\psi) \\
	\sdN{\sigma_1(\prg{contn})  \txteq  x := y.f; s} \\
	 \Same {\prg{this}}  {y}  {\sigma_1} {\Mtwo}\\
	%\phi_2 = \{\prg{local}:=\phi_1(\prg{local})[x\ \mapsto\ v],\ \prg{contn}:=s\}
	\sigma_2=\sigma_1[x\mapsto  \interpret{\sigma_1}{y.f} \} ][\prg{cont} \mapsto s ]
	}
	{\exec{\Mtwo}{\sigma_1}{\sigma_2}}
	\quad(\textsc{Read})
	\and
\infer
	{
	\sigma_1 (\prg{contn})  \txteq  x.f := y; s \\
	\Same {\prg{this}}  {x}  {\sigma_1} {\Mtwo}\\
	\sigma_2 = \sigma[\interpret{\sigma_1}{x}.f \mapsto\ \interpret{\sigma_1}{y} ][\prg{contn}\mapsto s]	
	}
	{\exec{\Mtwo}{\sigma_1}{\sigma_2}}
	{}
	\quad(\textsc{Write})
	\and
\infer
	{
	\sigma_1(\prg{contn})\  \txteq\  x := \red{\prg{new}\ C }; s \\
	fields(C)=\overline{f} \\
	\red{\overline v}   \mbox{ initial values for } \overline f\\
	\alpha \mbox{ fresh in } \sigma_1 \\
	% \phi' = \{\prg{local}:=[\prg{this} \mapsto \alpha],\overline{[p_i \mapsto \lfloor z_i \rfloor_{\sigma_1}}], \prg{contn} := s'\}\\
	\sigma_2 = \sigma_1[x\mapsto \alpha][\alpha  \mapsto  (\ C;\  \overline{f\, \mapsto \, v} \  ] [\prg{cont}\mapsto s]
		}
	{\exec{\Mtwo}{\sigma_1}{\sigma_2}}
	\quad(\textsc{New})
\end{mathpar}
\caption{\LangOO operational Semantics}
\label{f:loo-semantics}
\end{minipage}
\begin{minipage}{\textwidth}
\footnotesize
\begin{mathpar}
\infer
		{}
		{\eval{M}{\sigma}{v}{v}}
		\quad(\textsc{E-Val})
		\and
\infer
		{}
		{\eval{M}{\sigma}{x}{\interpret{\sigma}{x}}}
		\quad(\textsc{E-Var})
		\and
\infer
		{
		\eval{M}{\sigma}{e_1}{i_1}\\
		\eval{M}{\sigma}{e_2}{i_2}\\
		i_1 + i_2 = i
		}
		{
		\eval{M}{\sigma}{e_1 + e_2}{i}
		}
		\quad(\textsc{E-Add})
		\and
\infer
		{
		\eval{M}{\sigma}{e_1}{v}\\
		\eval{M}{\sigma}{e_2}{v}
		}
		{
		\eval{M}{\sigma}{e_1 = e_2}{\true}
		}
		\quad(\textsc{E-Eq}_1)
		\and
\infer
		{
		\eval{M}{\sigma}{e_1}{v_1}\\
		\eval{M}{\sigma}{e_2}{v_2}\\
		v_1 \neq\ v_2
		}
		{
		\eval{M}{\sigma}{e_1 = e_2}{\false}
		}
		\quad(\textsc{E-Eq}_2)
		\and
\infer
		{
		\eval{M}{\sigma}{e}{\true}\\
		\eval{M}{\sigma}{e_1}{v}
		}
		{
%		\eval{M}{\sigma}{\ifthenelse{e}{e_1}{e_2}}{v}
		\eval{M}{\sigma}{e}{v}
		}
		\quad(\textsc{E-If}_1)
		\and
\infer
		{
		\eval{M}{\sigma}{e}{\false}\\
		\eval{M}{\sigma}{e_2}{v}
		}
		{
%		\eval{M}{\sigma}{\ifthenelse{e}{e_1}{e_2}}{v}
		\eval{M}{\sigma}{e}{v}
		}
		\quad(\textsc{E-If}_2)
		\and
\infer
		{
		\eval{M}{\sigma}{e}{\alpha}
		}
		{
		\eval{M}{\sigma}{e.f}{\interpret{\sigma}{\alpha.f}}
		}
		\quad(\textsc{E-Field})
		\and
\infer
		{
		\eval{M}{\sigma}{e_1}{\alpha}\\
		\eval{M}{\sigma}{e_2}{v'}\\
		\prg{ghost}\ g(x : T)\{e\} : T'\ \in\ M(\class{\sigma}{\alpha})(\prg{gflds})\\
		\eval{M}{\sigma}{[v'/x]e}{v}
		}
		{
		\eval{M}{\sigma}{e_1.g(e_2)}{v}
		}
		\quad(\textsc{E-Ghost})
\end{mathpar}
\caption{\LangOO expression evaluation}
\label{f:evaluation}
\end{minipage}
\end{minipage}
\end{figure}

While the small-step operational semantics of \LangOO is given in Fig. \ref{f:loo-semantics},
specification satisfaction is defined over an abstracted notion of 
the operational semantics that models the open world. %, called \jm[]{\emph{external states semantics}}. 




An \emph{Initial} program state contains a single frame 
with a single local variable \prg{this} pointing to a single object 
in the heap of class \prg{Object}, and a continuation.
\begin{definition}[Initial Program State]
\label{def:initial}
A program state $\sigma$ is said to be an initial state ($\initial{\sigma}$)
if and only if
\begin{itemize}
\item
$\sigma\ =\  ( ((\prg{this}\ \mapsto\ \alpha), s); \  (\alpha \mapsto (\prg{Object}, \emptyset) 
\end{itemize} 
for some address $\alpha$ and some statement $s$.
\end{definition}


%We give the semantics of module pair execution in Definition \ref{def:pair-reduce}
%\begin{definition}[External State Semantics]
%\label{def:pair-reduce-appendix}
%For all internal modules $M_1$, external modules $M_2$, and program configurations $\sigma$ and $\sigma'$, 
%we say that $\reduction{M_1}{M_2}{\sigma}{\sigma'}$ if and only if
%\begin{itemize}
%\item
%$\class{\sigma}{\sigma(\prg{this})}\ \in\ M_2$ and
%\item
%$\class{\sigma'}{\sigma'(\prg{this})}\ \in\ M_2$ and 
%\end{itemize} 
%and
%\begin{itemize}
%\item
%$\exec{M_1\ \circ\ M_2}{\sigma}{\sigma'}$ or
%\item
%$M_1 \circ M_2, \sigma \leadsto \sigma_1 \leadsto \ldots \sigma_n \leadsto \sigma'$ and $\class{\sigma_i}{\sigma_i(\prg{this})} \in M_1$ for all $1 \leq i \leq\ n$
%\end{itemize}
%\end{definition}

Finally, we provide a semantics for expression evaluation is given in Fig. \ref{f:evaluation}. 
That is, given a module $M$ and a program state $\sigma$, expression $e$ evaluates to $v$
if $\eval{M}{\sigma}{e}{v}$. Note, the evaluation of expressions is separate from the operational
semantics of \LangOO, and thus there is no restriction on field access.



\newpage
