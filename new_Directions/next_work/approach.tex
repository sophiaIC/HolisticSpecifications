

\section{The problem and our approach}
\label{s:outline}
 
In this Section we introduce our problem and outline our approach. 
 We build on, and expand the  example as in OOSPLA-22 and FASE. .. but note that we propose a different approach, and we solve one more challenge than they,
 
  \subsubsection{Object Capabilities and Emergent Behaviour} % of modules}
\label{s:bank}
Our example  demonstrates the need to consider object capabilities as \emph{guards}, and the role
of {modules'} \emph{emergent behaviour} . 
%
in Sect.  \ref{s:approach:necopers} we show three versions of a banking module, $\ModA$, $\ModB$, and $\ModC$.
All three modules allow withdrawal of  money only when supplied with the password of the account.
Moreover, in $\ModA$ the password is immutable, in $\ModB$ it is unconditionally mutable, while in $\ModC$ the password may be modified but only if supplied with the old password. 


{Thus, in all three modules, the password is an object capability which \emph{enables} the withdrawal of the money. 
Moreover, in $\ModA$ and $\ModC$, the password also \emph{guards} the money.
Crucially however,  in $\ModB$ the password \emph{does not guard} the money. 
 Namely, using $\ModB$, it is possible to start in a state where the account's password is unknown, modify the password, and then withdraw the money. 
 This is not possible  in $\ModA$ and $\ModC$.
}

{Guarding is the crucial expectation that comes with object capabilities.\footnote{SAY THIS BETTER?} 
Hoare logics can specify enabling of effects (through per-method PRE/POST-condition pairs), but cannot specify guarding of effects.
The specification of capabilities as guards 
requires per-module guarantees, as proposed, \eg,  in OOPSLA-22, and in the current work.
\footnote{BOLD! Birkedahl and Dreyer might scream!}
%
% \vspace{.1cm}
\noindent
Crucial for such specifications are  two key concepts:
\begin{description}
\item[Capabilities] What are the capabilities (here the password), what effects do the capabilities {guard} (here the money), 
and who is expected or guaranteed not to obtain access to the capability.
\item[Emergent Behaviour] What effects may become possible through the interplay of several methods from a module  (here in ${\ModB}$  
{changing the password and then withdrawing}).
\end{description}}

\noindent
In this paper we address the following challenges
\begin{description}
\item[1st Challenge -- Specification of Object Capabilities as Guards] How to specify that object capabilities are guards over effects while taking the emergent behaviour into account?
\item[2nd Challenge -- Reasoning about external calls] How to verify that calls to external, unknown code, will not cause effects it is   ``not entitled'' to?
\item[3rd Challenge -- Verification of module's adherence to Specification] How to verify that modules adhere to their specification?
\end{description}


\noindent in Sections \ref{s:approach:necopers}, \ref{sec:how}, and \ref{sec:howThird} we give a high level overview of our solution to these challenges.
But before doing that, we make the example more concrete and discuss internal/external modules.

 \subsubsection{More Concretely}
% In the remainder of this paragraph we   explain
We revisit the example in some more detail:
We assume a  small, object oriented language with modules, with module-private fields
(readable/writeable only from other methods declared in the same module),  
and unforgeable and un-enumerable addresses.
Classes have a default constructor, which implicitly initializes all fields {of class type}  with the \prg{null} value,
{and all fields of \prg{int} type to \prg{0}}.
{Methods are module-private, unless defined as \prg{public} -- in this case they may be called by objects which do not belong to the same module, after dynamically checking that the types of the arguments correspond to the formal parameter types.}
   


{%We give three versions of a module which implements bank accounts. All three versions support the creation of accounts, and transfer of funds provided the correct password has been supplied. They only differ in the ability to modify the password. All three versions consists of an 
$\ModA$ consists of an empty class  \prg{Password} whose instances model  unique passwords}, and an \prg{Account} class with fields  password and  balance, an \prg{init} method to initialize the password, and  a \prg{transfer} method. 
%\susan{transfer is such a funny method that people may spend time thinking about it. Either pass the amount or put a footnote about why it is 100.}
%It was so as ensure the signature would fit the line, but we no longer need that, as we split into two figures. THANK you
An \prg{Account} is created with its \prg{balance} set to \prg{0}, and its password set to \prg{null}, \eg by executing \prg{acc = new Account}.
 After that, the password can be set to a new value,  \eg\,  \prg{p=new Password;} \prg{acc.set(p)}. 
It is impossible to modify the  non-\prg{null} password of an account, and thus once a password has been set, the account's {can be reduced only 
by parties with access to the account's password.}
%cannot be ``tampered'' with, unless the password has been supplied.
 

\begin{lstlisting}[mathescape=true, language=Chainmail, frame=lines]
module $\ModA$        
  class Password
  
  class Account
    field balance:int 
    field pwd: Password
    public method transfer(dest:Account, pwd':Password, amt:int) -> void
      if this.pwd==pwd'
        this.balance-=amt
        dest.balance+=amt
     public method set(pwd':Password) -> void
      if this.pwd==null
        this.pwd=pwd'
\end{lstlisting}
%
\noindent 
%We can capture the intended semantics of   \prg{transfer}  
%through  {a}  \funcSpec with pre- and post- conditions and \prg{MODIFIES} clauses as \emph{e.g.,} in \citeauthor{Leavens-etal07,dafny13}.
%The implementation of  \prg{transfer} in module  $\ModA$ meets
%this specification.
%
%\begin{lstlisting}[mathescape=true, frame=lines, language=Chainmail]
%$\Sclassic$  $\triangleq$
%   method transfer(dest:Account, pwd':Password) -> void  
%      ENSURES:
%            this.pwd$=$pwd' $\wedge$ this$\neq$dest  $\longrightarrow$  
%            this.balance$_{post} =$this.balance$_{pre}$-100 $\wedge$ dest.balance$_{post} =$dest.balance$_{pre}$+100
%      ENSURES:
%            this.pwd$\neq$pwd' $\vee$ this$=$dest  $\longrightarrow$ 
%            this.balance$_{post} =$this.balance$_{pre}$ $\wedge$ dest.balance$_{post} =$dest.balance$_{pre}$ 
%      MODIFIES:  this.balance, dest.balance        
%\end{lstlisting}
 
%  \vspace{.1cm}
 
Now consider  module \ModB, which allows any client to reset an account's password at any time.  \ModC, on the other hand  requires the existing password in order to change it.
  
  

\begin{tabular}{lll}
\begin{minipage}[b]{0.42\textwidth}
\begin{lstlisting}[mathescape=true, language=Chainmail, frame=lines]
module $\ModB$
  class Password
  
   class Account
    field balance:int 
    field pwd: Password 
    method transfer(..) ...
      ... as earlier ...
     public method set(pwd': Password)
      this.pwd=pwd'
      $\ \ \ $
\end{lstlisting}
\end{minipage}
&\ \ \  \ \   &%
\begin{minipage}[b]{0.45\textwidth}
\begin{lstlisting}[mathescape=true, language=chainmail, frame=lines]
module $\ModC$
  class Password

  class Account
    field balance:int 
    field pwd: Password 
    public method transfer(..) 
      ... as earlier ...
    public method set(pwd',pwd'': Password)
      if (this.pwd==pwd') 
        this.pwd=pwd''
\end{lstlisting}
\end{minipage} 
\end{tabular}

\noindent
Although the \prg{transfer} method is the same in
all three modules,   % and each one satisfies \Sclassic,
code  {such as}
\\ 
$\ \strut \hspace{.2in} $ \prg{p=new Password;  acc.set(p); acc.transfer(rogue\_account,p,1000)} 
\\ 
is enough to drain  \prg{acc} in \ModB without knowing the password.\footnote{CAREFUL: we had 
$\ \strut \hspace{.01in} $ \prg{an\_account.set(42); an\_account.transfer(rogue\_account,42)} but this was type incorrect!}
%
%This example demonstrates that we need to consider the 
\emph{Emergent behaviour} is key here: 
Even though the method \prg{transfer} in  \ModB is ``safe'' when considered in isolation, it is not safe when considered in conjunction with other methods from the same module. 
% We are therefore looking for a way to specify modules' emergent behaviour,  
%\susan{Do you want to say something about not possible to capture emergent behaviour using Hoare logic pre/post conditions that hold for individual methods? 
%{Good idea! I did it earlier on; Does it work?}}
   
\subsubsection{Internal and external modules, objects, and states}
 \label{s:concepts}

We are concerned with guarantees made in an \emph{open} setting; that is, a given module
$M$ must be programmed so that 
execution of $M$ together with \emph{any} number of external modules $\overline M$
will uphold these guarantees. % In the tradition of
%visible states semantics, we are  only interested in 
These guarantees must be upheld while 
 $\overline M$, the  \emph{\externalM} modules, are executing; in contrast, the internal module may 
temporarily break them,
so long they %are it 
hold {upon entry to external modules.}\footnote{Not quite true. How improve?} 
 
    
We therefore distinguish between  \emph{\internalO}
objects --- instances of classes defined in $M$ ---
and \emph{\externalO} objects defined in any other module.
We also distinguish between
  \emph{\internalC} and   \emph{\externalC} states - those whose receiver is internal or external respectively. 
 
  % % from \externalC objects
%  to \externalC objects. 
%{Looking at the code snippet from \S \ref{sec:how}, the call to \prg{set} on line 6 is an 
% internal call, while the call to \prg{make\_payment} is an external call -- from the external 
% object  \prg{this}  to the external object \prg{untrusted}.}
 
% % We are less
%% interested in calls made from \internalO objects to \internalO objects, because we only need 
%% establish the guarantees when the \externalM module is executing. And we 
%%
%%
%Because we only require guarantees while 
%the  \externalM module  is executing,
%we develop an \emph{external states} semantics, where
% any internal calls are executed in one, large, step.
%With external steps semantics,  the executing object (\prg{this}) is always   external. 
%  In line  with other work in the literature \cite{Permenev, Grossman, Albert}, we currently forbid 
%  calls from internal to  external objects
%  -- further details on call-backs in \S\ref{s:related}. 
%
%{For the purposes of the current work we are only interested in one internal, and one external module.
%But the interested reader might ask: what if there is more than one external module?
%The answer is that from the internal module's viewpoint, all external modules are considered as one;
%for this we provide a module linking operator with  the expected semantics -- more details in Def. \ref{def:pair-reduce} and \S \ref{app:loo}. 
%But from the external module's viewpoint, there may be more than one external module: for example, in \S \ref{sec:how}, 
%module $\ModParam{1}$ is external to the module   implementing class \prg{Account}, and the module 
%implementing the class of \prg{untrusted} is external to
%$\ModParam{1}$.
%}




%\fbox{\parbox{5.5in}{{\textbf{1st Challenge} We are looking for a specification language that expresses eventual access to capabilties, as well as the the emergent behaviour of modules.}}}% [14cm]{text in the box}}

%\vspace{.1cm}



\subsection{First Challenge: Specifications of Object Capabilities as Guards}
\label{s:approach:necopers}

%\susan{it sounds like capabilities and emergent behaviour are
%different things, perhaps Specifications for Capabilities including
%Emergent behaviour - or something similar. {Any better now?}}

Our specifications should describe {which effects are guarded by which capabilities}. 
In particular, lack of access to {a capability  should} guarantee  that {the effect guarded by that capability will not happen.}
  {These guarantees should hold  in the context of   emergent behaviour.}   
For this, we propose two %a specification language for modules, which has two main novel ingredients.
{novel concepts}:

 
\begin{description}
\item[Protection] to describe a lack of access to capabilities: An object $o$ is \emph{protected} from $o'$, formally $\protectedFrom {o} {o'}$,  if $o$ cannot obtain direct access to $o'$ unless an internal object affords it access. %to $o$. 
And $o$ is \emph{protected}, formally ${\inside{\prg{\it{o}}}}$, if $o$ is protected from all external objects transitively accessible from the currently executing method call. More in Def. 3.5, 4.5 and 4.6.
\item[Bounded, external, two-state invariants] to describe emergent behaviour:  $\TwoStatesQ {\overline{x:C}}  {A}{A'}$ expresses that if a reachable state $\sigma$ 
  satisfies ${\overline{x:C}}  \wedge A$, then all $\sigma$'s \emph{bounded future} states will satisfy $A'$ -- more in Def. 5.3. The concept of \emph{bounded} future includes further method calls and returns, but excludes returning from the call active in $\sigma$ -- more in Def 3.1. 
We consider executions of the internal module linked with any number of unknown, external modules -- cf Sect. 2.2.
 For $\sigma$ and its bounded future states  we only consider states whose receiver is \emph{external}. 
\end{description}

\kjx{not clear why these are ``two-state'' invariants - rather than
  $1+N$ states? \red{SD: ``two-state'' does bot mean that the states are immediately consecutive. Can you propose a better name}}
\kjx{seems $\protectedFrom {o} {o'}$ means $\exists i intl : \textrm{dominatesAllPathsFrom}(o)\textrm{To}(o')$ or something;
ditto  ${\inside{\prg{\it{o}}}}$ means 
    $\exists i intl : \textrm{dominatesAllPathsFrom}(topOfStack)\textrm{To}(o')$.
Not that I'm biased or anything.
 \red{SD: cI do not think so. We are asking for the penultimate element in the path. For example, if there is a path 1->2->3->4, and 2 is internal, and 3 is external, than your definition would say  that 4 is protected from 1, but with the def as given here it is not.
 \blue{SD:If we reversed the arrows, then indeed we could say that $\protectedFrom {o} {o'}$ iff the \textbf{set} of internal objects dominates all \textbf{reversed} paths from $o$ to $o'$. 
 Woulds that be more easily understandable than the current definition?
}}}

With these concepts we can now write specifications which  rule  out \ModB while permitting \ModA and
\ModC, namely:
\label{s:bankSpecEx}

  
 \begin{tabular}{lcll}
 $S_1$   & $\triangleq$   &  $\TwoStatesQ {\prg{a}:\prg{Account}}  {\inside{\prg{a}}} {\inside{\prg{a}}}$
 \\
 $S_2$   & $\triangleq$   &  $\TwoStatesQ {\prg{a}:\prg{Account}}  {\inside{\prg{a.password}}} {\inside{\prg{a.password}}}$
 \\
 $S_3$ & $\triangleq$   &
 $ \TwoStatesQ{\prg{a}:\prg{Account},\prg{b}:\prg{int}}  {\inside{\prg{a}} \wedge \prg{a.balance}=\prg{b}} 
 { \prg{a.balance}=\prg{b} }$
\\
$S_4$ & $\triangleq$   &
  $\forall \prg{a}:\prg{Account},\prg{b}:\prg{int}.$\\
  &  &  $\FirstState{\inside{\prg{a.password}} \wedge \prg{a.balance}=\prg{b}} 
\  \SecondState{ \prg{a.balance}\geq \prg{b} }$
 \end{tabular}




\noindent
 $S_1$ guarantees that the module does not leak accounts, 
 $S_2$ guarantees that the module does not leak passwords,
 $S_3$ guarantees that the balance is not modified unless there is unprotected access to the account, while 
 $S_4$ guarantees that the balance does not decrease unless there is unprotected access to the password.
 %
 All three modules satisfy $S_1$ and $S_3$. $\ModA$ and $\ModC$ also satisfy $S_2$ and $S_4$, but $\ModB$ does not satisfy $S_2$ nor $S_4$.
 
 
 \noindent
\textbf{NOTES} \notesep  Our specifications are expressed in terms of observable effects (\eg the password stays protected, the account's balance may increase), rather than in terms of individual methods (\eg\, \prg{set} or \prg{transfer}).
{This % gives our specifications the the vital advantage that our specifications can % be used to constrain
implies that they can charatcterize  any 
module with bank accounts which have a % {\textit{implementation}} of a bank account with a 
 balance and a password -- even as ghost fields --}, irrespective of the API offered, services  exported, or  dependencies on other parts of the system.\footnoteSD{does this come from OOPSLA? if so we need to rephrase}
\notesep
{Adherence to   specifications is not monotonic:
Adding a method to a module does not necessarily preserve adherence to
a specification, and while separate methods may adhere to a  specification, their combination does
not necessarily do so. 
{For example, \ModB's  \prg{tansfer} and \prg{set} satisfy $S_4$, but their interplay does not.}
%In this sense, and, similar to OOPSLA'22, our  specifications capture a module's \emph{emergent behaviour}. 
}

 \subsection{Second Challenge: Reasoning about external calls}
\label{sec:how}

{We now consider how we can reason about external method calls. 
The challenge, when calling a method on an external object, is that we have no specification for that method. 
However, as argued in the intro\footnoteSD{we need to indeed argue that in the intro}, we can use \susan{our} module's emergent behaviour  specification   in order to restrict the possible side-effects of the external call.
}

 
 
Consider that {our internal module included the following class} \prg{Shop}. It exports a public method \prg{buy} whose formal parameter \prg{buyer} is an \prg{Object} and thus may be an external object. 
The critical point is line 7, where we call a method on \prg{buyer}, and pass them the shop's account.
Is it possible that the buyer will use this opportunity  to drain the shop's account?
 




%{One might think that \SrobustB was not useful: normally, there will exist somewhere in the heap
%at least one external object  
%with access to the password --  if no such object existed, then {nobody} would be able to use the money of
%the account.
%And if such an object did exist, {then the premise of \SrobustB would not hold, and thus}
%the guarantee given by \SrobustB might seem vacuous.}
%
%{
%This is \emph{not} so: %there may exist  
%{in scopes   from which such external objects with access to the password
%are not (transitively) reachable, % . In such scopes,
%\SrobustB  guarantees that the balance of the account will not decrease.
%}
%We illustrate this through the following  code snippet:
 

\begin{lstlisting}
class Shop
  field myAccount : Account
  field inventory : Inventory    
  public method buy(buyer: Object, anItem: Item) -> void
    int price = anItem.price
    int oldBalance = this.myAccount.balance
    buyer.payMe(myAccount,price)
    if (this.myAccount.balance == oldBalance+price)
       this.send(buyer,anItem)
    else
       buyer.tell("you have not paid me") 
    public method send(buyer: Object, anItem: Item) -> void  
       ...            
\end{lstlisting}
 
\noindent
 
If the \prg{Account} comes from a module that satisfies $S_4$, and if at the point of the call to \prg{buy} -- \ie at line 4 -- the password was protected from the \prg{buyer}, then it is guaranteed that after the external call -- \ie after line 7 --  the shop's account's balance will not have decreased.
{The corresponding Hoare logic rule, which applying  $S_4$, would verifiy line 7, appears in Fig. 10.}

 

\vspace{.1cm}

  \noindent
 \textbf{NOTES}
%\susan{You've never said how a holistic spec is used.{Does the para above not use the holistic spec?}}
 \notesep We   applied $S_4$ at the point of call in line 7, even   though  
 the pre-state in $S_4$ requires % r  is  in terms 
 a \emph{stronger} property than the one we have.
 Namely,   $S_4$ requires   the password to be protected, while we only know 
  that the password is protected from the  \prg{buyer}.
It is safe to apply $S_4$,  because from the point of view of the callee at line 7 the password is indeed protected -- \cf Fig, 5,   Lemma 4.7, and the 
  Hoare logic rule in Fig.10.
 %,  and in the callee's \emph{bounded} future states,  the password is protected.
  \notesep  If at the point of call of \prg{buy} the password was not protected from the \prg{buyer}, then, of course, no guarantee can be made about the account's balance.
 \notesep Requiring that the account's balance was protected from all external objects would be far too strong; if that were the case, then the account would not be useful for any payments.
 \notesep Requiring that the account's balance was protected at the time of the call of \prg{buy} would also be far too strong; the module cannot impose restrictions on the callers' external state  upon the call to \prg{buy}.
 \notesep 
 We could specify the method \prg{buy} through the PRE/POST pair: $PRE: \protectedFrom {\prg{a.password}} {\prg{buyer}} \wedge {\prg{a.balance}}=\prg{b} -- POST: \prg{a.balance}\geq \prg{b}$. However this PRE/POST pair is a direct conclusion of $S_4$; thus, as \prg{Shop} is part of our internal module, the PRE/POST pair is superfluous.
 
 
 \subsection{Third Challenge: Proving that a module adheres to its specification}
 \label{sec:howThird}
 
{In order to prove that a module adheres to its specification, we need to prove that all  private methods satisfy their  PRE/POST post-conditions, and that all public methods satisfy the module's two-state invariants combined with their PRE/POST conditions -- if any.
This means that we need Hoare logic rules to reason about, among other things, protection.}

{For our work, we assume an underlying sound Hoare logic for \susan{our} language, and extend it with rules about protection, external calls, and the module's adherence to its 
specification - more in Fig. 7-11.
We prove that the extended Hoare logic is sound.
}

{Notice that while we are concerned with \emph{necessary conditions for effects}, we prove adherence to \emph{sufficient conditions which guarantee the absence of effects.}
We have therefore taken a similar path to .... TODO ...
}

 {The proof that the extended Hoare logic is sound is interesting, because we are arguing about the soundness of two interrelated systems: 
 the per-statement  Hoare logic, as well as the module's logic.
Moreover, we need to cater for the possibility that external calls eventually call public methods of the module, which in their turn make external calls etc.
For this we define a new measure of execution ...}


 
 %Because we only require guarantees while 
%the  \externalM module  is executing,
%we develop an \emph{external states} semantics, where
% any internal calls are executed in one, large, step.
%With external steps semantics,  the executing object (\prg{this}) is always   external. 
%  In line  with other work in the literature \cite{Permenev, Grossman, Albert}, we currently forbid 
%  calls from internal to  external objects
%  -- further details on call-backs in \S\ref{s:related}. 
%
%{For the purposes of the current work we are only interested in one internal, and one external module.
%But the interested reader might ask: what if there is more than one external module?
%The answer is that from the internal module's viewpoint, all external modules are considered as one;
%for this we provide a module linking operator with  the expected semantics -- more details in Def. \ref{def:pair-reduce} and \S \ref{app:loo}. 
%But from the external module's viewpoint, there may be more than one external module: for example, in \S \ref{sec:how}, 
%module $\ModParam{1}$ is external to the module   implementing class \prg{Account}, and the module 
%implementing the class of \prg{untrusted} is external to
%$\ModParam{1}$.
%}

 

\subsection{Our Contributions}

Our contributions are

\begin{enumerate}
\item
New capability assertions, $\protectedFrom {x} {y}$  and $\inside{x}$ 
\item
A new specification language for emergent behaviour  
\item
A Hoare logic extension, which handles external calls
\item
A logic which proves adherence to our specification language
\item
Proofs, and a worked out example
\end{enumerate}

 
