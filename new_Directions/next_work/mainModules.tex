%% For double-blind review submission, w/o CCS and ACM Reference (max submission space)
%\documentclass[acmsmall,review]{acmart}\settopmatter{printfolios=true,printccs=false,printacmref=false}
%% For double-blind review submission, w/ CCS and ACM Reference
\documentclass[acmsmall,review,anonymous,screen]{acmart}\settopmatter{printfolios=true,printacmref=false}
%% For single-blind review submission, w/o CCS and ACM Reference (max submission space)
%\documentclass[acmsmall,review]{acmart}\settopmatter{printfolios=true,printccs=false,printacmref=false}
%% For single-blind review submission, w/ CCS and ACM Reference
%\documentclass[acmsmall,review]{acmart}\settopmatter{printfolios=true}
%% For final camera-ready submission, w/ required CCS and ACM Reference
%\documentclass[acmsmall]{acmart}\settopmatter{}
\usepackage[shortlabels]{enumitem}
\usepackage{mathtools}
\usepackage{wrapfig}

\DeclareSymbolFont{arrows}{U}{FdSymbolC}{m}{n}

% \usepackage{ amssymb }
\DeclareTextFontCommand{\texttt}{\ttfamily}

%% Journal information
%% Supplied to authors by publisher for camera-ready submission;
%% use defaults for review submission.
%\acmJournal{PACMPL}
%\acmVolume{}
%\acmNumber{OOPSLA}  ONF = POPL or ICFP or OOPSLA
%\acmArticle{}
%\acmYear{2025}
%\acmMonth{1}
%\acmDOI{} % \acmDOI{10.1145/nnnnnnn.nnnnnnn}
%\startPage{1}

% opyright information
%% Supplied to authors (based on authors' rights management selection;
%% see authors.acm.org) by publisher for camera-ready submission;
%% use 'none' for review submission.
\setcopyright{none}
%\setcopyright{acmcopyright}
%\setcopyright{acmlicensed}
%\setcopyright{rightsretained}
%\copyrightyear{2018}           %% If different from \acmYear

%% Bibliography style
\bibliographystyle{ACM-Reference-Format}
% itation style
%% Note: author/year citations are required for papers published as an
%% issue of PACMPL.
%\citestyle{acmauthoryear}   %% For author/year citations
\citestyle{acmnumeric}

%%%%%%%%%%%%%%%%%%%%%%%%%%%%%%%%%%%%%%%%%%%%%%%%%%%%%%%%%%%%%%%%%%%%%%
%% Note: Authors migrating a paper from PACMPL format to traditional
%% SIGPLAN proceedings format must update the '\documentclass' and
%% topmatter commands above; see 'acmart-sigplanproc-template.tex'.
%%%%%%%%%%%%%%%%%%%%%%%%%%%%%%%%%%%%%%%%%%%%%%%%%%%%%%%%%%%%%%%%%%%%%%


%% Some recommended packages.
\usepackage{booktabs}   %% For formal tables:
                        %% http://ctan.org/pkg/booktabs
\usepackage{subcaption} %% For complex figures with subfigures/subcaptions
                        %% http://ctan.org/pkg/subcaption
 \usepackage{ stmaryrd }                       

\usepackage{relsize}
\usepackage{mathpartir}
\usepackage{amsmath}
\usepackage{amsthm}
\usepackage{listings}
\usepackage{xspace}
\usepackage{definitions}
\usepackage{multirow,bigdelim}
\usepackage{pbox}
\usepackage{courier}
\usepackage{soul}
\usepackage{centernot}
 

\newcommand\multibrace[3]{\rdelim\}{#1}{3mm}[\pbox{#2}{#3}]}


%%these COLOUR MACROS ARE ACTIVELY FUCKING EVIL
%%DO NOT DO THIS. EVER
%%AT LEast have some ovious way to TURN THEM OFF

\definecolor{ferngreen}{rgb}{0.31, 0.47, 0.26}
\newcommand{\kjx}[1]{{}}
\newcommand{\scd}[1]{{}}
%\newcommand{\sdN}[1]{{\color{dkgreen}{#1}}}
%\newcommand{\jm}[1]{{\color{magenta}{JM: #1}}}
\newcommand{\sdcomment}[1]{{}}
\newcommand{\secomment}[1]{{}}
\newcommand{\jncomment}[1]{{}}

\newcommand{\sd}[1]{{{#1}}}
\newcommand{\se}[1] {#1} %{{\color{brown}{#1}}}
\newcommand{\sue}[1] {{\color{brown}{#1}}}
\definecolor{amber}{rgb}{1.0, 0.75, 0.0}
\definecolor{amethyst}{rgb}{0.6, 0.4, 0.8}

\newcommand{\ponders}[3]{\marginpar{\tiny\itshape\raggedright\textcolor{#2}{\textbf{#1:} #3}}\ignorespaces}
\renewcommand{\ponders}[3]{{#3\ignorespaces}}  %TURNING IT OFF. NOT THAT IT FUCKING HEPLS

\marginparwidth=1.6cm \marginparsep=0cm
\newcommand{\TODO}[1]{} % {{\color{red}#1}}
\newcommand{\sophia}[1]{{#1}}
\newcommand{\toby}[1]{} % {\ponders{Toby}{purple}{#1}}
\newcommand{\susan}[2][]{{#2}\xspace}
\newcommand{\james}[1]{\ponders{James}{orange}{#1}}
\newcommand{\jm}[2][]{\ponders{Julian}{magenta}{#1} {#2}\xspace}
\newcommand{\mrr}[2][]{\ponders{Matthew Ross}{offblue}{{#1}} {{#2}}\xspace}
\newcommand{\sdNr}[2][]{\ponders{SD:}{amber}{#1} {#2}\xspace}
\newcommand{\sdN}[1]{{{#1}}}
\newcommand{\sdNO}[1]{\red{ {#1}}} %{\ponders{SD:}{amber}{ } {#1}\xspace}
\newcommand{\julian}[1]{{#1}\xspace}
\newcommand{\sdM}[1]{#1\xspace}

\newcommand{\sophiaPonder}[2][]{\ponders{Sophia}{blue}{#1}{#2}\xspace}
\renewcommand{\sophia}[2][]{}

\newcommand{\sdfootnote}[1]{}



\begin{document}

%% Title information
%\title[Specification and Proof of Necessary Conditions]{Specification
%and Proof of Necessary Conditions}         %% [Short Title] is
%optional;
%\title{Necessity Specifications are Necessary
%\title{Reasoning about External Calls with Limited Effects}
\title{Modules for Chainmail}
%\title{Limiting the Effects of External Calls}
%\title{Limited Effects for Reasoning about External Calls}
%\title{Reasoning about External Calls}

% \title{\Nec Specifications are Necessary for Robustness}

                                        %% when present, will be used in
                                        %% header instead of Full Title.
%\titlenote{with title note}             %% \titlenote is optional;
                                        % an be repeated if necessary;
                                        % ontents suppressed with 'anonymous'
%\subtitle{Subtitle}                     %% \subtitle is optional
%\subtitlenote{with subtitle note}       %% \subtitlenote is optional;
                                        % an be repeated if necessary;
                                        % ontents suppressed with 'anonymous'


%% Author information
% ontents and number of authors suppressed with 'anonymous'.
%% Each author should be introduced by \author, followed by
%% \authornote (optional), \orcid (optional), \affiliation, and
%% \email.
%% An author may have multiple affiliations and/or emails; repeat the
%% appropriate command.
%% Many elements are not rendered, and should be provided for metadata
%% extraction tools.


%% Author with two affiliations and emails.
\author{Julian Mackay}
%\authornote{with author2 note}          %% \authornote is optional;
                                        % an be repeated if necessary
\orcid{0000-0003-3098-3901}             %% \orcid is optional
\affiliation{
  %\position{Position2a}
  %\department{Engineering and Computer Science}             %% \department is recommended
  \institution{Victoria University of Wellington}           %% \institution is required
  %\streetaddress{Street2a Address2a}
  %\city{City2a}
  %\state{State2a}
  %\postcode{Post-Code2a}
  \country{New Zealand}                   %% \country is recommended
}
\email{julian.mackay@ecs.vuw.ac.nz}         %% \email is recommended

%% Author with single affiliation.
\author{Sophia Drossopoulou}
%\authornote{with author1 note}          %% \authornote is optional;
                                        % an be repeated if necessary
\orcid{0000-0002-1993-1142}             %% \orcid is optional
\affiliation{
  %\position{Position1}
  %\department{Department1}              %% \department is recommended
  \institution{Imperial College London}            %% \institution is required
  %\streetaddress{Street1 Address1}
  %\city{City1}
  %\state{State1}
  %\postcode{Post-Code1}
  \country{United Kingdom}                    %% \country is recommended
}
\email{scd@imperial.ac.uk}          %% \email is recommended

\author{James Noble}
%\authornote{with author2 note}          %% \authornote is optional;
                                        % an be repeated if necessary
\orcid{0000-0001-9036-5692}             %% \orcid is optional
\affiliation{
  %\position{Position2a}
  %\department{Department2a}             %% \department is recommended
  \institution{Creative Research \& Programming}           %% \institution is required
  \streetaddress{5 Fernlea Ave, Darkest Karori}
  \city{Wellington}
  %\state{State2a}
  \postcode{6012}
  \country{New Zealand}                   %% \country is recommended
}
\email{kjx@acm.org}         %% \email is recommended


\author{Susan Eisenbach}
%\authornote{with author2 note}          %% \authornote is optional;
                                        % an be repeated if necessary
\orcid{0000-0001-9072-6689}             %% \orcid is optional
\affiliation{
  %\position{Position2a}
  %\department{}             %% \department is recommended
  \institution{Imperial College London}           %% \institution is required
  %\streetaddress{Street2a Address2a}
  %\city{City2a}
  %\state{State2a}
  %\postcode{Post-Code2a}
  \country{United Kingdom}                   %% \country is recommended
}
\email{susan@imperial.ac.uk}         %% \email is recommended



%% Abstract
%% Note: \begin{abstract}...\end{abstract} environment must come
%% before \maketitle command


  

%% 2012 ACM Computing Classification System (CSS) concepts
%% Generate at 'http://dl.acm.org/ccs/ccs.cfm'.
\begin{CCSXML}
<ccs2012>
   <concept>
       <concept_id>10011007.10010940.10010992.10010993.10011683</concept_id>
       <concept_desc>Software and its engineering~Access protection</concept_desc>
       <concept_significance>300</concept_significance>
       </concept>
   <concept>
       <concept_id>10011007.10011074.10011099.10011692</concept_id>
       <concept_desc>Software and its engineering~Formal software verification</concept_desc>
       <concept_significance>500</concept_significance>
       </concept>
   <concept>
       <concept_id>10003752.10003790.10011741</concept_id>
       <concept_desc>Theory of computation~Hoare logic</concept_desc>
       <concept_significance>500</concept_significance>
       </concept>
   <concept>
       <concept_id>10011007.10011006.10011008.10011009.10011011</concept_id>
       <concept_desc>Software and its engineering~Object oriented languages</concept_desc>
       <concept_significance>300</concept_significance>
       </concept>
 </ccs2012>
\end{CCSXML}

\ccsdesc[300]{Software and its engineering~Access protection}
\ccsdesc[500]{Software and its engineering~Formal software verification}
\ccsdesc[500]{Theory of computation~Hoare logic}
\ccsdesc[300]{Object oriented programming~Object capabilities}



%% End of generated code

  
%% End of generated code


%% Keywords
% omma separated list
%%%%%%%%%%%%%%%%%%%%%\keywords{keyword1, keyword2, keyword3}  %% \keywords are mandatory in final camera-ready submission



%% Note: \maketitle command must come after title commands, author
% ommands, abstract environment, Computing Classification System
%% environment and commands, and keywords command.

\newcommand{\MCheck}[3]{\ensuremath{{#3} \, \vdash\, {#1}: {#2}} }% {{\ensuremath #1 \vdash #2 \prg{using} #3}}

\begin{document}
 
\begin{abstract}
Investigations into how much checking is needed when checking Chainmail modules' adherence to specifications.
And how local the checks will be


\end{abstract}


\maketitle 

\section{Notation}

$~ \ \ \  $ Assume modules $M_1$, $M_2$, ... with holistic specifications $S_1$, $S_2$, ... 

Use the notation 

\begin{itemize}
\item
$M \circ M'$ \ \ \ \  for the combination of two modules
\item
$S \wedge S'$ \ \ \ \  for the conjunction of two specifications
\item
$\MCheck {M} {S} {S'}$ \ \ \ for checking that module $M$ satisfies spec $S$. Do this  using the specs $S; S'$,
\item
$M \models S$ \ \ \  means that $M$ semantically satisfies $S$
\end{itemize}


\section{Intuitions}

How much do we need to check when combining modules? 
Sophia thinks that the following holds:

\vspace{.2cm}

\begin{center}
\begin{tabular}{||l|l||}
  \hline \hline  %  A
Assuming that ... & The checks below ensure $M_1 \circ M_2 \models S_1 \wedge S_2 $ \\
  \hline \hline 
 $M_1$ does not call outside $M_1$    & $\MCheck {M_1} {S_1} {\epsilon}$ \\
 $M_2$ does not call outside $M_2$  &  $\MCheck {M_2} {S_2} {\epsilon}$ \\
  \hline    
   
 $M_1$ does not call outside $M_1$    & $\MCheck {M_1} {S_1} {\epsilon}$ \\
 $M_2$ calls into $M_1$ and makes \emph{no} external calls  &  $\MCheck {M_2} {S_2} {S_1}$ \\
  \hline    
  
 $M_1$ does not call into $M_2$, and makes external calls    & $\MCheck {M_1} {S_1} {\epsilon}$ \\
 $M_2$ does not call into $M_1$, and makes external calls  &  $\MCheck {M_2} {S_2} {\epsilon}$ \\
  \hline    
 $M_1$ does not call into $M_2$, and makes external calls    & $\MCheck {M_1} {S_1} {\epsilon}$ \\
 $M_2$ calls into $M_1$, and makes external calls  &  $\MCheck {M_2} {S_1 \wedge S_2} {S_1}$ \\
\hline    
 $M_1$ calls into $M_2$, and makes external calls    & $\MCheck {M_1} {S_1 \wedge S_2} {S_1}$ \\
 $M_2$ calls into $M_1$, and makes external calls  &  $\MCheck {M_2} {S_1 \wedge S_2} {S_2}$ \\
  \hline \hline 

\end{tabular}
 \end{center}

\end{document}
