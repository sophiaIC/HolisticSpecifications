Software no longer stands alone \cite{swapsies}. Contemporary digital
watches have around ten times the processing power, memory, and
storage --- and a thousand times the network bandwidth --- of the
mobile phones or pocket computers first introduced twenty years ago.
Watches, phones, laptops, and larger computer systems are
intrinsically part of networks of software systems at all scales, from
the personal to the global.  The usefulness and productivity of these
devices rely critically on their being part of this open ecosystem ---
but the very openness of the ecosystem exposes every participant to
attacks from any other participant. Attacks on seemingly
inconsequential devices (wireless earbuds, say) can still cause
injuries and death (e.g.\ distracting a driver or pedestrian at
particularly ill-chosen moments).

This paper advances the state-of-the-art in providing protection
against such massive attacks \cite{massive-attack-protection}.  We
extend classical software verification techniques to  this
``open world'', demonstrating how to verify properties of software
that must interact with unverified, untrusted, unknown, or even
antagonistic contexts; and where control and data must flow from untrusted
to trusted components, from trusted to untrusted components, and then
back into the trusted components. We work within a framework of Hoare
logics and object capabilities, enabling a systematic approach to
verification, rather than specialised proofs constructed \textit{ab
  initio} for each
problem.  We expect our techniques can be incorporated into existing
program verification tools, thus paving the way towards the
verification of practical systems systems in the open world.




%% current apple watch 2024 
%% 64G storage
%% 18000Mhz  - 1.8Ghz 
%% 1G RAM
%% 2 main 64bit cores 
%% 100Mbits  netowrk
%%
%% first apple watch  2015
%% 512M dram
%% 8G storage
%% 500MhZ clock
%%
%% first iPhone 2007
%% 128 M Dram
%% 4/8/6G 
%% 400MhZ clock
%% 1 main core 32 bit
%% 100 kbits netowrk
