\newcommand{\paragraphSDD}[1]{\vspace{.01cm}{\textit{#1}}}

\subsection{Expressiveness} 

In   \S \ref{app:expressivity} we argue the expressiveness of our approach by comparing with example specifications  proposed in \cite{OOPSLA22,dd,VerX}.
We   summarize here.

 %% We continue the comparison of expresiveness between \emph{Chainmail} and \Nec, by 
 %% considering the examples studied in \cite{FASE}.
 
%\begin{example}[ERC20]

\paragraphSDD{DOM} is the motivating example  in \cite{dd}: Access to any DOM node
gives read/write  permissions to  all its \prg{parent} and \prg{children} nodes. 
These permissions are attenuated   through a \prg{Proxy} class, %which has a field \prg{node} pointing to a \prg{Node}, and a field \prg{height}, 
 which restricts the range of \prg{Node}s which may be modified through the use of the particular \prg{Proxy}. 
We can express such  attenuation   through two scoped invariants.
% The corresponding specification in \ref{OOPSLA22} is comparable, but not able to prove external calls. % not specific as to the frame from which any modification originated.

\paragraphSDD{DAO} %Decentralized Autonomous Organization\
 ~\cite{Dao}  is a well-known Ethereum contract   which was exploited with a re-entrancy bug in 2016, 
and lost \$50M. 
Our two state invariants  would have secured the DAO against that bug. % such a  bug. 
But note  that they are about precluded effects, and 
%. They are, essentially, simple object invariants and 
 could have been expressed with the techniques proposed already by \cite{MeyerDBC92}.
% \cite{OOPSLA22}  gives one  further specification, which says  that any reduction of funds can only be caused through a call to a specific method -- such specifications are beoind our scope.
 
 \paragraphSDD{ERC20} is a widely used % token 
 standard describing  basic functionality of Ethereum-based token 
contracts. 
The Solidity security model in not based on access to  capabilities but on who the caller  is. 
We  adapted our approach correspondingly, and 
%we change the meaning of $\inside{\re}$ to express that $\re$ did not make a method call.
%and we introduce aslightly modified form of two state invariants:  $\TwoStates{\overline {x:C}}{A}{A'}$   expresses that any execution which satisfies $A$, will preserve $A'$.
%
express 1) that  the owner of an account is always authorized on that account,  2) any execution which does not contain calls from a participant  authorized on some account will not affect the balance nor  who is authorized on  that account. 
% The specifications from \cite{OOPSLA22} are more API-specific, in that they pinpoint which method calls caused an effect, and less specific in that they do not pinpoint the frame from which the effect occurred. 