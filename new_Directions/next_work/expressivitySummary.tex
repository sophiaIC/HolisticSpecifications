\newcommand{\paragraphSDD}[1]{\vspace{.02cm}{\textit{#1}}}
 
\subsection{Expressiveness} 

We argue the expresseness of our approach  through a sequence of capability patterns studied in related approaches from the literature  
 \cite{OOPSLA22,dd,VerX,irisWasm23} written in our specification language.
These approaches %in  \cite{OOPSLA22,dd,VerX,irisWasm23}  
 are based on temporal logics \cite{VerX,OOPSLA22}, or on extensions of Coq/Iris \cite{dd,irisWasm23}, and
none offer a Hoare logic    for external calls.
%Other approaches in the literature are either unable to handle external method calls \cite{OOPSLA22}, or use  bespoke proofs \cite{dd,irisWasm23}, or model checking \cite{VerX}.
%Their specification languages are based on temporal logics \cite{VerX,OOPSLA22}, or on extensions of Coq and Iris \cite{dd,IrisWasm23}.
More in  \S \ref{app:expressivity}. 
% we argue our approach is able to prove comparable specifications to those proposed in  \cite{OOPSLA22,dd,VerX,irisWasm23}, in the presence of external method calls, using a Hoare logic.=≠≠≠q±±≠≠≠qq≠q≠≠www≠≠
We   summarize here.

 %% We continue the comparison of expresiveness between \emph{Chainmail} and \Nec, by 
 %% considering the examples studied in \cite{FASE}.
 
%\begin{example}[ERC20]

\paragraphSDD{DOM} % is the motivating example  in \cite{dd}:
Access to any DOM node
gives read/write  permissions to  all its \prg{parent} and \prg{children} nodes. 
These permissions are attenuated   through a \prg{Proxy} class, %which has a field \prg{node} pointing to a \prg{Node}, and a field \prg{height}, 
 which restricts the range of \prg{Node}s which may be modified through the use of the particular \prg{Proxy}. 
We  express such  attenuation   through two scoped invariants.
% The corresponding specification in \ref{OOPSLA22} is comparable, but not able to prove external calls. % not specific as to the frame from which any modification originated.

\paragraphSDD{DAO} %Decentralized Autonomous Organization\
 ~\cite{Dao}  is a well-known Ethereum contract   which was exploited with a re-entrancy bug in 2016, 
and lost \$50M. 
Our two state invariants  would have secured %the DAO 
 against that bug. % such a  bug. 
But note  that  they are about precluded effects, and 
%. They are, essentially, simple object invariants and 
thus expressible % could have been expressed 
 with techniques proposed in the 90's \cite{MeyerDBC92}.
% \cite{OOPSLA22}  gives one  further specification, which says  that any reduction of funds can only be caused through a call to a specific method -- such specifications are beoind our scope.
 
 \paragraphSDD{ERC20} is a widely used % token 
 standard describing  basic functionality of Ethereum-based token 
contracts. 
The Solidity security model is not based on access to  capabilities but on who the caller  is. 
We  adapted our approach correspondingly, and 
express 1) that  the owner of an account is always authorized on that account,  2) any execution which does not contain calls from a participant  authorized on some account will not affect the balance nor  who is authorized on  that account. 
% The specifications from \cite{OOPSLA22} are more API-specific, in that they pinpoint which method calls caused an effect, and less specific in that they do not pinpoint the frame from which the effect occurred. 

\paragraphSDD{Stack} is a Wasm module exporting separate functions to read or modify its contents \cite{irisWasm23}. We specify that in the absence of external access to the latter capability, the contents will not change.  