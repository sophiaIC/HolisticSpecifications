%\se{You use 'some effects' as the opposite of 'uncertain/unknown effects'. I think the effects shouldn't be some effects, but certain effects or specific effects. Some has the feel of randomness (but who knows which) which isn't meant}
%kjx fixed

\section{Introduction}
\label{s:intro}

\paragraph{External calls}

In today's complex software, internal, trusted, code  is tightly intertwined  with external, untrusted, code: external code
calls into internal code, internal code calls
out to external code  and external code  even calls back  into internal code --- all within the same call chain.


This paper addresses reasoning about \emph{external calls} --- when
trusted internal code calls out to untrusted, unknown external code.
This reasoning is hard because by ``external code'' we mean untrusted
code where we don't have a specification.
% cannot trust that the
% code implements any specification we might have, indeed where the
% external code 
It may even have been written by an attacker trying to subvert or destroy the whole system.  
%Worse still, external code can call back into internal code as part of the same call chain. SUSAN:said in the paragraph before.
%
\forget{
External code may be written years or decades after the core 
trusted system was released --- software plugins for audio and video
editors have lasted for more than thirty years, and new
trusted retrocomputing hosts (``emulators'')
can run code for machines back to the Manchester 
Baby \cite{MAME-SSEM}, where source code is long lost in the bits of
history \cite{Smalltalk-bits}.
 }
 

\vspace*{1mm}

\begin{tabular}{lll}
\begin{minipage}{.45\textwidth}
The code sketch to the right illustrates the problem. The challenge arises from the external call on line 6: 
It looks as if the possible effects are not limited. % We should use "limit" or "unlimited" because later we say ``not all clls will have unlimited effects''
%Are there any limits  to its possible effects? 
%Are there any limits  to its possible effects? 
What  if  \prg{untrst}   calls back into $M_{intl}$? %, \eg  by calling  \prg{m1}? -- chpped as obvious how. and shorter.
\end{minipage}
& \ \  \   &
\begin{minipage}{.45\textwidth}
\begin{lstlisting}[mathescape=true, language=Chainmail, frame=lines]
module $M_{intl}$        
  method m1 ..
    ...  $\mbox{trusted code}$ ...  
  method m2(untrst:external) 
    ... $\mbox{trusted code}$ ...
    untrst.unkn(this) //external call    
    ... $\mbox{trusted code}$ ...
\end{lstlisting}
\end{minipage}
\end{tabular}

 
 

\paragraph{\Tamed effects}  In practice, not all external calls will have unlimited effects.
{If the programming language supports encapsulation (\eg no address forging, private
 fields, %, ownership, no direct memory access 
 \etc) then internal modules can be  written \emph{defensively} \cite{MillerPhD}, to ensure that external calls  {have only limited effects}  on internal components.} %  can cause effects only if some conditions specific to that effect are met.
For example, a defensive implementation of the DAO  \cite{Dao} can ensure that  (a)
{no external object can cause the DAO's balance to fall below the sum of the balances of its subsidiary accounts}, 
and (b) {no external object can cause reduction of the balance of the DAO unless the causing object is  one of the DAO's account holders.}
% ; moreover, an external object should not cause 
%  that its overall balance %of the DAO 
% is sufficient to cover the sum of the balances of its individual
 %accounts, and
 % concomitantly (b) that its overall balance can only be reduced along
% with one or more of its account balances, and (c) only with the 
%  permission the each account's signatory.
 
 We say a module has \emph{\tamed an effect}, when no outgoing external call can trigger that effect.
  While the literature has explored external calls \cite{vanproving,cerise-jacm2024,vmsl-pldi2023,ddd}, ``robust safety'' \cite{gordonJefferyRobustSafety,robustSafetyPatrignani,abate2019journey}, \etc, to our knowledge, there is no widely accepted term limiting the range of effects resulting from external calls.
%
\Tamed effects help mitigate the uncertainty associated with outgoing external calls. 
With tamed effects, we can ensure that certain properties established before an outgoing external call will continue to be satisfied afterward.


\forget{
 Some \tamed effects depend only on internal state, \eg 
  (a) and (b) above depend only upon the DAO's overall balance,
  and the balance of each of its internal, subsidiary, accounts.
  (A call to manipulate an account with a DAO is always an internal call.)
Other \tamed effects can also depend on the state of external
  objects, \eg  (c) above also depends on accounts' signatories, and
  those signatories are necessarily external to the DAO, and thus its
  subsidiary accounts. 
Reasoning about  \taming the \emph{internal} effects of internal
  calls typically requires only an adaptation of
  techniques
 \forget{ \footnote{is there a name for such reasoning techiques?
  egotrip? biochip? penmanship? internship? wysiwyp?}}
from reasoning about object
  invariants \cite{staticsfull,DrossoFrancaMuellerSummers08,BarDelFahLeiSch04,objInvars,MuellerPoetzsch-HeffterLeavens06}.
  Reasoning about  taming the \emph{external} effects of external calls
requires rather more --- thus the topic of this paper.}

 
 \Taming of effects may be   \emph{conditional} or \emph{unconditional}.
For example, (a) is unconditional: the balance of the DAO is always, unconditionally, kept above the sum of the balances of its accounts.
On the other hand, (b) is conditional: % reduction of the balance of the DAO is conditionally \tamed: 
 reduction is possible, but only  if the causing, external, 
 object is an account holder. 
Reasoning about {unconditional}  \taming of  effects     typically requires only an adaptation of
  techniques from % reasoning about  
object  invariants \cite{staticsfull,DrossoFrancaMuellerSummers08,BarDelFahLeiSch04,objInvars,MuellerPoetzsch-HeffterLeavens06}.
  Reasoning about  {conditional} taming of    effects
requires rather more --- {hence} the topic of this paper. 

\paragraph{Effects \tamed by capabilities} 
To \tame  effects  in their code, 
programmers    rely on various kinds of  encapsulation -- \eg if address forging were possible, the range of potential effects from external calls would be unlimited.
In addition, to conditionally \tame effects, programmers often 
employ  the object capability model (OCAP)\cite{MillerPhD} --
 or capability model for short. 
Capabilities are transferable rights that allow the performance of one or more operations on a specific object. 
They are {\emph{necessary}} conditions for causing effects; callers can only produce effects if they possess the required capabilities. 
For example, a signatory can withdraw funds from a DAO only if they hold a ``withdraw'' capability for that specific account within that particular DAO.  



\paragraph{Our remit:  Specification and Verification for  \tamed effects}
In this paper we demonstrate how to reason about  code which \tames effects,
{including those \tamed by capabilities.}  
We can specify effects to be \tamed, 
and then we can \emph{prove} that a module has indeed \tamed the effects we've specified.

Recent work has developed logics to prove properties of programs employing object capabilities.
\citet{ddd} develop a logic to prove that code employing object
capabilities for encapsulation preserves invariants for
intertwined code, but without external calls. 
\forget{Relying on techniques from separation logic, they intend their proofs e.g.\ to support library components that can be reused in different contexts.}
\forget{--- which is implicitly support in our distinction between external and internal code.
 \footnote{\red{TO CHECK}\kjx{I have looked at it - good enough for
 me.}}}
  \citet{dd} \forget{work with Kripke logical relations describing program semantics, which} can describe and verify invariants
about multi-object structures and the availability and exercise of object capabilities.  %\citeauthor{dd} 
% They propose and solve several
% problems involving external calls. 
\forget{,  but require proofs specific to each individual problem, rather than a systematic Hoare logic.}
 Similarly,
{\citet{vmsl-pldi2023} propose a separation logic to support formal modular reasoning about communicating VMs, including in the presence of unknown VMs.
while \cite{irisWasm23} specify WASM modules, %and compose them in a simple target language, 
and prove that adversarial code  can only affect other modules through the functions} that they explicitly export. 
%Since external calls may in turn call any number of internal, public methods, 
%it is crucial  to prove that \taming of effects is preserved by the module's 
%\emph{emergent} behaviour, \ie any intertwining of external code with calls to internal, public methods.
\citet{CassezFQ24}  handles external calls by replacing them through an unbounded number of calls to the module's public methods.
 
 The approaches from above %, with partical exception of  \cite{{FASE,OOPSLA22} 
 are nor concerned with indirect, eventual access to capabilities.
\citet{FASE} and \citet{OOPSLA22} do describe such access; %concentrate on access to capabilities;
  the former proposes  ``holistic specifications'' to describe a module's emergent behaviour. 
and the latter develops 
a tailor-made logic to prove that modules which do not contain external calls adhere to such specifications.
%They propose a taylor-made proof for an external call.
%We argue that in order to \emph{prove} that effects are \tamed,
%we need a way to specify the conditions necessary for that effect, 
%and a way to prove that the   module in its entirety preserves this \taming.
%In order to prove that the module in its entirety preserves this \taming, 
% \vspace{.2cm} 
% Thus, no work so far has tackled Hoare logics which support external calls as well as \tamed effects.

\vspace{.1cm}

All  systems mentioned above can reason about the effects of specific external calls, but they rely on problem-specific, custom-made proofs rather than a Hoare logic.
 In contrast, we propose a Hoare logic that addresses access to capabilities, external calls, and the module's \tamed effects.

  
\paragraph{This paper's contributions} % In this paper we propose 
{(1) assertions to describe   access to capabilities}, 
(2) a specification language to describe \taming of effects,
% with new assertions to describe access to capabilities, 
3) a Hoare logic to reason about external calls and to prove that modules satisfy their \tamed effects specifications,
4)  proof of soundness,
5) a worked illustrative example \susan{with a mechanised proof in Coq}.
 

 \paragraph{Structure of this paper}
Sect.\ \ref{s:outline}   outlines the main ingredients of our approach in terms of an example, and explains the main contributions.
Sect.\ \ref{sect:underlying} outlines a simple object-oriented language used for our work, and Sect. \ref{s:auxiliary} \susan{contains essential concepts for our study}.
Sect.\ \ref{s:assertions}, and Sect \ref{sect:spec}  give syntax and semantics of assertions, and  specifications,
{while Sect. \ref{s:preserve} discusses preservation of satisfaction of assertions.}
Sect.\ \ref{sect:proofSystem} develops Hoare triples and quadruples to prove external calls, and that a module adheres to its \tamed effects specifications.
Sect.\ \ref{sect:sound:proofSystem} outlines a proof of soundness of
the Hoare logic. Finally, Sect.\ \ref{sect:related} discusses related
work and Sect.\ \ref{sect:conclusion} concludes. 
\forget{For the absence of a conclusion, we offer neither explanation
nor apology, but trust this careful attention to POPL tradition will
endear our work to the referees.}
 
 
%Our specifications reflect modules' emergent behaviour,
% We also develop  Hoare logics to reason about modules containing external calls.

% \vspace{.8cm}
% \red{I want to also make these points, but do not know where. The points are overlapping,  and need  polishing:} 
% \begin{enumerate}
% \item
% \Tamed effects are essential when reasoning about intertwined code, but have not yet been widely studied -- esp. the competition has not crystalliuzed their need. We have, but have not given them a name. Perhaps the tamed effects should appear even earlier?
% \item
%\Tamed effects rely on conditions which are not under the control of the module. 
%For example, the module may require that the credentials are shown before it reduces the balance, but it cannot control which external objects have access to the credentials. 
%\item
%\label{reduced}
%The competition's effects  are not flexible enough. 
%They rely on only exposing a reduced version of the module's interface:
%For example, Birkedahl'22   prove -- by  hand -- that the stack's depth with remain the same, but only because the interface of their module does not expose a \prg{pop} method. 
%Such an assumption is too strong. 
%A module which did not support methods to reduce the stack are not as interesting or useful ... \red{here would be better to talk of balance and accounts}.
%What we want to have is modules with a rich interface, which expose different parts  of their interface to different external objects.
%\item
%One might think that it is possible to ensure that access to an object only goes through a reduced interface (as in point  \ref{reduced}).
%Namely, an internal object $o_1$ may create a proxy object $o_2$ to itself, which only exposes the reduced interface, and then pass $o_2$ as  argument to an external object $o_3$ .
%However, this does not guarantee that $o_3$ does not have a different route to $o_1$, which does not pass through the proxy $o_2$
%\item
%\Tamed effects \emph{have} to be hypothetical. \footnote{ (or context-dependent)}.  
%Some external objects will have access to the capability, and others will not. 
%We can gaurantee that without the relevant capability the effect will not happen, and also the capabilities will not be leaked. 
%But we can never guarantee that the capability is not present.
%\item
%In the context of OCAP, we need to reflect not only whether an external object has direct access (through a field, aor method  argumet) to the capability, but also, whether an external object may eventually obtain  direct access to the capability,
%% Eg they say that the interface does not provide a method to reduce the balance; then the reduction will not happen.
%\item
%\Tamed effects should take emergent behaviour into account. 
%They should hold over any number of execution steps, and not just over one particular method call. 
%For example \tamed effects should preclude  a module which does require the credentials in order to reduce the balance, but leaks the credentials. Therefore, per-method specifications are not sufficient.
%\end{enumerate}
%
%\red{TODO} Explain OCAP; introduce the challenge of the external word; reasoning and works by Devries, Birkedahl, Vechev, and ours.
%Include in that in OCAPs the capabilities act as guards. 
%That this has been tackled in FASE/OOPSLA'22 and gets better tackled here
%Also the difference between necessary and sufficient.
%And relation to POPL'23 paper; perhaps also some of recent work by Toby 

%Overall structure -- I do not know in what  order though.
%
%1) In today's complex world,  internal calls external and external calls internal
%
%2) To mitigate the risk, we introduced encapsulation ... ownership, private, etc etc 
%
%
%3) The OCAP model proposes capabilties .... def from OCAP. This means that capabilty is enabler -- sufficient condtion, 
%
%4)  However, a better understanding is that capability is guard, ie necessary condition. This has been recognized at OOPSLA 2022.
%
%
%5) We need to develop robust software, so that external calls do not do not have unwarranted effects ( check what term do Birkedahl et al use ).
%
%6) When we reason about external calls we can consider all possible calls back into the nmodule (2022 work), or even better 
%  have a spec of the complete module. OOSPLA did the spec but could not do the reasoning. 
%
%7) Emergent behaviour; to describe this, we need specs about complete module. Earlier work in that vein: two state invariants from the 02's. , temporal loghics, and FASE, OOPSLA
%-- competititon does not do.



%reason in the external world is not only about preservation of simple invariants; it is about reasoning about possible effects. These are reduced/increased through the availability of capabilities.
%
%The "others" tackle the problem of the ambient authority in a 1-step manner, eg there is a module, and if it exports a capability this capability may be used. For example, xxxx 
%
%FASE/OOPSLA support unrestricted ambient authority, but not the reasoning part. For example, xxxx
%
%In this paper, we support unrestricted ambient authority, and also reasoning. We are inspired by the work in FASE/OOSPLA, but with the following differences
%
%1) replace the temporal operators of FASE/necessity operators of oopsla by 2-state spec-s 
%
%2) when reasoning about a module's adherence to a 2-state spec, replace the xxx rules from OOPSLA by just one rule -- inspired by Infeasibility logic
%
%3) add one Hoare logic rule to reason about calls to external call; and add Hoare logic rules that reason about the accessibility of objects from other objects


