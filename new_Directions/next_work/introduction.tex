\section{Introduction}
\label{s:intro}

\newcommand{\Tamed}{Tamed\xspace}
\newcommand{\tamed}{tamed\xspace}
\newcommand{\tame}{tame\xspace}
\newcommand{\tames}{tames\xspace}
\newcommand{\Taming}{Taming\xspace}
\newcommand{\taming}{taming\xspace}

\paragraph{Uncertainty in Intertwined code} In today's complex software, internal, trusted, code  is tightly intertwined  with external, untrusted, code:
External, unverified, untrusted code may call functions from the internal code.
And internal, trusted code may call functions from the  external  code.
This tight intertwining   introduces a high degree of uncertainty.

To  reduce  that uncertainty, programmers employ  
\emph{encapsulation  features}
 (\eg   modules, private fields and methods, ownership types, memory safety, capability safety,  etc, \cite{private}),
and \emph{encapsulation  programming patterns} (\eg monitors, proxies, membranes, object capabilities etc, \cite{mebranes}).
Such  features and patterns were first introduced in the 70s, and widely improved, employed, and studied since.\footnote{AUTHORS: careful, though; they were studied in a less adversary situation -- there was little worry of the untrusted code, Can we say this?}

Encapsulation features have been  used in   verification of internal, non-intertwined code  \cite{Muller, Leino}.
Verification of code employing such  features to help reduce the uncertainty in intertwined code
has only recently  been studied \cite{xxx} -- Dreyer, Devriese, Mackay,  

%The subject of this work is the specification  and verification of properties of internal code which 
% reduces the uncertainty 
% through the 
%employs objects capabilities to reduce the uncertainty when intertwined with external, unknown code. 
%
In this paper, we address the reasoning about the effects of  \emph{open calls}\footnote{Do we call them open or external?} 
\ie  internal code which contains calls to external, untrusted functions.
Such open calls may have effects on the state of the internal code. 
Reasoning about  open calls poses two challenges:  
We do not have a specification of the called method. 
And the external code may have any number of effects, and in particular may call back into the internal code.

\vspace{.1cm}

\begin{tabular}{lll}
\begin{minipage}{.5\textwidth}
The code sketch to the right  illustrates the point:  An internal, trusted module $M_{intl}$  has methods \prg{m1} and \prg{m2}.
Parameter  \prg{untrs} % in \prg{m2} 
 is an external object. 
The open call on line  6 is the challenge:
What effects can it have?
What % happens 
if that call later on calls back into $M_{intl}$, \eg  calls  \prg{m1}?
\end{minipage}
& \ \  &
\begin{minipage}{.4\textwidth}
\begin{lstlisting}[mathescape=true, language=Chainmail, frame=lines]
module $M_{intl}$        
  method m1 ..
    ...  $\mbox{trusted code}$ ...  
  method m2(untrs:external) 
    ... $\mbox{trusted code}$ ...
    untrs.unkn(this)  // open call    
    ... $\mbox{trusted code}$ ...
\end{lstlisting}
\end{minipage}
\end{tabular}


 

\paragraph{\Tamed effects}
%{\textbf
{Our remit
 is not to rule out effects caused by external objects, % at the open call,  
 but to {\tame} them:}
 %}
\footnote{AUTHORS: Do we like the term \tame? Can we think if a better one? t I do not like "conditional effects".  Guarded effects? Reduced effects?
An advantage of "tamed" is that it is a short word.}
We use the term \emph{\tamed  effect} for an effect which may be caused by an external object, but only if that external objects satisfies
 certain conditions specific to that effect.
\footnote{AUTHORS: I think this is key.}
For instance, an external object may cause the reduction of the balance of a bank account, 
but only if the causing object is in procession of the corresponding credentials.
 

%The subject of this paper is how to characterize the possible effects at the point of the open call.

\paragraph{Program Development for \tamed effects} %To \emph{develop} programs with \tamed effects, programmers have to employ
needs to
 encapsulation features and \emph{encapsulated properties}.
\footnote{AUTHORS: All terms used in this section have been used with many different meanings. 
We must warn the reader that we use our own meanings here. 
Do not know where to say that.}
Thus the module can preserve its internal invariants  even in the presence of open calls,
some effects can be ruled out altogether (\eg it is impossible for an account's balance to be negative), and
also ensure that effects are tamed.
 %
 Often  programs \taming effects are developed following the object capability model (OCAP)\cite{Miller}, where  
 capabilties are transferable rights to perform one or more operations on a given object.
Thus, capabilities \emph{enable} and  \emph{guard} effects:   
Capabilities \emph{guard} effects: callers can make the effect   happen, only if they have access to the corresponding capability.
 
 
% \vspace{.1cm}
\paragraph{Specification and Verification  for \tamed effects} But how specify such \tamed effects? How to \emph{prove} that a program's effects are, indeed, \tamed?
And can these proofs be  through Hoare logic, rather than in a per-case, tailor-made proof?
 
Recent work has developed logics to prove properties of programs employing object capabilities.\footnote{albeit not with the remit of   \tamed effects.}
Dreyer OOPSLA'18 develop a logic to prove that   code employing object capabilities as encapsulation features preserve invariants for intertwined code, but in the absence of open calls. 
 \footnote{\red{TO CHECK}}.
Devriese - 17? also develop xx logic for object capabilties and code invariants, they propose interesting problems for open calls, 
but can only prove these properties through tailor-made proofs rather than Hoare logics.
Birkedahl-PLDI'23 xxx The paper includes a property of an open call, but again through a tailor-made proof, and not as a Hoare logic.


Since open calls may in turn call any number of internal, public methods, 
it is crucial  to prove that \taming of effects is preserved by the module's 
\emph{emergent} behaviour, \ie any intertwining of external code with calls to internal, public methods.
\footnote{Properties which rely on the type system, are of course preserved, but what about more semantic properties?}
xxx-OOPSLA-22-workshop does this by replacing the open call though an unbounded number of calls to the module's public method.
 FASE instead proposes holistic specifications to describe   module's emergent behaviour. 
 OOPLSA'22 proposes a xxx-specification language for a module's emergent behaviour, and  tailor-made Hoare logic to prove that modules which do not contain open call adhere to such specifications.
%They propose a taylor-made proof for an open call.



%We argue that in order to \emph{prove} that effects are \tamed,
%we need a way to specify the conditions necessary for that effect, 
%and a way to prove that the   module in its entirety preserves this \taming.
%In order to prove that the module in its entirety preserves this \taming, 
% \vspace{.2cm} 
% Thus, no work so far has tackled Hoare logics which support open calls as well as \tamed effects.

\paragraph{This paper proposes} % In this paper we propose 
 ways to specify  \taming of effects for  programs which employ the OCAP model, 
 and a Hoare logic of quadruples which can reason about open calls, and prove module's emergent behaviour indeed \tames the effects as specified.
%Our specifications reflect modules' emergent behaviour,
% We also develop  Hoare logics to reason about modules containing open calls.

 \vspace{.8cm}
 \red{I want to also make these points, but do not know where. The points are overlapping,  and need  polishing:} 
 \begin{enumerate}
 \item
\Tamed effects rely on conditions which are not under the control of the module. 
For example, the module may require that the credentials are shown before it reduces the balance, but it cannot control which external objects have access to the credentials. 
\item
\label{reduced}
The competition's effects  are not flexible enough. 
They rely on only exposing a reduced version of the module's interface:
For example, Birkedahl'22   prove -- by  hand -- that the stack's depth with remain the same, but only because the interface of their module does not expose a \prg{pop} method. 
Such an assumption is too strong. 
A module which did not support methods to reduce the stack are not as interesting or useful ... \red{here would be better to talk of balance and accounts}.
What we want to have is modules with a rich interface, which expose different parts  of their interface to different external objects.
\item
One might think that it is possible to ensure that access to an object only goes through a reduced interface (as in point  \ref{reduced}).
Namely, an internal object $o_1$ may create a proxy object $o_2$ to itself, which only exposes the reduced interface, and then pass $o_2$ as  argument to an external object $o_3$ .
However, this does not guarantee that $o_3$ does not have a different route to $o_1$, which does not pass through the proxy $o_2$
\item
\Tamed effects \emph{have} to be hypothetical. \footnote{ (or context-dependent)}.  
Some external objects will have access to the capability, and others will not. 
We can gaurantee that without the relevant capability the effect will not happen, and also the capabilities will not be leaked. 
But we can never guarantee that the capability is not present.
\item
In the context of OCAP, we need to reflect not only whether an external object has direct access (through a field, aor method  argumet) to the capability, but also, whether an external object may eventually obtain  direct access to the capability,
% Eg they say that the interface does not provide a method to reduce the balance; then the reduction will not happen.
\item
\Tamed effects should take emergent behaviour into account. 
They should hold over any number of execution steps, and not just over one particular method call. 
For example \tamed effects should preclude  a module which does require the credentials in order to reduce the balance, but leaks the credentials. Therefore, per-method specifications are not sufficient.
\end{enumerate}

%\red{TODO} Explain OCAP; introduce the challenge of the open word; reasoning and works by Devries, Birkedahl, Vechev, and ours.
%Include in that in OCAPs the capabilities act as guards. 
%That this has been tackled in FASE/OOPSLA'22 and gets better tackled here
%Also the difference between necessary and sufficient.
%And relation to POPL'23 paper; perhaps also some of recent work by Toby 

%Overall structure -- I do not know in what  order though.
%
%1) In today's complex world,  internal calls external and external calls internal
%
%2) To mitigate the risk, we introduced encapsulation ... ownership, private, etc etc 
%
%
%3) The OCAP model proposes capabilties .... def from OCAP. This means that capabilty is enabler -- sufficient condtion, 
%
%4)  However, a better understanding is that capability is guard, ie necessary condition. This has been recognized at OOPSLA 2022.
%
%
%5) We need to develop robust software, so that external calls do not do not have unwarranted effects ( check what term do Birkedahl et al use ).
%
%6) When we reason about open calls we can consider all possible calls back into the nmodule (2022 work), or even better 
%  have a spec of the complete module. OOSPLA did the spec but could not do the reasoning. 
%
%7) Emergent behaviour; to describe this, we need specs about complete module. Earlier work in that vein: two state invariants from the 02's. , temporal loghics, and FASE, OOPSLA
%-- competititon does not do.



%reason in the open world is not only about preservation of simple invariants; it is about reasoning about possible effects. These are reduced/increased through the availability of capabilities.
%
%The "others" tackle the problem of the ambient authority in a 1-step manner, eg there is a module, and if it exports a capability this capability may be used. For example, xxxx 
%
%FASE/OOPSLA support unrestricted ambient authority, but not the reasoning part. For example, xxxx
%
%In this paper, we support unrestricted ambient authority, and also reasoning. We are inspired by the work in FASE/OOSPLA, but with the following differences
%
%1) replace the temporal operators of FASE/necessity operators of oopsla by 2-state spec-s 
%
%2) when reasoning about a module's adherence to a 2-state spec, replace the xxx rules from OOPSLA by just one rule -- inspired by Infeasibility logic
%
%3) add one Hoare logic rule to reason about calls to external call; and add Hoare logic rules that reason about the accessibility of objects from other objects


