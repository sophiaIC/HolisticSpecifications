
\begin{lemma}
\label{l:no:meth:calls}
If ${\hproves{M}  {A} {\ stmt\ }{A'} }$, then $stmt$ contains no method calls.
\end{lemma}

\begin{proof}
By induction on the rules in Fig. \ref{f:underly}.

\end{proof}

\subsection{Adaptation}
\label{appendix:adaptation}
 
 \newcommand{\SP}{$\strut \ \ \ \ $}


 We now discuss the proof of Lemma \ref{lemma:push:ass:state}.

 \vspace{0.5cm}
 
 \beginProofSub{lemma:push:ass:state}{l:push:stbl}
$~$ \\
To Show: \ \ \  $\Stable{\,  \PushASLong {(y_0,\overline y)} A\, }$
\\
By structural induction on $A$.\\
\completeProofSub

\vspace{1cm}

For parts \ref{lemma:push:ass:state:one},  \ref{lemma:push:ass:state:two}, and  \ref{lemma:push:ass:state:three}, we first prove the following auxiliary lemma:

\begin{auxLemma}
\label{l:push:pop:aux}
For all $\alpha$,   $\overline {\phi_1}$, $\overline {\phi_2}$, $\overline {\phi_2}$, $\phi$ and $\chi$\\
$\strut ~ \ \ \ \ \ (L1)\ \ \    M, (\overline {\phi_1},\chi) \models \protectedFrom \alpha {Rng(\phi)} \ \Longrightarrow \ M, (\overline {\phi_2}\cdot \phi,\chi) \models \inside \alpha$
\\
$\strut ~ \ \ \ \ \ (L2)\ \ \    M, (\overline {\phi_1}\cdot\phi,\chi) \models \inside \alpha   \wedge \extThis \ \ \Longrightarrow \ \ M, (\overline {\phi_2},\chi) \models \protectedFrom \alpha {Rng(\phi)} $\\
$\strut ~ \ \ \ \ \ (L3)\ \ \    M, (\overline {\phi_1}\cdot \phi_1,\chi) \models \inside \alpha   \wedge \extThis \ \ \wedge Rng(\phi)\subseteq Rng(\phi_1)\ \ \  \Longrightarrow \ \ M, (\overline {\phi_2},\chi) \models \protectedFrom \alpha {Rng(\phi)} $
\\\end{auxLemma}

\begin{proof}
$~$ \\
We first prove (L1): \\
~ \\
We define $\sigma_1 \triangleq (\overline {\phi_1},\chi)$, and  $\sigma_2 \triangleq (\overline {\phi_2}\cdot \phi,\chi) $.\\
The above definitions imply that: \\
\SP (1)\ \ $\forall \alpha',\forall \overline f.[\  \interpret {\sigma_1} {\alpha'.\overline f} =  \interpret {\sigma_2} {\alpha'.\overline f}\ ]$\\
\SP (2)\ \ $\forall \alpha'.[\  \Relevant {\alpha'} {\sigma_1} = \Relevant {\alpha'} {\sigma_2}\ ]$\\
\SP (3)\ \ $\LRelevantO {\sigma_2} = \bigcup_{\alpha'\in Rng(\phi)} \Relevant {\alpha'} {\sigma_2} $.\\
% $\strut ~ \ $\\
We now assume that\\
\SP (4)\ \ $M, \sigma_1 \models \protectedFrom \alpha {Rng(\phi)}$.\\
and want to show that\\
\SP (??)\ \ $M, \sigma_2 \models \inside \alpha$\\
From (4) and  by definitions, we obtain that\\
\SP (5)\ \ $\forall \alpha'\in Rng(\phi).\forall \alpha''\in \Relevant {\alpha'} {\sigma_1}.\forall f.[ \   M, \sigma_1 \models \alpha'':\prg{extl}\ \rightarrow \alpha''.f  \neq \alpha\ ]$, \ \ \ \ and also\\
\SP (6)\ \ $\alpha \notin Rng(\phi)$\\
From (5) and (3) we obtain:\\
\SP (7)\ \  $\forall \alpha' \in \LRelevantO {\sigma_2}.\forall f.[ \   M, \sigma_1 \models \alpha':\prg{extl}\ \rightarrow \alpha'.f  \neq \alpha\ ]$\\
From (7) and (1) and (2) we obtain:\\
\SP (8) \ \  $\forall \alpha' \in \LRelevantO {\sigma_2}.\forall f.[ \   M, \sigma_2 \models \alpha':\prg{extl}\ \rightarrow \alpha'.f  \neq \alpha\ ]$\\
From (8), by definitions, we obtain\\
 \SP (10)\ \ $M, \sigma_2 \models \inside \alpha$\\
 This completes the proof of (L1).
 \\
  $\strut ~ \ $\\
  We now prove (L2): \\
  ~ \\
 We define $\sigma_1 \triangleq (\overline {\phi_1}\cdot \phi,\chi)$, and  $\sigma_2 \triangleq (\overline {\phi_2},\chi) $.\\
The above definitions imply that: \\
\SP (1)\ \ $\forall \alpha',\forall \overline f.[\  \interpret {\sigma_1} {\alpha'.\overline f} =  \interpret {\sigma_2} {\alpha'.\overline f}\ ]$\\
\SP (2)\ \ $\forall \alpha'.[\  \Relevant {\alpha'} {\sigma_1} = \Relevant {\alpha'} {\sigma_2}\ ]$\\
\SP (3)\ \ $\LRelevantO {\sigma_1} = \bigcup_{\alpha'\in Rng(\phi)} \Relevant {\alpha'} {\sigma_1}$.\\
We   assume that\\
\SP (4)\ \  $M, \sigma_1 \models \inside \alpha \wedge \extThis$.\\
and want to show that\\
\SP (A?)\ \ $M, \sigma_2 \models \PushASLong  {Rng(\phi)} {A}$.\\
From (4), and unfolding the definitions, we obtain:\\
\SP (5)\ \  $\forall \alpha'\in \LRelevantO {\sigma_1}.\forall f.[ \   M, \sigma_1 \models \alpha':\prg{extl}\ \rightarrow \alpha'.f  \neq \alpha\ ]$, \ \ \ and\\
\SP (6)\ \ $\forall \alpha'\in Rng (\phi). [ \ \alpha'\neq \alpha \ ]$.\\
From(5), and using (3) and (2) we obtain:
\\
\SP (7)\ \  $\forall \alpha'\in Rng(\phi).\forall \alpha'' \in\Relevant {\alpha'} {\sigma_2}.\forall f.[ \   M, \sigma_2 \models \alpha'':\prg{extl}\ \rightarrow \alpha''.f  \neq \alpha\ ]$\\
From (5) and (7) and by definitions, we obtain
\\
\SP (8)\ \  $\forall \alpha'\in Rng (\phi).[ \   \models \alpha \protectedFrom \alpha {\alpha'}\ ]$.\\
From (8) and definitions we obtain (A?).\\
This completes the proof of (L2). 
 \\
  $\strut ~ \ $\\
  We now prove (L3): \\
  ~ \\
 We define $\sigma_1 \triangleq (\overline {\phi_1}\cdot \phi_1,\chi)$, and  $\sigma_2 \triangleq (\overline {\phi_2},\chi) $.\\
The above definitions imply that: \\
\SP (1)\ \ $\forall \alpha',\forall \overline f.[\  \interpret {\sigma_1} {\alpha'.\overline f} =  \interpret {\sigma_2} {\alpha'.\overline f}\ ]$\\
\SP (2)\ \ $\forall \alpha'.[\  \Relevant {\alpha'} {\sigma_1} = \Relevant {\alpha'} {\sigma_2}\ ]$\\
\SP (3)\ \ $\LRelevantO {\sigma_1} = \bigcup_{\alpha'\in Rng(\phi_1)} \Relevant {\alpha'} {\sigma_1}$.\\
We   assume that\\
\SP (4a)\ \  $M, \sigma_1 \models \inside \alpha \wedge \extThis$, and
\SP (4b)\ \ $Rng(\phi) \subseteq Rng(\phi_1)$\\
We  want to show that\\
\SP (A?)\ \ $M, \sigma_2 \models \PushASLong  {Rng(\phi)} {A}$.\\
From (4a), and unfolding the definitions, we obtain:\\
\SP (5)\ \  $\forall \alpha'\in \LRelevantO {\sigma_1}.\forall f.[ \   M, \sigma_1 \models \alpha':\prg{extl}\ \rightarrow \alpha'.f  \neq \alpha\ ]$, \ \ \ and\\
\SP (6)\ \ $\forall \alpha'\in Rng (\phi_1). [ \ \alpha'\neq \alpha \ ]$.\\
From(5), and   (3) and (2) and (4b) we obtain:
\\
\SP (7)\ \  $\forall \alpha'\in Rng(\phi).\forall \alpha'' \in\Relevant {\alpha'} {\sigma_2}.\forall f.[ \   M, \sigma_2 \models \alpha'':\prg{extl}\ \rightarrow \alpha''.f  \neq \alpha\ ]$ \\
From(6), and   (4b) we obtain:
\\
\SP (8)\ \ $\forall \alpha'\in Rng (\phi_1). [ \ \alpha'\neq \alpha \ ]$.\\
From (8) and definitions we obtain (A?).\\
This completes the proof of (L3). 

\end{proof}


\beginProofSub{lemma:push:ass:state}{lemma:push:ass:state:one}
$~$ \\
To Show: \ \ \  $(*)\ \ \ M, \sigma \models \PushASLong  {Rng(\phi)} {A}\  \ \ \ \ \  \ \ \    \Longrightarrow  \ \ \ \ M,  \PushSLong {\phi} {\sigma}   \models A$
\\ $~$ \\
 By  induction on the structure of $A$. For the case where $A$ has the form $\inside {\alpha.\overline f}$, we use lemma \ref{l:push:pop:aux},(L1), taking $\overline {\phi_1} = \overline { \phi_2}$, and $\sigma \triangleq (\overline {\phi_1},\chi).$
\\
\completeProofSub
 

\vspace{1cm}

\beginProofSub{lemma:push:ass:state}{lemma:push:ass:state:two}
$~$ \\
To Show \ \ \  $(*)\ \ \  M,  \PushSLong {\phi} {\sigma}   \models  A  \wedge \extThis    \ \  \ \  \Longrightarrow  \ \ \ \ M, \sigma \models \PushASLong  {Rng(\phi)} {A}$ 
\\
$~$ \\
We apply induction on the structure of $A$. For the case where $A$ has the form $\inside {\alpha.\overline f}$, we apply lemma \ref{l:push:pop:aux},(L2), using    $\overline {\phi_1} = \overline { \phi_2}$, and $\sigma \triangleq (\overline {\phi_1},\chi).$

\completeProofSub

\vspace{1cm}
\beginProofSub{lemma:push:ass:state}{lemma:push:ass:state:three}
$~$ \\
To Show:\ \ \   (*) \ \  $M, \sigma  \models  A  \wedge \extThis  \ \wedge \ M\cdot\Mtwo \models \PushSLong {\phi} {\sigma}   \ \  \ \ \  \  \Longrightarrow  \ \ \ \ M, \PushSLong {\phi} {\sigma} \models \PushASLong  {Rng(\phi)} {A}$
\\ 
$~$ \\
By induction on the structure of $A$. 
 For the case where $A$ has the form $\inside {\alpha.\overline f}$, we want to apply lemma \ref{l:push:pop:aux},(L3). We take  $\sigma$ to be $ (\overline {\phi_1}\cdot\phi_1, \chi)$, and $\overline {\phi_2}=\overline {\phi_1}\cdot\phi_1\cdot \phi$. Moreover,  $M\cdot\Mtwo \models \PushSLong {\phi} {\sigma}$ gives  that $Rng(\phi)\subseteq \LRelevantO {\sigma_2}$. Therefore, (*) follows by application of lemma \ref{l:push:pop:aux},(L3).\\
\completeProofSub