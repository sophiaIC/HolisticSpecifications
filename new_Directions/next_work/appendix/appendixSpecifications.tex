\section{Appendix to Section \ref{sect:spec} -- Specifications}
\label{app:spec}


 \begin{example}[Badly Formed Method Specifications]
$S_{9,bad\_1}$ is not a well-formed specification, because $A'$ is not a formal parameter, nor free in the precondition. 

   {\sprepost
		{\strut \ \ \ \ \ \ \ \ \ S_{9,bad\_1} }
		{  a:\prg{Account} \wedge  \inside{a} }
		{\prg{public Account}} {\prg{set}} {\prg{key'}:\prg{Key}}
		{   \inside{a}\wedge  \inside{a'.\prg{key}}  }
		{  true }
}
		
 {\sprepost
		{\strut \ \ \ \ \ \ \ \ \ S_{9,bad\_2} }
		{  a:\prg{Account} \wedge  \inside{a} }
		{\prg{public Account}} {\prg{set}} {\prg{key'}:\prg{Key}}
		{   \inside{a}\wedge  \inside{a'.\prg{key}}  }
		{  \prg{this}.\balance \ }

}
\end{example}

{ \begin{example}[More Method Specifications]
\label{ex:spesMore}
$S_7$ below  guarantees that
\prg{transfer} does not affect the balance of accounts different  from the receiver or argument, and  if the key supplied is not that of the receiver, then no account's balance is affected.  \
$S_8$ guarantees that if the key supplied is that of the receiver, the correct amount is transferred from the receiver to the destination.
 $S_9$ guarantees that \prg{set} preserves the protectedness of a key.

\small{
{\sprepost
		{\strut \ \ \ \ S_7} 
		{ a:\prg{Account}\wedge  a.\prg{\balance}=b \wedge
		(\prg{dst}\neq a\neq\prg{this} \vee \prg{key'}\neq a.\prg{\password})}
	               {\prg{public Account}} {\prg{transfer}} {\prg{dst}:\prg{Account},\prg{key'}:\prg{Key},\prg{amt}:\prg{nat}}
		{ a.\prg{\balance}=b}
		{\sdred{ a.\prg{\balance}=b}}
}
\\
{\sprepostLB
		{\strut \ \ \ \ \ \ \ \ \ S_8} 
		{  % b,b':\prg{int} \wedge  
		\prg{this}\neq \prg{dst}\wedge \prg{this}.\prg{\balance}=b \wedge  \prg{dst}.\prg{\balance}=b' }
		  {\prg{public Account}}
		  	  {\prg{transfer}} 
		   	 {\prg{dst}:\prg{Account},\prg{key'}:\prg{Key},\prg{amt}:\prg{nat} }
		{\prg{this}.\prg{\balance}=b-\prg{amt} \wedge \prg{dst}.\prg{\balance}=b'+\prg{amt} } 
		{    \prg{this}.\prg{\balance}=b \wedge  \prg{dst}.\prg{\balance}=b'   }
}
\\
{\sprepost
		{\strut \ \ \ \ \ \ \ \ \ S_9} 
		{  a:\prg{Account}\wedge
		 \inside{a.\prg{\password}}}
		{\prg{public Account}} {\prg{set}} {\prg{key'}:\prg{Key}}
		{ \inside{a.\prg{\password}}}
		{ \inside{a.\prg{\password} } }		
}
}
\end{example}
}
 
%??? OR should it be ??? (for the below, we need to add some implications in the Hoare logic for call, or for specs
%\\
%{\sprepost
%		{\strut \ \ \ \ S_7} 
%		{ a:\prg{Account}\wedge  a.\prg{\balance}=b \wedge
%		(\prg{dst}\neq a\neq\prg{this} \vee \prg{key'}\neq a.\prg{\password})}
%	               {\prg{public Account}} {\prg{transfer}} {\prg{dst}:\prg{Account},\prg{key'}:\prg{Key},\prg{amt}:\prg{nat}}
%		{ a.\prg{\balance}=b}
%		{\sdred{ false}}
%}
%\\
%{\sprepostLB
%		{\strut \ \ \ \ \ \ \ \ \ S_8} 
%		{  % b,b':\prg{int} \wedge  
%		\prg{this}\neq \prg{dst}\wedge \prg{this}.\prg{\balance}=b \wedge  \prg{dst}.\prg{\balance}=b' }
%		  {\prg{public Account}}
%		  	  {\prg{transfer}} 
%		   	 {\prg{dst}:\prg{Account},\prg{key'}:\prg{Key},\prg{amt}:\prg{nat} }
%		{\prg{this}.\prg{\balance}=b-\prg{amt} \wedge \prg{dst}.\prg{\balance}=b'+\prg{amt} } 
%		{    false  }
%}
%\\
%{\sprepost
%		{\strut \ \ \ \ \ \ \ \ \ S_9} 
%		{  a:\prg{Account}\wedge
%		 \inside{a.\prg{\password}}}
%		{\prg{public Account}} {\prg{set}} {\prg{key'}:\prg{Key}}
%		{ \inside{a.\prg{\password}}}
%		{ false }
%
%}




\forget{
 \label{example:twostatesarisfy}
\se{We revisit the modules and specifications from Sect. \ref{s:bankSpecEx}, and Example \ref{ex:spacesMore} :}


\begin{tabular}{lllllllll}
$\ModA  \not\models S_1$  &   $\ModA  \models S_2$  &     $\ModA \models S_3$    & $\ModA \models S_5$\\
 $\ModB \not\models S_1$  &   $\ModB \not\models S_2$     &  $\ModB  \not\models S_3$   & $\ModB \not\models S_5$ \\
 $\ModC  \not\models S_1$    & $\ModC \models S_2$ &   & $\ModC \not\models S_3$   & $\ModC \not\models S_5$ 
\end{tabular}
\end{example}
 

 
 \begin{example}
 \label{example:mprepost:sat:one}
 For  %method
 Example \ref{example:mprepostl}, we have
  $\ModA \models S_6$ and $\ModB \models S_6$ and  $\ModC \models S_6$.
Also,  $\ModA \models S_7$ and $\ModB \models S_7$ and  $\ModC \models S_7$.
However,   $\ModA  \models S_8$, while $\ModB  \not\models S_8$.
\end{example}

 \begin{example}
\label{example:mprepost:sat:two}
 For  %method
any   specification  $S \triangleq {\mprepost{A}{p\ C}{m}{x}{C}{A'} }$ and any module  $M$ which does not have a class $C$  with a method $m$ with formal parameter  types ${\overline C}$, we have that $M \models S$.
Namely, if a method were to be called with that signature on a $C$  from $M$, then execution would be stuck, and the requirements from Def. \ref{def:necessity-semantics}(3) would be trivially satisfied.
Thus,   $\ModC \models S_8$. %, even though $\ModC$ does not have a method \prg{set} with the signature given in $S_6$;
\end{example}
}

\subsection{Examples of Semantics of our Specifications}

\begin{example}
 \label{example:mprepost:sat:three}
We  revisit the specifications given in Sect. \ref{s:bankSpecEx},  the three  modules from Sect. \ref{s:bank}, and Example \ref{ex:spesMore}


\begin{tabular}{lllllllll}
$\ModA  \models S_1$  &   $\ModA  \models S_2$ &   $\ModA \models S_3$    & $\ModA \models S_5$\\
 $\ModB \models S_1$  &   $\ModB \not\models S_2$   &  $\ModB  \not\models S_3$   & $\ModB \not\models S_5$ \\
 $\ModC  \models S_1$    & $\ModC \models S_2$ & $\ModC \not\models S_3$   & $\ModC \not\models S_5$ 
\end{tabular}
\end{example}
 

 
 \begin{example}
 \label{example:mprepost:sat:four}
 For  %method
 Example \ref{example:mprepostl}, we have
  $\ModA \models S_7$ and $\ModB \models S_7$ and  $\ModC \models S_7$.
Also,  $\ModA \models S_8$ and $\ModB \models S_8$ and  $\ModC \models S_8$.
However,   $\ModA  \models S_9$, while $\ModB  \not\models S_9$.
\end{example}

 \begin{example}
\label{example:mprepost:sat:five}
 For  %method
any   specification  $S \triangleq {\mprepost{A}{p\ C}{m}{x}{C}{A'} }$ and any module  $M$ which does not have a class $C$  with a method $m$ with formal parameter  types ${\overline C}$, we have that $M \models S$.
Namely, if a method were to be called with that signature on a $C$  from $M$, then execution would be stuck, and the requirements from Def. \ref{def:necessity-semantics}(3) would be trivially satisfied.
Thus,   $\ModC \models S_8$. %, even though $\ModC$ does not have a method \prg{set} with the signature given in $S_6$;
\end{example}

\section{Expressiveness} 

We argue the expressiveness of our approach by comparing with example specifications  proposed in \cite{OOPSLA22}.

 %% We continue the comparison of expresiveness between \emph{Chainmail} and \Nec, by 
 %% considering the examples studied in \cite{FASE}.
 
%\begin{example}[ERC20]

\subsection{The DOM}  %\sophiaPonder[renamed Wrapper to Proxy]{  
\label{ss:DOM}
This is the motivating example in \cite{dd},
dealing with a tree of DOM nodes: Access to a DOM node
gives access to all its \prg{parent} and \prg{children} nodes, with the ability to
modify the node's \prg{property} -- where  \prg{parent}, \prg{children} and \prg{property}
are fields in class \prg{Node}. Since the top nodes of the tree
usually contain privileged information, while the lower nodes contain
less crucial third-party information, we must be able to limit 
 access given to third parties to only the lower part of the DOM tree. We do this through a \prg{Proxy} class, which has a field \prg{node} pointing to a \prg{Node}, and a field \prg{height}, which restricts the range of \prg{Node}s which may be modified through the use of the particular \prg{Proxy}. Namely, when you hold a \prg{Proxy}  you can modify the \prg{property} of all the descendants of the    \prg{height}-th ancestors of the \prg{node} of that particular \prg{Proxy}.  We say that
\prg{pr} has \emph{modification-capabilities} on \prg{nd}, where \prg{pr} is
a  \prg{Proxy} and \prg{nd} is a \prg{Node}, if the \prg{pr.height}-th  \prg{parent}
of the node at \prg{pr.node} is an ancestor of \prg{nd}.
%}

We specify this property as follows:
\\
$\strut \SPSP  S_{dom\_1}\ \  \triangleq \ \ \TwoStatesN{ nd:\prg{DomNode}}{\  \forall pr:\prg{Proxy}.[ \ may\_modi\!f\!y(pr, nd) \rightarrow \inside{pr}\  ] \ }$ 
\\
$\strut \SPSP  S_{dom\_2}\ \  \triangleq \ \  \forall{ nd:\prg{DomNode}, val:\prg{PropertyValue} }$.\\
$\strut \SPSP\strut \SPSP\strut \SPSP\strut \SPSP	\{ { \ \forall pr:\prg{Proxy}.[ \ may\_modi\!f\!y(pr, nd) \rightarrow \inside{pr}\  ]  \wedge nd.property = val} \ \} $ 
\\
where $may\_modi\!f\!y(pr, nd) \triangleq \exists k. [ \  nd.parent^k=pr.node.parent^{pr.height}\ ]$


Note that $S_{dom\_2}$ is strictly stronger than $S_{dom\_1}$
% THE MEANING IS OBVIOUS
% \footnote{\red{We have not defined strictly stronger in this document}.}

\vspace{.1cm}

In  \cite{OOPSLA22} this was specified as follows:
 
 \begin{lstlisting}[language = Chainmail, mathescape=true, frame=lines]
DOMSpec $\triangleq$ from nd : Node $\wedge$ nd.property = p  to nd.property != p
          onlyIf $\exists$ o.[ $\external {\prg{o}}$ $\wedge$ 
                       $( \  \exists$ nd':Node.[ $\access{\prg{o}}{\prg{nd'}}$ ]  $\vee$ 
                         $\exists$ pr:Proxy,k:$\mathbb{N}$.[$\, \access{\prg{o}}{\prg{pr}}$ $\wedge$ nd.parent$^{\prg{k}}$=pr.node.parent$^{\prg{pr.height}}$ ] $\,$ ) $\,$ ]
\end{lstlisting}

\prg{DomSpec} states that the \prg{property} of a node can only change if
some external object presently has 
access to a node of the DOM tree, or to some \prg{Proxy} with modification-capabilties
to the node that was modified.
The assertion $\exists {o}.[\ \external {\prg{o}} \wedge \access{\prg{o}}{\prg{pr}}\ ]$ is the contrapositive of our  $\inside{pr}$, but is is weaker than that, because it does not specify the frame from which $o$ is accessible.
Therefore, $\prg{DOMSpec}$ is a stronger requirement than $S_{dom\_1}$.

\subsection{DAO}
The Decentralized Autonomous Organization (DAO)~\cite{Dao}  is a well-known Ethereum contract allowing 
participants to invest funds. The DAO famously was exploited with a re-entrancy bug in 2016, 
and lost \$50M. Here we provide specifications that would have secured the DAO against such a 
bug. 
\\ 
$\strut \SPSP  S_{dao\_1}\ \  \triangleq \ \ \TwoStatesN{ d:\prg{DAO}}{\ \forall p:\prg{Participant}. [\ d.ether \geq d.balance(p) \ ]   \ }$ 
\\
$\strut \SPSP  S_{dao\_2}\ \  \triangleq \ \ \TwoStatesN{ d:\prg{DAO}}{\ \ d.ether \geq \sum_{p \in d.particiants} d.balance(p)\  \ }$ 


The specifications above say the following:
\\
\begin{tabular}{ll}
\begin{minipage}{.10\textwidth}
$\strut \SPSP  S_{edao\_1}$
\end{minipage}
&
\begin{minipage}{.85\textwidth}
guarantees that the DAO holds more ether than the balance  of any of its  participant's.
\end{minipage}
\\
\\
\begin{minipage}{.10\textwidth}
$\strut \SPSP  S_{dao\_2}$ 
\end{minipage}
&
\begin{minipage}{.85\textwidth}
guarantees that that the DAO holds more ether than the sum  of the balances held by DAO's participants.
\end{minipage}
\end{tabular}

$S_{dao\_2}$  is stronger than $S_{dao\_1}$. They would both have precluded the DAO bug. Note that these specifications  do not mention capabilities. 
They are, essentially, simple class invariants and could have been expressed with the techniques proposed already by \cite{MeyerDBC92}.
The only difference is that $S_{dao\_1}$ and $S_{dao\_2}$ are two-state invariants, which means that we require that they are \emph{preserved},
\ie if they hold in one (observable) state they have to hold in all successor states,
while class invariants are one-state, which means they are required to hold in all (observable) states.
\footnote{This should have been explained somewhere earlier.}

\vspace{0.5cm}
We now compare with the specification given in \cite{OOPSLA22}.
\prg{DAOSpec1} in similar to  $S_{dao\_1}$: iy
says that no participant's balance may ever exceed the ether remaining 
in DAO. It is, essentially, a one-state invariant.


\begin{lstlisting}[language = Chainmail, mathescape=true, frame=lines]
DAOSpec1 $\triangleq$ from d : DAO $\wedge$ p : Object
            to d.balance(p) > d.ether
            onlyIf false
\end{lstlisting}
\prg{DAOSpec1}, similarly to $S_{dao\_1}$,   in that it enforces a class invariant of \prg{DAO}, something that could be enforced
by traditional specifications using class invariants.


 \cite{OOPSLA22}  gives one more   specification: 
 
 \begin{lstlisting}[language = Chainmail, mathescape=true, frame=lines]
DAOSpec2 $\triangleq$ from d : DAO $\wedge$ p : Object
            next d.balance(p) = m
            onlyIf $\calls{\prg{p}}{\prg{d}}{\prg{repay}}{\prg{\_}}$ $\wedge$ m = 0 $\vee$ $\calls{\prg{p}}{\prg{d}}{\prg{join}}{\prg{m}}$ $\vee$ d.balance(p) = m
\end{lstlisting}

 \prg{DAOSpec2} states that if after some single step of execution, a participant's balance is \prg{m}, then 
either 
\begin{description}
\item[(a)] this occurred as a result of joining the DAO with an initial investment of \prg{m}, 
\item[(b)] the balance is \prg{0} and they've just withdrawn their funds, or 
\item[(c) ]the balance was \prg{m} to begin with
\end{description}

%\subsection{Safe}
%\cite{FASE} used as a running example   a Safe, where a treasure 
%was secured within a \texttt{Safe} object, and access to the treasure was only granted by 
%providing the correct password. 
%\red{Sophia proposes that we drop the Safe}
%\ Using \Nec, we express \texttt{SafeSpec}, that requires that the treasure cannot be 
%removed from the safe without knowledge of the secret.
%\begin{lstlisting}[language = Chainmail, mathescape=true, frame=lines]
%SafeSpec $\triangleq$ from s : Safe $\wedge$ s.treasure != null
%            to s.treasure == null
%            onlyIf $\neg$ inside(s.secret)
%\end{lstlisting}
%
%The module  \prg{SafeModule} described  below satisfies  \prg{SafeSpec}.
%
%\begin{lstlisting}[frame=lines]
%module SafeModule
%     class Secret{}
%     class Treasure{}
%     class Safe{
%         field treasure : Treasure
%         field secret : Secret
%         method take(scr : Secret){
%              if (this.secret==scr) then {
%                   t=treasure
%                   this.treasure = null
%                   return t } 
%          }
% }
%\end{lstlisting}

\subsection{ERC20}

The ERC20 \cite{ERC20} is a widely used token standard describing the basic functionality of any Ethereum-based token 
contract. 
This functionality includes issuing tokens, keeping track of tokens belonging to participants, and the 
transfer of tokens between participants. Tokens may only be transferred if there are sufficient tokens in the 
participant's account, and if either they (using the \prg{transfer} method) or someone authorised by the participant (using the \prg{transferFrom} method) initiated the transfer. 

For an $e:\prg{ERC20}$, the term $e.balance(p)$  indicates the number of tokens in   participant $p$'s  account at $e$.
The 
assertion $e.allowed(p,p')$ expresses that participant $p$ has been authorised to spend moneys from $p'$'s account at $e$.
 
The security model in Solidity is not based on having access to a capability, but on who the caller of a method is. 
Namely, Solidity supports the  construct \prg{sender} which indicates the identity of the caller.
Therefore, for Solidity, we adapt our approach in two significant ways:
we change the meaning of $\inside{\re}$ to express that $\re$ did not make a method call.
Moreover, we introduce a new, slightly modified form of two state invariants of the form $\TwoStates{\overline {x:C}}{A}{A'}$ which expresses that any execution which satisfies $A$, will preserve $A'$.
% \footnote{\red{I think that something deeper is going on here, I think that SOLIDITY id more in the style of ACLs, than OCAPs.}}

% \footnote{\red{AUTHORS: the way we had written the spec of ECR20 at OOPSLA, the tokens are taken out of the ERC20 and just disappear rather than being transferred to a another account. Was that correct?{ It would be easy to change}}}

We specify the guarantees of   ERC20  as follows:
\\
$\strut \SPSP  S_{erc\_1}\ \  \triangleq \ \ \TwoStatesN{ e:\prg{ERC20},p:\prg{Participant}}  {\  e.allowed(p,p)   \ } $ 
\\
$\strut \SPSP  S_{erc\_2}\ \  \triangleq \ \ \TwoStatesLB{ e:\prg{ERC20},p,p':\prg{Participant},n:\mathbb{N}} 
 {\ \forall p'.[\,(e.allowed(p',p) \rightarrow   \inside{p'}\, ] \ } { \ e.balance(b)=n \ } $ 
\\
$\strut \SPSP  S_{erc\_3}\ \  \triangleq \ \ \TwoStatesLB{ e:\prg{ERC20},p,p':\prg{Participant}}  {\ \forall p'.[\,(e.allowed(p',p) \rightarrow   \inside{p'}\, ] \ } { \ \neg (e.allowed(p'',p) \ } $ 

The specifications above say the following:
\\
\begin{tabular}{ll}
\begin{minipage}{.10\textwidth}
$\strut \SPSP  S_{erc\_1}$
\end{minipage}
&
\begin{minipage}{.85\textwidth}
guarantees that the the owner of an account is always authorized on that account -- this specification is expressed using the original version of two-state invariants.
\end{minipage}
\\
\\
\begin{minipage}{.10\textwidth}
$\strut \SPSP  S_{erc\_2}$ 
\end{minipage}
&
\begin{minipage}{.85\textwidth}
guarantees that any execution which does not contain calls from a participant $p'$ authorized on $p$'s account will not affect the balance of $e$'s account. Namely, if the execution starts in a state in which $ e.balance(b)=n$, it will lead to a state where $ e.balance(b)=n$ also holds.
\end{minipage}
\\
\\
\begin{minipage}{.10\textwidth}
$\strut \SPSP  S_{erc\_3}$ 
\end{minipage}
&
\begin{minipage}{.85\textwidth}
guarantees that any execution which does not contain calls from a participant $p'$ authorized on $p$'s account will not affect the balance of $e$'s account. That is, f the execution starts in a state in which $ \neg (e.allowed(p'',p)$, it will lead to a state where $ \neg (e.allowed(p'',p)$ also holds.
\end{minipage}
\end{tabular}

%$\strut \SPSP  S_{erc\_1}$ guarantees that the the owner of an account is always authorized on that account -- this specification is expressed using the original version of two-state invariants.
%\\
%$\strut \SPSP  S_{erc\_2}$ guarantees that any execution which does not contain calls from a participant $p'$ authorized on $p$'s account will not affect the balance of $e$'s account. Namely, if the execution starts in a state in which $ e.balance(b)=n$, it will lead to a state where $ e.balance(b)=n$ also holds.
%\\
%$\strut \SPSP  S_{erc\_3}$ guarantees that any execution which does not contain calls from a participant $p'$ authorized on $p$'s account will not affect the balance of $e$'s account. That is, f the execution starts in a state in which $ \neg (e.allowed(p'',p)$, it will lead to a state where $ \neg (e.allowed(p'',p)$ also holds.

\vspace{1cm}

We compare with the specifications given in \cite{OOPSLA22}:
 Firstly, \prg{ERC20Spec1} 
says that if the balance of a participant's account is ever reduced by some amount $m$, then
that must have occurred as a result of a call to the \prg{transfer} method with amount $m$ by the participant,
or the \prg{transferFrom} method with the amount $m$ by some other participant.
\begin{lstlisting}[language = Chainmail, mathescape=true, frame=lines]
ERC20Spec1 $\triangleq$ from e : ERC20 $\wedge$ e.balance(p) = m + m' $\wedge$ m > 0
              next e.balance(p) = m'
              onlyIf $\exists$ p' p''.[$\calls{\prg{p'}}{\prg{e}}{\prg{transfer}}{\prg{p, m}}$ $\vee$ 
                     e.allowed(p, p'') $\geq$ m $\wedge$ $\calls{\prg{p''}}{\prg{e}}{\prg{transferFrom}}{\prg{p', m}}$]
\end{lstlisting}
Secondly, \prg{ERC20Spec2} specifies under what circumstances some participant \prg{p'} is authorized to 
spend \prg{m} tokens on behalf of \prg{p}: either \prg{p} approved \prg{p'}, \prg{p'} was previously authorized,
or \prg{p'} was authorized for some amount \prg{m + m'}, and spent \prg{m'}.
\begin{lstlisting}[language = Chainmail, mathescape=true, frame=lines]
ERC20Spec2 $\triangleq$ from e : ERC20 $\wedge$ p : Object $\wedge$ p' : Object $\wedge$ m : Nat
              next e.allowed(p, p') = m
              onlyIf $\calls{\prg{p}}{\prg{e}}{\prg{approve}}{\prg{p', m}}$ $\vee$ 
                     (e.allowed(p, p') = m $\wedge$ 
                      $\neg$ ($\calls{\prg{p'}}{\prg{e}}{\prg{transferFrom}}{\prg{p, \_}}$ $\vee$ 
                              $\calls{\prg{p}}{\prg{e}}{\prg{allowed}}{\prg{p, \_}}$)) $\vee$
                     $\exists$ p''. [e.allowed(p, p') = m + m' $\wedge$ $\calls{\prg{p'}}{\prg{e}}{\prg{transferFrom}}{\prg{p'', m'}}$]
\end{lstlisting}
%\end{example}

\prg{ERC20Spec1} is related to $S_{erc\_2}$. Note that \prg{ERC20Spec1} is more API-specific, as it expresses the precise methods which caused the modificatiation of the balance.
% \red{... More comparisons to do here| }
%
%\subsection{Crowdsale}
\jm[]{\Nec is able to encode the motivating example of \citeasnoun{VerX}: 
an escrow smart contract that ensures that the contract may not be coerced to 
pay out or refund more money than has been raised.
The motivating \prg{Crowdsale} example consists of a \prg{Crowdsale} contract 
for crowd sourcing funding. A \prg{Crowdsale} object consists of an \prg{Escrow} object,
an amount raised, a funding goal, and a closing time in which the goal must be met for 
the fund to be successful. An \prg{Escrow} consists of a ledger of investors and how much
they have invested. There are several properties that \citeasnoun{VerX} sought to encode,
and we have provided the encoding of those specifications in Fig. \ref{f:verx:encoding}.
\prg{R0} states that if an investor claims a refund from an escrow, then the balance of 
the escrow decreases by the amount the investor had deposited in the escrow. 
\prg{R1} states that if at anytime the escrow has not yet succeeded, then the deposits must
be less than the balance of the escrow. 
\prg{R2\_1} and \prg{R2\_2} combine to express a single property: no one may ever withdraw and 
then subsequently claim a refund or visa versa.
\prg{R3} states that if the funding goal is ever met, then no one may subsequently claim a refund.}

\begin{figure}[htb]
\begin{lstlisting}[language=chainmail]
class Crowdsale {
Escrow escrow;
  closeTime, raised, goal : int;
  method init() {
    if escrow == null
      escrow := new Escrow(new Object);
    	  closeTime := now + 30 days;
    	  raised := 0;
    	  goal := 10000 * 10**18;
  }
  method invest(investor : Object, value : int) {
    if raised < goal
      escrow.deposit(investor, value);
      raised += value;
  }
  method close() {
    if now > closeTime || raised >= goal
      if raised >= goal
        escrow.close();
      else
        escrow.refund();
  }
}
\end{lstlisting}
\caption{Crowdsale Contract}
\label{f:verx:crowdsale}
\end{figure}

\begin{figure}[htb]
\begin{lstlisting}[language=chainmail]
confined class Escrow {
  owner, beneficiary : Object;
  mapping(Object => uint256) deposits;
  OPEN, SUCCESS, REFUND : Object;
  state : Object;
  method init(o : Object, b : Object) {
    if owner == null || beneficiary == null
      owner := o;
      beneficiary := b;
      OPEN := new Object; SUCCESS := new Object; REFUND := new Object;
      state := OPEN;
      
  method close() {state = SUCCESS;}
  method refund() {state = REFUND;}
  method deposit(value : int, p : Object) {
    deposits[p] := deposits[p] + value;
  }
  method withdraw() {
    if state == SUCCESS
      return this.balance;
  }
  method claimRefund(p : Object) {
    if state == REFUND
      int amount := deposits[p];
      deposits[p] := 0;
      return amount;
  }
}
\end{lstlisting}
\caption{Escrow Contract}
\label{f:verx:escrow}
\end{figure}

\begin{figure}[htb]
\begin{lstlisting}[mathescape=true, language=chainmail]
(R0) $\triangleq$ e : Escrow $\wedge$ $\calls{\_}{\prg{e}}{\prg{claimRefund}}{\prg{p}}$
          next e.balance = nextBal onlyIf nextBal = e.balance - e.deposits(p)
(R1) $\triangleq$ e : Escrow $\wedge$ e.state $\neq$ e.SUCCESS $\longrightarrow$ sum(deposits) $\leq$ e.balance
(R2_1) $\triangleq$ e : Escrow $\wedge$ $\calls{\_}{\prg{e}}{\prg{withdraw}}{\prg{\_}}$
           to $\calls{\_}{\prg{e}}{\prg{claimRefund}}{\prg{\_}}$ onlyIf false
(R2_2) $\triangleq$ e : Escrow $\wedge$ $\calls{\_}{\prg{e}}{\prg{claimRefund}}{\prg{\_}}$
           to $\calls{\_}{\prg{e}}{\prg{withdraw}}{\prg{\_}}$ onlyIf false
(R3) $\triangleq$ c : Crowdsale $\wedge$ sum(deposits) $\geq$ c.escrow.goal
         to $\calls{\_}{\prg{c.escrow}}{\prg{claimRefund}}{\prg{\_}}$ onlyIf false
\end{lstlisting}
\caption{Encoding VerX Crowdsale Example in Necessity}
\label{f:verx:encoding}
\end{figure}

 
