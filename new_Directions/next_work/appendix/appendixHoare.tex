\section{Appendix to Section \ref{sect:proofSystem} } \label{app:proof}
\label{app:hoare}


\subsection{Preliminaries: Specification Lookup,  Renamings, Underlying Hoare Logic}

Definition \ref{d:promises} is broken down as follows:  $S_1 \txtin  S_2$ says that $S_1$ is textually included in $S_2$; \ \ $S \thicksim S'$ says that $S$ is a safe renaming of $S'$; \ \   $\promises M S$ says that $S$ is a safe renaming of one of the specifications given for $M$. 
 
In particular, a safe renaming of  ${ \TwoStatesN {\overline {x:C}} {A}  }$ can replace any of the variables $\overline x$.  
A safe renaming  of ${\mprepostN{A_1}{p\ D}{m}{y}{D}{A_2} {A_3}}$  can replace  the formal parameters ($\overline y$) by actual parameters  ($\overline {y'}$) but requires the actual parameters  not to include \prg{this}, or \prg{res}, (\ie $\prg{this}, \prg{res}\notin \overline{y'}$). -- %The latter requirement 
Moreover, it can replace  the free variables which do not overlap with the formal parameters or the receiver ( $\overline{x}=\fv(A_1)\setminus\{{\overline y},\prg{this}\}$).

\begin{definition}
For a module $M$ and a specification $S$, we define:
\label{d:promises}
\begin{itemize}
\item
$S_1 \txtin  S_2  \ \ \ \triangleq\ \ \   S_1 \txteq  S_2$, or  $S_2 \txteq  S_1 \wedge S_3$, or $S_2\txteq S_3 \wedge S_1$,  or   $S_2 \txteq S_3 \wedge  S_1 \wedge S_4$ for some $S_3$, $S_4$.
\item
$S  \thicksim  S'$ \ \ \  is defined by cases

\begin{itemize}
\item
$ { \TwoStatesN {\overline {x:C}} {A}  }   \thicksim  { \TwoStatesN {\overline {x':C}} {A'[\overline{x'/x}]} } $
\item
$ {\mprepostN{A_1}{p\ D}{m}{y}{D}{A_2} {A_3}}  \thicksim
 {\mprepostN{A_1'}{p\ D}{m}{y'}{D}{A_2'} {A_3'}} $
  \\
 $\strut \hspace{2cm}    \ \  \triangleq\ \ \  A_1 = A_1'[\overline{y/y'}][\overline{x/x'}], \ \ A_2 = A_2'[\overline{y/y'}][\overline{x/x'}], \ \ A_3 = A_3'[\overline{y/y'}][\overline{x/x'}], \ \ \wedge$\\
 $\strut \hspace{2cm}\ \ \ \ \ \ \ \ \ \prg{this}, \prg{res}\notin \overline{y'}, \ \ \overline{x}=\fv(A_1)\setminus\{{\overline y},\prg{this} \} $
  \end{itemize} 
  
 \item  
 $\promises M  S \ \ \ \triangleq
 \ \ \ \exists S'.[ \ \ S'  \txtin \SpecOf M\ \ \wedge\ \ S' \thicksim S \ \ ]$
  \end{itemize} 
  \end{definition} 
  
The restriction on renamings of method specifications that  the actual parameters should not to include \prg{this}  or \prg{res} 
 is necessary because \prg{this} and \prg{res}  denote different objects from the point of the caller  than from the point of the callee.
It means that we are not able to verify a method call whose actual parameters include \prg{this} or \prg{res}. 
This is not a serious restriction: we can encode any such method call by preceding it with assignments to fresh local variables, \prg{this'}:=\prg{this}, and   \prg{res'}:=\prg{res}, and using \prg{this'} and \prg{res'} in the call.

 

\begin{example}
\label{e:rename}
The specification  from  Example \ref{example:mprepostl} can be renamed as 

\small
{
   {\sprepost
		{\strut \ \ \ \ \ \ \ \ \ S_{9r}} 
		{  a1:\prg{Account}, a2: \prg{Account}\wedge  \inside{a1}\wedge  \inside{a2.\prg{key}} }
		{\prg{public Account}} {\prg{set}} {\prg{nKey}:\prg{Key}}
		{   \inside{a1}\wedge  \inside{a2.\prg{key}} }
		{{   \inside{a1}\wedge  \inside{a2.\prg{key}} }}		
}}

\end{example}

\begin{axiom}
\label{ax:ul}
{Assume   Hoare logic with judgements % of the form 
\ $M \vdash_{ul} \{ A \} stmt \{ A' \}$, 
with  $\Stable{A}$,  $\Stable{A'}$. }
\end{axiom}

\subsection{Types}
\label{types}

The rules in Fig. \ref{f:types} allow triples to talk about the types 
Rule {\sc{types-1}} promises that types of local variables do not change.
Rule {\sc{types-2}} generalizes {\sc{types-1}} to any statement, provided that  there already exists a triple for that statement.

\begin{figure}[tht]
$
\begin{array}{c}
 \begin{array}{lcl}
\inferrule[\sc{types-1}]
	{  stmt \ \mbox{contains no method call} \\
	stmt  \ \mbox{contains   no assignment to $x$}}
	{\hproves{M}  {x:C} {\ stmt\ }{x:C} }
\\
\\
% THIS RULE IS NOT SOUND -- we need to adapt it
%{
\inferrule[\sc{types-2}]
	{ \hprovesN{M}  {A} {\ s\ }  {A'} {A''}  }
	{\hprovesN{M}  {x:C \wedge A} {\ s\ }  {x:C\wedge A'} {A''}}
% \end{array}
\end{array}
\end{array}
 $
\caption{Types}
\label{f:types}
\end{figure}


In {\sc{types-1}} we restricted to statements which do not contain method calls  in order to make lemma   \ref{l:no:meth:calls}  valid.

 
\subsection{Second Phase - more}

in Fig. \ref{f:substructural:app}, we    extend the Hoare Quadruples Logic with substructural rules, rules for conditionals, case analysis, and a contradiction rule.
For the conditionals we assume the obvious operational. semantics, but do not define it in this paper

\begin{figure}[htb]
$
\begin{array}{c}
\begin{array}{lcl}
\inferruleSD{[\sc{combine}]}
	{  \begin{array}{l}
	\hprovesN{M}  {A_1} {\ s\ } {A_2}  {A} \\ % \hspace{1.4cm}  
	\hprovesN{M}  {A_3} {\ s\ } {A_4} {A}
	\end{array}
	}
	{ \hprovesN{M}  {A_1 \wedge A_3 }{\ s\ } {A_2 \wedge A_4} {A} }
& &
\inferruleSD{[\sc{sequ}]}
	{  \begin{array}{l} 
	\hprovesN{M}  {A_1} {\ s_1\ } {A_2}  {A}  \\ % \hspace{1.4cm} 
	\hprovesN{M}  {A_2} {\ s_2\ } {A_3} {A}
	\end{array}
	}
	{   \hprovesN{M}  {A_1   }{\ s_1; \, s_2\ } {  A_3} {A} }
\end{array}
\\ \\
\inferruleSD{ \hspace{3cm} [\sc{consequ}]}
	{
%	\begin{array}{l}
	 { \hprovesN  {M}  {A_4} {\, s\, } { A_5} {A_6}  }
	 \hspace{1.4cm} 
	 M \vdash A_1 \rightarrow A_4 
	 \hspace{1.4cm} 
	{ M \vdash A_5   \rightarrow  A_2  }
	 \hspace{1.4cm}   
	{  M \vdash A_6 \rightarrow A_3 }
%	\end{array}
	}
	{   \hprovesN{M}  {A_1 }{\ s\ } {A_2} {A_3} }
  \end{array}
 $
 
 $
\begin{array}{c}
\inferruleSD{\hspace{2.5cm} [\sc{If\_Rule}]}
	{
	 \begin{array}{c}
	  \hprovesN {M}   
		{\  A \wedge Cond \  }
		{\    stmt_1   \ }
 		{\ A' \ }
		{\ A'' \ }
	\\
	    \hprovesN {M}   
		{\  A \wedge \neg Cond \  }
		{\    stmt_2   \ }
 		{\ A' \ }
		{\ A'' \ }	
	\end{array}
	}	
 	{  	
	\hprovesN {M}   
		{\  A \  }
		{\  \prg{if}\ Cond\ \prg{then}\ stmt_1\ \prg{else}\ stmt_2 \ \ }
		{\ A' \ }
		{\ A'' \ }
}
\\
\\
\begin{array}{lcl}
{
\inferruleSD{\hspace{0.5cm} [\sc{Absurd}]}
	{	
	}	 
 	{  	
	\hprovesN {M}   
		{\  false \  }
		{\  \ stmt \ \ }
		{\ A' \ }
		{\ A'' \ }
}
} & &
{
\inferruleSD{\hspace{0.5cm} [\sc{Cases}]}
	{ \begin{array}{l}
	\hprovesN {M}   
		{\  A \wedge A_{1}  \  }
		{\  \ stmt \ \ }
		{\ A' \ }
		{\ A'' \ }
		\\
		\hprovesN {M}   
		{\   A \wedge A_{2} \  }
		{\  \ stmt \ \ }
		{\ A' \ }
		{\ A'' \ }
	\end{array}	
	}	 
 	{  	
	\hprovesN {M}   
		{\  A \wedge (A_1 \vee A_2) \  }
		{\  \ stmt \ \ }
		{\ A' \ }
		{\ A'' \ }
}
}
\end{array}
\end{array}
$
 \vspace{-.5cm}
\caption{Hoare Quadruples -    substructural rules, and conditionals }
\label{f:substructural:app}
\end{figure}

\subsection{Extend the semantics and Hoare logic to accommodate scalars and conditionals}
\label{s:app:scalars}

{We extend the notion of protection to also allow it to apply to scalars. }

\begin{definition}[Satisfaction  of Assertions  -- Protected From]
\label{def:chainmail-protection-from-ext}
extending the definition of Def 
\ref{def:chainmail-protection-from}. We use $\alpha$ to range over addresses, $\beta$  to range over scalars, and   $\gamma$ to range over addresses or scalars.

\noindent
We define  $\satisfiesA{M}{\sigma}{\protectedFrom{{\gamma}} {{\gamma_{o}}}} $ as:
\begin{enumerate}
\item
\label{cProtectedNew}
 $\satisfiesA{M}{\sigma}{\protectedFrom{{\alpha}} {{\alpha_{o}}}}   \ \ \ \triangleq $ 
  \begin{itemize}
 \item
$\alpha\neq \alpha_0$,
 \ \ \ \  and% \\
 \item
$\forall n\in\mathbb{N}. \forall f_1,...f_n..
[\ \ \interpret{\sigma}{\alpha_{o}.f_1...f_n}=\alpha \ \ \  \Longrightarrow \ \ \  \satisfiesA{M}{\sigma}{ {\interpret{\sigma}{\alpha_{o}.f_1...f_{n-1}}}:C} \ \wedge \ C\in M\ \ ]$
\end{itemize}
\item
 $\satisfiesA{M}{\sigma}{\protectedFrom{{\gamma}} {{\beta_{o}}}}   \ \ \ \triangleq  \ \ \ true$
 \item
 $\satisfiesA{M}{\sigma}{\protectedFrom{{\beta}} {{\alpha_{o}}}}   \ \ \ \triangleq  \ \ \ false$
  \item
$\satisfiesA{M}{\sigma}{\protectedFrom{{\re}} {{\re_{o}}}} \ \ \ \triangleq $ \\
  $\exists \gamma, \gamma_{o}. [\  \ \eval{M}{\sigma}{{\re}}{\gamma}\ \wedge \eval{M}{\sigma}{{\re_0}}{\gamma_0} \  \wedge \ 
  \satisfiesA{M}{\sigma}{\protectedFrom{{\gamma}} {{\gamma_{o}}}}
 \ \  ]$
 \end{enumerate}
 \end{definition} 



{The definition from above gives rise to further cases of  protection; we supplement the triples from 
Fig. \ref{f:protection} with some further inference rules, given   in Fig. \ref{f:protection:conseq:ext}.}
Namely, any expression $\re$ is protected from a scalar (rules {\sc{Prot-In}}, {\sc{Prot-Bool}} and {\sc{Prot-Str}}).
Moreover, if starting at some $\re_o$ and following any   sequence of fields $\overline f$ we reach   internal objects or  scalars (\ie never reach an external object), then any $\re$ is protected from $\re_o$ (rule {\sc{Prot\_Intl}}).


\begin{figure}[htb]
\begin{mathpar}
\inferrule
	{M \vdash \re_o : \prg{int} \rightarrow \protectedFrom{\re}{\re_o} }
	{}
	\quad[\textsc{Prot-Int}]
	\and
\inferrule
	{M \vdash \re_o : \prg{bool} \rightarrow \protectedFrom{\re}{\re_o} }
	{}
	\quad[\textsc{Prot-Bool}]
	\and
\inferrule
	{M \vdash \re_o : \prg{str} \rightarrow \protectedFrom{\re}{\re_o} }
	{}
	\quad[\textsc{Prot-Str}]
%	THE BELOW IS USNOUND WHEN WE HAVE SCALARS\and
% {
%\inferrule
%	{M \vdash  \protectedFrom{\re}{\re_o}   \rightarrow \re \neq e'} 
%	{}
%	\quad[\textsc{Prot-Neq]}
%  }
  \and
  {
\inferrule
	{M \vdash  A \rightarrow \forall \overline f.[\ \re_o.\overline f\internal\, \vee\, \re_o.\overline f:\prg{int} \,  \vee\,    \re_o.\overline f:\prg{bool}\,  \vee\,  \re_o.\overline f:\prg{str}\ ] }
	{M \vdash  A \rightarrow \protectedFrom {\re} {\re_o} }
	\quad[\textsc{Prot-Intl]}
  } 
\end{mathpar}
\caption{Protection for Scalar and Internal Types}
%  Extended Consequence Rules for Protection that include rules for protection involving scalars.}
\label{f:protection:conseq:ext}
\end{figure}



 %
%
%The rules {\sc{Call\_Int}} and {\sc{Call\_Int\_Adapt}}   are weaker forms of  the  rule {\sc{Call\_Int\_Combine}} given here. 
%Even though  {\sc{Call\_Int\_Combine}} is sound, we did not include it in the presented system, for the sake of simplicity and separation of concerns.
%A similar stronger rule can be expressed for {\sc{Call\_Ext}} and  for {\sc{Call\_Ext\_Adapt}}.
% \\
%$
%{\begin{array}{c}
%  \inferruleSD  {\hspace{4.7cm} [\sc{Call\_Int\_Combine}]}
% 	{
% 	  	\begin{array}{c}
% 		  \promises  M { \mprepostN  {A_{1a} \wedge A_{1r} }  {p\ C} {m} {y} {C} { A_{2a} \wedge A_{2r}}   {A_3}  }
%		\\
%		A_1'\txteq    A_{1a}[y_0/\prg{this}]  \wedge {\PushAS {y}{A_{1r}[y_0/\prg{this}]}}
%		\\
%		A_2'\txteq    A_{2a}[u/res,y_0/\prg{this}]  \wedge {\PushAS {y}{A_{2r}[u/res,y_0/\prg{this}]}}\
%       	\end{array}
% 		}
% 	{  \hprovesN {M} 
%						{ \  y_0:C, {\overline {y:C}} \wedge A_1' \ }  
%						{ \ u:=y_0.m(y_1,.. y_n)\    }
%						{ \ A_2'\ }
%						{  \  A_3 \ }	
%      }
%\end{array}
%}

\vspace{1cm} Finally, we discuss the proof


\newcommand{\SP}{$\strut \ \ \ \ $}

\beginProof{l:no:meth:calls}
By induction on the rules in Fig. \ref{f:underly}.

\completeProof 

\subsection{Adaptation}
\label{appendix:adaptation}
 
 
 We now discuss the proof of Lemma \ref{lemma:push:ass:state}.

 \vspace{0.5cm}
 
 \beginProofSub{lemma:push:ass:state}{l:push:stbl}
$~$ \\
To Show: \ \ \  $\Stable{\,  \PushASLong {(y_0,\overline y)} A\, }$
\\
By structural induction on $A$.\\
\completeProofSub

\vspace{1cm}

For parts \ref{lemma:push:ass:state:one} and \ref{lemma:push:ass:state:two}, we first prove the following auxiliary lemma:

\begin{auxLemma}
\label{l:push:pop:aux}
For all $\alpha$,   $\overline {\phi_1}$, $\overline {\phi_2}$, $\overline {\phi_2}$, $\phi$ and $\chi$\\
$\strut ~ \ \ \ \ \ (L1)\ \ \    M, (\overline \phi_1,\chi) \models \protectedFrom \alpha {Rng(\phi)} \ \Longrightarrow \ M, (\overline \phi_2\cdot \phi,\chi) \models \inside \alpha$
\\
$\strut ~ \ \ \ \ \ (L2)\ \ \    M, (\overline {\phi_1}\cdot\phi,\chi) \models \inside \alpha   \wedge \extThis \ \ \Longrightarrow \ \ M, (\overline {\phi_2},\chi) \models \protectedFrom \alpha {Rng(\phi)} $
$\strut ~ \ \ \ \ \ (L3)\ \ \    M, (\overline {\phi_1},\chi) \models \inside \alpha   \wedge \extThis \ \ \wedge Rng(\phi)\subseteq \LRelevantO {\sigma_2} \ \ \  \Longrightarrow \ \ M, (\overline {\phi_2},\chi) \models \protectedFrom \alpha {Rng(\phi)} $
\\\end{auxLemma}

\begin{proof}
We first prove (L1): \\
~ \\
We define $\sigma_1 \triangleq (\overline {\phi_1},\chi)$, and  $\sigma_2 \triangleq (\overline {\phi_2}\cdot \phi,\chi) $.\\
The above definitions imply that: \\
\SP (1)\ \ $\forall \alpha',\forall \overline f.[\  \interpret {\sigma_1} {\alpha'.\overline f} =  \interpret {\sigma_2} {\alpha'.\overline f}\ ]$\\
\SP (2)\ \ $\forall \alpha'.[\  \Relevant {\alpha'} {\sigma_1} = \Relevant {\alpha'} {\sigma_2}\ ]$\\
\SP (3)\ \ $\LRelevantO {\sigma_2} = \bigcup_{\alpha'\in Rng(\phi)} \Relevant {\alpha'} {\sigma_2} $.\\
% $\strut ~ \ $\\
We now assume that\\
\SP (4)\ \ $M, \sigma_1 \models \protectedFrom \alpha {Rng(\phi)}$.\\
and want to show that\\
\SP (??)\ \ $M, \sigma_2 \models \inside \alpha$\\
From (4) and  by definitions, we obtain that\\
\SP (5)\ \ $\forall \alpha'\in Rng(\phi).\forall \alpha''\in \Relevant {\alpha'} {\sigma_1}.\forall f.[ \   M, \sigma_1 \models \alpha'':\prg{extl}\ \rightarrow \alpha''.f  \neq \alpha\ ]$, \ \ \ \ and also\\
\SP (6)\ \ $\alpha \notin Rng(\phi)$\\
From (5) and (3) we obtain:\\
\SP (7)\ \  $\forall \alpha' \in \LRelevantO {\sigma_2}.\forall f.[ \   M, \sigma_1 \models \alpha':\prg{extl}\ \rightarrow \alpha'.f  \neq \alpha\ ]$\\
From (7) and (1) and (2) we obtain:\\
\SP (8) \ \  $\forall \alpha' \in \LRelevantO {\sigma_2}.\forall f.[ \   M, \sigma_2 \models \alpha':\prg{extl}\ \rightarrow \alpha'.f  \neq \alpha\ ]$\\
From (8), by definitions, we obtain\\
 \SP (10)\ \ $M, \sigma_2 \models \inside \alpha$\\
 This completes the proof of (L1).
 \\
  $\strut ~ \ $\\
  We now prove (L2): \\
  ~ \\
 We define $\sigma_1 \triangleq (\overline {\phi_1}\cdot \phi,\chi)$, and  $\sigma_2 \triangleq (\overline {\phi_2},\chi) $.\\
The above definitions imply that: \\
\SP (1)\ \ $\forall \alpha',\forall \overline f.[\  \interpret {\sigma_1} {\alpha'.\overline f} =  \interpret {\sigma_2} {\alpha'.\overline f}\ ]$\\
\SP (2)\ \ $\forall \alpha'.[\  \Relevant {\alpha'} {\sigma_1} = \Relevant {\alpha'} {\sigma_2}\ ]$\\
\SP (3)\ \ $\LRelevantO {\sigma_1} = \bigcup_{\alpha'\in Rng(\phi)} \Relevant {\alpha'} {\sigma_1}$.\\
We   assume that\\
\SP (4)\ \  $M, \sigma_1 \models \inside \alpha \wedge \extThis$.\\
and want to show that\\
\SP (A?)\ \ $M, \sigma_2 \models \PushASLong  {Rng(\phi)} {A}$.\\
From (4), and unfolding the definitions, we obtain:\\
\SP (5)\ \  $\forall \alpha'\in \LRelevantO {\sigma_1}.\forall f.[ \   M, \sigma_1 \models \alpha':\prg{extl}\ \rightarrow \alpha'.f  \neq \alpha\ ]$, \ \ \ and\\
\SP (6)\ \ $\forall \alpha'\in Rng (\phi). [ \ \alpha'\neq \alpha \ ]$.\\
From(5), and using (3) and (2) we obtain:
\\
\SP (7)\ \  $\forall \alpha'\in Rng(\phi).\forall \alpha'' \in\Relevant {\alpha'} {\sigma_2}.\forall f.[ \   M, \sigma_2 \models \alpha'':\prg{extl}\ \rightarrow \alpha''.f  \neq \alpha\ ]$,
From (5) and (7) and by definitions, we obtain
\\
\SP (8)\ \  $\forall \alpha'\in Rng (\phi).[ \   \models \alpha \protectedFrom \alpha {\alpha'}\ ]$.\\
From (8) and definitions we obtain (A?).\\
This completes the proof of (L2). 
\end{proof}


\beginProofSub{lemma:push:ass:state}{lemma:push:ass:state:one}
$~$ \\
To Show: \ \ \  $(*)\ \ \ M, \sigma \models \PushASLong  {Rng(\phi)} {A}\  \ \ \ \ \  \ \ \    \Longrightarrow  \ \ \ \ M,  \PushSLong {\phi} {\sigma}   \models A$
%\\ 
%\SP \\
%We first prove the property that for all $\alpha$,   $\overline {\phi_1}$, $\overline {\phi_2}$, $\phi$ and $\chi$\\
%$\strut ~ \ \ \ \ \ (**)\ \ \    M, (\overline \phi_1,\chi) \models \protectedFrom \alpha {Rng(\phi)} \ \Longrightarrow \ M, (\overline \phi_2\cdot \phi,\chi) \models \inside \alpha$
%\\
%We define $\sigma_1 \triangleq (\overline {\phi_1},\chi)$, and  $\sigma_2 \triangleq (\overline {\phi_2}\cdot \phi,\chi) $.\\
%The above definitions imply that: \\
%\SP (1)\ \ $\forall \alpha',\forall \overline f.[\  \interpret {\sigma_1} {\alpha'.\overline f} =  \interpret {\sigma_2} {\alpha'.\overline f}\ ]$\\
%\SP (2)\ \ $\forall \alpha'.[\  \Relevant {\alpha'} {\sigma_1} = \Relevant {\alpha'} {\sigma_2}\ ]$\\
%\SP (3)\ \ $\LRelevantO {\sigma_2} = \bigcup_{\alpha'\in Rng(\phi)} \Relevant {\alpha'} {\sigma_2} $.\\
%% $\strut ~ \ $\\
%We now assume that\\
%\SP (4)\ \ $M, \sigma_1 \models \protectedFrom \alpha {Rng(\phi)}$.\\
%and want to show that\\
%\SP (??)\ \ $M, \sigma_2 \models \inside \alpha$\\
%From (4) and  by definitions, we obtain that\\
%\SP (5)\ \ $\forall \alpha'\in Rng(\phi).\forall \alpha''\in \Relevant {\alpha'} {\sigma_1}.\forall f.[ \   M, \sigma_1 \models \alpha'':\prg{extl}\ \rightarrow \alpha''.f  \neq \alpha\ ]$, \ \ \ \ and also\\
%\SP (6)\ \ $\alpha \notin Rng(\phi)$\\
%From (5) and (3) we obtain:\\
%\SP (7)\ \  $\forall \alpha' \in \LRelevantO {\sigma_2}.\forall f.[ \   M, \sigma_1 \models \alpha':\prg{extl}\ \rightarrow \alpha'.f  \neq \alpha\ ]$\\
%From (7) and (1) and (2) we obtain:\\
%\SP (8) \ \  $\forall \alpha' \in \LRelevantO {\sigma_2}.\forall f.[ \   M, \sigma_2 \models \alpha':\prg{extl}\ \rightarrow \alpha'.f  \neq \alpha\ ]$\\
%From (8), by definitions, we obtain\\
% \SP (10)\ \ $M, \sigma_2 \models \inside \alpha$\\
% This completes the proof of (**).
% $\strut ~ \ $\\
 By  induction on the structure of $A$. For the case where $A$ has the form $\inside {\alpha.\overline f}$, we use lemma \ref{l:push:pop:aux},(L1), taking $\overline {\phi_1} = \overline { \phi_2}$, and $\sigma \triangleq (\overline {\phi_1},\chi).$
\\
\completeProofSub
 

\vspace{1cm}

\beginProofSub{lemma:push:ass:state}{lemma:push:ass:state:two}
$~$ \\
To Show \ \ \  $(*)\ \ \  M,  \PushSLong {\phi} {\sigma}   \models  A  \wedge \extThis    \ \  \ \  \Longrightarrow  \ \ \ \ M, \sigma \models \PushASLong  {Rng(\phi)} {A}$ 
\\
$~$ \\
We apply induction on the structure of $A$. For the case where $A$ has the form $\inside {\alpha.\overline f}$, we apply lemma \ref{l:push:pop:aux},(L2), using    $\overline {\phi_1} = \overline { \phi_2}$, and $\sigma \triangleq (\overline {\phi_1},\chi).$

\completeProofSub

\vspace{1cm}
\beginProofSub{lemma:push:ass:state}{lemma:push:ass:state:three}
$~$ \\
To Show:\ \ \   (*) \ \  $M, \sigma  \models  A  \wedge \extThis  \ \wedge \ M\cdot\Mtwo \models \PushSLong {\phi} {\sigma}   \ \  \ \ \  \  \Longrightarrow  \ \ \ \ M, \PushSLong {\phi} {\sigma} \models \PushASLong  {Rng(\phi)} {A}$
\\ 
$~$ \\
By induction on the structure of $A$. 
 For the case where $A$ has the form $\inside {\alpha.\overline f}$, we want to apply lemma \ref{l:push:pop:aux},(L3). We take  $\sigma$ to be $ (\overline {\phi_1}, \chi)$, and $\overline {\phi_2}=\overline {\phi_2}\cdot \phi)$. Moreover,  $M\cdot\Mtwo \models \PushSLong {\phi} {\sigma}$ gives  that $Rng(\phi)\subseteq \LRelevantO {\sigma_2}$. Therefore, (*) follows by application of lemma \ref{l:push:pop:aux},(L3).
\completeProofSub

 
