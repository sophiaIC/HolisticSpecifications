Using our Hoare logic, we have developed a mechanised proof that, indeed, $M_{good} \vdash S_2 \wedge S_3$.
In appendix \ref{s:app:example}, included in the auxilliary material, we outline the main ingredients of that proof. 
We  expand our  semantics and logic to deal with scalars and conditionals, % and  rewrite the example in the syntax of \Loo.
and then highlight the most interesting proof steps of that proof.
The mechanised proof models \LangOO, the assertion language, the specification language, and our Hoare logic. 
Our mechanisation is primarily concerned with demonstrating the application of our approach to our running example, 
but does use several underlying definitions: i.e. an underlying traditional Hoare logic and simple properties of our 
assertion language. In particular, we assume the existence of an underlying Hoare logic (in the same manner as \S \ref{sect:proofSystem}), 
and make several assumptions of properties that underlying logic, all of which are standard in Hoare logics. 
We further assume several unsurprising rules of consequence for the underlying assertion language such as 
transitivity of consequence, the law of the excluded middle, or equality of field accesses from equality of the object.
All assumptions in our mechanised proof are clearly indicated in the associated artifact.


% Finally, we discuss why  $M_{good} \not\vdash S_2 $
