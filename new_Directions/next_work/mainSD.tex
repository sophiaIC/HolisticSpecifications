%% For double-blind review submission, w/o CCS and ACM Reference (max submission space)
%\documentclass[acmsmall,review]{acmart}\settopmatter{printfolios=true,printccs=false,printacmref=false}
%% For double-blind review submission, w/ CCS and ACM Reference
\documentclass[acmsmall,review,anonymous,screen]{acmart}\settopmatter{printfolios=true,printacmref=false}
%% For single-blind review submission, w/o CCS and ACM Reference (max submission space)
%\documentclass[acmsmall,review]{acmart}\settopmatter{printfolios=true,printccs=false,printacmref=false}
%% For single-blind review submission, w/ CCS and ACM Reference
%\documentclass[acmsmall,review]{acmart}\settopmatter{printfolios=true}
%% For final camera-ready submission, w/ required CCS and ACM Reference
%\documentclass[acmsmall]{acmart}\settopmatter{}
\usepackage[shortlabels]{enumitem}
\usepackage{mathtools}

\DeclareSymbolFont{arrows}{U}{FdSymbolC}{m}{n}

% \usepackage{ amssymb }
\DeclareTextFontCommand{\texttt}{\ttfamily}

%% Journal information
%% Supplied to authors by publisher for camera-ready submission;
%% use defaults for review submission.
\acmJournal{PACMPL}
\acmVolume{}
\acmNumber{POPL} % CONF = POPL or ICFP or OOPSLA
\acmArticle{}
\acmYear{2025}
\acmMonth{1}
\acmDOI{} % \acmDOI{10.1145/nnnnnnn.nnnnnnn}
\startPage{1}

%% Copyright information
%% Supplied to authors (based on authors' rights management selection;
%% see authors.acm.org) by publisher for camera-ready submission;
%% use 'none' for review submission.
\setcopyright{none}
%\setcopyright{acmcopyright}
%\setcopyright{acmlicensed}
%\setcopyright{rightsretained}
%\copyrightyear{2018}           %% If different from \acmYear

%% Bibliography style
\bibliographystyle{ACM-Reference-Format}
%% Citation style
%% Note: author/year citations are required for papers published as an
%% issue of PACMPL.
%\citestyle{acmauthoryear}   %% For author/year citations
\citestyle{acmnumeric}

%%%%%%%%%%%%%%%%%%%%%%%%%%%%%%%%%%%%%%%%%%%%%%%%%%%%%%%%%%%%%%%%%%%%%%
%% Note: Authors migrating a paper from PACMPL format to traditional
%% SIGPLAN proceedings format must update the '\documentclass' and
%% topmatter commands above; see 'acmart-sigplanproc-template.tex'.
%%%%%%%%%%%%%%%%%%%%%%%%%%%%%%%%%%%%%%%%%%%%%%%%%%%%%%%%%%%%%%%%%%%%%%


%% Some recommended packages.
\usepackage{booktabs}   %% For formal tables:
                        %% http://ctan.org/pkg/booktabs
\usepackage{subcaption} %% For complex figures with subfigures/subcaptions
                        %% http://ctan.org/pkg/subcaption
 \usepackage{ stmaryrd }                       

\usepackage{relsize}
\usepackage{mathpartir}
\usepackage{amsmath}
\usepackage{amsthm}
\usepackage{listings}
\usepackage{xspace}
\usepackage{definitions}
\usepackage{multirow,bigdelim}
\usepackage{pbox}
\usepackage{courier}
\usepackage{soul}
\usepackage{centernot}
 

\newcommand\multibrace[3]{\rdelim\}{#1}{3mm}[\pbox{#2}{#3}]}

\definecolor{ferngreen}{rgb}{0.31, 0.47, 0.26}
\newcommand{\kjx}[1]{{\color{ferngreen}{(KJX: #1)}}}
\newcommand{\scd}[1]{{\color{blue}{#1}}}
%\newcommand{\sdN}[1]{{\color{dkgreen}{#1}}}
%\newcommand{\jm}[1]{{\color{magenta}{JM: #1}}}
\newcommand{\sdcomment}[1]{{\ensuremath{\blacksquare}}\footnote{\color{dkgreen}{SD: #1}}}
\newcommand{\secomment}[1]{{\ensuremath{\blacksquare}}\footnote{\se{#1}}}
\newcommand{\jncomment}[1]{{\ensuremath{\blacksquare}}\footnote{\kjx{#1}}}

\newcommand{\sd}[1]{{\color{blue}{#1}}}
 \newcommand{\tobyM}[1]{#1} %[1]{{\color{purple}{Toby: #1}}}
\newcommand{\se}[1]{{\color{brown}{#1}}}
\definecolor{amber}{rgb}{1.0, 0.75, 0.0}
\definecolor{amethyst}{rgb}{0.6, 0.4, 0.8}

\newcommand{\ponders}[3]{\marginpar{\tiny\itshape\raggedright\textcolor{#2}{\textbf{#1:} #3}}\ignorespaces}
\marginparwidth=1.6cm \marginparsep=0cm
\newcommand{\TODO}[1]{} % {{\color{red}#1}}
\newcommand{\sophia}[1]{{\color{blue}#1}}
\newcommand{\toby}[1]{} % {\ponders{Toby}{purple}{#1}}
\newcommand{\susan}[2][]{\ponders{Susan}{brown}{#1} \textcolor{brown}{#2}\xspace}
\newcommand{\james}[1]{\ponders{James}{orange}{#1}}
\newcommand{\jm}[2][]{\ponders{Julian}{magenta}{#1} \textcolor{magenta}{#2}\xspace}
\newcommand{\mrr}[2][]{\ponders{Matthew Ross}{offblue}{{#1}} \textcolor{offblue}{{#2}}\xspace}
\renewcommand{\sdr}[2][]{\ponders{SD:}{blue}{#1} \textcolor{blue}{#2}\xspace}
\newcommand{\sdNr}[2][]{\ponders{SD:}{amber}{#1} \textcolor{amethyst}{#2}\xspace}
\newcommand{\sdN}[1]{{\color{teal}{#1}}}
\newcommand{\mrrz}[1]{\textcolor{offblue}{{#1}}\xspace}
\newcommand{\Mrr}[2][]{\ponders{Matthew Ross}{teal}{{#1}} \textcolor{teal}{{#2}}\xspace}
\newcommand{\Mrrz}[1]{\textcolor{teal}{{#1}}\xspace}
\newcommand{\julian}[1]{\textcolor{green}{#1}\xspace}
\newcommand{\sdM}[1]{\textcolor{amethyst}#1\xspace}
% \newcommand{\sdred}[1]{{\textcolor{amethyst}#1}} %{\textcolor{amethyst}#1\xspace}
\newcommand{\sophiaPonder}[2][]{\ponders{Sophia}{blue}{#1} \textcolor{blue}{#2}\xspace}
\renewcommand{\sophia}[2][]

\newcommand{\sdfootnote}[1]{\footnote{#1}}



\begin{document}

%% Title information
%\title[Specification and Proof of Necessary Conditions]{Specification
%and Proof of Necessary Conditions}         %% [Short Title] is
%optional;
%\title{Necessity Specifications are Necessary}
\title{Reasoning about External Calls}
% \title{\Nec Specifications are Necessary for Robustness}

                                        %% when present, will be used in
                                        %% header instead of Full Title.
%\titlenote{with title note}             %% \titlenote is optional;
                                        %% can be repeated if necessary;
                                        %% contents suppressed with 'anonymous'
%\subtitle{Subtitle}                     %% \subtitle is optional
%\subtitlenote{with subtitle note}       %% \subtitlenote is optional;
                                        %% can be repeated if necessary;
                                        %% contents suppressed with 'anonymous'


%% Author information
%% Contents and number of authors suppressed with 'anonymous'.
%% Each author should be introduced by \author, followed by
%% \authornote (optional), \orcid (optional), \affiliation, and
%% \email.
%% An author may have multiple affiliations and/or emails; repeat the
%% appropriate command.
%% Many elements are not rendered, but should be provided for metadata
%% extraction tools.


%% Author with two affiliations and emails.
\author{Julian Mackay}
%\authornote{with author2 note}          %% \authornote is optional;
                                        %% can be repeated if necessary
\orcid{0000-0003-3098-3901}             %% \orcid is optional
\affiliation{
  %\position{Position2a}
  %\department{Engineering and Computer Science}             %% \department is recommended
  \institution{Victoria University of Wellington}           %% \institution is required
  %\streetaddress{Street2a Address2a}
  %\city{City2a}
  %\state{State2a}
  %\postcode{Post-Code2a}
  \country{New Zealand}                   %% \country is recommended
}
\email{julian.mackay@ecs.vuw.ac.nz}         %% \email is recommended

%% Author with single affiliation.
\author{Sophia Drossopoulou}
%\authornote{with author1 note}          %% \authornote is optional;
                                        %% can be repeated if necessary
\orcid{0000-0002-1993-1142}             %% \orcid is optional
\affiliation{
  %\position{Position1}
  %\department{Department1}              %% \department is recommended
  \institution{Imperial College London}            %% \institution is required
  %\streetaddress{Street1 Address1}
  %\city{City1}
  %\state{State1}
  %\postcode{Post-Code1}
  \country{United Kingdom}                    %% \country is recommended
}
\email{scd@imperial.ac.uk}          %% \email is recommended

\author{James Noble}
%\authornote{with author2 note}          %% \authornote is optional;
                                        %% can be repeated if necessary
\orcid{0000-0001-9036-5692}             %% \orcid is optional
\affiliation{
  %\position{Position2a}
  %\department{Department2a}             %% \department is recommended
  \institution{Creative Research \& Programming}           %% \institution is required
  \streetaddress{5 Fernlea Ave, Darkest Karori}
  \city{Wellington}
  %\state{State2a}
  \postcode{6012}
  \country{New Zealand}                   %% \country is recommended
}
\email{kjx@acm.org}         %% \email is recommended


\author{Susan Eisenbach}
%\authornote{with author2 note}          %% \authornote is optional;
                                        %% can be repeated if necessary
\orcid{0000-0001-9072-6689}             %% \orcid is optional
\affiliation{
  %\position{Position2a}
  %\department{}             %% \department is recommended
  \institution{Imperial College London}           %% \institution is required
  %\streetaddress{Street2a Address2a}
  %\city{City2a}
  %\state{State2a}
  %\postcode{Post-Code2a}
  \country{United Kingdom}                   %% \country is recommended
}
\email{susan@imperial.ac.uk}         %% \email is recommended



%% Abstract
%% Note: \begin{abstract}...\end{abstract} environment must come
%% before \maketitle command


  

%% 2012 ACM Computing Classification System (CSS) concepts
%% Generate at 'http://dl.acm.org/ccs/ccs.cfm'.
\begin{CCSXML}
<ccs2012>
   <concept>
       <concept_id>10011007.10010940.10010992.10010993.10011683</concept_id>
       <concept_desc>Software and its engineering~Access protection</concept_desc>
       <concept_significance>300</concept_significance>
       </concept>
   <concept>
       <concept_id>10011007.10011074.10011099.10011692</concept_id>
       <concept_desc>Software and its engineering~Formal software verification</concept_desc>
       <concept_significance>500</concept_significance>
       </concept>
   <concept>
       <concept_id>10003752.10003790.10011741</concept_id>
       <concept_desc>Theory of computation~Hoare logic</concept_desc>
       <concept_significance>500</concept_significance>
       </concept>
   <concept>
       <concept_id>10011007.10011006.10011008.10011009.10011011</concept_id>
       <concept_desc>Software and its engineering~Object oriented languages</concept_desc>
       <concept_significance>300</concept_significance>
       </concept>
 </ccs2012>
\end{CCSXML}

\ccsdesc[300]{Software and its engineering~Access protection}
\ccsdesc[500]{Software and its engineering~Formal software verification}
\ccsdesc[500]{Theory of computation~Hoare logic}
\ccsdesc[300]{Object oriented programming~Object capabilities}



%% End of generated code

  
%% End of generated code


%% Keywords
%% comma separated list
%%%%%%%%%%%%%%%%%%%%%\keywords{keyword1, keyword2, keyword3}  %% \keywords are mandatory in final camera-ready submission



%% Note: \maketitle command must come after title commands, author
%% commands, abstract environment, Computing Classification System
%% environment and commands, and keywords command.


   \begin{abstract}
 
In today's complex software, internal, trusted, code % needs to cooperate with external code  of unknown provenance and trustworthiness.
is tightly intertwined  %cooperates 
with external, untrusted, code: external code may call into internal  code, and internal code may call out to external code.
% In particular, trusted  code may be called by external, unverified, untrusted code; moreover, it may call such external, unverified, untrusted code
This tight intertwining   introduces a high degree of uncertainty which developers strive to mitigate:
They %want to preserve   important properties of their data, and  
 want to prevent the external code %\susan{when calling into the internal code,} SD removed this bit -- not sure it helps
 from  
 causing certain effects unless it is ``entitled'' to.

The OCAP (object capabilities) model has proposed means to mitigate this uncertainty:
Capabilities are transferable rights to perform one or more operations on a given object.
Thus, capabilities \emph{enable} effects.  
But even more: % strongly than that:  
Capabilities \emph{guard} effects: callers can make the effect   happen, only if they have access to the corresponding capability.
%Capabilities are unforgeable, and can only be acquired through explicit passing from caller to callee, or at object creation.
  As a result, sharing of capabilities has to be carefully managed.
%
Our work addresses the specification and verification of code which relies on capabilities as guards.

% -- something not yet sufficiently  tackled by the OCAP  literature.
{\textbf {For the specification:}} To describe
 not being ``entitled'' to some effect, we give a means to guarantee  that lack of (eventual) access to certain capabilities ensures
that certain effects will not take place.
 To reflect that the guarantees
should hold in the presence of internal code calling and being called by external code, we 
% are about eventual} effects we 
propose \susan{\emph{scoped invariants}}, where the notion of the future is \susan{constrained} by the currently executing method. 
%We propose means to specify the preservation of such properties, and the meaning of being ``entitled'' to some effect. 
To define the   management of sharing of capabilities, we propose the concept of  protected capability which guarantees  that access by external objects
 is controlled by the internal objects.
  
%  
%  Our specification language can express that capabilities are necessary (rather than sufficient) conditions for certain effects
%(thus going further than traditional object capabilities).
% We propose 
% With these ingredients, internal code may call external code in the knowledge that ...
{\textbf {For the verification: }} We  propose Hoare logic inference rules  \sdN{to prove} that calls to external code are safe. 
We also propose inference rules  \sdN{to prove}    that  {our} modules satisfy our specifications.
We then \sdN{demonstrate}  that a sound   Hoare logic augmented by the proposed rules is also sound. 
We use a motivating example  and   prove its properties using our logic.


\end{abstract}

 % must come before maetitle
%
%\maketitle 
%\section{Introduction}
%Current systems are complex, and  built out of many different components of different provenance and different degree of trustworthiness.

%\begin{flushright}
%  \textit{nice things are nicer than nasty ones}\\
%  Kingsley Amis, ``Lucky Jim.''
%\end{flushright}

\begin{flushright}
  \textit{Condition B is hard to formalize, \\since it requires saying precisely what a bad
plan is,\\ and we do not attempt to do so.}\\
  Leslie Lamport
\end{flushright}
\susan[that is the best quote I could find - from Byzantine Generals]{}
\sophia[better than Amis's quote, I think]{}

The days of single, monolithic programs are long gone.  Contemporary
software is built over decades, by combining modules and components of
different provenance and different degrees of trustworthiness, and can
interact with almost every other program, device, or person.
In order for the resulting complex system to be correct, we need to be
able to reason about individual components to ensure that they behave
correctly, i.e.\ that good things happen when our programs run.
For example: if I send an email to a valid address, it will be
delivered to its recipient, or if I provide the right password, I can
transfer money from one of my bank accounts to another. 
To prove that good things can
happen, program verification systems can use witnesses, e.g.\ a
precondition, a postcondition describing the good thing (the desired
effect), and a code snippet, whose execution will establish the
effect, given the precondition.  The critical point here is that the
precondition is a \emph{sufficient} condition for the code snippet to
make the good thing happen: given the precondition, executing a
correct code snippet is guaranteed to achieve the postcondition.

Unfortunately, in a system of any complexity, knowing that good things
will happen is not enough: we also need to be sure that bad things
cannot happen. For example: we also need to be sure that an email can
only be read by the intended recipient; or that I can only transfer
money if I provide the right password. To address this problem,
we need to consider the \emph{necessary} conditions under which some
postcondition can be achieved (otherwise\susan[replaced 'or' with otherwise as I thought or was slightly confusing]{} some bad thing can happen):
it is necessary that someone is sent an email before they can read it;
it is necessary that the correct password is provided before money can
be transferred.  If a necessary condition can never be achieved, then
bad things guarded by that condition cannot happen.

The challenge here is twofold: How do we specify the bad things we are
concerned about, and how do we prove that the bad things we've
specified do not happen?  These challenges are difficult because we
cannot refer to just one component of a software system.  A sufficient
specification can deal with a single component in isolation --- a
single function for pre- and postconditions; a single class or data
structure for invariants. A necessary specification, however, must
provide guarantees which encompass the software system in its
entirety, and constrain the emergent behaviour of all its components,
for an open system, all possible sequences of API invocations.


Rather than considering the sufficient conditions to
achieve a given effect, we express necessary preconditions for the
holistic system. For example, a third party will not get to read my
email unless I forward it to them - the forward is a necessary
precondition for the reading. To reason about necessary conditions we
develop our Logic of Necessity. 

There has been work in the past to expand specifications of systems beyond the specification of good things.  blah, blah, blah  \citeauthor{FASE} were also concerned with 
 holistic specifications and necessary conditions, which they expressed through time operators.
 \Chainmail appears to be less rich than Chainmail, however we have been able to specify all their examples. In addition, \Chainmail has a proof system, which Chainmail lacked.
%To specify the bad things, we cannot
%refer to one particular piece of code, % and cannot refer to one witness;
% instead, we need to make a guarantee which encompasses all possible
% functions executable by a module in all their possible sequences or even interleavings --
 %we need to consider their \emph{emerging} behaviour.
 %Rather than considering the sufficient conditions to achieve a certain effect,
 %we express the necessary precondition. For example,    a third party 
 %will not get to read my email unless I forward it to them -- the forwarding
% is a necessary precondition for the reading.
 
 %To reason about necessary conditions, we develop a special
 %logic with such necessity triples. The most basic such holistic assertions 
 %can be derived from classical assertions (sufficient \jm[old:truples]{triples})
 %under assumptions of encapsulation;
 %they can then be further combined using our logic of necessity.




We illustrate our ideas in terms of the following example:
\prg{Account}s have a balance and a password. One may transfer
\prg{100} units from one \prg{Account} to another, but only provided
that the caller provides the right password. Below we show version
\sf{I} of the code for a class \prg{Account}.  We use a Java-like
syntax, and assume that fields are ``class-wide'' private (again as in
Java, so methods may read and write fields of any instance of that
class) and that passwords are unforgeable and not enumerable (again as
in Java, albeit without reflection).

 
\begin{lstlisting}[language=Chainmail]
class Account{
   field balance:int 
   field pwd: Password 
   method transfer(dest:Account, pwd':Password) -> void {
      if (this.pwd==pwd') {
            this.balance-=100;  dest.balance+=100; }  }
}
\end{lstlisting}

as well as a ``classical'' specification of the method \prg{transfer}:


 (ClassicSpec)$  \ \ $  $\triangleq$
\begin{lstlisting}[mathescape=true, frame=lines, language=Chainmail]
  method transfer(dest:Account, pwd':Password) -> void {
       ( PRE:  this.balance=bal1 $\wedge$ this.pwd==pwd' $\wedge$ dest.balance=bal2 $\wedge$ dest=/=this 
         POST: this.balance == bal1-100 $\wedge$  dest.balance == bal2+100 )
       ( PRE: this.balance=bal1 $\wedge$ this.pwd=/=pwd' $\wedge$ dest.balance=bal2
         POST: this.balance == bal1 $\wedge$  dest.balance=bal2 )
       ( PRE: a : Account $\wedge$ a=/=this $\wedge$ a=/=dest  $\wedge$ a.balance=bal $\wedge$ a.pwd=pwd1
         POST:  a.balance=bal $\wedge$ a.pwd=pwd1)
       ( PRE: a : Account $\wedge$ a.pwd=pwd1  
         POST: a.pwd=pwd1)       
\end{lstlisting}\footnote{Perhaps omit some of the lines here, but we do need them all in the full discussion}
 
  
 
 Now consider two further versions of the class account, given in Fig. \ref{fig:ExampleAccount}.
 In version {\sf{II}} and version {\sf{III}} the class has an additional method, \prg{set}, which enables the resetting of the password.
The method \prg{transfer} in all three versions of the class \prg{Account} satisfies the (ClassicSpec), 
however, while executing the first and third version of \prg{Account} won't exhibit unwanted behaviour, the second version doesn't preclude it.
%Namely version II allows any client to change the password of the account, and then to repeatedly withdraw money from it.
  
% On the other hand, we expect our software -- even if complex -- to provide some simple, high level
%guarantees, e.g. email addressed to me personally will not be read by a third party unless I explicitly 
%forwarded it to them.
%We except  our software to  behave correctly, even when used by a careless or malicious third party. 
%Such use of a software often consist of a sequence of actions performed on the module. 
%
%Software components respond to single actions, 
%or to sequences of such single actions. 
%When thinking about a software component we want think about the behaviour of each 
%action in isolation, but also about the \emph{emergent behaviour}, ie all
% the possible effects of the combinations of these actions. 
  
  
 
 \begin{figure}[hbtp]
 \begin{tabular}{lll}
\begin{minipage}{0.35\textwidth}
\begin{lstlisting}[language=chainmail]
class Account{
   field balance:int 
   field pwd: Password 
   method transfer(..) 
        ... as earlier ...
   method set(pwd':Password){
         this.pwd=pwd' }
}   
\end{lstlisting}
\end{minipage}
  &\ \ \  \ \   &
\begin{minipage}{0.50\textwidth}
\begin{lstlisting}[language=chainmail]
class Account{
   field balance:int 
   field pwd: Password 
   method transfer(..) 
        ... as earlier ...
   method set(pwd',pwd'':Password){
         if (this.pwd==pwd'){this.pwd=pwd''} }
}
\end{lstlisting}
\end{minipage} 
 \end{tabular}
  \caption{class \prg{Account} version II, and \ \ \ \ \ \  class \prg{Account} version III}
 \label{fig:ExampleAccount}
 \end{figure}

 
 The flaw in version {\sf{II}} arises from \emph{emergent} behaviour as \prg{set} 
 can be used to overwrite the
 password, and then using the new password \prg{transfer}  can be called.
% If we want the \prg{Account} class to be robust, we must prohibit the password from being freely available.
 Therefore, we propose a holistic specification which says that
 the \prg{balance} of an \prg{Account} reduces only if an object which does not belong to the
 class \prg{Account} has access to the password:
 
 \begin{lstlisting}[language = Chainmail, mathescape=true, frame=lines]
(HolisticSpec)  $\triangleq$  a:Account $\wedge$ a.balance==bal
                   to a.balance < bal
                   onlyIf $\exists$ o.[$\external{\texttt{o}}$ $\wedge$ $\access{\texttt{o}}{\texttt{a.pwd}}$]
\end{lstlisting}
 
%\jm[good point. fixed]{WHY DOES o HAVE A PRIME?}
 
 In more detail, the specification from above says that if in the current
 configuration \prg{a} is an \prg{Account},
 % with balance \prg{bal}, 
 and in some future configuration \prg{a} will have a balance less than the current one, then, in the \emph{current} configuration
 there must exist some object \prg{o}, which is \emph{external} to our module (does not belong to class
 \prg{Account}), and which has access to \prg{a}'s password.
 
 
 Thus, having access to the password is a necessary condition for the balance to reduce.
 Note, that the specification only talks about effects (here the balance reducing), and does not
 talk about individual methods (such as \prg{set} or \prg{transfer}).
 

 The holistic specification language, \Chainmail, extends traditional specifications with
notions of access, %(which object points to which)
control 
%(which methods are called)
\footnote{Sophia -perhaps drop as confusing? Susan -I think both of the bracketed phrases should be omitted, as we have already used the word access when describing the passwords and we don't mention control anywhere}, 
 the distinction between external and internal objects, and necessary conditions, 
 whereby the assertion $A~to~A'~onlyIf~A''$ expresses that if in in the current configuration
 $A$ holds, and in some future configuration $A'$, 
\susan[{changed first " to '}]{}
 will hold, then $A''$ must also hold in the 
 current configuration.  
  
 The contributions of this paper are as follows:
 
 \begin{enumerate}
 \item
 We propose \Chainmail, a specification language with which to
express holistic specifications. 
 \item
  We propose a Logic of Necessity for writing holistic assertions and for proving a module's adherence to said assertions.
 This logic builds on top of classical pre- post- conditions.
 \item
 We prove soundness of our logic.  
 \item
 We use our logic to prove adherence to the holistic assertion of the example
outlined  in this section
 \end{enumerate}
 
 We have developed a Coq proof of soundness of our approach, and a Cow proof of the
 example in this paper. We make these proofs available as supplementary material.
% QUESTIONS:
% \begin{itemize}
% \item
% holistic assertion -- different name or say we adopt from FASE; holistic -> comprehensive, integrated, aggregate \susan[I think we should keep holistic]{}
% \item
% Name for our logic?  Holistic logic? What is we have a different take on holistic next year? Necessity triples? 
%\item
%If our \Chainmail is simpler than the FASE Chainmail, then we need to argue that we can express all FASE examples
% \end{itemize}


The rest of the paper is organised as follows: .... 
%Section
%~\ref{sect:motivate:Bank} 
%\sd{gives an example from the literature} which we will use 
%to elucidate key points of \Chainmail.
%%motivates our work via an example, and then section
%~\ref{sect:chainmail} presents the \Chainmail\ specification
%language.  Section~\ref{sect:formal} introduces the formal model
%underlying \Chainmail, and then section~\ref{sect:assertions} defines
%the 
%semantics of \Chainmail's assertions.
%% SD the below is NOT ture
%%full details are relegated toappendices.   
%Section~\ref{sect:example} shows how key points of 
%exemplar problems can be specified in \Chainmail,
%section~\ref{sect:discussion}
%discusses our design, \ref{sect:related} considers related
%work, and section~\ref{sect:conclusion} concludes.
%We relegate various details to appendices.




%\section{The problem and our approach}
%\label{s:outline}  
% \section{The Meaning of Necessity}
\label{s:semantics}

 
In this section we define our \Nec specification language.
%In order to do that, 
\scd{We} first define 
an underlying programming language, \Loo  (\S \ref{sub:Loo}).
%modification to description of type system. Not sure if this is clear enough.]{
%We base \Loo on \LangOO, as defined by 
%\cite{FASE}. Whereas \LangOO was untyped, \Loo 
%%includes a simple type
%%system to allow for
%has type based restrictions on external access to private data.}
We then  define an assertion language, \SpecO,  which can talk about
 the contents of the state, as well \scd{as about} 
provenance, permission and control (\S \ref{sub:SpecO}).
Finally, we define the syntax and semantics of 
\Nec specifications  (\S \ref{s:holistic-guarantees}).
% our \Nec specification language  (\S \ref{s:holistic-guarantees}).


\subsection{\Loo}
\label{sub:Loo} 
%\jm[TODO: mention the type system and the restriction on external method calls]{}
%% We introduce a simple object-oriented language, \Loo, upon 
%% which our specification language sits.
 \Loo is a  simple  imperative, \scd{sequential}, 
class based, typed, object oriented language.
%\susan[Java fields are package private not class private - are fields so important that we need to say this?]{Fields are private to the class where they are defined.}
 Given its simplicity, %  the simplicity of \Loo, we do notdefine it here, instead, 
 we direct the reader to Appendix \ref{app:loo} for 
the full definitions, \scd{and} introduce here only % syntax and operational semantics.
 the concepts most relevant to the
treatment of the open world guarantees.
%\jm[]{\Loo fields are private in the way fields in Java are private,
%the privacy is class-wide, i.e. they may only be read or written to by 
%objects of the same class.}
\scd{\Loo is based} on \LangOO from
\cite{FASE}, \scd{with some small variations}, as well as 
the addition of a % while \LangOO is untyped, \Loo 
 a simple type system -- more in \sophiaPonder[TODO]{reference}.
%has type based restrictions on external access to private data.}


A \Loo state $\sigma$ consists of a 
heap $\chi$, and a  {stack $\psi$, which is a sequence of frames $\phi$}.
A frame $\phi$ consists of
local variable map, and a continuation, i.e. a sequence of statements to be executed.
 A statement may assign to variables, create new objects and push them to the heap, 
perform field reads and writes on objects,  or
 call methods on those objects. 

%Program 
 Modules are mappings
from class names to class definitions. 
Execution takes place in the context of  a module $M$ and   \scd{a state $\sigma$}.
It is % Execution
 \scd{defined} through a small-step, unsurprising, semantics, of the form \ \ 
   $M, \sigma \leadsto \sigma'$,\  c.f. Appendix \ref{app:loo}.
The   continuation of the top frame contains the statements \scd{that will be} % currently being 
executed next.
 % chopped, as generic 
 % There are several properties  of \Loo that are important to the central topic of this paper. 
 
As discussed in \S \ref{s:approach}, we are interested in guarantees which hold
\sophiaPonder[need to remind reader of internal/external]{during execution of an internal, 
known, trusted module $M$ when linked together with any
unknown, untrusted, module $M'$.} These guarantees need only hold 
when the external module is executing; \scd{we} are not concerned if they are
temporarily broken by the internal module. Therefore, we are only interested in states where the
executing object (the \prg{this}) is an external object. 
For this, we define the  \emph{external state semantics}, of the form 
$\reduction{M'}{M}{\sigma}{\sigma'}$, where $M'$ is the external
module, and $M$ is the internal module, and where we
collapse all internal steps into one single step.

 

\begin{definition}[External State Semantics]
\label{def:pair-reduce}
For  
% If we say "internal module", it is sounds as something makes the module be internal
  modules $M$,  $M'$, and program states $\sigma$, $\sigma'$, 
we say that $\ \ \ \ \ \ \ \ \reduction{M'}{M}{\sigma}{\sigma'}\ \ \ \ \ \ \ \ $ if and only if there exist 
$n\in\mathbb{N}$, \scd{and states $\sigma_0$,...$\sigma_n$}, such that
\begin{itemize}
% SD changed because the old version was slightly wrong
\item
$\sigma$=$\sigma_1$, and  $\sigma'$=$\sigma_n$,
\item
$M' \circ M, \sigma_i \leadsto \sigma_{i+1}$  \ \ \ for all $i\in [0..n)$,
\item
$\class{\sigma}{\sigma.\prg{this}}, \class{\sigma'}{\sigma'.\prg{this}}\in M'$,
\item
$\class{\sigma_i}{\sigma_i.\prg{this}} \in M$\ \ \ for all $i\in [1..n)$.
\end{itemize} 
\end{definition}
% In Definition \ref{def:pair-reduce}  % of external state semantics makes reference to a 
The function
$\class{\sigma}{x}$  looks up 
the class of   the object referred by variable $x$ in the heap of $\sigma$. 
  % for a specific variable in a specific program.
% SD not a variable, and no program.
The module linking operator $\circ$, applied to two modules, $M\circ M'$, % . The operator $\circ$
 %  results in the union of two disjoint modules.
combines the two modules into one module in the obvious way, provided their
domains are disjoint.
Full details in  Appendix \ref{app:loo}.
\begin{figure}[htb]
%  \vspace*{-2.5mm}
%  \begin{center}
%   \begin{minipage}{0.80\textwidth}
%     \begin{center}
%       \includegraphics[width=\linewidth]{diagrams/VisibleStates.pdf}
%     \end{center}
%   \end{minipage}
%   \end{center}
%   \vspace*{-2.5mm}
TODO new figure. \scd{should only have (A) and (B)}
   \caption{External States Semantics
     (Def. \ref{def:pair-reduce}). %
     % 
     (A) $\exec{{\color{blue}M'} \circ {\color{orange}M}}{\sigma_1}{\ldots}\leadsto \sigma_9$\ \ \and \ \ \ 
     (B) $\reduction{{\color{blue}M'}}{{\color{orange}M}}{\sigma_2}{\ldots}\leadsto \sigma_9$
    %  (c) $\reduction{{\color{orange}M'}}{{\color{blue}M}}{\sigma_1}{\ldots}\leadsto \sigma_8$
    }
   \label{fig:VisibleStates}
 \end{figure}
 
Fig. \ref{fig:VisibleStates} provides a simple graphical description of 
our external states semantics: \scd{(A) is the ``normal'' execution after 
linking two modules into one: \ $M' \circ M, ... \leadsto ...$. (B) is the
 external states execution when $M'$ is external,\   $\reduction{M'}{M}{...}{...}$  (B).}
\sophiaPonder[moved to earlier]{We  use} the notation\ \  $\reductions{M'}{M}{\sigma}{\sigma'}$ \ 
to denote
zero or more reduction steps starting at state $\sigma$ and ending at state $\sigma'$, in the context of internal module 
$M$ and external module $M'$.
 





\sophiaPonder[better connection, avoided repetition, tighten]{Not only are we unconcerned 
about internal states, 
%  (where \prg{this} belongs to a class from the inernal module),
 we are also unconcerned with  states which cannot ever arise from execution.
% 
% which are upheld by the internal 
%module. Moreover,    these guarantees  need to be satisfied only at `reachable' states,
%and need not be satisfied at states that are
%not reachable -- this is described formally in Definition \ref{def:arising}. 
\emph{Arising} states are those that  may arise by external states execution
% of  a given internal module linked with an external module.
% We describe the states of interest as the \emph{arising states}. 
starting at some initial configuration:}



\begin{definition}[Arising Program State]
\label{def:arising}
For   modules $M$ and  $M'$, a program state $\sigma$ is 
said to be an \emph{arising} state, formally \ \ \ $\arising{M}{M'}{\sigma}$,\ \ \ 
if and only if there exists some $\sigma_0$ such that $\initial{\sigma_0}$ and
$\reductions{M'}{M}{\sigma_0}{\sigma}$.
\end{definition}

% Definition \ref{def:arising} uses the definition for 
% In the definition above we used 
\scd{A state is \emph{Initial}} 
% he definition of which can be found in  Definition \ref{def:initial}. 
% which  characterize states at the start of program execution.
%The heap of an initial state should 
if its heap contains a single object of class \prg{Object}, and
its  stack   consists of a single frame, whose local variable map is a
mapping from \prg{this} to the single object, and whose continuation \scd{is}  any statement.
% to be executed.
More in Def. \ref{def:initial}and Def. \ref{def:arising}.
%%========================
%%old
%Finally, \Loo has a simple class based type system with the following properties:
%\begin{description}
%\item[(1)]
%Classes may be optionally annotated as \prg{enclosed}
%\item[(2)]
%Methods of non-\prg{enclosed} classes may not return objects of \prg{enclosed} classes
%\item[(3)]
%Modules are typed in isolation of other modules, thereby implicitly prohibiting
%method calls from internal objects to external objects.
%\end{description}
%The type system \jm[]{is included to allow a simple and convenient way to prove encapsulation of \SpecO assertions,
%however it is not fundamental to the definition of either the  \Nec specification language or logic, and could easily be substituted out 
%for any other encapsulation system.}
%%==================================

\jm[]{Finally, \Loo has a simple class based type system where 
classes may be optionally annotated as \enclosed. An object 
of an \enclosed type may not be accessed by module-external
objects. This helps simplify the demonstrative proof of the running 
Bank Account example (\S\ref{s:examples}). In order to enforce this 
restriction, methods of non-\enclosed classes are prohibited from
returning \enclosed objects.}

\jm[]{\Loo's type system has one further implication: modules are typed 
in isolation of other modules, thereby implicitly prohibiting
method calls from internal objects to external objects.}

\jm[]{The type system is included to allow a simple and convenient way to prove that access to certain 
objects is restricted. This restriction simplifies the proof of our running example to a manageable degree,
however it is not fundamental to the definition of either the  \Nec specification language or logic, 
and could easily be removed.}


\subsection{\SpecO}
\label{sub:SpecO}
\SpecO extends the expressiveness of standard specification languages
with assertion forms capturing key concepts of software security:
 \emph{permission}, \emph{provenance}, and \emph{control}.
 \emph{Permission} and \emph{Provenance} are inspired by the capabilities literature, while
\emph{Control} assists construction of proofs.
%That is, \SpecO specifications are able to specify which objects have
%access to which other object (\emph{permission}), whether an object's origin
%is internal or external to known code (\emph{viewpoint}), or which objects call which 
%methods (\emph{control}). 

\subsubsection{Syntax}

%\begin{figure}[t]
%\footnotesize
%\[
%\begin{syntax}
%\syntaxElement{A}{}
%		{
%		\syntaxline
%				{e}
%				{e : C}
%				{\neg A}
%				{A\ \wedge\ A}
%				{A\ \vee\ A}
%				{\all{x}{A}}
%				{\ex{x}{A}}
%		\endsyntaxline
%		}
%		{
%		\syntaxline
%				{\access{x}{y}}
%				{\internal{x}}
%				{\external{x}}
%		\endsyntaxline
%		}
%		{
%		\syntaxline
%				{\calls{x}{y}{m}{\overline{z}}}
%		\endsyntaxline
%		}
%\endSyntaxElement\\
%\end{syntax}
%\]
%\caption{\SpecO Assertions}
%\label{f:chainmail-syntax}
%\end{figure}
%

\begin{definition}
Assertions ($A$) in the basic specification language, \SpecO, are defined as follows

\label{f:chainmail-syntax}
%\footnotesize
\[
\begin{syntax}
\syntaxElement{A}{}
		{
		\syntaxline
				{e}
				{e : C}
				{\neg A}
				{A\ \wedge\ A}
				{A\ \vee\ A}
				{\all{x}{A}}
				{\ex{x}{A}}
		\endsyntaxline
		}
		{
		\syntaxline
				{\access{x}{y}}
				{\internal{x}}
				{\external{x}}
%		\endsyntaxline
%		}
%		{
%		\syntaxline
				{\calls{x}{y}{m}{\overline{z}}}
		\endsyntaxline
		}
\endSyntaxElement\\
\end{syntax}
\]


\end{definition}


Def. \ref{f:chainmail-syntax} gives the assertion syntax of \SpecO. % the \SpecO specification language.
An assertion may be an expression, a class assertion, the usual connectives and quantifiers, along 
with the following non-standard assertion forms:
\begin{itemize}
\item
\emph{Permission} ($\access{x}{y}$): % Which objects have access to which other objects (i.e.
  $x$ has access to $y$.
\item
{\emph{Provenance}} ($\internal{x}$ and $\external{y}$): %Which objects are internal or external to our component.
 $x$ is internal, and $y$ is external.
\item
\emph{Control} ($\calls{x}{y}{m}{\overline{z}}$): 
$x$ calls method $m$ on object $y$ with arguments $\overline{z}$.
\end{itemize}

\subsubsection{Semantics of \SpecO}
The semantics of \SpecO assertions is given in Definition \ref{def:chainmail-semantics}. 
\SpecO makes use of several language features of 
\Loo that can be found in Appendix \ref{app:loo}. Specifically, $\eval{M}{\sigma}{e}{v}$
is the evaluation relation for expressions, and is interpreted as expression $e$ evaluates
to value $v$ in the context of program state $\sigma$, with module $M$. The full
semantics of expression evaluation are given in Fig. \ref{f:evaluation}. It should 
be noted that expressions in \Loo may be recursively defined, and thus evaluation may not
necessarily terminate, however the logic remains classical because recursion is restricted
to expressions, and not generally to assertions.

Further, Definition \ref{def:chainmail-semantics} uses the interpretation of variables
within a specific frame or state: i.e. $\interpret{\phi}{x} = v$, meaning that $x$ maps to
value $v$ in the local variable map of frame $\phi$, and $\interpret{\sigma}{x} = v$ meaning $x$ 
maps to value $v$ in the top most frame of $\sigma$'s stack. And the term  $\interpret{\sigma}{x.f} = v$
has the obvious meaning.


\begin{definition}[Satisfaction of \SpecO Assertions] 
\label{def:chainmail-semantics}
We define satisfaction of an assertion $A$ by a program state $\sigma$ with internal module $M$ as:
\begin{itemize}
\item
$\satisfiesA{M}{\sigma}{e}$ \ \ \ iff \ \ \  $\eval{M}{\sigma}{e}{\true}$
\item
$\satisfiesA{M}{\sigma}{e : C}$ \ \ \ iff \ \ \  $\eval{M}{\sigma}{e}{\alpha}$ \textit{and} $\class{\sigma}{\alpha} = C$
\item
$\satisfiesA{M}{\sigma}{\neg A}$ \ \ \ iff \ \ \  ${M},{\sigma}\nvDash{A}$
\item
$\satisfiesA{M}{\sigma}{A_1\ \wedge\ A_2}$ \ \ \ iff \ \ \  $\satisfiesA{M}{\sigma}{A_1}$ and 
$\satisfiesA{M}{\sigma}{A_2}$
\item
$\satisfiesA{M}{\sigma}{A_1\ \vee\ A_2}$ \ \ \ iff \ \ \  $\satisfiesA{M}{\sigma}{A_1}$ or 
$\satisfiesA{M}{\sigma}{A_2}$
\item
$\satisfiesA{M}{\sigma}{\all{x}{A}}$ \ \ \ iff \ \ \  
$\satisfiesA{M}{\sigma[x \mapsto \alpha]}{A}$, \ 
\ \ for some $x$ fresh in $\sigma$, and for all $\alpha\!\in\!\sigma.\prg{heap}$.
\item
$\satisfiesA{M}{\sigma}{\ex{x}{A}}$ \ \ \ iff \ \ \  
$\satisfiesA{M}{\sigma[x \mapsto \alpha]}{A}$, \ 
\ \ for some $x$ fresh in $\sigma$, and for some $ \alpha\!\in\!\sigma.\prg{heap}$. 
\item
$\satisfiesA{M}{\sigma}{\access{x}{y}}$ \ \ \ iff \ \ \  
\begin{itemize}
\item
$\interpret{\sigma}{x.f}={\interpret{\sigma}{y}}$ for some $f$, \  or
\item
there exists some $z$, and some frame $\phi$ in the stack of $\sigma$ such that {$\interpret{\sigma}{x}=\interpret{\phi}{\prg{this}}$}, {and $\interpret{\sigma}{y}=\interpret{\phi}{z}$}
\end{itemize}
\item
$\satisfiesA{M}{\sigma}{\internal{x}}$ \ \ \ iff \ \ \  
$\textit{classOf}(\sigma,x) \in M$
\item
$\satisfiesA{M}{\sigma}{\external{x}}$ \ \ \ iff \ \ \  
$\textit{classOf}(\sigma,x) \not\in M$
\item
$\satisfiesA{M}{\sigma}{\calls{x}{y}{m}{z_1, \ldots, z_n}}$ \ \ \ iff \ \ \ 
\begin{itemize}
\item
$\sigma.\prg{contn} = (\_ := y'.m(z'_1,\ldots,z'_n))$, % and is superfluous, enums are ands, unless expltly stated   
\item
$\satisfiesA{M}{\sigma}{x = \prg{this}}$
%$\interpret{\sigma}{x} = \interpret{\sigma}{\prg{this}}$  % and
\item
$\satisfiesA{M}{\sigma}{y = y'}$
%$\sd{\interpret{\sigma}{y} = \interpret{\sigma}{y'}}$ % and
\item
$\satisfiesA{M}{\sigma}{z_i = z'_i}$\ \ \ for all $1\!\leq i\!\leq n$
%$\interpret{\sigma}{z_i} = \interpret{\sigma}{z'_i}$ \ \ \ for all $1\!\leq i\!\leq n$
\end{itemize}
\end{itemize}
\end{definition}

 
Finally, we define what it means for a module to satisfy an assertion:
a module $M$ satisfies an assertion $A$, if all states $\sigma$
arising from external steps execution of that
module with any other external module, satisfy $A$. 
 
\begin{definition} [Assertion Satisfaction by Modules]
\label{def:mdl-sat}
For a module $M$ and assertion $A$, we say that\ \  $\satisfies{M}{A}$ \ \ if and only if 
for all modules $M'$, and all $\sigma$, if $\arising{M'}{M}{\sigma}$, then $\satisfiesA{M}{\sigma}{A}$.
\end{definition}


% Thus, satisfaction by a module  allows us to talk 
% about what is true for a given module without introducing the 
% details of specific program configurations, a critical component 
% of constructing our Logic of Necessity in Section \ref{s:inference}. 

A proof system for such assertions is indicated by a judgment of the form $\proves{M}{A}$. 
We will not define such a judgment, and just rely on its existence (cf. Theorem 3.2).
We define soundness of such a judgment in the usual way:

\begin{definition}[Soundness of \SpecO Provability]
\label{ax:specW-prove-soundness}
A judgment of the form $M \vdash A$ is \emph{sound}, if for all
  all modules $M$ and assertions $A$, \ if $\proves{M}{A}$ then $\satisfies{M}{A}$.
\end{definition}

\subsubsection{Wrapping}

We define a useful shorthand: the $\wrapped{}$ predicate  states 
that only \internalO objects have access to some object.
That object may be either \internalO or \externalO.
\begin{definition}[Wrapped]
$\wrapped{o}\ \triangleq\ \all{x}{\neg \access{x}{o}\ \vee\ \internal{x}}$
\end{definition}
Wrapped is critical as it captures the conditions under \jm[]{an interaction
with the \internalM module is necessary}. 
\jm[]{As an example, consider the Bank Account example from Section \ref{s:outline}: if} only \internalO
objects have access to an account's password, then
it follows that access to the password may not 
be gained except by an interaction with the \internalM
module, and subsequently if the \internalM module
is secure we know that the password cannot be leaked.
\jm[]{This predicate forms a core aspect of Part 2 mentioned in Section \ref{s:outline},
and is elaborated on more in Section \ref{s:classical-proof}.}
 
 

%\begin{figure}[t]
%\begin{mathpar}
%\infer
%		{M;\ M',\ \sigma\ \vdash\ e : \prg{intrnl}}
%		{M;\ M',\ \sigma\ \vdash\ e : \prg{encap}}
%		\and
%\infer
%		{M;\ M',\ \sigma\ \vdash\ e : \prg{intrnl}}
%		{M;\ M',\ \sigma\ \vdash\ e.f : \prg{encap}}
%		\and
%\infer
%		{M;\ M',\ \sigma\ \vdash\ e : \prg{intrnl}}
%		{M;\ M',\ \sigma\ \vdash\ e.g(e') : \prg{encap}}
%\end{mathpar}
%\caption{Encapsulated Expressions}
%\label{f:intrnl}
%\end{figure}
	
%	\begin{figure}[h]
%	\[
%	\begin{array}{llr}
%	A & ::= & \textit{Assertions}\\  
%	| & e & \\
%	| & e\ :\ C & \\
%	| & e\ \in\ S & \\
%	| & A\ \prg{in}\ S & \\
%	| & \access{x}{y} \\
%	| & \internal{x} \\
%	| & \external{x} \\
%%	| & \mut x y f &\\
%%	| & \gives x y z &\\
%	| & \calls{x}{y}{m}{args} \\
%	| & \changes{S}{A} \\
%	| & \neg A & \\
%	| & A\ \wedge\ A & \\
%	| & A\ \vee\ A & \\
%	| & A\ \longrightarrow\ A & \\
%	| & \forall\ x.\ [A] & \\
%	| & \exists\ x.\ [A] & \\
%	| & \forall\ S.\ [A] & \\
%	| & \exists\ S.\ [A] &
%	\end{array}
%%	\begin{array}{llr}
%%	s & ::= & \textit{Source}\\
%%	| & \prg{int} & \\
%%	| & \prg{ext} & \\
%%	| & \_ &
%%	\end{array}
%	\]
%	\caption{Assertions}
%	\label{f:assertions_triple2}
%	\end{figure}





\subsection {\Nec Specifications}
\label{s:holistic-guarantees}

Our \Nec specification language extends \SpecO with novel 
%\emph{Necessity Specifications}.
\sophiaPonder[]{\emph{necessity operators}}.
In this Section we define its syntax (Def. \ref{f:holistic-syntax}) and semantics 
%(Def. \ref{d:necessOpers}) 
(Def. \ref{def:necessity-semantics})
%of
%\emph{\NecessitySpecifications}.
We have the following three operators:

% described below:





\paragraph{Only If}
[$\onlyIf{A_1}{A_2}{A}$]: If an arising program state satisfies $A_1$, and after some execution, a state program state satisfying $A_2$ is reached, 
then the original program state must have also satisfied $A$.
e.g. if the balance of a bank account changes over time, then there must be some external object in the current 
program state that has access to the account's password.

\paragraph{Single-Step Only If}
[$\onlyIfSingle{A_1}{A_2}{A}$]: If an arising program state satisfies $A_1$, and after a single step of execution, a state satisfying $A_2$ is reached, 
then the original program state must have also satisfied $A$.
e.g. if the balance of a bank account changes over a single execution step, then that execution step must be a method call to the bank \prg{transfer} method.

\paragraph{Only Through}
[$\onlyThrough{A_1}{A_2}{A}$]: If an arising program state satisfies $A_1$, and after some execution, a state satisfying $A_2$ is reached, then program execution must have passed through some state satisfying $A$.
e.g. if the balance of an account changes over time, then the bank's \prg{transfer} method must have been called 
in some intermediate state. Note 
that the intermediate state where $A$ is true might be the initial state,
or final state.

%\begin{figure}[t]
\begin{definition}
Assertions ($H$)
% in   \Chainmail \NecessitySpecifications  
are defined as follows:
% \footnotesize

\[
\begin{syntax}
\syntaxElement{H}{}
		{
		\syntaxline
				{A}
				{\onlyIf{A_1}{A_2}{A_3}}
				{\onlyThrough{A_1}{A_2}{A_3}}
		\endsyntaxline
		}
		{
		\syntaxline
				{\onlyIfSingle{A_1}{A_2}{A_3}}
		\endsyntaxline
		}
\endSyntaxElement\\
\end{syntax}
\]
%\caption{Syntax of \Chainmail Necessity Specifications}
\label{f:holistic-syntax}
\end{definition}
%\end{figure}


All three operators talk about assertions satisfied in the 
current state as well as of assertions satisfied in some future state. 
These assertions may contain variables, whose denotation might change during
program execution: the 
map may change, variables may be overwritten, or the entire local variable maps may be lost on a method return.
For this reason, before we provide the semantics of our \Nec specification language, we first introduce an adaptation operator
to account for variable renaming throughout the execution of a program.
\begin{definition}
\label{d:adapt}
$\adapt{\sigma'}{\sigma} \triangleq (\chi, \{\prg{local} := \beta[\overline{z} \mapsto \beta(\overline{z}')], \prg{contn}:= [\overline{z'}/\overline{z}]c\} : \psi)$
where 
\begin{itemize}
\item
$\sigma = (\_, \{\prg{local}:=\beta; \prg{contn}:=\_\} : \_)$, and
$\sigma' = (\chi, \{\prg{local}:=\beta', \prg{contn}:=c\} : \psi)$
\item
$dom(\beta') = \overline{z'}$, $dom(\beta) \cap \overline{z} = \emptyset$, and $|\overline{z'}| = |\overline{z}|$
\end{itemize}
\end{definition}
\jm[]{Def. \ref{d:adapt} allows satisfaction to take variable renaming during evaluation into account. 
As an example, consider the following code snippet.}
\begin{lstlisting}[frame=lines]
x.f := y
x := z
\end{lstlisting}
\jm[]{After the evaluation of line 1 , \texttt{x} has access to \texttt{y}, however when we overwrite \texttt{x} in line 2, 
this is no longer necessarily true, as \texttt{x} might now refer to a different object, even though the object previously referred to 
by \texttt{x} has not changed. A similar issue might occur when either calling a method, or returning from a method, as 
the entire variable map changes under such circumstances. Adaptation (Def. \ref{d:adapt}) provides a convenient way to refer to 
objects across time, while ignoring rewrites and new frames.}


We can now define our semantics,  $M \models H$, 
%the semantics of the Necessity Specifications,
%  in Definition \ref{def:necessity-semantics}.  The definition goes 
by cases over the \sophiaPonder[we were saying three forms]{four}possible syntactic forms of $H$: 


\noindent
\begin{definition}[\Nec Semantics]\susan[Can this be called Semantics or Necessity Specification Language]{}\jm[used to be called ``Necessity Specifications'']{}
\label{def:necessity-semantics}
For any assertions $A$, $A_1$, $A_2$, and $A$,  we define \\


$\bullet$ \ $\satisfies{M}{{A}}$ \ \ \ iff\ \ \ for all $M'$, $\sigma$,\ if $\arising{M}{M'}{\sigma}$, then $\satisfiesA{M}{\sigma}{A}$. (see Def. \ref{def:mdl-sat})\\

%$\bullet$ \ $\satisfies{M}{{A}}$ \ \ \ as defined in \ref{def:mdl-sat} \\

$\bullet$ \ $\satisfies{M}{\onlyIf {A_1}{A_2}{A}}$ \ \ iff\ \  for all $M'$, $\sigma$, $\sigma'$, such that $\arising{M}{M'}{\sigma}$; \\ % and\\

\begin{tabular}{lr}
$\;\;\;\;$- $\satisfiesA{M}{\sigma}{A_1}$  & \rdelim\}{3}{3mm}[$\;\;\;\Rightarrow\;\;\;$  $\satisfiesA{M}{\sigma}{A}$] \\
$\;\;\;\;$- $\satisfiesA{M}{\sigma' \triangleleft \sigma}{A_2}$   \\
$\;\;\;\;$- $\reductions{M}{M'}{\sigma}{\sigma'}$   \\
\end{tabular}\\ 

$\bullet$ \  $\satisfies{M}{\onlyIfSingle {A_1}{A_2}{A}}$\ \ iff\ \   for all $M'$, $\sigma$,   $\sigma'$, such that $\arising{M}{M'}{\sigma}$: \\

\begin{tabular}{lr}
$\;\;\;\;$- $\satisfiesA{M}{\sigma}{A_1}$  & \rdelim\}{3}{3mm}[$\;\;\;\Rightarrow\;\;\;$  $\satisfiesA{M}{\sigma}{A}$] \\
$\;\;\;\;$- $\satisfiesA{M}{\sigma' \triangleleft \sigma}{A_2}$   \\
$\;\;\;\;$- $\reduction{M}{M'}{\sigma}{\sigma'}$   \\
\end{tabular}\\ 

%% here as it was 
%$\bullet$ \  $\satisfies{M}{\onlyThrough {A_1}{A_2}{A}}$ \ \ iff\ \  for all $M'$, $\sigma$,   $\sigma'$, such that $\arising{M}{M'}{\sigma}$, and \\
%\begin{tabular}{lr}
%$\;\;\;\;$- $\satisfiesA{M}{\sigma}{A_1}$  & 
%\rdelim\}{3}{3mm}%[\makecell{Some really \\ longer text}]
%[$\;\;\;\Rightarrow\;\;\;$\pbox{9cm}{then for all $\sigma_1, \ldots, \sigma_n$ such that $\reduction{M}{M'}{\sigma}{\sigma_1}\leadsto \ldots \sigma_n \leadsto \sigma'$
%there exists some $\sigma_i$ such that $\satisfiesA{M}{\sigma_i \triangleleft \sigma}{A}$ where $0\leq i \leq n$, or $\satisfiesA{M}{\sigma}{A}$, or $\satisfiesA{M}{\sigma' \triangleleft \sigma}{A}$}] \\
%$\;\;\;\;$- $\satisfiesA{M}{\sigma' \triangleleft \sigma}{A_2}$   \\
%$\;\;\;\;$- $\reductions{M}{M'}{\sigma}{\sigma'}$   \\
%\end{tabular}\\ 
%$\bullet$ \  $\satisfies{M}{\onlyThrough {A_1}{A_2}{A}}$ \ \ iff\ \  for all $M'$, $\sigma_1$,   $\sigma_n$, such that $\arising{M}{M'}{\sigma}$: \\
  
$\bullet$ \  $\satisfies{M}{\onlyThrough {A_1}{A_2}{A}}$ \ \ iff\ \  for all $M'$, $\sigma_1$,   $\sigma_n$, such that $\arising{M}{M'}{\sigma_1}$: \\

\begin{tabular}{lr}
$\;\;\;\;$- $\satisfiesA{M}{\sigma_1}{A_1}$  & 
\rdelim\}{3}{3mm}%[\makecell{Some really \\ longer text}]
[$\;\;\;\Rightarrow\;\;\;$\pbox{9cm}{$\forall \sigma_2, \ldots, \sigma_{n-1}$.  \\ 
(\ \ $\forall i\!\in\![1..n).\ \reduction{M}{M'}{\sigma_i}{\sigma_{i+1}}$   \ $\Rightarrow$
$\exists i\!\in\![1..n]. \  \satisfiesA{M}{\sigma_i \triangleleft \sigma_1}{A}$ \ \ )   }] \\
$\;\;\;\;$- $\satisfiesA{M}{\sigma_n\triangleleft \sigma}{A_2}$   \\
$\;\;\;\;$- $\reductions{M}{M'}{\sigma}{\sigma_n}$   \\
\end{tabular} 
\end{definition} 



 
%With our language as defined in Definition \ref{def:necessity-semantics},
We are now able to state what the necessary preconditions to critical functions in 
software are, including safety properties of software in the open world. The semantics
of \emph{Single-Step Only If} allow for the statement of such necessary preconditions
for any execution step for any program to achieve a certain outcome. The semantics
of \emph{Only If} and \emph{Only Through} allow us to raise these necessary preconditions
to any arbitrary number of execution steps, and thus allow for reasoning about 
the execution of an entire program.
 
Looking back at the example from the Introduction,   it holds that
\\
\strut $\hspace{1in}$ \prg{Mod1} $\models$ \prg{NecessityBankSpec}
 \\
\strut $\hspace{1in}$ \prg{Mod2} $\not\models$ \prg{NecessityBankSpec}
 \\
\strut $\hspace{1in}$ \prg{Mod3} $\models$ \prg{NecessityBankSpec}
 

 
For more specification examples, consider the
bank account discussed in Section \ref{s:intro}. We have already shown
how we can specify knowledge of an account's password using \prg{NecessityBankSpec},
but we are also able to write other useful properties about the bank account. 
 
\begin{lstlisting}[language = Chainmail, mathescape=true, frame=lines]
NecessityBankSpec'  $\triangleq$  from a:Account $\wedge$ a.balance==bal
                       nxt a.balance < bal
                       onlyIf $\exists$ o.[$\external{\texttt{o}}$ $\wedge$ $\calls{\prg{o}}{\prg{a}}{\prg{transfer}}{\prg{\_, \_, \_}}$]
\end{lstlisting}
 
\prg{NecessityBankSpec$^\prime$} states that if over a single step the balance of an account decreases, then it must have occurred as 
a result of a call to \prg{transfer}.
 
\begin{lstlisting}[language = Chainmail, mathescape=true, frame=lines]
NecessityBankSpec''  $\triangleq$  from a:Account $\wedge$ a.password == pwd
                        to a.password != pwd
                        onlyThrough $\exists$ o.[$\external{\texttt{o}}$ $\wedge$ $\calls{\prg{o}}{\prg{a}}{\prg{set}}{\prg{pwd, \_}}$]
\end{lstlisting}
 
\prg{NecessityBankSpec$^{\prime\prime}$} states that if over an arbitrary number of execution steps, the password of an account changes,
then it follows that there must have been some intervening execution step that was a call to \prg{set} on the account 
with the correct password. Both of these specifications are important, and are both used as intermediate steps
when we present the full proof of \prg{NecessityBankSpec} later in Section \ref{s:examples}.
\Nec thus provides us with a rich language for talking about the necessary conditions
under which critical actions within of our software are allowed to occur. 
%\jm[]{It is worth discussing the semantics of \Nec specifications and their 
%relation to typical logical consequence and Hoare logic. A classical Hoare triple, 
%$\hoare{P}{C}{Q}$, denotes that any program state that satisfies $P$, after execution 
%of program $C$, will result in a program state that satisfies $Q$. Thus, $P$ represents 
%a subset of program states that after execution of $C$ results in a program state satisfying $Q$.
%Conversely, $Q$ represents a superset of program states resulting from the execution of $C$ in 
%a program state satisfying $P$. Thus, we can soundly strengthen the left hand side ($P$), and weaken
%the right hand side ($Q$). This intuition extends to all three specifications. 
%For example, from \prg{NecessityBankSpec'}, while it is somewhat contrived, we are able to
%strengthen the ``left hand side'' by adding information, and weaken the ``right hand side'', 
%and conclude that}
%\begin{lstlisting}[language = Chainmail, mathescape=true, frame=lines]
%NecessityBankSpec'''  $\triangleq$  from a:Account $\wedge$ a.balance == bal $\wedge$ bal == x + y $\wedge$ y > 0
%                       nxt a.balance == x
%                       onlyIf $\exists$ o.[$\external{\texttt{o}}$ $\wedge$ $\access{\prg{o}}{\prg{a}}$]
%\end{lstlisting}
%\jm[]{This follows because in the above specification, (a)\prg{x < bal}, and thus \prg{a.balance = x} implies and $a.balance < bal$,
%and (b) if an object calls a method on another object, it follows that it has access to that object.
%More generally, given a Single-Step Only-If specification, 
%$\onlyIfSingle{A_1}{A_2}{A}$, $A_1$ and $A_2$ represent a subset of single step execution paths starting from a program state 
%satisfying $A_1$ and reaching a program state satisfying $A_2$, that have $A$ as a necessary precondition. 
%In the same way the converse is true, i.e. $A$ represents a superset of initial program states
%for execution starting at a state satisfying $A_1$ and reaching a state satisfying $A_2$ after a single step of execution.
%As with Hoare logic, we are able to soundly strengthen the left hand side ($A_1$ and $A_2$)
%and strengthen the right hand side ($A$). This intuition also extends to Only-If and Only-Through specifications. 
%In some places later in this paper, we use the distinction ``\emph{left hand side}'' of a \Nec specification
%to denote the two left most assertions in the specification, and ``\emph{right hand side}'' to denote
%the necessity precondition.}
 




\subsection{Assertion Encapsulation}
\jm[lemmas? does A => enc(A') imply A => enc($\neg$A')?]{}
 %In order to reason about necessary requirements in an open world,
% and those assertions that may change due 
% to computation by external, unknown code.
In Section \ref{s:outline} we used the concept of encapsulation of \SpecO assertions 
 when proving adherence to \Nec specifications.
An assertion $A$ is encapsulated by a module $M$ if it cannot be invalidated unless an
internal method is called. 
Here we refine this concept, to allow for ``conditional'' encapsulation:
$M\ \vDash A\ \Rightarrow\ \encaps{A'}$ expresses that in states which satisfy $A$, the assertion 
$A'$ cannot be invalidated, unless a method from $M$ was called.

\begin{definition}[Assertion Encapsulation]
\label{def:encapsulation}
For % an internal module. -- SDL internal is nit an inherrent property
a module $M$ and assertion $A$, we define an assertion $A'$ as being 
encapsulated, written\ \  $M\ \vDash A\ \Rightarrow\ \encaps{A'}$, \ \ if and only if
%$M\ \vDash\ \onlyIfSingle{A}{\neg A}{\calls{x}{y}{m}{\overline{z}}\ \wedge\ \external{x}\ \wedge\ \internal{y}}$
for all external modules $M'$, and program states $\sigma$ and $\sigma'$
such that $\arising{M}{M'}{\sigma}$:

\begin{tabular}{lr}
$\;\;\;\;$- $\reduction{M}{M'}{\sigma}{\sigma'}$  & \rdelim\}{4}{3mm}[$\;\;\;\Rightarrow\;\;\;$  $\exists x,\ \overline{z}. (\satisfiesA{M}{\sigma}{\calls{\_}{x}{m}{\overline{\sd{z}}}\ \wedge\ \internal{x}})$] \\
$\;\;\;\;$- $\satisfiesA{M}{\sigma}{A}$   \\
$\;\;\;\;$- $\satisfiesA{M}{\sigma}{A'}$   \\
$\;\;\;\;$- $\satisfiesA{M}{\sigma' \triangleleft \sigma}{\neg A'}$   \\
\end{tabular}
\end{definition}

%%%
%%%%  

\section{Bits and Bobs: points to make earlier}

\subsection{one}
In the introduction show a method where
\begin{lstlisting}
    void cautious(untrusted : Object, acc: Account){
    PRE acc.passwd PRT-FROM untrusted
    POST acc.passwd PRT-FROM untrusted
     ...
        ext-call does not pass acc.pwd
     ...
   }
\end{lstlisting}

Even external object creation may return an old, preexisting object.
Nevertheless, we are able to prove the above. Make a diagram that has an untrust', which has access to a.pwd. 


\subsection{two}
In our setting, we must deal with the possibility  that any untrusted object may point to any other untrusted object.
Therefore, any reference given to an untrusted object may end up, in principle,  eventually given to all of them.
Therefore if there was an untrusted object $u$ that had reference to $o$,  and we knew that $o$ was protected from $z$ now, we are not
allowed   to deduce that $o$ will still be protected in the "deep" future.
However, if we consider the "shallow" future, we can do better than that.

Namely, if  before an untrusted call with receiver and arguments $\overline z$ we know that $o$ is protected from $\overline z$, and if we know that the internal objects do not leak $o$ to the the external world, then we know that during execution of the untrusted call (and also all nested trusted or untrusted calls), $o$ will be protected (ie no locally accessible external object will obtain direct access), and that after the call, $o$ will still be protected from $\overline z$.

Going back to the discussion where $u$ \kjx{refers}
\st{has an (un-mitigated?)reference} to $o$,  if at the untrusted call
with receiver and arguments $\overline z$, we know that $o$ is
protected from $\overline z$, then this does not preclude
\kjx{$u$ referring to $\overline z$, but it does preclude $\overline
  z$ referring to $u$ -- because if $\overline z \rightarrowtail u$
  then $\overline z \rightarrowtail o$.}
\st{that $u$ has unmitigated access to $\overline z$, but it does precludes that any of  $\overline z$ has unmitigated access to $u$ (because if had, then it would also have had unmitigated access to $o$ itself).}

\subsection{three}
NOTE: JAMES asked Do we need to worry about well-fomedness? Eg what if
we had an assertion $3 \wedge 5 \rightarrow 66$?

  

\subsection{four}
Say that protected-from is a heap property, whereas protected is a
heap-frame property.


\subsection{five}
Say authority implies eventual permission. Ie we want to guarantee that an untrusted object will not get permission to a capability. Lack of eventual permissions bounds authority.

\subsection{six}
Execution takes many modules, but satisfaction only one module.


\paragraph{The next sections are organized as follows}
\label{s:semantics}
% In this section we define {the}  \SpecLang specification language.  
We first define an underlying programming language, \LangOO (\S \ref{sub:Loo}).
We then define an assertion language, \AssertLang, which can talk about the
contents of the state, as well as about protection (\S \ref{sub:SpecO}).  Finally, we define the syntax and
semantics of  \SpecLang
specifications (\S \ref{s:holistic-guarantees}).

 


%%%
%%%
%
%\section{\LangOO, The programming language }  

\subsection{\LangOO syntax runtime configurations}
\label{sub:Loo} 
{This work} is based \LangOO, a {small}, imperative, sequential,  class based, typed, object-oriented language. 
 {We believe, however, that the work can easily adapted to any capability safe language with some form of encapsulation. 
Wrt to encapsulation and  capability safety},  \LangOO supports private fields, private and public methods, unforgeable addresses, and no ambient authority (no static methods, no address manipulation).
 It has a simple concept of module with module-private fields and methods, described in Sect. \ref{sect:execution}.
 The definition of \LangOO  {can be found in the appendices  \cite{necessityFull}}, and is  similar to   OOPSLA-22.\footnoteSD{any differences?}

A \LangOO state, $\sigma$,  consists of a  heap $\chi$, and a   stack $\psi$. A stack is is a sequence of frames, $\phi$.
A frame consists of a local variable map, and a continuation, \ie a sequence of statements to be executed.


 
\paragraph{Notation} We adopt the following, unsurprising, notation:
\begin{itemize}
\item
{An object is uniquely identified by the address that points to it. We shall be talking of objects $o$, $o'$ when talking less formally, and of addresses, $\alpha$, $\alpha'$, $\alpha_1$, ...  when more formal.}
\item
$x$, $x'$,  ..., $y$, ... $z$, ... are variables;  $\va$, $\va'$ ... are either addresses or variables, we call these \emph{\atoms}.
\item
$\alpha \in \sigma$ means that $\alpha$ is defined in the heap of $\sigma$, and $x\in \sigma$ means that $x$ is defined in the top frame of $\sigma$.
Conversely,  $\alpha\notin\sigma$ and $x\notin\sigma$ %, and $\va \notin A$ h
 have the obvious meanings.
\item
$\interpret{\sigma}{\alpha}$  is $\alpha$; and $\interpret{\sigma}{x}$  is the value to which  $x$  is mapped in the top-most frame of $\sigma$'s stack, 
and $\interpret{\sigma}{e.f}$ looks up in $\sigma$'s heap the value of $f$ for the object  $\interpret{\sigma}{e}$.
Note that $\interpret{\sigma}{e}$ is not defined when $e$ contains a method call or a ghost field.
\item The substitution  $\sigma[x \mapsto \alpha]$ is applied to the top frame of $\sigma$, and $\sigma[\overline{x \mapsto \alpha}]$ % applies the substitutions $\overline{x \mapsto \alpha}$ to the top frame.
has the expected meaning.
\item
{$\phi.\prg{local\_map}$ is the local variable map of $\phi$}, and $\sigma.\prg{cont}$ is the continuation in the top frame.
\item
$text_1 \txteq text_2$ expresses that $text_1$ and $text_2$ are textually equal.
\end{itemize}

  

  
\subsection{\LangOO Execution}
\label{sect:execution}

%Central to our work is the distriction between the 
 \LangOO execution is described by a small steps operational semantics of the shape $\leadstoOrig  {\Mtwo} {\sigma}   {\sigma'}$.\\
  $\Mtwo$ stands for one or more modules, where a
  module,  $M$, maps class names to class definitions. 
   
{The semantics enforces dynamically a simple form of module-wide privacy: 
Fields may be read or written only if the class of the object whose field is being read or written, and the class of the object which is reading or writing belong to the same module.}
Private methods may be called only if the class of the receiver (the object whose method is being called), and the class of the caller (the object which is calling) belong to the same module.
Public methods may always be called.

The semantics is as unsurprising in all remaining aspects  :  
In $\sigma$, the  top frame's continuation contains the statement to be  executed next.  
 Statements may assign to variables, allocate new objects, 
perform field reads and writes on objects, and
 call methods on those objects. 
When a method is called, a new frame is pushed onto the stack; this frame  maps \prg{this} and the formal parameters to  the values for the receiver and other arguments, and the continuation to the body of the method.  When the continuation is ground\footnoteSD{TODO check and define}, the frame is popped and the value from the last frame's continuation is entered into the appropriate part in the caller's continuation. 
%In other aspects, the semantics is unsurprisring.%we return from that call, its frame is  popped, and execution continues in the context of the calling method. 
%The relation $\leadstoOrigStar  {\Mtwo} {\_}   {\_}$  is the reflexive, transitive closure of $\leadstoOrig  {\Mtwo} {\_}   {\_}$ .


{Fig. \ref{fig:UpSemantics} illustrates  such  execution steps:  disks indicate states;
 horizontal $\leadstoN$-arrows denote   steps  within the same  call; upwards arrows denote  method calls;
 %(pushing a new frame onto the stack);  
 downwards arrows denote method returns. % (popping the top of the stack). 
 Here,   $\leadstoOrig {\Mtwo}{\sigma_8}   {\sigma_9} $ is a step within the same call, $\leadstoOrig {\Mtwo}{\sigma_9}   {\sigma_{10}} $ is a method call   
with $\leadstoOrig {\Mtwo}{\sigma_{12}}   {\sigma_{13}} $ %is a method return  (from the call to $m_a$), 
the corresponding return. 
 {Note that  $\leadstoOrigStar  {\Mtwo} {\sigma}   {\sigma'}$ may involve  any number of  calls or returns: 
 $\leadstoOrigStar  {\Mtwo} {\sigma_8}   {\sigma_{12}}$ involves one call and no return,
while $\leadstoOrigStar  {\Mtwo} {\sigma_{10}}   {\sigma_{15}}$,   involves no calls and two returns.
% In section \ref{sect:bounded}, we will define a derived relation, called bounded execution, where the number of returns may not exceed the number of calls.
}
} 

\begin{figure}[htb]
\begin{tabular}{|c|}
 \hline %  \\ -- this added one vertical space
\resizebox{7cm}{!}{
\includegraphics[width=\linewidth]{diagrams/bounded.png}
} 
 \\
\hline
%\begin{tabular}{lclclclcl}
%$\leadstoOrig  {\Mtwo} {\sigma_8}    {\sigma_9} $  & & 
%$\leadstoOrig  {\Mtwo} {\sigma_9}    {\sigma_{10}} $ &  &
%$\leadstoOrig  {\Mtwo} {\sigma_{12}}   {\sigma_{13}} $ & & 
%$\leadstoOrig  {\Mtwo} {\sigma_{13}}    {\sigma_{14}} $  &  &
%$\leadstoOrig {\Mtwo}{\sigma_{14}}   {\sigma_{15}} $
%\\
%\hline
%$\leadstoOrigStar  {\Mtwo} {\sigma_8}   {\sigma_{12}}$ & & \ & & \ & & $\leadstoOrigStar  {\Mtwo} {\sigma_{10}}   {\sigma_{15}}$
%\\
%\hline
%\end{tabular}
%\\
%\hline
\end{tabular}
   \caption{Illustrating   $\leadstoOrig  {\Mtwo} {\sigma}    {\sigma'}$ 
    }
   \label{fig:UpSemantics}
 \end{figure}
 
%\susan{I don't think the two lines at the bottom of the figure are needed here and in the next figure. {agree, but highlight.}} 
 
%{Note that $\leadstoOrig {\Mtwo}{\sigma_{8}}   {\sigma_{9}} $ and $\leadstoOrig {\Mtwo}{\sigma_{13}}   {\sigma_{14}} $ are steps within the same call, but 
%$\leadstoOrig {\Mtwo}{\sigma_{14}}   {\sigma_{15}} $ and $\leadstoOrig {\Mtwo}{\sigma_{17}}   {\sigma_{18}} $ are not. %steps within the same call,
%% even though all four states ($\sigma_{13}$, $\sigma_{14}$, $\sigma_{17}$, and $\sigma_{18}$), have the same number of frames on their stack.
%We want a semantics to reflect whether execution steps happen within the bounds of certain call. For this, we define \emph{bounded execution}, 
%$\leadstoBounded {\Mtwo} {\sigma} {\sigma''} {\sigma'}$ 
%which are execution steps which lead from $\sigma$ to $\sigma'$ while not popping  $\sigma''$-s top frame.
%This will be defined in Section \ref{sect:bounded}.
%}

%\subsection*{Applicability} 
%While our work is based on the particular language  \LangOO , % a simple, imperative, typed, object oriented  language with unforgeable addresses and private fields, we believe that % our approach
%we believe that it is applicable to several programming paradigms, and  that   unforgeability and privacy
% can be replaced  by lower level mechanisms such as capability machines \cite{vanproving,davis2019cheriabi}.


\section{From \LangOO to \AssertLang}

{To develop the semantics of our assertions language, \AssertLang, we   build three auxiliary concepts on  top of  \LangOO: scoped execution, method calls and returns, and reachable objects.}


  
  
 \subsection{\cancel{Scoped Execution, and scoped future} \se{Scopes}}
 \label{sect:bounded}

{The semantics from the earlier section allows arbitrary numbers of method calls and returns. 
In particular, it is possible to start with a state $\sigma$ and perform more returns than calls --
\eg $\leadstoOrigStar  {\Mtwo} {\sigma_{8}}   {\sigma_{15}}$  in  Fig. \ref{fig:UpSemantics}.
{In the sense of $\rightarrow^*$,  the state $\sigma_{15}$  is one of the future  states for $\sigma_8$.}

 
{For} the purposes of our work, we   need an {additional} notion, of  \emph{scoped} future:  
the scoped future of a state consists of all states which  can be reached through any   
 steps, including method calls and returns, but   {stopping before returning}   
from the method executing in the scoping state}. 
\forget{We say the currently executing method \emph{scopes} the execution.}
Thus, the {scoped} future  of $\sigma_8$   includes only
  $\sigma_9$, $\sigma_{10}$, $\sigma_{11}$, $\sigma_{12}$, $\sigma_{13}$, and $\sigma_{14}$, but \emph{not} $\sigma_{15}$  -- the latter results from returning from $\sigma_{8}$'s top continuation.  
 {Similarly, $\sigma_{18}$ is not in the scoped future of $\sigma_8$, even though the two states have the same stack depth -- between $\sigma_{8}$  and  $\sigma_{18}$ we returned from the top continuation in $\sigma_{8}$.}
 
 

To capture this  notion, we define  % from the viewpoint of an original state, 
{\emph{execution scoped by a state} $\sigma\bd$, which %  is ``bounded'' so as not to ever 
allows any steps, except for popping   $\sigma\bd$'s top frame:}
 
 
\begin{definition}[Scoped Execution]
\label{def:shallow:term}
We define relations \    $\leadstoBoundedThree {\Mtwo} {\sigma} {\sigma\bd} {\sigma'}$ \ and\  $\leadstoBounded  {\Mtwo} {\sigma} {\sigma'}$ as:

\begin{itemize}
\item
 $\leadstoBoundedThree {\Mtwo} {\sigma} {\sigma\bd}  {\sigma'}$ \    iff \ \   $\leadstoOrig {\Mtwo} {\sigma} {\sigma'} % $\\
% $\strut  \hspace{3.6cm}\ 
\ \  \wedge $\\
$\strut  \hspace{2.9cm}\ \      \exists \phi,\!\psi,\!\psi_1,\!\psi_2.[ \  \sigma\bd\! =\! (\phi\cdot\psi,\_) \ \wedge \ \sigma\! =\! (\psi_1\cdot \psi, \_)
\ \wedge\ \sigma'\! =\! (\psi_2\cdot \psi, \_)\ \wedge\ {\psi_1,\psi_2\!\neq\! \epsilon}] $ 
\item
 $\leadstoBoundedStarThree  {\Mtwo}  {\sigma}  {\sigma\bd} {\sigma'}$\ \  iff \ \ $\sigma=\sigma'\ \ \vee$\\
$\strut  \hspace{3cm}\ \ \exists n\!\in\!\mathbb{N},\sigma_0,..,\sigma_n.[\ \forall i\!\! \in\!\! [0..n).\leadstoBoundedThree {\Mtwo}  {\sigma_i}  {\sigma\bd} {\sigma_{i+1}} \ \wedge\ \sigma=\sigma_0\ \wedge\ \sigma'=\sigma_n\ ]$
 \item
{  $\leadstoBounded  {\Mtwo} {\sigma}   {\sigma'}$\ \  \ \ \ \ \ \   \ \ \ \ iff \ \ \ $\leadstoBoundedThree {\Mtwo} {\sigma} {\sigma}  {\sigma'}$}
  \item
{  $\leadstoBoundedStar {\Mtwo}  {\sigma}  {\sigma'}  $\ \ \ \ \ \ \ \ \   \ iff \ \ \ $\leadstoBoundedStarThree {\Mtwo}  {\sigma}  {\sigma} {\sigma'}$}\ \  
\item
{  $\leadstoBoundedStarFin {\Mtwo}  {\sigma}  {\sigma'}  $\ \ \ \ \ \   \ iff \ \ \ $\leadstoBoundedStar {\Mtwo}  {\sigma}  {\sigma'}  \ \wedge\ \ \neg(\exists 
\sigma''.\ \leadstoBoundedThree {\Mtwo} {\sigma'}  {\sigma} {\sigma''} \ )$ }
 \end{itemize}
\end{definition}
 

We continue with  Fig. \ref{fig:UpSemantics}. Here $\leadstoOrig {\Mtwo} {\sigma_{14}}  {\sigma_{15}}$ 
 but    $\notLeadstoBoundedThree {\Mtwo}  {\sigma_{14}} {\sigma_{14}} {\sigma_{15}}$
and  $\notLeadstoBounded  {\Mtwo}  {\sigma_{14}}   {\sigma_{15}}$
--  this step would pop  $\sigma_{14}$'s
 top frame. 
% Therefore,   $\leadstoOrigStar {\Mtwo} {\sigma_8}  {\sigma_{15}}$ 
%  but  $\notLeadstoBoundedStarThree {\Mtwo} {\sigma_8} {\sigma_8} {\sigma_{15}}$, and  $\notLeadstoBoundedStarThree {\Mtwo}  {\sigma_8} {\sigma_{15}}$.
 Also, $\leadstoOrigStar {\Mtwo} {\sigma_8}  {\sigma_{18}}$ 
 but  $\notLeadstoBoundedStar {\Mtwo} {\sigma_8}   {\sigma_{18}}$  -- even though $\sigma_8$ and $\sigma_{18}$ have the same depth of stack, they belong to different calls.

  
Lemma \ref{lemma:orig:to:bounded}  states that
\sdN{any execution} which is part of an execution which started at some initial state, is scoped by that state.
%\ (\ref{otbOne})\  Any state in the future of some  initial state is in the  scoped future   of that state.
%\ (\ref{otbTwo})\ %  
% \sdN{If a state $\sigma$ is in the future of some initial state $\sigma_{init}$, then any execution from $\sigma$ to $\sigma'$  is scoped by   $\sigma_{init}$.}
%While an execution from $\sigma$ to $\sigma'$ need not be scoped ($\leadstoOrigStar {\Mtwo} {\sigma}  {\sigma'}$ does not imply $ \leadstoBoundedStar {\Mtwo} {\sigma}  {\sigma'}$), 
%if $\sigma$ is in the future of some initial state $\sigma_{init}$, then $\sigma$ is reachable from some initial state $\sigma_{init}$
% then $\sigma_{init}$  bounds the execution from $\sigma$ to   $\sigma'$. 
% \ (\ref{otbThree})\  generalizes %  Lemma \ref{lemma:orig:to:bounded},  part
% (\ref{otbTwo}): {For any execution starting at some initial state $\sigma_{init}$ and leading from $\sigma_m$ to $\sigma_n$, 
% we can find $\sigma_k$, a predecessor of $\sigma_m$, which  scopes the execution from $\sigma_m$ to $\sigma_n$; moreover all
%  predecessors  for $\sigma_k$ scope that execution, while its successors do not.}
% % there exists a further, earlier, state $\sigma_o$, such that both $\sigma$ and $\sigma'$ are  
%%part   of $\sigma_o$'s bounded future. Moreover, from $\sigma_o$'s viewpoint, $\sigma'$ is a transitive bounded successor of $\sigma$.}
%Bounded semantics impose restrictions on the set of future states, but only from the viewpoint of {the bounding} \susan{binding?%\red{SD: I thought that binding goes with "bind" and "bounding" goes with "bounded" -- i.e., the state that provides the bound}} state. 
\forget{
Lemma \ref{lemma:orig:to:bounded}  states that:  
\ (\ref{otbOne})\  All executions starting at an initial state  are in scope.
\ (\ref{otbTwo})\  {While an execution from $\sigma$ to $\sigma'$ need not be in scope  ($\leadstoOrigStar {\Mtwo} {\sigma}  {\sigma'}$ does not imply $ \leadstoBoundedStar {\Mtwo} {\sigma}  {\sigma'}$), if $\sigma$ is in the scoped future of some initial state $\sigma_{init}$, 
 then $\sigma_{init}$  scopes the execution from $\sigma$ to   $\sigma'$. }
 \ (\ref{otbThree})\  generalizes %  Lemma \ref{lemma:orig:to:bounded},  part
 (\ref{otbTwo}): {For any execution starting at some initial state $\sigma_{init}$ and leading from $\sigma_m$ to $\sigma_n$, 
 we can find $\sigma_k$, a predecessor of $\sigma_m$, which  scopes the execution from $\sigma_m$ to $\sigma_n$; moreover all
  predecessors  for $\sigma_k$ scope that execution, while its successors do not.}
  }
 % there exists a further, earlier, state $\sigma_o$, such that both $\sigma$ and $\sigma'$ are  
%part   of $\sigma_o$'s bounded future. Moreover, from $\sigma_o$'s viewpoint, $\sigma'$ is a transitive bounded successor of $\sigma$.}


 \begin{lemma}
\label{lemma:orig:to:bounded}
For all $\overline M$, all $n,m\in \mathbb{N}$ with $m\leq n$, $\sigma_0$, ... $\sigma_n$,  $\sigma_{init}$, $\sigma$, $\sigma'$, where
$\sigma_{init}$ is an initial state:\footnote{An \emph{Initial} state's heap contains a single object of class \prg{Object}, and
its  stack   consists of a single frame, whose local variable map is a mapping from \prg{this} to the single object, and whose continuation is  any statement.
(See Definition %s \ref{def:initial} and 
\ref{def:arising} and the 
{appendices %of the full paper 
\cite{necessityFull}).}} 
\begin{itemize} % {enumerate} 
%\item 
%\label{otbOne}
%$\leadstoOrigStar {\Mtwo} {\sigma_{init}}  {\sigma}\ \ \Longrightarrow\ \  \leadstoBoundedStar {\Mtwo}  {\sigma_{init}} {\sigma}$.
\item 
\label{otbTwo}
$\leadstoOrigStar {\Mtwo} {\sigma_{init}}  {\sigma}\ \wedge\ \leadstoOrigStar {\Mtwo} {\sigma}  {\sigma'}\ \ \Longrightarrow\ \  \leadstoBoundedStarThree {\Mtwo} {\sigma} {\sigma_{init}} {\sigma'}$.
%there exists a $\sigma_o$, such that $\leadstoBoundedStar {\Mtwo} {\sigma_{init}}  {\sigma_o}$, and
% $\leadstoBoundedStar {\Mtwo} {\sigma_o}  {\sigma}$, and $\leadstoBoundedStar {\Mtwo} {\sigma_o}  {\sigma'}$, and $\leadstoBoundedStar {\Mtwo} {\sigma_o}  {\sigma'}$. More specifically, 
% $\leadstoBoundedStarThree {\Mtwo} {\sigma} {\sigma_o} {\sigma'}$.
%\item
%\label{otbThree}
% $\forall i\!\in\! [0..m).\ \leadstoOrig  {\Mtwo} {\sigma_{i}}  {\sigma_{i+1}}\ \wedge\ \sigma_{init}=\sigma_0$ \\
% % \ \wedge \  \sigma=\sigma_m \ \wedge\ \sigma' =\sigma_n  $ \\
%\strut \hspace{0.5cm} $\ \ \Longrightarrow \ \  \exists k\leq m.[\ \ \ \forall i\leq k.[\  % \leadstoBoundedStar  {\Mtwo}   {\sigma_{init}} {\sigma_i}\ \wedge \ 
%\leadstoBoundedStarThree  {\Mtwo}  {\sigma_m}  {\sigma_{i}} {\sigma_n} ]\ \ \  \wedge \ \ \ 
%% $\\ \strut \hspace{3cm} $
% \forall i> k. [\  \notLeadstoBoundedStarThree  {\Mtwo}  {\sigma_m}  {\sigma_{i}} {\sigma_n}\ ] \ \ \ ]$.
%\end{enumerate} 
\end{itemize}
\end{lemma}
 



{We revisit  Fig. \ref{fig:UpSemantics}, and assume that $\sigma_6$ is an initial state.
We have that $\leadstoOrigStar {\Mtwo} {\sigma_{10}}  {\sigma_{14}}$ and $ \notLeadstoBoundedStar {\Mtwo}  {\sigma_{10}} {\sigma_{14}}$.
However, because $\sigma_6$ is an initial state, we have $\leadstoBoundedStarThree {\Mtwo}  {\sigma_{10}} {\sigma_{6}}  {\sigma_{14}}$ -- part (\ref{otbTwo}) 
from above. 
Moreover, we have that  $\forall i\in [6..9].\leadstoBoundedStarThree {\Mtwo}  {\sigma_{10}} {\sigma_{i}}  {\sigma_{14}}$ -- part (\ref{otbThree}) 
from above. 
}

  \subsection{{Reachable  Objects}}

 {A central concept to our work is object \emph{protection}, which we will define in   Sect. \ref{sect:protect}: It requires that no external object  
reachable from the top frame  can have unmitigated access to that object.}
%
%{The  \SpecLang  specifications support universal quantification over  objects; such specifications 
%are applicable  to all objects in the heap witch, are however, either locally reachable (i.e. there is in the heap a path from the an 
%object on the top frame to the particular object), or globally reachable (i.e. there is in the heap a path from the an 
%object on some frame to the particular object.)
%%In this section  we will formally define these concepts.}\footnoteSD{TODO we need a better motivation for these concepts.}
%
An object $\alpha$ is  locally reachable, $ \LRelevant \alpha \sigma $, if it is reachable from the top frame on the stack of $\sigma$,
and it is globally reachable, $\GRelevant \alpha \sigma$, if it is reachable from any  frame on the stack of $\sigma$.
 
\begin{definition} We define 
\begin{itemize}
\item
$ \LRelevant \alpha \sigma $ \ \ iff\ \  
$\exists \phi.[\ \sigma=(\phi\cdot\_, \_)$ and $\Relevant \alpha \phi \sigma\ ]$. % for some $\phi$
\item
$\GRelevant \alpha \sigma$  \ \ iff\ \  
$\exists \phi.[\ \sigma=(\_\cdot\phi\cdot\_, \_)$ and $\Relevant \alpha \phi \sigma\ ]$. % for some $\phi$
\end{itemize}
where\\
$\strut\ \ \ \  \ \ \ \ \ \ \Relevant \alpha \phi \sigma $  \ \ \ \ \ \ \ iff\ \  
$\exists n\in \mathbb{N}.\exists \prg{f}_1,... \prg{f}_n.\exists \prg{x}.[ \ \interpret{\sigma}{\phi(x).\prg{f}_1.....\prg{f}_n} = \alpha \ \ ]$.

\end{definition}

 \begin{figure}[htb]
\begin{tabular}{|c|c|c|}
\hline \\
\resizebox{3.5cm}{!}{
\includegraphics[width=\linewidth]{diagrams/heap.png}
} 
&
\resizebox{5cm}{!}{
\includegraphics[width=\linewidth]{diagrams/locReachA.png}
} 
&
\resizebox{5cm}{!}{
\includegraphics[width=\linewidth]{diagrams//locReachb.png}
} 
\\
\hline
 a heap
&
Locally Reachable from $\phi_1$
&
Locally Reachable from $\phi_2$
\\
\hline \hline
\end{tabular}
   \caption{A heap, and Locally Reachable Objects. % from $\phi_1$ and $\phi_2$. 
   The distinction of objects into  green or pink is explained in later chapters}
   \label{fig:LReachable}
 \end{figure}

We illustrate these concepts in Fig. \ref{fig:LReachable}: In the middle pane the top frame is $\phi_1$ which maps \prg{this} to $o_1$; all objects are locally reachable. 
In the right pane the top frame is $\phi_2$, which maps \prg{this} to $o_3$, and $x$ to $o_7$; now $o_1$ and $o_2$ are no longer locally reachable.

Lemma  \ref{lemma:relevant} % describes properties of global reachability. 
says that  (\ref{oneGR}) Locally reachable objects are globally reachable. 
(\ref{twoGR}) 
% Any object which will be globally reachable at some future state  and which exists object in the current state, is globally reachable the current state: that is, 
Globally unreachable objects may not become reachable in the future.
\footnoteSD{cite "only connectivity begets connectivity"}
(\ref{threeLR}) A pre-existing object, locally reachable after any number of scoped execution steps, was locally reachable at the first step.


\begin{lemma}
\label{lemma:relevant}
For all module sets $\Mtwo$, states $\sigma$, $\sigma'$,   address $\alpha$, and $\overline \alpha$:
%\footnoteSD{{TODO decide whether $o$ or $\alpha$}}
%and  variables ${\overline z}$, and statements $s:$
\begin{enumerate}
\item
\label{oneGR}
$ \LRelevant \alpha \sigma\ \ \Longrightarrow \ \   \GRelevant \alpha \sigma$
\item
\label{twoGR}
${\leadstoOrigStar {\Mtwo}  {\sigma}  {\sigma'}} \ \ \wedge \ \  \GRelevant \alpha {\sigma'} \ \ \wedge\ \  {\alpha\in \sigma} \ \ \ \Longrightarrow \ \  \ \GRelevant \alpha {\sigma}$.
\item
\label{threeLR}
${\leadstoBoundedStar {\Mtwo}  {\sigma}    {\sigma'}} \ \ \wedge \ \   \LRelevant \alpha {\sigma'}\  \ \wedge\ \  {\alpha\in \sigma} \ \ \ \Longrightarrow \ \ \ \LRelevant \alpha {\sigma}$.
\end{enumerate}
\end{lemma}

{Consider Fig.  \ref{fig:UpSemantics}. %, and Fig.  \ref{fig:UpSemanticsBounded}.
Lemma \ref{lemma:relevant}.\ref{threeLR}  promises that any objects locally reachable in $\sigma_{14}$ which already existed in $\sigma_{8}$, were locally reachable in $\sigma_{8}$. However, the lemma is only  applicable to scoped execution, and as 
$\notLeadstoBoundedStar {\Mtwo} {\sigma_8}  {\sigma_{17}}$, 
the lemma does not promise that  objects locally reachable in $\sigma_{17}$ which already existed in $\sigma_{8}$, were locally accessible in $\sigma_{8}$ -- namely it could be that objects are made globally reachable upon method return, during the step from $\sigma_{14}$ to $\sigma_{15}$.}

 \subsection{\red{23 April - newest:} Method Calls and Returns}

 
\sdN{
Method calls and returns %in particular external calls, are a central consideration of our work. \sdN{In particular, every call, whether external or internal, 
push/pop \se{stack} frames, %onto the stack, 
and thus restrict/extend the set of locally reachable objects. 
The operator   $ \PushS  {\alpha} {\sigma}$ describes the effect of pushing a stack frame,
and Lemma \ref{lemma:push:N} describes its effect on locally reachable objects.
%
% therefore a characterization of the states resulting immediately from a method call or a method return is crucial. 
% which a method is called or returned from is important.
% The set ${\_} \triangledown {\_}$ does exactly that:
%\footnoteSD{We need to say that all we care about is what is reachable in the rnew state vs what w
 }
 
 {
\begin{definition}
\label{def:push:frame}
Given a state $\sigma$, addresses $\overline \alpha$, and variables or addresses $\overline \va$, we define
\begin{itemize}
\item
$ \PushS  {\alpha} {\sigma} \ =\ \{ \ \sigma' \ \mid\ \exists \phi,\psi  \ s.t.\ \ 
%\\  $\strut \hspace{2.3cm}  
\sigma=(\psi, \chi)\  \wedge\  \sigma'=(\phi\!\cdot\!\psi, \chi) \ \wedge\    rng(\phi)=\overline \alpha \ \   \ \}$
\item
{$ \PushS  {\va}  {\sigma}$ is short for  $ \PushS  {\interpret {\sigma} {\va}} {\sigma} $.}
\end{itemize}
 \end{definition}
}\footnoteSD{\red{DO we also need that $\overline \alpha$ were locally reachable in $\sigma$?}}

 
 
 %Thus, if $\sigma_a$ is the result of pushing a frame onto $\sigma_b$, then $\sigma_a\in   \PushS  {\alpha} {\sigma_b}$. 
%{Similarly, if $\sigma_b$ is the result of popping a frame from $\sigma_a$, then $\sigma_a\in   \PushS  {\alpha} {\sigma_b}$.}
 % below Therefore, as 
 Lemma \ref{lemma:push:N} says that % $\pushSymbol$ characterizes  calls and returns. 
%\  (\ref{pushOne}):\  If $\leadstoOrig {\Mtwo} {\sigma}   {\sigma'} $ and
%$\sigma$'s continuation starts with a method call), then $\sigma'$ is the result of pushing a frame with the receiver and arguments onto $\sigma$'s stack. \ 
%%  $\sigma'\!\in\!\PushS   {\alpha} {\sigma}$ then $\sigma'$ is a  \emph{callee} state of $\sigma$ --  {a direct successor state of    $\sigma$  after calling a method}. % with receiver and arguments $\overline \alpha$). 
% (\ref{pushTwo}):\  If $\leadstoOrig {\Mtwo} {\sigma}   {\sigma'} $ and $\sigma$'s continuation is the last statement before a method returns, then $\sigma'$   results by popping the top frame % containing   some arguments ($\overline \alpha$) form
% from $\sigma$'s  stack. 
%Finally, (\ref{oneLR}):\
any    object  which is locally reachable  {right after pushing a frame} was also locally reachable before 
pushing that frame. 
% {entering} the call.
%\kjx{pretty sure we need one other case: (3a) a locally reachable
%  obbject after a call could have been newly created by that call
%  --
%  \red{SD: Thank you for the comment. 
%  (3) is supposed to talk about immediately after entering the call. It was missing an assumption -- in blue. 
%  Please check the text and the maths.}}
 
  
\begin{lemma}% [$\pushSymbol$  for calls and returns]
\label{lemma:push:N}
For all states $\sigma$, $\sigma'$, modules $\Mtwo$, variables, $x$, $\overline {y}$, and address $\alpha'$.  

% If \ $\leadstoOrig {\Mtwo} {\sigma}   {\sigma'} $, \ then 
\begin{itemize} % \begin{enumerate}
%\item
%\label{pushOne}
%$   \sigma.\prg{cont} \txteq x:=y_0.m(y_1,..y_n); \_\ \  \  \Longrightarrow \ \  \sigma'\in   \PushS  {{ \interpret \sigma y}} {\sigma}   $
%\item
%\label{pushTwo}
%$  \sigma.\prg{cont}\txteq\red{z;\_} \ \   \Longrightarrow \ \  
%\exists   \overline \alpha. \  [\ \  \sigma\!\in\! \PushS  {\alpha} {\sigma'}\  \ ]$
%% \wedge\  \sigma.\prg{cont}\txteq\alpha' \ \ \wedge   \sigma'.\prg{cont}\txteq x:=\alpha';stmt \ \ \wedge 
%%\\
% % $\strut \hspace{4.6cm}\sigma.\prg{pop}.\prg{cont}\txteq x:=y_0.m(y_1,..y_n); stmt \  ] $
\item
\label{oneLR}
{$ \sigma'\!\in\! \PushS {\alpha} {\sigma}   \ \wedge  \  \LRelevant {\overline \alpha} {\sigma}  \ \wedge\    \LRelevant \alpha {\sigma'} \ \  \Longrightarrow\ \ 
%$ \\ $\strut \hspace{2cm}
 \LRelevant \alpha {\sigma}$
}

\end{itemize}  % \end{enumerate}

\end{lemma}
%\footnote{A stronger lemma can be proven:
%% Lemma \ref{lemma:push:N:S} says that % $\pushSymbol$ characterizes  calls and returns. 
%\  (\ref{pushOne}):\  If $\leadstoOrig {\Mtwo} {\sigma}   {\sigma'} $ and $\sigma'\!\in\!\PushS   {\alpha} {\sigma}$ then $\sigma'$ is a  \emph{callee} state of $\sigma$ --  {a direct successor state of    $\sigma$  after calling a method}. % with receiver and arguments $\overline \alpha$): \ 
%$\sigma'\in   \PushS  {\alpha} {\sigma}  \ \ \Longleftrightarrow \ \ 
%\exists x, m.[\ \ \sigma.\prg{cont} \txteq x:=y_0.m(y_1,..y_n); \_\ \ \wedge \ \overline \alpha = \overline{ \interpret \sigma y} \ \ ] $
%\\
% (\ref{pushTwo}):\  If $\leadstoOrig {\Mtwo} {\sigma}   {\sigma'} $ and $\sigma\! \in\! \PushS   {\alpha} {\sigma'}$  then $\sigma'$ is a  \emph{caller} state  of $\sigma$ -- {a direct successor  of $\sigma$}  after returning from a method:
% \ \ 
% $\sigma\in   \PushS  {\alpha} {\sigma'}  \ \ \Longleftrightarrow \ \ \exists x, \alpha'.[\ \  \sigma.\prg{cont}\txteq\alpha' \ \ \wedge \ \ \sigma'.\prg{cont}\txteq x:=\alpha';\_ \ \ ] $
% }
  
%\begin{lemma}% [$\pushSymbol$  for calls and returns]
%\label{lemma:push:N:S}
%For all states $\sigma$, $\sigma'$, modules $\Mtwo$, and addresses $\overline \alpha$. If \ $\leadstoOrig {\Mtwo} {\sigma}   {\sigma'} $, \ then 
%\begin{enumerate}
%\item
%\label{pushOne}
%$\sigma'\in   \PushS  {\alpha} {\sigma}  \ \ \Longleftrightarrow \ \ 
%\exists x, m.[\ \ \sigma.\prg{cont} \txteq x:=y_0.m(y_1,..y_n); \_\ \ \wedge \ \overline \alpha = \overline{ \interpret \sigma y} \ \ ] $
%\item
%\label{pushTwo}
%$\sigma\in   \PushS  {\alpha} {\sigma'}  \ \ \Longleftrightarrow \ \ \exists x, \alpha'.[\ \  \sigma.\prg{cont}\txteq\alpha' \ \ \wedge \ \ \sigma'.\prg{cont}\txteq x:=\alpha';\_ \ \ ] $
%\item
%\label{oneLR}
%{${\leadstoOrig  {\Mtwo}  {\sigma}  {\sigma'}} \ \ \wedge\ \  \sigma'\in \PushS {\alpha} {\sigma}  \ \ \wedge \ \  \LRelevant {\overline \alpha} {\sigma} \ \ \wedge\ \   \LRelevant \alpha {\sigma'} \ \ \ \Longrightarrow$
%\\
%$\strut \hspace{2cm} \LRelevant \alpha {\sigma}$
% \end{enumerate}
% \end{lemma}
%}
 

 Consider Fig. \ref{fig:UpSemantics} again: $\sigma_8\!\in\!   \PushS  {\alpha} {\sigma_7}$ for some $\overline \alpha$ -- {thus $\sigma_8$ is a callee state for 
 $\sigma_7$}. Also, 
 $\sigma_{14}\!\in\!\PushS  {\alpha'} {\sigma_{15}}$ for some $\overline {\alpha'}$ -- {thus $\sigma_{15}$ is a caller state for 
 $\sigma_{14}$}.
\footnote{ $\overline \alpha$ may differ from $\overline {\alpha'}$, because between $\sigma_8$ and $\sigma_{15}$ there may 
 have been assignments to local variables.} % -- only the receiver will have remained the same.
 
Lemma \ref{lemma:call:return}
states that method calls  correspond  to pushing of frames (here $\sigma_2 \in \PushS  {\interpret {\sigma_1} {y}} {\sigma_1}$),
 that they leave the values of the caller's local variables unmodified ($\forall z. \interpret {\sigma_1} {z} = \interpret {\sigma_4} {z}$), and
method returns  correspond  to popping  a frame  (here $\sigma_3 \in  \PushSLong  {( {\interpret {\sigma_1} {y}},{\overline \alpha})} {\sigma_4}$), where the local map of the frame being popped contains the original arguments ($\overline {{\interpret {\sigma_1} {y}}}$) as well as some additional addresses ($\overline \alpha$).
The latter property -- that  the frame being popped from $\sigma_3$ contained the original arguments passed in to $\sigma_2$ -- holds because 
 we forbid assignments to formal parameters (but we do allow  assignments to the other local variables).
 We use $s_0\txteq s;\_$ as short for $s_0\txteq s$  or  $s_0\txteq s;s'$ for some $s'$.

 
 \begin{lemma}
 \label{lemma:call:return}
 For any modules $\Mtwo$, states $\sigma_1$, $\sigma_2$, $\sigma_3$, and $\sigma_4$, variables $x$, $y_0... y_n$:
% \\
% If  $\sigma_1.\prg{cont}\txteq x:= y_0.m(y_1,...y_n);\_$, and $\leadstoOrig {\Mtwo} {\sigma_1}   {\sigma_2} $ and 
% $\leadstoBoundedStarFin  {\Mtwo}  {\sigma_2}  {\sigma_3}$, and $\leadstoOrig {\Mtwo} {\sigma_3}   {\sigma_4} $, then
% \\
% $\sigma_2 \in \PushS  {\interpret {\sigma_1} {y}} {\sigma_1}\ \wedge\  \forall i. \interpret {\sigma_1} {y_i} = \interpret {\sigma_4} {y_i}\ \wedge\  \exists \overline \alpha. [\ \sigma_3 \in  \PushSLong  {( {\interpret {\sigma_1} {y}},{\overline \alpha})} {\sigma_4} \ ]$
 
 $  
   \left. %\{
   \begin{array}{l}  \ \strut \ \ \sigma_1.\prg{cont}\txteq x:= y_0.m(y_1,...y_n);\_ ,\\
    \ \strut \ \  \leadstoOrig {\Mtwo} {\sigma_1}   {\sigma_2} , \\
     \ \strut \ \  \leadstoBoundedStarFin  {\Mtwo}  {\sigma_2}  {\sigma_3}, \\
  \ \strut \ \  \leadstoOrig {\Mtwo} {\sigma_3}   {\sigma_4}, 
    \end{array} 
\right \}
 \mbox{implies} \ \ 
  \begin{cases}
     \ \strut \ \  \sigma_2 \in \PushS  {\interpret {\sigma_1} {y}} {\sigma_1},\\
     \ \strut \  \exists z.\ \sigma_3.\prg{cont}\txteq z;\_  \\
       \ \strut \ \       \forall z. \interpret {\sigma_1} {z} = \interpret {\sigma_4} {z},\\  
        \ \strut \ \       \exists \overline \alpha. [\ \sigma_3 \in  \PushSLong  {( {\interpret {\sigma_1} {y}},{\overline \alpha})} {\sigma_4} \ ]
    \end{cases} 
$
 \end{lemma}

% \subsection{\red{EVEN NEWER:} Method Calls and Returns}
%
% 
%{
%Calls , and in particular external calls, are a central consideration of our work; therefore a characterization of the states resulting immediately from a method call or a method return is crucial. 
%% which a method is called or returned from is important.
% The set ${\_} \triangledown {\_}$ does exactly that:
% }
% 
% {
%\begin{definition}
%\label{def:push:frame}
%Given a state $\sigma$, addresses $\overline \alpha$, and variables or addresses $\overline \va$, we define
%\begin{itemize}
%\item
%$ \PushS  {\alpha} {\sigma} \ =\ \{ \ \sigma' \ \mid\ \exists \phi,\psi,\chi,\phi_n,\phi',\ s.t\ $\\
%$\strut \hspace{2.3cm}  \sigma=(\phi\!\cdot\!\psi, \chi)\  \wedge\  \sigma'=(\phi_n\!\cdot\!\phi'\!\!\cdot\!\psi, \chi) \ \wedge\   rng(\phi)=rng(\phi') \ \wedge\ rng(\phi_n)=\overline \alpha \ \   \ \}$
%\item
%{$ \PushS  {\va}  {\sigma}$ is short for  $ \PushS  {\interpret {\sigma} {\va}} {\sigma} $.}
%\end{itemize}
% \end{definition}
%}\footnote{\red{DO we also need that $\overline \alpha$ were locally reachable in $\sigma$?}}
%
% 
%%Thus, if $\sigma_a$ is the result of pushing a frame onto $\sigma_b$, then $\sigma_a\in   \PushS  {\alpha} {\sigma_b}$. 
%%{Similarly, if $\sigma_b$ is the result of popping a frame from $\sigma_a$, then $\sigma_a\in   \PushS  {\alpha} {\sigma_b}$.}
% % below Therefore, as 
% Lemma \ref{lemma:push} says that % $\pushSymbol$ characterizes  calls and returns. 
%\  (\ref{pushOne}):\  If $\leadstoOrig {\Mtwo} {\sigma}   {\sigma'} $ and $\sigma'\!\in\!\PushS   {\alpha} {\sigma}$ then $\sigma'$ is a  \emph{callee} state of $\sigma$ --  {a direct successor state of    $\sigma$  after calling a method}. % with receiver and arguments $\overline \alpha$). 
% (\ref{pushTwo}):\  If $\leadstoOrig {\Mtwo} {\sigma}   {\sigma'} $ and $\sigma\! \in\! \PushS   {\alpha} {\sigma'}$  then $\sigma'$ is a  \emph{caller} state  of $\sigma$ -- {a direct successor  of $\sigma$}  after returning from a method.
%  Also, (\ref{oneLR}):\ A locally reachable object  {right after entering a call} was also locally reachable before {entering} the call.
%%\kjx{pretty sure we need one other case: (3a) a locally reachable
%%  obbject after a call could have been newly created by that call
%%  --
%%  \red{SD: Thank you for the comment. 
%%  (3) is supposed to talk about immediately after entering the call. It was missing an assumption -- in blue. 
%%  Please check the text and the maths.}}
% 
%  
%\begin{lemma}% [$\pushSymbol$  for calls and returns]
%\label{lemma:push}
%For all states $\sigma$, $\sigma'$, modules $\Mtwo$, and addresses $\overline \alpha$. If \ $\leadstoOrig {\Mtwo} {\sigma}   {\sigma'} $, \ then 
%\begin{enumerate}
%\item
%\label{pushOne}
%$\sigma'\in   \PushS  {\alpha} {\sigma}  \ \ \Longleftrightarrow \ \ 
%\exists x, m.[\ \ \sigma.\prg{cont} \txteq x:=\alpha_0.m(\alpha_1,...\alpha_n); \_\ \ ] $
%\item
%\label{pushTwo}
%$\sigma\in   \PushS  {\alpha} {\sigma'}  \ \ \Longleftrightarrow \ \ \exists x, \alpha'.[\ \  \sigma.\prg{cont}\txteq\alpha' \ \ \wedge \ \ \sigma'.\prg{cont}\txteq x:=\alpha';\_ \ \ ] $
%\item
%\label{oneLR}
%{${\leadstoOrig  {\Mtwo}  {\sigma}  {\sigma'}} \ \ \wedge\ \  \sigma'\in \PushS {\alpha} {\sigma}  \ \ \wedge \ \  \LRelevant {\overline \alpha} {\sigma} \ \ \wedge\ \   \LRelevant \alpha {\sigma'} \ \ \ \Longrightarrow$
%\\
%$\strut \hspace{2cm} \LRelevant \alpha {\sigma}$
%}
%
%\end{enumerate}
%
%\end{lemma}
%
% 
%
% Consider Fig. \ref{fig:UpSemantics} again: $\sigma_8\!\in\!   \PushS  {\alpha} {\sigma_7}$ for some $\overline \alpha$ -- {thus $\sigma_8$ is a callee state for 
% $\sigma_7$}. Also, 
% $\sigma_{14}\!\in\!\PushS  {\alpha'} {\sigma_{15}}$ for some $\overline {\alpha'}$ -- {thus $\sigma_{15}$ is a caller state for 
% $\sigma_{14}$}.
%\footnote{ $\overline \alpha$ may differ from $\overline {\alpha'}$, because between $\sigma_8$ and $\sigma_{15}$ there may 
% have been assignments to local variables.} % -- only the receiver will have remained the same.


% \label{s:underlying}
%   
%\section{Assertions} %\AssertLang -- the assertion language}
%\label{s:assertions}
%%\section{ Assertions}
%\label{sect:assertions}

\subsection{The syntax of Expressions and Assertions}

%\secomment
\susan{if you like this I will write macros so as not to have parameter lists with the keywords, alternatively this list could just have the identifiers and no descriptions\sd{not clear what is meant}}
In section~\ref{sect:chainmail} we introduced our assertion language \Chainmail with keywords 
$\CanAccess{}{}$ to check whether one object can call another, $\Calls{}$ for the current function call, 
$\Changes{}$ to check whether the next configuration will affect validity of some assertion, and 
 $\Next {}$ or $\Future {}$  for expressing an assertion will hold at
the immediate successor execution point or at some future point, and
$\Prev{}$ or $\Past{}$ to express  that an assertion held at the immediately previous or
some point in the past, and  $\Using{}{}$, 
for expressing that an assertion holds in
the sub-configuration determined by a witness.

The keywords enable \Chainmail assertions to support 
reflection over various aspects of the current 
runtime configurations, reflection over past or future configurations, and 
reflection over sub-configurations.
Assertions can contain logical operators and interestingly, the existential and universal quantifiers may quantify over object addresses, as well as 
over sets of addresses, numbers, and sequences of field identifiers of a given length.


%\secomment
\susan{if you prefer this paragraph then link it back to section 4. \sd{Yes, it may now be superfluous.}}


Assertions, $\A$, support standard logical operators, 
reflection over various aspects of the current 
runtime configurations, reflection over past or future configurations, and 
reflection over sub-configurations.
The standard logical operators are, unsurprisingly,
 $\wedge$, $\vee$, $\rightarrow$, $\neg$, $\exists$ and $\forall$.
Interestingly, the existential and universal quantifiers may quantify over object addresses, but also 
over sets of addresses, numbers, and sequences of field identifiers of a given length.
When reflecting over the current state, we can reflect over the class and contents of objects
(\eg \x:\prg{ClassId} or \x.\f=\y.\f'), whether an
object has direct access to (and thus may call on) another object $\CanAccess{\_}{\_}$,
and the current function call $\Calls{\prg{\_},\prg{\_},\prg{\_},\prg{\_}}$.
We can also talk about whether the next configuration will affect the 
validity of some assertion $\Changes{\_}$
\footnote{Note that $\Changes{\_}$ may be encoded; do we keep it?
The reason to keep it is that we can then talk of "permission" and "authority" }.  
We also support {\em temporal} modifiers, where $\Next \A$ or $\Future \A$  express  that $\A$ will hold at
the immediate successor execution point or at some future point, while
$\Prev \A$ or $\Past \A$ express  that $\A$ held at the immediately previous or
some point in the past.
Finally, we support a {\em spatial modifier}, $\Using{\A}{S}$, 
which expresses that assertion $\A$ holds in
the sub-configuration determined by the witness \prg{S}.


\begin{definition}[Assertions] The syntax of simple expressions $\SE$) and assertions ($\A$) is:
\label{def:assertions}

 $\begin{array}{lcl}
  \SE & ::= &  \prg{true}  \ \mid\ \prg{false}  \    \mid\ \prg{null}  \ \mid \ \x  \ \mid \ \SE.\f    \ \mid \ \SE.\f^n \   \ \mid\  \ \\
 ~ \\
\A &\ ::=\  &   \SE  \ \mid \ \SE > \SE \ \mid \  \SE=\SE  \ \mid \ \SE \equiv \SE\ \mid \   \SE:\prg{ClassId}  \ \mid \
    \SE\in\prg{S}   \ \mid  \ \A \rightarrow \A  \ \mid\  \  \\
 &   &  \exists \x.\A  \ \mid \  \exists \prg{S}:SET.\A  \ \mid \  \exists fs:FLD^k.\A
 \ \mid \  \exists k:\mathbb{N}.\A  \ \mid\  \
\\
 &    & \CanAccess x y \ \mid\  \ \Changes e \ \mid\  \Calls{\prg{x},\prg{y},\prg{m},\prg{z}} \ \mid\  \\  
 &    &  \Next \A  \ \mid \   \Future \A \ \mid \  \Prev \A    \ \mid \  \Past \A \ \mid \ \Using \A \prg{S }  \ \mid\  \
% \\
 \\
  &   &  \A \wedge \A  \ \mid\  \ \A \vee \A  \ \mid\  \ \neg A   \ \mid\  \ \forall \x.\A  \ \mid \  \forall \prg{S}:SET.\A  \ \mid \  \forall fs:FLD^k.\A
 \ \mid \  \forall k:\mathbb{N}.\A
\end{array}$


\end{definition}

Note that the operators $\wedge$, $\vee$,  $\neg$ and $\forall$  could have been defined  through the usual shorthands, \eg, $\neg \A$ is short for
$\A \rightarrow \ff$ \etc, but here we give full definitions instead.
 Validity of assertions has the format $\M\mkpair \M', \sigma \models \A$, where  $\M$ is the internal module, whose internal workings
 are opaque to the external, client module $\M'$.

\subsection{Configuration adaptation and configuration restrictions}
In order to define whether a runtime configuration satisfies an assertion we need two auxiliary concepts:
the adaptation of a runtime configuration to another one, and the restriction of a runtime configuration to only the set of objects from a
given set.

We need adaptation to deal with time, and the corresponding changes of scope. For example, the assertion
$\Future {\x.\f=\prg{3}}$, is satisfied if in some {\em future} configuration, the field  \f\, of the object that is pointed at by \x\, in the {\em current} configuration has the value \prg{3}; note that in the future  configuration, \x\, may be pointing to a different object, or may
even no longer be in scope (\eg if a nested call or a nesting call is executed).
Therefore, we introduce the operator \  $\adapt\;$,  \ \ which combines runtime configurations: $\sigma \adapt \sigma'$ adapts the second configuration to the top frame's view of the former: it returns a new configuration whose stack has  the top frame as taken from $\sigma$ and where the \prg{contn} has been consistently renamed, while the heap is taken from $\sigma'$. This allows us to interpret expressions  in the newer (or older) configuration $\sigma'$ but with the variables bound according to the top frame from $\sigma$; \eg we can obtain that value of \prg{x}.\prg{f} in configuration  $\sigma'$ even if \prg{x} was out of scope. The consistent renaming of the code allows the correct modelling of execution (as needed,   for the semantics of  nested time assertions, as \eg in $\Future {\x.\f=\prg{3} \wedge \Future {\x.\f=\prg{5}}}$


 \begin{definition}[Adaptation on Runtime Configurations]  The operator $\adapt$\ \  is a binary operator on runtime configurations.
 \label{def:config:adapt}
 $~ $

\begin{itemize}
\item
$\sigma \adapt \sigma' \triangleq (\phi''\cdot\psi',\chi')$  \IFF $\sigma=(\phi\cdot\_,\_)$, and $\sigma'= (\phi'\cdot\psi',\chi')$, and
 \\
$\ \strut \ \ \hspace{1.45in} $
$\phi$=$(\prg{contn},varMap)$, and $\phi'$=$(\prg{contn}',varMap')$, and
 \\
$\ \strut \ \ \hspace{1.45in} $     % $\phi''$ such that
  $\phi''=(\, \prg{contn}'[\prg{zs}/\prg{zs}' ],\,varMap'[\prg{zs}'\mapsto varMap(\prg{zs})]\, ) $, where
 \\
$\ \strut \ \ \hspace{1.45in} $
\prg{zs}=$dom(varMap)$, and
 \\
$\ \strut \ \ \hspace{1.45in} $      $\prg{zs}'$ is a set  of variables with  the  same cardinality as \prg{zs}, and
 \\
$\ \strut \ \ \hspace{1.45in} $   all variables in
$\prg{zs}'$  are fresh in $varMap$ and in $varMap'$.


\end{itemize}

\end{definition}

 On the other hand, an assertion of the form $\Using{A}{S}$ promises that $\A$ holds in subconfiguration, whose heap is restricted to the objects from \prg{S}.

 \begin{definition}[Restriction on Runtime Configurations]  The restriction operator~$\;\restrct{} {} $ applied to a runtime configuration $\sigma$ and a set $R$ is defined as follows:
 \label{def:config:restrct}
 $~ $

\begin{itemize}
\item
$\restrct {\sigma}{\prg{S}} \ \triangleq \ (\phi, \chi')$, \IFF  $\sigma$=$(\phi,\chi)$, \ and  \  $dom(\chi')=\interp {\prg{S}} {\sigma}$, and  \\
$\ \strut \ \ \hspace{1.2in} $
 $\forall \alpha\!\in\!dom(\chi').[ \ClassOf {\alpha} {\chi'} =  \ClassOf {\alpha} {\chi}\ \wedge \ \forall \f.  \chi'(\alpha,\f)=\chi(\alpha,\f)]$.
\end{itemize}
\end{definition}

\subsection{Satisfaction of assertions}



\begin{definition}[Interpretations for simple expressions]

For any runtime configuration, $\sigma$, and any $k\in \mathbb{N}$, and any simple expression, $\SE$, we define its interpretation as follows:

\begin{itemize}
     \item
  $\interp {\prg{true}}{\sigma}$ $ \triangleq$   \prg{true}, \ and \ \    $\interp {\prg{false}}{\sigma}$ $ \triangleq$ \prg{false}, \ and \ \
   $\interp {\prg{null}}{\sigma}$ $ \triangleq$  \prg{null}
  \item
  $\interp {\x}{\sigma}$ $ \triangleq$ $\phi(\x)$  \ \ if \ \ $\sigma$=$(\phi\cdot\_,\_)$
  \item
  $\interp {\SE.\prg{f}}{\sigma}$ $ \triangleq$ $\chi({\interp {\SE}{\sigma}}, \prg{f})$  \ \ if \ \ $\sigma$=$(\_,\chi)$
   \item
     $\interp {\SE.\prg{f}^0}{\sigma}$ $ \triangleq$  $\interp {\SE}{\sigma}$, \ \ \ and \ \ \ $\interp {\SE.\prg{f}^{k+1}}{\sigma}$ $ \triangleq$   $\chi({\interp {\SE.\prg{f}^k}{\sigma}})(\prg{f})$, where $\sigma$=$(\_,\chi)$.
   \end{itemize}
\end{definition}

\begin{lemma}[Interpretation corresponds to execution]
For any simple expression $\SE$, runtime configuration $\sigma$, and value $v$:

\begin{itemize}
     \item
  $\interp \SE {\sigma}$ = $v$\ \     if and only if \ \ $\M_\emptyset, \sigma[\prg{contn}\mapsto \SE] \leadsto v$,\\
  where $\M_\emptyset$ stands for the empty module.
  \item
   $\interp \SE {\sigma}$ = $v$\ \     if and only if \ \ $\M, \sigma[\prg{contn}\mapsto \SE] \leadsto v$ \ \ \ for any module $\M$ .
   \end{itemize}
   \end{lemma}

   \begin{proof} The  first guarantee is proven structural induction  over the definition of $\SE$.
   The second guarantee  is a corollary of the first guarantee  and of lemma \ref{lemma:linking:properties}.\end{proof}


\begin{definition}[Satisfaction of  Assertions] We define below when a configuration satisfies an assertions. We first extend the definition of interpretation
to simple expressions.
\label{def:valid:assertion}

We first consider simpler assertions which only involve expressions:

\begin{itemize}
\item
$\M\mkpair \M', \sigma \models\SE$ \IFF  $\interp{\SE}{\sigma}$ = \prg{true}.
\item
$\M\mkpair \M', \sigma \models\SE>\SEPrime$ \IFF $\interp{\SE}{\sigma}$ > $\interp{\SEPrime}{\sigma}$.
\item
$\M\mkpair \M', \sigma \models\SE=\SEPrime$ \IFF $\interp{\SE}{\sigma}$ = $\interp{\SEPrime}{\sigma}$.
\item
$\M\mkpair \M', \sigma \models\SE\equiv\SEPrime$ \IFF $\SE$ and $\SEPrime$ are textually identical.
\item
$\M\mkpair \M', \sigma \models \SE:\prg{ClassId}$ \IFF $\ClassOf {\interp{\SE}{\sigma}} {\sigma}$ = $\prg{ClassId}$.
\item
$\M\mkpair \M', \sigma \models \SE\in \prg{S}$ \IFF $\interp{\SE}{\sigma}\in \interp{\prg{S}}{\sigma}$.
\end{itemize}

Next, we consider assertions involving existential quantifiers over program variables, field sequences, sets and numbers.

\begin{itemize}
\item
$\M\mkpair \M', \sigma \models \exists x.\A$ \IFF
$\M\mkpair \M', \sigma[\prg{z}\mapsto \alpha] \models  \A[\prg{x}/\prg{z}]$ \ for some  $\alpha\in dom(\sigma)$, and   \prg{z} free in $\sigma$ and $\A$.\item
$\M\mkpair \M', \sigma \models \exists \prg{S}:\prg{SET}\!.\,\A$ \IFF  $\M\mkpair \M', \sigma[\prg{Q}\mapsto R] \models  \A[\prg{S}/\prg{Q}]$ \\
$\strut ~ \hspace{1.4in} $ for some set of addresses $R\subseteq dom(\sigma)$, and   \prg{Q} free in $\sigma$ and $\A$.

\item
$\M\mkpair \M', \sigma \models  \exists \prg{fs}:\prg{FLD}^k\!.\,\A$ \IFF
$\M\mkpair \M', \sigma \models  \A[\prg{fs}/\prg{f}_1.\f_2.\,...\,\prg{f}_k]$\  for  $k$ field identifiers $\prg{f}_1$,..,$\prg{f}_k$.
\item
$\M\mkpair \M', \sigma \models  \exists \prg{n}:\prg{Nat}.\A$ \IFF  $\M\mkpair \M', \sigma \models \A[\prg{n}/k]$\ \ for some $k\in\mathbb{N}$.

\end{itemize}

And now, we consider the assertions which involve space, time or control:

\begin{itemize}
\item
$\M\mkpair \M', \sigma \models  \CanAccess{\prg{x}}{\prg{y}}$   \IFF  \begin{itemize}
\item
$\interp {\x} {\sigma}$=$\interp {\y} {\sigma}$, or
\item
$\interp {\x.\f} {\sigma}$=$\interp {\y} {\sigma}$  for some field \prg{f},  or
\item
$\interp {\x} {\sigma}$=$\interp {\this} {\sigma}$ and
  $\interp {\y} {\sigma}$=$\interp {\z} {\sigma}$,
\
and \z\ appears in  $\sigma$.\prg{contn}.\footnote{
That is, $\CanAccess{\prg{x}}{\prg{y}}$ expresses that \x has a {\em direct} path to \y.
In more detail, in the current frame,
either \x and \y\, are  aliases, or \x points to an object which has a field
whose value is the same as \y, or \x is the currently executing object and \y is
 a local variable or formal parameter which appears in the code in the continuation.
 %That means, that variables which were introduced into the variable map in order to give meaning to existentially quantified assertions are not considered.
 }
 \end{itemize}
 \item
 $\M\mkpair \M', \sigma \models   \Changes{\prg{e}}$  \IFF
 $\exists \sigma'.\, [\ \ \M\mkpair \M',\sigma \leadsto \sigma' \ \wedge \interp{e}{\sigma} \neq \interp{e}{\sigma\triangleleft \sigma'} \ \  ]$.
   \item
$\M\mkpair \M', \sigma \models  \Calls{\prg{x},\prg{y},\prg{m},\prg{z}}$ \IFF \
 $\sigma.\prg{contn}$=\prg{u.m(v);\_}\ \ for some variables \prg{u} and \prg{v},  \ and \
\\ $\strut ~ \hspace{1.4in} $
$\interp{\prg{this}}{\sigma}$=$\interp{\prg{x}}{\sigma}$, \ and\ $\interp{\prg{y}}{\sigma}$=$\interp{\prg{u}}{\sigma}$,
 \ and\ $\interp{\prg{z}}{\sigma}$=$\interp{\prg{v}}{\sigma}$.
 \item
  $\M\mkpair \M', \sigma \models  \Next \A $
  \IFF
  $\exists \sigma'.\, [\ \ \M\mkpair \M',\phi \leadsto  \sigma' \ \wedge \M\mkpair \M',\sigma\adapt\sigma' \models \A \ \  ]$,
 \\
$\strut ~ \hspace{1.4in} $  and where $\phi$ is
so that $\sigma$=$(\phi\cdot\_,\_)$.\footnote{$\M\mkpair \M', \sigma \models  \Future \A $ holds if
$\A$ holds in some configuration $\sigma'$ which arises from execution of $\phi$, where $\phi$ is the top frame of $\sigma$. By requiring that $\phi \leadsto^* \sigma' $ rather than
$\sigma \leadsto^* \sigma' $ we are restricting the set of possible future configurations to
just those that are caused by the top frame.
Namely, we do not want to also consider the effect of  enclosing function calls.
This allows us to write more natural specifications
when giving necessary conditions for some future effect.
}
\item
  $\M\mkpair \M', \sigma \models  \Future \A $
  \IFF
  $\exists \sigma'.\, [\ \ \M\mkpair \M',\phi \leadsto^* \sigma' \ \wedge \M\mkpair \M',\sigma\adapt\sigma' \models \A \ \  ]$,
 \\
$\strut ~ \hspace{1.4in} $  and where $\phi$ is
so that $\sigma$=$(\phi\cdot\_,\_)$.  
  \item
 $\M\mkpair \M', \sigma \models  \Prev \A $ \IFF
 $\forall \sigma_1, \sigma_2. [\ \ \Initial{\sigma_1}\ \wedge \   \M\mkpair \M', \sigma  \leadsto^*  \sigma_2 \ \wedge \   \M\mkpair \M', \sigma_2  \leadsto   \sigma  
$
 \\
$\strut ~ \hspace{1.9in} $  $ \longrightarrow \ \ \   
 \M\mkpair \M', \sigma\adapt\sigma_2  \models \A\ \
 ]$\footnote{past includes the present, perhaps change this}
 \item
 $\M\mkpair \M', \sigma \models  \Past \A $ \IFF
 $\forall \sigma_1, ... \sigma_n. [\ \ \Initial{\sigma_1}\ \wedge \  \sigma_n=\sigma 
  \ \wedge \ \forall i\in[1..n). \M\mkpair \M', \sigma_{i} \leadsto  \sigma_{i+1}
$
 \\
$\strut ~ \hspace{1.9in} $  $ \longrightarrow \ \ \  \exists j\in [1..n-1).
 \M\mkpair \M', \sigma\adapt\sigma_j  \models \A\ \
 ]$\footnote{past includes the present, perhaps change this}
 \item
 $\M\mkpair \M', \sigma \models \Using {\A} {\prg{S}}$
 \IFF
 $\M\mkpair \M', \restrct \sigma {\prg{S}} \models  \A  $.
 \item
  \sd{$\M\mkpair \M', \sigma \models \External {\prg{e}}$}
  \IFF
$\sd{{\interp{\SE}{\sigma}} {\sigma}\notin dom(\M)}$
\end{itemize}

The remaining assertions introduce the remaining logical operators (\ie $\wedge$, $\vee$, $\neg$ and the universal quantifiers). These could be encoded in terms of the preceding operators, but we nevertheless give their meaning explicitly here.

\begin{itemize}
\item
$\M\mkpair \M', \sigma \models \A \rightarrow \A' $ \IFF  $\M\mkpair \M', \sigma \models \A $ implies $\M\mkpair \M', \sigma \models \A' $
\item
$\M\mkpair \M', \sigma \models  \A \wedge \A'$   \IFF  $\M\mkpair \M', \sigma \models  \A $
and $\M\mkpair \M', \sigma \models  \A'$.
\item
$\M\mkpair \M', \sigma \models  \A \vee \A'$   \IFF  $\M\mkpair \M', \sigma \models  \A $
or $\M\mkpair \M', \sigma \models  \A'$.
\item
$\M\mkpair \M', \sigma \models  \neg\A$   \IFF  $\M\mkpair \M', \sigma \models  \A $
does not hold.
\item
$\M\mkpair \M', \sigma \models \forall x.\A$ \IFF
$\sigma[\prg{z}\mapsto \alpha] \models  \A[\prg{x}/\prg{z}]$ \ for all  $\alpha\in dom(\sigma)$, and   all \prg{z} free in $\sigma$ and $\A$.\item
$\M\mkpair \M', \sigma \models \forall \prg{S}\!\!:\!\!\prg{SET}.\A$ \IFF  $\M\mkpair \M', \sigma[\prg{Q}\mapsto R] \models  \A[\prg{S}/\prg{Q}]$ \\
$\strut ~ \hspace{1.4in} $ for all sets of addresses $R\subseteq dom(\sigma)$, and  all \prg{Q} free in $\sigma$ and $\A$.

\item
$\M\mkpair \M', \sigma \models \forall \prg{fs}\!\!:\!\!\prg{FLD}^k.\A$ \IFF
$\M\mkpair \M', \sigma \models  \A[\prg{fs}/\prg{f}_1.\f_2.\,...\,\prg{f}_k]$\  for  all  field identifiers $\prg{f}_1$,..,$\prg{f}_k$.
\item
$\M\mkpair \M', \sigma \models  \forall \prg{n}:\prg{Nat}.\A$ \IFF  $\M\mkpair \M', \sigma \A[\prg{n}/k]$\ \ for all $k\in\mathbb{N}$.
\end{itemize}\end{definition}
 
We define equivalence of   assertions in the usual sense: two assertions are equivalent if they are satisfied  in
the context of the same configurations.
Similarly, an assertion entails another assertion, iff all configurations 
which satisfy the former also satisfy the latter.  

\begin{definition}[Equivalence and entailments of assertions]
$ ~ $

\begin{itemize}
\item
$\A \equiv \A'\  \IFF\    \forall \sigma.\, \forall \M, \M'. \ [\ \ \M\mkpair \M', \sigma \models \A\ \mbox{ if and only if }\ \M\mkpair \M', \sigma \models \A'\ \ ].$
\item
$\A \subseteqq \A'\  \IFF\    \forall \sigma.\, \forall \M, \M'. \ [\ \ \M\mkpair \M', \sigma \models \A\ \mbox{ implies }\ \M\mkpair \M', \sigma \models \A'\ \ ].$
\end{itemize}
\end{definition}



\begin{lemma}[Assertions are classical-1]
For all runtime configurations $\sigma$,    assertions $\A$ and $\A'$, and modules $\M$  and $\M'$, we have
\begin{enumerate}
\item
$\M\mkpair \M', \sigma \models \A$\ or\ $\M\mkpair \M', \sigma \models \neg\A$
\item
$\M\mkpair \M', \sigma  \models \A \wedge \A'$ \SP if and only if \SP $\M\mkpair \M', \sigma \models \A$ and $\M\mkpair \M', \sigma  \models \A'$
\item
$\M\mkpair \M', \sigma  \models \A \vee \A'$ \SP if and only if \SP $\M\mkpair \M', \sigma  \models \A$ or  $\sigma \models \A'$
\item
$\M\mkpair \M', \sigma  \models \A \wedge \neg\A$ never holds.
\item
$\M\mkpair \M', \sigma  \models \A$ and  $\M\mkpair \M', \sigma  \models \A \rightarrow \A'$  implies
$\M\mkpair \M', \sigma  \models \A '$.
\end{enumerate}
\end{lemma}
\begin{proof} By application of the corresponding definitions from \ref{def:valid:assertion}.\end{proof}.

\begin{lemma}[Assertions are classical-2]
For     assertions $\A$, $\A'$, and $\A''$ the following equivalences hold
\label{lemma:basic_assertions_classical}
\begin{enumerate}
\item
$ \A \wedge\neg \A \ \equiv \  \prg{false}$
\item
$ \A \vee \neg\A   \ \equiv \  \prg{true}$
\item
$ \A \wedge \A'  \ \equiv \  \A' \wedge \A$
\item
$ \A \vee \A'  \ \equiv \  \A' \vee \A$
\item
$(\A \vee \A') \vee \A'' \ \equiv \  \A \vee (\A' \vee\A'')$
\item
$(\A \vee \A') \wedge \A'' \ \equiv \  (\A \wedge \A')\, \vee\, (\A \wedge \A'')$
\item
$(\A \wedge \A') \vee \A'' \ \equiv \  (\A \vee \A')\, \wedge\, (\A \vee \A'')$
\item
$\neg (\A \wedge \A') \  \ \equiv \  \neg  \A   \vee\, \neg \A''$
\item
$\neg (\A \vee \A') \  \ \equiv \  \neg  \A   \wedge\, \neg \A''$
\item
$\neg (\exists \prg{x}.\A )  \  \ \equiv \  \forall \prg{x}.(\neg  \A)$
\item
$\neg (\exists k:\mathbb{N}.\A )  \  \ \equiv \  \forall  k:\mathbb{N}.(\neg  \A)$
\item
$\neg (\exists \prg{fs}:FLD^k.\A )  \  \ \equiv \  \forall \prg{fs}:FLD^k.(\neg  \A)$
\item
$\neg (\forall \prg{x}. \A)  \  \ \equiv \  \  \exists \prg{x}.\neg(\A )$
\item
$\neg (\forall k:\mathbb{N}. \A)  \  \ \equiv \  \  \exists k:\mathbb{N}.\neg(\A )$
\item
$\neg (\forall \prg{fs}:FLD^k. \A)  \  \ \equiv \  \  \exists \prg{fs}:FLD^k.\neg(\A )$
\end{enumerate}
\end{lemma}
\begin{proof}
All points follow by application of the corresponding definitions from \ref{def:valid:assertion}.
 \end{proof}

Notice that satisfaction is not preserved with growing configurations; for example, the assertion $\forall \x. [\ \x : \prg{Purse} \rightarrow \x.\prg{balance}>100\ ]$ may hold in a smaller configuration, but not hold in an extended configuration. Nor is it preserved with configuratio s getting smaller; consider \eg $\exists \x. [\ \x : \prg{Purse} \wedge \x.\prg{balance}>100\ ]$

% \begin{definition}
%We say that $\sigma \vdash \A$ if for any  \x\, is free in $\A$ and any
%  any term $\x.\f_1...\f_n$ appearing in $\A$,
% the interpretation $\interp{\x.\f_1...\f_n} \sigma$ is defined.
%\end{definition}
%
%Note that if we take $n=0$ in the definition above we obtain as corollary that   all variables that appear free in $\A$ they  are in the domain of the top frame in $\sigma$.
%
%\begin{lemma}[Preservation of satisfaction] $ $
%\label{lemma:preserve:valid}
%\begin{itemize}
%\item
%If  $\sigma \vdash \A$ and $\M\mkpair \M',  \sigma \vdash \A$ and   $\sigma' \subconf \sigma$, \  then  \ $\M\mkpair \M',  \sigma' \models \A$.
%\end{itemize}
%\end{lemma}

Finally, we define satisfaction of assertions by modules: A module $\M$ satisfies an assertion $\A$ if for all modules $\M'$, in all configurations arising from executions of $\M\mkpair\M'$, the assertion $\A$ holds.

\begin{definition}
\label{def:module_satisfies}
For any module $\M$, and  assertion $\A$, we define:
\begin{itemize}
\item
$\M \models \A$ \IFF  $\forall \M'.\, \forall \sigma\!\in\!\Arising{\M\mkpair\M'}.\   \M\mkpair\M', \sigma \models \A$
\end{itemize}
\end{definition}

The next sections contain full details. Section~\ref{sect:LangOO} defines a small oo language,  \LangOO, in terms of its  syntax, the structure of its runtime configurations $\sigma$, and its operational semantics in terms of a judgment with   the form $\M \mkpair \M', \sigma \leadsto \sigma'$. Section \ref{sect:assertions} gives the full definition of assertions $\A$,   when assertions are valid in given runtime configurations in terms of a judgment with   the form $\M \mkpair \M', \sigma \models \A$, and finally defines modules' adherence to assertions  in terms of a judgment with   the form $\M \models \A$.



%
%
% \section{Preservation of Assertions}
% \label{s:preserve}
% Program logics require some form of framing, \ie conditions under which  satisfaction of  assertions is preserved across program execution. 
This is the subject of the current Section.

We start with  Lemma \ref{l:assrt:unaffect}  which says that satisfaction of an assertion is not affected by replacing a variable by its value, nor by changing the continuation in a state.


\begin{lemma}
\label{lemma:addr:expr}
\label{l:assrt:unaffect}
For all $M$, $\sigma$, $\alpha$, $x$, $\re$, $stmt$, and $A$:

\begin{enumerate}
\item
\label{one:ad:exp}
\label{l:assrt:unaffect:one}
$\satisfiesA{M}{\sigma}{A}   \ \ \ \Longleftrightarrow\ \ \ \satisfiesA{M}{\sigma}{A[{\interpret \sigma x}/x]}   $ 
% holds, but not used
%\item
%$\eval{M}{\sigma}{{\re}}{\alpha}  \ \ \Longrightarrow\ \  [ \ \satisfiesA{M}{\sigma}{A} \  \Longleftrightarrow\   \  \satisfiesA{M}{\sigma}{A[\alpha/\re]} \  \  ]$
\item
$ \satisfiesA{M}{\sigma}{A}   \ \ \ \Longleftrightarrow\ \ \ \satisfiesA{M}{\sigma[\prg{cont}\mapsto stmt]}{A}$ 
\end{enumerate}

\end{lemma}

 \noindent
We now move to assertion preservation across method call and return. % (Sect \ref{s:preserve:call:ret}),   or heap modification (Sect \ref{s:preserve:encaps}).

\subsection{Stability} %{Stability: Preservation across method call and return}
\label{s:preserve:call:ret}
In most program logics, satisfaction of  variable-free assertions  is preserved when pushing/popping frames
-- \ie immediately after entering a method or  returning from it.
This, however, is not   the case for our assertions, where protection depends not only of the heap but also 
on the mapping from the top frame. % set of objects reachable from the top frame.
% whether $\alpha$ is protected, \ie whether   $\inside {\alpha}$ holds, depends on the heap as well as the set of objects reachable from the top frame;
% the latter  changes when the frame changes.
%This is shown, 
\Eg  Fig. \ref{fig:Protected} where 
$\sigma_2 \not\models \inside {o_6}$, but after pushing a frame, we have $\sigma_3  \models \inside {o_6}$.
%$\sigma_1 \not\models \inside {o_3}$, then $\sigma_2 \models \inside {o_3}$, and then $\sigma_3  \not\models \inside {o_3}$
 
 

%Nevertheless,  
{Assertions} which do  not contain  $\inside {\_}$, called   $\Stable {\_}$, are preserved when pushing/popping  frames.  
{Less strictly}, assertions which do  not contain $\inside {\_}$ in \emph{negative} positions, called $\Pos {\_}$,  are preserved when pushing  {internal} frames provided that the range of the new frame contains locally reachable addresses.
These properties are stated in Lemma \ref{l:stbl:preserves} --
\cf    \A\ \ref{app:preserve} for full definitions and proofs. 

\begin{lemma}
For all  states $\sigma$, frames $\phi$,   all assertions $A$ with  $\fv(A)=\emptyset $
\label{l:preserve:asrt}
\label{l:stbl:preserves} 
\begin{itemize}
\item 
$\Stable{A} \  \ \  \Longrightarrow  \  \ \  [\ \ M, \sigma \models A \ \ \Longleftrightarrow \ \  M,{\PushSLong \phi \sigma} \models A\ \ ]$
\item 
\label{l:preserve:asrt:two}
$\Pos{A}   \ \wedge    \ Rng(\phi)\!\subseteq\! \LRelevantO   \sigma\ \wedge\  M, {\PushSLong \phi \sigma} \models  \intThis\  \wedge  \ M, \sigma \models A $
% \\
%$\strut \hspace{2cm} \ \  \ \ 
$\  \ \Longrightarrow \  \ M,{\PushSLong \phi \sigma} \models A\ $
\end{itemize}
\end{lemma}


While $Stb^+$ assertions \emph{are} preserved  when pushing  internal frames,   they  are \emph{not} necessarily preserved when pushing  external frames   (\cf  Ex. \ref{push:does:not:preserve}), \emph{nor} when popping frames   (\cf Ex. \ref{ex:pop:does:not:preserve}). 

 
\begin{example}[$Stb^+$  not always preserved by External Push]
\label{push:does:not:preserve}
%In    Fig. \ref{fig:Protected}, where $\sigma_2$ by pushing external frame onto $\sigma_1$, and 
%$\sigma_1 \models \inside {o_3}$ but $\sigma_2 \not\models \inside {o_3}$.
% Bad syntax above, but fits in one line
In Fig. \ref{fig:Protected} we have $\sigma_2 \models \inside {o_4}$, but when pushing   frame $\phi_3$ with an external receiver 
 and $o_4$ as argument,  we have $\sigma_3 \not\models \inside {o_4}$.
%   $\sigma_3$ emerges by pushing an external frame onto $ \sigma_2$’s stack we have $\sigma_2 \models \inside {o_3}$ and $\sigma_3 \not\models \inside {o_3}$ 
\end{example}
 

\begin{example}[$Stb^+$ not always preserved by Pop]
\label{ex:pop:does:not:preserve}
Take Fig. \ref{fig:Protected} again: Returning from $\sigma_3$ would give 
$\sigma_2$. And we have $\sigma_3 \models \inside {o_6}$, but after method return we have  $\sigma_2 \not\models \inside {o_6}$.
\end{example}


% \vspace{.1cm}
{We work with  $Stb^+$  assertions   (the  $Stbl$ requirement is too strong). 
But  we need to address the lack of preservation of $Stb^+$ assertions  {for external method calls and returns}.
 %,  makes it difficult to  ensure certain (perhaps weaker) assertions hold when entering   or returning from a method. 
We do the former   through    \emph{adaptation}   ($\pushSymbolInText$ in Sect \ref{s:viewAndProtect}), and the latter through  
\emph{\scoped satisfaction} (\S \ref{s:scoped:valid}).
  }
   
 \subsection{Adaptation } % and Protection}
 \label{s:preserve:encaps}
 \label{s:viewAndProtect}
 
 As  discussed in \S  \ref{sec:howThird}, it is possible to know that a certain object will be protected ($\inside \re$) after pushing a frame, if we know enough about the arguments of that frame. 
 %  for an assertion  not to be satisfied at the caller
% but to be  satisfied at the called viewpoint {(the callee)}. %:  right before a method call, $A$ might not be satisfied,
We define the operator $\pushSymbolInText$ which  translates an assertion from the viewpoint of a callee whose arguments are $\overline y$, to that of the caller.

\begin{definition}
\label{def:push}
[The $\pushSymbolInText$  operator]  

$
\begin{array}{c}
\begin{array}{l}
\begin{array}{rclcrcl}
  \PushAS y {\inside \re} & \triangleq &  \protectedFrom \re {\overline {y} }
  & \ \ \  \ &
  \PushAS y   {(A_1  \wedge  A_2)} & \triangleq &  (\PushAS y  { A_1})  \wedge  ( \PushAS y  {A_2} )  
\\ 
 \PushAS y {(\protectedFrom \re {\overline {u}})} &  \triangleq& \protectedFrom \re {\overline {u}} 
  & &
 \PushAS y  {(\forall x:C.A)} & \triangleq & \forall x:C.({\PushAS y A} )  
  \\  
  \PushAS y  {(\external \re)} &  \triangleq & {\external \re}  %   \PushAS y  {(\external \re)} & \triangleq &   {\external \re}
  & & 
  \PushAS y  {(\neg A)} &  \triangleq & \neg( {\PushAS y A} )  
    \\
     \PushAS y  {\re} &  \triangleq&   \re %    \PushAS y  {(\internal \re)} &  \triangleq & {\internal \re}
    & &
    \PushAS y  {(\re:C)} &  \triangleq&   \re:C 
 \end{array}
\end{array}
\end{array}
$
\label{f:Push}
\end{definition}

Only the first equation in  Def.  \ref{def:push}  is interesting: for $\re$ to be {protected at  a} callee with argument $\overline y$, it should be protected from   % the call's 
these arguments -- thus
  $\PushAS y {\inside \re} = \protectedFrom {\re} {\overline {y}}$. 
The notation $\protectedFrom {\re} {\overline {y}}$   stands for $\protectedFrom \re {y_0}\  \wedge\  ...  \wedge \protectedFrom \re {y_n}$, assuming that $\overline y$=${y_0, ... y_n}$.

Lemma \ref{lemma:push:ass:state}  states that %  (\ref{lemma:push:ass:state:one}) and (\ref{lemma:push:ass:state:two}):  
indeed, $\pushSymbolInText$ adapts assertions from the callee to the caller, and is the counterpart to the %states' 
operator $\pushSymbol$.
In particular:\ \ 
 (\ref{lemma:push:ass:state:one}):\ If the caller,   $\sigma$,  satisfies  $\PushASLong  {Rng(\phi)} {A}$, then  the callee,   ($\PushASLong {\phi} {\sigma}$), satisfies $A$.
\ \ (\ref{lemma:push:ass:state:two}):\ The opposite implication holds if the callee is external.
\ \ (\ref{l:push:stbl}):\    $\pushSymbolInText$ turns an assertion into a stable assertion.
\ \ (\ref{lemma:internal:adapted}): \ {In internal states, an assertion implies its adapted version. % (Lemma \ref{lemma:internal:adapted}).}
  
%% The below was fine when we had a more detailed version
% (1) If \ the caller state
%$\sigma$ satisfies  $\PushAS y A$, then  the callee's state  ($\PushASLong {\phi} {A}$) % (after pushing a frame with the values of $\overline y$) 
%will satisfy $A$.  
%(2) is the opposite: {If the called state ($\PushASLong {\phi} {A}$) % (in which we pushed a frame with the values for $\overline y, \overline z$
% satisfies $A$, then the caller state will satisfy 
%$\PushAS {y} {A}$. } % will hold {in the caller state}. % after popping that frame.

\begin{lemma} 
\label{lemma:push:ass:state}
For  any  state  $\sigma$, assertion $A$,   % with $\fv(A)=\emptyset$,  %variables   $\overline y$, 
frame $\phi$, and variables $\overline y$: % with $Range (\phi)=\overline {\interpret \sigma y}$:


\begin{enumerate}
 \item
 \label{lemma:push:ass:state:one}
$\fv(A)=\emptyset \ \ \wedge \ \  M, \sigma \models \PushASLong  {Rng(\phi)} {A}\ \  \ \ \ \ \  \ \ \    \Longrightarrow  \ \ \ \ M,  \PushSLong {\phi} {\sigma}   \models A$
\item
 \label{lemma:push:ass:state:two}
$\fv(A)=\emptyset \ \ \wedge \ \ M,  \PushSLong {\phi} {\sigma}   \models  A  \wedge \extThis    \ \  \ \ \    \Longrightarrow  \ \ \ \ M, \sigma \models \PushASLong  {Rng(\phi)} {A}$
\item
\label{l:push:stbl}
$\overline y\neq [] \ \ \ \ \Longrightarrow \ \ \ \ \Stable{ \PushAS y A}$
\item
\label{lemma:internal:adapted}
$M, \PushSLong {\phi} {\sigma}  \models A \wedge \intThis \ \ \Longrightarrow M, \PushSLong {\phi} {\sigma}  \models  \PushASLong {Rng(\phi)} {A}$.
\end{enumerate}
\end{lemma}
}



Proofs of the lemma can be found in \A\ \ref{appendix:adaptation}. We next discuss examples demonstrating why the \prg{extl} and \prg{intl} requirements are necessary for (\ref{lemma:push:ass:state:two}) and (\ref{lemma:internal:adapted}) above, and why  original and adapted versions of assertions are not always comparable.

\begin{example}[Comparing Validity of Adapted and Original versions]
\label{push:does:not:imply}
\notesep   $A$ does not imply $\PushAS {y} {A}$:\  \Eg  take 
  a   $\sigma_1$ where $\interpret {\sigma_1} {\prg{this}}$ = $o_1$, and $o_1$ is {external},  and there is no other object. Then, we have
$\_,\sigma_1 \models \inside {\prg{this}}$ and $\_,\sigma_1 \not\models \protectedFrom {\prg{this}} {\prg{this}}$.
\notesep Nor does  $\PushAS {y} {A}$  imply $A$.\  \Eg  take a $\sigma_2$ where $\interpret {\sigma_2} {\prg{this}}$ = $o_1$,
 $\interpret {\sigma_2} {x}$ = $o_2$ , and  $\interpret {\sigma_2} {x.f}$ = $o_3$, and $o_2$ is external, and there are no other objects or fields.
 Then $\_,\sigma_2 \models   \protectedFrom {x.f} {\prg{this}}$ but  $\_,\sigma_2 \not\models \inside {x.f}$.
\end{example}






 
\subsection{Encapsulation} % : Preservation across heap modifications}
\label{s:preserve:encaps}
 
Proofs of adherence to specifications  hinge on the expectation that some,  specific, assertions are always satisfied unless some internal (and thus known) computation took place. 
{We call such assertions   \emph{encapsulated}.}
 

The judgment $M\ \vdash \encaps{A}$  expresses that satisfaction of $A$ involves looking into the state of  
internal objects only,  \cf. Def \ref{d:encaps:sytactic}.
{On the other hand, $M  \models \encaps{A}$ says that assertion $A$  is  \emph{encapsulated} by a module $M$, \ie in all possible states % which arise from execution of module $M$ with any other  module $\Mtwo$, 
execution which involves $M$ and any set of other modules $\Mtwo$, always satisfies  $A$  unless the execution  included internal execution steps}.
 

\begin{definition}[An assertion $A$ is \emph{encapsulated} by module $M$] $~$ 
\label{d:encaps}
%\begin{itemize}
%\item 
%$M \vdash \encaps{A}  \ \   \triangleq  \ \  \exists \Gamma.[\ M; \emptyset \vdash \encaps{A}; \Gamma\ ]$ \ \  as defined in Fig. \ref{f:encaps}.
%\item

$ % \begin{equation}
    M\ \models \encaps{A}\ \   \triangleq  \ \   
    \begin{cases}
     \forall \Mtwo, \sigma, \sigma',  \overline{\alpha}, \overline{x} \mbox{ with } \overline{x}=\fv(A)\\ % , \mbox{and }  \arising{M\madd\Mtwo}{\sigma}:\\
   \ \ \ \  [\ \ \  \satisfiesA{M}{\sigma}{({A[{\overline {\alpha/x}}]} \ \wedge\ \externalexec)}\  \wedge\ { \leadstoBounded {M\madd\Mtwo}  {\sigma}{\sigma'}} % \\    \ \ \ \ \ \   \ \ \ 
   \ \ \Longrightarrow\ \ 
  %  \ \ \ \ \ \  \ 
   {M},{\sigma'}\models{A[{\overline {\alpha/x}}]} \ \  ]
    \end{cases}
 $% \end{equation}
 %\end{itemize}
  \end{definition}
  
 
  \begin{lemma}[Encapsulation Soundness]
\label{lem:encap-soundness}
% A judgement of the form $\proves{M}{\encaps{A}}$  is\  \emph{sound}, \ if 
For all modules $M$, and assertions $A$: 

$\strut \hspace{1.5cm} \proves{M}{\encaps{A}} \ \ \ \ \Longrightarrow\ \ \ \ \satisfies{M}{\encaps{A}}$.
\end{lemma}


%%
%\section{ Specifications}
%\label{sect:spec}    
%\section{ \SpecLang Specifications}
\label{sect:spec}

We will now discuss the syntax and semantics of \SpecLang specifications, and illustrate through examples

\subsection{SpecLang syntax and semantics}

The syntax of  \SpecLang %specifications 
is given below:
 
\begin{definition}  

\noindent
{\emph{{Syntax of \SpecLang Specifications}}}

\label{f:holistic-syntax}
\[
\begin{syntax}
\syntaxElement{S}{}
		  {\syntaxline
                                 {\OneStateQ {\overline {x:C}} {A} }	
				{\TwoStatesQ {\overline {x:C}} {A} {A} }	
				{ A[\!\![ \,C\!::\!m(\overline{x:C})\,]\!\!] A }
				{S\, \wedge \, S}
		 \endsyntaxline
		}
\endSyntaxElement\\
\end{syntax}
\]
where for any specification ${\TwoStatesQ {\overline {x:C}} {A} {A} }$ we require that    {$fv(A), fv(A')\subseteq \overline x$}.
\end{definition}


%\subsubsection*{Arising States} % and {Arising} External States}
\footnoteSD{TODO motivate;
Here what we had: As discussed in \S \ref{s:approach}, 
{open world specifications need to be able to provide}
guarantees which hold
during execution of an internal, 
known, trusted module $M$ when linked together with any
unknown, untrusted, module $M_{ext}$. These guarantees need only hold 
when the external module is executing; we are not concerned if they are
temporarily broken by the internal module. Therefore, we are only interested in states where the
executing object (\prg{this}) is an external object. 
To express our focus on external states, we define the  \emph{external states semantics}, of the form 
$\reduction{M_{ext}}{M}{\sigma}{\sigma'}$, where $M_{ext}$ is the external
module, and $M$ is the internal module, and where we
collapse all internal steps into one single step.
}
{{The specifications $S$ describe expected properties between two different states. The first state must be a state that may \emph{arise} from the execution of the module, defined below. We require that execution started at an \emph{intitial} state. %A state $\sigma$ is \emph{arising},}  written $\arising{\sigma}{M}$, {if it  may arise}  % by observable states} 
%by execution
% starting at some initial configuration:


\begin{definition}[Arising  States]
\label{def:arising}
For modules $\overline M$ we define arising  states as follows:

\begin{itemize}
\item
 a state $\sigma$ is 
{ an \emph{arising} state, formally \ \ \  $\arising{\sigma}{\Mtwo}$,\ \ \ if  there exists some $\sigma_0$ such that $\initial{\sigma_0}$ and
$\Mtwo, {\sigma_0} \leadsto^* {\sigma}$.}
%\item
%{A a state $\sigma$ is 
%called an \emph{arising} state, formally\ \ \ \  $\extArising{\sigma}{M_{ext}}{M}$,\ \ \ \
%if and only if $\arising{\sigma}{M_{ext}*M}$ and $M, \sigma \models \external{\prg{this}}$.}
\end{itemize}
\end{definition}
  
We now move to the semantics of \SpecLang Specifications, and   define what it means for  a module  $M$ to satisfy specification  $S$, written as $M \vDash S$.  
 
\begin{definition}% [Semantics of \SpecLang Specifications]

We define $\satisfies{M}{{S}}$ by cases over the three 
possible syntactic forms.

\label{def:necessity-semantics}

\begin{enumerate}
\item
$
\satisfies{M}{ \OneStateQ {\overline {x:C}} {A} } \  \ \ \ \ \ \ \ \ \ \ \ \  \mbox{iff}  \     
    \begin{cases}
     \mbox{for all }   \Mtwo  \mbox{ and }  \sigma:\\
   \ \ \ \  [\ \ \   \arising{\sigma}{M\madd\Mtwo}  \ \  \wedge\ \ M, \sigma  \models {\external{\prg{this}}} \\
    \ \ \ \ \ \   \ \ \ \Longrightarrow\\
    \ \ \ \ \ \  \  \satisfiesA{M}{\sigma}{\forall \overline{x:C}.A}\ \  \  ] 
    \end{cases} 
 $
 \item
 $\satisfies{M}{\TwoStatesQ {\overline {x:C}} {A}{A'}}   \ \, \ \ \  \mbox{iff}  \   \
    \begin{cases}
     \mbox{for all }   \Mtwo  \mbox{ and }  \sigma  \mbox{ and }  \overline{\alpha}    \\
 \ \ \ \  [ \ \ { \arising{\sigma}{M\madd \Mtwo }\   \  \wedge\ \ \GRelevant {\overline \alpha}  \sigma}\ \ \ \wedge  \\
  \ \ \ \  \ \ \  \satisfiesA {M}   {\sigma}  {(\overline {\alpha :C} \ \wedge\  {A[\overline{\alpha/x}]}\ \wedge \  {\external{\prg{this}}}) } \ \ \ \wedge   \\
 \ \ \ \   \ \ \ {\leadstoBoundedStar {M\madd \Mtwo}{\sigma}  {\sigma'} } \ \ \wedge\ \  M, \sigma' \models {\external{\prg{this}}}  \\
 \ \ \ \   \ \ \ \ \  \ \Longrightarrow   \\
 \ \ \ \   \ \ \  {\satisfiesA{M}{\sigma'}{ A'[\overline{\alpha/x}]}  }  \ \ \  ]  
     \end{cases} 
 $
 \item
\sdN{
 $\satisfies{M}{ A[\!\![ \,C\!::\!m(\overline{x:C})\,]\!\!] A' }   \ \,  \mbox{iff}  \   \ 
    \begin{cases}
     \mbox{for all }   \Mtwo  \mbox{ and }  \sigma  \mbox{ and variables} y,\overline{y}    \\
   \ \ \ \  [ \ \ { \arising{\sigma}{M\madd \Mtwo} }\    \ \ \wedge \ \  \satisfiesA {M}   {\sigma}  {(y: C \ \wedge \ \overline {y :C} \ \wedge  \  {A}) } \ \ \ \wedge  \\
 \ \ \ \   \ \ \  \sigma.\prg{cont}=y.m(\overline y) \ \ \wedge\ \ \   {\leadstoBoundedStar {M\madd \Mtwo}{\sigma}  {\sigma'} }    \\
 \ \ \ \   \ \ \ \ \  \ \Longrightarrow   \\
 \ \ \ \   \ \ \  {\satisfiesA{M}{\sigma'}{ A'[ {\interpret{\sigma}{y}}/y,  \overline{{\interpret{\sigma}{y}}/y}] }}\ \ \  ]  
    \end{cases} 
    $
}
 \item
 $\satisfies{M}{S\, \wedge\, S'}$  \ \ \ \ \ \ \ \ \ \ \ \ \ \ \ \ \  \ \ \  iff  \  \ \  \   $\satisfies{M}{S}\ \wedge \ \satisfies{M}{S'}$
\end{enumerate}

\red{TODO: In (3) above it has to  be a terminating execution}

%\begin{tabular}{l l c l }
%
%$(1)$ & $\satisfiesA{M}{ \OneStateQ {\overline {x:C}} {A} }$ & iff & 
%for all $\Mtwo$, and $\sigma$  \\
%  & & & $[ \ \  \arising{\sigma}{M\madd\Mtwo}  \ \  \wedge\ \ M, \sigma  \models {\external{\prg{this}}} % \ \wedge 
%$\\
%& & & $\ \ \ \ \  \ \Longrightarrow \  $ \\ % \\ & & &  $ \satisfiesA{M}{\sigma[\overline{x\mapsto o}]}{A} $
%& & & \ \ \   {$ \satisfiesA{M}{\sigma}{\forall \overline{x:C}.A}\ \ ] $}
%\\
%\\
%$(2)$ & $\satisfies{M}{\TwoStatesQ {\overline {x:C}} {A}{A'}}$& iff & 
%for all $\Mtwo$, $\sigma$,   and $\overline{\alpha}$   \\
% & & & $[ \ \ { \arising{\sigma}{M\madd \Mtwo }\   \  \wedge\ \ \GRelevant {\overline \alpha}  \sigma}\ \ \ \wedge$ \\
% & & & $\ \ \  \satisfiesA {M}   {\sigma} {\red{(\overline {\alpha :C}}\ \wedge\  \red{A[\overline{\alpha/x}]}\ \wedge \  {\external{\prg{this}}}) } \ \ \ \wedge $ \\
% % \ \ \wedge \satisfiesA{M}  \sigma   {\external{\prg{this}}} \ \ \wedge $ \\
% %\ \    \satisfiesA{M}   {\sigma[\overline{x\mapsto \alpha}]}{(\overline {x:C} \wedge A)} \ \ \ \wedge}$ \\ 
% & & & $\ \ \ {\leadstoBoundedStar {M\madd \Mtwo}{\sigma}  {\sigma'} } \ \ \wedge\ \  M, \sigma' \models {\external{\prg{this}}}$ \\
%& & & $\ \ \ \ \  \ \Longrightarrow $ \\
%& & & $\ \ \  \satisfiesA{M}{\sigma'}{\red{A'[\overline{\alpha/x}]}} \ \ ] $
%\\
%\\
%$\bullet$ &  $\satisfies{M}{S\, \wedge\, S'}$ &   iff   & $\satisfies{M}{S}\ \wedge \ \satisfies{M}{S'}$
%\end{tabular} 

 
\end{definition} 


\footnoteSD{First bullet: This means that we require all objects to satisfy even if not locally relevant. Second Bullet: notice that we are asking for globally relevant objects}  
\footnoteSD{{TODO: Make an example that demonstrates the difference if in the second bullet we had asked for locally relevant objects ${\overline o}$.}}
\footnoteSD{{TODO Notice that we assume that $\overline x$ are not free in $A$ -- cf Barendregt convention.}}
\footnoteSD{TODO: explain why we did not require the stronger $\leadstoFin{M_{ext}\!\circ \!M}{\sigma}{\sigma'}$ rather than $\leadstoBoundedStar {M_{ext}\!\circ \!M}{\sigma}  {\sigma'}$.}
% Note that the requirements that $\extArising{\sigma}{M_{ext}}{M}$ and $\leadstoFin{M_{ext}\circ M}{\sigma}{\sigma'} $ imply that
% $M, \sigma' \models {\external{\prg{this}}}$



We demonstrate the meaning of ${\TwoStatesQ {\overline {x:C}} {A_0}{A_0}}$ in Fig. \ref{fig:illusrPreserve} where we refine the execution shown in Fig. \ref{fig:UpSemantics}, and take it that the pink states, \ie   ${\sigma_6}$-${\sigma_9}$ and $\sigma_{13}$-$\sigma_{17}$, and  $\sigma_{20}$, $\sigma_{21}$ are external, and the green states, \ie   ${\sigma_{10}}$,  ${\sigma_{11}}$,   ${\sigma_{12}}$,  ${\sigma_{18}}$, and  ${\sigma_{19}}$, are internal. 
 
 \begin{figure}[htb]
\begin{tabular}{|c|}
\hline  % \\
\resizebox{8cm}{!}{
\includegraphics[width=\linewidth]{diagrams/preserves.png}, 
} 
\\
\hline
% \\
\begin{tabular}{lcl}
$\leadstoBoundedStar {...} {\sigma_6}   {\sigma_9} $ & & thus, $A_0$ guaranteed to be preserved from $\sigma_6$ to $\sigma_9$.\\
$\leadstoBoundedStar {...} {\sigma_6}   {\sigma_{13}} $ & & thus, $A_0$ guaranteed to be preserved from $\sigma_6$ to $\sigma_{13}$.\\
$\leadstoBoundedStar {...} {\sigma_6}   {\sigma_{19}} $, \ \  but $..,\sigma_{19}\not \models {\external{\prg{this}}}$ & & thus, $A_0$ not guaranteed to be preserved from $\sigma_6$ to $\sigma_{19}$.\\
$\leadstoBoundedStar {...} {\sigma_6}  {\sigma_{20}} $  \ \   & & thus, $A_0$  guaranteed to be preserved from $\sigma_6$ to $\sigma_{20}$.\\
$\notLeadstoBoundedStar {...} {\sigma_8}  {\sigma_{20}} $  \ \   & & thus, $A_0$  not guaranteed to be preserved from $\sigma_8$ to $\sigma_{20}$.\\
%$ {\sigma_6} \leadsto^*  \sigma_9 $ preserves $A_0$ & &
%$ {\sigma_6} \leadsto^*  \sigma_{10} $ does not preserve $A_0$ \\
%$ {\sigma_8} \leadsto^*  \sigma_{14} $ preserves $A_0$ & &
%$ {\sigma_8} \leadsto^* \sigma_{17} $ does not preserve $A_0$\\
%$ {\sigma_7} \leadsto^*  \sigma_{16} $  preserve $A_0$ & &
%$ {\sigma_9} \leadsto^*  \sigma_{19} $ does not preserve $A_0$
%\\
\hline
\end{tabular}
\end{tabular}
   \caption{Illustrating  the meaning on ${\TwoStatesQ {\overline {x:C}} {A_0}{A_0}}$  -- refining Fig. \ref{fig:UpSemantics}.  }
   \label{fig:illusrPreserve} 
 \end{figure}
 
\subsection{\SpecLang Examples}
\noindent
\sdN{We now revisit the specifications given in Sect. \ref{s:bankSpecEx}, and also consider % $S_{2a}$, a variation of $S_2$ which guarantees that a non-null password does not change, and 
 $S_5$ which guarantees   that   non-null passwords do not change:
}

\begin{tabular}{lcll}
%$S_{2a}$   &     $\triangleq$   &   ${\TwoStatesQ {\prg{a}:\prg{Account}.\prg{p}:\prg{Password}}  {\prg{p}=\prg{a.password} \wedge \inside{\prg{p}}}{\inside{\prg{p}}} }$
% \\
$S_5$ & $\triangleq$   & ${\TwoStatesQ {\prg{a}:\prg{Account}.\prg{p}:\prg{Password}}  {\prg{null}\neq \prg{p}=\prg{a.password}} {\prg{p}=\prg{a.password}} }$
 \end{tabular}


\noindent
\sdN{
With the three  modules from Sect. \ref{s:bank}, we obtain:
}

%\begin{tabular}{lcll}
%$S_o$   &     $\triangleq$   & $\OneStateQ{\prg{a}:\prg{Account} } {\inside{\prg{a}}} $
% \end{tabular}
\begin{tabular}{lllllllll}
$\ModA  \models S_1$  &   $\ModA  \models S_2$ &  $\ModA \models S_3$ &   $\ModA \models S_4$    & $\ModA \models S_5$\\
 $\ModB \models S_1$  &   $\ModB \not\models S_2$   &  $\ModB  \models S_3$   &  $\ModB  \not\models S_4$   & $\ModB \not\models S_5$ \\
 $\ModC  \models S_1$    & $\ModC \models S_2$ &   $\ModC \models S_3$    &$\ModC \not\models S_4$   & $\ModC \not\models S_5$ 
\end{tabular}
 
\subsection{\SpecLang Implications}

\begin{definition}[Satisfaction of Assertions by a module] 
\label{def:assertion-inference-semantics}
We define satisfaction of an assertion $A$ by a  module $M$ as:
\begin{itemize}
\item
{
$M \vDash A$   \ \ \ iff \ \ \  $\forall \overline{M}. \forall \sigma
[\ \    \arising{\sigma}{M\madd\overline{M}}\   \  \wedge\ \  \satisfiesA {M}   {\sigma} {\external{\prg{this}}} 
\   \ \Longrightarrow \ \ \satisfiesA{M}{\sigma}{A}\ \ ]$
}\footnote{Not sure about the need for external and arising.}
\end{itemize}
\end{definition}

%TODO: Here we will say that assertions are classical, as proven in FASE

\begin{definition}[Stronger Specifications] 
\label{def:specification-implication-semantics}
Specification $S$ is stronger than another specification $S'$  in the context of a  module: 
 \begin{itemize}[itemsep=5pt]
\item 
$\stronger M  S  {S'}$   \ \ \ iff \ \ \  $M\models S$ implies $M \models S'$
\item
$\strongerEq M  S  {S'}$   \ \ \ iff \ \ \ $\stronger M  S  {S'}$  \ and \  $\stronger M   {S'} S$    
\end{itemize}
\end{definition}

%Some properties of $M \models \_  \subseteq \_ $ are given below:
%
%\begin{lemma}
%For assertions $A$, $A'$, variables $\overline y$, and $\overline x$, specifications $S$, $S'$, $S''$, and module $M$:
%\begin{itemize} [topsep=6pt,itemsep=5pt,parsep=0pt,partopsep=0pt]
%%\item
%% $\stronger M {\OneStateQ {\overline {x:C}}  {A}}  {\TwoStatesQ {\overline {x:C}} {A}{A}} $ 
%%    \item
%%  $\strongerEq  M  {\OneStateQ    {y:\prg{Object}}   {\forall \overline {x:C}[ A ] } } 
%%    {\OneStateQ {\overline {x:C}}  {A}} $.
%\item
%$\strongerEq M    {\TwoStatesQ {\overline {x:C}} {A}{A'}}    {\TwoStatesQ {\overline {y:C}} {A[y/x]}{A'[y/x]}}$
%\item
%$  M  \models    \overline {x:C} \wedge A_1'  \rightarrow A_1$ \ \ \  and \ \ \
%$  M  \models  \overline {x:C} \wedge A_2'  \rightarrow A_2$  \ \ \ \ 
%implies\\
% $\strut \hspace{5cm} \stronger M  {\TwoStatesQ {\overline {x:C}} {A_1}{A_2}}     {\TwoStatesQ {\overline {x:C}} {A_1'}{A_2'}}$
%
%\item
%$\stronger M  S {S''}$ and $\stronger M {S''} {S'}$\ \  \ implies\  \ \ $\stronger M S  {S'}$.
%
%\end{itemize}
%
%\end{lemma}




%
%%%% %\subsection{Expressiveness}
%\label{s:expressiveness}
%
%We discuss expressiveness of \Nec operators, by comparing 
%them with one another, with temporal operators, and with other examples from the literature.
%
%\paragraph{Relationship between Necessity Operators}
%The three \Nec \ operators
%are related by generality. 
%%An 
% \emph{Only If} ($\onlyIf{A_1}{A_2}{A}$) implies
%  \emph{Single-Step Only If} ($\onlyIfSingle{A_1}{A_2}{A}$), since if $A$ is 
%a necessary precondition for multiple steps, then it must be a necessary 
%precondition for a single step. 
% \emph{Only If} also implies 
%an \emph{Only Through}, where the intermediate state is the starting state
%of the execution.  There is no further relationship between 
%\emph{Single-Step Only If} and \emph{Only Through}.
%
%
%\paragraph{Relationship with Temporal Logic}
%Two of the three \Nec operators can be expressed in traditional
%  temporal logic: 
%  ${\onlyIf{A_1}{A_2}{A}}$
%can be expressed  %%put in to get better line breaks
% as 
% $A_1\ \wedge\ \Diamond A_2\ \longrightarrow\ A$, and
% $\onlyIfSingle{A_1}{A_2}{A}$
%can be expressed  %%put in to get better line breaks
% as $\ A_1\ \wedge\ \bigcirc A_2\ \longrightarrow\ A$
% (where $\Diamond$ denotes any future state,  and
% $\bigcirc$ denotes the next state).
% Critically, 
%$\onlyThrough{A_1}{A_2}{A}$ cannot be encoded in temporal logics
%  without ``nominals'' (explicit state references), because the state where $A$ 
% holds must be between the state where $A_1$ holds, and the state
% where $A_2$ holds; and this must be so on \emph{every} execution path
% from $A_1$ to  $A_2$ \cite{hybridLogic2021,nominal-seplogic2020}.
% TLA+, for example, cannot describe ``only through'' conditions
% \cite{tlabook}, but we have found ``only through'' conditions critical
% to our proofs. 
%
%
%% \subsection{More Examples expressed in \Nec}
%% do not say \Nec Specifications
%% because it is language that is expressive, not the specification
%
% \sdfootnote{  SD chopped the below -- some of it had moved to earlier, rest perhaps not that illuminating
%
%In this section we introduce some further specification examples, and use them to elucidate finer points
%in the semantics of \Nec. % We also  discuss which modules    satisfy  which specifications.
%
% \subsubsection{More examples of the Bank}
%Looking back at the examples from  \S\ref{s:bankSpecEx},   it holds that
%%\\
%%\strut $\hspace{.2in}$  \ModA$\vDash$ \SrobustA    $\hspace{.6in}$ \ModB$\vDash$ \SrobustA
%%  $\hspace{.6in}$ \ModC$\vDash$ \SrobustA
%%  \\
%%\strut   $\hspace{.2in}$  \ModA$\vDash$ \SrobustB    $\hspace{.6in}$ \ModB$\nvDash$ \SrobustB
%%  $\hspace{.6in}$ \ModC$\vDash$ \SrobustB
%  \\
%  $\begin{array}{llll}
%  \ \  \ \ \ \ \ & \ModA \vDash  \SrobustA    \ \ \ \ \ \ & \ModB \vDash \SrobustA \ \ \ \ \ \
%  &  \ModC \vDash \SrobustA
%  \\
% &  \ModA \vDash  \SrobustB    \ \ \ \ \ & \ModB \nvDash \SrobustB \ \ \ \ \ 
%  &  \ModC \vDash \SrobustB
%  \end{array}$
% 
%
% 
%Consider now another four \Nec specifications:
% 
%\begin{lstlisting}[language = Chainmail, mathescape=true, frame=lines]
%     $\text{\SRobustNextAcc}$   $\triangleq$  from a:Account $\wedge$ a.balance==bal  next a.balance < bal
%                        onlyIf $\exists$ o.[$\external{\texttt{o}}$ $\wedge$ $\access{\prg{o}}{\prg{a.pwd}}$]                                           
%
%     $\text{\SRobustNextCall}$  $\triangleq$  from a:Account $\wedge$ a.balance==bal  next a.balance < bal
%                        onlyIf $\exists$ o.[$\external{\texttt{o}}$ $\wedge$ $\calls{\prg{o}}{\prg{a}}{\prg{transfer}}{\prg{\_, \_, \_}}$]
%                       
%     $\text{\SRobustToCall}$   $\triangleq$  from a:Account $\wedge$ a.balance==bal to a.balance < bal
%                        onlyIf $\exists$ o.[$\external{\texttt{o}}$ $\wedge$ $\calls{\prg{o}}{\prg{a}}{\prg{transfer}}{\prg{\_, \_, \_}}$]  
%                                          
%     $\text{\SRobustThroughCall}$  $\triangleq$  from a:Account $\wedge$ a.balance==bal to a.balance < bal
%                       onlyThrough $\exists$ o.[$\external{\texttt{o}}$ $\wedge$ $\calls{\prg{o}}{\prg{a}}{\prg{transfer}}{\prg{\_, \_, \_}}$]
%
%\end{lstlisting}
%
%
%{The specification \SRobustNextAcc  states that
%the balance of an account decreases \emph{in one step}, only if an external object has access to the 
%password. It a weaker specification than \SrobustB, because it applies when the 
%decrease   takes place in \emph{one} step, rather than in \emph{a number} of steps.}
%Even though \ModB does not satisfy \SrobustB, it does satisfy \SRobustNextAcc:
% 
%  $\begin{array}{llll}
%  \   & \ModA \vDash \SRobustNextAcc  \   & \ModB \vDash \SRobustNextAcc \  
%  & \ModC \vDash \SRobustNextAcc \\
%  
%  \end{array}$
%
%\vspace{.07in} % SD thinks some space is needed here
%
%The specifications \SRobustNextCall and   \SRobustToCall are similar:
%they both say that a decrease of the balance can only happen if the current statement is a call to \prg{transfer}.  
%The former considers   a \emph{single} step, while the latter allows for \emph{any number} of steps. 
% \SrobustB is slightly different, because it  says that such a decrease is only possible if some \emph{intermediate}
% step called \prg{transfer}.
% All three   modules satisfy  \SRobustNextCall.  
%On the other hand, the code \prg{a1=new Account; a2.transfer}$(...)$ decrements the balance of \prg{a2} and
%does call \prg{transfer} but not as a first step; therefore, none of the modules satisfy 
%\SRobustToCall. That is:
%
%
% $\begin{array}{llll}
%  & \ModA \vDash \SRobustNextCall     & \ModB \vDash \SRobustNextCall   
%  & \ModC \vDash \SRobustNextCall
%  \\
%  & \ModA \nvDash \SRobustToCall     & \ModB \nvDash \SRobustToCall  
%  & \ModC \nvDash \SRobustToCall
%  \end{array}$
%  
%  \vspace{.07in} % SD thinks some space is needed here
%
% Finally, \SRobustThroughCall is a weaker requirement than \SRobustToCall, because it only asks
%  that the \prg{transfer} method is called in \emph{some intermediate} step. 
%  All modules satisfy it:
% 
% 
%   $\begin{array}{llll}
%  & \ModA \vDash\SRobustThroughCall     & \ModB \vDash \SRobustThroughCall  
%  & \ModC \vDash \SRobustThroughCall
%   \end{array}$
%
%}




\subsection{The DOM} %\sophiaPonder[renamed Wrapper to Proxy]{  
\label{ss:DOM}
%This is the motivating example in \cite{dd},
%dealing with a tree of DOM nodes: Access to a DOM node
%gives access to all its \prg{parent} and \prg{children} nodes, with the ability to
%modify the node's \prg{property} -- where  \prg{parent}, \prg{children} and \prg{property}
%are fields in class \prg{Node}. Since the top nodes of the tree
%usually contain privileged information, while the lower nodes contain
%less crucial third-party information, we must be able to limit 
% access given to third parties to only the lower part of the DOM tree. We do this through a \prg{Proxy} class, which has a field \prg{node} pointing to a \prg{Node}, and a field \prg{height}, which restricts the range of \prg{Node}s which may be modified through the use of the particular \prg{Proxy}. Namely, when you hold a \prg{Proxy}  you can modify the \prg{property} of all the descendants of the    \prg{height}-th ancestors of the \prg{node} of that particular \prg{Proxy}.  We say that
%\prg{pr} has \emph{modification-capabilities} on \prg{nd}, where \prg{pr} is
%a  \prg{Proxy} and \prg{nd} is a \prg{Node}, if the \prg{pr.height}-th  \prg{parent}
%of the node at \prg{pr.node} is an ancestor of \prg{nd}.
%%}
%
%
%The specification \prg{DOMSpec} states that the \prg{property} of a node can only change if
%some external object presently has 
%access to a node of the DOM tree, or to some \prg{Proxy} with modification-capabilties
%to the node that was modified.
\begin{lstlisting}[language = Chainmail, mathescape=true,xleftmargin=2em,frame=lines,framexleftmargin=1.5em]
DOMNecSpec $\triangleq$ from nd : Node $\wedge$ nd.property = p  to nd.property != p
          onlyIf $\exists$ o.[ $\external {\prg{o}}$ $\wedge$ 
                       $( \  \exists$ nd':Node.[ $\access{\prg{o}}{\prg{nd'}}$ ]  $\vee$ 
                         $\exists$ pr:Proxy,k:$\mathbb{N}$.[$\, \access{\prg{o}}{\prg{pr}}$ $\wedge$ 
                         nd.parent$^{\prg{k}}$=pr.node.parent$^{\prg{pr.height}}$ ] $\,$ ) $\,$ ]
\end{lstlisting}
\begin{lstlisting}[language = Chainmail, mathescape=true,xleftmargin=2em,frame=lines,framexleftmargin=1.5em]
DOMSpec1 $\triangleq$ $\forall$ nd : Node
				$\openBr$$\inside{\texttt{nd}}$}{$\inside{\texttt{nd}}$$\closeBr$
DOMSpec2 $\triangleq$ $\forall$ nd : Node, p : Object
				$\openBr$$\inside{\texttt{nd}}$ $\wedge$ nd.property = p $\wedge$ 
				 $\forall$ pr : Proxy, [nd.parent$^\texttt{k}$ =  pr.node.parent$^\texttt{pr.height}$ $\longrightarrow$ $\inside{\texttt{pr}}$]$\closeBr$
				$\openBr$nd.property = p$\closeBr$
\end{lstlisting}

\subsection{The DAO}
\label{ss:DAO}

\begin{lstlisting}[language = Chainmail, mathescape=true, frame=lines]
DAONecSpec1 $\triangleq$ from d : DAO $\wedge$ p : Object
            to d.balance(p) > d.ether
            onlyIf false
\end{lstlisting}
\begin{lstlisting}[language = Chainmail, mathescape=true, frame=lines]
DAONecSpec2 $\triangleq$ from d : DAO $\wedge$ p : Object
            next d.balance(p) = m
            onlyIf $\calls{\prg{p}}{\prg{d}}{\prg{repay}}{\prg{\_}}$ $\wedge$ m = 0 $\vee$ $\calls{\prg{p}}{\prg{d}}{\prg{join}}{\prg{m}}$ $\vee$ d.balance(p) = m
\end{lstlisting}

\begin{lstlisting}[language = Chainmail, mathescape=true, frame=lines]
DAOSpec1 $\triangleq$ $\forall$ d : DAO, p : Object.
            $\openBr$d.balance(p) > d.ether$\closeBr$
\end{lstlisting}

\subsection{Safe}
\label{ss:Safe}


\begin{lstlisting}[language = Chainmail, mathescape=true, frame=lines]
SafeNecSpec $\triangleq$ from s : Safe $\wedge$ s.treasure != null
            to s.treasure == null
            onlyIf $\neg$ inside(s.secret)
\end{lstlisting}
\begin{lstlisting}[language = Chainmail, mathescape=true, frame=lines]
SafeSpec $\triangleq$ $\forall$ s : Safe, t : Object
		$\openBr$s.treasure = t $\wedge$ $\inside{\texttt{s.secret}}$$\closeBr$
		$\openBr$s.treasure = t$\closeBr$
\end{lstlisting}


\subsection{Crowdsale}
\label{ss:Crowdsale}
\begin{lstlisting}[mathescape=true, language=chainmail]
(NecR0) $\triangleq$ e : Escrow $\wedge$ $\calls{\_}{\prg{e}}{\prg{claimRefund}}{\prg{p}}$
          next e.balance = nextBal onlyIf nextBal = e.balance - e.deposits(p)
(NecR1) $\triangleq$ e : Escrow $\wedge$ e.state $\neq$ e.SUCCESS $\longrightarrow$ sum(deposits) $\leq$ e.balance
(NecR2_1) $\triangleq$ e : Escrow $\wedge$ $\calls{\_}{\prg{e}}{\prg{withdraw}}{\prg{\_}}$
           to $\calls{\_}{\prg{e}}{\prg{claimRefund}}{\prg{\_}}$ onlyIf false
(NecR2_2) $\triangleq$ e : Escrow $\wedge$ $\calls{\_}{\prg{e}}{\prg{claimRefund}}{\prg{\_}}$
           to $\calls{\_}{\prg{e}}{\prg{withdraw}}{\prg{\_}}$ onlyIf false
(NecR3) $\triangleq$ c : Crowdsale $\wedge$ sum(deposits) $\geq$ c.escrow.goal
         to $\calls{\_}{\prg{c.escrow}}{\prg{claimRefund}}{\prg{\_}}$ onlyIf false
\end{lstlisting}


%\paragraph{More examples}
%%or {\sc{VerX}}. 
%% Nevertheless, 
%%we believe that
%%it  is powerful enough for the purpose of straightforwardly
%%expressing robustness requirements. 
%In order to investigate \Nec's expressiveness,  
%we used it for
%examples provided in the literature. 
%% In this section we considered the DOM, %  example, proposed by  \citeasnoun{dd}. 
%In \jm[]{the appendices 
%%of the full paper 
%\cite{necessityFull}} % Appendix \ref{s:expressiveness:appendix},
%we compare with examples proposed by  \citeasnoun{FASE}, and \citeasnoun{VerX}.
 


%
%\section{Hoare Logic} %Proving Open Calls and Adherence to \SpecLang Specifications}
%\label{sect:proofSystem}
% \section{Proving Adherence to \SpecLang Specifications}

\subsection{Semantics of  a Hoare Triples -- Soundness of Hoare Logic}

We will develop a  Hoare logic with judgments of the form  $M\ \vdash\  \{\, A \,  \}\ e\  \{\, A' \, \}$ which promise that execution of the expression $e$ in a configuration with satisfies $A$ will lead to a configuration that satisfied $A'$. 
%{
%\begin{definition}[Semantics of Hoare triples and quadruples]
%
%For modules $M$, and assertions $A$, $A'$ and $A''$ we define the semantics of Hoare-triples, 
% $M\ \models\  \{\, A \,  \}\ e\  \{\, A' \, \}$, and Hoare quadruples, $M\ \models\  \{\, A \,  \}\ e\  \{\, A' \, \}\, \parallel\, A''$, as follows:
%\begin{itemize}
%\item
%$M\ \models\  \{\, A \,  \}\ e\  \{\, A' \, \}\, \parallel\, A''$ \\
% iff \\
% for a all $M_{ext}$, for all $\sigma$ such that $\sigma \in Arising ...$ \\
%$M,\sigma \ \models \ A  \ \wedge\  
%\sigma.cont$=$e \ \wedge\  M\circ M_{ext}, \sigma \leadsto^* \sigma'$  
%\\
%$\Longrightarrow$ \\
%$M_{int},\sigma' \ \models \ A'' \ \wedge\  (\ \sigma'.cont$ is a value $\ \Longrightarrow\ $$M_{int},\sigma' \ \models \ A'\ ). $
%\end{itemize}
%\end{definition}
%}
%
%\subsection{Hoare Rules}
%\label{s:inference}
%
%Note that $M_{int}\ \models\  \{\, A \,  \}\ e\  \{\, A' \, \}\, \parallel\, A''$  does {\emph not} imply that $M_{int}\ \models\  \{\, A \,  \}\ e\  \{\, A' \, \}$

%In this Section we provide a proof system for constructing 
%proofs of the \SpecLang specifications defined in \S \ref{s:holistic-guarantees}.
%Such proofs consist of 
% three parts:  
%\begin{description} 
%\item[(Part 1)]
%Proving Assertion Encapsulation (\S \ref{s:encaps-proof})
%\item[(Part 2)]
%Proving that   method bodies adhere to specifications written in \AssertLang (\S \ref{s:classical-proof})
%\item[(Part 3)]
%Proving that modules adhere to \SpecLang specifications (\S \ref{s:module-proof})
%\end{description}
%
%Part 1 is, to a certain extent, orthogonal to the main aims of our work;
%in this paper we propose a simple approach based on the type system, while also acknowledging that 
%better solutions are possible.
%For Parts 2-3, we 
%TODO
% came up with the key ideas outlined in  \S \ref{s:approach}, which we
% develop in more detail in \S \ref{s:classical-proof}-\S \ref{s:emergent-proof}.}

\begin{axiom}
We assume a judgment of the form $M \vdash A$, which had the property that\\
\strut \hspace{5cm} $M \vdash A $ \ \ \ \ implies \ \ \ \ $M \vDash A$
\end{axiom}


\subsection {Assertion Encapsulation}
\label{s:encaps-proof}

{
{\SpecLang proofs  hinge on the fact that some assertions cannot be invalidated unless some 
} internal (and thus known)
computation took place. 
{We refer to this property as \emph{Assertion Encapsulation},}
}
formally $M\ \vDash  \encaps{A}$, which states that 
 assertion $A$ is encapsulated by module $M$.


\subsubsection{Semantics of Assertion Encapsulation}

{An assertion $A$  is  encapsulated by a module $M$ under condition $A'$,
if in all possible states which arise from execution of module $M$ with any other external module $M_{ext}$, and which satisfy $A'$, 
the validity of $A$} 
{ can only be changed via computations internal to that module} -- \emph{i.e.},  via a call to
a method from $M$, i.e.,
calls to objects defined in $M$ but accessible from the
outside.


\begin{definition}[Assertion Encapsulation]
\label{def:encapsulation}
An assertion $A$ is \emph{encapsulated} by module $M$ and assertion $A'$, written as
\begin{itemize}
 \item     $M\ \vDash \encaps{A}$
 \end{itemize}
if, for all external modules $M_{ext}$, and all states $\sigma$, $\sigma'$
such that $\arising{M_{ext}}{M}{\sigma}$, {and variables $\overline{x}$ and objects $\overline{o}$}:

\begin{itemize}
 \item
  $\satisfiesA{M}{\sigma}{A}$,  \ \ \ \   $\overline{x}=Free(A)$, \  \  \ \ $\overline{o}=\sigma(\overline{x})$,\ \ \ \ $\reduction{M_{ext}}{M}{\sigma}{\sigma'}$, \ \ \ ${M},{\sigma'[\overline{y}\mapsto{\overline{o}}]}\not\vDash{A}$
%\item $\overline{x}=Free(A)$, $\overline{o}=\sigma(\overline{x})$
%\item $\reduction{M_{ext}}{M}{\sigma}{\sigma'}$   
%\item ${M},{\sigma'[\overline{y}\mapsto{\overline{o}}]}\not\vDash{A}$
 \end{itemize}

implies

 \begin{itemize}
 \item   $\exists y,\,m,\, \overline{y} .[\ \sigma.\prg{cont}= y.m(\overline{y})\ \wedge\  \satisfiesA{M}{\sigma}{\internal{y}} \ ]
$
 \end{itemize}
\end{definition}

Revisiting the examples from \S~\ref{s:outline}, % we can see
both \ModB and \ModC encapsulate   the  equality of the \prg{balance} of an account to some value \prg{bal}: 
Namely, this equality can only be invalidated through calling  methods on internal objects.
 
{On the other hand, assume two further modules, $Mod_{ul}$ and $Mod_{pl}$: both these modules use ledgers to keep a map between accounts and their balances, which export functions that allow the update of this map. In $Mod_{ul}$ the ledger is \emph{not} protected, while in $Mod_{pl}$ the ledger \emph{is} protected. Then, in the former, the and the balance of an account is \emph{not} encapsulated, and in the latter it  \emph{is} encapsulated. } 
%
\\
\strut \hspace{1cm}
$\ModB\ \vDash\ \encaps{ \prg{a}:\prg{Account}\ \wedge \prg{a.balance}=\prg{bal}}$
\\
\strut \hspace{1cm}
$\ModC\ \vDash \encaps{ \prg{a}:\prg{Account}\ \wedge \prg{a.balance}=\prg{bal}}$
\\
\strut \hspace{1cm} {$Mod_{ul}\ \not\vDash \encaps{ \prg{a}:\prg{Account}\ \wedge \prg{a.balance}=\prg{bal}}$}
\\
\strut \hspace{1cm} {$Mod_{pl}\  \vDash \encaps{ \prg{a}:\prg{Account}\ \wedge \prg{a.balance}=\prg{bal}}$}

\noindent
{Note that in the above, the term \prg{a.balance} is a ghost field.}



The property that a variable is protected from another one is not encapsulated, but  the property that a variable is protected \emph{is} encapsulated, regardless of the module. Note also, that  encapsulation of an assertion does not imply encapsulation of its negation; 
 for example,  ${\inside{x}}$ is encapsulated (as per below), but $\neg  {\inside{x}}$ is not.


\begin{lemma}
For any module $M$, and variables $x$ and $y$:
\begin{enumerate} 
\item $M \vDash \encaps{\inside{x}}$
\item $M \not\vDash {\encaps{\neg\inside{x}}}$
\item $M \not\vDash {\encaps{\protectedFrom{x}{z}}}$
\item $M \vDash A \rightarrow A' \ \ \wedge\ \  M \vDash  \encaps{A}$ \ \ implies \ \ $M \vDash  \encaps{A'}$.
\end{enumerate}
\end{lemma}

\begin{proof} Sketches:

(1) because if $y$ is protected, and since the fields are all private ... the only way .. 
\\
(2) Take a state where $\neg\inside{x}$ and that there is only external object that has access to $x$, and that this object becomes no longer accessible -- eg through field override. That means that we now have  $\inside{x}$.
\\
(3) it is always possible that another external object $z'$ has unprotected access to $x$; if $z'$ has access to $z$, then it can give it access to $x$ without invoking am internal method. (3) as a counterexample, 
\\
(4) Use that $M \vDash A \rightarrow A'$ implies $M \vDash \neg A'  \rightarrow \neg  A$. BUT CAREFUL, TODO! with the potential of non-termination on assertions!!
\end{proof}
%The key consequence of soundness is that -- SD dropped; it is   not a consequence of soundness!

\subsubsection{Deriving  Assertion Encapsulation}

{In general},  code that does not contain 
calls to a {given} module is guaranteed not to invalidate any assertions encapsulated by that module.
 Assertion encapsulation has been used in proof systems to {address}   the  {frame} problem
 \cite{objInvars,encaps}. 

We  do not mandate how this property should be derived -- instead, we rely on a judgment 
$M\ \vdash  \encaps{A}$ provided by some external system. \footnote{This is simpler than the oopsla-33 setting}
Thus, \SpecLang is parametric over the derivation of the encapsulation
     judgment; in fact, several ways to do that are possible \cite{TAME2003,ownEncaps,objInvars}. For example,
 the appendices of
    \cite{necessityFull} present a 
	%Appendix~\ref{s:encap-proof} and
    %Figure~\ref{f:asrt-encap}  we present a 
    rudimentary system that is sufficient to support our example
    proof.  


%As we have already stated at the beginning of this section,
%encapsulation is a deep topic that is well studied in the literature, 
%and is not the focus of this paper. For now, we simply assume the existence 
%of a proof system for encapsulation as it is secondary to the central topic 
%of this paper. We need only assert that such an algorithmic proof system 
%must be sound (Definition \ref{lem:encap-soundness}).
%% \susan[I commented out what was there as I thought it was repetious]
%% {We are assuming the existence of a proof system for encapsulation and only need to assert that such an algorithmic proof system nust be sound.}
%% The construction of the algorithmic system is not central to our work,
%% because, as we shall see in later sections, our logic 
%% does not rely on the specifics of an encapsulation algorithm, only its soundness.

Like OOPSLA 22 Our logic does not {deal with, nor} rely on, the specifics of  how   encapsulation
{is derived}.
  % model, 
{Instead, it relies} on an encapsulation judgment and expects it to be sound:

\begin{definition}[Encapsulation Soundness]
\label{lem:encap-soundness}
A judgement of the form $\proves{M}{\encaps{A}}$  is\  \emph{sound}, \ if 
for all modules $M$, and assertions $A$:\\

$\strut \hspace{1.5cm} \proves{M}{\encaps{A}} \ \ \ \ $ implies $\ \ \ \ \satisfies{M}{\encaps{A}}$.
\end{definition}




\subsubsection{Types for Assertion Encapsulation}
\label{types}
TODO: something simple ere 

\subsection{Assertion Inference}




\subsection{Weaker/Stronger Specifications}

We assume   the existence of a function $HS$ which allows us to look up the holistic specification of a module. 
In Figure \ref{fig:si} we   define a judgment $\stronger M S  {S'}$ which expresses that the specification $S$ is stronger than the specification $S'$ under module $M$.  

\begin{figure}[hbt]
$
\begin{array}{c}
\begin{array}{lcl}
\inferrule [HS-1]
	{ \\
	}
	{\strongerI M {S_1 \wedge S_2}  {S_1}
	}
&  & 
\inferrule [HS-2]
	{ \\
	}
	{\strongerI M {S_1 \wedge S_2}  {S_2}
	}
\\
\\
\inferrule [HS-rename-1]
	{ \\\\
	\strongerI M S {\OneStateQ {\overline {x:C}} {A} }
	\\
	\overline {y} \mbox{ free in } A
	}
	{\strongerI M S {\OneStateQ {\overline {y:C}} {A[\overline y/\overline x]} }
	}
&  & 
\inferrule [HS-rename-2]
	{ \\\\
	\strongerI M S {\TwoStatesQ {\overline {x:C}} {A} {A'} }
	\\
	\overline {y} \mbox{ free in } A, A'
	}
	{\strongerI M S {\TwoStatesQ {\overline {y:C}} {A[\overline y/\overline x]} {A'[\overline y/\overline x]} }
	}

\\
\\
\inferrule [HS-3]
	{ \\ 
	M \vdash (\overline {x:C} \wedge A) \rightarrow A' }
	{\strongerI M  {\OneStateQ {\overline {x:C}} {A} }  {\OneStateQ {\overline {x:C}} {A'} } }
	&  &
\inferrule [HS-4]
	{ \\ }
	{\strongerI M {\OneStateQ {\overline {x:C}} {A} } {\TwoStatesQ {\overline {x:C}} {A} {A} }
	}
\end{array}
\\
\\	
\inferrule [HS-5]
	{ \\ 
	M \vdash ({\overline {x:C}} \wedge A_1) \rightarrow A_1' \ \hspace{.5cm} M \vdash ({\overline {x:C}} \wedge A_2') \rightarrow A_2 }
	{\strongerI M   {\TwoStatesQ {\overline {x:C}} {A_1'}{A_2'} }   {\TwoStatesQ {\overline {x:C}} {A_1}{A_2} }
	}		
\end{array}
$
\label{fig:si}
\caption{Specification Implication}
\end{figure}

\begin{lemma}
For all modules $M$, and specifications $S$ and $S'$, we have that\\
\strut \hspace{2cm} $\strongerI M  S  {S'}    \ \ \ \ \Longrightarrow\ \ \ \ \stronger M S {S'}'$
\end{lemma}

We now define what it means for a module $M$ to promise a specification $S$:

\begin{definition}
Given module $M$ snd specification $S$:

\strut \hspace{2cm} $\promises M S$ \ \ \ \  iff \ \ \ \  $\strongerI M {HS(M)} S$
\end{definition}

Notice, that $\promises M S$ is only based on the spec of $M$, and does not guarantee that indeed $M$ satisfies $S$.

%\subsection{Proving method bodies while using  \AssertLang specifications}
%\label{s:classical-proof}
% 
%We now develop a Hoare logic, which can prove assertions of the from \\
%\strut \hspace{1cm} $\hproves{M}{A}{\prg{s}}{A'}$.\\
%where \prg{s} is a statement in \Loo, and $A$ and $A'$ are assertions in \AssertLang.
%
%The challenges here are 1) that \AssertLang assertions support, on top of the classical features, also ??what-shall-we-call-them? protection features, and 2) we need to reason about calls to external modules.
%
%
%We assume that there exists some
%proof system  that   allows us to prove 
% specifications of the form  $\hproves{M}{A}{\prg{s}}{A'}$.
%{We further assume that such a proof system is sound, i.e. that 
%if xxx TODO 
%% if $\hproves{M}{\hoare{P}{\prg{res = x.m($\overline{z}$)}}{Q}}$, then 
%% for every program state $\sigma$ that satisfies $P$, the execution of the method call \prg{x.m($\overline{z}$)}
% % esults in a program state satisfying $Q$.}
% We then expand the proof rules as follows ....
 


\subsection{Reasoning about protection}
We expand that logic with rules about protection, as in Fig. \ref{f:protection}. Essentially, the only what that the "protection" of an object can decrease is if we call an eternal method, and pass it an internal object as argument. This is then covered by the rule in Fig. \ref{f:external:calls}.

\begin{figure}[hbt]
$
\begin{array}{c}
\inferrule[\sc{prot-1}]
	{ }
	{\hproves{M} 
						{\ \protectedFrom{x}{z}\ \wedge \ \internal y }
						{\ y.f=y'\ }
						{\ \protectedFrom{x}{z}\ }
	}
	\\\\

%\inferrule[\sc{prot-1}]
%	{ }
%	{\hproves{M} 
%						{\ \protectedFrom{x}{z}\ \wedge \ \internal v }
%						{\ v=v'\ }
%						{\ \protectedFrom{x}{z}\ }
%	}
%	\\\\

	\inferrule[\sc{prot-2}]
	{ }
	{\hproves{M} 
						{\ \protectedFrom{x}{z}\  \wedge\ \internal {y'} \ \wedge\  x \neq y'}
						{\ y.f=y'\ }
						{\ \protectedFrom{x}{z}\ }
	}
	\\\\

	\inferrule[\sc{prot-3}]
	{ }
	{\hproves{M} 
						{\ \protectedFrom{x}{z}\ \wedge\  \protectedFrom{x}{y'} }
						{\ y.f=y'\ }
						{\ \protectedFrom{x}{z}\ }
	}
	\\\\

%	\inferrule[\sc{prot-2}]
%	{ }
%	{\hproves{M} 
%						{\ \protectedFrom{x}{z}\ \wedge\ \external v\ \wedge\  z\neq v }
%						{\ v=v'\ }
%						{\ \protectedFrom{x}{z}\ }
%	}
%	\\\\


%	\inferrule[\sc{prot-4}]
%	{ }
%	{\hproves{M} 
%						{\ \protectedFrom{x}{v}\ \wedge\ \external z\  }
%						{\ z=v\ }
%						{\ \protectedFrom{x}{z}\ }
%	}
%	\\\\

	\inferrule[\sc{prot-4}]
	{ }
	{\hproves{M} 
						{\ \protectedFrom{x}{z}\ \wedge\ z \neq \this}
						{\ y =y'.f\ }
						{\ \protectedFrom{x}{z}\ }
	}
	\\\\

%\inferrule[\textsc{prot-5}]
%	{}
%	{\hproves{M}
%			{}
%	}
\end{array}
$
\caption{Protection Logic}
\label{f:protection}
\end{figure}

TO-DISCUSS 1)Should f it be $\in HS(M)$ or something more general. ie can it be implied from .. ? Premise of  \textsc{prot-3} may be too strong!
2) What about the universally quantified $\overline x$? 3)  Note that $y'$ and $y$ talk of a different variable.
 

Explanations: \textsc{xxxl} states that   yyy
  
 
\subsection{Reasoning about calls}

We now show how to reason about external calls

\begin{figure}[hbt]
\begin{mathpar}
% the below are speacil cases of the last one
%\inferrule[\sc{ExtCall-1}]
%{ 
%	      \\\\
%		M \vdash A\ \rightarrow \  ( \  \external \{ z, z', z''\}\ \wedge\ \internal x \ \wedge\  \ y:C \ \wedge \ 
%  x\neq y \
%  \wedge\ \protectedFrom {y} {\{z,z',z''\} }\ )
%	\\\\ 
%	\promises M   {\TwoStatesQ {\overline {y:C}} {\inside y}{\inside y}} 
%		}
%	{\hproves{M} 
%						{ A }
%						{ \ z.m(z',x)\  }
%						{ \protectedFrom {y} {\{z,z',z''\} } }
%	}\\ \\
%\inferrule[\sc{ExtCall-2}]
%	{ 
%	      \\\\
%		M \vdash A\ \rightarrow \  ( \  \external \{ z, z', z''\}\ \wedge\ \internal x \ \wedge\  \ y:C \ \wedge \   x\neq y.f \
%  \wedge\ \protectedFrom {y.f} {\{z,z',z''\} }\ )
%		\\\\ 
%	\promises M   {\TwoStatesQ {\overline {y:C}} {\inside y.f}{\inside y.f}} 
%		}
%	{\hproves{M} 
%						{ A }
%						{ \ z.m(z',x)\  }
%						{ \protectedFrom {y.f} {\{z,z',z''\} } }
%	}
%\ \\
\inferruleSD{[\sc{ExtCall}]}
	{ 
		 M \vdash A\ \rightarrow \ \red{ {\external{z}}  }
		  \\
   	\promises M   {\TwoStatesQ {\overline {x:C}} {A_1}{A_2}}
           \\
		M \vdash {\lift A  {\{z,\overline {u}\}}  {\overline y}} \  \rightarrow \ (\  \red{ {\overline {x:C}}\ {\wedge\ A_1}} \  )	 
		}
	{   \hproves{M} 
						{ \ A\  }
						{ \ z.m(\overline u)\  }
						{ \  \llower {A_2}{(z,\overline y)} \ \wedge\ { \preserve  A  {(z,\overline y)} M }  \ }	
}
\end{mathpar}
\caption{Internal and External Calls Logic}
\label{f:external:calls}
\end{figure}

$\begin{array}{lcll}
\lift {v=v'} {\overline  y}  & = & v=v' 
\\
\lift {x.f=v} {\overline y} & = & x.f=v 
\\
\lift{ \inside x}  {\overline  y}  & = &   \inside x 
\\
\lift{\protectedFrom x {\overline {u}} }  {\overline  y}  & = &   \inside x & \mbox{if } { x\not\in \overline{y}}
\\
   & = &   \prg{true} & \mbox{otherwise}

\\
\lift {A_1 \wedge A_2} {\overline   y}  & = & \lift { A_1} {\overline y}    \ \wedge \lift {A_2} {\overline y}  
\\
\lift {\neg A} {\overline z} {\overline y}  & = & \neg (\lift {A} {\overline y}  )& \mbox{if  $A$ is protection-free}
\\
  & = &  \prg{true} & \mbox{otherwise}
\\
\lift {\forall \overline{x:C}.[ A ]} {\overline   y}  & = & \forall \overline{x:C}.[ \lift A  {\overline y}  ] & 
\\
\lift {\exists \overline{x:C}.[ A ]} {\overline  y}  & = & \exists \overline{x:C}.[ \lift A  {\overline y}   ] & 
\end{array}
$

$\begin{array}{lcll}
\\
\\
\llower {v=v'} {\overline   y}   & = & v=v' 
\\
\llower {x.f=v} {\overline   y}  & = & x.f=v 
\\
\llower{ \inside x}  {\overline   y}  & = &   \prg{true}
\\
\llower{ \protectedFrom x {\overline {u}} }  {\overline   y}  & = &     \protectedFrom x {\overline {u}}  
\\
\llower {A_1 \wedge A_2} {\overline   y}  & = & \llower { A_1} {\overline   y} \ \wedge\ \llower {A_2}   {\overline   y} 
\\
\llower {\neg A}  {\overline   y}  & = & \neg (\llower  {A} {\overline   y} ) & \mbox{if  $A$ is protection-free}
\\
  & = &  \prg{true} & \mbox{otherwise}
\\
\llower {\forall \overline{x:C}.[ A ]} {\overline   y}   & = & \forall \overline{x:C}.[ \llower A   {\overline   y} ] & 
\\
\llower {\exists \overline{x:C}.[ A ]} {\overline   y}  & = & \exists \overline{x:C}.[ \llower A  {\overline   y} ] & 
\end{array}
$

{\small{
$\begin{array}{lcll}
\\
\\
\preserve {v=v'} {\overline  y} M  & = &  v=v' 
\\
\preserve {x.f=v} {\overline  y} M & = & x.f=v  & \mbox{if} \ \ \promises M   {\TwoStatesQ {\overline {x':C},x:D} {x.f = v \wedge A_1}{x.f=v}} 
\\ 
& & & \mbox{and} \ \ M \vdash {\lift A  {\overline   y}} \  \rightarrow \ (\, {\overline {x:D}} \wedge A_1\, )
\\
   & = &   \prg{true} & \mbox{otherwise}
   \\
\preserve  { \inside x}  {\overline   y}  M & = &   \inside x 
\\
\preserve  { \protectedFrom x {\overline {u}} }  {\overline z} {\overline y} M  & = &   { \protectedFrom x {\overline {u}} }  & \mbox{if} \ \ \promises M   {\TwoStatesQ {\overline {x':C},x:D} {x.f = v \wedge A_1}{x.f=v}} 
\\ 
& & & \mbox{and} \ \ M \vdash {\lift A  {\overline  y}} \  \rightarrow \ (\, {\overline {x:D}} \wedge A_1\, )
\\
   & = &   \prg{true} & \mbox{otherwise}

\\
\preserve  {A_1 \wedge A_2} {\overline  y}  M & = & \preserve { A_1}  {\overline y}  M \ \ \wedge \\
& &  \preserve { A_2}  {\overline   y}  M
\\
\preserve  {\neg A}{\overline  y}  M   & = & \neg (\preserve {A} {\overline   y}  M)& \mbox{if  $A$ is protection-free}
\\
  & = &  \prg{true} & \mbox{otherwise}
\\
\preserve {\forall \overline{x:C}.[ A ]} {\overline   y}  M   & = & \forall \overline{x:C}.[ \preserve A  {\overline  y}  M ] & 
\\
\preserve {\exists \overline{x:C}.[ A ]} {\overline  y}  M   & = & \exists \overline{x:C}.[ \preserve A {\overline   y}  M  ] & 
\end{array}
$
}}

\subsection{Proving \SpecLang Specifications}

\subsubsection{Deriving sub-specifications}


\label{s:module-proof}

\begin{figure}[thb]
%\footnotesize
$
\begin{array}{c}
\inferrule [Two-State]
	{
	\\\\
	M \vdash \encaps{\overline {x:C}\, \wedge \, A}
	\\\\
	\textit{for all}\ \  \textit{public methods  from } D,\ \textit{with}\ \prg{mBody}(m,D,M)=\overline{y:D}\{\  s \ \}\\\\
				% \strut \hspace{3cm}
				\ \  {\hproves{M}{ \overline{x:C}\ \wedge \ A\ \wedge \ \prg{this}:\prg{D} \wedge\ \overline{y:D}  } {\ s\ } {\ A\ }} \ \parallel \  A 
	}
	{
	M\ \vdash\ {\TwoStatesQ {\overline {x:C}} {A} {A} }
	}
\\\\
\inferrule [External-Safe]
	{
		A_{ois}=\overline{\OneStateQ{\overline {y:C}}{A''}}\ \mbox{all object invariants in } HS(M)\ \ \ \ \ \ \ \  A_{strng}=A_{ois}\wedge A' 
		\\\\
 		 \hproves{M} {A_{strng} } {\ s\ } {\ A\ } \
 		\\\\ 	
				\forall  s', z, m.[\ \ 
				 (\  s = s'; z.m(\_); \_\ \wedge \ 
				  \hproves{M} {A_{strng}\ } {\ s'\ } {\  \external{z}\  }  \ \ \
				  \Longrightarrow\ \ \ \hproves{M}{A_{strng} } {\ s'\ } {\ A\ } \ \ ]
	}
	{
	{\hproves{M}{ A'}   {\ s\ } {\ A\ } }\  \parallel \  A   
	}
\\\\ 
\inferrule [One-State]
	{
 	M \vdash \encaps{\overline {x:C}\, \wedge \, A}
 	\\\\
 	\forall \mbox{ public } D. [\ \hproves{M}{\forall \overline{x:C}.[A]} {\ y=\prg{new}\ D\ } {\ \forall \overline{x:C}.[A]\ }	\ ]
   \\\\
 	\textit{for all}\ \  \textit{public methods  from } D,\ \textit{ with } \prg{mBody}(m,D,M)=\overline{y:D}\{\  s \ \}\\\\
%				% \strut \hspace{3cm}
 			\ \  {\hproves{M}{ \forall \overline{x:C}.[A]\ \wedge \ \prg{this}:\prg{D} \wedge\ \overline{y:D}  } {\ s\ } {\ \forall \overline{x:C}.[A]\ }} \ \parallel \  {\forall \overline{x:C}.[A]} 
	}
	{
	M\ \vdash\ \OneStateQ{\overline {x:C}}{A}
	}
\\\\
\begin{array}{lcl}
\inferrule[Weaken]
{
M \vdash S \\ \strongerI M S {S'}
}
{
M \vdash S'
}
&\ \ \  &
\inferrule[Multi]
	{
	M\ \vdash\ S 
	\\
	M\ \vdash\ S' 
	}
	{
	M\ \vdash\ S \wedge S'
	}
\end{array}

\end{array}
$
\caption{Inferring that module satisfies its specification}
\label{f:module:invariats}
\end{figure}

TODO: Does the consequence rule require that the assertions are encapsulated? And if an assertion is encapsulated, is its consequence also encapsulated?

The rules also require that the variables in the quatifiers do not appear in the bodies, and are disjoint from the parameters.
TODO explain. Also, we only look at the methods exported from the module.  Also, we ned to add some dynamic type checking to the language, ie the method call crashes if actual params do not fir the formal types. OR we type them all as \prg{Object}.

TODO: shall we drop one-state invariants? Do not know how to prove them here. But they are in the spirit of capabilities literature.

%\subsection{Soundness of the \SpecLang Logic}
%
% 
%\label{s:soundness}
%
%We will now prove soundness of the  \SpecLang Logic. For this, we will first prove soundness of our extended Hoare logic.
%
%
%\begin{lemma}
%Assuming a sound \SpecO proof system, $\proves{M}{A}$, and  and
%a sound encapsulation inference system, $\proves{M}{\encaps{A}}$. Then:
%\begin{itemize}
%\item
%The inference system  $M\ \vdash\  \{\, A \,  \}\ e\  \{\, A' \, \}$  defined in the previous section is sound.
%\end{itemize}
%\end{lemma}
%
%\begin{proof}
%Take arbitrary modules  $M$, $M'$, expression $e$,  assertions $A$, $A'$ and $A''$ and assume
%\begin{enumerate}
%\item
% $M\ \vdash\  \{\, A \,  \}\ e\  \{\, A' \, \}$ 
% \item
% $M,\sigma \ \models \ A$
% \item
%$ \sigma.cont$=$e$ 
%\item
%$M\circ M', \sigma \leadsto^* \sigma' \ \ \wedge\ \ \sigma'.cont$ is a value
%\end{enumerate}
%We want to show that
%\begin{enumerate}
%\item
%$M,\sigma' \ \models \ A' $
%\end{enumerate}
%The proof proceeds by induction over a lexicographic ordering over the tuples $(M, A, e, A', \sigma, \sigma')$ This ordering is the tuple of ($m_{cl}$, $m_{ext}$), where 
%$m_{cl}$ is the length of the maximal sequence of proof steps in "classical Hoare logic, ie excluding a step {\sc{ExtCall}} involved in proving that  ie excluding a step {\sc{ExtCall}}, and the $m_{ext}$ is the number of external calls that occurred ... {TODO: this needs to be refined!}
%
%\end{proof}

For the proof of soundness we will use the following two lemmas that give guarantees about preservation of properties when pushing new frames onto the stack, and when popping frames from the stack. We use the notation $\sigma \bullet \phi$ to indicate that the frame $\phi$ has been pushed on top of the ...\footnote{{these lemmas could also appear earlier ... or later}}

 
\begin{lemma}
For any module $M$, assertion $A$, variables $\overline x$, and variables $\overline x$,  % values $\overline v$, 
and any continuation, $cont$, and any states $\sigma$ and $\sigma'$ where 
$\sigma=(\psi,h)$, and $\sigma'=(((\overline {z \mapsto \sigma(x)}), cont)\cdot \psi, h)$,  
we have

\begin{itemize}
\item
$M,\sigma\ \models\  A$  \ \ implies \ \  $M,\sigma' \ \models\  \lift {A} {\overline x}   $
\item
$M,\sigma'\ \models\  A$  \ \ implies \ \  $M,\sigma  \ \models\  \llower  {A}  {\overline   x} $ 
\end{itemize}
\end{lemma}
\begin{proof}
in the next section
% by induction on the derivation of $\proves{M}{S}$.
\end{proof}


 
% 
% 
% \section{Soundness} % of our proof system}
% \label{sect:sound:proofSystem}
%

We now outline some interesting aspects when proving soundness of the logic from \S \ref{sect:proofSystem}.

 
\paragraph{\Scoped Satisfaction} 
\label{s:scoped:valid}


Remember that an assertion which held at the end of a method execution, need not hold upon return from it -- \cf Ex. \ref{ex:pop:does:not:preserve}, and   \ref{ex:motivate:scopes}. To address this, we introduce \emph{scoped satisfaction}: % of assertions: 
 $ \satDAssertFrom M  \sigma k   A$   says that $\sigma$ satisfies $A$ from $k$ onwards, if it satisfies it in $k$-th frame,  and all the frames above it. 
% That is   $ \satDAssertFrom M  \sigma k   A\ \triengleq\   
\ie $\forall j. [\  k\leq j \leq \DepthSt \sigma \ \Rightarrow \ M, \RestictTo \sigma j \models A \ ]$.
%And define  and $k\leq n$ and $\forall j. [\  k\leq j \leq n \ \Rightarrow \ M, ((\phi_1\cdot ... \phi_j), \chi) \models A'\ ]$ where $A'$ is $A$ whose free variables have been substituted according to $\phi_n$ -- \cf Def. \ref{def:restrict}.
We also introduce \emph{scoped quadruples},  $\satisfiesD {M} {\quadruple  {A} }   {\sigma}   {A'} {A''}$, which promise for all $k\leq \DepthSt \sigma$,  if $\sigma$ satisfies $A$ from $k$ onwards, and executes its continuation to termination, then the final state will satisfy $A'$ from $k$ onwards, and that  all intermediate external states will satisfy $A''$ from $k$ onwards - \cf Def \ref{def:restrict}.
More in \A\ \ref{s:scoped:valid}.
% We define   $\satisfiesD {M} {\quadruple  {A} }   {stmt}   {A'} {A''} $ and  $\satisfiesD {M} {S}$ accordingly.
\Scoped satisfaction is stronger than shallow:   
 
\begin{lemma}[\Scoped   vs Shallow Satisfaction]
For all $M$, $A$, $A'$, $A''$, $\sigma$, $stmt$:  
\begin{itemize}
\item
 $\satisfiesD {M} {\quadruple  {A} }   {\sigma}   {A'} {A''}   \ \ \ \Longrightarrow \ \ \   \satisfies {M} {\quadruple  {A} }   {\sigma}   {A'} {A''}$

%\item
% $\satisfiesD {M} {\quadruple  {A} }   {stmt}   {A'} {A''}   \ \ \ \Longrightarrow \ \ \   \satisfies {M} {\quadruple  {A} }   {stmt}   {A'} {A''}$::q

%\item 
%$\satisfiesD {M} {S}  \ \ \ \Longrightarrow \ \ \ \satisfies {M} {S}$
\end{itemize}
\end{lemma}

  %%%%%%%%

\paragraph{Soundness of the Hoare Triples Logic}
\label{sect:prove:triples:sound}

We require the assertion logic,  $M\vdash A$, and  the    underlying Hoare logic,  $M\ \vdash_{ul}\  \triple A {stmt} {A'}$,   to be be  sound. %, \cf Axiom \ref{ax:ul:sound}.
We   prove  properties of protection, and 
 soundness of the inference system for triples $M \vdash  \triple A {stmt} {A'} $ -- \cf \A\ \ref{s:sound:app:triples}.

 

\begin{Theorem}
\label{l:triples:sound}
For module  $M$   such that  $\vdash M$, and for any assertions $A$,  $A'$, $A''$ and statement  $stmt$:
\begin{center}
$M\ \vdash\  \triple A {stmt} {A'}  \ \ \ \  \Longrightarrow  \ \ \ \ \satisfiesD {M} {\quadruple {A} {stmt} {A'} {A''}}$
\end{center}
\end{Theorem}
 

\paragraph{Summarised Execution}
\label{s:summaized}

%Soundness of   rule {\sc{Call\_Ext}} raises the challenge that 
Execution of an external call may consist of any number of external
transitions, interleaved with calls to public internal methods, which in
turn may make any number of further internal calls (public or private),  % chopped "whether" so as to be more succinct
and these, again may call external methods.
For the   proof of soundness,  internal and external transitions use different arguments.
 For  external transitions we consider small steps  and  argue in terms of  preservationon of  encapsulated properties,
while for internal calls, we use large steps, and appeal to the method's specification.
Therefore, we define  \emph{sumarized} executions, where  internal calls are collapsed into one. large step, \eg below:
  

\label{sect:termExecs}


% \vspace{.1cm}

\begin{tabular}{lll}
\begin{minipage}{.41\textwidth}
%Original execution %, \textbf{three}  internal \& {four} external calls. 
% 
 \resizebox{4.1cm}{!}{
\includegraphics[width=\linewidth]{diagrams/summaryA.png}
} 
 \end{minipage}
&  \begin{minipage}{.14\textwidth}
summarized\\
$\strut \ \ \ \ \ $ to 
\end{minipage}
 &
\begin{minipage}{.41\textwidth}
~ \\
~ \\
\resizebox{4.1cm}{!}
{
\includegraphics[width=\linewidth]{diagrams/summaryB.png}
} \end{minipage}
\end{tabular}
 
 


\vspace{.1cm}

% We now revisit external executions interleaved with public method calls:   
%In the appendix, we prove 
Lemma \ref{lemma:external_breakdown:term} % from  the  Appendix   
says that any terminating execution 
 starting in an external state  consists of a  sequence of  external states interleaved with terminating executions
  of public methods.
Lemma  \ref{lemma:external_exec_preserves_more} says that such an execution preserves an encapsulated assertion $A$  
provided that all these finalising internal executions  %(the public methods called at $\sigma_1$, ... $\sigma_n$) 
also preserve $A$.
% It is used to prove  soundness of the rule {\sc{ExtCall}}\footnoteSD{perhaps also {\sc{ExtCall\_WithSpec\_Strong}}}
% 
%In the appendix we prove lemmas \ref{lemma:external_breakdown:term} and \ref{lemma:external_exec_preserves_more} 
%which guarantee  that if $\sigma$ is external and $ \leadstoBoundedStarFin {\Mtwo}  {\sigma}  {\sigma'}$, 
%then there exist $\sigma_1$, ... $\sigma_n$, such that
% $\WithExtPub {\Mtwo\cdot M} {\sigma\bd}  {\sigma}  {\sigma'} {\sigma_1...\sigma_n}$.
% Conversely, if $A$ is encapsulated, 
% and $\WithExtPub {\Mtwo\cdot M} {\sigma\bd}  {\sigma}  {\sigma'} {\sigma_1...\sigma_n}$, and 
% the calls at  $\sigma_1$, ... $\sigma_n$ preserve $A$,  then $A$ is preserved from $\sigma$ to $\sigma'$.
  


  %%%%%%%%%%%%%%%%%%%%%%%%%%%%%%%%%%%%%%%%%%%%%%%%%%%

\paragraph{ Soundness of the Hoare Quadruples Logic}

Proving soundness of our quadruples in  some cases  requires  induction on the execution while in other cases  requires induction on the derivation of the quadruples.  We address this   through  a well-founded ordering that combines both, \cf 
\label{sect:prove:wellfounded}
\label{sect:prove:sound:quadruples}
  Def.  \ref{def:measure}  and  lemma \ref{lemma:normal:two}. 
  Finally, in \ref{s:app:proof:sketch;quadruples}, we prove soundness:
 

\begin{theorem}
\label{t:quadruple:sound}
\label{thm:soundness}
For module  $M$,   assertions $A$, $A'$, $A''$,   state  $\sigma$, and specification $S$:

\begin{enumerate}[(A)]
\item
 $:\strut \   \vdash M  \ \ \ \wedge \ \ \  M\ \vdash\  \quadruple {A} {stmt} {A'} {A''}  \ \ \ \ \ \ \ \Longrightarrow \ \ \ \ \ \  \ \ \  M\ \modelsD\  \quadruple {A} {stmt} {A'} {A''}$
 \item
  $:\strut \  \  \proves{M}{S}\ \ \ \ \ \ \Longrightarrow\ \ \ \ \ \  \ \ \ {M} \modelsD {S}$
 
\end{enumerate}

\end{theorem}

%The proofs make use of summarized executions, well-founded orderings, and various assertion preservation properties.  %
%\begin{theorem}[Soundness]
%\label{thm:soundness}
%%Assume an \SpecO proof system, $\proves{M}{A}$, 
%%an encapsulation inference system, $\proves{M}{\encaps{A}}$,
%%% Axiom xx, and 
%% and  that on top of these systems we built
%% the \SpecLang logic according to zzzz,  then, for    all modules $M$, and all \SpecLang specifications  $S$:
% For any module $M$, and specification $S$:
% 
% $$\proves{M}{S}\ \ \ \ \ \ \ \mbox{implies}\ \ \ \ \ \  \ \ \ {M} \modelsD {S}$$
%\end{theorem}
%
%We now prove soundness of the inference system $M \vdash  \quadruple A {stmt} {A'} {A''}$, using summarized executions from the earlier section, and the ordering $_ \ll \_$:   Proof outlines for these theorems can be found in \A \ref{s:app:proof:sketch;quadruples} and \ref{s:app:proof:sketch;overall}. 

%
%Theorem. \ref{thm:soundness} demonstrates 
% that the   \SpecLang logic is sound with respect to the semantics of \SpecLang specifications.
% The \SpecLang logic parametric wrt to the algorithms for proving validity of assertions
% $\proves{M}{A}$, and 
% assertion encapsulation ($\proves{M}{\encaps{A}}$), and is sound
% provided that these two proof systems are sound.

%
%  \section{Our Example Proven} % of our proof system}
% \label{sect:example:proof:short}
%Using our Hoare logic, we have developed a mechanised proof in Coq, that, indeed, $M_{good} \vdash S_2 \wedge S_3$.
This proof is \sdnew{part of the accompanying artifact.}
 
 Our proof models  \LangOO, the assertion language, the specification language, and the Hoare logic from \S \ref{s:hoare:first},  \S   \ref{s:hoare:second},  \S  \ref{sect:wf},  \S \aref{F}{\ref{app:hoare}} and Def. \ref{def:push}.
In keeping with   the start of  \S \ref{sect:proofSystem}, our proof assumes the existence of an underlying Hoare logic,  
and several, standard, properties of that underlying logic, the assertions logic (\eg equality of objects implies equality of field accesses) and of type systems
(\eg  fields of objects of different types cannot be aliases of one another).
All assumptions  are clearly indicated in the associated artifact.
 Appendix \aref{H}{\ref{s:app:example}}  %included in the auxiliary material, 
outlines   that proof. 

% Finally, we discuss why  $M_{good} \not\vdash S_2 $

%
%  \section{Conclusion: Summary, Related Work and Further Work}
% \label{sect:related}
%\label{sect:conclusion}
% \paragraph{Behavioural Specification Languages} 

Hatcliff et al.\ \cite{behavSurvey2012} provide an excellent survey of
contemporary specification approaches.  With a lineage back to Hoare
logic \cite{Hoare69}, Meyer's Design by Contract \cite{Meyer97} was the
first popular attempt to bring verification techniques to
object-oriented programs as a ``whole cloth'' language design in
Eiffel.  Several more recent specification languages are now making
their way into practical and educational use, including JML
\cite{Leavens-etal07}, Spec$\sharp$ \cite{BarLeiSch05}, Dafny
\cite{dafny} and Whiley \cite{whiley15}. Our approach builds upon
these fundamentals, particularly Leino \& Shulte's
%\kjx{and Naumann's} 
formulation of
two-state invariants \cite{usingHistory}, and Summers and
Drossopoulou's Considerate Reasoning \cite{Considerate}.
%
In general, these approaches assume a closed system, where modules
can be trusted to co{\"o}perate. In this paper we aim to
% illustrate the kinds of techniques required
work
in an open system where modules'
invariants must be protected irrespective of the behaviour of the rest
of the system.

%% \sd{\Chainmail assertions are} guarantees upheld throughout program execution. 
%% Other systems which give such ``permanent'' guarantees are  type systems, 
%% which ensure that well-formed programs  always produce well-formed runtime
%% configurations, or information flow control systems \cite{infoflow}, which ensure that values 
%% classified as high  will not be passed into contexts classified as low. 
%% Such  guarantees %made by types or information flow control
%%  are  practical to check, but   too coarse grained
%% for the purpose of fine-grained,  module-specific specifications. 


%% \Chainmail\ specifications can cross-cut the code they are
%% specifying; \sd{therefore,} they are related to
%% aspect-oriented specification
%% languages such as AspectJML \cite{AspectJML} and AspectLTL
%% \cite{AspectLTL}.
%% %
%% AspectJML is an aspect-oriented extension to JML;
%%  in much the same way that AspectJ is an aspect-oriented extension to
%% Java \cite{AspectJ}.  AspectJML offers AspectJ-style pointcuts 
%% that allow the definition of crosscutting specifications, such as 
%% shared pre- or post-conditions for a range of method calls. 
%% % SD removed the below, as I do not understand it.
%% % These crosscutting specifications can be checked dynamically along with
%% % traditional object-oriented JML assertions. In contrast, \Chainmail\
%% %specifications naturally cross-cut implementation and specification
%% %modules without any special notation, although, lacking wildcards,
%% %\Chainmail\ is not as flexible as AspectJML. 
%% % % SD removed the below, because I do not think it is important
%% %To our knowledge, the
%% %semantics of AspectJML have yet to be defined formally, although
%% %earlier work by Molderez and Janssens describes the formal core of a
%% %similar language \cite{DbCAspectJ}.

%% AspectLTL \cite{AspectLTL} is a specification language based on Linear
%% Temporal Logic (LTL). \sd{It} %AspectLTL 
%% adds cross-cutting aspects to more
%% traditional LTL module specifications: these aspects can further
%% constrain specifications in modules. In that sense, AspectLTL and
%% \Chainmail\ %both 
%% \sd{use} similar implicit join point models, rather than
%% importing AspectJ style explicit pointcuts as in AspectJML.
%% %% % SD removed the below, because I do not think it is important
%% %  AspectLTL
%% %has a formal definition, as does \Chainmail; unlike \Chainmail,
%% %AspectLTL has support for automated reasoning with an efficient
%% %synthesis algorithm.

%% % \paragraph{Concurrent Reasoning} Deny-Guarantee \cite{DenyGuarantee}
%% % distinguishes between assertions guaranteed by a thread, and actions
%% % denied to all other threads. Deny properties correspond to our
%% % requirements that certain properties be preserved by all code linked
%% % to the current module. Compared with our work, deny-guarantee assumes
%% % co{\"o}peration: composition is legal only if  threads adhere  to
%% % their deny properties. In our work, a module has to be robust  and
%% % ensure that these properties cannot be affected by  other code. 


%% %Finally, 
%% \sd{Our} work is also related to the causal obligations in Helm et
%% al.'s behavioural contracts \cite{helm90}. Causal obligations allow
%% programmers to specify e.g.\ that whenever one object receives a
%% message (such as a subject in the Observer pattern having its value
%% changed) that object must send particular messages off to other objects
%% (e.g.\ the subject must notify its observers). \Chainmail's control
%% %SD: not "control flow"
%%  operator % (`$\Calls{\_} {\_} {\_} {\_} $) 
%%  %allows  programmers to make
%%  \sd{supports}  similar specifications, (e.g. 
%%  ${\Calls{\_}  {\prg{setValue}} {\prg{s}} {\prg{v}}}  \rightarrow \Future{\Calls{\prg{s}}{\prg{notify}}{\prg{s.observer}}{\prg{v}}}$ --- when a subject receives a \prg{setValue} method,
%%   it must ``forward'' those messages to the observer.

\paragraph{Defensive Consistency}

%cute but wrong.
%To misparaphrase Tolstoy, secure systems are all alike;
%every insecure system is insecure in its own way
%\cite{WikipediaAnnaKareninaPrinciple}.

In an open world, we cannot rely on the kindness of strangers: rather
we have to ensure our code is correct regardless of whether it
interacts with friends or foes.
Attackers 
\textit{``only have to be lucky once''} while secure systems 
\textit{``have to be lucky always''} \cite{IRAThatcher}.
% SD 
Miller \cite{miller-esop2013,MillerPhD} defines the necessary approach
as \textbf{defensive consistency}: \textit{``An object is defensively
  consistent when it can defend its own invariants and provide correct
  service to its well behaved clients, despite arbitrary or malicious
  misbehaviour by its other clients.''}  Defensively consistent
modules are particularly hard to design, to write, to understand, and
to verify: but
% they have the great advantage that
they make it much
easier to make guarantees about systems composed of multiple components
\cite{Murray10dphil}.


\paragraph{Object Capabilities and Sandboxes.}
{{\em Capabilities} as a means to support the development of concurrent and distributed system  were developed in the 60's
by Dennis and Van Horn \cite{Dennis66}, and were adapted to the
programming languages setting in the 70's \cite{JamesMorris}. 
{\em Object capabilities} were first introduced~\cite{MillerPhD} in the early 2000s},
 and many recent % work attempts to manage
studies manage
to verify  safety or correctness of object capability programs.
Google's Caja \cite{Caja} applies   sandboxes, proxies, and wrappers
 to limit components'
access to \textit{ambient} authority.
% --- that is, capabilities that
%can be obtained from the wider environment, rather than being granted
%to a component explicitly.
Sandboxing has been validated
formally: Maffeis et al.\ \cite{mmt-oakland10} develop a model of
JavaScript, demonstrate that it obeys two principles of
object capability systems
%  (``connectivity begets connectivity'' and
%``no authority amplification''), and then % uses these principles to
and show  how untrusted applications can be prevented from interfering with
the rest of the system. 
Recent programming languages % and web systems
\cite{CapJavaHayesAPLAS17,CapNetSocc17Eide,DOCaT14} including Newspeak
\cite{newspeak17}, Dart \cite{dart15}, Grace \cite{grace,graceClasses}
and Wyvern \cite{wyverncapabilities} have adopted the object
capability model.

%% \paragraph{Verification of Dynamic Languages}
%% A few formal verification frameworks  address JavaScript's highly
%% dynamic, prototype-based semantics. Gardner et al.\ \cite{Gardner12}
%%  developed a formalisation of JavaScript based on separation logic
%% % that they have used
%% and verified   examples. Xiong and Qin et
%% al.\ \cite{XiongPhd,Qin11}  worked on similar lines.
%% % More substantially,
%% Swamy et al.\ \cite{JSDijkstraMonad}  recently
%% developed a mechanised verification technique for JavaScript based on
%% the Dijkstra Monad in the F* programming language.  Finally, Jang et
%% al.\ \cite{Quark} % have %  managed to provide
%% developed a machine-checked proof of
%% five important properties of a web browser --- again similar to our
%% % \prg{any\_code} 
%% invariants --- such as
%% % \textit{``no tab may interfere with
%% %  another tab''} and 
%% \textit{``cookies may not be shared across
%%   domains''} by writing the minimal kernel of the browser in Haskell.
  
%%   \paragraph{JavaScript analyses.}
%% More practically, 
%% Karim et al. apply static analysis on
%% Mozilla's JavaScript Jetpack extension framework \cite{adsafe}, including
%%  pointer analyses. % In a different direction,
%% Bhargavan et al.\ \cite{DefJS}
%% extend language-based sandboxing techniques to support defensive
%% components that can execute successfully  in otherwise untrusted
%% environments.   Politz et
%% al.\ \cite{ADsafety} use a JavaScript type checker to check
%% properties such as
%% % \textit{``widgets cannot obtain direct references
%%  % to DOM nodes''} and
%%  \textit{``multiple widgets on the same page
%%   cannot communicate.''}
%% % --- somewhat similar in spirit to our \textbf{Pol\_4}.
%% Lerner et al.\ extend this system to ensure browser
%% extensions observe \textit{``private mode''} browsing conventions,
%% such as that \textit{``no private browsing history retained''}
%% \cite{Lerner2013b}.  Dimoulas et al.\ \cite{DPCC14} generalise the
%% language and type checker based approach to enforce explicit policies,
%% % although the policies  are restricted to
%% that  describe  which components  may
%% access, or may influence the use of, particular capabilities.
%% Alternatively, Taly et al.\ \cite{secureJS}
%% model  JavaScript APIs in Datalog, and then
%% carry out a Datalog search for an ``attacker'' from the set of all
%% valid API calls. 



\paragraph{Verification of Object Capability Programs}
Murray made the first attempt to formalise defensive consistency and
correctness~\cite{Murray10dphil}.  Murray's model was rooted in
counterfactual causation~\cite{Lewis_73}: an object is defensively
consistent when the addition of untrustworthy clients cannot cause
well-behaved clients to be given incorrect service.  Murray formalised
defensive consistency very abstractly, over models of (concurrent)
object-capability systems in the process algebra CSP~\cite{Hoare:CSP},
without a specification language for describing effects, such as what
it means for an object to provide incorrect service.  Both Miller and
Murray's definitions are intensional, describing what it means for an
object to be defensively consistent.


Dro\-sso\-pou\-lou and Noble \cite{capeFTfJP,capeFTfJP14} have
analysed Miller's Mint and Purse example \cite{MillerPhD} 
% SD Chope details by
% expressing it in Joe-E 
% a Java subset without reflection and static
%fields, 
%and in Grace \cite{capeFTfJP14}, 
and discussed the six
capability policies 
% that characterise the correct behaviour of the
% program, 
as proposed in \cite{MillerPhD}.
%We argued that these policies require a novel
%approach to specification, and showed some first ideas on how to use
%temporal logic.
In %  an unpublished technical report
\cite{WAS-OOPSLA14-TR}, {they} % Drossopoulou and Noble
sketched a  specification language,  \sd{used}  it to  
specify the six policies from \cite{MillerPhD}, % however,
%{their} partial formalisation showed that % they allowed
\sd{showed} that several possible interpretations were possible, %.  They also 
\sd{and} uncovered
the need for another four further policies.
%  and formalised them as well, showing how different implementations of the underlying Mint and Purse
% systems coexist with different policies \cite{capeIFM14},
They also
  sketched how 
a trust-sensitive 
example (the escrow exchange) could be verified in an open world
\cite{swapsies}. 
% In contrast, our work focuses on the semantics of the  \Chainmail\ specification
% language and how it can be used to provide holistic specifications for
% robust programs.
\sd{Their work does not support the concepts of control, time, or space, as in \Chainmail,
but it offers a primitive expressing trust.}
 
Devriese et al.\ \cite{dd}  have deployed
   \sd{powerful} %rather more complex
  theoretical techniques to address similar problems:  % Devrise et al.\ 
  \sd{They} show how step-indexing, Kripke worlds, and representing objects
as state machines with public and private transitions can be used to
reason about % object-oriented programs in general.
\sd{object capabilities}.
Devriese have demonstrated solutions to a range of exemplar problems,
including the DOM wrapper (replicated in our
Section~\ref{sect:example:DOM}) and a mashup application.
% Although the formal techniques are much more sophisticated than we
%apply here, and consequently 
% not true can e.g.\ reason about recursion where we
%cannot, there are some similarities, e.g.\ with the 
\sd{Their} distinction
between public and private transitions %being related 
\sd{is similar} to the
distinction between internal and external objects.

More recently, Swasey et al.\ \cite{ddd}  designed OCPL, a logic
for object capability patterns, that supports specifications and
proofs for object-oriented systems in an open world.  
% The key idea here is to 
\sd{They} % say it simpler
draw on verification techniques for security and
information flow: separating internal implementations (``high values''
which must not be exposed to attacking code) from interface objects
(``low values'' which may be exposed).  OCPL supports defensive
consistency % (Swasey et al.\ use 
(\sd{they} use the term ``robust safety'' from the
security community \cite{Bengtson}) via a proof system that ensures
low values can never leak high values to external attackers. 
%\susan{How does this imply that high values can be exposed?}
%\james{typo fixed: it's LOW values that can be exposed}
This means that low values \textit{can} be exposed to external code,
and the behaviour of the system is described by considering attacks only
on low values.  %OCPL is a program logic, and Swasey
\sd{They} use that logic to
prove a number of object-capability patterns, including
sealer/unsealer pairs, the caretaker, and a general membrane.

Schaefer et al.\ \cite{schaeferCbC} have recently
% taken a similar approach to Swasey,
% adding support for
\sd{added}  support for information-flow security % in a setting 
\sd{using} refinement to ensure correctness (in this case confidentiality) by
construction. 
% Although designed to support
% confidentialty, it seems likely that the information-flow guarantees
% could also be used to ensure robustness.  
By enforcing encapsulation, \sd{all} % used to say both
these approaches share similarity with techniques such as
ownership types \cite{ownalias,NobPotVitECOOP98}, which also
protect internal implementation objects from accesses that cross
encapsulation boundaries.  Banerjee and Naumann demonstrated that by
ensuring confinement, ownership
systems can enforce representation independence (a property close to
``robust safety'') some time ago \cite{Banerjee:2005}.

 
\Chainmail\ differs from Swasey, Schaefer's, and Devriese's work in a number of ways:
% \citet{ddd} and \citet{schaeferCbC} 
\sd{They} are primarily concerned \sd{with} %about
mechanisms that ensure encapsulation (aka 
confinement) while we abstract away from any mechanism via the
$\External{}$ predicate. 
\sd{They use powerful mathematical techniques
% , such as Kripke worlds and step-indexing 
which  the users need  to understand in order to write their specifications,
while \Chainmail users only need  to understand  first order logic and 
the holistic operators presented in this paper.}
% While \Chainmail's $\Using{}{}$ is related to Banerjee
% and Naumann's region sets, the assertion languages here are mostly
% traditional (extensions of) Hoare logics --- Swasey et al.\ build on a
%concurrent separation logic. 
\sd{ Finally, none of these systems offer the kinds of
holistic assertions addressing control flow, change, or temporal
operations that are at the core of \Chainmail's approach.
}

Scilla \cite{scillaOOPSLA19} is a minimalistic typed functional
language for writing smart contracts that compiles to the Ethereum
bytecode. Scilla's semantic model is restricted, assuming actor based
communication and restricting recursion,  thus facilitating static
analysis of Scilla contracts and ensuring termination.
Scilla is able to demonstrate that a number of popular Ethereum
contracts avoid type errors, out-of-gas resource failures, and
preservation of virtual currency. 
Scilla's semantics are defined formally, but have not yet been represented in a
mechanised model.

%% \kjx{NPChecker \cite{NPcheckerOOPSLA19} analyses Ethereum smart
%% contracts to detect bugs related to nondeterministic
%% execution. NPChecker undertakes an information flow
%% analysis to detect potential read-write hazards
%% particularly reentrancy and ordering dependencies.
%% \textbf{We don't do concurrency. Do we need this one? I don't think so}
%% }


Finally, the recent VerX tool is able to verify a range of
specifications for solidity contracts automatically \cite{VerX}.
Similar to \Chainmail, VerX has a specification language based on
temporal logic.  VerX offers three temporal operators (always, once,
prev) but only within a past modality, while \Chainmail\ has two
temporal operators, both existential, but with both past and future
modalities.   VerX specifications can also include predicates th	at
model the current invocation on a contract (similar to \Chainmail's
``calls''), can access variables, and compute sums (only) over
collections. \Chainmail\ is strictly more expressive as a
specification language, including quantification over objects and sets
(so can compute arbitrary reductions on collections) and of course
specifications for permission (``access''), space (``in'') and
viewpoint (``external'') which have no analogues in VerX. 
Unlike \Chainmail, VerX includes a practical tool that has
been used to verify   a hundred properties across case studies of
twelve Solidity contracts.
%\textbf{(ideally also say something about proof status)}}

\jm{
\paragraph{Incorrectness Logic.} O'Hearn~\cite{IncorrectnessLogic} defined a Hoare
Logic for modelling program incorrectness. O'Hearn's Incorrectness Logic
is based on a Reverse Hoare Logic \cite{reverseHoare}, and empowers programmers to 
specify preconditions under which specific errors and program states may result. 
Incorrectness Logic provides a sound and compositional way to reason about 
the presence of bugs rather than the absence of bugs. 
As with Hoare logic, Incorrectness Logic provides a system
for reasoning about sufficent conditions for post-conditions to hold.
However, where Hoare logic specifies the shape of the result of execution 
of all program states that satisfy the precondition, Incorrectness Logic
specifies that all states that satisfy the postcondition are reachable
from those that satisfy the precondition. This suits the specification
of program errors, as it allows for the exclusion of false negatives.
In comparison, \Chainmail, as with Hoare Logic, is concerned with correctness
(as seen in the exemplars of Section \ref{sect:problemdriven}). 
Extending the comparison, \Chainmail differs from both Hoare Logic and Incorrectness,
in the ability to specify, not just sufficient conditions, but necessary conditions for 
reaching certain program states. Neither Incorrectness Logic, 
nor Hoare Logic allows for such specifications.
}

% SD chopped as did not like
%As with Swasey et al.\ this work does not provide a holistic
%assertion language like \Chainmail.
% SD Chopped, as it sounds as if their is not real code, which is debatable
% and what is an extensional framework? they would say that theirs is too.
%In contrast, \Chainmail\ is
%meant for describing and reasoning about real code, and we provide an
%expressive, extensional framework for evaluating defensive consistency
%in actual open systems.
%


%%%%%%%%%%%%%%%%%%%%%%%%%%%%%%%%%%%%%%%%%%%%%%%%%%%%%%%%%%%%
%%NOTES:
%% the other thing this section needs to do, particularly with Devrise, is to lay out precisely the way our work is more limited than theirs:
%% (Swasey, I'm more and more convinced, is just ownership-via-a-proof-system) 
%% we don't step-index, don't have logical relations, etc: what do we lose by NOT having those things
%% (or what do we gain by having those things...

%% The "deep" comparison with Swasey and with Devirese (and also
%% information flow control and temporal logics) needs to say why these
%% works are not as good (expressive? easy to understand?) as ours.
%% Currently the Related work just mentions them, but does not answer the
%% question as to why our work is important when theirs already has been
%% published.




%% *Difference between Spec Languages and Chainmail*  One way to tackle
%%  this would be to enumerate which elements of Chainmail appear at
%%  other works, which do not, and claim that Chainmail’s novelty is the
%%  good combination of these elements


%% Eg: reflection about contents of stack and heap (in classical Hoare
%% Logics), two state assertions (JML etc), invariants (Hoare and Meyer),
%% internal/external (Liskov?, Noble et al,modules in Neumann and also
%% O’Hearn), time (temporal logic, but they do not have the other stuff),
%% Control (none?), Space (in Sep. logic, and in effects, buyt the
%% meaning is different), Permissions (our earlier work, and less
%% flexible approaches such as owenrship types and perhaps also
%% oinformation flow control), Authority (effect systems and modifies
%% clauses, and perhaps also Bierman&Parkison abstract predicates, but
%% there it is tied to pre-post conditions.


%% Also, point out difference between our invariants and Hoare
%% triples. Subtle and needs thinking







%%%%%%%%%%%%%%%%%%%%%%%%%%%%%%%%%%%%%%%%%%%%%%%%%%%%%%%%%%%%

%% Neither effort addresses specification languages for security and
%% robustness, provides Hoare logics to reason about object-capability
%% programs.

%% , model protocols that dynamically ascribe trust
%% \cite{swapsies,lefthand} or quantify the damage an untrustworthy
%% object can do.






% \kjx{History-Based Specification and Verification of Scalable
%  Concurrent and Distributed Systems --- ICFEM15}


% \paragraph{Specifying Design Patterns}

% Techniques for specifying Design Patterns go back at least to 
% Helm's contracts \cite{Helm92}.

% more importantly: work on formalisation of design patterns.
% (again look at JC grant, even if refs are 5 years old)
% let's be shameless here...



% This search is similar to the quantification over
% potential code snippets in our model.
% The problem posed by the Escrow example is that it establishes a two-way
% dependency between trusted and untrusted systems --- precisely the
% kind of dependencies these techniques prevent.

% %These approaches are all based on static analyses.
%  The WebSand
% \cite{flowcaps11,sabelfeld-inlining2012} and Jeeves \cite{jeeves2012}
% projects use dynamic techniques to monitor safe execution of information flow policies.
%  Richards et al.\ \cite{FlacJS}   extended this approach by
% incorporating explicit dynamic ownership of objects (and thus of
% capabilities) and policies that may examine the history of objects'
% computations. While these dynamic techniques can restrict or terminate
% the execution of a component that breaches its security policies, they
% cannot guarantee in advance that such violations can never happen.
% While information flow policies are concerned with the flow of objects (and thus also capabilities)
% across the program code, our work is more concerned with the identification of the objects which protect
% the services.

%Compared with all these approaches, our work   focuses on
%\textit{general} techniques for specifying (and ultimately verifying)
%capability policies, whereas these systems are generally much more
%\textit{specific}: focusing on one (or a small number) of actual
%policies. % This seems to be because contemporary object capability
%programming is primarily carried out in JavaScript, but
% There are few

 
% \paragraph{Relational models of trust.}
% Artz and Gil \cite{artz-trust-survey-2007} survey various
% types of trust in computer science generally, although trust has also
% been studied in specific settings, 
% %
% ranging from peer-to-peer systems \cite{aberer-trust-p2p-2001} and
% cloud computing \cite{habib-trust-cloud-2011} 
% to mobile ad-hoc networks \cite{cho-trust-survey-adhocnets-2011}, 
% the internet of things \cite{lize-trust-IoT-2014}, 
% online dating \cite{norcie-trust-online-dating},
% and as a component of a wider socio-technical system
% \cite{cho-trust-sustainable-2013,walter-trust-cloud-ecis2013}. 
% %
% Considering trust (and risk) in systems design, Cahill et al.'s overview
% of the \textsc{Secure} project \cite{cahill-trust-pervasive-2003}
% gives a good introduction to both theoretical and practical issues of
% risk and trust, including a qualitative analysis of an e-purse
% example. This project builds on Carbone's trust model
% \cite{carbone-formal-trust-2003} which offers a core semantic model of
% trust based on intervals to capture both trust and uncertainty in that
% trust. Earlier Abdul-Rahman proposed using separate relations for
% trust and recommendation in distributed systems
% \cite{abdul-rahman-distributed-trust-1998}, more recently Huang and
% Nicol preset a first-order formalisation that makes the same
% distinction \cite{huang-formal-semantics-trust-calculus-2010}.
% Solhaug and St{\o}len \cite{solhaug-trust-uncertainty-2011} 
% consider how risk and trust are related to uncertainties over
% actual outcomes versus knowledge of outcomes.
% Compared with our work, these approaches produce models of trust
% relationships between high-level system components 
% (typically treating risk as uncertainty in trust) 
% but do not link those relations to the system's code. 



% \paragraph{Logical models of trust.}
% \sd{A detailed study of how web-users decide whether to trust appears in \cite{GilArtz}.}
% \sd{Starting with \cite{Lampson92},} various different logics have been used to measure trust in different
% kinds of systems.
% Murray and Lowe \cite{murray10-infoflow} model object capability
% programs in CSP, and use a model checker to ensure program executions
% do not leak authority.
% Carbone et al.\ \cite{carbone-formal-trust-2011}
% use linear temporal logic to model specific trust relationships in service
% oriented architectures.
% Ries et al.\ \cite{habib-trust-CertainLogic-2011} evaluate trust under
% uncertainty by evaluating Boolean expressions in terms of real values
% for average rating, certainty, and initial expectation.
% % Perhaps closer to our work, Aldini
% Aldini \cite{aldini-calculus-trust-IFIPTM2014} describes a temporal logic for
% trust that supports model checking to verify some trust properties.
% Primiero and Taddeo \cite{primiero-modal-theory-trust-2011} have
% developed a modal type theory that treats trust as a second-order
% relation over base relations between
% counterparties. Merro and Sibilio
% \cite{merro-calculus-trust-adhoc-facs2011} developed a trust model for
% a process calculus based on labelled transition systems.
% Compared with our proposal, these approaches use
% process calculi or other abstract logical models of systems, rather
% than engaging directly with the system's code.






%%%% %%%% %%%% %%%% %%%% %%%% %%%% %%%% %%%% %%%% %%%% %%%% %%%% %%%% 
%%%% %%%% %%%% %%%% %%%% %%%% %%%% %%%% %%%% %%%% %%%% %%%% %%%% %%%% 






%
%%\section{Conclusion}
%%\label{sect:conclusion}
%%%Susan:Please read first bit as I have just written it
%\se{When you write a module that is to be used with other code, the last thing you want to happen is that some other code uses it to cause effects that you never intended. Our specification language \Chainmail has been designed, so that developers whose modules are going to be used in the wild, have the language to constrain the usage of their code. In addition to classical function by function specification techniques, we have shown that a holistic or whole program approach is needed to make open world code robust. We have shown} 
% going to the old one, as running out of space.
% also, the new one brings new words, and I think all th words in concluson should have appeared earlier
In this paper we have motivated the need for holistic specifications,
presented the \Chainmail specification language for writing such
specifications, and shown 
how \Chainmail can be used to give holistic
specifications of key exemplar problems: the bank account,  the
wrapped DOM, the ERC20, and and the DAO.

To focus on the key attributes of a holistic specification language,
% we have tried to keep the
\sd{we have kept  \Chainmail simple, only requiring an understanding of first order logic.}
\sd{We believe that the holistic features (permission, control, time, space and viewpoint),
are intuitive concepts %for ptogrammers. 
when reasoning informally, and were pleased to have been able to provide their
formal semantics in what  we  argue is a simple manner.}
% below not true, we do have recusrions  
%do not even support recursive procedures to avoid circularities in the
%metatheory, let alone concurrency, exceptions, distribution,
%networking, etc. 

\sd{The development of the semantics of \Chainmail assertions posed several interesting 
challenges, \eg the treatment of the open world requires two-module execution
and the concept of external objects,
recursion is confined to ghostfields and assertions require termination of included expressions,
space required the concept of restricting runtime configurations,
and time required adaptation operators which apply bindings from one configuration to another.}  

\sd{\Chainmail is powerful enough to express many key examples from the
literature; nevertheless, it lacks several important features: It provides 
recursion  only in a restricted form, it has a rather inflexible notion of
module and does not support hierarchies of modules, and knows nothing about
concurrency or distribution.  We plan to remove these restrictions by applying
techniques such as step-indexing \cite{stepindex}, but hope to keep any mathematical 
sophitsication in the
model of \Chainmail without exposing it to the person who writes the specification.  We are also
 interested in extending \Chainmail\ to situations
where internal modules are typed, but the external modules are untyped.
%
We also plan to extend \Chainmail to support reasoning about
conditional trust in programs, and to quantify the risks involved in
interacting with untrustworthy software \cite{swapsies}.
}

\sd{To make these kinds of specifications
practically useful,  we plan to develop logics for proving adherence of module's code to holistic specs, as well
as logics for using holistic specs in the proof of open programs. We want to develop 
dynamic monitoring  and model checking techniques for our specifications. 
And finally, we plan to automate reasoning with these logics.}
  
%%
%
%
%
% %  \Input{furtherQuestions}
%%%  
The specification of  a \prg{Purse} appears in Fig. \ref{fig:PurseSpec}. This specification is at level 1, and therefore pre-conditions for not mention $\obeys x {Purse}$ but postconditions do.
The specification of  an  \prg{Escrow} appears in Fig. \ref{fig:EscrowSpec}. This specifications is at level 2, and therefore both pre- and post-conditions may mention  $\obeys x {Purse}$ .

\newcommand{\sMT}{sellerM} 
\newcommand{\bMT}{buyerM}
\newcommand{\sGT}{sellerG} 
\newcommand{\bGT}{buyerG}
\newcommand{\sM}{\prg{\sMT}}
\newcommand{\bM}{\prg{\bMT}}
\newcommand{\sG}{\prg{\sGT}}
\newcommand{\bG}{\prg{\sGT}}


\subsection{Purse}

The \prg{Purse} has a ghost field (or abstract predicate), \prg{SameMint(\_)}, which guarantees that the receiver and argument belong to the same \prg{Mint}.  It also has another ghost field, which is the balance of the account.

\textbf{To discuss} \begin{enumerate}
\item
implicit obeys for the receiver
\item
\prg{SameMint} definition
\item
\prg{SameMint} role
\end{enumerate}
%   fields sellerMoney, sellerGoods, buyerMoney, buyerGoods //  Purse
%   fields price, amt   // Number
% had to drop this one

\begin{figure*}[t]
\begin{lstlisting}[mathescape=true, language=Chainmail, frame=lines]
$\textbf{specification}$ $Purse$ {
    
    $\textbf{ghost}$ balance:int
    $\textbf{ghost}$ SameMint(x):bool  
         
    $\textbf{scoped-invr}\  \forall b:nat.[\ \inside{\prg{this}}  \ \wedge \prg{this}.\prg{balance}\geq b \ ]$
    $\textbf{invr}\   \forall p:\prg{Object}[\ \ \prg{this.SameMint(p)}\ \ ]$ // enough?
    $\textbf{scoped-invr}\   \forall p:\prg{Object}.,\forall b:bool.[\ \ \neg \prg{this.SameMint(p)}\ \ ]$ // enough?
    $\textbf{invr}\  \prg{this}.\prg{balance}\geq 0$       
    $\textbf{invr}\  \forall p:\prg{Object}.[\ \ \prg{this.SameMint(p)} \ \longrightarrow \  \obeys {\prg{p}} {Purse}\ \ ]$
     
    true  // implicit $\obeys {\prg{this}} {Purse}$
         $\textbf{\{}$ this.transfer(p,amt) $\textbf\}$ : bool
    res $\wedge$  this.SameMint(p) // implicit $\obeys {\prg{p}} {Purse}$
    $\vee$
    $\neg$res $\wedge$ [ this==p $\vee$ this.balance$\leq$amt $\vee$ $\neg(\, \prg{this.SameMint(p)}\,)$ ]

    this$\neq$p $\wedge$ this.balance=bm$\geq$amt  $\wedge$   this.SameMint(p)  
     // implicit $\obeys {\prg{this}} {Purse} \ \wedge\ \    \obeys {\prg{p}} {Purse}$
         $\textbf{\{}$ this.transfer(p,amt) $\textbf\}$ : bool
    res $\wedge$ this.balance=balT-amt $\wedge$ p.balance=balP+amt 

}

\end{lstlisting}
\caption{Specification of  $Purse$ }
\label{fig:PurseSpec}
 \end{figure*}
 
 NOTE: Julian proposed to replace lines 7 and 8 above with   \\
 $\ \ \  $ $\textbf{scoped-invr}\   \forall p:\prg{Object}.,\forall b:bool.[\ \ \neg \prg{this.SameMint(p)}\ \ ]$ 
But this would require us to ecplain that $b$ is a value and not a variable. HMHHHHH Is this also a problem in the OOPSLA-25 paper!!!

A possible implementation of \prg{Purse} appears below. The Purses have a field storing therr Mint and their balance

\begin{lstlisting}[mathescape=true, language=Chainmail, frame=lines]
$\textbf{class}$ $PurseImpl\_a\ \textbf{implements} Purse$ {
    
    $\textbf{field}$  myMint: Mint
    $\textbf{field}$  myBalance: int
     
    $\textbf{ghost}$ SameMint(x) $\textbf{is}$ this.myMint = x.myMint
    $\textbf{ghost}$ balance $\textbf{is}$ this.myBalance
    
    method transfer(p:Object,amt:nat)  // external
    	if p.myMint == this.myMint and this.blance>= amt 
	then
		this.balance -= amt
		p.balance += amt
		// will throw exception if p is not from class \prg{Purse}
		return true
	else
		return false	
  
}
\end{lstlisting}

A anotherimplementation of \prg{Purse} appears below. The Purses have a field storing their Mint, but their balances are stored in a 
lookup table in the Mint

\begin{lstlisting}[mathescape=true, language=Chainmail, frame=lines]
 $\textbf{class}$ Mint{

	field myPurses ... a list of Purse
	field myBalances .., a map from Purse to int
	
	method inMint(p:Purse) : bool    //  internal
	...	
	method getBalance{p:Purse): int
	...
	method setBalance(p:Purse,amt:int): void
	...	
}
 $\textbf{class}$Purse{
    
    $\textbf{field}$  myMint: Mint
     
    $\textbf{ghost}$ SameMint(x) $\textbf{is}$ this.myMint.inMint(x)
    $\textbf{ghost}$ balance $\textbf{is}$ this.myMint = x.myMint
    
    method transfer(p:Object,amt:nat) // external 
    	if myMint.inMint(p) and myMint.getBalance(this)>= amt 
	then
		myMint.setBalance(this,...)
		myMint.setBalance(p,...)
		return true
	else
		return false	
  
}
\end{lstlisting}



\begin{figure*}[t]
\begin{lstlisting}[mathescape=true, language=Chainmail, frame=lines]
$\textbf{specification}$ $Escrow$ {
    
    //   1$^{st}$ case:
    $\obeys  {\{\sM,\sG\}} {Purse}$ $\wedge$ 
    $\sM$.SameMint($\bM$) $\wedge$  $\sG$.SameMint($\bG$)  $\wedge$ price, amt:$\mathbb{N}$  $\wedge$
    $\bM$.balance=bm$\geq$price  $\wedge$  $\sM$.balance=sG$\geq$amt $\wedge$ ...
    $\obeys  {p} {Purse}$   $\wedge$ p.balance = pM
           $\textbf{\{}$ this.deal($\sM,\, \bM,\, \sG,\, \bG$, price, amt) $\textbf\}$
    res $\wedge$
    $\bM$.balance=bM-price $\wedge$ $\sM$.balance=....  $\wedge$
    p.balance=pM

     //   2$^{nd}$ case:
    $\obeys  {\sM } {Purse}$ $\wedge$ $\neg(\obeys  { \bM} {Purse})$ $\wedge$ 
    ....
    $\obeys  {p} {Purse}$ $\wedge$ $\protectedFrom {p} {\{ \bM, \sG, \bG \}}$  $\wedge$ p.balance = pM
          $\textbf{\{}$ this.deal($\sM,\, \bM,\, \sG,\, \bG$, price, amt) $\textbf\}$
    $\neg$ res $\wedge$
    .... $\wedge$
    p.balance=pM
     
}

\end{lstlisting}
\caption{Specification of  $Escrow$.\prg{::deal} -- Incomplete}
\label{fig:EscrowSpec}
 \end{figure*}
%
%%\section{Discussion}
%%
%% \paragraph{Behavioural Specification Languages} 

Hatcliff et al.\ \cite{behavSurvey2012} provide an excellent survey of
contemporary specification approaches.  With a lineage back to Hoare
logic \cite{Hoare69}, Meyer's Design by Contract \cite{Meyer97} was the
first popular attempt to bring verification techniques to
object-oriented programs as a ``whole cloth'' language design in
Eiffel.  Several more recent specification languages are now making
their way into practical and educational use, including JML
\cite{Leavens-etal07}, Spec$\sharp$ \cite{BarLeiSch05}, Dafny
\cite{dafny} and Whiley \cite{whiley15}. Our approach builds upon
these fundamentals, particularly Leino \& Shulte's
%\kjx{and Naumann's} 
formulation of
two-state invariants \cite{usingHistory}, and Summers and
Drossopoulou's Considerate Reasoning \cite{Considerate}.
%
In general, these approaches assume a closed system, where modules
can be trusted to co{\"o}perate. In this paper we aim to
% illustrate the kinds of techniques required
work
in an open system where modules'
invariants must be protected irrespective of the behaviour of the rest
of the system.

%% \sd{\Chainmail assertions are} guarantees upheld throughout program execution. 
%% Other systems which give such ``permanent'' guarantees are  type systems, 
%% which ensure that well-formed programs  always produce well-formed runtime
%% configurations, or information flow control systems \cite{infoflow}, which ensure that values 
%% classified as high  will not be passed into contexts classified as low. 
%% Such  guarantees %made by types or information flow control
%%  are  practical to check, but   too coarse grained
%% for the purpose of fine-grained,  module-specific specifications. 


%% \Chainmail\ specifications can cross-cut the code they are
%% specifying; \sd{therefore,} they are related to
%% aspect-oriented specification
%% languages such as AspectJML \cite{AspectJML} and AspectLTL
%% \cite{AspectLTL}.
%% %
%% AspectJML is an aspect-oriented extension to JML;
%%  in much the same way that AspectJ is an aspect-oriented extension to
%% Java \cite{AspectJ}.  AspectJML offers AspectJ-style pointcuts 
%% that allow the definition of crosscutting specifications, such as 
%% shared pre- or post-conditions for a range of method calls. 
%% % SD removed the below, as I do not understand it.
%% % These crosscutting specifications can be checked dynamically along with
%% % traditional object-oriented JML assertions. In contrast, \Chainmail\
%% %specifications naturally cross-cut implementation and specification
%% %modules without any special notation, although, lacking wildcards,
%% %\Chainmail\ is not as flexible as AspectJML. 
%% % % SD removed the below, because I do not think it is important
%% %To our knowledge, the
%% %semantics of AspectJML have yet to be defined formally, although
%% %earlier work by Molderez and Janssens describes the formal core of a
%% %similar language \cite{DbCAspectJ}.

%% AspectLTL \cite{AspectLTL} is a specification language based on Linear
%% Temporal Logic (LTL). \sd{It} %AspectLTL 
%% adds cross-cutting aspects to more
%% traditional LTL module specifications: these aspects can further
%% constrain specifications in modules. In that sense, AspectLTL and
%% \Chainmail\ %both 
%% \sd{use} similar implicit join point models, rather than
%% importing AspectJ style explicit pointcuts as in AspectJML.
%% %% % SD removed the below, because I do not think it is important
%% %  AspectLTL
%% %has a formal definition, as does \Chainmail; unlike \Chainmail,
%% %AspectLTL has support for automated reasoning with an efficient
%% %synthesis algorithm.

%% % \paragraph{Concurrent Reasoning} Deny-Guarantee \cite{DenyGuarantee}
%% % distinguishes between assertions guaranteed by a thread, and actions
%% % denied to all other threads. Deny properties correspond to our
%% % requirements that certain properties be preserved by all code linked
%% % to the current module. Compared with our work, deny-guarantee assumes
%% % co{\"o}peration: composition is legal only if  threads adhere  to
%% % their deny properties. In our work, a module has to be robust  and
%% % ensure that these properties cannot be affected by  other code. 


%% %Finally, 
%% \sd{Our} work is also related to the causal obligations in Helm et
%% al.'s behavioural contracts \cite{helm90}. Causal obligations allow
%% programmers to specify e.g.\ that whenever one object receives a
%% message (such as a subject in the Observer pattern having its value
%% changed) that object must send particular messages off to other objects
%% (e.g.\ the subject must notify its observers). \Chainmail's control
%% %SD: not "control flow"
%%  operator % (`$\Calls{\_} {\_} {\_} {\_} $) 
%%  %allows  programmers to make
%%  \sd{supports}  similar specifications, (e.g. 
%%  ${\Calls{\_}  {\prg{setValue}} {\prg{s}} {\prg{v}}}  \rightarrow \Future{\Calls{\prg{s}}{\prg{notify}}{\prg{s.observer}}{\prg{v}}}$ --- when a subject receives a \prg{setValue} method,
%%   it must ``forward'' those messages to the observer.

\paragraph{Defensive Consistency}

%cute but wrong.
%To misparaphrase Tolstoy, secure systems are all alike;
%every insecure system is insecure in its own way
%\cite{WikipediaAnnaKareninaPrinciple}.

In an open world, we cannot rely on the kindness of strangers: rather
we have to ensure our code is correct regardless of whether it
interacts with friends or foes.
Attackers 
\textit{``only have to be lucky once''} while secure systems 
\textit{``have to be lucky always''} \cite{IRAThatcher}.
% SD 
Miller \cite{miller-esop2013,MillerPhD} defines the necessary approach
as \textbf{defensive consistency}: \textit{``An object is defensively
  consistent when it can defend its own invariants and provide correct
  service to its well behaved clients, despite arbitrary or malicious
  misbehaviour by its other clients.''}  Defensively consistent
modules are particularly hard to design, to write, to understand, and
to verify: but
% they have the great advantage that
they make it much
easier to make guarantees about systems composed of multiple components
\cite{Murray10dphil}.


\paragraph{Object Capabilities and Sandboxes.}
{{\em Capabilities} as a means to support the development of concurrent and distributed system  were developed in the 60's
by Dennis and Van Horn \cite{Dennis66}, and were adapted to the
programming languages setting in the 70's \cite{JamesMorris}. 
{\em Object capabilities} were first introduced~\cite{MillerPhD} in the early 2000s},
 and many recent % work attempts to manage
studies manage
to verify  safety or correctness of object capability programs.
Google's Caja \cite{Caja} applies   sandboxes, proxies, and wrappers
 to limit components'
access to \textit{ambient} authority.
% --- that is, capabilities that
%can be obtained from the wider environment, rather than being granted
%to a component explicitly.
Sandboxing has been validated
formally: Maffeis et al.\ \cite{mmt-oakland10} develop a model of
JavaScript, demonstrate that it obeys two principles of
object capability systems
%  (``connectivity begets connectivity'' and
%``no authority amplification''), and then % uses these principles to
and show  how untrusted applications can be prevented from interfering with
the rest of the system. 
Recent programming languages % and web systems
\cite{CapJavaHayesAPLAS17,CapNetSocc17Eide,DOCaT14} including Newspeak
\cite{newspeak17}, Dart \cite{dart15}, Grace \cite{grace,graceClasses}
and Wyvern \cite{wyverncapabilities} have adopted the object
capability model.

%% \paragraph{Verification of Dynamic Languages}
%% A few formal verification frameworks  address JavaScript's highly
%% dynamic, prototype-based semantics. Gardner et al.\ \cite{Gardner12}
%%  developed a formalisation of JavaScript based on separation logic
%% % that they have used
%% and verified   examples. Xiong and Qin et
%% al.\ \cite{XiongPhd,Qin11}  worked on similar lines.
%% % More substantially,
%% Swamy et al.\ \cite{JSDijkstraMonad}  recently
%% developed a mechanised verification technique for JavaScript based on
%% the Dijkstra Monad in the F* programming language.  Finally, Jang et
%% al.\ \cite{Quark} % have %  managed to provide
%% developed a machine-checked proof of
%% five important properties of a web browser --- again similar to our
%% % \prg{any\_code} 
%% invariants --- such as
%% % \textit{``no tab may interfere with
%% %  another tab''} and 
%% \textit{``cookies may not be shared across
%%   domains''} by writing the minimal kernel of the browser in Haskell.
  
%%   \paragraph{JavaScript analyses.}
%% More practically, 
%% Karim et al. apply static analysis on
%% Mozilla's JavaScript Jetpack extension framework \cite{adsafe}, including
%%  pointer analyses. % In a different direction,
%% Bhargavan et al.\ \cite{DefJS}
%% extend language-based sandboxing techniques to support defensive
%% components that can execute successfully  in otherwise untrusted
%% environments.   Politz et
%% al.\ \cite{ADsafety} use a JavaScript type checker to check
%% properties such as
%% % \textit{``widgets cannot obtain direct references
%%  % to DOM nodes''} and
%%  \textit{``multiple widgets on the same page
%%   cannot communicate.''}
%% % --- somewhat similar in spirit to our \textbf{Pol\_4}.
%% Lerner et al.\ extend this system to ensure browser
%% extensions observe \textit{``private mode''} browsing conventions,
%% such as that \textit{``no private browsing history retained''}
%% \cite{Lerner2013b}.  Dimoulas et al.\ \cite{DPCC14} generalise the
%% language and type checker based approach to enforce explicit policies,
%% % although the policies  are restricted to
%% that  describe  which components  may
%% access, or may influence the use of, particular capabilities.
%% Alternatively, Taly et al.\ \cite{secureJS}
%% model  JavaScript APIs in Datalog, and then
%% carry out a Datalog search for an ``attacker'' from the set of all
%% valid API calls. 



\paragraph{Verification of Object Capability Programs}
Murray made the first attempt to formalise defensive consistency and
correctness~\cite{Murray10dphil}.  Murray's model was rooted in
counterfactual causation~\cite{Lewis_73}: an object is defensively
consistent when the addition of untrustworthy clients cannot cause
well-behaved clients to be given incorrect service.  Murray formalised
defensive consistency very abstractly, over models of (concurrent)
object-capability systems in the process algebra CSP~\cite{Hoare:CSP},
without a specification language for describing effects, such as what
it means for an object to provide incorrect service.  Both Miller and
Murray's definitions are intensional, describing what it means for an
object to be defensively consistent.


Dro\-sso\-pou\-lou and Noble \cite{capeFTfJP,capeFTfJP14} have
analysed Miller's Mint and Purse example \cite{MillerPhD} 
% SD Chope details by
% expressing it in Joe-E 
% a Java subset without reflection and static
%fields, 
%and in Grace \cite{capeFTfJP14}, 
and discussed the six
capability policies 
% that characterise the correct behaviour of the
% program, 
as proposed in \cite{MillerPhD}.
%We argued that these policies require a novel
%approach to specification, and showed some first ideas on how to use
%temporal logic.
In %  an unpublished technical report
\cite{WAS-OOPSLA14-TR}, {they} % Drossopoulou and Noble
sketched a  specification language,  \sd{used}  it to  
specify the six policies from \cite{MillerPhD}, % however,
%{their} partial formalisation showed that % they allowed
\sd{showed} that several possible interpretations were possible, %.  They also 
\sd{and} uncovered
the need for another four further policies.
%  and formalised them as well, showing how different implementations of the underlying Mint and Purse
% systems coexist with different policies \cite{capeIFM14},
They also
  sketched how 
a trust-sensitive 
example (the escrow exchange) could be verified in an open world
\cite{swapsies}. 
% In contrast, our work focuses on the semantics of the  \Chainmail\ specification
% language and how it can be used to provide holistic specifications for
% robust programs.
\sd{Their work does not support the concepts of control, time, or space, as in \Chainmail,
but it offers a primitive expressing trust.}
 
Devriese et al.\ \cite{dd}  have deployed
   \sd{powerful} %rather more complex
  theoretical techniques to address similar problems:  % Devrise et al.\ 
  \sd{They} show how step-indexing, Kripke worlds, and representing objects
as state machines with public and private transitions can be used to
reason about % object-oriented programs in general.
\sd{object capabilities}.
Devriese have demonstrated solutions to a range of exemplar problems,
including the DOM wrapper (replicated in our
Section~\ref{sect:example:DOM}) and a mashup application.
% Although the formal techniques are much more sophisticated than we
%apply here, and consequently 
% not true can e.g.\ reason about recursion where we
%cannot, there are some similarities, e.g.\ with the 
\sd{Their} distinction
between public and private transitions %being related 
\sd{is similar} to the
distinction between internal and external objects.

More recently, Swasey et al.\ \cite{ddd}  designed OCPL, a logic
for object capability patterns, that supports specifications and
proofs for object-oriented systems in an open world.  
% The key idea here is to 
\sd{They} % say it simpler
draw on verification techniques for security and
information flow: separating internal implementations (``high values''
which must not be exposed to attacking code) from interface objects
(``low values'' which may be exposed).  OCPL supports defensive
consistency % (Swasey et al.\ use 
(\sd{they} use the term ``robust safety'' from the
security community \cite{Bengtson}) via a proof system that ensures
low values can never leak high values to external attackers. 
%\susan{How does this imply that high values can be exposed?}
%\james{typo fixed: it's LOW values that can be exposed}
This means that low values \textit{can} be exposed to external code,
and the behaviour of the system is described by considering attacks only
on low values.  %OCPL is a program logic, and Swasey
\sd{They} use that logic to
prove a number of object-capability patterns, including
sealer/unsealer pairs, the caretaker, and a general membrane.

Schaefer et al.\ \cite{schaeferCbC} have recently
% taken a similar approach to Swasey,
% adding support for
\sd{added}  support for information-flow security % in a setting 
\sd{using} refinement to ensure correctness (in this case confidentiality) by
construction. 
% Although designed to support
% confidentialty, it seems likely that the information-flow guarantees
% could also be used to ensure robustness.  
By enforcing encapsulation, \sd{all} % used to say both
these approaches share similarity with techniques such as
ownership types \cite{ownalias,NobPotVitECOOP98}, which also
protect internal implementation objects from accesses that cross
encapsulation boundaries.  Banerjee and Naumann demonstrated that by
ensuring confinement, ownership
systems can enforce representation independence (a property close to
``robust safety'') some time ago \cite{Banerjee:2005}.

 
\Chainmail\ differs from Swasey, Schaefer's, and Devriese's work in a number of ways:
% \citet{ddd} and \citet{schaeferCbC} 
\sd{They} are primarily concerned \sd{with} %about
mechanisms that ensure encapsulation (aka 
confinement) while we abstract away from any mechanism via the
$\External{}$ predicate. 
\sd{They use powerful mathematical techniques
% , such as Kripke worlds and step-indexing 
which  the users need  to understand in order to write their specifications,
while \Chainmail users only need  to understand  first order logic and 
the holistic operators presented in this paper.}
% While \Chainmail's $\Using{}{}$ is related to Banerjee
% and Naumann's region sets, the assertion languages here are mostly
% traditional (extensions of) Hoare logics --- Swasey et al.\ build on a
%concurrent separation logic. 
\sd{ Finally, none of these systems offer the kinds of
holistic assertions addressing control flow, change, or temporal
operations that are at the core of \Chainmail's approach.
}

Scilla \cite{scillaOOPSLA19} is a minimalistic typed functional
language for writing smart contracts that compiles to the Ethereum
bytecode. Scilla's semantic model is restricted, assuming actor based
communication and restricting recursion,  thus facilitating static
analysis of Scilla contracts and ensuring termination.
Scilla is able to demonstrate that a number of popular Ethereum
contracts avoid type errors, out-of-gas resource failures, and
preservation of virtual currency. 
Scilla's semantics are defined formally, but have not yet been represented in a
mechanised model.

%% \kjx{NPChecker \cite{NPcheckerOOPSLA19} analyses Ethereum smart
%% contracts to detect bugs related to nondeterministic
%% execution. NPChecker undertakes an information flow
%% analysis to detect potential read-write hazards
%% particularly reentrancy and ordering dependencies.
%% \textbf{We don't do concurrency. Do we need this one? I don't think so}
%% }


Finally, the recent VerX tool is able to verify a range of
specifications for solidity contracts automatically \cite{VerX}.
Similar to \Chainmail, VerX has a specification language based on
temporal logic.  VerX offers three temporal operators (always, once,
prev) but only within a past modality, while \Chainmail\ has two
temporal operators, both existential, but with both past and future
modalities.   VerX specifications can also include predicates th	at
model the current invocation on a contract (similar to \Chainmail's
``calls''), can access variables, and compute sums (only) over
collections. \Chainmail\ is strictly more expressive as a
specification language, including quantification over objects and sets
(so can compute arbitrary reductions on collections) and of course
specifications for permission (``access''), space (``in'') and
viewpoint (``external'') which have no analogues in VerX. 
Unlike \Chainmail, VerX includes a practical tool that has
been used to verify   a hundred properties across case studies of
twelve Solidity contracts.
%\textbf{(ideally also say something about proof status)}}

\jm{
\paragraph{Incorrectness Logic.} O'Hearn~\cite{IncorrectnessLogic} defined a Hoare
Logic for modelling program incorrectness. O'Hearn's Incorrectness Logic
is based on a Reverse Hoare Logic \cite{reverseHoare}, and empowers programmers to 
specify preconditions under which specific errors and program states may result. 
Incorrectness Logic provides a sound and compositional way to reason about 
the presence of bugs rather than the absence of bugs. 
As with Hoare logic, Incorrectness Logic provides a system
for reasoning about sufficent conditions for post-conditions to hold.
However, where Hoare logic specifies the shape of the result of execution 
of all program states that satisfy the precondition, Incorrectness Logic
specifies that all states that satisfy the postcondition are reachable
from those that satisfy the precondition. This suits the specification
of program errors, as it allows for the exclusion of false negatives.
In comparison, \Chainmail, as with Hoare Logic, is concerned with correctness
(as seen in the exemplars of Section \ref{sect:problemdriven}). 
Extending the comparison, \Chainmail differs from both Hoare Logic and Incorrectness,
in the ability to specify, not just sufficient conditions, but necessary conditions for 
reaching certain program states. Neither Incorrectness Logic, 
nor Hoare Logic allows for such specifications.
}

% SD chopped as did not like
%As with Swasey et al.\ this work does not provide a holistic
%assertion language like \Chainmail.
% SD Chopped, as it sounds as if their is not real code, which is debatable
% and what is an extensional framework? they would say that theirs is too.
%In contrast, \Chainmail\ is
%meant for describing and reasoning about real code, and we provide an
%expressive, extensional framework for evaluating defensive consistency
%in actual open systems.
%


%%%%%%%%%%%%%%%%%%%%%%%%%%%%%%%%%%%%%%%%%%%%%%%%%%%%%%%%%%%%
%%NOTES:
%% the other thing this section needs to do, particularly with Devrise, is to lay out precisely the way our work is more limited than theirs:
%% (Swasey, I'm more and more convinced, is just ownership-via-a-proof-system) 
%% we don't step-index, don't have logical relations, etc: what do we lose by NOT having those things
%% (or what do we gain by having those things...

%% The "deep" comparison with Swasey and with Devirese (and also
%% information flow control and temporal logics) needs to say why these
%% works are not as good (expressive? easy to understand?) as ours.
%% Currently the Related work just mentions them, but does not answer the
%% question as to why our work is important when theirs already has been
%% published.




%% *Difference between Spec Languages and Chainmail*  One way to tackle
%%  this would be to enumerate which elements of Chainmail appear at
%%  other works, which do not, and claim that Chainmail’s novelty is the
%%  good combination of these elements


%% Eg: reflection about contents of stack and heap (in classical Hoare
%% Logics), two state assertions (JML etc), invariants (Hoare and Meyer),
%% internal/external (Liskov?, Noble et al,modules in Neumann and also
%% O’Hearn), time (temporal logic, but they do not have the other stuff),
%% Control (none?), Space (in Sep. logic, and in effects, buyt the
%% meaning is different), Permissions (our earlier work, and less
%% flexible approaches such as owenrship types and perhaps also
%% oinformation flow control), Authority (effect systems and modifies
%% clauses, and perhaps also Bierman&Parkison abstract predicates, but
%% there it is tied to pre-post conditions.


%% Also, point out difference between our invariants and Hoare
%% triples. Subtle and needs thinking







%%%%%%%%%%%%%%%%%%%%%%%%%%%%%%%%%%%%%%%%%%%%%%%%%%%%%%%%%%%%

%% Neither effort addresses specification languages for security and
%% robustness, provides Hoare logics to reason about object-capability
%% programs.

%% , model protocols that dynamically ascribe trust
%% \cite{swapsies,lefthand} or quantify the damage an untrustworthy
%% object can do.






% \kjx{History-Based Specification and Verification of Scalable
%  Concurrent and Distributed Systems --- ICFEM15}


% \paragraph{Specifying Design Patterns}

% Techniques for specifying Design Patterns go back at least to 
% Helm's contracts \cite{Helm92}.

% more importantly: work on formalisation of design patterns.
% (again look at JC grant, even if refs are 5 years old)
% let's be shameless here...



% This search is similar to the quantification over
% potential code snippets in our model.
% The problem posed by the Escrow example is that it establishes a two-way
% dependency between trusted and untrusted systems --- precisely the
% kind of dependencies these techniques prevent.

% %These approaches are all based on static analyses.
%  The WebSand
% \cite{flowcaps11,sabelfeld-inlining2012} and Jeeves \cite{jeeves2012}
% projects use dynamic techniques to monitor safe execution of information flow policies.
%  Richards et al.\ \cite{FlacJS}   extended this approach by
% incorporating explicit dynamic ownership of objects (and thus of
% capabilities) and policies that may examine the history of objects'
% computations. While these dynamic techniques can restrict or terminate
% the execution of a component that breaches its security policies, they
% cannot guarantee in advance that such violations can never happen.
% While information flow policies are concerned with the flow of objects (and thus also capabilities)
% across the program code, our work is more concerned with the identification of the objects which protect
% the services.

%Compared with all these approaches, our work   focuses on
%\textit{general} techniques for specifying (and ultimately verifying)
%capability policies, whereas these systems are generally much more
%\textit{specific}: focusing on one (or a small number) of actual
%policies. % This seems to be because contemporary object capability
%programming is primarily carried out in JavaScript, but
% There are few

 
% \paragraph{Relational models of trust.}
% Artz and Gil \cite{artz-trust-survey-2007} survey various
% types of trust in computer science generally, although trust has also
% been studied in specific settings, 
% %
% ranging from peer-to-peer systems \cite{aberer-trust-p2p-2001} and
% cloud computing \cite{habib-trust-cloud-2011} 
% to mobile ad-hoc networks \cite{cho-trust-survey-adhocnets-2011}, 
% the internet of things \cite{lize-trust-IoT-2014}, 
% online dating \cite{norcie-trust-online-dating},
% and as a component of a wider socio-technical system
% \cite{cho-trust-sustainable-2013,walter-trust-cloud-ecis2013}. 
% %
% Considering trust (and risk) in systems design, Cahill et al.'s overview
% of the \textsc{Secure} project \cite{cahill-trust-pervasive-2003}
% gives a good introduction to both theoretical and practical issues of
% risk and trust, including a qualitative analysis of an e-purse
% example. This project builds on Carbone's trust model
% \cite{carbone-formal-trust-2003} which offers a core semantic model of
% trust based on intervals to capture both trust and uncertainty in that
% trust. Earlier Abdul-Rahman proposed using separate relations for
% trust and recommendation in distributed systems
% \cite{abdul-rahman-distributed-trust-1998}, more recently Huang and
% Nicol preset a first-order formalisation that makes the same
% distinction \cite{huang-formal-semantics-trust-calculus-2010}.
% Solhaug and St{\o}len \cite{solhaug-trust-uncertainty-2011} 
% consider how risk and trust are related to uncertainties over
% actual outcomes versus knowledge of outcomes.
% Compared with our work, these approaches produce models of trust
% relationships between high-level system components 
% (typically treating risk as uncertainty in trust) 
% but do not link those relations to the system's code. 



% \paragraph{Logical models of trust.}
% \sd{A detailed study of how web-users decide whether to trust appears in \cite{GilArtz}.}
% \sd{Starting with \cite{Lampson92},} various different logics have been used to measure trust in different
% kinds of systems.
% Murray and Lowe \cite{murray10-infoflow} model object capability
% programs in CSP, and use a model checker to ensure program executions
% do not leak authority.
% Carbone et al.\ \cite{carbone-formal-trust-2011}
% use linear temporal logic to model specific trust relationships in service
% oriented architectures.
% Ries et al.\ \cite{habib-trust-CertainLogic-2011} evaluate trust under
% uncertainty by evaluating Boolean expressions in terms of real values
% for average rating, certainty, and initial expectation.
% % Perhaps closer to our work, Aldini
% Aldini \cite{aldini-calculus-trust-IFIPTM2014} describes a temporal logic for
% trust that supports model checking to verify some trust properties.
% Primiero and Taddeo \cite{primiero-modal-theory-trust-2011} have
% developed a modal type theory that treats trust as a second-order
% relation over base relations between
% counterparties. Merro and Sibilio
% \cite{merro-calculus-trust-adhoc-facs2011} developed a trust model for
% a process calculus based on labelled transition systems.
% Compared with our proposal, these approaches use
% process calculi or other abstract logical models of systems, rather
% than engaging directly with the system's code.






%%%% %%%% %%%% %%%% %%%% %%%% %%%% %%%% %%%% %%%% %%%% %%%% %%%% %%%% 
%%%% %%%% %%%% %%%% %%%% %%%% %%%% %%%% %%%% %%%% %%%% %%%% %%%% %%%% 






%%
%% %Susan:Please read first bit as I have just written it
%\se{When you write a module that is to be used with other code, the last thing you want to happen is that some other code uses it to cause effects that you never intended. Our specification language \Chainmail has been designed, so that developers whose modules are going to be used in the wild, have the language to constrain the usage of their code. In addition to classical function by function specification techniques, we have shown that a holistic or whole program approach is needed to make open world code robust. We have shown} 
% going to the old one, as running out of space.
% also, the new one brings new words, and I think all th words in concluson should have appeared earlier
In this paper we have motivated the need for holistic specifications,
presented the \Chainmail specification language for writing such
specifications, and shown 
how \Chainmail can be used to give holistic
specifications of key exemplar problems: the bank account,  the
wrapped DOM, the ERC20, and and the DAO.

To focus on the key attributes of a holistic specification language,
% we have tried to keep the
\sd{we have kept  \Chainmail simple, only requiring an understanding of first order logic.}
\sd{We believe that the holistic features (permission, control, time, space and viewpoint),
are intuitive concepts %for ptogrammers. 
when reasoning informally, and were pleased to have been able to provide their
formal semantics in what  we  argue is a simple manner.}
% below not true, we do have recusrions  
%do not even support recursive procedures to avoid circularities in the
%metatheory, let alone concurrency, exceptions, distribution,
%networking, etc. 

\sd{The development of the semantics of \Chainmail assertions posed several interesting 
challenges, \eg the treatment of the open world requires two-module execution
and the concept of external objects,
recursion is confined to ghostfields and assertions require termination of included expressions,
space required the concept of restricting runtime configurations,
and time required adaptation operators which apply bindings from one configuration to another.}  

\sd{\Chainmail is powerful enough to express many key examples from the
literature; nevertheless, it lacks several important features: It provides 
recursion  only in a restricted form, it has a rather inflexible notion of
module and does not support hierarchies of modules, and knows nothing about
concurrency or distribution.  We plan to remove these restrictions by applying
techniques such as step-indexing \cite{stepindex}, but hope to keep any mathematical 
sophitsication in the
model of \Chainmail without exposing it to the person who writes the specification.  We are also
 interested in extending \Chainmail\ to situations
where internal modules are typed, but the external modules are untyped.
%
We also plan to extend \Chainmail to support reasoning about
conditional trust in programs, and to quantify the risks involved in
interacting with untrustworthy software \cite{swapsies}.
}

\sd{To make these kinds of specifications
practically useful,  we plan to develop logics for proving adherence of module's code to holistic specs, as well
as logics for using holistic specs in the proof of open programs. We want to develop 
dynamic monitoring  and model checking techniques for our specifications. 
And finally, we plan to automate reasoning with these logics.}

%
%  
%
%
%%% Bibliography
%  \bibliography{Case,more,Response1}    
%
%
%\clearpage
%
%%\section{If we have space, somewhere}
%%
%% \subsection{Discussion of the semantics of assertions}
%% 
%% {Both existential and universal quantification (defined in \ref{quant1} and \ref{quant2}) is done over all objects which are transitively 
%%accessible any frame in the stack (as in OOPSLA). But note that $\satisfiesA{M}{\sigma} {\inside {y}}$ only considers objects that are locally reachable ..
%%
%%We do not include quantification over primitive types such as integers as \LangOO is too simple. The 
%%Coq mechanisation does include primitive types.}
%%\footnoteSD{TODO: Do we prove the implications as in TACAS, or just rely on TACAS? -- perhaps the former, since we have some new primitives? hmhh}
%%
%%\subsubsection{Alternative Definition of Protection}
%%
%%We could have given a weaker definition of protection, which would require that $o$ is protected from $o'$ if $o'$ can obtain direct access to $o$ only if some internal object executes a method. 
%%The projection of this weaker definition into the heap structure is as below
%%
%%
%%$\satisfiesA{M}{\sigma}{ {\alpha\ }\textbf{weakPrtFrom}\ {\alpha_o} }$  \ \ \ iff \\
%%\begin{itemize}
%%\item $\satisfiesA{M}{\sigma}{\internal{\alpha_0}}$, \\ or
%%\item 
%%$\forall n\in\mathbb{N}. \forall f_1,...f_n.$\\
%%$
%%[\ \ \interpret{\sigma}{\alpha_{o}.f_1...f_n}=\alpha \ \ \  \Longrightarrow \ \ \ \exists k<n. \satisfiesA{M} {\sigma} {\internal{{\interpret{\sigma}{\alpha_{o}.f_1...f_{k}}}}}\ \ ]$
%%\end{itemize}
%%
%%
%%With this, weaker, definition, in Fig. \ref{fig:ProtectedFrom}, object $6$ would be weakly-protected from $2$. Namely,  here 
%%for $2$  to get access to $6$ it suffices for $3$ to introduce $5$ to $2$, and does not require that some internal object introduces $6$ to some external object\footnoteSD{TODO? Hmhh This would have been clearer if there was also a field pointing from 3 to 5}.
%%!TEX encoding = UTF-8 Unicode
%%Note however that this alternative definition does not satisfy Lemma \ref{lemma:rel:abs:prot}. For a counterexample,   consider pushing onto the frame $\phi_1$ from Fig. \ref{fig:ProtectedFrom}, a frame $\phi_4$ where $\phi_4(\prg{this})$=$2$. Then, Lemma
%%\ref{lemma:rel:abs:prot} would promise that $6$ is protected from that frame, even though it is not.
%
%\appendix
%\subsection{Weaker/Stronger Specifications}

\begin{figure}[hbt]
$
\begin{array}{c}
\begin{array}{lclclcl}
\inferrule [HS-1]
	{ \\
	}
	{\strongerI M {S_1 \wedge S_2}  {S_1}
	}
&  & 
\inferrule [HS-2]
	{ \\
	}
	{\strongerI M {S_1 \wedge S_2}  {S_2}
	}
	&  & 
\inferrule [HS-3]
	{ \\
	}
	{\strongerI M {S_1 \wedge S_2}  {S_2 \wedge S_1}
	}
&  & 
\inferrule [HS-4]
	{ 
	{\strongerI M {S_1}  {S_3}}
	}
	{\strongerI M {S_1 \wedge S_2}  {S_3 \wedge S_2}
	}
\\
\end{array}
\\
\\	
\inferrule [HS-5]
	{ \\ 
	M \vdash ({\overline {x:C}} \wedge A_1) \rightarrow A_1' \ \hspace{.5cm} M \vdash ({\overline {x:C}} \wedge A_2') \rightarrow A_2 }
	{\strongerI M   {\TwoStatesQ {\overline {x:C}} {A_1'}{A_2'} }   {\TwoStatesQ {\overline {x:C}} {A_1}{A_2} }
	}
\\
\\	
% the below is weaker than the rule tht follows it
% \inferrule [HS-6]
%	{  }
%	{
%		{\strongerI M   { \TwoStatesQ {\overline {x:C}} {A_1}{A_2}\wedge  {\TwoStatesQ {\overline {x:C}} {A_1'}{A_2'}  } }   {\TwoStatesQ {\overline {x:C}} {A_1\wedge A_1'}{A_2\wedge A_2'} }  }
%	}
%\\
%\\	
\inferrule [HS-6]
	{   \forall i, j: x'_i \txtneq  x''_j }
	{
		{\strongerI M   { \TwoStatesQ {\overline {x:C},\overline{x':C'}} {A_1}{A_2}\wedge  {\TwoStatesQ {\overline {x:C,x'':C''}} {A_1'}{A_2'}  } }   {\TwoStatesQ {\overline {x:C,x':C',x'':C''}} {A_1\wedge A_1'}{A_2\wedge A_2'} }  }
	}
\\ \\
\inferrule [HS-7]
	{   M \vdash A_1 \rightarrow A_1' }
	{
		{\strongerI M   { \TwoStatesQ {\overline {x:C}} {A_1}{A_2 \wedge A_2'} }   {\TwoStatesQ {\overline {x:C}} {A_1}{A_2} \wedge \TwoStatesQ {\overline {x:C}} {A_1'}{A_2'} } }
	}	
	\\
	\\
\inferrule [HS-8]
	{   {\strongerI M {S_1} {S_2}} \ \ \ \   {\strongerI M {S_2} {S_1}} }
	{
		M \vdash S_1 \equiv S_2 
	}			
\end{array}
$
\label{fig:si}
\caption{Specification Implication}
\end{figure}
%%% Appendix
%%\appendix
\appendix
%
\section{\LangOO - full defintion}
\label{app:loo}


We introduce \LangOO, a simple, typed, class-based, object-oriented language.
To reduce the complexity of our formal models, \LangOO lacks many
common languages features, omitting static fields and methods, interfaces,
inheritance, subsumption, exceptions, and control flow.  
 \LangOO includes ghost fields,  that may only be used in the specification language.
The ghost fields may be defined recursively.
%
%\kjx{
%These features are
%well-understood: their presence (or absence) would not chanage the
%results we claim nor the structures of the proofs of those results.
%Similarly, while Loo is typed, we don't present or mechanise
%its type system. 
%Our results and proofs rely only upon type
%soundness --- in fact, we only need that an expression of
%type $T$ (where $T$ is a class $C$ declared in module $M$)
%will evaluate to an instance of some class from $M$,
%with the same confinement status as $C$.
%Featherweight Java extended with modules and assignment
%will more than suffice \cite{IgaPieWadTOPLAS01}.
%% well-understood that it is too boring to present here or to mechanise anew --- 
%%
%}


\subsection{Syntax}
The syntax of \LangOO is given in Fig. \ref{f:loo-syntax}.
\LangOO modules ($M$) map class names ($C$) to class definitions ($\textit{ClassDef}$).
A class definition consists of % \jm[]{an optional annotation \enclosed},
a list of field definitions, ghost field definitions, and method definitions.
{Fields, ghost fields, and methods all have types: {types are
    classes}. Ghost fields may be optionally 
annotated as \texttt{intrnl}, requiring the argument to have an internal type, and the 
body of the ghost field to only contain references to internal objects. This is enforced by the
limited type system of \LangOO.}
A program state ($\sigma$) is represented as a heap ($\chi$), stack ($\psi$) pair, 
where a heap is a map from addresses ($\alpha$) to objects ($o$), and a stack is a non-empty list of frames ($\phi$). A frame consists of a local variable
map and a continuation ($c$) that represents the statements that are yet to be executed ($s$).
% or a hole waiting to be filled by a method return in the frame above ($x := \bullet; s$).
A statement is either a field read ($x := y.f$), a field write ($x.f := y$), a method call
($u :=y_0.m(\overline{y})$), a constructor call ($\prg{new}\ C(\overline{x})$), a method return statement
($\prg{return}\ x$), or a sequence of statements ($s;\ s$).

\LangOO also includes syntax for expressions $e$ that may %only
be used in writing
specifications or the definition of ghost fields.


\begin{figure}[t]
\footnotesize
\[
\begin{syntax}
\syntaxID{x, y, z}{Variable}
\syntaxID{C, D}{Class Id.}
\syntaxID{f}{Field Id.}
\syntaxID{g}{Ghost Field Id.}
\syntaxID{m}{Method Id.}
\syntaxID{\alpha}{Address Id.}
\syntaxInSet{i}{\IntSet}{Integer}
\syntaxElement{v}{Value}
		{
		\syntaxline
				{\alpha}
				{i}
				{\true}
				{\false}
				{\nul}
		\endsyntaxline
		}
\endSyntaxElement\\
\\
\syntaxElement{Mdl}{Module Def.}
		{
		\syntaxline{\overline{C\ \mapsto\ CDef}}\endsyntaxline
		}
\endSyntaxElement\\
\syntaxElement{CDef}{Class Def.}
		{
		[An]\ \prg{class}\ C\ 
		\{\ \prg{cnstr}:= (\overline{x : T})\{s\};\ \prg{flds}:=\overline{fld};\ \prg{mths}:=\overline{mth};\ \prg{gflds}:=\overline{gfld};\  \}		
		}
\endSyntaxElement\\
\syntaxElement{An}{Class Annotation}
		{\enclosed}
\endSyntaxElement\\
\syntaxElement{T}{Type}
		{
		\syntaxline
%				{\_}
				{C}
		\endsyntaxline
		}
\endSyntaxElement\\
\syntaxElement{s}{Statement}
		{
		\syntaxline
				{\sdN{x:=y}}
				{\sdN{x:=v}}
				{x:=y.f}
				{x.f:=y}
				{x:=y_0.m(\overline{y})}
%		\endsyntaxline
%		}
%		{
%		\syntaxline
				{\new{C}{\overline{x}}}
				{\red{x}}
%				{\return{x}}
				{s;\ s}
		\endsyntaxline
		}
\endSyntaxElement\\
%\syntaxElement{c}{Continuation}
%		{
%		\syntaxline
%				{\sdN{s; \ x}}
%				{\sdN{x}}
%		\endsyntaxline
%		}
%\endSyntaxElement\\
\syntaxElement{mth}{Method Def.}
		{
		\prg{method}\ m\ (\overline{x : T})\sdN{:T}\{\ s\ \}
		}
\endSyntaxElement\\
\syntaxElement{fld}{Field Def.}
		{\syntaxline
			{\prg{field}\ f\ :\ T}
		\endsyntaxline}
\endSyntaxElement\\
\\
\syntaxElement{gfld}{Ghost Field Def.}
		{\syntaxline
			{\prg{ghost}\ g\ (\overline{x : T})\{\ gt\ \} : T}
			{\prg{ghost}\ \prg{intrnl}\ g\  (\overline{x : T})\{\ gt\ \} : T}
		\endsyntaxline}
\endSyntaxElement\\
\syntaxElement{\sdN{gt}}{\sdN{Ghost Term}}
		{
		\syntaxline
				{x}
				{v}
				{gt + gt}
				{gt = gt}
				{gt < gt}
	%	\endsyntaxline
	%	}
	%	{
	%	\syntaxline
				{\prg{if}\ gt\ \prg{then}\ gt\ \prg{else}\ gt}
				{gt.f}
				{t.g(gt)}
		\endsyntaxline
		}
\\
\endSyntaxElement\\
\syntaxElement{\sigma}{Program Config.}
		{(\prg{stack}:=\psi; \prg{heap}:=\chi)}
\endSyntaxElement\\
\syntaxElement{\psi}{Stack}
		{\syntaxline{\phi}{\phi \sdN{\cdot} \psi}\endsyntaxline}
\endSyntaxElement\\
\syntaxElement{\phi}{Frame}
		{\{\prg{local}:=\overline{x\mapsto v};\ \prg{contn}:=\sdN{s}\}}
\endSyntaxElement\\
\syntaxElement{\chi}{Heap}
		{\overline{\alpha \mapsto o}}
\endSyntaxElement\\
\syntaxElement{o}{Object}
		{\{\prg{class}:=C;\ \prg{flds}:=\overline{f \mapsto v} \}}
\endSyntaxElement\\
\end{syntax}
\]
\caption{\LangOO Syntax}
\label{f:loo-syntax}
\end{figure}

\footnoteSD{\red{JULIAN: Do we need the selectors? SD}}

\subsection{Semantics}
\LangOO is a simple object oriented language, and the operational semantics 
(given in Fig. \ref{f:loo-semantics} and discussed later)
do not introduce any novel or surprising features. The operational 
semantics make use of several helper definitions that we 
define here.

We provide a definition of reference interpretation in Definition \ref{def:interpret}
\begin{definition}
\label{def:interpret}
For a program state $\sigma = (\chi, \phi : \psi)$, we provide the following function definitions:
\begin{itemize}
\item
$\interpret{\sigma}{x}\ \triangleq\ \phi.(\prg{local})(x)$
\item
$\interpret{\sigma}{\alpha.f}\ \triangleq\ \chi(\alpha).(\prg{flds})(f)$
\item
$\interpret{\sigma}{x.f}\ \triangleq\ \interpret{\sigma}{\alpha.f}$ where $\interpret{\sigma}{x}=\alpha$
\end{itemize}
\end{definition}
That is, a variable $x$, or a field access on a variable $x.f$ 
has an interpretation within a program state of value $v$
if $x$ maps to $v$ in the local variable map, or the field
$f$ of the object identified by $x$ points to $v$.

Definition \ref{def:class-lookup} defines the class lookup function an object 
identified by variable $x$.
\begin{definition}[Class Lookup]
\label{def:class-lookup}
For program state $\sigma = (\chi, \phi \cdot\psi)$, class lookup is defined as 
$$\class{\sigma}{x}\ \triangleq\ \chi(\interpret{\sigma}{x}).(\prg{class})$$
\end{definition}

Definition \ref{def:meth-lookup} defines the method lookup function for a method
call $m$ on an object of class $C$.
\begin{definition}[Method Lookup]
\label{def:meth-lookup}
For module $\Mtwo$, class $C$, and method name $m$, method lookup is defined as 
$$\meth{\Mtwo}{C}{m}\ \triangleq\ M(C).\prg{mths}(m)\ \ \mbox{for some} M\in\Mtwo$$
\end{definition}

We borrow the definition  of module linking, given in Definition \ref{def:linking}.
\begin{definition}
\label{def:linking}
For all modules $\Mtwo$ and $M$, if the domains of$\Mtwo$ and $M$ are disjoint, 
we define the module linking function as $M\cdot \Mtwo\ \triangleq\ M\ \cup\ M'$.
\end{definition}
That is,  their linking is the union of the two if their domains are disjoint.


\newcommand{\Same}[4]{{SameModule(#1,#2,#3,#4)}}

Finally, we define what it means for two objects to come from the same module
\begin{definition}[Same Module]
\label{def:class-lookup}
For program state $\sigma$,  modules $\Mtwo$, and variables $x$ and $y$, we defone
$$\Same {x} {y} {\sigma}{\Mtwo}\ \triangleq\ \exists C, C', M[ \ M\in \Mtwo \wedge C, C'\in M \wedge  \class{\sigma}{x}=C \wedge \class{\sigma}{y} =C'\ ]$$
\end{definition}


Fig. \ref{f:loo-semantics} gives the operational semantics of \LangOO. 
Program state $\sigma_1$ reduces to $\sigma_2$ in the context of
modules$\Mtwo$ if $\exec{\Mtwo}{\sigma_1}{\sigma_2}$. The semantics in Fig. \ref{f:loo-semantics}
are unsurprising, but it is notable that reads (\textsc{Read}) and writes (\textsc{Write})
are restricted to the class that the field belongs to,
\sdN{and methods  may only be called if public, or from same module as current receiver.}
\begin{figure}[t]
\begin{minipage}{\textwidth}
\begin{minipage}{\textwidth}
\footnotesize
\begin{mathpar}
\infer
	{
	\sigma_1 = (\sdN{\overline{\phi}\cdot\phi},  \chi)\\
        \sdN{\phi}.(\prg{contn})  \txteq   u := y_0.m(\overline{y}); s \\
       % \phi_1' = \phi_1[\prg{contn} := (x := \bullet; s)]\\
	\meth{\Mtwo}{\class{\sigma_1}{y}}{m} = p \ C::m(\overline{x : T})\red{:T}\{s'\}\\
        	{\sdN{p=\prg{public} \ \vee \ \Same{\prg{this}} {y_0} {\sigma_1}{\Mtwo} }} \\
	\phi' = \{\prg{local}:= ([\prg{this}\ \mapsto\ \interpret{\sigma_1}{y_0}]\overline{[x\ \mapsto\ \interpret{\sigma_1}{y}]}), \prg{contn}:=s'\} \\
	  \sigma_2 = \sdN{(\overline{\phi}\cdot\phi\cdot\phi'},\chi)\\
	}
	{\exec{\Mtwo}{\sigma_1}{\sigma_2}}
	\quad(\textsc{Call})
	\and
\infer
	{
%	\sigma_1 = (\chi, \phi_1 \cdot\psi) \\
%	\sigma_2 = (\chi, \phi_2 \cdot\psi) \\
	\sdN{\sigma_1.(\prg{contn}) =  x := y.f; s} \\
	 \Same {\prg{this}}  {y}  {\sigma_1} {\Mtwo}\\
	%\phi_2 = \{\prg{local}:=\phi_1.(\prg{local})[x\ \mapsto\ v],\ \prg{contn}:=s\}
	\sigma_2=\sigma_1[x\mapsto  \interpret{\sigma_1}{y.f} \} ]
	}
	{\exec{\Mtwo}{\sigma_1}{\sigma_2}}
	\quad(\textsc{Read})
	\and
\infer
	{
 %	\sigma_1 = (\overline{\phi}\cdot\phi,\chi ) \\
	\sigma_1 .(\prg{contn}) =  x.f := y; s \\
	\Same {\prg{this}}  {x}  {\sigma_1} {\Mtwo}\\
%	\phi' = \phi [ \prg{contn}:=s ] \\ % \{\prg{local}:=\phi_1.(\prg{local}),\ \prg{contn}:=s\}\\
%	\chi' = \chi[\interpret{\sigma_1}{x}.f \mapsto\ \interpret{\sigma_1}{y} ]\\
	\sigma_2 = \sigma[\prg{contn}:=s][\interpret{\sigma_1}{x}.f \mapsto\ \interpret{\sigma_1}{y} ]	
	(\overline{\phi}\cdot\phi',\chi' )\\
	}
	{\exec{\Mtwo}{\sigma_1}{\sigma_2}}
	{}
	\quad(\textsc{Write})
	\and
\infer
	{
	\sigma_1.(\prg{contn})\ =\  x := \prg{new}\ C(\overline{z}); s \\
	fields(C)=\overline{f} \\
	\alpha \mbox{ fresh in } \sigma_1 \\
	% \phi' = \{\prg{local}:=[\prg{this} \mapsto \alpha],\overline{[p_i \mapsto \lfloor z_i \rfloor_{\sigma_1}}], \prg{contn} := s'\}\\
	\sigma_2 = \sigma_1[x\mapsto \alpha][\prg{cont}\mapsto s][\alpha  \mapsto  \{\prg{class}:=C, \prg{flds}:=\overline{f\ \mapsto\ \interpret {\sigma_1} {x}} ] 
%	\sigma_1 = (\chi, \phi \cdot\psi) \\
%	\phi.(\prg{contn}) = (x := \prg{new}\ C(\overline{z}); s)\\
%	M(C).(\prg{constr}) = (\overline{p : T})\{ s' \} \\
%	\phi' = \{\prg{local}:=[\prg{this} \mapsto \alpha],\overline{[p_i \mapsto \lfloor z_i \rfloor_{\sigma_1}}], \prg{contn} := s'\}\\
%	\sigma_2 = (\chi[\alpha\ \mapsto\ \{\prg{class}:=C, \prg{flds}:=\overline{f\ \mapsto\ \nul}], \phi' \cdot\phi[\prg{contn}\ :=\ (x := \bullet; s)] \cdot\psi)	
		}
	{\exec{\Mtwo}{\sigma_1}{\sigma_2}}
	\quad(\textsc{New})
	\and
\infer
	{
	\sigma_1 = (\sdN{\overline{\phi}\cdot}\phi_ \cdot\phi', \chi) \\
	\phi'.(\prg{contn}) \txtin  \red{z} \\ % )\ \textit{or}\ \phi_1.(\prg{contn}) = (\red{z})\\
	\phi.(\prg{contn})   \txteq  \red{x := y_0.m(\overline y)}; s \\
	\phi''= \phi[x \mapsto \interpret{\sigma_1}{z}][\prg{contn}:= \red{s} ]\\ 
	%(\prg{local} := \phi.(\prg{local})[x \mapsto \interpret{\sigma_1}{z}],  \prg{contn}:= \red{s} )   \\
	\sigma_2 = (\sdN{\overline{\phi}\cdot} \phi'', \chi)
	}
	{\exec{\Mtwo}{\sigma_1}{\sigma_2}}
	{}
	\quad(\textsc{Return})
\end{mathpar}
\caption{\LangOO operational Semantics}
\label{f:loo-semantics}
\end{minipage}
\begin{minipage}{\textwidth}
\footnotesize
\begin{mathpar}
\infer
		{}
		{\eval{M}{\sigma}{v}{v}}
		\quad(\textsc{E-Val})
		\and
\infer
		{}
		{\eval{M}{\sigma}{x}{\interpret{\sigma}{x}}}
		\quad(\textsc{E-Var})
		\and
\infer
		{
		\eval{M}{\sigma}{e_1}{i_1}\\
		\eval{M}{\sigma}{e_2}{i_2}\\
		i_1 + i_2 = i
		}
		{
		\eval{M}{\sigma}{e_1 + e_2}{i}
		}
		\quad(\textsc{E-Add})
		\and
\infer
		{
		\eval{M}{\sigma}{e_1}{v}\\
		\eval{M}{\sigma}{e_2}{v}
		}
		{
		\eval{M}{\sigma}{e_1 = e_2}{\true}
		}
		\quad(\textsc{E-Eq}_1)
		\and
\infer
		{
		\eval{M}{\sigma}{e_1}{v_1}\\
		\eval{M}{\sigma}{e_2}{v_2}\\
		v_1 \neq\ v_2
		}
		{
		\eval{M}{\sigma}{e_1 = e_2}{\false}
		}
		\quad(\textsc{E-Eq}_2)
		\and
\infer
		{
		\eval{M}{\sigma}{e}{\true}\\
		\eval{M}{\sigma}{e_1}{v}
		}
		{
%		\eval{M}{\sigma}{\ifthenelse{e}{e_1}{e_2}}{v}
		\eval{M}{\sigma}{e}{v}
		}
		\quad(\textsc{E-If}_1)
		\and
\infer
		{
		\eval{M}{\sigma}{e}{\false}\\
		\eval{M}{\sigma}{e_2}{v}
		}
		{
%		\eval{M}{\sigma}{\ifthenelse{e}{e_1}{e_2}}{v}
		\eval{M}{\sigma}{e}{v}
		}
		\quad(\textsc{E-If}_2)
		\and
\infer
		{
		\eval{M}{\sigma}{e}{\alpha}
		}
		{
		\eval{M}{\sigma}{e.f}{\interpret{\sigma}{\alpha.f}}
		}
		\quad(\textsc{E-Field})
		\and
\infer
		{
		\eval{M}{\sigma}{e_1}{\alpha}\\
		\eval{M}{\sigma}{e_2}{v'}\\
		\prg{ghost}\ g(x : T)\{e\} : T'\ \in\ M(\class{\sigma}{\alpha}).(\prg{gflds})\\
		\eval{M}{\sigma}{[v'/x]e}{v}
		}
		{
		\eval{M}{\sigma}{e_1.g(e_2)}{v}
		}
		\quad(\textsc{E-Ghost})
\end{mathpar}
\caption{\LangOO expression evaluation}
\label{f:evaluation}
\end{minipage}
\end{minipage}
\end{figure}

While the small-step operational semantics of \LangOO is given in Fig. \ref{f:loo-semantics},
specification satisfaction is defined over an abstracted notion of 
the operational semantics that models the open world. %, called \jm[]{\emph{external states semantics}}. 




An \emph{Initial} program state contains a single frame 
with a single local variable \prg{this} pointing to a single object 
in the heap of class \prg{Object}, and a continuation.
\begin{definition}[Initial Program State]
\label{def:initial}
A program state $\sigma$ is said to be an initial state ($\initial{\sigma}$)
if and only if
\begin{itemize}
\item
$\sigma.\prg{heap} = [\alpha\ \mapsto\ \{\prg{class}:=\prg{Object};\ \prg{flds}:=\emptyset\}]$ and
\item
$\sigma.\prg{stack} = \{\prg{local}:=[\prg{this}\ \mapsto\ \alpha];\ \prg{contn}:= s\}$
\end{itemize} 
for some address $\alpha$ and some statement $s$.
\end{definition}


%We give the semantics of module pair execution in Definition \ref{def:pair-reduce}
%\begin{definition}[External State Semantics]
%\label{def:pair-reduce-appendix}
%For all internal modules $M_1$, external modules $M_2$, and program configurations $\sigma$ and $\sigma'$, 
%we say that $\reduction{M_1}{M_2}{\sigma}{\sigma'}$ if and only if
%\begin{itemize}
%\item
%$\class{\sigma}{\sigma.(\prg{this})}\ \in\ M_2$ and
%\item
%$\class{\sigma'}{\sigma'.(\prg{this})}\ \in\ M_2$ and 
%\end{itemize} 
%and
%\begin{itemize}
%\item
%$\exec{M_1\ \circ\ M_2}{\sigma}{\sigma'}$ or
%\item
%$M_1 \circ M_2, \sigma \leadsto \sigma_1 \leadsto \ldots \sigma_n \leadsto \sigma'$ and $\class{\sigma_i}{\sigma_i.(\prg{this})} \in M_1$ for all $1 \leq i \leq\ n$
%\end{itemize}
%\end{definition}

Finally, we provide a semantics for expression evaluation is given in Fig. \ref{f:evaluation}. 
That is, given a module $M$ and a program state $\sigma$, expression $e$ evaluates to $v$
if $\eval{M}{\sigma}{e}{v}$. Note, the evaluation of expressions is separate from the operational
semantics of \LangOO, and thus there is no restriction on field access.



\newpage

%\clearpage
%\section{Appendix to Section 4 -- Auxiliary Concepts to Operational Semantics}
%% No appendix needed
%% Section has no lemmas
%% \clearpage
%%\section{Appendix to Section 5 -- Assertions}

\sdN{In order to prove \ref{lemma:addr:expr} from the next appendix, we first formulate and prove the following auxiliary lemma, which allows us to replace any variable $x$ in an extended expression $\re$, by its interpretation

\begin{lemma}
\label{aux:lemma:vars:eval}
For all extended expressions $\re$,  addresses $\alpha$ and  variables $x$, so that $x \in dom(\sigma):$

\begin{itemize}
\item
$\eval{M}{\sigma}{{\re}} {\alpha} \ \ \ \Longleftrightarrow \ \ \ \ \eval{M}{\sigma}{{\re[{\interpret \sigma x}/x] }} {\alpha}$
\end{itemize}
\end{lemma}
}

\sdN{
Note that in the above we require that $x\in dom( \sigma)$, in order to ensure that the replacement $[{\interpret \sigma x}/x]$ is well-defined.
On the other hand, 
we do not require  that $x\in \fv(\re)$, because 
if  $x\notin \fv(\re)$, then ${\re[{\interpret \sigma x}/x] } \txteq \re$ and the guarantee from above becomes a tautology. 
}

\sdN{
\noindent
\textbf{Proof of Lemma 
\ref{aux:lemma:vars:eval}} % Take any $M$ $e$, $\sigma$, and $x$
 The proof goes by induction on the structure of $\re$ -- as defined in Def. \ref{f:chainmail-syntax} -- and according to the expression evaluation rules from Fig. \ref{f:evaluation}.
\noindent
\textbf{End of Proof}
}
%\clearpage
%\section{Appendix to Section \ref{s:preserve} -- Preservation of Satisfaction }
\label{app:preserve}

 
\beginProof{lemma:addr:expr}

We first prove that for  any $M$ $A$, $\sigma$

\begin{enumerate}
\item
To show that  $ \satisfiesA{M}{\sigma}{A}\ \ \ \ \Longleftrightarrow \ \ \ \ \ \satisfiesA{M}{\sigma}{A[{\interpret \sigma x}/x]} $ 

The proof goes by induction on the structure of $A$,   application of  Defs.  \ref{def:chainmail-semantics}, \ref{def:chainmail-protection-from}, and  \ref{def:chainmail-protection}.
%\item Assume that   $\satisfiesA{M}{\sigma}{A[{\interpret \sigma x}/x]} $. To show that  $ \satisfiesA{M}{\sigma}{A}$. 


\item
To show that $ \satisfiesA{M}{\sigma}{A}   \ \ \ \Longleftrightarrow\ \ \ \satisfiesA{M}{\sigma[\prg{cont}\mapsto stmt]}{A}$ 

The proof goes by induction on the structure of $A$,   application of  Defs.  \ref{def:chainmail-semantics}, \ref{def:chainmail-protection-from}, and  \ref{def:chainmail-protection}.


\end{enumerate}

The lemma itself then follows form (1) and (2) proven above.

\completeProof

 
In addition to what is claimed in Lemma  \ref{lemma:addr:expr}, it  also holds that 
\begin{lemma}
$\eval{M}{\sigma}{{\re}}{\alpha}  \ \ \Longrightarrow\ \  [ \ \satisfiesA{M}{\sigma}{A} \  \Longleftrightarrow\   \  \satisfiesA{M}{\sigma}{A[\alpha/\re]} \  \  ]$
\end{lemma}

\begin{proof} by induction on the structure of $A$,   application of  Defs.  \ref{def:chainmail-semantics}, \ref{def:chainmail-protection-from}, and  \ref{def:chainmail-protection}, and , and auxiliary lemma \ref{aux:lemma:vars:eval}.

\end{proof}

\subsection{Stability}

%The proof of lemma \ref{lemma:addr:expr} is by induction on the structure of $A$.

We first give complete definitions for the concepts of $  \Stable {\_]}$ and $\Pos {\_}$

\vspace{.2cm}

% An assertion is {\emph {stable}}, written as $\Stable A$ if it does not contain $\inside {\re}$ assertions:
\begin{definition}
\label{def:Basic}
%\begin{figure}[hbt]
[$\Stable{\_}$] assertions: %  defined below

$
\begin{array}{l}
 \begin{array}{c}
  \Stable {\inside{\re}}  \triangleq  false \\
    \Stable {\protectedFrom \re {\overline {u}}} =  
  \Stable  {\internal \re} =  %   \PushAS y  {(\external \re)} & \triangleq &   {\external \re}
    \Stable {\re}=   
     \Stable {\re:C}\   \triangleq \    true
 \end{array}
  \\
 \begin{array}{lcl}
 \Stable  {A_1  \wedge  A_2}\  \triangleq\     \Stable  { A_1}  \wedge    \Stable  {A_2}    &
\lclSPACE  &  
 \Stable  {\forall x:C.A} =\Stable  {\neg A} \   \triangleq\   \Stable A
 \end{array}
 \end{array}
$
\label{f:Basic}
 \end{definition}


 \begin{definition}
% \begin{figure}[hbt]
[$\Pos{\_}$] assertions: %  defined below

$
\begin{array}{l}
 \begin{array}{c}
  \Pos {\inside{\re}} =  \Pos {\protectedFrom \re {\overline {u}}} =  
  \Pos  {\internal \re} =   
    \Pos {\re}=   
     \Pos {\re:C}\   \triangleq \    true
     \\
% \end{array}
%  \\
% \begin{array}{lll}
 \Pos  {A_1  \wedge  A_2}\  \triangleq\     \Pos  { A_1}  \wedge    \Pos  {A_2}  
  \\ 
 \Pos  {\forall x:C.A}   \triangleq\   \Pos A
\\ % \   &
  \Pos {\neg A}  \triangleq \Stable A % \Neg A
\end{array}
 \end{array}
 $
 \label{def:Pos}
\end{definition}

%\begin{definition}
%[$\Neg{\_}$] assertions:
%
%$
%\begin{array}{l}
% \begin{array}{lll}
%  \Neg {\inside{\re}}\  \triangleq \    false &    \Neg {\protectedFrom \re {\overline {u}}} =  
%  \Neg  {\internal \re} =   
%    \Neg {\re}=   
%     \Neg {\re:C}\   \triangleq \    true
%\\ 
% \Neg  {A_1  \wedge  A_2}\  \triangleq\     \Neg  { A_1}  \wedge    \Neg  {A_2}   \ \   
% &    
% \Neg  {\forall x:C.A}   \triangleq\   \Neg A \ \ 
%&
%  \Neg {\neg A}  \triangleq \Pos A
% \end{array}
% \end{array}
%$
%\label{def:Neg}
%\end{definition}

The definition  of $\Pos{\_}$ is  less  general than would be   possible. \Eg $(\inside {x} \rightarrow  x.f=4) \rightarrow xf.3=7$  does not satisfy our definition of $\Pos {\_}$.
We have given these less general definitions in order to simplify both our defintions and our proofs.

\beginProof {l:preserve:asrt} 
Take any  state  $\sigma$, frame  $\phi$,  assertion  $A$,  
 
 
\begin{itemize}
\item 
To show\\
  $\Stable{A}\ \wedge \  \fv(A)=\emptyset \ \ \  \Longrightarrow \ \  \ \  [\ \ M, \sigma \models A \ \ \Longleftrightarrow \ \  M,{\PushSLong \phi \sigma} \models A\ \ ]$

By induction on the structure of the definition of $\Stable{A}$.

\item 
To show\\
 $\Pos{A}\ \wedge \  \fv(A)=\emptyset \  \wedge \     {M\cdot\Mtwo \models {\PushSLong \phi \sigma}}\ \wedge\ 
  \ M, \sigma \models A \  \wedge \  M, {\PushSLong \phi \sigma} \models  \intThis \ \ \Longrightarrow$ \\
  $\strut \hspace{2cm}  \ \  M,{\PushSLong \phi \sigma} \models A $

By induction on the structure of the definition of $\Pos{A}$.
The only interesting case is when $A$ has the form $\inside {\re}$. Because 
$ fv(A)=\emptyset$, we know that $\interpret {\sigma} {\re}$=$\interpret {\PushSLong \phi \sigma}  {\re}$. Therefore, we assume that 
 $\interpret {\sigma} {\re} = \alpha$ for some $\alpha$, assume that $ M,  \sigma  \models \inside \alpha$, and want to show that  $ M,{\PushSLong \phi \sigma} \models \inside \alpha$. 
 From $   {M\cdot\Mtwo \models {\PushSLong \phi \sigma}}$ we obtain that
 $Rng(\phi) \subseteq Rng(  \sigma)  $. 
 From this, we obtain that
  $  \LRelevantO {\PushSLong \phi \sigma} \subseteq \LRelevantO  {\sigma}$.
  The rest follows by unfolding and folding Def. \ref{def:chainmail-protection}.

  \end{itemize}
 
\completeProof




\subsection{Encapsulation}

{
Proofs of adherence to {\SpecLang specifications  hinge on the expectation that some, 
specific, assertions cannot be invalidated unless some 
} internal (and thus known) computation took place. 
{We call such assertions   \emph{encapsulated}.}
}
We define the  judgement,  $M\ \vdash  \encaps{A}$, in terms of the judgment  $M; \Gamma \vdash \encaps A  ; \Gamma'$ from Fig. \ref{f:encaps:aux}.
This judgements ensures   that any objects whose fields  are read  in the validation of $A$ are internal, 
that $\protectedFrom {\_} {\_}$  does not appear in $A$,  and that protection assertions (ie $\inside{}$ or $\protectedFrom {\_} {\_}$) do not appear in negative positions in $A$. The second environment in this judgement, $\Gamma'$, is used to keep track of any variables introduces in that judgment, \eg we would have that\\
$\strut \hspace{1cm} M_{good}, \emptyset\ \vdash\ \encaps{a:\prg{Account}\wedge  k:\prg{Key} \wedge \inside{a.\prg{key}} \wedge a.\prg{key}\neq k}; \ (a:\prg{Account}, k:\prg{Key}$.


We assume a type judgment $M; \Gamma \vdash e :  \prg{intl}$ which says that in the context of $\Gamma$, the expression $e$ belongs to a class from $M$.
We also assume that the judgement $M; \Gamma \vdash e :  \prg{intl}$ can deal with ghostfields -- namely, ghost-methods have to be type checked in the contenxt of $M$ and therefor they will only read the state of internal objects.
Note that it is possible for $M; \Gamma \vdash \encaps {\re}; \Gamma'$ to hold and 
$M; \Gamma \vdash  e : \prg{intl}$ not to hold -- \cf rule {\sc{Enc\_1}}.


\begin{figure}[thb]
$
\begin{array}{l}
\begin{array}{lclcl}
\inferruleSDNarrow 
{~ \strut  {\sc{Enc\_1}}}
{  
\begin{array}{l}
M; \Gamma \vdash \re : \prg{intl} \\
M; \Gamma \vdash \encaps{\re};\  \Gamma'
\end{array}
}
{
M; \Gamma \vdash \encaps{\re.f};\  \Gamma'
}
& &
\inferruleSDNarrow 
{~ \strut  {\sc{Enc\_2}}}
{  
\begin{array}{l}
  \\
M; \Gamma \vdash \encaps{\re};\  \Gamma'
\end{array}
}
{
M; \Gamma \vdash \encaps{\re: C};\  (\Gamma', \re:C)
}
& &
\inferruleSDNarrow 
{~ \strut  {\sc{Enc\_3}}}
{   
\begin{array}{l}
M; \Gamma \vdash \encaps{A};\ \Gamma'  \\
 A \mbox{\ does\ not\ contain\ $\inside{\_}$}
% Chooped, superfluous because if 1st requirement
%\\ A \mbox{\ does\ not\ contain\ \ $\protectedFrom {\_} {\_}$}
\end{array}
}
{
M; \Gamma \vdash \encaps{ \neg A};\  \Gamma'  
}
\\ \\
\inferruleSDNarrow 
{~ \strut  {\sc{Enc\_4}}}
{  
\begin{array}{l}
M; \Gamma \vdash \encaps{A_1};\ \Gamma''   \\
  M; \Gamma'' \vdash \encaps{ A_2};\ \Gamma' 
  \end{array} 
}
{
M; \Gamma \vdash \encaps{A_1 \wedge A_2};\  \Gamma'
}
& &
\inferruleSDNarrow 
{~ \strut  {\sc{Enc\_5}}}
{  
\\
M; \Gamma, {x:C} \vdash \encaps {A};\ \Gamma' 
}
{
M; \Gamma \vdash \encaps {\forall {x:C}. A};\  \Gamma'
}
& & 
\inferruleSDNarrow 
{~ \strut  {\sc{Enc\_6}}}
{  
\\
M; \Gamma \vdash \encaps{\re};\  \Gamma'
}
{
M; \Gamma \vdash \encaps{\re: \prg{extl} };\  \Gamma'
}
\end{array}
\\ \\
\inferruleSDNarrow 
{~ \strut  {\sc{Enc\_7}}}
{  
M; \Gamma \vdash \encaps{\re}; \Gamma' 
}
{
M; \Gamma \vdash \encaps{\inside{\re}}; \ \Gamma' 
}
\end{array}
$
\caption{The judgment $M; \Gamma \vdash \encaps  {A}; \Gamma'$}
\label{f:encaps:aux}
\label{f:encaps}
\end{figure}


 

\begin{definition}[An assertion $A$ is \emph{encapsulated} by module $M$] $~$ \\
\label{d:encaps:sytactic}
\begin{itemize}
\item 
$M \vdash \encaps{A}  \ \   \triangleq  \ \  \exists \Gamma.[\ M; \emptyset \vdash \encaps{A}; \Gamma\ ]$ \ \  as defined in Fig. \ref{f:encaps}.
 \end{itemize}
  \end{definition}

To motivate the design of our judgment $M; \Gamma \vdash \encaps{A}; \Gamma'$,  we first give a semantic notion of encapsulation:


\begin{definition}  An assertion $A$ is semantically encapsulated by module $M$:
\label{d:encaps:sem}

\begin{itemize}
\item
$
%\begin{equation} 
    M\ \models \encaps{A}\ \   \triangleq  \   
     \forall \Mtwo, \sigma, \sigma'.[   \ \  \satisfiesA{M}{\sigma}{(A  \wedge \externalexec)}\  \wedge\ { \leadstoBounded {M\madd\Mtwo}  {\sigma}{\sigma'}} % \\    \ \ \ \ \ \   
        \  \Longrightarrow\  
    {M},{\sigma'}\models {\as \sigma A} \ \  ]
%  \end{equation}
  $
\end{itemize}
\end{definition}

\noindent
\textbf{More on Def. \ref{d:encaps:sem}} {If the definition \ref{d:encaps:sem} or in lemma \ref{d:encaps} we had used the more general execution, $\leadstoOrig  {M\madd\Mtwo}  {\sigma}{\sigma'}$, rather than the scoped execution,  $\leadstoBounded {M\madd\Mtwo}  {\sigma}{\sigma'}$,
 then fewer assertions would have been encapsulated.}
Namely, assertions like    $\inside {x.f}$ would not be encapsulated.
Consider, \eg, a heap $\chi$, with objects $1$, $2$, $3$ and $4$, where  $1$, $2$ are external, and $3$, $4$ are internal, and  $1$ has fields pointing to $2$ and $4$, and $2$ has a field pointing to $3$, and $3$ has a field $f$ pointing to $4$. Take  state $\sigma$=$(\phi_1\!\cdot\!\phi_2,\chi)$, where $\phi_1$'s receiver is $1$,  $\phi_2$'s receiver is $2$,   and there are no local variables. 
We have  $...\sigma\models \externalexec \wedge \inside {3.f}$. 
We  return from the most recent all, 
getting  $\leadstoOrig  {...}  {\sigma}{\sigma'}$ where $\sigma'=(\phi_1,\chi)$; and have   $...,\sigma'\not\models  \inside {3.f}$.

\begin{example}
\label{ex:not:encaps}
For an assertion $A_{bal}\  \triangleq\ a:\prg{Account}\wedge a.\prg{balance}=b$, % can only be invalidated through internal  methods.  %on internal objects.
and modules \ModB and  \ModC  from \S~\ref{s:outline}, we have  \ \ \ $\ModB\ \models\ \encaps{ A_{bal} }$, \ \ \ and \ \ \ $\ModB\ \models\ \encaps{ A_{bal} }$.
\end{example}


\begin{example} Assume   further modules, $\ModD$ and $\ModE$,  which  use ledgers mapping  accounts to their balances, and export functions that update this map. In  $\ModD$ the ledger is  part of the {internal} module, %\emph{not} protected, while 
while in $\ModE$ it is part of the  {external} module.
Then  \ \ $\ModD \ \not\models\encaps{ A_{bal}} $, \ \  and \ \ $\ModE  \models \encaps{ A_{bal}} $.
Note that in both $\ModD$ and $\ModE$, the term \prg{a.balance} is a ghost field. 
\end{example}

\begin{note} Relative protection % (a variable is protected from another one) 
is not encapsulated, (\eg $M \not\models {\encaps{\protectedFrom{x}{y}}}$), even though    absolute protection is
(\eg $M \models \encaps{\inside{x}}$).
Encapsulation of an assertion does not imply encapsulation of its negation; 
 for example,  $M \not\models {\encaps{\neg\inside{x}}}$.
\end{note}

\noindent
\textbf{More on Def. \ref{d:encaps:sytactic}} This definition is less permissive than necessary. 
For example $M \not\vdash \encaps{\neg ( {\neg \inside {x}})}$ even though 
 $M  \models \encaps{\neg ( {\neg \inside {x}})}$.
 Namely, 
$\neg (\neg \inside {x}) \equiv  \inside {x}$ and $M \vdash {\encaps{  \inside {x}}}$.
A more permissive, sound, definition, is not difficult, but not the main aim of this work.
We gave this, less permissive definition, in order to simplify the definitions and the proofs. 


 \vspace{.1cm}

 \beginProof{lem:encap-soundness} This says that $M \vdash {\encaps A}$ implies that $M \vdash {\encaps A}$.
 \\
 We fist prove that\\
 $\strut \ \ \ \ $  (*) Assertions $A_{poor}$ which do not contain $\inside {\_}$ or $\protectedFrom {\_} {\_}$ are preserved by any external step.\\
 Namely, such an assertion only depends on the contents of the fields of internal objects, and these are not modified  by external steps.
 Such an $A_{poor}$ is defined essentially through
\\ 
 $
\begin{syntax}
\syntaxElement{\ \ \ \  A_{poor}}{}
		{
		\syntaxline
				{{\re}}
				{{\re} : C}
				{\neg A_{poor}}
				{A_{poor}\ \wedge\ A_{poor}}
				{\all{x:C}{A_{poor}}}
				{\external{{\re}}}
		\endsyntaxline
		}
\endSyntaxElement\\
\end{syntax}
$
\\
 We can prove (*) by induction on the structure of $A_{poor}$ and case analysis on the execution step.
  
\vspace{.05cm}   
\noindent
We then prove Lemma \ref{lem:encap-soundness} by induction on the structure of $A$. 

\noindent 
--- The cases {\sc{Enc\_1}}, {\sc{Enc\_2}},  and {\sc{Enc\_6}} 
are  straight application of (*). 

\noindent 
--- The case {\sc{Enc\_3}} also follows from (*), because any $A$ which satisfies $\encaps {A}$ and which does not contain $\inside {\_}$
is an $A_{poor}$ assertion.

\noindent 
--- The cases {\sc{Enc\_4}} and {\sc{Enc\_5}}  follow  by induction hypothesis.

\noindent 
--- The case {\sc{Enc\_7}} is more interesting. \\
We assume that $\sigma$ is an external state,   that $\leadstoBounded {...} {\sigma} {\sigma'}$, and that $.. \sigma \models \inside \re$.
By definition, the latter means that\\
 $\strut \ \ \ \ $  (**) no locally reachable external object in $\sigma$ has a field ponting to $\re$, \\
 $\strut \ \ \ \ \ \ \ \ \ \ \ $ nor is $\re$ one of the variables.\\
We proceed by case analysis on the step $\leadstoBounded {...} {\sigma} {\sigma'}$.
\\
- If that step was an assignment to a local variable $x$, then this does not affect $\inside {\as \sigma \re}$ because in $\interpret \sigma {\re}$=$\interpret {\sigma'} {\as \sigma {\re}}$,
and $... \sigma' \models x \neq re$.
\\
- If that step was an assignment to an external object's field, of the form $x.f :=y$ then this does not affect $\inside {\as \sigma \re}$ either.
This is so, because $\encaps {\re}$ gives that $\interpret \sigma {\re}$=$\interpret \sigma' {\as \sigma {\re}}$ -- namely the evaluation of $\re$ does not read $x$'s fields, since $x$ is external. 
And moreover, the assignment $x.f :=y$ cannot create a new, unprotected  path to $\re$ (unprotected means here that the penultimate element in that path is external), because then we would have had in $\sigma$  an unprotected path from $y$ to $\re$.
\\
- If that step was a method call, then we apply lemma \ref{lemma:push:N} which says that all objects reachable in $\sigma'$ were already reachable in $\sigma$.
\\
- Finally, we do not consider method return (\ie the rule {\sc{Return}}),   because we are looking at $\leadstoBounded {...} {\_} {\_}$ execution steps rather than $\leadstoOrig  {...}  {\_}{\_}$ steps.
\hspace{3cm}
\completeProof

 

%\TODO{{\red{TODO}: Say that there are more ways to define the encaps judgement. ALSO: show the rule for %the ghostfields. }}


%\clearpage
%\section{Appendix to Section \ref{sect:spec} -- Specifications}
%\clearpage
%\section{Expressiveness} 

\label{app:expressivity}


\begin{figure}[tbh]
\begin{lstlisting}[language = Chainmail, mathescape=true, frame=lines]
module DOM 
   class Node
      field cnt: int
      field parent: Node 
      
   class Proxy
      field hght: nat
      field node: Node
	
      public method set(newCnt:int, up:nat): void
            if this.hght >= up
               setPrivate(newCnt, up)
            else
               return
	       
      private method setPrivate(newCnt:int, up:nat): void
            if up==0 then
               this.node.cnt := nwCnt
            else
               setPrivate(newCnt, up-1)
\end{lstlisting}
\caption{The DOM module  -- classes \prg{Node} and \prg{Proxy}}
\label{fig:DoMCode}
\end{figure}

We argue the expressiveness of our approach by comparing with example specifications  proposed in \cite{OOPSLA22,dd,irisWasm23}.

\subsection{The DOM}  %\sophiaPonder[renamed Wrapper to Proxy]{  
\label{ss:DOM}

\subsubsection{The Problem} This is the motivating example in \cite{dd}. It
deals with a tree of DOM nodes: Access to a DOM node
gives access to all its \prg{parent}s, % simoplified it and \prg{children} nodes, 
with the ability to modify the node's contents   -- where  \prg{parent} %\prg{children} 
and \prg{cnt} are fields in class \prg{Node}. Since the top nodes of the tree
usually contain privileged information, while the lower nodes contain
less crucial third-party information, we must be able to limit 
 access given to third parties to only the lower part of the DOM tree. 
 
To do this,   \citet{dd}  propose   
  a \prg{Proxy} class, which has a field \prg{node} pointing to a \prg{Node}, and a field height (\prg{hght}), which restricts the range of \prg{Node}s which may be modified through the use of the particular \prg{Proxy}. Namely,   a \prg{Proxy}  may modify 
the \prg{cnt} of all the ancestors of its  \prg{node}, up to the   \prg{hght}-th ancestor of that \prg{node}. 

A possible implementation of such \prg{Node} and \prg{Proxy} classes is shown in Fig.  \ref{fig:DoMCode}, 
while the creation of a tree, three proxies, and  passing the proxies to the external world is shown in Fig \ref{fig:DoMCodeClient}.




\begin{figure}[tbh]
\begin{lstlisting}[language = Chainmail, mathescape=true, frame=lines]
module DOM 
   ... as before ...
   class Example
      public method demo(untrst:external) : void 
          //  create a tree of $\prg{Node}s$
         nd1 := new Node; nd1.cnt := 1
         nd2 := new Mode; nd2.parent:= nd1; nd2,ctns:=2;
         nd3 := new Mode; nd3.parent:= nd1; nd3,ctns:=3;
         nd4 := new Mode; nd4.parent:= nd2; nd4,ctns:=4;
         nd5 := new Mode; nd5.parent:= nd2; nd5,ctns:=5;
         
         //  create three $\prg{Proxy}s$
         prx10 := newProxy; prx10.height:=2; prx10.node:=nd4;
         prx11 := newProxy; prx10.height:=1; prx10.node:=nd4; 
         prx12 := newProxy; prx10.height:=0; prx10.node:=nd4;
          
         // External calls:         
         untrst.meth_A();       // no modification in the tree
         
         untrst.meth_B(nd2);    // no modification in the tree     
         
         untrst.meth_C(prx12);  // nd4.cnt may have changed; 
                                // all else has stayed the same
         
         nd4.ctns:=4            // re-establish old value of nd4.ctns
         untrst.meth_A( );      // nd4.cnt may have changed again; 
                                // all else has stayed the same
         
         untrst.meth_C(prx12);  // nd4.cnt, nd2.cnt, and nd1.ctns may have changed; 
                                // all else has stayed the same
      
        // Revocation:
         prx12.height := 1;     // revocation
         nd1.ctns:=1            // re-establigh the old value of nd1.ctns
         untrst.meth_C(prx12);  // nd1.ctns does not change     
\end{lstlisting}
\caption{The DOM module continued -- class \prg{Example} }
\label{fig:DoMCodeClient}
\end{figure}
In Fig. \ref{f:DOM:Tree:Diagrs}  we show a diagram of  a tree consisting of the five nodes, \prg{nd1}, \prg{nd2},  \prg{nd3}, \prg{nd4} and \prg{nd5}, 
and three proxies, \prg{prx10}, \prg{prx11},  and \prg{prx12}. These are created in lines 6-15 in the code in Fig. \ref{fig:DoMCodeClient}.
Moreover, in Fig. \ref{f:DOM:Diagrs}  we revisit the diagram, and show highlight that three objects have capability on \prg{nd4.cnt},
while two objects have capability on \prg{nd2.cnt}, and finally, only one object has capability on  \prg{nd1.cnt}.

We now consider the remaining code in Fig. \ref{fig:DoMCodeClient}:

\begin{itemize}
\item 
The first external call, in line 18, does not affect the DOM tree or is \prg{cnt}, because \prg{untrst} has no access to it.
 \item 
The second external call, in line 20, also does not affect the DOM tree or is \prg{cnt} -- even though \prg{untrst} has access to a \prg{Node}, because the class has no public method, none of its fields may be modified. 
 \item 
The third external call, in line 22, may affect the contents of \prg{nd4.ctns}, because we have passed the capability \prg{prx12} in the call. 
From then on, the capability is no longer protected.
 \item 
Before the fourth external call, on line 25, we restore the value of \prg{nd4.cnt} to what it was earlier. Nevertheless, even though the fourth external call, on 
line 26, is identical to the from line 19, it may result in a modification of \prg{nd4.ctns}. This is so, because \prg{prx12} is no longer protected.
 \item 
The fifth external call, on line 29, may result in the modification of \prg{nd4.ctns}, \prg{nd2.ctns}, and \prg{nd1.ctns}, because we have passed the capability \prg{prx10}.
 \item 
We now revoke the capability of \prg{prx12}, on line 34, so that it can only affect \prg{nd4.cnt} and \prg{nd2.cnt}. We re-establish the value of \prg{nd1.cnt} on line 35, and on line 36 we make an external call. We know that this call cannot affect the value of \prg{nd1.cnt}.
\footnote{Another example of a revocation would be to set the   \prg{parent} of \prg{nd4} to be \prg{nd3}; then, \prg{prx11} 
would have capability on \prg{nd4} and \prg{nd3}.
On the other hand, setting the \prg{parent} of \prg{nd4}  to be \prg{nd1} would augment the capability of \prg{prx11};
such an augmentation would, however be forbidden by $S_{dom_4}$, if the corresponding capability was had already been exported to the external world.}
\end{itemize}



 
\begin{figure}[th] 
\resizebox{7cm}{!}{
\includegraphics[width=\linewidth]{diagrams/Dom_A.png}
} 
\caption{DOM tree and Proxies, created as in the code from Fig \ref{fig:DoMCodeClient}}
 \label{f:DOM:Tree:Diagrs}
 \end{figure}



\begin{figure}[th] 
\begin{tabular}{|c|c|c|}
\hline
\resizebox{3,8cm}{!}{
\includegraphics[width=\linewidth]{diagrams/Dom_B.png}
}
&
\resizebox{3.8cm}{!}{
\includegraphics[width=\linewidth]{diagrams/Dom_C.png}
}
&
\resizebox{3,8cm}{!}{
\includegraphics[width=\linewidth]{diagrams/Dom_D.png}
}
\\
\hline
\prg{prx10}, \prg{prx10}, and \prg{prx12} are the  
&
\prg{prx10} and  \prg{prx11}  are the 
&
\prg{prx10}  is the\ 
\\
capabilities for \prg{nd4.cnt} 
&
capabilities for \prg{nd2.cnt}  
&
capability for \prg{nd1.cnt}  
\\
\hline
\end{tabular}
\caption{Capabilities for the DOM tree from Fig. \ref{f:DOM:Tree:Diagrs}. In particular, notice the many-to-many relation between capabilities and effects: For example, there are three capabilities on \prg{nd4.cnt}. And \prg{prx10} has capabilities on \prg{cnt} of three objects.}
%\prg{nd1.cnt} \prg{nd2.cnt} and \prg{nd4.cnt}.}
 \label{f:DOM:Diagrs}
 \end{figure}

\subsubsection{Specifications}
We now discuss possible specifications. For this, we will use  the predicate $may\_modify \subseteq \prg{Proxy} \times \prg{Node}$, which says that 
\prg{prx} has \emph{modification-capabilities} on \prg{nd}, where \prg{prx} is
a  \prg{Proxy} and \prg{nd} is a \prg{Node}, if \prg{nd} is the \prg{prx}$k^{th}$  parent
of   \prg{pr.node} where $k \leq \prg{nd}.\prg{hght}$.
The formal definition is as follows:
\\
$\strut \SPSP   may\_modi\!f\!y(\prg{prx}, \prg{nd}) \triangleq \exists k:\mathbb{N}. [ \  \prg{prx}.\prg{node}.\prg{parent}^k=\prg{nd}\ \wedge k\leq   \prg{prx}.\prg{hght}]$
\\
Thus, a \prg{Proxy} object \prg{prx}, which satisfies $may\_modi\!f\!y(\prg{prx},\prg{nd})$ is a capability which may modify the node $\prg{nd}$. 

 
\vspace{.1cm}

\noindent
We now use $may\_modify$ in writing the specifications.
Below, $S_{dom\_1}$ mandates  that nodes, $\prg{nd}$ are not leaked. 
Moreover, $S_{dom\_2}$  and $S_{dom\_3}$   mandate that the \prg{parent} node of a \prg{Node} cannot be modified by external code: 
{Similarly, $S_{dom\_4}$ mandates  that proxies which $may\_modi\!f\!y(\prg{prx},\prg{nd})$ are not leaked.}
{Finally, $S_{dom\_5}$ mandates   that proxies which $may\_modi\!f\!y(\prg{prx},\prg{nd})$ are not leaked, and that when a node $nd$ is protected, and  all proxies that can  $may\_modi\!f\!y(\prg{prx},\prg{nd})$ are protected, then  the $cnt$ of $nd$ cannot be modified by external code.}
\\
$\strut \SPSP  S_{dom\_1}\ \  \triangleq \ \ \TwoStatesN{ \prg{nd}:\prg{Node}}{\  \inside{\prg{nd}}\  ] \ }$ 
\\
\\
$\strut \SPSP  S_{dom\_2}\ \  \triangleq \ \ \TwoStatesN{ \prg{nd}:\prg{Node},\prg{nd'}:\prg{Node}}{\  \prg{nd.parent}=\prg{nd}'   \ }$ 
\\
$\strut \SPSP  S_{dom\_3} \  \triangleq \ \ \TwoStatesN{ \prg{nd}:\prg{Node} }{\  \prg{nd.parent}=\prg{null} \ }$ 
\\
\noindent
\\
$\strut \SPSP  S_{dom\_4}\ \  \triangleq \ \ \TwoStatesN{ \prg{nd}:\prg{Node}}{\  \forall \prg{prx}:\prg{Proxy}.[ \ may\_modi\!f\!y(\prg{prx}, \prg{nd}) \rightarrow \inside{\prg{prx}}\  ] \ }$ 
\\
\\
$\strut \SPSP  S_{dom\_5}\ \  \triangleq \ \  \forall{ \prg{nd}:\prg{Node}.\forall \prg{val}:\prg{Object} }$.\\
$\strut \SPSP\strut  \SPSP\strut \SPSP	\{  \  \inside{\prg{nd}} \wedge \forall \prg{prx}:\prg{Proxy}.[ \ may\_modi\!f\!y(\prg{prx}, \prg{nd} ) \rightarrow \inside{\prg{prx}}\  ]  \wedge \prg{nd.cnt} = \prg{val} \ \}  $ 


 Note that $S_{dom\_5}$ is strictly stronger than $S_{dom\_4}$. 
The module shown in Fig. \ref{fig:DoMCode} and \ref{fig:DoMCodeClient} does satisfy  all the  specifications from above. 

The careful reader might worry that the method \prg{demo} in class \prg{Example} in Fig. \ref{fig:DoMCodeClient} 
breaks specifications $S_{dom\_1}$, and $S_{dom\_4}$, since it does export \prg{nd1}  and \prg{prx12}. 
However, there is no external state between the creation of   \prg{nd1} and its export, and so our scoped invariant is preserved. 
Similarly, there is no external state between the creation of   \prg{prx10} and its export, and so our scoped invariant is preserved. 


 To prove that the external calls in Fig. \ref{fig:DoMCodeClient} preserve the stated properties  we need to apply the specifications  $S_{dom\_2}$, 
 and $S_{dom\_5}$, but we do not need   $S_{dom\_1}$ or $S_{dom\_3}$.
 


\subsubsection{Other Specifications}

\citet{OOPSLA22} specify this as:
 
 \begin{lstlisting}[language = Chainmail, mathescape=true, frame=lines]
DOMSpec $\triangleq$ from nd : Node $\wedge$ nd.property = p  to nd.property != p
  onlyIf $\exists$ o.[ $\external {\prg{o}}$ $\wedge$ 
     $( \exists$ nd':Node.[ $\access{\prg{o}}{\prg{nd'}}$ ]  $\vee$ 
     $\,\;\exists$ pr:Proxy,k:$\mathbb{N}$.[$\, \access{\prg{o}}{\prg{pr}}$ $\wedge$ nd.parent$^{\prg{k}}$=pr.node.parent$^{\prg{pr.height}}$ ] $\,$ ) $\,$ ]
\end{lstlisting}

\prg{DomSpec} states that the \prg{property} of a node can only change if
some external object presently has 
access to a node of the DOM tree, or to some \prg{Proxy} with modification-capabilties
to the node that was modified.
The assertion $\exists {o}.[\ \external {\prg{o}} \wedge \access{\prg{o}}{\prg{pr}}\ ]$ is the contrapositive of our  $\inside{pr}$, but is is weaker than that, because it does not specify the frame from which $o$ is accessible.
Therefore, $\prg{DOMSpec}$ is a stronger requirement than $S_{dom\_1}$.

 





% \pagebreak 
\subsection{DAO}
The Decentralized Autonomous Organization (DAO)~\cite{Dao}  is a well-known Ethereum contract allowing 
participants to invest funds. The DAO famously was exploited with a re-entrancy bug in 2016, 
and lost \$50M. Here we provide specifications that would have secured the DAO against such a 
bug. 
\\ 
$\strut \SPSP  S_{dao\_1}\ \  \triangleq \ \ \TwoStatesN{ d:\prg{DAO}}{\ \forall p:\prg{Participant}. [\ d.ether \geq d.balance(p) \ ]   \ }$ 
\\
$\strut \SPSP  S_{dao\_2}\ \  \triangleq \ \ \TwoStatesN{ d:\prg{DAO}}{\ \ d.ether \geq \sum_{p \in d.particiants} d.balance(p)\  \ }$ 


The specifications above say the following:
\\
\begin{tabular}{ll}
\begin{minipage}{.10\textwidth}
$\strut \SPSP  S_{edao\_1}$
\end{minipage}
&
\begin{minipage}{.85\textwidth}
guarantees that the DAO holds more ether than the balance  of any of its  participant's.
\end{minipage}
\\
\\
\begin{minipage}{.10\textwidth}
$\strut \SPSP  S_{dao\_2}$ 
\end{minipage}
&
\begin{minipage}{.85\textwidth}
guarantees that that the DAO holds more ether than the sum  of the balances held by DAO's participants.
\end{minipage}
\end{tabular}

$S_{dao\_2}$  is stronger than $S_{dao\_1}$. They would both have precluded the DAO bug. Note that these specifications  do not mention capabilities. 
They are, essentially, simple class invariants and could have been expressed with the techniques proposed already by \cite{MeyerDBC92}.
The only difference is that $S_{dao\_1}$ and $S_{dao\_2}$ are two-state invariants, which means that we require that they are \emph{preserved},
\ie if they hold in one (observable) state they have to hold in all successor states,
while class invariants are one-state, which means they are required to hold in all (observable) states.
\footnote{This should have been explained somewhere earlier.}

\vspace{0.5cm}
We now compare with the specification given in \cite{OOPSLA22}.
\prg{DAOSpec1} in similar to  $S_{dao\_1}$: iy
says that no participant's balance may ever exceed the ether remaining 
in DAO. It is, essentially, a one-state invariant.


\begin{lstlisting}[language = Chainmail, mathescape=true, frame=lines]
DAOSpec1 $\triangleq$ from d : DAO $\wedge$ p : Object
            to d.balance(p) > d.ether
            onlyIf false
\end{lstlisting}
\prg{DAOSpec1}, similarly to $S_{dao\_1}$,   in that it enforces a class invariant of \prg{DAO}, something that could be enforced
by traditional specifications using class invariants.


 \cite{OOPSLA22}  gives one more   specification: 
 
 \begin{lstlisting}[language = Chainmail, mathescape=true, frame=lines]
DAOSpec2 $\triangleq$ from d : DAO $\wedge$ p : Object
            next d.balance(p) = m
            onlyIf $\calls{\prg{p}}{\prg{d}}{\prg{repay}}{\prg{\_}}$ $\wedge$ m = 0 $\vee$ $\calls{\prg{p}}{\prg{d}}{\prg{join}}{\prg{m}}$ $\vee$ d.balance(p) = m
\end{lstlisting}

 \prg{DAOSpec2} states that if after some single step of execution, a participant's balance is \prg{m}, then 
either 
\begin{description}
\item[(a)] this occurred as a result of joining the DAO with an initial investment of \prg{m}, 
\item[(b)] the balance is \prg{0} and they've just withdrawn their funds, or 
\item[(c) ]the balance was \prg{m} to begin with
\end{description}

%\subsection{Safe}
%\cite{FASE} used as a running example   a Safe, where a treasure 
%was secured within a \texttt{Safe} object, and access to the treasure was only granted by 
%providing the correct password. 
%\red{Sophia proposes that we drop the Safe}
%\ Using \Nec, we express \texttt{SafeSpec}, that requires that the treasure cannot be 
%removed from the safe without knowledge of the secret.
%\begin{lstlisting}[language = Chainmail, mathescape=true, frame=lines]
%SafeSpec $\triangleq$ from s : Safe $\wedge$ s.treasure != null
%            to s.treasure == null
%            onlyIf $\neg$ inside(s.secret)
%\end{lstlisting}
%
%The module  \prg{SafeModule} described  below satisfies  \prg{SafeSpec}.
%
%\begin{lstlisting}[frame=lines]
%module SafeModule
%     class Secret{}
%     class Treasure{}
%     class Safe{
%         field treasure : Treasure
%         field secret : Secret
%         method take(scr : Secret){
%              if (this.secret==scr) then {
%                   t=treasure
%                   this.treasure = null
%                   return t } 
%          }
% }
%\end{lstlisting}

\subsection{ERC20}

The ERC20 \cite{ERC20} is a widely used token standard describing the basic functionality of any Ethereum-based token 
contract. 
This functionality includes issuing tokens, keeping track of tokens belonging to participants, and the 
transfer of tokens between participants. Tokens may only be transferred if there are sufficient tokens in the 
participant's account, and if either they (using the \prg{transfer} method) or someone authorised by the participant (using the \prg{transferFrom} method) initiated the transfer. 

For an $e:\prg{ERC20}$, the term $e.balance(p)$  indicates the number of tokens in   participant $p$'s  account at $e$.
The 
assertion $e.allowed(p,p')$ expresses that participant $p$ has been authorised to spend moneys from $p'$'s account at $e$.
 
The security model in Solidity is not based on having access to a capability, but on who the caller of a method is. 
Namely, Solidity supports the  construct \prg{sender} which indicates the identity of the caller.
Therefore, for Solidity, we adapt our approach in two significant ways:
we change the meaning of $\inside{\re}$ to express that $\re$ did not make a method call.
Moreover, we introduce a new, slightly modified form of two state invariants of the form $\TwoStates{\overline {x:C}}{A}{A'}$ which expresses that any execution which satisfies $A$, will preserve $A'$.
% \footnote{\red{I think that something deeper is going on here, I think that SOLIDITY id more in the style of ACLs, than OCAPs.}}

% \footnote{\red{AUTHORS: the way we had written the spec of ECR20 at OOPSLA, the tokens are taken out of the ERC20 and just disappear rather than being transferred to a another account. Was that correct?{ It would be easy to change}}}

We specify the guarantees of   ERC20  as follows:
\\
ç
\\
$\strut \SPSP  S_{erc\_2}\ \  \triangleq \ \ \TwoStatesLB{ e:\prg{ERC20},p,p':\prg{Participant},n:\mathbb{N}} 
 {\ \forall p'.[\,(e.allowed(p',p) \rightarrow   \inside{p'}\, ] \ } { \ e.balance(b)=n \ } $ 
\\
$\strut \SPSP  S_{erc\_3}\ \  \triangleq \ \ \TwoStatesLB{ e:\prg{ERC20},p,p':\prg{Participant}}  {\ \forall p'.[\,(e.allowed(p',p) \rightarrow   \inside{p'}\, ] \ } { \ \neg (e.allowed(p'',p) \ } $ 

The specifications above say the following:
\\
\begin{tabular}{ll}
\begin{minipage}{.10\textwidth}
$\strut \SPSP  S_{erc\_1}$
\end{minipage}
&
\begin{minipage}{.85\textwidth}
guarantees that the the owner of an account is always authorized on that account -- this specification is expressed using the original version of two-state invariants.
\end{minipage}
\\
\\
\begin{minipage}{.10\textwidth}
$\strut \SPSP  S_{erc\_2}$ 
\end{minipage}
&
\begin{minipage}{.85\textwidth}
guarantees that any execution which does not contain calls from a participant $p'$ authorized on $p$'s account will not affect the balance of $e$'s account. Namely, if the execution starts in a state in which $ e.balance(b)=n$, it will lead to a state where $ e.balance(b)=n$ also holds.
\end{minipage}
\\
\\
\begin{minipage}{.10\textwidth}
$\strut \SPSP  S_{erc\_3}$ 
\end{minipage}
&
\begin{minipage}{.85\textwidth}
guarantees that any execution which does not contain calls from a participant $p'$ authorized on $p$'s account will not affect who else is authorized on that account. That is, if the execution starts in a state in which $ \neg (e.allowed(p'',p)$, it will lead to a state where $ \neg (e.allowed(p'',p)$ also holds.
\end{minipage}
\end{tabular}

%$\strut \SPSP  S_{erc\_1}$ guarantees that the the owner of an account is always authorized on that account -- this specification is expressed using the original version of two-state invariants.
%\\
%$\strut \SPSP  S_{erc\_2}$ guarantees that any execution which does not contain calls from a participant $p'$ authorized on $p$'s account will not affect the balance of $e$'s account. Namely, if the execution starts in a state in which $ e.balance(b)=n$, it will lead to a state where $ e.balance(b)=n$ also holds.
%\\
%$\strut \SPSP  S_{erc\_3}$ guarantees that any execution which does not contain calls from a participant $p'$ authorized on $p$'s account will not affect the balance of $e$'s account. That is, f the execution starts in a state in which $ \neg (e.allowed(p'',p)$, it will lead to a state where $ \neg (e.allowed(p'',p)$ also holds.

\vspace{1cm}

We compare with the specifications given in \cite{OOPSLA22}:
 Firstly, \prg{ERC20Spec1} 
says that if the balance of a participant's account is ever reduced by some amount $m$, then
that must have occurred as a result of a call to the \prg{transfer} method with amount $m$ by the participant,
or the \prg{transferFrom} method with the amount $m$ by some other participant.
\begin{lstlisting}[language = Chainmail, mathescape=true, frame=lines]
ERC20Spec1 $\triangleq$ from e : ERC20 $\wedge$ e.balance(p) = m + m' $\wedge$ m > 0
              next e.balance(p) = m'
              onlyIf $\exists$ p' p''.[$\calls{\prg{p'}}{\prg{e}}{\prg{transfer}}{\prg{p, m}}$ $\vee$ 
                     e.allowed(p, p'') $\geq$ m $\wedge$ $\calls{\prg{p''}}{\prg{e}}{\prg{transferFrom}}{\prg{p', m}}$]
\end{lstlisting}
Secondly, \prg{ERC20Spec2} specifies under what circumstances some participant \prg{p'} is authorized to 
spend \prg{m} tokens on behalf of \prg{p}: either \prg{p} approved \prg{p'}, \prg{p'} was previously authorized,
or \prg{p'} was authorized for some amount \prg{m + m'}, and spent \prg{m'}.
\begin{lstlisting}[language = Chainmail, mathescape=true, frame=lines]
ERC20Spec2 $\triangleq$ from e : ERC20 $\wedge$ p : Object $\wedge$ p' : Object $\wedge$ m : Nat
              next e.allowed(p, p') = m
              onlyIf $\calls{\prg{p}}{\prg{e}}{\prg{approve}}{\prg{p', m}}$ $\vee$ 
                     (e.allowed(p, p') = m $\wedge$ 
                      $\neg$ ($\calls{\prg{p'}}{\prg{e}}{\prg{transferFrom}}{\prg{p, \_}}$ $\vee$ 
                              $\calls{\prg{p}}{\prg{e}}{\prg{allowed}}{\prg{p, \_}}$)) $\vee$
                     $\exists$ p''. [e.allowed(p, p') = m + m' $\wedge$ $\calls{\prg{p'}}{\prg{e}}{\prg{transferFrom}}{\prg{p'', m'}}$]
\end{lstlisting}
%\end{example}

\prg{ERC20Spec1} is related to $S_{erc\_2}$. Note that \prg{ERC20Spec1} is more API-specific, as it expresses the precise methods which caused the modification of the balance.

\subsection{Wasm, Iris, and the stack}

%
In \cite{irisWasm23}, they consider inter-language safety for Wasm. They develop Iris-Wasm, a mechanized higher-order separation logic mechanized in Coq and the Iris framework. Using Iris-Wasm, with the aim to
specify and verify individual modules separately, and then compose them modularly in a simple host language
featuring the core operations of the WebAssembly JavaScript Interface. They develop a 
logical relation that enforces robust safety: unknown, adversarial code can only affect other modules through
the functions that they explicitly export. 
They do not offer however a logic to deal with the effects of external calls.

As a running example, they use a \prg{stack} module, which is an array of values, and exports functions to inspect the stack contents or modify its contents. 
Such a setting can be expressed in our language through a \prg{stack} and a \prg{modifier} capability.
Assuming a predicate $Contents(\prg{stack},\prg{i},\prg{v})$, which expresses that the contents of \prg{stack} at index \prg{i} is \prg{v}, we can specify the stack through
 
 $$\strut \SPSP  S_{stack}\ \  \triangleq \ \ \TwoStatesN{ s:\prg{Stack},i:\mathbb{N},\prg{v}:\prg{Value}} 
 {\ \inside{\prg{s.modifier}} \ \wedge \ Contents(\prg{s},i,\prg{v})\  }$$  
 
 
 
 In that work, they provide a tailor-made proof that indeed, when the stack makes an external call, passing only the inspect-capability, the contents will not change. 
 However, because the language is essentially functional, they do not consider the possibility that the external call might already have stored the modifier capability.
 Moreover, the proof does not make use of a Hoare logic.  
 
 \subsection{Sealer-Unsealer pattern} 
 The sealer-unsealer pattern, proposed by \citet{JamesMorris}, is a security  pattern  to enforce data
abstraction while interoperating with untrusted  code. He proposes a function
\prg{makeseal} which generating pairs of functions (\prg{seal}, \prg{unseal} ), such that \prg{seal} takes a value $v$ and returns a low-integrity value $v'$.
The function \prg{unseal} when given $v'$ will return $v$. But there is no other way to obtain $v$ out of $v'$ except throughthe use of the \prg{usealer}.
Thus, $v'$ can securely be shared with untrusted code.
This pattern has been studied by \citet{ddd}.

We formulate this pattern here. 
As we are working with an object oriented rather than a functional language, we assume the existence of a class 
\prg{DynamicSealer} with two methods, \prg{seal}, and \prg{unseal}. 
And we define a predicate $Sealed(v,v',us)$ to express that $v$ has been sealed into $v'$ and can be unsealed using $us$.

Then, the scoped invariants 

 $$\strut \SPSP  S_{sealer\_1}\ \  \triangleq \ \ \TwoStatesN{ \prg{v},\prg{v}',\prg{us}: \prg{Object} }
 {\   \inside{\prg{us}} \ \wedge\ Sealed(\prg{v},\prg{v}',\prg{us}) }$$  
 
 $$\strut \SPSP  S_{sealer\_2}\ \  \triangleq \ \ \TwoStatesN{ \prg{v},\prg{v}',\prg{us}: \prg{Object} }
 {\ \inside{\prg{v}} \ \wedge \ \inside{\prg{us}} \ \wedge\ Sealed(\prg{v},\prg{v}',\prg{us}) }$$  

\noindent
 expresses that the unsealer is not leaked to external code ($S_{sealer\_1}$), and that if the external world has no access to the high-integrity value $\prg{v}$ nor to the its unsealer  \prg{us}, then it will not get access to the value  ($S_{sealer\_2}$). 

%
%Moreover, we would add method specifications to the \prg{seal}, and \prg{unseal} methods as follows, where $res.1$ and $res.2$ stand for taking the 1st or 2nd element out of a tuple.
%
%  {\sprepost
%		{\strut \ \ \ \ \ \ \ \ \ S_{sealer\_seal}} 
%		{  v, o:\prg{Object \wedge  \inside{o} }
%		{\prg{public DynamicSealer}} {\prg{seal}} {\prg{v}:\prg{Object}}
%		{   \inside{o}\wedge\ Sealed(v, res.1, res.2) \wedge   \inside{res.2}}
%		{   \inside{o}  \wedge   \inside{res.2} }
%}}
%
% 
%  {\sprepost
%		{\strut \ \ \ \ \ \ \ \ \ S_{sealer\_unseal\_1}} 
%		{ \prg{v'}, o:\prg{Object} \wedge  \inside{o} \wedge\ Sealed(v, \prg{v'}, \prg{us}) \wedge \prg{us}=\preg{us}' }
%		{\prg{public DynamicSealer}} {\prg{unseal}} {\prg{v'}:\prg{Object}, \prg{us'}:\prg{Object}}
%		{ \   Sealed(v, \prg{v'}, \prg{us}) \wedge res=\prg{v} }
%		{   \inside{o} }
%}
%
%
%{\sprepost
%  {\strut \ \ \ \ \ \ \ \ \ S_{sealer\_unseal\_2}}
%                  { \prg{v'}, o:\prg{Object} \wedge  \inside{o} \wedge\ Sealed(v, \prg{v'}, \prg{us}) \wedge \prg{us}\neq\preg{us}' }
%		{\prg{public DynamicSealer}} {\prg{unseal}} {\prg{v'}:\prg{Object}, \prg{us'}:\prg{Object}}
%		{ \   Sealed(v, \prg{v'}, \prg{us}) \wedge res=\preg{v}' }
%		{   \inside{o} }
%}





%\section{Crowdsale}
\jm[]{It is notable that \Nec is able to encode the motivating example of \citeauthor{VerX}: 
an escrow smart contract that ensures that the contract may not be maliciously coerced.
The motivating \prg{Crowdsale} example consists of a \prg{Crowdsale} contract 
for crowd sourcing funding. A \prg{Crowdsale} object consists of an \prg{Escrow} object,
an amount raised, a funding goal, and a closing time in which the goal must be met for 
the fund to be successful. An \prg{Escrow} consists of a ledger of investors and how much
they have invested. There are several properties that \citeauthor{VerX} sought to encode,
and we have provided the encoding of those specifications in Fig. \ref{f:verx:encoding}.
\prg{R0} states that if an investor claims a refund from an escrow, then the balance of 
the escrow decreases by the amount the investor had deposited in the escrow. 
\prg{R1} states that if at anytime the escrow has not yet succeeded, then the deposits must
be less than the balance of the escrow. 
\prg{R2\_1} and \prg{R2\_2} combine to express a single property: no one may ever withdraw and 
then subsequently claim a refund or visa versa.
\prg{R3} states that if the funding goal is ever met, then no one may subsequently claim a refund.}

\begin{figure}[htb]
\begin{lstlisting}[language=chainmail]
contract Crowdsale {
Escrow escrow;
  uint256 closeTime;
  uint256 raised = 0;
  uint256 goal = 10000 * 10**18;
  function constructor() {
    escrow = new Escrow(0x1234);
    closeTime = now + 30 days;
  }
  function invest() payable {
    require(raised < goal);
    // fix: uncomment pre-condition below:
    // require(now<=closeTime);
    escrow.deposit.value(msg.value)(msg.sender);
    raised += msg.value;
  }
  function close() {
    require(now > closeTime || raised >= goal);
    if (raised >= goal) {
      escrow.close();
    } else {
      escrow.refund();
    }
  }
}
\end{lstlisting}
\caption{Crowdsale Contract}
\label{f:verx:crowdsale}
\end{figure}

\begin{figure}[htb]
\begin{lstlisting}[language=chainmail]
contract Escrow {
  address owner, beneficiary;
  mapping(address => uint256) deposits;
  enum State {OPEN, SUCCESS, REFUND}
  State state = OPEN;
  constructor(address b) {
    owner = msg.sender;
    beneficiary = b;
  }
  modifier onlyOwner {
    require(msg.sender == owner);
  }
  function close() onlyOwner {state = SUCCESS;}
  function refund() onlyOwner {state = REFUND;}
  function deposit(address p) onlyOwner payable {
    deposits[p] = deposits[p] + msg.value;
  }
  function withdraw() {
    require(state == SUCCESS);
    beneficiary.transfer(this.balance);
  }
  function claimRefund(address p) {
    require(state == REFUND);
    uint256 amount = deposits[p];
    deposits[p] = 0;
    p.call.value(amount)();
  }
}
\end{lstlisting}
\caption{Escrow Contract}
\label{f:verx:escrow}
\end{figure}

\begin{figure}[htb]
\begin{lstlisting}[mathescape=true, language=chainmail]
(R0) $\triangleq$ e : Escrow $\wedge$ $\calls{\_}{\prg{e}}{\prg{claimRefund}}{\prg{p}}$
          next e.balance = nextBal onlyIf nextBal = e.balance - e.deposits(p)
(R1) $\triangleq$ e : Escrow $\wedge$ e.state $\neq$ SUCCESS $\longrightarrow$ sum(deposits) $\leq$ e.balance
(R2_1) $\triangleq$ e : Escrow $\wedge$ $\calls{\_}{\prg{e}}{\prg{withdraw}}{\prg{\_}}$
           to $\calls{\_}{\prg{e}}{\prg{claimRefund}}{\prg{\_}}$ onlyIf false
(R2_2) $\triangleq$ e : Escrow $\wedge$ $\calls{\_}{\prg{e}}{\prg{claimRefund}}{\prg{\_}}$
           to $\calls{\_}{\prg{e}}{\prg{withdraw}}{\prg{\_}}$ onlyIf false
(R3) $\triangleq$ c : Crowdsale $\wedge$ sum(deposits) $\geq$ c.escrow.goal
         to $\calls{\_}{\prg{c.escrow}}{\prg{claimRefund}}{\prg{\_}}$ onlyIf false
\end{lstlisting}
\caption{Encoding VerX Crowdsale Example in Necessity}
\label{f:verx:encoding}
\end{figure}

%\clearpage
%\section{Appendix to Section \ref{sect:proofSystem} } \label{app:proof}
\label{app:hoare}


\subsection{Preliminaries: Specification Lookup,  Renamings, Underlying Hoare Logic}

Definition \ref{d:promises} is broken down as follows:  $S_1 \txtin  S_2$ says that $S_1$ is textually included in $S_2$; \ \ $S \thicksim S'$ says that $S$ is a safe renaming of $S'$; \ \   $\promises M S$ says that $S$ is a safe renaming of one of the specifications given for $M$. 
 
In particular, a safe renaming of  ${ \TwoStatesN {\overline {x:C}} {A}  }$ can replace any of the variables $\overline x$.  
A safe renaming  of ${\mprepostN{A_1}{p\ D}{m}{y}{D}{A_2} {A_3}}$  can replace  the formal parameters ($\overline y$) by actual parameters  ($\overline {y'}$) but requires the actual parameters  not to include \prg{this}, or \prg{res}, (\ie $\prg{this}, \prg{res}\notin \overline{y'}$). -- %The latter requirement 
Moreover, it can replace  the free variables which do not overlap with the formal parameters or the receiver ( $\overline{x}=\fv(A_1)\setminus\{{\overline y},\prg{this}\}$).

\begin{definition}
For a module $M$ and a specification $S$, we define:
\label{d:promises}
\begin{itemize}
\item
$S_1 \txtin  S_2  \ \ \ \triangleq\ \ \   S_1 \txteq  S_2$, or  $S_2 \txteq  S_1 \wedge S_3$, or $S_2\txteq S_3 \wedge S_1$,  or   $S_2 \txteq S_3 \wedge  S_1 \wedge S_4$ for some $S_3$, $S_4$.
\item
$S  \thicksim  S'$ \ \ \  is defined by cases

\begin{itemize}
\item
$ { \TwoStatesN {\overline {x:C}} {A}  }   \thicksim  { \TwoStatesN {\overline {x':C}} {A'[\overline{x'/x}]} } $
\item
$ {\mprepostN{A_1}{p\ D}{m}{y}{D}{A_2} {A_3}}  \thicksim
 {\mprepostN{A_1'}{p\ D}{m}{y'}{D}{A_2'} {A_3'}} $
  \\
 $\strut \hspace{2cm}    \ \  \triangleq\ \ \  A_1 = A_1'[\overline{y/y'}][\overline{x/x'}], \ \ A_2 = A_2'[\overline{y/y'}][\overline{x/x'}], \ \ A_3 = A_3'[\overline{y/y'}][\overline{x/x'}], \ \ \wedge$\\
 $\strut \hspace{2cm}\ \ \ \ \ \ \ \ \ \prg{this}, \prg{res}\notin \overline{y'}, \ \ \overline{x}=\fv(A_1)\setminus\{{\overline y},\prg{this} \} $
  \end{itemize} 
  
 \item  
 $\promises M  S \ \ \ \triangleq
 \ \ \ \exists S'.[ \ \ S'  \txtin \SpecOf M\ \ \wedge\ \ S' \thicksim S \ \ ]$
  \end{itemize} 
  \end{definition} 
  
The restriction on renamings of method specifications that  the actual parameters should not to include \prg{this}  or \prg{res} 
 is necessary because \prg{this} and \prg{res}  denote different objects from the point of the caller  than from the point of the callee.
It means that we are not able to verify a method call whose actual parameters include \prg{this} or \prg{res}. 
This is not a serious restriction: we can encode any such method call by preceding it with assignments to fresh local variables, \prg{this'}:=\prg{this}, and   \prg{res'}:=\prg{res}, and using \prg{this'} and \prg{res'} in the call.

 

\begin{example}
\label{e:rename}
The specification  from  Example \ref{example:mprepostl} can be renamed as 

\small
{
   {\sprepost
		{\strut \ \ \ \ \ \ \ \ \ S_{9r}} 
		{  a1:\prg{Account}, a2: \prg{Account}\wedge  \inside{a1}\wedge  \inside{a2.\prg{key}} }
		{\prg{public Account}} {\prg{set}} {\prg{nKey}:\prg{Key}}
		{   \inside{a1}\wedge  \inside{a2.\prg{key}} }
		{{   \inside{a1}\wedge  \inside{a2.\prg{key}} }}		
}}

\end{example}

\begin{axiom}
\label{ax:ul}
{Assume   Hoare logic with judgements % of the form 
\ $M \vdash_{ul} \{ A \} stmt \{ A' \}$, 
with  $\Stable{A}$,  $\Stable{A'}$. }
\end{axiom}

\subsection{Types}
\label{types}

The rules in Fig. \ref{f:types} allow triples to talk about the types 
Rule {\sc{types-1}} promises that types of local variables do not change.
Rule {\sc{types-2}} generalizes {\sc{types-1}} to any statement, provided that  there already exists a triple for that statement.

\begin{figure}[tht]
$
\begin{array}{c}
 \begin{array}{lcl}
\inferrule[\sc{types-1}]
	{  stmt \ \mbox{contains no method call} \\
	stmt  \ \mbox{contains   no assignment to $x$}}
	{\hproves{M}  {x:C} {\ stmt\ }{x:C} }
\\
\\
% THIS RULE IS NOT SOUND -- we need to adapt it
%{
\inferrule[\sc{types-2}]
	{ \hprovesN{M}  {A} {\ s\ }  {A'} {A''}  }
	{\hprovesN{M}  {x:C \wedge A} {\ s\ }  {x:C\wedge A'} {A''}}
% \end{array}
\end{array}
\end{array}
 $
\caption{Types}
\label{f:types}
\end{figure}


In {\sc{types-1}} we restricted to statements which do not contain method calls  in order to make lemma   \ref{l:no:meth:calls}  valid.

 
\subsection{Second Phase - more}

in Fig. \ref{f:substructural:app}, we    extend the Hoare Quadruples Logic with substructural rules, rules for conditionals, case analysis, and a contradiction rule.
For the conditionals we assume the obvious operational. semantics, but do not define it in this paper

\begin{figure}[htb]
$
\begin{array}{c}
\begin{array}{lcl}
\inferruleSD{[\sc{combine}]}
	{  \begin{array}{l}
	\hprovesN{M}  {A_1} {\ s\ } {A_2}  {A} \\ % \hspace{1.4cm}  
	\hprovesN{M}  {A_3} {\ s\ } {A_4} {A}
	\end{array}
	}
	{ \hprovesN{M}  {A_1 \wedge A_3 }{\ s\ } {A_2 \wedge A_4} {A} }
& &
\inferruleSD{[\sc{sequ}]}
	{  \begin{array}{l} 
	\hprovesN{M}  {A_1} {\ s_1\ } {A_2}  {A}  \\ % \hspace{1.4cm} 
	\hprovesN{M}  {A_2} {\ s_2\ } {A_3} {A}
	\end{array}
	}
	{   \hprovesN{M}  {A_1   }{\ s_1; \, s_2\ } {  A_3} {A} }
\end{array}
\\ \\
\inferruleSD{ \hspace{3cm} [\sc{consequ}]}
	{
%	\begin{array}{l}
	 { \hprovesN  {M}  {A_4} {\, s\, } { A_5} {A_6}  }
	 \hspace{1.4cm} 
	 M \vdash A_1 \rightarrow A_4 
	 \hspace{1.4cm} 
	{ M \vdash A_5   \rightarrow  A_2  }
	 \hspace{1.4cm}   
	{  M \vdash A_6 \rightarrow A_3 }
%	\end{array}
	}
	{   \hprovesN{M}  {A_1 }{\ s\ } {A_2} {A_3} }
  \end{array}
 $
 
 $
\begin{array}{c}
\inferruleSD{\hspace{2.5cm} [\sc{If\_Rule}]}
	{
	 \begin{array}{c}
	  \hprovesN {M}   
		{\  A \wedge Cond \  }
		{\    stmt_1   \ }
 		{\ A' \ }
		{\ A'' \ }
	\\
	    \hprovesN {M}   
		{\  A \wedge \neg Cond \  }
		{\    stmt_2   \ }
 		{\ A' \ }
		{\ A'' \ }	
	\end{array}
	}	
 	{  	
	\hprovesN {M}   
		{\  A \  }
		{\  \prg{if}\ Cond\ \prg{then}\ stmt_1\ \prg{else}\ stmt_2 \ \ }
		{\ A' \ }
		{\ A'' \ }
}
\\
\\
\begin{array}{lcl}
{
\inferruleSD{\hspace{0.5cm} [\sc{Absurd}]}
	{	
	}	 
 	{  	
	\hprovesN {M}   
		{\  false \  }
		{\  \ stmt \ \ }
		{\ A' \ }
		{\ A'' \ }
}
} & &
{
\inferruleSD{\hspace{0.5cm} [\sc{Cases}]}
	{ \begin{array}{l}
	\hprovesN {M}   
		{\  A \wedge A_{1}  \  }
		{\  \ stmt \ \ }
		{\ A' \ }
		{\ A'' \ }
		\\
		\hprovesN {M}   
		{\   A \wedge A_{2} \  }
		{\  \ stmt \ \ }
		{\ A' \ }
		{\ A'' \ }
	\end{array}	
	}	 
 	{  	
	\hprovesN {M}   
		{\  A \wedge (A_1 \vee A_2) \  }
		{\  \ stmt \ \ }
		{\ A' \ }
		{\ A'' \ }
}
}
\end{array}
\end{array}
$
 \vspace{-.5cm}
\caption{Hoare Quadruples -    substructural rules, and conditionals }
\label{f:substructural:app}
\end{figure}

\subsection{Extend the semantics and Hoare logic to accommodate scalars and conditionals}
\label{s:app:scalars}

{We extend the notion of protection to also allow it to apply to scalars. }

\begin{definition}[Satisfaction  of Assertions  -- Protected From]
\label{def:chainmail-protection-from-ext}
extending the definition of Def 
\ref{def:chainmail-protection-from}. We use $\alpha$ to range over addresses, $\beta$  to range over scalars, and   $\gamma$ to range over addresses or scalars.

\noindent
We define  $\satisfiesA{M}{\sigma}{\protectedFrom{{\gamma}} {{\gamma_{o}}}} $ as:
\begin{enumerate}
\item
\label{cProtectedNew}
 $\satisfiesA{M}{\sigma}{\protectedFrom{{\alpha}} {{\alpha_{o}}}}   \ \ \ \triangleq $ 
  \begin{itemize}
 \item
$\alpha\neq \alpha_0$,
 \ \ \ \  and% \\
 \item
$\forall n\in\mathbb{N}. \forall f_1,...f_n..
[\ \ \interpret{\sigma}{\alpha_{o}.f_1...f_n}=\alpha \ \ \  \Longrightarrow \ \ \  \satisfiesA{M}{\sigma}{ {\interpret{\sigma}{\alpha_{o}.f_1...f_{n-1}}}:C} \ \wedge \ C\in M\ \ ]$
\end{itemize}
\item
 $\satisfiesA{M}{\sigma}{\protectedFrom{{\gamma}} {{\beta_{o}}}}   \ \ \ \triangleq  \ \ \ true$
 \item
 $\satisfiesA{M}{\sigma}{\protectedFrom{{\beta}} {{\alpha_{o}}}}   \ \ \ \triangleq  \ \ \ false$
  \item
$\satisfiesA{M}{\sigma}{\protectedFrom{{\re}} {{\re_{o}}}} \ \ \ \triangleq $ \\
  $\exists \gamma, \gamma_{o}. [\  \ \eval{M}{\sigma}{{\re}}{\gamma}\ \wedge \eval{M}{\sigma}{{\re_0}}{\gamma_0} \  \wedge \ 
  \satisfiesA{M}{\sigma}{\protectedFrom{{\gamma}} {{\gamma_{o}}}}
 \ \  ]$
 \end{enumerate}
 \end{definition} 

{The definition from above gives rise to further cases of  protection; we supplement the triples from 
Fig. \ref{f:protection} with some further inference rules, given   in Fig. \ref{f:protection:conseq:ext}.}


\begin{figure}[htb]
\begin{mathpar}
\inferrule
	{M \vdash x : \prg{int} \rightarrow \protectedFrom{y}{x} }
	{}
	\quad[\textsc{Prot-Int}]
	\and
\inferrule
	{M \vdash x : \prg{bool} \rightarrow \protectedFrom{y}{x} }
	{}
	\quad[\textsc{Prot-Bool}]
	\and
\inferrule
	{M \vdash x : \prg{str} \rightarrow \protectedFrom{y}{x} }
	{}
	\quad[\textsc{Prot-Str}1]
	\and
 {
\inferrule
	{M \vdash  \protectedFrom{e}{e'}   \rightarrow e\neq e'} 
	{}
	\quad[\textsc{Prot-Neq]}
  }
\end{mathpar}
\caption{Protection for Scalar Types}
%  Extended Consequence Rules for Protection that include rules for protection involving scalars.}
\label{f:protection:conseq:ext}
\end{figure}



\begin{lemma}
\label{l:no:meth:calls}
If ${\hproves{M}  {A} {\ stmt\ }{A'} }$, then $stmt$ contains no method calls.
\end{lemma}

\begin{proof}
By induction on the rules in Fig. \ref{f:underly}.

\end{proof}

\subsection{Adaptation}
\label{appendix:adaptation}
 
 \newcommand{\SP}{$\strut \ \ \ \ $}


 We now discuss the proof of Lemma \ref{lemma:push:ass:state}.

 \vspace{0.5cm}
 
 \beginProofSub{lemma:push:ass:state}{l:push:stbl}
$~$ \\
To Show: \ \ \  $\Stable{\,  \PushASLong {(y_0,\overline y)} A\, }$
\\
By structural induction on $A$.\\
\completeProofSub

\vspace{1cm}

For parts \ref{lemma:push:ass:state:one},  \ref{lemma:push:ass:state:two}, and  \ref{lemma:push:ass:state:three}, we first prove the following auxiliary lemma:

\begin{auxLemma}
\label{l:push:pop:aux}
For all $\alpha$,   $\overline {\phi_1}$, $\overline {\phi_2}$, $\overline {\phi_2}$, $\phi$ and $\chi$\\
$\strut ~ \ \ \ \ \ (L1)\ \ \    M, (\overline {\phi_1},\chi) \models \protectedFrom \alpha {Rng(\phi)} \ \Longrightarrow \ M, (\overline {\phi_2}\cdot \phi,\chi) \models \inside \alpha$
\\
$\strut ~ \ \ \ \ \ (L2)\ \ \    M, (\overline {\phi_1}\cdot\phi,\chi) \models \inside \alpha   \wedge \extThis \ \ \Longrightarrow \ \ M, (\overline {\phi_2},\chi) \models \protectedFrom \alpha {Rng(\phi)} $\\
$\strut ~ \ \ \ \ \ (L3)\ \ \    M, (\overline {\phi_1}\cdot \phi_1,\chi) \models \inside \alpha   \wedge \extThis \ \ \wedge Rng(\phi)\subseteq Rng(\phi_1)\ \ \  \Longrightarrow \ \ M, (\overline {\phi_2},\chi) \models \protectedFrom \alpha {Rng(\phi)} $
\\\end{auxLemma}

\begin{proof}
$~$ \\
We first prove (L1): \\
~ \\
We define $\sigma_1 \triangleq (\overline {\phi_1},\chi)$, and  $\sigma_2 \triangleq (\overline {\phi_2}\cdot \phi,\chi) $.\\
The above definitions imply that: \\
\SP (1)\ \ $\forall \alpha',\forall \overline f.[\  \interpret {\sigma_1} {\alpha'.\overline f} =  \interpret {\sigma_2} {\alpha'.\overline f}\ ]$\\
\SP (2)\ \ $\forall \alpha'.[\  \Relevant {\alpha'} {\sigma_1} = \Relevant {\alpha'} {\sigma_2}\ ]$\\
\SP (3)\ \ $\LRelevantO {\sigma_2} = \bigcup_{\alpha'\in Rng(\phi)} \Relevant {\alpha'} {\sigma_2} $.\\
% $\strut ~ \ $\\
We now assume that\\
\SP (4)\ \ $M, \sigma_1 \models \protectedFrom \alpha {Rng(\phi)}$.\\
and want to show that\\
\SP (??)\ \ $M, \sigma_2 \models \inside \alpha$\\
From (4) and  by definitions, we obtain that\\
\SP (5)\ \ $\forall \alpha'\in Rng(\phi).\forall \alpha''\in \Relevant {\alpha'} {\sigma_1}.\forall f.[ \   M, \sigma_1 \models \alpha'':\prg{extl}\ \rightarrow \alpha''.f  \neq \alpha\ ]$, \ \ \ \ and also\\
\SP (6)\ \ $\alpha \notin Rng(\phi)$\\
From (5) and (3) we obtain:\\
\SP (7)\ \  $\forall \alpha' \in \LRelevantO {\sigma_2}.\forall f.[ \   M, \sigma_1 \models \alpha':\prg{extl}\ \rightarrow \alpha'.f  \neq \alpha\ ]$\\
From (7) and (1) and (2) we obtain:\\
\SP (8) \ \  $\forall \alpha' \in \LRelevantO {\sigma_2}.\forall f.[ \   M, \sigma_2 \models \alpha':\prg{extl}\ \rightarrow \alpha'.f  \neq \alpha\ ]$\\
From (8), by definitions, we obtain\\
 \SP (10)\ \ $M, \sigma_2 \models \inside \alpha$\\
 This completes the proof of (L1).
 \\
  $\strut ~ \ $\\
  We now prove (L2): \\
  ~ \\
 We define $\sigma_1 \triangleq (\overline {\phi_1}\cdot \phi,\chi)$, and  $\sigma_2 \triangleq (\overline {\phi_2},\chi) $.\\
The above definitions imply that: \\
\SP (1)\ \ $\forall \alpha',\forall \overline f.[\  \interpret {\sigma_1} {\alpha'.\overline f} =  \interpret {\sigma_2} {\alpha'.\overline f}\ ]$\\
\SP (2)\ \ $\forall \alpha'.[\  \Relevant {\alpha'} {\sigma_1} = \Relevant {\alpha'} {\sigma_2}\ ]$\\
\SP (3)\ \ $\LRelevantO {\sigma_1} = \bigcup_{\alpha'\in Rng(\phi)} \Relevant {\alpha'} {\sigma_1}$.\\
We   assume that\\
\SP (4)\ \  $M, \sigma_1 \models \inside \alpha \wedge \extThis$.\\
and want to show that\\
\SP (A?)\ \ $M, \sigma_2 \models \PushASLong  {Rng(\phi)} {A}$.\\
From (4), and unfolding the definitions, we obtain:\\
\SP (5)\ \  $\forall \alpha'\in \LRelevantO {\sigma_1}.\forall f.[ \   M, \sigma_1 \models \alpha':\prg{extl}\ \rightarrow \alpha'.f  \neq \alpha\ ]$, \ \ \ and\\
\SP (6)\ \ $\forall \alpha'\in Rng (\phi). [ \ \alpha'\neq \alpha \ ]$.\\
From(5), and using (3) and (2) we obtain:
\\
\SP (7)\ \  $\forall \alpha'\in Rng(\phi).\forall \alpha'' \in\Relevant {\alpha'} {\sigma_2}.\forall f.[ \   M, \sigma_2 \models \alpha'':\prg{extl}\ \rightarrow \alpha''.f  \neq \alpha\ ]$\\
From (5) and (7) and by definitions, we obtain
\\
\SP (8)\ \  $\forall \alpha'\in Rng (\phi).[ \   \models \alpha \protectedFrom \alpha {\alpha'}\ ]$.\\
From (8) and definitions we obtain (A?).\\
This completes the proof of (L2). 
 \\
  $\strut ~ \ $\\
  We now prove (L3): \\
  ~ \\
 We define $\sigma_1 \triangleq (\overline {\phi_1}\cdot \phi_1,\chi)$, and  $\sigma_2 \triangleq (\overline {\phi_2},\chi) $.\\
The above definitions imply that: \\
\SP (1)\ \ $\forall \alpha',\forall \overline f.[\  \interpret {\sigma_1} {\alpha'.\overline f} =  \interpret {\sigma_2} {\alpha'.\overline f}\ ]$\\
\SP (2)\ \ $\forall \alpha'.[\  \Relevant {\alpha'} {\sigma_1} = \Relevant {\alpha'} {\sigma_2}\ ]$\\
\SP (3)\ \ $\LRelevantO {\sigma_1} = \bigcup_{\alpha'\in Rng(\phi_1)} \Relevant {\alpha'} {\sigma_1}$.\\
We   assume that\\
\SP (4a)\ \  $M, \sigma_1 \models \inside \alpha \wedge \extThis$, and
\SP (4b)\ \ $Rng(\phi) \subseteq Rng(\phi_1)$\\
We  want to show that\\
\SP (A?)\ \ $M, \sigma_2 \models \PushASLong  {Rng(\phi)} {A}$.\\
From (4a), and unfolding the definitions, we obtain:\\
\SP (5)\ \  $\forall \alpha'\in \LRelevantO {\sigma_1}.\forall f.[ \   M, \sigma_1 \models \alpha':\prg{extl}\ \rightarrow \alpha'.f  \neq \alpha\ ]$, \ \ \ and\\
\SP (6)\ \ $\forall \alpha'\in Rng (\phi_1). [ \ \alpha'\neq \alpha \ ]$.\\
From(5), and   (3) and (2) and (4b) we obtain:
\\
\SP (7)\ \  $\forall \alpha'\in Rng(\phi).\forall \alpha'' \in\Relevant {\alpha'} {\sigma_2}.\forall f.[ \   M, \sigma_2 \models \alpha'':\prg{extl}\ \rightarrow \alpha''.f  \neq \alpha\ ]$ \\
From(6), and   (4b) we obtain:
\\
\SP (8)\ \ $\forall \alpha'\in Rng (\phi_1). [ \ \alpha'\neq \alpha \ ]$.\\
From (8) and definitions we obtain (A?).\\
This completes the proof of (L3). 

\end{proof}


\beginProofSub{lemma:push:ass:state}{lemma:push:ass:state:one}
$~$ \\
To Show: \ \ \  $(*)\ \ \ M, \sigma \models \PushASLong  {Rng(\phi)} {A}\  \ \ \ \ \  \ \ \    \Longrightarrow  \ \ \ \ M,  \PushSLong {\phi} {\sigma}   \models A$
\\ $~$ \\
 By  induction on the structure of $A$. For the case where $A$ has the form $\inside {\alpha.\overline f}$, we use lemma \ref{l:push:pop:aux},(L1), taking $\overline {\phi_1} = \overline { \phi_2}$, and $\sigma \triangleq (\overline {\phi_1},\chi).$
\\
\completeProofSub
 

\vspace{1cm}

\beginProofSub{lemma:push:ass:state}{lemma:push:ass:state:two}
$~$ \\
To Show \ \ \  $(*)\ \ \  M,  \PushSLong {\phi} {\sigma}   \models  A  \wedge \extThis    \ \  \ \  \Longrightarrow  \ \ \ \ M, \sigma \models \PushASLong  {Rng(\phi)} {A}$ 
\\
$~$ \\
We apply induction on the structure of $A$. For the case where $A$ has the form $\inside {\alpha.\overline f}$, we apply lemma \ref{l:push:pop:aux},(L2), using    $\overline {\phi_1} = \overline { \phi_2}$, and $\sigma \triangleq (\overline {\phi_1},\chi).$

\completeProofSub

\vspace{1cm}
\beginProofSub{lemma:push:ass:state}{lemma:push:ass:state:three}
$~$ \\
To Show:\ \ \   (*) \ \  $M, \sigma  \models  A  \wedge \extThis  \ \wedge \ M\cdot\Mtwo \models \PushSLong {\phi} {\sigma}   \ \  \ \ \  \  \Longrightarrow  \ \ \ \ M, \PushSLong {\phi} {\sigma} \models \PushASLong  {Rng(\phi)} {A}$
\\ 
$~$ \\
By induction on the structure of $A$. 
 For the case where $A$ has the form $\inside {\alpha.\overline f}$, we want to apply lemma \ref{l:push:pop:aux},(L3). We take  $\sigma$ to be $ (\overline {\phi_1}\cdot\phi_1, \chi)$, and $\overline {\phi_2}=\overline {\phi_1}\cdot\phi_1\cdot \phi$. Moreover,  $M\cdot\Mtwo \models \PushSLong {\phi} {\sigma}$ gives  that $Rng(\phi)\subseteq \LRelevantO {\sigma_2}$. Therefore, (*) follows by application of lemma \ref{l:push:pop:aux},(L3).\\
\completeProofSub
 %
%
%The rules {\sc{Call\_Int}} and {\sc{Call\_Int\_Adapt}}   are weaker forms of  the  rule {\sc{Call\_Int\_Combine}} given here. 
%Even though  {\sc{Call\_Int\_Combine}} is sound, we did not include it in the presented system, for the sake of simplicity and separation of concerns.
%A similar stronger rule can be expressed for {\sc{Call\_Ext}} and  for {\sc{Call\_Ext\_Adapt}}.
% \\
%$
%{\begin{array}{c}
%  \inferruleSD  {\hspace{4.7cm} [\sc{Call\_Int\_Combine}]}
% 	{
% 	  	\begin{array}{c}
% 		  \promises  M { \mprepostN  {A_{1a} \wedge A_{1r} }  {p\ C} {m} {y} {C} { A_{2a} \wedge A_{2r}}   {A_3}  }
%		\\
%		A_1'\txteq    A_{1a}[y_0/\prg{this}]  \wedge {\PushAS {y}{A_{1r}[y_0/\prg{this}]}}
%		\\
%		A_2'\txteq    A_{2a}[u/res,y_0/\prg{this}]  \wedge {\PushAS {y}{A_{2r}[u/res,y_0/\prg{this}]}}\
%       	\end{array}
% 		}
% 	{  \hprovesN {M} 
%						{ \  y_0:C, {\overline {y:C}} \wedge A_1' \ }  
%						{ \ u:=y_0.m(y_1,.. y_n)\    }
%						{ \ A_2'\ }
%						{  \  A_3 \ }	
%      }
%\end{array}
%}



 


\begin{lemma}
\label{l:no:meth:calls}
If ${\hproves{M}  {A} {\ stmt\ }{A'} }$, then $stmt$ contains no method calls.
\end{lemma}

\begin{proof}
By induction on the rules in Fig. \ref{f:underly}.

\end{proof}

\subsection{Adaptation}
\label{appendix:adaptation}
 
 \newcommand{\SP}{$\strut \ \ \ \ $}


 We now discuss the proof of Lemma \ref{lemma:push:ass:state}.

 \vspace{0.5cm}
 
 \beginProofSub{lemma:push:ass:state}{l:push:stbl}
$~$ \\
To Show: \ \ \  $\Stable{\,  \PushASLong {(y_0,\overline y)} A\, }$
\\
By structural induction on $A$.\\
\completeProofSub

\vspace{1cm}

For parts \ref{lemma:push:ass:state:one},  \ref{lemma:push:ass:state:two}, and  \ref{lemma:push:ass:state:three}, we first prove the following auxiliary lemma:

\begin{auxLemma}
\label{l:push:pop:aux}
For all $\alpha$,   $\overline {\phi_1}$, $\overline {\phi_2}$, $\overline {\phi_2}$, $\phi$ and $\chi$\\
$\strut ~ \ \ \ \ \ (L1)\ \ \    M, (\overline {\phi_1},\chi) \models \protectedFrom \alpha {Rng(\phi)} \ \Longrightarrow \ M, (\overline {\phi_2}\cdot \phi,\chi) \models \inside \alpha$
\\
$\strut ~ \ \ \ \ \ (L2)\ \ \    M, (\overline {\phi_1}\cdot\phi,\chi) \models \inside \alpha   \wedge \extThis \ \ \Longrightarrow \ \ M, (\overline {\phi_2},\chi) \models \protectedFrom \alpha {Rng(\phi)} $\\
$\strut ~ \ \ \ \ \ (L3)\ \ \    M, (\overline {\phi_1}\cdot \phi_1,\chi) \models \inside \alpha   \wedge \extThis \ \ \wedge Rng(\phi)\subseteq Rng(\phi_1)\ \ \  \Longrightarrow \ \ M, (\overline {\phi_2},\chi) \models \protectedFrom \alpha {Rng(\phi)} $
\\\end{auxLemma}

\begin{proof}
$~$ \\
We first prove (L1): \\
~ \\
We define $\sigma_1 \triangleq (\overline {\phi_1},\chi)$, and  $\sigma_2 \triangleq (\overline {\phi_2}\cdot \phi,\chi) $.\\
The above definitions imply that: \\
\SP (1)\ \ $\forall \alpha',\forall \overline f.[\  \interpret {\sigma_1} {\alpha'.\overline f} =  \interpret {\sigma_2} {\alpha'.\overline f}\ ]$\\
\SP (2)\ \ $\forall \alpha'.[\  \Relevant {\alpha'} {\sigma_1} = \Relevant {\alpha'} {\sigma_2}\ ]$\\
\SP (3)\ \ $\LRelevantO {\sigma_2} = \bigcup_{\alpha'\in Rng(\phi)} \Relevant {\alpha'} {\sigma_2} $.\\
% $\strut ~ \ $\\
We now assume that\\
\SP (4)\ \ $M, \sigma_1 \models \protectedFrom \alpha {Rng(\phi)}$.\\
and want to show that\\
\SP (??)\ \ $M, \sigma_2 \models \inside \alpha$\\
From (4) and  by definitions, we obtain that\\
\SP (5)\ \ $\forall \alpha'\in Rng(\phi).\forall \alpha''\in \Relevant {\alpha'} {\sigma_1}.\forall f.[ \   M, \sigma_1 \models \alpha'':\prg{extl}\ \rightarrow \alpha''.f  \neq \alpha\ ]$, \ \ \ \ and also\\
\SP (6)\ \ $\alpha \notin Rng(\phi)$\\
From (5) and (3) we obtain:\\
\SP (7)\ \  $\forall \alpha' \in \LRelevantO {\sigma_2}.\forall f.[ \   M, \sigma_1 \models \alpha':\prg{extl}\ \rightarrow \alpha'.f  \neq \alpha\ ]$\\
From (7) and (1) and (2) we obtain:\\
\SP (8) \ \  $\forall \alpha' \in \LRelevantO {\sigma_2}.\forall f.[ \   M, \sigma_2 \models \alpha':\prg{extl}\ \rightarrow \alpha'.f  \neq \alpha\ ]$\\
From (8), by definitions, we obtain\\
 \SP (10)\ \ $M, \sigma_2 \models \inside \alpha$\\
 This completes the proof of (L1).
 \\
  $\strut ~ \ $\\
  We now prove (L2): \\
  ~ \\
 We define $\sigma_1 \triangleq (\overline {\phi_1}\cdot \phi,\chi)$, and  $\sigma_2 \triangleq (\overline {\phi_2},\chi) $.\\
The above definitions imply that: \\
\SP (1)\ \ $\forall \alpha',\forall \overline f.[\  \interpret {\sigma_1} {\alpha'.\overline f} =  \interpret {\sigma_2} {\alpha'.\overline f}\ ]$\\
\SP (2)\ \ $\forall \alpha'.[\  \Relevant {\alpha'} {\sigma_1} = \Relevant {\alpha'} {\sigma_2}\ ]$\\
\SP (3)\ \ $\LRelevantO {\sigma_1} = \bigcup_{\alpha'\in Rng(\phi)} \Relevant {\alpha'} {\sigma_1}$.\\
We   assume that\\
\SP (4)\ \  $M, \sigma_1 \models \inside \alpha \wedge \extThis$.\\
and want to show that\\
\SP (A?)\ \ $M, \sigma_2 \models \PushASLong  {Rng(\phi)} {A}$.\\
From (4), and unfolding the definitions, we obtain:\\
\SP (5)\ \  $\forall \alpha'\in \LRelevantO {\sigma_1}.\forall f.[ \   M, \sigma_1 \models \alpha':\prg{extl}\ \rightarrow \alpha'.f  \neq \alpha\ ]$, \ \ \ and\\
\SP (6)\ \ $\forall \alpha'\in Rng (\phi). [ \ \alpha'\neq \alpha \ ]$.\\
From(5), and using (3) and (2) we obtain:
\\
\SP (7)\ \  $\forall \alpha'\in Rng(\phi).\forall \alpha'' \in\Relevant {\alpha'} {\sigma_2}.\forall f.[ \   M, \sigma_2 \models \alpha'':\prg{extl}\ \rightarrow \alpha''.f  \neq \alpha\ ]$\\
From (5) and (7) and by definitions, we obtain
\\
\SP (8)\ \  $\forall \alpha'\in Rng (\phi).[ \   \models \alpha \protectedFrom \alpha {\alpha'}\ ]$.\\
From (8) and definitions we obtain (A?).\\
This completes the proof of (L2). 
 \\
  $\strut ~ \ $\\
  We now prove (L3): \\
  ~ \\
 We define $\sigma_1 \triangleq (\overline {\phi_1}\cdot \phi_1,\chi)$, and  $\sigma_2 \triangleq (\overline {\phi_2},\chi) $.\\
The above definitions imply that: \\
\SP (1)\ \ $\forall \alpha',\forall \overline f.[\  \interpret {\sigma_1} {\alpha'.\overline f} =  \interpret {\sigma_2} {\alpha'.\overline f}\ ]$\\
\SP (2)\ \ $\forall \alpha'.[\  \Relevant {\alpha'} {\sigma_1} = \Relevant {\alpha'} {\sigma_2}\ ]$\\
\SP (3)\ \ $\LRelevantO {\sigma_1} = \bigcup_{\alpha'\in Rng(\phi_1)} \Relevant {\alpha'} {\sigma_1}$.\\
We   assume that\\
\SP (4a)\ \  $M, \sigma_1 \models \inside \alpha \wedge \extThis$, and
\SP (4b)\ \ $Rng(\phi) \subseteq Rng(\phi_1)$\\
We  want to show that\\
\SP (A?)\ \ $M, \sigma_2 \models \PushASLong  {Rng(\phi)} {A}$.\\
From (4a), and unfolding the definitions, we obtain:\\
\SP (5)\ \  $\forall \alpha'\in \LRelevantO {\sigma_1}.\forall f.[ \   M, \sigma_1 \models \alpha':\prg{extl}\ \rightarrow \alpha'.f  \neq \alpha\ ]$, \ \ \ and\\
\SP (6)\ \ $\forall \alpha'\in Rng (\phi_1). [ \ \alpha'\neq \alpha \ ]$.\\
From(5), and   (3) and (2) and (4b) we obtain:
\\
\SP (7)\ \  $\forall \alpha'\in Rng(\phi).\forall \alpha'' \in\Relevant {\alpha'} {\sigma_2}.\forall f.[ \   M, \sigma_2 \models \alpha'':\prg{extl}\ \rightarrow \alpha''.f  \neq \alpha\ ]$ \\
From(6), and   (4b) we obtain:
\\
\SP (8)\ \ $\forall \alpha'\in Rng (\phi_1). [ \ \alpha'\neq \alpha \ ]$.\\
From (8) and definitions we obtain (A?).\\
This completes the proof of (L3). 

\end{proof}


\beginProofSub{lemma:push:ass:state}{lemma:push:ass:state:one}
$~$ \\
To Show: \ \ \  $(*)\ \ \ M, \sigma \models \PushASLong  {Rng(\phi)} {A}\  \ \ \ \ \  \ \ \    \Longrightarrow  \ \ \ \ M,  \PushSLong {\phi} {\sigma}   \models A$
\\ $~$ \\
 By  induction on the structure of $A$. For the case where $A$ has the form $\inside {\alpha.\overline f}$, we use lemma \ref{l:push:pop:aux},(L1), taking $\overline {\phi_1} = \overline { \phi_2}$, and $\sigma \triangleq (\overline {\phi_1},\chi).$
\\
\completeProofSub
 

\vspace{1cm}

\beginProofSub{lemma:push:ass:state}{lemma:push:ass:state:two}
$~$ \\
To Show \ \ \  $(*)\ \ \  M,  \PushSLong {\phi} {\sigma}   \models  A  \wedge \extThis    \ \  \ \  \Longrightarrow  \ \ \ \ M, \sigma \models \PushASLong  {Rng(\phi)} {A}$ 
\\
$~$ \\
We apply induction on the structure of $A$. For the case where $A$ has the form $\inside {\alpha.\overline f}$, we apply lemma \ref{l:push:pop:aux},(L2), using    $\overline {\phi_1} = \overline { \phi_2}$, and $\sigma \triangleq (\overline {\phi_1},\chi).$

\completeProofSub

\vspace{1cm}
\beginProofSub{lemma:push:ass:state}{lemma:push:ass:state:three}
$~$ \\
To Show:\ \ \   (*) \ \  $M, \sigma  \models  A  \wedge \extThis  \ \wedge \ M\cdot\Mtwo \models \PushSLong {\phi} {\sigma}   \ \  \ \ \  \  \Longrightarrow  \ \ \ \ M, \PushSLong {\phi} {\sigma} \models \PushASLong  {Rng(\phi)} {A}$
\\ 
$~$ \\
By induction on the structure of $A$. 
 For the case where $A$ has the form $\inside {\alpha.\overline f}$, we want to apply lemma \ref{l:push:pop:aux},(L3). We take  $\sigma$ to be $ (\overline {\phi_1}\cdot\phi_1, \chi)$, and $\overline {\phi_2}=\overline {\phi_1}\cdot\phi_1\cdot \phi$. Moreover,  $M\cdot\Mtwo \models \PushSLong {\phi} {\sigma}$ gives  that $Rng(\phi)\subseteq \LRelevantO {\sigma_2}$. Therefore, (*) follows by application of lemma \ref{l:push:pop:aux},(L3).\\
\completeProofSub
%\clearpage
%\section{Appendix to Section \ref{sect:sound:proofSystem} -- Soundness of the Hoare Logics}

\subsection{Expectatons}
\begin{axiom}
\label{lemma:axiom:enc:assert:ul}
\label{ax:ul:sound}
We require a sound logic of assertions ($M \vdash A$), and a sound Hoare logic , \ie that for all $M$, $A$, $A'$, $stmt$:
\begin{center}
$M \vdash A   \ \ \ \  \Longrightarrow  \ \ \ \  \forall \sigma.[\ M, \sigma \models A\ ]$.\\
% \end{center}
%\end{axiom}
%
%\noindent
%Moreover, we assume that the  \ie for all $A$, $A'$, $stmt$:\ \ \  
%
%\begin{axiom}
% \begin{center}
{$M\ \vdash_{ul}\  \triple A {stmt} {A'}  \ \ \ \  \Longrightarrow  \ \ \ \ \satisfies  {M} { \triple A {stmt} {A'}}$ }
 \end{center}
\end{axiom}

\subsection{\Scoped satisfaction of assertions}
\label{s:scoped:mean}{

\begin{definition}% [State Restriction, and Multi-level Sartisfaction]
\label{def:restrict}
For a state $\sigma$, and a number $i\in \mathbb{N}$ with $i \leq \DepthSt {\sigma}$,   module $M$, and assertions $A$, $A'$ we define: % $\RestictTo {\sigma} {i}$:
\begin{itemize}
\item
$  \satDAssertFrom M  \sigma k   A  \ \  \ \triangleq \  \ \  
  k\leq  \DepthSt {\sigma} \ \wedge \  \forall i\!\in\![k...\DepthSt {\sigma}].[\ M,{\RestictTo {\sigma}{i}} \models A[\overline{ {{\interpret \sigma z}/ z}}]\ ] \ \  \mbox{where} \ \
  \overline z=\fv(A).$ 
\end{itemize}
\end{definition}
 
 Remember the definition of  $\RestictTo  \sigma k$, which returns a new state whose top frame is the $k$-th frame from $\sigma$. Namely, $\RestictTo {(\phi_1...\phi_i...\phi_n,\chi)} {i}\ \ \ \triangleq \ \ \ (\phi_1...\phi_i,\chi)$
  
 
\begin{lemma}
\label{l:shallow:scoped}
For a states $\sigma$, $\sigma'$, numbers $k,k'\in \mathbb{N}$, assertions  $A$, $A'$, frame $\phi$ and variables $\overline z$, $\overline u$:
\begin{enumerate}
\item
$ \satDAssertFrom M  \sigma { \DepthSt \sigma}   A \ \ \Longleftrightarrow\ \ M,\sigma \models A\ $
\item
$ \satDAssertFrom M  \sigma {k} A \ \wedge\  k\leq k'\  \  \   \Longrightarrow \ \ \satDAssertFrom M  \sigma {k'} A$ 
\item 
\label{shallow:to:scoped}
$ M,\sigma \models A \ \wedge\ \Stable A \  \ \Longrightarrow \  \  \forall k  \leq  \DepthSt \sigma.[ \ \satDAssertFrom M  \sigma k   A \ ]$
\item
\label{fourSD}
$ M  \models A \rightarrow A'\  \  \   \Longrightarrow \ \ \forall \sigma. \forall k\leq  { \DepthSt \sigma}.[ \ \satDAssertFrom M  \sigma {k} A
\ \Longrightarrow \  \satDAssertFrom M  \sigma {k} A'\ ]$
\item
\label{fiveSD}
$\Pos A \ \wedge\ \fv(A)=\overline z\ \wedge\ \sigma'=\sigma  \pushSymbol \phi\ \wedge\ \ M, \sigma' \models \intThis \ \wedge\   \overline {\interpret \sigma z= {\interpret {\sigma'} u}}\ \ \Longrightarrow$ \\
$\strut \hspace{4cm} [\ \satDAssertFrom M  \sigma {k} A  
\ \ \Longleftrightarrow \ \ \satDAssertFrom M  {\sigma'} {k} {A[{\overline{u/z}}]}\ ]$
\end{enumerate}
\end{lemma}
 
 



\noindent
\vspace{.1cm}
{\textbf{Proof Sketch}} 

\begin{enumerate}
\item
By unfolding and folding the definitions.
\item
By unfolding and folding the definitions.
\item
By induction on the definition of $\Stable {\_}$.
\item
By contradiction: Find a $\sigma$, a $k$ and   such that  
$\forall i\geq k.[M,{\RestictTo {\sigma}{i}} \models A[\overline{ {{\interpret \sigma z}/ z}}]$, and
$\exists j\geq k.[M,{\RestictTo {\sigma}{j}} \not\models A'[[\overline{ {{\interpret \sigma z}/ z}}]$
 such that $\overline z = \fv(A)$.
 Take $\sigma''\triangleq  \RestictTo {\sigma}{j}$, and then we have that
 $M, \sigma'' \models A[[\overline{ {{\interpret \sigma z}/ z}}]$ and  $M,  \sigma'' \not\models A'[[\overline{ {{\interpret \sigma z}/ z}}]$.
 This contradicts $ M  \models A \rightarrow A'$.
 Here we are also using the property that $M \models A$  and $u\notin \fv(A)$ implies $M \models A[u/z]$ -- this is needed because we have free variables in $A$ which are not free in $A[...]$ 
 {\footnoteSD{NOTE TO AUTHORS  proof hinges on the fact that we consider the "restricted" state, $\sigma''$ a "dully-fledged" state, and the fact that we no longer require "Arising".}}
 {\footnoteSD{SD  wondered whether  \ref{l:shallow:scoped}.\ref{four} would still hold if we allowed the assertions to "reflect" on the frame, to say things eg like "this and x are the only local variables". But such assertions would not have the property \ref{l:shallow:scoped}.\ref{fiveSD}}}
 \item
 TODO, or bring it after the next lemma.
\end{enumerate}
\noindent
%\vspace{.1cm}
{\textbf{End Proof Sketch}} 



\subsection{Shallow and \Scoped Semantics of Hoare tuples}

\begin{definition}[\Scoped Satisfaction of Quadruples by States]
\label{def:restrict}
For modules $\Mtwo$, $M$, state $\sigma$,  
and assertions $A$, $A'$ and  $A''$
\begin{itemize}
\item
$ {\satDAssertQuadrupleFrom \Mtwo  M  \sigma   {A} {A'} {A''} } \ \ \triangleq \ \ $  \\
$\strut \hspace{.5cm} \forall k, \overline{z}, \sigma',\sigma''.[
  \satDAssertFrom M  \sigma k   {A}  \  
  \ \Longrightarrow$\\
$\strut \hspace{4.5cm}    [ \ {\leadstoBoundedStarFin {M\madd \Mtwo}{\sigma}  {\sigma'} }\  \Rightarrow \    \satDAssertFrom M  {\sigma'} k   {\sdN{A'}}   \ ]
%[\overline{\interpret \sigma z/z}]}} \ ]
 \ \wedge$\\
$\strut \hspace{4.5cm}    [ \ {\leadstoBoundedStar  {M\madd \Mtwo}{\sigma}  {\sigma''} }\  \Rightarrow\      \satDAssertFrom M  {\sigma''}  k  {(\externalexec \rightarrow A''[\sdN{\overline{\interpret \sigma z/z}}])} \ ] $\\
$\strut \hspace{2.3cm}\ \ \ \ ]  $ \\
$\strut \hspace{2.3cm}\ \ \ \  \mbox{where }  \sdN{ \overline z= \fv(A)}$ %\!\setminus\! \vs(\sigma.\prg{cont}),\ \overline w=\fv(A)  $ 
\end{itemize}
\end{definition}



 
\begin{lemma} 
For all $M$, $\Mtwo$ $A$, $A'$, $A''$ and $\sigma$:
\begin{itemize}
\item
$ {\satDAssertQuadrupleFrom \Mtwo  M  \sigma   {A} {A'} {A''} } \ \Longrightarrow \ \
  {\satAssertQuadruple  \Mtwo  M   {A}  \sigma  {A'} {A''} } $
\end{itemize}
\end{lemma}

\label{sect:HLmeans}

We  define the {\emph {meaning}} of  our Hoare triples, $\triple {A} {stmt} {A'}$,  in the usual way, \ie that execution of $stmt$ in a state that satisfies $A$ leads to a state which satisfies $A'$.  
In addition to that, Hoare quadruples, $\quadruple {A} {stmt} {A'} {A''}$, promise that any external future states scoped by $\sigma$ will satisfy $A''$.
We give both a weak and a shallow version of the semantics


 \begin{definition}[\Scoped Semantics of Hoare triples]
For modules $M$, and assertions $A$, $A'$   we define:
%  the semantics of Hoare-triples,   $M\ \models\  \{\, A \,  \}\ stmt\  \{\, A' \, \}$ as follows:
\begin{itemize}
\item
\label{def:hoare:sem:one}
$\satisfies  {M} {  \{\, A \,  \}\ stmt\  \{\, A' \, \} }\ \ \ \triangleq$\\
{$\strut  \ \ \ \forall    \Mtwo. \forall  \sigma.[ \ \   %\arising{\sigma}{M\madd \Mtwo }\   \wedge\  
 \sigma.\prg{cont}\txteq stmt\   \Longrightarrow \ 
{\satAssertQuadruple  \Mtwo  M   {\  A\ }  \sigma  { A'  } {true} } \ \ ]$%  \ \ \  ]  $
}
 \item
 \label{def:hoare:sem:two}
$\satisfies {M} {\quadruple {A} {stmt} {A'} {A''}}  \ \ \  \triangleq$ \\
{$\strut  \ \ \  \forall    \Mtwo. \forall  \sigma.[ \  \  \sigma.\prg{cont}\txteq stmt\   \Longrightarrow \ 
{\satAssertQuadruple  \Mtwo  M    {\  A\ }  \sigma  { A'  } {A''} } \ \ ]$%  \ \ \  ]  $
}
\item
\label{def:hoare:sem:three}
$\satisfiesD {M} {  \{\, A \,  \}\ stmt\  \{\, A' \, \} }\ \ \ \triangleq$\\
{$\strut  \ \ \ \forall    \Mtwo. \forall  \sigma.[ \ \   %\arising{\sigma}{M\madd \Mtwo }\   \wedge\  
 \sigma.\prg{cont}\txteq stmt\   \Longrightarrow \ 
{\satDAssertQuadrupleFrom \Mtwo  M  \sigma   {\  A\ } { A'  } {true} } \ \ ]$%  \ \ \  ]  $
}
 
 \item
 \label{def:hoare:sem:four}
$\satisfiesD {M} {\quadruple {A} {stmt} {A'} {A''}}  \ \ \  \triangleq$ \\
{$\strut  \ \ \  \forall    \Mtwo. \forall  \sigma.[ \ \    \sigma.\prg{cont}\txteq stmt\   \Longrightarrow \ 
{\satDAssertQuadrupleFrom \Mtwo  M  \sigma   {\  A\ } { A'  } {A''} } \ \ ]$%  \ \ \  ]  $
}

\end{itemize}
\end{definition}



 
 \begin{lemma}[\Scoped   vs Shallow Semantics of Quadruples]
For all $M$, $A$, $A'$, and $stmt$:
\begin{itemize}
\item
$\satisfiesD {M} {\quadruple {A} {stmt} {A'} {A''}}   \ \ \ \Longrightarrow \ \ \ \satisfies  {M} {\quadruple {A} {stmt} {A'} {A''}} $
\end{itemize}
\end{lemma}
 
 
 
\subsection{\Scoped satisfaction of specifications} 
%\subsection{\Scoped satisfaction of specifications -- \red{better term than \Scoped?}}  
\label{sect:HLmeans}



We now give a \scoped meaning to specifications: 

\begin{definition} [\Scoped Semantics of  Specifications]

We define $\satisfiesD{M}{{S}}$ by cases: %  over the three possible syntactic forms.

\label{def:necessity-semantics:strong}

\begin{enumerate}
\item
{ 
$\satisfiesD{M}{\TwoStatesN {\overline {x:C}} {A}} \ \  \ \triangleq   \ \ \ 
\forall \sigma.[\  \satisfiesD {M} {\quadruple {\externalexec \wedge \overline {x:C} \wedge A} {\sigma} {A} {A} } \ ] $
}
  \item
 {$\satisfiesD{M} { \mprepostN {A_1}{p\ D}{m}{y}{D}{A_2} {A_3}    }\  \ \ \   \triangleq  \ \ \ $}\\ 
 {
$\strut  \ \ \   \ \ \ \ \ \ \ \ \   \    \forall   y_0,\overline y, \sigma[ \ \ \ \sigma\txteq {u:=y_0.m(y_1,..y_n)} \ \ \Longrightarrow \ \ 
\satisfiesD {M} {\quadruple  {A_1'} }   {\sigma}   {A_2' } {A_3' }  \  \ \  ]$ } \\
$\strut \ \ \   \ \ \ \ \ \ \ \ \   \  \mbox{where}$\\
$\strut  \ \ \   \ \ \ \ \ \ \ \ \   \  \ A_1' \triangleq   y_0:D,{\overline {y:D}}   \wedge   A[y_0/\prg{this}],\  \  A_2' \triangleq A_2[u/res,y_0/\prg{this}],\ \ A_3' \triangleq A_3[y_0/\prg{this}] $
 \item
 $\satisfiesD{M}{S\, \wedge\, S'}$\ \ \  \ \ \  $\triangleq$  \  \ \  \   $\satisfiesD{M}{S}\ \wedge \ \satisfiesD{M}{S'}$
\end{enumerate}
\end{definition}

 \begin{lemma}[\Scoped   vs Shallow Semantics of Quadruples]
For all $M$, $S$:
\begin{itemize}
\item
$\satisfiesD {M} {S}   \ \ \ \Longrightarrow \ \ \ \satisfies  {M} {S} $
\end{itemize}
\end{lemma}

\subsection{Soundness of the Hoare Triples Logic}
\label{s:sound:app:triples}

\begin{auxLemma}
\label{l:no:call}
For any module $M$, assertions $A$, $A'$ and $A''$, and statement $stmt$ which does not contain any method calls:
\begin{center}
$  \satisfiesD {M} {\triple {A} {stmt} {A'} }  \ \ \Longrightarrow\ \  \satisfiesD {M} {\quadruple {A} {stmt} {A'} {A''}}$
\end{center}
\end{auxLemma}




%%%%%%%%%%%%%%%%%%%%%%%%%%%%%%%%%%%%%%%%%%%%%%%%%%%%%%%%%%%%%%%%%%%%
\subsubsection{Lemmas about protection}
\label{s:app:protect:lemmas}

\begin{definition}

$\LRelevantKO  {\sigma} {k}\ \ \ \triangleq\ \ \  \{ \alpha \mid  \exists i.[ \ k \leq i \leq \depth \sigma \ \wedge \ \alpha \in \LRelevantO  {\RestictTo \sigma i}\ ]$
\end{definition}
 

{
\begin{lemma} For all $\sigma$, $\sigma'$, and $\alpha$:
\begin{itemize}
\item
$\leadstoBounded  {\Mtwo\cdot M}  {\sigma}  {\sigma'}\ \  \Longrightarrow\ \ 
  (\LRelevantKO   {\sigma} {k}\ \setminus\   \LRelevantKO {\sigma'} {k}) \cap \sigma = \emptyset$
\item
$\leadstoBoundedStar  {\Mtwo\cdot M}  {\sigma}  {\sigma'}\ \  \Longrightarrow\ \ 
  (\LRelevantKO   {\sigma} {k}\ \setminus\   \LRelevantKO {\sigma'} {k}) \cap \sigma = \emptyset$
\end{itemize}
\end{lemma}
 }
 
 {
 \begin{lemma} For all $\sigma$, $\sigma'$, and $\alpha$:
\begin{itemize}
\item
$\leadstoBounded  {\Mtwo\cdot M}  {\sigma}  {\sigma'}\ \wedge \  \sigma \models \external{\alpha} \ \wedge\  {\interpret{\sigma} {\alpha.f}} \neq {\interpret{\sigma'} {\alpha.f}}
 \ \ \Longrightarrow\ \  M,\sigma \models \extThis$
\end{itemize}
\end{lemma}
  } 
    
  
 
   {
 \begin{lemma} For all $\sigma$, $\sigma'$, and $\alpha$:
 \label{lemma:notInside:implies}
\begin{itemize}
\item
$ \satDAssertFrom M  {\sigma} k   {\inside{\alpha}}  \ \wedge \ \leadstoBounded  {\Mtwo\cdot M}  {\sigma}  {\sigma'}\ \wedge \  \notSatDAssertFrom M  {\sigma'} k   {\inside{\alpha}}
 \ \ \Longrightarrow\ \ $\\
 $\exists \alpha_o,f.[\  \alpha_0 \in \LRelevantKO  {\sigma} {k} \ \wedge \ M, \sigma' \models \external {\alpha_o} \ \wedge \ \interpret {\sigma'} {\alpha_o.f} = \alpha\  \wedge$\\
$\strut \hspace{3cm} [ \alpha_o \notin \sigma\ \vee\ \alpha_o \notin \LRelevantKO {\sigma} {k}\ \vee\  \interpret {\sigma} {\alpha_o.f} \neq \alpha\ ]\ \ \ \ \ ] $
\end{itemize}
\end{lemma}
}

Lemma \ref{lemma:inside:preserved}  guarantees that internal code which does not include method calls preserves absolute protection. 
It is used in the proof of soundness of the inference rule {\sc{Prot-1}}.

  {
 \begin{lemma} For all $\sigma$, $\sigma'$, and $\alpha$:
 \label{lemma:inside:preserved} 
\begin{itemize}
\item
$ \satDAssertFrom M  {\sigma} k   {\inside{\alpha}}  \ \wedge \ M, \sigma \models \intThis \ \wedge \ \sigma.\prg{cont} \mbox{ contains no method calls } \ \wedge\ \leadstoBounded   {\Mtwo\cdot M}  {\sigma}  {\sigma'}\  \ \ \Longrightarrow\ \ \satDAssertFrom M  {\sigma'} k   {\inside{\alpha}}$
\item
$ \satDAssertFrom M  {\sigma} k   {\inside{\alpha}}  \ \wedge \ M, \sigma \models \intThis \ \wedge \ \sigma.\prg{cont}  \mbox{ contains no method calls } \ \wedge\ \leadstoBoundedStar  {\Mtwo\cdot M}  {\sigma}  {\sigma'}\  \ \ \Longrightarrow\ \ \satDAssertFrom M  {\sigma'} k   {\inside{\alpha}}$
\end{itemize}
\end{lemma}
}

\subsubsection{Lemmas about relative protection}


  {
 \begin{lemma} For all $\sigma$, $\sigma'$, and $\alpha$:
\begin{itemize}
\item
$  M, \sigma  \models    { \protectedFrom \alpha {\alpha_o}}  \ \wedge\   \sigma.\prg{heap}= \sigma'.\prg{heap} \ \ \Longrightarrow\ \  M, \sigma' \models      { \protectedFrom \alpha {\alpha_o}} $
\end{itemize}
\end{lemma}
}

{
 \begin{lemma} For all $\sigma$,  and $\alpha$, $\alpha_o$, $\alpha_1$, $\alpha_2$:
 \label{l:prtFrom}
\begin{itemize}
\item
$ M, \sigma  \models    {\protectedFrom \alpha  {\alpha_o}}  \  \wedge \ \  M, \sigma  \models    {\protectedFrom \alpha  {\alpha_1}}    \   \ 
\Longrightarrow\ \ M, \sigma[\alpha_2,f \mapsto \alpha_1] \models  \protectedFrom\alpha   {\alpha_o}$
\end{itemize}
\end{lemma}
}

{
\begin{definition}
\begin{itemize}
\item
$M, \sigma \models \internalPaths{\re} \ \ \triangleq \ \ \forall \overline{f}.[\  M, \sigma \models \internal{\re.\overline{f}}\ ]$
\end{itemize}
\end{definition}
}

{
 \begin{lemma} For all $\sigma$, and $\alpha_o$ and $\alpha$:
\begin{itemize}
\item
$M, \sigma \models \internalPaths{\alpha_o}  \    \ \ \Longrightarrow\ \ M, \sigma \models {\protectedFrom \alpha {\alpha_o}}$
\end{itemize}
\end{lemma}
}

\noindent
\vspace{.2cm}
 {\textbf{Proof Sketch Theorem \ref{l:triples:sound}} 
The proof goes by case analysis over the rule applied to obtain $M \vdash \{ A \}\ stmt \  \{ A' \} $:

\begin{description} 

\item[{\sc{extend}}] 
By  soundness of the underlying Hoare logic (axiom \ref{ax:ul:sound}), we obtain that  $M\ \models\ \triple {A} {stmt}   {A'}$.
By axiom \ref{ax:ul} we also obtain that $\Stable{A}$ and  $\Stable{A'}$. 
This, together with   Lemma \ref{l:shallow:scoped}, part \ref{shallow:to:scoped}, gives us that
$\satisfiesD {M} {\triple {A} {stmt} {A'} }$. 
By the assumption of {\sc{extend}}, $stmt$ does not contain any method call. Rest follows by lemma \ref{l:no:call}.

\item[{\sc{types-1}}] 

Follows from type system, the assumption of {\sc{types-1}} and lemma \ref{l:no:call}.

\item[{\sc{Prot-1}}] by Lemma \ref{lemma:inside:preserved}.

\item[{\sc{Prot-4}}] by Lemma \ref{l:prtFrom}

\end{description}
\noindent
\vspace{.1cm}
{\textbf{End Proof Sketch}} 

%%%%%%%%%%%%%%%%%%%%%%%%%%%%%%%%%%%%%%%%%%%%%%%%%%%%%%%%
\newcommand{\Ao}{A_{o}}
\newcommand{\Ain}{A_{in}}
\newcommand{\Aout}{A_{out}}

\subsection{Sequences, Sets, Substitutions and Free Variables}


Our system makes heavy use of textual substitution,   textual inequality, and the concept of free variables in assertions. 
 
In this subsection we introduce some notation and some lemmas to deal with these concepts.
These concepts and lemmas are by no means novel; we list them here so as to use them more easily in the subsequent proofs.


\begin{definition}[Sequences,   Disjointness, and Disjoint Concatenation]
For any variables $v$, $w$, and sequences of variables $\overline v$, $\overline w$ we define:
\begin{itemize}
\item
 $v \in \overline w \ \ \triangleq \ \  \exists  \overline {w_1},  \overline {w_1}[\  {\overline w} = \overline {w_1}, v, \overline {w_2} \ ]$
\item
$v \# w \ \ \triangleq \ \ \neg(v \txteq w)$.
\item
$\overline v \subseteq \overline w \ \ \triangleq \ \ \forall v.[\ v \in  \overline v\ \Rightarrow\ v \in  \overline w\ ]$
\item
$\overline v \#  \overline w \ \ \triangleq \ \ \forall v \in  \overline v. \forall w \in  \overline w.  [ \ v \# w\  ]$
\item
$ \overline v \cap \overline w \ \ \triangleq \ \  \overline u, \ \ \ \ \mbox{such that}\   \forall u.[ \ \ u  \in   \overline v \cap \overline w \ \ \Leftrightarrow\ \  [ \ u\in \overline v\ \wedge\ u\in \overline w\  ]$
\item
$ \overline v \setminus \overline w \ \ \triangleq \ \  \overline u, \ \ \ \ \mbox{such that}\   \forall u.[ \ \ u  \in  \overline v \setminus \overline w \ \ \Leftrightarrow\ \  [ \ u\in \overline v\ \wedge\ u\notin \overline w\  ]$
\item
$\overline v; \overline w \ \ \triangleq \ \ \overline v$, $\overline w$ \ \ if $\overline v \#  \overline w $ \ \ \ and  undefined otherwise.
\end{itemize}
\end{definition}

\begin{lemma}[Substitutions and Free Variables]
\label{l:sfs}
For any sequences of variables $\overline x$, $\overline y$, $\overline z$, $\overline v$, $\overline w$, a variable $w$, any assertion $A$, we have
\begin{enumerate}
\item
\label{l:sfs:zero}
$ \overline x[\overline{y/x} ] = \overline y $
\item
\label{l:sfs:zero:one}
$ \overline {x} \# \overline y \ \   \Rightarrow \ \  \overline y[\overline{z / x} ] = \overline y $
\item
\label{l:sfs:one}
$\overline {z} \subseteq \overline y \ \   \Rightarrow \ \  \overline y[\overline{z / x} ] \subseteq \overline y $
\item
\label{l:sfs:two}
$\overline {y} \subseteq \overline z \ \   \Rightarrow \ \  \overline y[\overline{z / x} ] \subseteq \overline z $
\item
\label{l:sfs:three}
 $\overline x \# \overline y \ \ \Rightarrow \ \ {\overline z}[\overline{y / x}]  \# \overline x $ 
 \item
 \label{l:sfs:four}
 $\fv(A[\overline{y / x}] )\, =\, \fv(A)[\overline{y / x}] $
 \item
 \label{l:sfs:five}
 $\fv(A)\, =\, \overline x; \overline v, \ \ \   \fv(A[\overline{y / x}] )\, = \,   \overline y; \overline w 
 \ \ \ \Longrightarrow\ \ \ 
 \overline v \, = \, (\overline {y}\cap\overline{v}); \overline w $
  \item
  \label{l:sfs:sixa}
 $ \overline v \# \overline x   \# \overline y   \# \overline u  \ \ \    \ \ \ \Longrightarrow\ \ \ 
w[ \overline {u/x} ][ \overline {v/y} ]  \txteq w[ \overline {v/y} ][ \overline {u/x} ]  $

 \item
  \label{l:sfs:six}
 $ \overline v \# \overline x   \# \overline y   \# \overline u  \ \ \    \ \ \ \Longrightarrow\ \ \ 
A[ \overline {u/x} ][ \overline {v/y} ]  \txteq A[ \overline {v/y} ][ \overline {u/x} ]  $
\item
  \label{l:sfs:seven}
 $( fv (A[ \overline {y/x} ])\setminus \overline y)\, \# \, \overline x$
% \item
%  \label{l:sfs:six}
%$\fv(A)\, =\, \overline x; \overline v, \ \ \   \fv(A[\overline{y / x}] )\, = \,   \overline y; \overline w, \ \ \ \overline y\, \# \, \overline x; \overline v
% \ \ \ \Longrightarrow\ \ \ 
% \overline v \, = \,\overline w \ $ 
\end{enumerate}

\end{lemma}

\noindent 
\textbf{Proof of Lemma \ref{l:sfs}}
\begin{enumerate}
\item
 by induction on  the number of elements in $\overline x$ 
\item
 by induction on  the number of elements in $\overline y$ 

\item
 by induction on  the number of elements in $\overline y$ 
\item
 by induction on  the number of elements in $\overline y$ 
\item
 by induction on the structure of $A$ 
 \item
 by induction on the structure of $A$ 
\item
Assume that\\
$(ass1)\ \ \  \fv(A)\, =\, \overline x; \overline v,$
\\
$(ass2)\ \ \  \fv(A[\overline{y / x}] )\, = \,   \overline y; \overline w$\\
We define:
\\
$(a) \ \ \  \overline {y_0} \triangleq \overline v \cap \overline y, \ \ \  \overline {v_2} \triangleq \overline v \setminus \overline y, \ \ \ \overline {y_1} = \overline {y_0}[\overline {x/y}]$
\\
This gives:\\
$(b) \ \ \ \overline {y_0} \#   \overline {v_2}$
\\
$(c)\ \ \ \overline v =  \overline {y_0}; \overline {v_2}$\
\\
$(d) \ \ \  \overline {y_1}  \subseteq \overline y$
\\
$(e) \ \ \ \overline {v_2}[\overline{y / x}] = \overline {v_2}$, \ \ \ from assumption and (a) we have $\overline x \# \overline v_2$ and by Lemma \ref{l:sfs}) part (\ref{l:sfs:zero:one})
\\
We now calculate \\
$\begin{array}{lcll}
\ \ \  \fv(A[\overline{y / x}] )   & = &  (\overline x; \overline v)[\overline{y / x}] & \mbox{ by (ass1), and Lemma \ref{l:sfs} part (\ref{l:sfs:three}).}
\\ 
& = &  (\overline x; \overline {y_0}; \overline {v_2})[\overline{y / x}] & \mbox{ by (c) above }
\\
& = &   \overline x[\overline{y / x}], \, \overline {y_0}[\overline{y / x}], \overline {v_2}[\overline{y / x}] & \mbox{ by distributivity of $[../..]$ }
\\
& = &   \overline y, \, \overline {y_1}, \overline {v_2}  & \mbox{ by Lemma \ref{l:sfs} part (\ref{l:sfs:zero}), (a), and (e). }
\\
& = &   \overline y; \overline {v_2}  & \mbox{ because (d), and  $ \overline y \# \overline {v_2}$ }
\\
\ \ \  \fv(A[\overline{y / x}] )   & = &  \overline y; \overline w  & \mbox{ by (ass2)}
\end{array}
$
\\
The above gives that $\overline {v_2} = \overline {w}$. This, together with (a) and (c) give that $\overline {v} = (\overline {y}\cap\overline{v});\overline{w}$ 
% Thus, from (c) and the above we obtain  $ \overline v =  \overline {y_0}; \overline {w}$; and (a) gives $\overline {y_0}\subseteq \overline {y}$
% \item
% Follows from (\ref{l:sfs:five}).
\item
By case analysis on whether $w \in \overline x$ ... etc
\item
By induction on the structure of $A$, and the guarantee from (\ref{l:sfs:sixa}).
\item
We take a variable sequence $\overline z$ such that \\
$(a) \  \  \fv(A  ) \subseteq \overline{x}; \overline z$
\\
This gives that\\
$(b) \  \   \overline{x} \# \overline z$
\\
Part (\ref{l:sfs:four}) of our lemma and (a) give\\
$(c) \  \  \fv(A[\overline{y / x}] ) \subseteq \overline{y}, \overline z$
\\
Therefore
\\
$(d) \  \  \fv(A[\overline{y / x}] ) \setminus {\overline y } \subseteq  \overline z$
\\
The above, together with (b) conclude the proof
 \end{enumerate}


\noindent
\textbf{End Proof}


\begin{lemma}[Substitutions and Adaptations]
\label{l:sybbs:adapt}
For any sequences of variables $\overline x$, $\overline y$, sequences of expressions $\overline e$, and   any assertion $A$, we have
\begin{itemize}
\item
$ \overline x \# \overline y \ \ \ \Longrightarrow \ \ \  \PushAS {y} {(A[\overline {e/x}])}  \txteq   (\PushAS {y} {A})[\overline{ e/x}] $
\end{itemize}

\end{lemma}

\noindent
\textbf{Begin Proof}

\noindent 
We first consider $A$ to be $\inside e_0$, and just take one variable. Then, \\
$\strut \ \ \ \ { \PushAS  {y} { (\inside {e_0}[e/x] ) } }
 \ \txteq\  {\protectedFrom {e_0[ {e/x}]} {y}}$, \\
and\\
$\strut \ \ \ \ (\PushAS {y} {\inside {e_0}})[e/x] \  \txteq\  \protectedFrom {e_0[{e/x}]} {y[ {e/x}]}$. \\
When $x \# y$  then the two assertions from above  are textually equal.
The rest follows by induction on the length of $\overline x$ and the structure of $A$.
 

\noindent
\textbf{End Proof}




%%%%%%%%%%%%%%%%%%%%%%%%%%%%%%%%%%%%%%%%%%%%%%%%%%%%%%%%%%
%%%%%%%%%%%%%%%%%%%%%%%%%%%%%%%%%%%%%%%%%%%%%%%%%%%%%%%%%%
\subsection{Preservation of assertions when pushing or popping frames}

In this section we  discuss the preservation of satisfaction of assertions when calling methods or returning from methods -- \ie when pushing or popping  frames. 
Namely, since  pushing/popping frames  does not modify the heap, these operations should preserve satisfaction of some assertion $A$, up to the fact that a) passing an object as a parameter of a a result might break its protection, and 
b) the bindings of variables change  with pushing/popping frames.
To deal with a)  upon method call, 
we   require that the fame being pushed or the frame to which we return is internal ($M, \sigma' \models \intThis$), or require the adapted version of an assertion (\ie  ${\PushAS  {v} { A}}$ rather than $A$).
To deal with b) we either require that there are no free variables in $A$, or we break the free variables of $A$ into two parts, \ie $\fv(\Ain) =  \overline{v_1};\overline{v_2}$, where the value of $\overline{v_3}$ in the caller is the same as that of  $\overline{v_1}$ in the called frame.
%To deal with b) upon method return, we also break the free variables of $A$ into two parts, \ie $\fv(\Ain) =  \overline{v_1}, \overline{v_2}$, where the value of $\overline{v_1}$ in the callee is the same as that of  i$\overline{v_3}$ in the caller.
%And in both cases   $\overline {v_2}$ do not necessarily have a value in the callee frame.
This   is described in  lemmas \ref{l:calls} -  \ref{l:calls:return}.

We have four lemmas: Lemma \ref{l:calls} describes preservation from a caller to an internal called, lemma \ref{l:calls:external}
describes preservation from a caller to any called, Lemma \ref{l:calls:return} describes preservation from an internal called to a caller, and  Lemma \ref{l:calls:return:ext} describes preservation from an any called to a caller, 
These four lemmas are used in the soundness proof for the four Hoare rules   about method calls, as given in Fig. \ref{f:calls}. 

In the rest of this subsection we will first  state and discuss each of the lemmas,
and we will then prove them.
  
 
 %%%% From caller to called %%%
\begin{lemma}[From caller to internal called]
\label{l:calls}

For any assertion $\Ain$, states $\sigma$, $\sigma'$,  
variables  $\overline{v_1}$,    $\overline{v_2}$,  $\overline{v_3}$,  $\overline{v_4}$,  $\overline{v_6}$,  % addresses $\overline \alpha_2$ and $\overline \alpha_4$, 
statement $stmt$, and frame $\phi$.

\noindent
% $ ~ $ % and $\overline{y'}$,  \\
If 
\begin{enumerate}[(i)]
\item 
\label{l:calls:r:one}
$ \Pos \Ain$,  
\item 
\label{l:calls:r:two}
$\fv(\Ain) =  \overline{v_1}; \overline{v_2} $\footnote{As we said earlier. this gives  also that the variable sequences  are pairwise disjoint, \ie $\overline{v_1}\#\overline{v_2}$.},
\ \ \ 
$\fv(\Ain[\overline {v_3/v_1}]) =  \overline{v_3}; \overline{v_4} $, \ \ \ \ 
$ \overline {v_6}\triangleq\overline{v_2}\cap\overline{v_3}; \overline{v_4} $, 
\item
$\sigma'=\sigma  \pushSymbol \phi  \ \ \ \  \wedge\ \ \  Rng(\phi)= \overline{\interpret {\sigma} {v_3} }\ \ \ \ \wedge \ \  \ \ \overline {\interpret {\sigma'}  {v_1} } = \overline {\interpret {\sigma} {v_3} }$, 

\end{enumerate}

\noindent
then

\begin{enumerate}[(a)]
\item
\label{l:calls:callee:one}
$\satDAssertFrom M  \sigma k   \Ain[\overline {v_3/v_1}] % [\overline {\alpha_2/v_2}] \ \   
\ \ \wedge\ \ M, \sigma' \models \intThis
%   $  \\  $\strut \hspace{2cm} 
 \hfill \Longrightarrow  \ \ \  \   \ \satDAssertFrom M  {\sigma'} k  {\Ain[\overline { {\interpret {\sigma} {v_6}} / {v_6} } ] } $
 
\item

\label{l:calls:callee:two}
$ \satDAssertFrom M  \sigma k    {\PushASLong  {(\overline {v_3})} {(\Ain[\overline {v_3/v_1}])}} 
%  $ \\  $\strut \hspace{2cm}  
\hfill \Longrightarrow  \ \ \  \   \ \ \  \  
M, \sigma' \models   {\Ain\overline { {\interpret {\sigma} {v_6}} / {v_6} } ] }$.

\end{enumerate}

\end{lemma}

In lemma \ref{l:calls},   state $\sigma$ is the state right before pushing the new frame on the stack,   while 
state $\sigma'$ is the state right after pushing the frame on the stack.
That is, $\sigma$ is the last state before entering the method body, and $\sigma'$ is the first state after entering the method body.
$\Ain$ stands for the method's precondition, while the variables $\overline {v_1}$ stand for the formal parameters of the method,
and $\overline {v_3}$ stand for the actual parameters of the call.
Therefore, $\overline {v_1}$ is the domain of the new frame, and $\overline {\sigma} {v_3}$ is its range.
The variables $\overline {v_6}$ are the free variables of $\Ain$ which are not in  $\overline {v_1}$ -- \cf Lemma \ref{l:sfs} part
( \ref{l:sfs:five}).
Therefore if (\ref{l:calls:callee:one})  the callee is internal, and 
 $\Ain[\overline {v_3/v_1}]$ holds  at the call point, or
 if (\ref{l:calls:callee:two}) ${\PushASLong  {(\overline {v_3})} {(\Ain[\overline {v_3/v_1}])}}$
 holds  at the call point, 
 then $\Ain[\overline {.../v_61}]$  holds right after pushing $\phi$ onto the stack.
% Moreover, (\ref{l:calls:callee:two})  , then $\Ain[\overline {.../v_6}]$  holds right after pushing $\phi$ onto the stack.
Notice the difference in the conclusion in (\ref{l:calls:callee:one}) and (\ref{l:calls:callee:two}): in the first case we have \scoped satisfaction, while in the second case we only have shallow satisfaction.



%%% calling external

\begin{lemma}[From caller to any called]
\label{l:calls:external}

For any assertion $\Ain$, states $\sigma$, $\sigma'$,  
variables   $\overline{v_3}$  % addresses $\overline \alpha_2$ and $\overline \alpha_4$, 
statement $stmt$, and frame $\phi$.

\noindent
% $ ~ $ % and $\overline{y'}$,  \\
If 
\begin{enumerate}[(i)]
\item 
\label{l:calls:re:one}
$ \Pos \Ain$,  
\item 
\label{l:calls:re:two}
$\fv(\Ain) =  \emptyset$,
\item
\label{l:calls:re:three}
$\sigma'=\sigma  \pushSymbol \phi  \ \ \ \  \wedge\ \ \  Rng(\phi)= \overline{\interpret {\sigma} {v_3} },$
\end{enumerate}

\noindent
then

\begin{enumerate}[(a)]
\item

\label{l:calls:callee:three}
$ \satDAssertFrom M  \sigma k    {\PushASLong  {(\overline {v_3})} {\Ain} } 
%  $ \\  $\strut \hspace{2cm}  
\hfill \Longrightarrow  \ \ \  \   \ \ \  \  
M, \sigma' \models   {\Ain} $.


\item
\label{l:calls:callee:four}
$\satDAssertFrom M  \sigma k    {(\Ain  \wedge  ({\PushASLong  {(\overline {v_3})} {\Ain} } ) )}  
% $  \\   $\strut \hspace{2cm} 
 \hfill \Longrightarrow  \ \ \  \   \satDAssertFrom M  {\sigma'} k  {\Ain}$


\end{enumerate}

\end{lemma}

In lemma \ref{l:calls:external},  as in  lemma \ref{l:calls}, 
  $\sigma$ stands for the last state before entering the method body, and $\sigma'$ for the first state after entering the method body.
$\Ain$ stands for a module invariant in which all free variables have been substituted by addresses.
The lemma is intended for external calls, and therefore we have no knowledge of the method's formal parameters.
The variables   $\overline {v_3}$ stand for the actual parameters of the call, and therefore 
 $\overline {\interpret {\sigma} {v_3}}$ is the range of the new frame.
Therefore if (\ref{l:calls:callee:three})   the adapted version,
 ${\PushASLong  {(\overline {v_3})} {\Ain} }$, holds  at the call point,
 then the unadapted version, $\Ain$  holds right after pushing $\phi$ onto the stack.
 Notice that  even though the premise of (\ref{l:calls:callee:three}) requires \scoped satisfaction, the conclusion promises
 only weak satisfaction.
 Moreover, if (\ref{l:calls:callee:four})   the adapted as well as the unadapted version,
 $\Ain \wedge {\PushASLong  {(\overline {v_3})} {\Ain} }$
 holds  at the call point,
 then the unadapted version, $\Ain$  holds right after pushing $\phi$ onto the stack.
 % Moreover, (\ref{l:calls:callee:two})  , then $\Ain[\overline {.../v_6}]$  holds right after pushing $\phi$ onto the stack.
Notice the difference in the conclusion in (\ref{l:calls:callee:three}) and (\ref{l:calls:callee:four}): in the first case we have  shallow satisfaction, while in the second case we only have \scoped satisfaction.


%%% From called back to caller

\begin{lemma}[From internal called to caller]
\label{l:calls:return}

For any assertion $\Aout$, states $\sigma$, $\sigma'$,  
variables  $\overline{v_1}$,      $\overline{v_3}$,   $\overline{v_5}$,  % addresses $\overline \alpha_2$ and $\overline \alpha_4$, 
statement $stmt$.

\noindent
 If 
 
\begin{enumerate}[(i)]
\item 
\label{l:calls:r:one}
$ \Pos \Aout$,  
\item 
\label{l:callers:r:two}
$\fv(\Aout) \subseteq  \overline{v_1} $,
\item
$  \overline {\interpret {\sigma'} {v_5}}, \overline {\alpha_4} \subseteq \LRelevantO \sigma  \ \ \wedge \ \  M, \sigma' \models \intThis$,
 \item
\label{l:callers:three}
$\sigma'= (\sigma \popSymbol)[\overline{ v_4\! \mapsto\! \alpha_4}][\prg{cont}\!\mapsto\! stmt]
\ \ \wedge \ \   \overline {\interpret {\sigma} {v_1} }\, =\,  \overline {\interpret {\sigma'}  {v_3} }$, 
  \end{enumerate}
  
\noindent
then
% $\Longrightarrow$


\begin{enumerate} [a]
\item
\label{l:calls:caller:one}
$\satDAssertFrom M  \sigma k   \Aout  \ \  \wedge\ \ \DepthSt {\sigma'} \geq k  \ 
%$ \\ $\strut \hspace{2cm} 
 \hfill \Longrightarrow  \ \ \  \   \satDAssertFrom M  {\sigma'} k   {\Aout[\overline {v_3 / v_1}]  }$ .

 
 \item
 \label{l:calls:caller:two}
 $M, \sigma \models  {\Aout}\ % \ \wedge\ \  \overline {\interpret {\sigma'} {v_5}} \subseteq \LRelevantO \sigma
%  $ \\   $\strut \hspace{2cm} 
 \hfill \Longrightarrow  \ \ \  \  %
 \satDAssertFrom M  {\sigma'} k   {\PushASLong  {\overline {v_5}}    {(\Aout[\overline { v_3/v_1}])} }$.

 
\end{enumerate}

\end{lemma}

In lemma \ref{l:calls:return},  
 state  $\sigma$ stands for the last state in the method body, and $\sigma'$ for the first state after exiting  the method call.
$\Aout$ stands for a method postcondition.
The lemma is intended for internal calls, and therefore we know the method's formal parameters.
The variables   $\overline {v_1}$ stand for the formal  parameters of the method, and  
  $\overline {v_3}$ stand for the actual parameters of the call, and therefore 
  the  the formal parameters of the called have the same values as the actual parameters in the caller
  $  \overline {\interpret {\sigma} {v_1} }\, =\,  \overline {\interpret {\sigma'}  {v_3} }$.
% $\overline {\interpret {\sigma} {v_3}}$ is the range of the new frame.
Therefore   ( \ref{l:calls:caller:one})  and  (\ref{l:calls:caller:two})  
promise that if the postcondition $\Aout$ holds before popping the frame, then it also holds after popping frame after replacing the 
the formal parameters by the actual parameters $\Aout[\overline{v_3/v_1}]$.
As in earlier lemmas, there is an important difference between  (\ref{l:calls:caller:one}) and (\ref{l:calls:caller:two}):
In (\ref{l:calls:caller:one}), we require \emph{deep satisfaction at the called}, 
and obtain at the deep satisfaction of the \emph{unadapted} version ($\Aout[\overline{v_3/v_1}]$) at the return point;
while in (\ref{l:calls:caller:two}), we only require \emph{shallow satisfaction at the called}, 
and obtain deep satisfaction of the \emph{adapted} version (${\PushASLong  {\overline {v_5}}    {(\Aout[\overline { v_3/v_1}])} }$),
at the return point.


\begin{lemma}[From any called to caller]
\label{l:calls:return:ext}

For any assertion $\Aout$, states $\sigma$, $\sigma'$,  
variables  $\overline{v_1}$,      $\overline{v_3}$,   $\overline{v_5}$,  % addresses $\overline \alpha_2$ and $\overline \alpha_4$, 
statement $stmt$.

\noindent
 If 
 
\begin{enumerate}[(i)]
\item 
$ \Pos \Aout$,  
\item 
 $\fv(\Aout) = \emptyset$,
\item
$  \overline {\interpret {\sigma'} {v_5}}, \overline {\alpha_4} \subseteq \LRelevantO \sigma $,
 \item
$\sigma'= (\sigma \popSymbol)[\overline{ v_4\! \mapsto\! \alpha_4}][\prg{cont}\!\mapsto\! stmt]
$, 
  \end{enumerate}
  
\noindent
then
% $\Longrightarrow$


\begin{enumerate}[(a)]
\item
 \label{l:ext:return:one}
 $M, \sigma \models  {\Aout}\  
 \hfill \Longrightarrow  \ \ \  \  \satDAssertFrom M  {\sigma'} k  {\PushASLong  {\overline {v_5}}    {\Aout}}$.

\item
\label{l:ext:return:two}
$\satDAssertFrom M  \sigma k   \Aout  \ \  \wedge\ \ \DepthSt {\sigma'} \geq k  \ 
%$ \\ $\strut \hspace{2cm} 
 \hfill \Longrightarrow  \ \ \  \   \satDAssertFrom M  {\sigma'} k   {\Aout \, \wedge\, {\PushASLong  {\overline {v_5}}    {\Aout}}} $

 
\end{enumerate}

\end{lemma}


In lemma  \ref{l:calls:return:ext}, similarly to  lemma \ref{l:calls:return},  
 state  $\sigma$ stands for the last state in the method body, and $\sigma'$ for the first state after exiting  the method call.
$\Aout$ stands for a method postcondition.
The lemma is meant to apply to external calls, and therefore, we do not know the method's formal parameters, 
$\Aout$ is meant to stand for a module invariant where all the free variables have been substituted by addresses --
\ie $\Aout$ has no free variables.
The variables $\overline {v_3}$ stand for the actual parameters of the call, and therefore 
 % $\overline {\interpret {\sigma} {v_3}}$ is the range of the new frame.
Therefore    (\ref{l:ext:return:one})  and  (\ref{l:ext:return:two})
promise that if the postcondition $\Aout$ holds before popping the frame, then it its adapted version 
also holds after popping frame (${\PushASLong  {\overline {v_5}}    {\Aout}}$).
 As in earlier lemmas, there is an important difference between   (\ref{l:ext:return:one})  and  (\ref{l:ext:return:two})
In  (\ref{l:ext:return:one}),  we require \emph{shallow satisfaction at the called}, 
and obtain   deep satisfaction of the \emph{adapted} version (${\PushASLong  {\overline {v_5}}    {\Aout}}$) at the return point;
while in (\ref{l:ext:return:two}), we  require \emph{deep satisfaction at the called}, 
and obtain deep satisfaction of the   \emph{conjuction}  of the \emph{unadapted} with the \emph{adapted} version (${\Aout \, \wedge\, {\PushASLong  {\overline {v_5}}    {\Aout}}}$),
at the return point.

\subsubsection{Use of these Lemmas}

As we said earlier, Lemmas \ref{l:calls}-\ref{l:calls:external} are used to prove the soundnes of the Hoare logic rules for method calls.
In the proof of soundness of {\sc{Call\_Int}}. we will use Lemma \ref{l:calls} part (\ref{l:calls:callee:one}) and Lemma \ref{l:calls:return} part (\ref{l:calls:caller:one}).  
In the proof of soundness of {\sc{Call\_Int\_Adapt}} we will use  Lemma \ref{l:calls} part (\ref{l:calls:callee:two}) and Lemma \ref{l:calls:return} part (\ref{l:calls:caller:two}).
In the proof of soundness of {\sc{Call\_Ext\_Adapt}} we will use  Lemma \ref{l:calls:external} part (\ref{l:calls:callee:three}) and Lemma \ref{l:calls:return:ext} part (\ref{l:ext:return:one}).
In the proof of soundness of {\sc{Call\_Ext\_Adapt\_Strong}} we will use  Lemma \ref{l:calls:external} part (\ref{l:calls:callee:four}) and Lemma \ref{l:calls:return:ext} part (\ref{l:ext:return:two}).
  

\subsection{Well-founded ordering}

 \begin{definition}
\label{def:measure}
For a module $M$, and modules $\Mtwo$,   we define a measure, $\measure {A, \sigma,A',A''} {M,\Mtwo} $, and based on it, a well founded ordering $(A_1,\sigma_1,A_2, A_3) \ll_{M,\Mtwo}  (A_4,\sigma_2,A_5,A_6)$
as follows:
\begin{itemize}
\item
 $\measure {A, \sigma,A',A''} {M,\Mtwo} \  \ \triangleq \ \  (m, n)$,  \ \ \  where
\begin{itemize}
\item
$m$ is the minimal number of execution steps so that $ \leadstoBoundedStarFin {M\cdot \Mtwo} {\sigma}    {\sigma'}$  for some $\sigma'$, {and $\infty$ otherwise}.
 \item
  $n$ is minimal depth of all proofs of $M \vdash \quadruple {A} {\sigma.\prg{cont}} {A'} {A''} $.
\end{itemize}
 \item
 $(m,n) \ll (m',n')$\ \  \ \ $\triangleq$ \ \  \ \ $ m<m'\vee  (m=m'  \wedge n < n')   $.
\item
$(A_1,\sigma_1,A_2, A_3) \ll_{M,\Mtwo}  (A_4,\sigma_2,A_5, A_6)$  \  \  $\triangleq$ \ \ 
$\measure {A_1, \sigma_1,A_2, A_3} {M,\Mtwo}  \ll \measure {A_4, \sigma_2,A_5. A_6} {M,\Mtwo} $
\end{itemize}
\end{definition}

%\subsection{Our Well-founded ordering}

\begin{lemma}
\label{lemma:normal:two}
For any modules $M$ and $\Mtwo$,  the relation $\_ \ll_{M,\Mtwo}  \_$ is well-founded.
\end{lemma}

%%%%%%%%%%%%%%%%%%%%%%%%%%%%%%%%%%%%%%%%%%%%%%%%%%%%%%%%%%

\subsection{Public States, properties of executions consisting of several steps}

We t define a state to be public, if the currently executing method is public.

\begin{definition}
We use the form
$M, \sigma \models \pubMeth$ to express that the currently executing method is public.\footnote{This can be done by looking in the caller's frame -- ie the one right under the topmost frame -- the class of the current receiver and the name of the currently executing method, and then looking up the method definition in the module $M$; if not defined there, then it is not \prg{public}. }
Note that $\pubMeth$ is not part of the assertion language.
\end{definition}

 \begin{auxLemma}[Enclosed Terminating Executions]\footnoteSD{TODO find better name for the aux lemma}
 \label{lemma:encl:tem}
 For   modules $\Mtwo$,   states $\sigma$, $\sigma'$, $\sigma_1$:
\begin{itemize}
\item
$  \leadstoBoundedStarFin {\Mtwo}  {\sigma}  {\sigma'} \  \wedge \  \leadstoBoundedStar  {\Mtwo}  {\sigma}  {\sigma_1} 
% $\\ $
\ \ \  \Longrightarrow\ \ \  % $\\ $  
 \exists \sigma_2.[\ \ \leadstoBoundedStarFin {\Mtwo} {\sigma_1}  {\sigma_2}  
\ \wedge\ 
\leadstoBoundedStarThree  {\Mtwo}  {\sigma_2}  {\sigma}   {\sigma'} \ \ ]$
\end{itemize}

\end{auxLemma} 

\begin{auxLemma}[Executing  sequences]
\label{lemma:subexp}
For modules $\Mtwo$, statements $s_1$, $s_2$,  states $\sigma$, $\sigma'$, $\sigma'''$:
\begin{itemize}
\item
$ \sigma.\texttt{cont} = s_1; s_2 \ \ \wedge\ \  \leadstoBoundedStarFin {\Mtwo}  {\sigma}  {\sigma'}\ \ 
\wedge \ \
\leadstoBoundedStar {\Mtwo}  {\sigma}  {\sigma''}\
$\\
$  \Longrightarrow$\\
$   \exists \sigma''.[\ \ \ \ \   \leadstoBoundedStarFin {\Mtwo} {\sigma[\texttt{cont}\mapsto s_1]}  {\sigma''}  
\ \wedge\ 
\leadstoBoundedStarFin {\Mtwo} {\sigma''[\texttt{cont}\mapsto s_2]}   {\sigma'} \  \wedge$
\\
$\strut \hspace{1.2cm}  [ \ \ \leadstoBoundedStar {\Mtwo} {\sigma[\texttt{cont}\mapsto s_1]}   {\sigma''}\ \vee \ \leadstoBoundedStar {\Mtwo}  {\sigma''[\texttt{cont}\mapsto s_2]}   {\sigma'''}\ ]\ \ \ \ \ \ \ \  \ ] $
\end{itemize}
\end{auxLemma}

\subsection{Summarised Executions}

We repeat the two diagrams given in \S \ref{s:summaized}.

\begin{tabular}{lll}
\begin{minipage}{.45\textwidth}
The diagram opposite  shows such an execution:
  $ \leadstoBoundedStarFin {\Mtwo\cdot M}    {\sigma_2}  {\sigma_{30}}$ consists of 4 calls to external objects,
and 3 calls to internal objects.
The calls to external objects are from $\sigma_2$ to $\sigma_3$,  from $\sigma_3$ to $\sigma_4$, from $\sigma_9$ to $\sigma_{10}$, 
and  from $\sigma_{16}$ to $\sigma_{17}$.
 The calls to internal objects are from $\sigma_5$ to $\sigma_6$, rom $\sigma_7$ to $\sigma_8$, and from $\sigma_{21}$ to $\sigma_{23}$. 
\end{minipage}
& \ \  &
\begin{minipage}{.4\textwidth}
\resizebox{6.2cm}{!}
{
\includegraphics[width=\linewidth]{diagrams/summaryA.png}
} \end{minipage}
\end{tabular}

\begin{tabular}{lll}
\begin{minipage}{.45\textwidth}
 In terms of our example, we want to summarise the execution of the two ``outer'' internal, public methods into the 
 ``large'' steps $\sigma_6$ to $\sigma_{19}$ and $\sigma_{23}$ to $\sigma_{24}$.
 And are not concerned with the states reached from these two public method executions.  
\end{minipage}
& \ \  &
\begin{minipage}{.4\textwidth}
\resizebox{6.2cm}{!}
{
\includegraphics[width=\linewidth]{diagrams/summaryB.png}
} \end{minipage}
\end{tabular} 

\noindent 
In order to express such summaries, Def. \ref{def:exec:sum} introduces the following concepts:
\begin{itemize}
\item
 ${\leadstoBoundedThreeStarExt {\Mtwo\cdot M} {\sigma\bd}  {\sigma}  {\sigma'}}$ \ \ \  execution from $\sigma$ to $\sigma'$ scoped by $\sigma\bd$, involving  external states only.
\item
${\WithPub {\Mtwo\cdot M}    {\sigma}  {\sigma'} {\sigma_1}}$ \  \ \  \ \ \ \ \ \ \ \  ${\sigma}$ is an external state  calling an internal public method, and \\
$\strut \hspace{3.25cm}$ $\sigma'$ is the state after return from the public method, and \\
$\strut \hspace{3.25cm}$  $\sigma_1$ is the first state upon entry to the public method.  
%$\WithExtPub {\Mtwo\cdot M} {\sigma\bd}  {\sigma}  {\sigma'} {\epsilon}$ \ \     \ \  $\triangleq$ \ \ 
%$\leadstoBoundedThreeStarExt {\Mtwo\cdot M} {\sigma\bd}  {\sigma}  {\sigma''}$
%\item
%$\WithExtPub {\Mtwo\cdot M} {\sigma\bd}  {\sigma}  {\sigma'} {\sigma_1...\sigma_n}$   \ \  int
%\item
%$\leadstoBoundedExtPub {\Mtwo\cdot M}    {\sigma}  {\sigma'} $   \ \ \ \ \   \ \ \  \ \ \ \   \ \ \ \  $\triangleq$   \ \ 
%  $ \exists n\in \mathbb{N}. \exists \sigma_1,...\sigma_n. \ \WithExtPub {\Mtwo\cdot M} {\sigma}  {\sigma}  {\sigma'} {\sigma_1...\sigma_n} 
%$
\end{itemize}
  
  \noindent
Continuing with our example, we have the following execution summaries:
\begin{enumerate}
\item
${\leadstoBoundedThreeStarExt {\Mtwo\cdot M} {\sigma_3}  {\sigma_3}  {\sigma_5}}$\ \ \ 
Purely external execution from $\sigma_3$ to $\sigma_5$, scoped by $\sigma_3$.
\item
${\WithPub {\Mtwo\cdot M}    {\sigma_5}  {\sigma_{20}} {\sigma_{6}}}$. \ \ \ 
Public method call from external state $\sigma_5$ into  nternal state $\sigma_6$ returning to $\sigma_{20}$. 
Note that this   summarises two  internal method executions ($\sigma_{6}-\sigma_{19}$, and $\sigma_8-\sigma_{14}$),
and two external method executions ($\sigma_{6}-\sigma_{19}$, and $\sigma_8-\sigma_{14}$).
\item
 ${\leadstoBoundedThreeStarExt {\Mtwo\cdot M} {\sigma_3}  {\sigma_{20}}  {\sigma_{21}}}$.
 \item
${\WithPub {\Mtwo\cdot M}    {\sigma_{21}}  {\sigma_{25}} {\sigma_{23}}}$. \ \ \ 
Public method call from  external state ${\sigma_{21}}$ into internal state $\sigma_{23}$, and returning to external state $\sigma_{25}$.
 \item
  ${\leadstoBoundedThreeStarExt {\Mtwo\cdot M} {\sigma_3}  {\sigma_{25}}  {\sigma_{28}}}$.
\ \ \ 
  Purely external execution from $\sigma_{25}$ to $\sigma_{28}$, scoped by ${\sigma_3}$.
\end{enumerate}


\begin{definition}
\label{def:exec:sum}
For any module $M$  where $M$ is the internal module, external modules $\Mtwo$, and states $\sigma\bd$,  $\sigma$,  $\sigma_1$, ... $\sigma_n$, and $\sigma'$, we define:

\begin{enumerate}

% 1
\item
 ${\leadstoBoundedThreeStarExt {\Mtwo\cdot M} {\sigma\bd}  {\sigma}  {\sigma'}}$ \ \ \ \ \   $\triangleq$ \ \ 
$
\begin{cases}
M, \sigma  \models  \extThis\  \wedge\  \\
[ \ \ \ 
\sigma=\sigma' \, \wedge\,  \EarlierS  {\sigma\bd}  {\sigma} \, \wedge\,  \EarlierS  {\sigma\bd}  {\sigma''}\ \ \ \ \ \vee\\
\ \ \ \exists \sigma''[\,  \leadstoBoundedThree {\Mtwo\cdot M} {\sigma}  {\sigma\bd}   {\sigma''} \  \wedge\  
{\leadstoBoundedThreeStarExt {\Mtwo\cdot M} {\sigma\bd}  {\sigma''}  {\sigma'}}\, ] \ \ \ ]
\end{cases}
$

% 2
\item
${\WithPub {\Mtwo\cdot M}    {\sigma}  {\sigma'} {\sigma_1}}$ \  \ \  \ \ \ \ \ \ \ \ $\triangleq$ \ \ 
$\begin{cases}
M, \sigma  \models \extThis \ \wedge \\
\exists   \sigma_1'\ [ \ \   \leadstoBoundedThree  {\Mtwo\cdot M} {\sigma} {\sigma}  {\sigma_{1}}\, \wedge\,  M, \sigma_1 \models \pubMeth \ \wedge \\ 
\strut \ \ \ \ \  \ \ \ \ \ \   \leadstoBoundedStarFin {\Mtwo\cdot M} {\sigma_1}  {\sigma_1'}  \ \wedge \   \leadstoBounded  {\Mtwo\cdot M} {\sigma_1'}      {\sigma'} \ \ ] 
\end{cases}
$

% 3
\item
$\WithExtPub {\Mtwo\cdot M} {\sigma\bd}  {\sigma}  {\sigma'} {\epsilon}$ \ \      \  $\triangleq$ \ \ 
$\leadstoBoundedThreeStarExt {\Mtwo\cdot M} {\sigma\bd}  {\sigma}  {\sigma''}$

% 4 
\item
\label{four:defg23a}
$\WithExtPub {\Mtwo\cdot M} {\sigma\bd}  {\sigma}  {\sigma'} {\sigma_1}$  \ \ \  $\triangleq$ \ \ 
$\exists \sigma_1',\sigma_2'.  
\begin{cases}
 \ \   {\leadstoBoundedThreeStarExt {\Mtwo\cdot M} {\sigma\bd}  {\sigma}  {\sigma_1'}}\ \wedge\ 
{\WithPub {\Mtwo\cdot M}    {\sigma_1'}  {\sigma_2'} {\sigma_1}}  \ \ \wedge \\
 \ \  {\leadstoBoundedThreeStarExt {\Mtwo\cdot M} {\sigma\bd}  {\sigma_2'}  {\sigma'}}   \\
  \end{cases}$

 
 %5
\item
\label{four:defg23}
$\WithExtPub {\Mtwo\cdot M} {\sigma\bd}  {\sigma}  {\sigma'} {\sigma_1...\sigma_n}$   \ \  $\triangleq$ \ \ 
$\exists \sigma_1'.[ \  \
 \WithExtPub {\Mtwo\cdot M} {\sigma\bd}  {\sigma}  {\sigma_1'} {\sigma_1} 
  \ \wedge \ 
    {\WithExtPub {\Mtwo\cdot M} {\sigma\bd}  {\sigma_1'}  {\sigma'} {\sigma_2...\sigma_n} }   \  \ ]
$

% 6
\item
\label{six:g23}
$\leadstoBoundedExtPub {\Mtwo\cdot M}    {\sigma}  {\sigma'} $    \ \ \   \ \ \  \ \ \ \   \ \ \ \  $\triangleq$   \ \ 
 $ \exists n\!\in\! \mathbb{N}. \exists \sigma_1,...\sigma_n. \ \WithExtPub {\Mtwo\cdot M} {\sigma}  {\sigma}  {\sigma'} {\sigma_1...\sigma_n} 
$
\end{enumerate}
\end{definition}

\vspace{.1cm}

Note   that 
${\leadstoBoundedThreeStarExt {\Mtwo\cdot M} {\sigma\bd}  {\sigma}  {\sigma'}}$ implies that $\sigma$ is external, but does not
imply that $\sigma'$ is external.
${\leadstoBoundedThreeStarExt {\Mtwo\cdot M} {\sigma}  {\sigma}  {\sigma'}}$. 
On the other hand, $\WithExtPub {\Mtwo\cdot M} {\sigma\bd}  {\sigma}  {\sigma'} {\sigma_1...\sigma_n}$ implies that $\sigma$ and $\sigma'$ are external, and  $\sigma_1$, ... $\sigma_1$  are internal and public.
Finally, note that   in part (\ref{six:g23}) above it is possible that $n=0$, and so 
$\leadstoBoundedExtPub {\Mtwo\cdot M}    {\sigma}  {\sigma'} $  also holds when
Finally, note that the decomposition used in (\ref{four:defg23}) is not unique, but since we only care for the public states this is of no importance.

\vspace{.2cm}

Lemma \ref{lemma:external_breakdown:term} says that\\
\begin{enumerate}
\item
Any terminating execution which starts at an external state ($\sigma$) consists of a number of external states interleaved with another number of terminating calls to public methods.
\item
Any execution execution which starts at an external state ($\sigma$) and reaches another state ($\sigma'$) also consists of a number of external states interleaved with another number of terminating calls to public methods, which may be followed by a call to some public method (at $\sigma_2$), and from where another execution, scoped by $\sigma_2$  reaches $\sigma'$.
\end{enumerate}


 \begin{auxLemma}
\label{lemma:external_breakdown:term}[Summarised Executions]
For   module $M$, modules $\Mtwo$, and states $\sigma$, $\sigma'$:
\\
\\
If $M,\sigma \models \extThis$, then
\begin{enumerate}
\item
\label{lemma:external_breakdown:term:one}
$\leadstoBoundedStarFin {M\cdot \Mtwo}  {\sigma}  {\sigma'}  \ \ \  \ 
\Longrightarrow \ \ \  \ \leadstoBoundedExtPub {\Mtwo\cdot M}    {\sigma}  {\sigma'}$
\item
\label{lemma:external_breakdown:two}
$\leadstoBoundedStar  {M\cdot \Mtwo}  {\sigma}  {\sigma'}  \ \ \  \ \ \  
\Longrightarrow$ 

\begin{enumerate}
\item
$\strut \ \ \ \ \ \ \ \    \leadstoBoundedExtPub {\Mtwo\cdot M}    {\sigma}  {\sigma'}\ \ \ \  \vee$
\item
$\strut \ \ \ \ \ \ \ \    \exists \sigma_c,\sigma_d.[\ 
\leadstoBoundedExtPub {\Mtwo\cdot M}    {\sigma}  {\sigma_c} 
\wedge\ \leadstoBounded  {\Mtwo\cdot M}    {\sigma_c}  {\sigma_d} 
\wedge \ M, \sigma_c \models \pubMeth \wedge \leadstoBoundedStar  {\Mtwo\cdot M}    {\sigma_d}  {\sigma'} \ ]
$
\end{enumerate}
\end{enumerate}
\end{auxLemma}
 



\begin{auxLemma}
\label{lemma:external_exec_preserves_more}[Preservation of Encapsulated Assertions]
For any module $M$, modules $\Mtwo$,  assertion  $A$, and 
% variables $\overline x$, and addresses $\overline \alpha$,
 states $\sigma\bd$, $\sigma$, $\sigma_1$ ... $\sigma_n$, $\sigma_a$, $\sigma_b$ and $\sigma'$:

\noindent
If

\noindent
 $\strut \hspace{.5cm} M \vdash \encaps A \   \wedge   \ fv(A)=\emptyset \  \wedge \ 
\satDAssertFrom M {\sigma} k A \ \wedge \ k \leq \DepthSt {\sigma\bd}$. 

\noindent
Then

\begin{enumerate}
% ΟΝΕ
\item
\label{lemma:external_exec_preserves_more:one}
$  M, \sigma  \models \extThis \ \wedge \  \leadstoBoundedThree  {\Mtwo\cdot M} {\sigma} {\sigma\bd}  {\sigma'} 
\ \ \Longrightarrow \ \ \ \satDAssertFrom M {\sigma'} k A$

% TWO
\item
$   \leadstoBoundedThreeStarExt {\Mtwo\cdot M} {\sigma\bd}  {\sigma}  {\sigma'} 
\ \ \Longrightarrow \ \ \ \satDAssertFrom M {\sigma'} k A$

% THREE
\item
\label{lemma:external_exec_preserves_more:three}
$ \WithExtPub {\Mtwo\cdot M} {\sigma\bd}  {\sigma}  {\sigma'} {\sigma_1...\sigma_n}\ \ \wedge $\\
 $\strut \ \ \ \  \  \forall i\in [1..n]. \forall \sigma_{f}.[ \ \  \satDAssertFrom M {\sigma_i} k A  \ \wedge \  \leadstoBoundedStarFin {M\cdot \Mtwo}  {\sigma_i}  {\sigma_{f}} \ 
\ \ \Longrightarrow \ \  \satDAssertFrom M {\sigma_f} k A \ ]$\\
$\Longrightarrow $
\\
 $\strut \ \ \ \  \ \satDAssertFrom M {\sigma'} k A $ 
 \\
  $\strut \ \ \ \  \  \wedge $
  \\
 $\strut \ \ \ \  \  \forall i\in [1..n].   \satDAssertFrom M {\sigma_i} k A $
 \\
 $\strut \ \ \ \  \  \wedge $
  \\
 $\strut \ \ \ \  \  \forall i\in [1..n]. \forall \sigma_{f}.[ \ \    \leadstoBoundedStarFin {M\cdot \Mtwo}  {\sigma_i}  {\sigma_{f}} \ 
\ \ \Longrightarrow \ \  \satDAssertFrom M {\sigma_f} k A \ ]$ 


\end{enumerate}

\end{auxLemma}

\noindent
\textbf{Proof Sketch}

\begin{enumerate}
\item
 is proven by Def. of $\encaps{\_}$ and the fact $\DepthSt {\sigma'} \geq \DepthSt {\sigma\bd}$ and therefore $k\leq  \DepthSt {\sigma'}$.
In particular, the step $\leadstoBoundedThree  {\Mtwo\cdot M} {\sigma} {\sigma\bd}  {\sigma'}$ may push or pop a frame onto $\sigma$.
If it pops a frame, then $\satDAssertFrom M {\sigma'} k A $ holds by definition.
If is pushes a frame, then $M, \sigma' \models A$, by lemma \ref{lem:encap-soundness}; this gives that $\satDAssertFrom M {\sigma'} k A $.

\item   by induction on the number of steps in $ \leadstoBoundedThreeStarExt {\Mtwo\cdot M} {\sigma\bd}  {\sigma}  {\sigma'} $, and using (1).

\item
 by induction on the number of states   appearing in ${\sigma_1...\sigma_n}$, and using (2).
\end{enumerate}

\textbf{End Proof Sketch}



%%%%%%%%%%%%%%%%%%%%%%%%%%%%%%%%%%%%%%%%%%%%%%%%%%%%%%%%%%%%%%%%%%%%%%%
\subsection{Proof Sketch of Theorem \ref{t:quadruple:sound} }
\label{s:app:proof:sketch;quadruples}
\noindent
\vspace{.2cm}
  {\textbf{Proof Sketch}} 


\noindent
Take any $M$, $\Mtwo$, with\\ 
$\strut \ \ \hspace{2.3cm} \ \ (1) \ \ \vdash M $.
\\
We will prove that\\
$\strut \ \ \hspace{2.3cm} \ \ (*)\ \ \forall \sigma, A, A', A''.$\\
$\strut \ \ \hspace{2.3cm} \ \ \ \ \  \ \ [ \ M\ \vdash\  \quadruple {A} {\sigma.\prg{cont}} {A'} {A''}  \ \ \Longrightarrow \ \    M\ \modelsD\  \quadruple {A} {stmt} {A'} {A''}\ ]$.\\
by induction on the well-founded ordering  $\_ \ll_{M,\Mtwo}  \_$.
\\
Take $\sigma$, $A$, $A'$, $A''$, $\overline z$, $\overline \alpha$, $\sigma'$, $\sigma''$  arbitrary. Assume that\\
$\strut \ \ \hspace{2.3cm} \ \ (2) \ \ M\ \vdash\  \quadruple {A} {\sigma.\prg{cont}} {A'} {A''}$\\
$\strut \ \ \hspace{2.3cm} \ \ (3) \ \ \fv(A)\setminus \vs(\sigma.\prg{cont} ) = \overline z$\\
$\strut \ \ \hspace{2.3cm} \ \ (4) \ \ \satDAssertFrom M  \sigma k   A[\overline{\alpha/z}]$\\
To show\\
$\strut \ \ \hspace{2.3cm} \ \ (\alpha) \ \    \leadstoBoundedStarFin {\Mtwo\cdot M}  {\sigma}  {\sigma'}\ \ \Longrightarrow\ \     \satDAssertFrom M  {\sigma'} k   A'[\overline{\alpha/z}]$\\
$\strut \ \ \hspace{2.3cm} \ \ (\beta) \ \    \leadstoBoundedStar  {\Mtwo\cdot M}  {\sigma}  {\sigma''}\ \ \ \Longrightarrow\ \     \satDAssertFrom M  {\sigma''}  k  \extThis \rightarrow A''[\overline{\alpha/z}]$
 
 \vspace{.2cm}
\noindent
We proceed by case analysis on the  rule applied in the last step of the proof of (2). We only describe some cases.

\begin{description} 
 
 \item[{\sc{mid}}] 
 
 By Theorem \ref{l:triples:sound}.

\newcommand{\SPS}{\strut \ \ \hspace{0.5cm} \ \ }
 
\item[{\sc{sequ}}] 
Therefore, there exist statements $stmt_1$ and $stmt_2$, and assertions  $A_1$, $A_2$ and $A''$, so that $A_1\txteq A$, and $A_2 \txteq A'$, and $\sigma.\prg{cont}\txteq  stmt_1; stmt_2$, and $\overline z = \fv(A) \setminus (\vs(stmt_1) \cup \vs(stmt_2))$, and
the proof of  $(2)$ %$M\ \vdash\  \{\, A \,  \}\ s_1;s_2\  \{\, A' \, \}$ in $P$ 
has as its immediate predecessors proofs for \\
$\SPS (5)\ \ M\ \vdash\  \quadruple {A_1} {stmt_1} {A_2} {A''}$,\ \ \ \  and\\
$\SPS (6)\ \  M\ \vdash\  \quadruple {A_2} {stmt_2} {A_3} {A''}$.

Define $\sigma_1$ as  $\sigma_1 \triangleq \sigma[\prg{cont} \mapsto stmt_1]$.
Therefore,  $(A_1,\sigma_1,A_2, A'') \ll_{M,\Mtwo} (A_1,\sigma,A_3, A'')$. 

\vspace{.1cm}
Proving $(\alpha)$. Assume that $\leadstoBoundedStarFin {\Mtwo\cdot M}  {\sigma}  {\sigma'}$. Then, by lemma \ref{lemma:subexp} we obtain that there exists a $\sigma_1'$ such that $\leadstoBoundedStarFin {\Mtwo\cdot M}  {\sigma_1}  {\sigma_1'}$,
and $\leadstoBoundedStarFin {\Mtwo\cdot M}  {\sigma_2}  {\sigma'}$, where $\sigma_2 \triangleq \sigma_1'[\prg{cont} \mapsto stmt_2]$. Moreover,  $(A_2,\sigma_2,A_3,A'') \ll_{M,\Mtwo} (A_1,\sigma,A_3, A'')$.

We will use the following abbreviations: \\
$\SPS (7)\ \ \overline {z_3}\triangleq\vs(stmt_1)\setminus\vs(stmt_2) \ \ \ \ \ \overline {z_4}\triangleq\vs(stmt_2)\setminus\vs(stmt_1) $.\\
Then we have\\
$\SPS  (8)\ \ \fv(A)\setminus\vs(stmt_1) = \overline {z}\cup \overline{z_4}  \ \ \ \ \ 
\fv(A)\setminus\vs(stmt_2) = \overline {z}\cup\overline{z_3} $.
\\
 From (4), and Lemma \ref{lemma:addr:expr}, we obtain\\ 
 $\SPS (9) \ \ \satDAssertFrom M  {\sigma_1} k   {A_1[\overline{\alpha/z}][\overline{{\interpret {\sigma_1} {z_4}} /z_4}]}$\\
 We apply the induction hypothesis, and obtain\\
  $\SPS (10) \ \ \satDAssertFrom M  {\sigma_1'} k   {A_2[\overline{\alpha/z}][\overline{{\interpret {\sigma_1} {z_4}} /z_4}]}$\\
By Lemmas \ref{l:var:unaffect} and \ref{l:assrt:unaffect} and (7) we have that ${\overline{{\interpret {\sigma_1} {z_4}} /z_4}]}$=${\overline{{\interpret {\sigma_1'} {z_4}} /z_4}]}$=${\overline{{\interpret {\sigma_2} {z_4}} /z_4}]}$, and this, applied to (10) and again with Lemma \ref{lemma:addr:expr} gives us\\
 $\SPS (11) \ \ \satDAssertFrom M  {\sigma_2} k   {A_2[\overline{\alpha/z}]}$\\
We apply Lemma \ref{lemma:addr:expr} again, and obtain\\
 $\SPS (12) \ \ \satDAssertFrom M  {\sigma_2} k   {A_2[\overline{\alpha/z}][{\overline{{\interpret {\sigma_2} {z_3}} /z_3}}]}$\\
 We apply again the induction hypothesis, and obtain\\
  $\SPS (13) \ \ \satDAssertFrom M  {\sigma'} k   {A_3[\overline{\alpha/z}][\overline{{\interpret {\sigma_1} {z_3}} /z_3}]}$\\
By similar argument as earlier, we obtain:\\
 $\SPS (\alpha') \ \ \satDAssertFrom M  {\sigma'} k   {A_3[\overline{\alpha/z}]}$ 
 
 \vspace{.1cm}
Proving $(\beta)$.
Let us now take an arbitrary $\sigma''$ such that   $\leadstoBoundedStar  {\Mtwo\cdot M}  {\sigma}  {\sigma''}$. By
lemma  \ref{lemma:subexp}, either $\leadstoBoundedStar  {\Mtwo\cdot M}  {\sigma_1}  {\sigma''}$ or
$\leadstoBoundedStar  {\Mtwo\cdot M}  {\sigma_2}  {\sigma''}$. 

In the first case, where $\leadstoBoundedStar  {\Mtwo\cdot M}  {\sigma_1}  {\sigma''}$ we apply the inductive hypothesis, and obtain $(\beta')$, ie that \ \ $ M, \sigma'' \models \extThis \rightarrow A''[\overline{\alpha/z}]$.

 The second case is analogous.
 
  \item[{\sc{combine}}]  by induction hypothesis, and unfolding and folding the definitions
  
 \item[{\sc{consequ}}]  using Lemma \ref{l:shallow:deep} part \ref{fourSD}  and axiom \ref{lemma:axiom:enc:assert:ul}

%
%
\item[{\sc{Call\_Int}}]
 
 Therefore, there exist $u$, $y_o$, $C$, $\overline y$,  $A_{pre}$, $A_{post}$, and $A_{mid}$, such that \\
 $\SPS (5) \ \ \sigma.\prg{cont}\txteq u:=y_0.m(\overline y)$,\\
   $\SPS (5a) \ \ \fv(A)\setminus \{ u, y_0, ... y_n \}  =\overline z $
\\ 
$\SPS (6) \  \ \promises  M {\mprepostN {A_{pre}}{D}{m}{y}{D}{A_{post}} {A_{mid}}}$, \\
$\SPS (7) \  \ A \txteq y_0:D ,\overline {y:D}\ \wedge \  A_{pre}[y_0/\prg{this}],\ \  \ \ 
A'  \txteq A_{post}[y_0/\prg{this},u/res],\ \ \ 
%$\strut \hspace{2cm}  \  
A'' \txteq  A_{mid}$. 
\\
Also, \\
$\SPS (8) \ \ \leadstoBounded  {\Mtwo\cdot M}  {\sigma}  {\sigma_1}$, \\
 where \\
$\SPS (8a) \ \ \ \sigma_1\triangleq (\PushSLong { (\prg{this}\mapsto {\interpret{\sigma} {y_0}},{\overline{y \mapsto {\interpret{\sigma} {y}}}})}\sigma [\prg{cont}\mapsto stmt_m]$, \\% \\  and where\\
$\SPS (8b) \ \ \   \prg{mBody}(m,D,M)=\overline{y:D}\{\    stmt_m\ \}$ .\\
By (1), (7), and definition of $\vdash M$ in Section \ref{sect:wf}, we obtain\\
$\SPS (9) \ \ M \vdash  \quadruple { \ \prg{this}:D,\overline{y:D}\, \wedge\, A_{pre}\  } {\ stmt_m } {\ A_{post}\ } {A_{mid}}$.\\
From (8) and (9) we obtain  \\
$\SPS (10) \ \ (\prg{this}:D,\overline{y:D}\, \wedge\, A_{pre},\sigma_1,A_{post}, A_{mid}) \ll_{M,\Mtwo} (A,\sigma,A', A'')$. 
\\
We want to apply Lemma   \ref{l:calls} part \ref{l:calls:callee:one}. For this, we take
\\
$\SPS (11a) \ \  \overline {v_1} \triangleq \overline y$, \ \ $\overline {v_2} \triangleq  \prg{this}$, \ \ $\overline {v_4} \triangleq  y_0$, 
%and  $\overline{v_3} \triangleq \fv(y_0:D ,\overline {y:D}\ \wedge \  A_{pre}) \setminus \{ \overline {v_1}, \overline {v_2} \}$
\\
These definitions  \footnote{Also, because receiver and argument have ti be pairwise disjoint} give that 
\\
$\SPS (11b) \ \  \overline {v_1}\cap   \overline{v_2} = \emptyset $
\\
% Moreover, because of and the wellformedness requirements of specifications,  we also have
%$\SPS (10c) \ \ \overline {v_1}\cap  \overline{v_3} = \emptyset$.\\
Because of (8a), we have that \\
$\SPS (11c) \ \ \overline {\interpret {\sigma_1} {v_1} } = \overline {\interpret {\sigma} {v_1} }$, \ \ and\ \  $\overline {\interpret {\sigma_1}  {v_2} } = \overline {\interpret {\sigma} {v_4} }$,\\
Because of (5a), (7)  and because $u\notin \fv(y_0:D ,\overline {y:D}\ \wedge \  A_{pre})$\footnote{TODO: refer to the appropriate definition}, we obtain that \\
$\SPS (11d) \ \ \fv(y_0:D ,\overline {y:D}\ \wedge \  (A_{pre}[\prg{this}/y_0]) \setminus \{ \overline {v_1}, \overline {v_2} \} =  \overline z$.    
\\
From   (11a)-(11d) we see that the requirements  \ref{l:calls:r:one}-\ref{l:calls:r:four} of Lemma \ref{l:calls}  are satisfied.  Moreover: \\
$\SPS (11e) \ \ \ (\prg{this}:D,\overline{y:D}\, \wedge\, A_{pre})[y_0/\prg{this}] \txteq  y_0:D,\overline{y:D}\, \wedge\, (A_{pre}[y_0/\prg{this}] )$\\
 Therefore, with (7), (4), (11a) - (11e) and  
 Lemma   \ref{l:calls} part \ref{l:calls:callee:one} we obtain\\
$\SPS (12) \ \  \satDAssertFrom M  {\sigma_1} {k}   {(\prg{this}:D,\overline{y:D}\, \wedge\, A_{pre})[\overline{\alpha/z}]}$.

 \vspace{.1cm}
Proving $(\alpha)$. Assume that   $\leadstoBoundedStarFin  {\Mtwo\cdot M}  {\sigma}  {\sigma'}$. Then, by the operational semantics, we obtain that 
there exists state $\sigma_1'$, such that \\
$\SPS (13) \ \ \leadstoBoundedStarFin  {\Mtwo\cdot M}  {\sigma_1}  {\sigma_1'}$ \\
$\SPS (14) \ \ \sigma'=(\sigma_1'\popSymbol)[u \mapsto {\interpret {\sigma_1'} {res}}][\prg{cont}\mapsto \epsilon]$.
\\
We want to apply the induction hypothesis on (12) and (13). For this, we need to determine the value of  $\fv(\prg{this}:D,\overline{y:D}\, \wedge\, A_{pre})  \setminus \vs(stmts_m)$.  We do this below. From (5a) and (7) we obtain:\\
$\SPS (15a) \ \ \fv(\prg{this}:D,\overline{y:D}\, \wedge\, A_{pre}) =  \{ y_0,\, \overline{y},\, \overline z \} $\\
With    (15a) and (7) and (5a) \\
$\SPS (15b) \ \  \fv(y_0:D,\overline{y:D}\, \wedge\, A_{pre}[y_0/\prg{this}) =  \{ \prg{this},\, \overline{y},\,  \overline z \} $
\\
From (15b), {\sc{Method}}, we have \\
$\SPS (15c) \ \  (\vs(stmt_m) \ \{ \prg{this},\, \overline y \})\wedge \fv(A_{pre}) =  \emptyset$
\\
From (15a) and (15c) we obtain;\\
$\SPS (15c) \ \  \fv(\prg{this}:D,\overline{y:D}\, \wedge\, A_{pre})  \setminus \vs(stmts_m) =  \overline {z'}, \overline {z}$
\\
for some $\overline {z'}$ such that
\\
$\SPS (15c)  \ \ \overline {z'} \subseteq  \{ \prg{this},\, \overline{y} \}$.
\\
From (12) and Lemma\footnote{The lemma that allow substitions of variables for their values}, we obtain:
\\
$\SPS (16) \ \  \satDAssertFrom M  {\sigma_1} {k}   {(\prg{this}:D,\overline{y:D}\, \wedge\, A_{pre})[\overline{\alpha/z}][\overline{\interpret {\sigma_1} {z'} /z'}]}$.
\\
We now apply the induction hypothesis on (9), (13), (15c) and (16), and obtain:
\\ 
$\SPS (17) \ \  \satDAssertFrom M  {\sigma_1'} {k}   {(A_{post}))[\overline{\alpha/z}][\overline{\interpret {\sigma_1} {z'} /z'}]}$.
\\
Because of Lemma \footnote{add the lemma that says no assignment to formal parameters, and perhaps one that allows the removal of replacements}, we also obtain that:
\\
$\SPS (18) \ \  \satDAssertFrom M  {\sigma_1'} {k}   {(A_{post})[\overline{\alpha/z}]}$.
\\
We now want to apply  \ref{l:calls} part \ref{l:calls:caller:one}. For this, we take
\\
$\SPS (19a) \ \  \overline {v_1} \triangleq \overline y$, \ \ $\overline {v_2} \triangleq  y_0, u$, \ \ $\overline {v_4} \triangleq  \prg{this}, res$, \\ $\overline {v_5} \triangleq  \prg{u}$, and
$\overline \alpha \triangleq \interpret {\sigma_1'} {res}$.
\\
This construction gives that  \\
$\SPS (19b) \ \  \overline {v_1}\cap   \overline{v_2} = \emptyset $
\\
and that
\\
$\SPS (19c) \ \ \overline {\interpret {\sigma_1'} {v_1} } = \overline {\interpret {\sigma'} {v_1} }$, \ \ and\ \  $\overline {\interpret {\sigma_1'}  {v_2} } = \overline {\interpret {\sigma'} {v_4} }$.
\\
We also see that \\
$\SPS (19d) \ \ k \leq \DepthSt {\sigma'}$
\\
We now apply Lemma \ref{l:calls} part \ref{l:calls:caller:one}, and obtain \\
$\SPS (20) \ \ \satDAssertFrom M  {\sigma_1'} {k}   {(A_{post})[y_0,u/\prg{this},res][\overline{\alpha/z}]}$.





 \vspace{.1cm}
Proving $(\beta)$. Take a $\sigma''$. Assume that\\
%$\SPS (12) \ \ \leadstoBoundedStar   {\Mtwo\cdot M}  {\sigma}  {\sigma''}$\\
%$\SPS (13) \ \ \ {\Mtwo\cdot M}, \sigma'' \models \extThis$.\\
%Then, from (12) and (8) we also obtain that\\
%$\SPS (14) \ \ \leadstoBoundedStar   {\Mtwo\cdot M}  {\sigma_1}  {\sigma''}$\\
%By (10), application of the induction hypothesis on (9),  (12), and (13), we obtain that\\
%$\SPS (\beta') \ \  \satDAssertFrom M  {\sigma''} {k}   {A_{mid}[\overline{\alpha/z}]}$.\\
%and using (7) we are done.
%


\item[{\sc{Call\_Int\_Adapt}}] is similar to {\sc{Call\_Int}}. 
We highlight the differences in \npgreen{green}.
\\
Therefore, there exist there exist $u$, $y_o$, $C$, $\overline y$,  $A_{pre}$, $A_{post}$, and $A_{mid}$, such that \\
 $\SPS (5) \ \ \sigma.\prg{cont}\txteq u:=y_0.m(\overline y)$,  \ \ \ \ $\overline z = \fv(A)\setminus \{ u, y_0, ... y_n \}$
\\ 
$\SPS (6) \  \ \promises  M {\mprepostN {A_{pre}}{D}{m}{y}{D}{A_{post}} {A_{mid}}}$, \\
$\SPS (7) \  \ A \txteq y_0\!:\!D,\overline {y:D}\ \wedge \ \PushASLongG {({y_0, \overline y})} {(A_{pre} [y_0/\prg{this}])}$,\\
$\SPS \ \ \ \ \  \ A'  \txteq \PushASLongG  {({y_0, \overline y})} {(A_{post}[y_0/\prg{this},u/res])},\ \ \ \ \ \ 
  \  A'' \txteq  A_{mid}$. 
\\
Also, \\
$\SPS (8) \ \ \leadstoBounded  {\Mtwo\cdot M}  {\sigma}  {\sigma_1}$, \\
 where \\
$\SPS (8a) \ \ \ \sigma_1\triangleq (\PushSLong { (\prg{this}\mapsto {\interpret{\sigma} {y_0}},{\overline{y \mapsto {\interpret{\sigma} {y}}}})}\sigma [\prg{cont}\mapsto stmt_m]$, \ \ \ \  \ \  and where\\
$\SPS (8b) \ \ \   \prg{mBody}(m,D,M)=\overline{y:D}\{\    stmt_m\ \}$ .\\
By (1), (7), and definition of $\vdash M$ in Section \ref{sect:wf}, we obtain\\
$\SPS (9) \ \ M \vdash  \quadruple { \ \prg{this}:D,\overline{y:D}\, \wedge\, A_{pre}\  } {\ stmt_m } {\ A_{post}\ } {A_{mid}}$.\\
From (8) and (9) we obtain  \\
$\SPS (10) \ \ (\prg{this}:D,\overline{y:D}\, \wedge\, A_{pre},\sigma_1,A_{post}, A_{mid}) \ll_{M,\Mtwo} (A,\sigma,A', A'')$. 
\\
By (1), (7)   and Lemma   \ref{l:calls} part \npgreen{\ref{l:calls:callee:two}},  we obtain\\
% $\SPS (11) \ \  \satDAssertFrom M  {\sigma_1} {\npgreen{\DepthSt {\sigma_1}}}  {(\prg{this}:D,\overline{y:D}\, \wedge\, (A_{pre}[\overline{\alpha/z}])}$.
$\SPS (11) \ \ \npgreen{ M, \sigma_1 \models ...}$

 \vspace{.1cm}
Proving $(\alpha)$. Assume that   $\leadstoBoundedStarFin  {\Mtwo\cdot M}  {\sigma}  {\sigma'}$. Then, by the operational semantics, we obtain that 
there exists state $\sigma_1'$, such that \\
$\SPS (12) \ \ \leadstoBoundedStarFin  {\Mtwo\cdot M}  {\sigma_1}  {\sigma_1'}$ \\
$\SPS (13) \ \ \sigma'=(\sigma_1'\popSymbol)[u \mapsto {\interpret {\sigma_1'} {res}}][\prg{cont}\mapsto \epsilon]$.
\\
Using (9), (10), and because by Barendregt we can assume that $\fv(A)\cap \vs(stmt_m) = \emptyset$, we can apply the inductive hypothesis, and obtain\\
$\SPS (14) \ \  \satDAssertFrom M  {\sigma_1'} {\npgreen{\DepthSt {\sigma_1'}}}    {\prg{this}:D,\overline{y:D}\, \wedge\, (A_{post}[\overline{\alpha/z}])}$.
\\
By (13), (14) and Lemma  \ref{lemma:push:ass:state} part \ref{lemma:push:ass:state:two}, and $\models \sigma_1'$, we obtain\\
$\SPS (16) \ \  \satDAssertFrom M  {\sigma'}  {\npgreen{\DepthSt {\sigma'}}}  {\npgreen {\PushAS y {(A_{post}[y_0/ \prg{this},u/res][\overline{\alpha/z}])}}}$.\\
 and using Lemma \npgreen{\ref{l:adapt:stable} and \ref{l:shallow:deep} part \ref{fiveSD}}, and \npgreen{because $\overline z \cap \overline y = \emptyset$}   we obtain \\
$\SPS (\alpha') \ \  \satDAssertFrom M  {\sigma'}  {\npgreen{k}}  {\npgreen {(\PushAS y {(A_{post}[y_0/ \prg{this},u/res])})}[\overline{\alpha/z}]}$

 \vspace{.1cm}
Proving $(\beta)$. Similar.

\item[{\sc{Call\_Ext\_Adapt}}] is similar to {\sc{Call\_Int\_Adapt}}. with the difference that it also uses Lemma \ref{lemma:external_exec_preserves_more}, part (3).
 It uses  Lemmas
 \ref{lemma:push:ass:state} part \ref{lemma:push:ass:state:one}, then \ref{lemma:external_exec_preserves_more}, part (3),  and   \ref{l:adapt:stable} and \ref{l:shallow:deep}
 then again \ref{lemma:push:ass:state}, part \ref{lemma:push:ass:state:two}.
 To fulfill the second premise of Lemma  \ref{lemma:external_exec_preserves_more}, part (3), we make use of the fact that $\vdash M$, and apply the rule {\sc{Method}}, and the inductive hypothesis.
\end{description}
\noindent
\vspace{.1cm}
  {\textbf{End Proof Sketch}} 

\subsection{Proof Sketch of Theorem  \ref{thm:soundness}}

\label{s:app:proof:sketch;overall}
 {\textbf{Proof Sketch}}
 By induction on the  cases for the specification $S$. If it is a method spec, then the theorem follows from \ref{t:quadruple:sound}. If it is a conjunction, then by inductive hypothesis.
 \\
 The interesting case is $S \txteq {\TwoStatesN {\overline {x:C}} {A}}$.
 \\
 Assume that 
 $ \satDAssertFrom M  {\sigma} k A[\overline {\alpha/x} ]$, that  $M,\sigma \models \extThis$,
 that $\leadstoBoundedStar  {M\cdot \Mtwo}  {\sigma}  {\sigma'}$, and that $M,\sigma \models \extThis$,
 \\ 
 We want to show that $ \satDAssertFrom M  {\sigma'} k A[\overline {\alpha/x} ]$.
 \\
 Then, by lemma
 \ref{lemma:external_breakdown:term}, we obtain that either \\
$\strut \ \ \ \ \ $ (1)\  $\leadstoBoundedExtPub {\Mtwo\cdot M}    {\sigma}  {\sigma'}$, or\\
$\strut \ \ \ \ \ $  (2)\  $\exists \sigma_1,\sigma_2.[\
\leadstoBoundedExtPub {\Mtwo\cdot M}    {\sigma}  {\sigma_1}
\wedge\ \leadstoBounded  {\Mtwo\cdot M}    {\sigma_1}  {\sigma_2}
\wedge \ M, \sigma_2 \models \pubMeth \wedge \leadstoBoundedStar  {\Mtwo\cdot M}    {\sigma_2}  {\sigma'} \ ]$
\\
In Case (1), we apply  \ref{lemma:external_exec_preserves_more}, part (3).  In order to fulfill the second premise of Lemma  \ref{lemma:external_exec_preserves_more}, part (3), we make use of the fact that $\vdash M$,   apply the rule {\sc{Method}}, and theorem \ref{t:quadruple:sound}.
This gives us $ \satDAssertFrom M  {\sigma'} k A[\overline {\alpha/x} ]$
\\
In Case (2), we proceed as in (1) and obtain that $ \satDAssertFrom M  {\sigma_1} k A[\overline {\alpha/x} ]$. Because $M \vdash \encaps A$, we also obtain that 
$ \satDAssertFrom M  {\sigma_2} k A[\overline {\alpha/x} ]$.
Since we are now executing a public method, and because $\vdash M$, we can apply {\sc{Invariant}}, and theorem \ref{t:quadruple:sound}, and obtain $ \satDAssertFrom M  {\sigma'} k A[\overline {\alpha/x} ]$\\
 \vspace{.1cm}
  {\textbf{End Proof Sketch}} 

%\clearpage
% \section{Example: Taming Effects for the Shop/Account}
In Section \ref{s:outline} we introduced a \verb|Shop| that allows clients to make purchases through the
\verb|buy| method, a method that included a method call to an unknown external object (\verb|buyer.pay(...)|).
Here we use our Hoare logic from Section \ref{sect:proofSystem} to prove that the \verb|buy| method
does not expose the \verb|Shop|'s  \verb|Account|, its \verb|Key|, or allow the \verb|Account|'s balance to be illicitly modified. 
More specifically, we prove the following scoped invariants are satisfied by \verb|M|$_{good}$\\
%$\strut \SPSP  S_1\ \  \triangleq \ \ \TwoStatesN {\prg{a}:\prg{Account}}  {\inside{\prg{a}}} $ 
%\\
$\strut  \SPSP  S_1\ \  \triangleq \ \ \TwoStatesN  {\prg{a}:\prg{Account}}  {\inside{\prg{a.key}}} $ 
 \\
%$\strut  \SPSP   S_3\ \  \triangleq \ \ \TwoStatesN {\prg{a}:\prg{Account},\prg{b}:\prg{int}}  {\inside{\prg{a}} \wedge \prg{a.\balance}=\prg{b}}  $
%\\
$\strut  \SPSP  S_2\ \  \triangleq \ \ \TwoStatesN{ \prg{a}:\prg{Account},\prg{b}:\prg{int} } {\inside{\prg{a.key}} \wedge \prg{a.\balance} \geq \prg{b} } $\\
\verb|M|$_{good}$ is given below.
\begin{lstlisting}[mathescape=true, language=Chainmail, frame=lines]
module M$_{good}$
  ...   
  class Shop
    field accnt:Account, invntry:Inventory, clients:[external]    
  
    public method buy(buyer:external, anItem:Item)
      int price = anItem.price
      int oldBlnce = this.accnt.blnce
      buyer.pay(this.accnt, price)     // $\red{\mbox{external\ call}}!$
      if (this.accnt.blnce == oldBlnce+price)  this.send(buyer,anItem)
      else
         buyer.tell("you have not paid me") 
     
      private method send(buyer:external, anItem:Item)  
       ... 
  class Key
  class Account
    field blnce:int 
    field key:Key
    public method transfer(dest:Account, key':Key, amt:int)
      if (this.key==key')
        this.blnce-=amt
        dest.blnce+=amt
     public method set(key':Key)
      if (this.key==null)  this.key=key'
\end{lstlisting}

For brevity we only show that \verb|buy| satisfies our scoped invariants, as the all other methods of 
the \verb|M|$_{good}$ interface are relatively simple, and do not make any external calls. 
Our approach follows the 3 phases outlined in Section \ref{sect:proofSystem}. That is, in
phase 1 we use more an assumed underlying Hoare logic and more traditional Hoare triples to prove the adherence of internal code to
the specification. In phase 2 we use Hoare quadruples to prove external calls adhere
to the specificaiton, and finally in phase 3 we use raise the results from phase 1 and 2 to proved
the entire module satisfies the specification.

\begin{lemma}
$M_{good} \vdash S_1$
\end{lemma}
\begin{proof}
We construct our proof tree using a top down approach.  That is, we start with our goal
$$M_{good} \vdash \TwoStatesN {\prg{a}:\prg{Account}}  {\inside{\prg{\prg{a.key}}}}$$
and apply  \textsc{Invariant-Hope} from Fig. \ref{f:wf}.
From this we are left with a subgoal for each method $m$ in  class $C$ with parameters $\overline{y:D}$ and body $stmt$ in the public interface of $M_{good}$:
\small
\begin{align*}
M_{good} \vdash
		& \{ \ \prg{this}:\prg{C},\, \overline{y:D},\, \prg{a}:\prg{Account}\, \wedge\, 
\external z \wedge\ {\PushASLong {z} {\inside{\prg{a.key}}}}\} \\
		& stmt\ \\
		& \{\ {\PushASLong {z} {\inside{\prg{a.key}}}}\ \}\ ||\ \{\ {\PushASLong {z} {\inside{\prg{a.key}}}} \ 			\} 
\end{align*}
\normalsize
This is relatively simple to prove for methods without external method calls, as we can rely entirely on the Hoare logic extension described in Section \ref{sect:proofSystem}. The only  method within $M_{good}$ that contains 
an external method call is  the \prg{buy} method within the \prg{Shop} class. Thus, we need to prove:
\small
\begin{align*}
\text{(1)}  \ \ \ \ M_{good} \vdash & \{  \prg{this}:\prg{Shop}, \prg{buyer} : \prg{external}, \prg{anItem} : \prg{Item},\, \prg{a}:\prg{Account}\,\wedge\, \external z \wedge\ {\PushASLong {z} {\inside{\prg{a.key}}}} \} \\
		  &stmt\ \\  &\{ {\PushASLong {z} {\inside{\prg{a.key}}}} \} \ \ \  || \ \ \ \{ {\PushASLong {z} {\inside{\prg{a.key}}}} \}
\end{align*}
\normalsize
Where $stmt$ is the body of \prg{buy}. We now construct this proof using the quadruple rules described in \ref{sect:proofSystem}. By our hoare triple extension and the \textsc{Mid} rule that raises triples to quaruples, the first two lines
of \prg{buy} preserve the precondition in (1) above, and further that 
\small
$$
\text{(2)} \ \ \ \ \prg{price} : \prg{int}
$$ 
$$
\text{(3)} \ \ \ \ \prg{this.accnt} : \prg{Account}
$$ 
We now need to prove that the external method call \prg{buyer.pay(this.accnt, price)} protects the \prg{key}. i.e.
\small
\begin{align*}
M_{good} \vdash & \{  \prg{this}:\prg{Shop}, \prg{buyer} : \prg{external}, 
				  \prg{anItem} : \prg{Item},\, \prg{a}:\prg{Account},\,
				  \prg{this.accnt}:\prg{Account},\, \\
				& \prg{price} : \prg{int},\,
				  \wedge\, 
				  \external z \wedge\ {\PushASLong {z} {\inside{\prg{a.key}}}} \} \\
		  		& \prg{buyer.pay(this.accnt, price)}\ \\  
		  		& \{ {\PushASLong {z} {\inside{\prg{a.key}}}} \} \ \ \  || \ \ \ \{ {\PushASLong {z} {\inside{\prg{a.key}}}} \}
\end{align*}
\normalsize
or simplifying the precondition to what we need and applying the ``pop''.
\small
\begin{align*}
\text{(4)} \ \ \ \ 
M_{good} \vdash & \{  \prg{this}:\prg{Shop}, \prg{buyer} : \prg{external}, 
				  \prg{anItem} : \prg{Item},\, \prg{a}:\prg{Account},\,
				  \prg{this.accnt}:\prg{Account},\, \\
				& \prg{price} : \prg{int},\,
				  \wedge\, 
				  \external z \wedge\ 
				  {\protectedFrom{\prg{a.key}}{z}} \} \\
		  		& \prg{buyer.pay(this.accnt, price)}\ \\  
		  		& \{ {\protectedFrom{\prg{a.key}}{z}} \} \ \ \  || \ \ \ \{ {\protectedFrom{\prg{a.key}}{z}} \}
\end{align*}
\normalsize

This is discharged by the rule \textsc{Call\_Ext\_Adapt}, giving us the desired result.

\jm{(julian: I did the below bit because I thought it would be necessary, but I'm not sure that it is. I feel like this might be a problem. Don't we need to know that \prg{a.key} is protected from the arguments to \prg{buyer.pay}?????)}

From the definition of protection \jm{(Julian: we need some proof rules for this)}, 
we have 
$$
\text{(5)} \ \ \ \ \protectedFrom{\prg{a.key}}{\prg{this.accnt}}
$$
since $\prg{this.accnt}:\prg{Account}$ and $\prg{Account} \in M_{good}$.
We also get
$$
\text{(6)} \ \ \ \ \protectedFrom{\prg{a.key}}{\prg{price}}
$$
because \prg{price} is a prmitive and is thus everything is protected from 
it. \jm{(julian: we need to add this.)}

This gives us:
\small
\begin{align*}
\text{(7)} \ \ \ \ 
M_{good} \vdash & \{  \prg{this}:\prg{Shop}, \prg{buyer} : \prg{external}, 
				  \prg{anItem} : \prg{Item},\, \prg{a}:\prg{Account},\,
				  \prg{this.accnt}:\prg{Account},\, \\
				& \prg{price} : \prg{int},\,
				  \wedge\, 
				  \external z \wedge\ 
				  {\protectedFrom{\prg{a.key}}{z}}\, \wedge\, 
				  \protectedFrom{\prg{a.key}}{\prg{price}}\, \wedge\, \\
				& \protectedFrom{\prg{a.key}}{\prg{this.accnt}}\} \\
		  		& \prg{buyer.pay(this.accnt, price)}\ \\  
		  		& \{ {\protectedFrom{\prg{a.key}}{z}} \} \ \ \  || \ \ \ \{ {\protectedFrom{\prg{a.key}}{z}} \}
\end{align*}
\normalsize


\jm{Julian: I don't think below is necessary...}\\
\jm{Julian: I think there might be a problem here. If \prg{buyer.pay} had an external 
object as an
argument, then there would be no way we'd able to prove that \prg{a.key} was 
protected from that external object.}
Now we use the fact the \jm{law of excluded middle} and \textsc{Conseq} to introduce $\prg{a} = \prg{this.accnt}\ \vee\ \prg{a} \not= \prg{this.accnt}$ to the precondition.
\small
\begin{align*}
\text{(5)} \ \ \ \ 
M_{good} \vdash & \{  \prg{this}:\prg{Shop}, \prg{buyer} : \prg{external}, 
				  \prg{anItem} : \prg{Item},\, \prg{a}:\prg{Account},\,
				  \prg{this.accnt}:\prg{Account},\, \\
				& \prg{price} : \prg{int},\,
				  \wedge\, 
				  \external z \wedge\ 
				  {\protectedFrom{\prg{a.key}}{z}}\, \wedge \,
				  \prg{a} = \prg{this.accnt}\ \vee\ \prg{a} \not= \prg{this.accnt} \} \\
		  		& \prg{buyer.pay(this.accnt, price)}\ \\  
		  		& \{ {\protectedFrom{\prg{a.key}}{z}}  \} \ \ \  || \ \ \ \{ {\protectedFrom{\prg{a.key}}{z}} \}
\end{align*}
\normalsize
Now \jm{the traditional  Hoare rule for disjunction} gives us two subgoals:
\small
\begin{align*}
\text{(6.1)} \ \ \ \ 
M_{good} \vdash & \{  \prg{this}:\prg{Shop}, \prg{buyer} : \prg{external}, 
				  \prg{anItem} : \prg{Item},\, \prg{a}:\prg{Account},\,
				  \prg{this.accnt}:\prg{Account},\, \\
				& \prg{price} : \prg{int},\,
				  \wedge\, 
				  \external z \wedge\ 
				  {\protectedFrom{\prg{a.key}}{z}}\, \wedge \,
				  \prg{a} = \prg{this.accnt} \} \\
		  		& \prg{buyer.pay(this.accnt, price)}\ \\  
		  		& \{ {\protectedFrom{\prg{a.key}}{z}}  \} \ \ \  || \ \ \ \{ {\protectedFrom{\prg{a.key}}{z}} \}
\end{align*}
\normalsize
\small
\begin{align*}
\text{(6.2)} \ \ \ \ 
M_{good} \vdash & \{  \prg{this}:\prg{Shop}, \prg{buyer} : \prg{external}, 
				  \prg{anItem} : \prg{Item},\, \prg{a}:\prg{Account},\,
				  \prg{this.accnt}:\prg{Account},\, \\
				& \prg{price} : \prg{int},\,
				  \wedge\, 
				  \external z \wedge\ 
				  {\protectedFrom{\prg{a.key}}{z}}\, \wedge \,
				  \prg{a} \not= \prg{this.accnt} \} \\
		  		& \prg{buyer.pay(this.accnt, price)}\ \\  
		  		& \{ {\protectedFrom{\prg{a.key}}{z}}  \} \ \ \  || \ \ \ \{ {\protectedFrom{\prg{a.key}}{z}} \}
\end{align*}
\normalsize
Both (5.1) and (5.2) can now be discharged in the same manner, since in both cases, \prg{a.key} is protected from the arguments of \prg{buyer.pay}.  Finally, by application of \textsc{Conseq} and \textsc{Call\_Ext\_Adapt} we get our goal.
%\small
%\begin{align*}
%\text{(3)} \ \ \ \ M_{good} \vdash & \{  \prg{this}:\prg{Shop}, \prg{buyer} : \prg{external}, \prg{anItem} : \prg{Item}, \prg{price} : \prg{int} \, \wedge\, 
%		 \prg{a} : \prg{Account}\, \wedge\,  \inside{\prg{a.key}} \}  \\
%		 & \prg{buyer.(this.accnt, price)} \ \{ \inside{\prg{a.key}} \} \ \ \  || \ \ \  \{ \inside{\prg{a.key}} \}
%
%\normalsize
\end{proof}



\begin{lemma}
$M_{good} \vdash S_2$
\end{lemma}
\begin{proof}
We construct our proof tree using a top down approach.  That is, we start with our goal
$$M_{good} \vdash \TwoStatesN {\prg{a}:\prg{Account}, \prg{b} : \prg{int}}  {\inside{\prg{a.key}} \wedge \prg{a.blnce} \geq \prg{b}}$$
and apply  \textsc{Invariant} from Fig. \ref{f:wf}.
From this we are left with a subgoal for each method $m$ in  class $C$ with parameters $\overline{y:D}$ and body $stmt$ in the public interface of $M_{good}$:
\small
\begin{align*}
M_{good} \vdash
		& \{ \ \prg{this}:\prg{C},\, \overline{y:D},\, \prg{a}:\prg{Account}\, \wedge\, 
\external z \wedge\ {\PushASLong {z} {(\inside{\prg{a.key}} \wedge \prg{a.blnce} \geq \prg{b})}}\} \\
		& stmt\ \\
		& \{\ {\PushASLong {z} {(\inside{\prg{a.key}} \wedge \prg{a.blnce} \geq \prg{b})}}\ \}\ ||\ \{\ {\PushASLong {z} {(\inside{\prg{a.key}} \wedge \prg{a.blnce} \geq \prg{b})}} \ 			\} 
\end{align*}
\normalsize
This is relatively simple to prove for methods without external method calls, as we can rely entirely on the Hoare logic extension described in Section \ref{sect:proofSystem}. The only  method within $M_{good}$ that contains 
an external method call is  the \prg{buy} method within the \prg{Shop} class. Thus, we need to prove:
\small
\begin{align*}
\text{(1)} \ \ \ \ M_{good} \vdash & \{  \prg{this}:\prg{Shop}, \prg{buyer} : \prg{external}, \prg{anItem} : \prg{Item} \, \wedge\, \neg\inside{this}\wedge \\ 
		&\neg\inside{\prg{buyer}} \wedge \neg\inside{\prg{anItem}}, \wedge\, \prg{a} : \prg{Account}\, \wedge\,  \inside{\prg{a.key}} \} 
		 stmt\ \{ \inside{\prg{a.key}} \} \ \ \  || \ \ \ \{ \inside{\prg{a.key}} \}
\end{align*}
\normalsize
Where $stmt$ is the body of \prg{buy}. We now construct this proof using the quadruple rules described in \ref{sect:proofSystem}. By our hoare triple extension and the \textsc{Mid} rule that raises triples to quaruples, the first two lines
of \prg{buy} preserve the precondition in (1) above, and further that 
\small
$$
\text{(2)} \ \ \ \ \prg{price} : \prg{int}
$$ 
$$
\text{(3)} \ \ \ \ \prg{this.accnt} : \prg{Account}
$$ 
We now need to prove that the external method call \prg{buyer.pay(this.accnt, price)} protects the \prg{key}. i.e.
\small
\begin{align*}
M_{good} \vdash & \{  \prg{this}:\prg{Shop}, \prg{buyer} : \prg{external}, 
		\prg{anItem} : \prg{Item}, \prg{price} : \prg{int} \, \wedge\, 
		\neg\inside{this}\wedge \\ 
		&\neg\inside{\prg{buyer}} \wedge \neg\inside{\prg{anItem}}, \wedge\, \prg{a} : \prg{Account}\, \wedge\,  \inside{\prg{a.key}} \}  \\
		 & \prg{buyer.(this.accnt, price)} \ \{ \inside{\prg{a.key}} \} \ \ \  || \ \ \  \{ \inside{\prg{a.key}} \}
\end{align*}
\normalsize
or simplifying the precondition to what we need:
\small
\begin{align*}
\text{(3)} \ \ \ \ M_{good} \vdash & \{  \prg{this}:\prg{Shop}, \prg{buyer} : \prg{external}, \prg{anItem} : \prg{Item}, \prg{price} : \prg{int} \, \wedge\, 
		 \prg{a} : \prg{Account}\, \wedge\,  \inside{\prg{a.key}} \}  \\
		 & \prg{buyer.(this.accnt, price)} \ \{ \inside{\prg{a.key}} \} \ \ \  || \ \ \  \{ \inside{\prg{a.key}} \}
\end{align*}
\normalsize
Now, by application of \textsc{Conseq} and \textsc{Call\_Ext\_Adapt} we get our goal.
%\small
%\begin{align*}
%\text{(3)} \ \ \ \ M_{good} \vdash & \{  \prg{this}:\prg{Shop}, \prg{buyer} : \prg{external}, \prg{anItem} : \prg{Item}, \prg{price} : \prg{int} \, \wedge\, 
%		 \prg{a} : \prg{Account}\, \wedge\,  \inside{\prg{a.key}} \}  \\
%		 & \prg{buyer.(this.accnt, price)} \ \{ \inside{\prg{a.key}} \} \ \ \  || \ \ \  \{ \inside{\prg{a.key}} \}
%
%\normalsize
\end{proof}

%$$M_{shop} \vdash \{{\inside{\texttt{a.key}}}\}\ {\texttt{public}\ \texttt{Shop}}::{\texttt{buy}}{(\texttt{buyer} : \texttt{external})}\ \{{\inside{\texttt{a.key}}}\}\ ||\ \{{\inside{\texttt{a.key}}} \}$$
%\end{proof}
%\begin{lemma}
%$$M_{good} \vdash \{{\inside{\texttt{a.key}}}\}\ {\texttt{public}\ \texttt{Shop}}::{\texttt{buy}}{(\texttt{buyer} : \texttt{external}, \texttt{price} : \texttt{int})}\ \{{\inside{\texttt{a.key}}}\}\ ||\ \{{\inside{\texttt{a.key}}} \}$$
%\end{lemma}
%\begin{lemma}
%$$M_{good} \vdash \{{\inside{\texttt{a.key}}}\}\ {\texttt{public}\ \texttt{Shop}}::{\texttt{buy}}{(\texttt{buyer} : \texttt{external}, \texttt{price} : \texttt{int})}\ \{{\inside{\texttt{a.key}}}\}\ ||\ \{{\inside{\texttt{a.key}}} \}$$
%\end{lemma}
%\clearpage
%HERE THE EXAMPLE MORE OR LESS AS WAS WRITTEN, with many comments in red
%\input{appendixExampleOLD}
%%\clearpage
%% \section{Encapsulation}
%%
%\subsection{Proving Encapsulation}
\label{s:encap-proof}

%We start by giving providing the syntax for type contexts in Fig. \ref{f:context-syntax}.
%\begin{figure}[t]
%\[
%\begin{syntax}
%\syntaxElement{\Gamma}{Type Context}
%		{
%		\syntaxline
%				{\emptyset}
%				{\alpha : C,\ \Gamma}
%		\endsyntaxline
%		}
%\endSyntaxElement\\
%\end{syntax}
%\]
%\caption{}
%\label{f:context-syntax}
%\end{figure}
%We construct type contexts out of assertions using the following rules:
%\begin{mathpar}
%\infer
%		{}
%		{\textit{Env}(\alpha : C) = \alpha : C,\ \emptyset}
%		\and
%\infer
%		{}
%		{\textit{Env}(A_1\ \wedge\ A_2) = \textit{Env}(A_1) \cup \textit{Env}(A_2)}
%\end{mathpar}
%\begin{definition}[Assertion Encapsulation]
%For all modules $M$, and assertions $A$, and $A'$ we say $M\ \vdash\ A\ \Rightarrow\ A'$ if and only if M
%\end{definition}

\kjx{
Assertion encapsulation (Definition \ref{def:encapsulation}) is
critical to our approach.  Assertion encapsulation ensures that a
change in satisfaction of an assertion can only depend on computation
\textit{internal} to the module in which the assertion is encapsulated
--- this is related to the footprint of an
assertion \cite{objInvars,encaps}.
If the footprint of an assertion is contained 
within a module, then that assertion is encapsulated,
however there are assertions that are encapsulated by a module 
whose footprint is not contained within the module. 
Specifically, the assertion $\inside{x}$ is not 
contained within an module $M$ since its due to the
universal quantification contained withing 
$\inside{x}$, the footprint consists of portions 
of the heap that are external to $M$. $\inside{x}$ is 
encapsulated by $M$ since if only objects that derive 
from $M$ have access to $x$, it follows that a method call
on $M$ is required to gain access to $x$.
Necessity Logic itself does not depend on the details
of the encapsulation scheme --- only that we can determine
whether an assertion is encapsulated within a particular
part of the program.  For reasons of simplicity, 
we have adopted an encapsulation model for \Loo
based on 
\citeauthor{confined}'s \textit{Confined Types} [\citeyear{confined}]
(and we rely on their proof).
%
% We see no reason why a different
% encapsulation mechanism could not be used instead.
%
% KJX move to related work?
%
Confined types partition the objects accessible to code within a
module, based on those objects' defining classes and modules:
\begin{itemize}
\item instances of non-\enclosed classes %defined in a module
constitute their defining module's encapsulation
boundary \cite{TAME2003},
and may be accessed anywhere.
\item instances of \enclosed classes %defined in the module
are encapsulated \inside their defining module.
\item instances of \enclosed classes defined in \emph{other} modules
are not accessible elsewhere
\item instances of non-\enclosed classes defined in \emph{other}
modules are visible, however methods may only be invoked on such 
objects when the confinement system guarantees the particular instance
is only accessible \inside \emph{this} module.
\end{itemize}
%
\noindent \Loo's Confined Types rely on three syntactic restrictions
to enforce this encapsulation model:
\begin{itemize}
\item \enclosed class declarations must be annotated.
\item \enclosed objects may not be returned by methods of non-\enclosed
classes.
\item Ghost fields 
may be annotated as \prg{intrnl}; if so, they must only refer to objects \inside
their defining module --- i.e.\ either defined directly in that module, or
instances of non-\enclosed classes defined in \emph{other} modules
where those particular instances are only ever accessed within the
defining module.
\end{itemize}
}
\jm[Yes, I think that's right. I'm not 100\% sure that assertion 
encapsulation is defined by the footprint, unless I misunderstand 
footprint. It is possible for an assertion to be encapsulated, 
but depend on external objects. For example $\neg\access{x}{y}$:
if $\inside{y}$ is true, then $\neg\access{x}{y}$ is encapsulated,
even if $x$ is external.]{}

%% Using Confined Types for \Loo means that modules needing to encapsulate
%% assertions must meet the following ownership restrictions: 

%% however we assume several properties enforced by the type system, including 
%% simple ownership properties:
%% \begin{itemize}
%% \item
%% Method calls may not be made to external, non-module methods.
%% \item
%% \jm[]{Classes may be optionally annotated as \enclosed: their instances (``\enclosed'' objects) are marked as \enclosed, and may not be returned by methods of non-\enclosed classes.}
%% \item
%% Ghost fields may be annotated as \prg{intrnl} and thus may only include and be passed references to objects belonging to module internal classes.
%% \end{itemize}
%% %
%% \kjx
%% These rules enforce a simple model 
%%   Instances of classes confined in a module are internal to that module;
%% however, non-confined 

%% %
%% These encapsulation properties are easily enforceable, and we
%% do not define the type system as ownership types have been 
%% well covered in the literature. 
%% We specifically use a simple ownership system to model 
%% encapsulation as the theory has been well established by others, 





\jm[]{
%To assist in the definition of our simple encapsulation system,
We define internally evaluated expressions ($\intrnl{\_}$) 
whose evaluation only inspects internal objects or primitvies (i.e. integers or booleans).}
\jm[]{\begin{definition}[Internally Evaluated Expressions]
For all modules $M$, assertions $A$, and expressions $e$, 
$\satisfies{M}{\givenA{A}{\intrnl{e}}}$ if and only if for all heaps $\chi$, stacks $\psi$, and frames $\phi$
such that $\satisfiesA{M}{(\chi, \phi : \psi)}{A}$, we have for all values $v$, such that $\eval{M}{(\chi, \phi : \psi)}{e}{v}$
then $\eval{M}{(\chi', \phi' : \psi)}{e}{v}$, where 
\begin{itemize}
\item $\chi'$ is the internal portion of $\chi$, i.e. \\
$\chi' = \{\alpha \mapsto o| \alpha \mapsto o \in \chi\ \wedge \ o.(\prg{cname}) \in M \}$ and
\item $\phi'.(\prg{local})$ is the internal portion of the $\phi.(\prg{local})$ i.e. \\
$\phi' = \{x \mapsto v| x \mapsto v \in \chi\ \wedge \ (v \in \IntSet\ \vee\ v = \true\ \vee\ v = false)\ \vee\ (\exists \alpha, \ v = \alpha \wedge \class{(\chi, \phi : \psi)}{\alpha} \in M\}$
\end{itemize}
\end{definition}}


The encapsulation proof system consists of two relations 
\begin{itemize}
\item
Purely internal expressions: $\proves{M}{\givenA{A}{\intrnl{e}}}$ and
\item
Assertion encapsulation: $\proves{M}{\givenA{A}{\encaps{A'}}}$
\end{itemize}

Fig. \ref{f:intrnl} gives proof rules for an expression comprising purely internal objects.
\jm[]{Primitives are $Enc_e$ (\textsc{Enc$_e$-Int}, \textsc{Enc$_e$-Null}, \textsc{Enc$_e$-True}, and \textsc{Enc$_e$-False}).
Addresses of internal objects are $Enc_e$ (\textsc{Enc$_e$-Obj}). Field accesses with internal types of $Enc_e$ expressions
are themselves $Enc_e$ (\textsc{Enc$_e$-Field}). Ghost field accesses annotated as $Enc_e$ on $Enc_e$ 
expressions are themselves $Enc_e$ (\textsc{Enc$_e$-Ghost}).}

\begin{figure}[h]
\footnotesize
\begin{mathpar}
\infer
		{}
		{\proves{M}{\givenA{A}{\intrnl{i}}}}
		\quad(\textsc{Enc$_e$-Int})
		\and
\infer
		{}
		{\proves{M}{\givenA{A}{\intrnl{\nul}}}}
		\quad(\textsc{Enc$_e$-Null})
		\and
\infer
		{}
		{\proves{M}{\givenA{A}{\intrnl{\true}}}}
		\quad(\textsc{Enc$_e$-True})
		\and
\infer
		{}
		{\proves{M}{\givenA{A}{\intrnl{\false}}}}
		\quad(\textsc{Enc$_e$-False})
		\and
\infer
		{
		\proves{M}{A\ \longrightarrow\ \alpha : C}\\
		C\ \in\ M
		}
		{
		\proves{M}{\givenA{A}{\intrnl{\alpha}}}
		}
		\quad(\textsc{Enc$_e$-Obj})
		\and
\infer
		{
		\proves{M}{\givenA{A}{\intrnl{e}}}\\
		\proves{M}{A\ \longrightarrow\ e : C}\\
		[\prg{field}\ \_\ f\ :\ D]\ \in\ M(C).(\prg{flds}) \\
		D\ \in\ M
		}
		{
		\proves{M}{\givenA{A}{\intrnl{e.f}}}
		}
		\quad(\textsc{Enc$_e$-Field})
		\and
\infer
		{
		\proves{M}{\givenA{A}{\intrnl{e_1}}}\\
		\proves{M}{\givenA{A}{\intrnl{e_2}}}\\
%		\proves{M}{\givenA{A}{\intrnl{e}}} \\
		\proves{M}{A\ \longrightarrow\ e_1 : C} \\
		\prg{ghost}\ \prg{intrnl}\ g(x : \_)\{e\} \in M(C).(\prg{gflds})
		}
		{
		\proves{M}{\givenA{A}{\intrnl{e_1.g(e_2)}}}
		}
		\quad(\textsc{Enc$_e$-Ghost})
\end{mathpar}
\caption{Internal Proof Rules}
\label{f:intrnl}
\end{figure}


\jm[]{Fig. \ref{f:asrt-encap} gives proof rules for whether an assertion is encapsulated, that is whether 
a change in satisfaction of an assertion requires interaction with the internal module.
An \prg{Intrl} expression is also an encapsulated assertion (\textsc{Enc-Exp}). A field
access on an encapsulated expression is an encapsulated expression. Binary and ternary operators
applied to encapsulated expressions are themselves encapsulated assertions (\textsc{Enc-=}, \textsc{Enc-+}, \textsc{Enc-<}, \textsc{Enc-If}).
An internal object may only lose access to another object via internal computation (\textsc{Enc-IntAccess}).
Only internal computation may grant external access to an $\wrapped{\_}$ object (\textsc{Enc-Inside}$_1$).
If an object is $\wrapped{\_}$, then nothing (not even internal objects) may gain access
to that object except by internal computation (\textsc{Enc-Inside}$_2$).
If an assertion $A_1$ implies assertion $A_2$, then $A_1$ implies the encapsulation of any assertion that
$A_2$ does. Further, if an assertion is encapsulated, then any assertion that is implied by it is also encapsulated.
These two rules combine into an encapsulation rule for consequence (\textsc{Enc-Conseq}).}

\begin{figure}[h]
\footnotesize
\begin{mathpar}
\infer
		{\proves{M}{\givenA{A}{\intrnl{e}}}}
		{\proves{M}{\givenA{A}{\encaps{e}}}}
		\quad(\textsc{Enc-Exp})
		\and
\infer
		{\proves{M}{\givenA{A}{\intrnl{e}}}}
		{\proves{M}{\givenA{A}{\encaps{e.f}}}}
		\quad(\textsc{Enc-Field})
		\and
\infer
		{
		\proves{M}{\givenA{A}{\encaps{e_1}}} \\
		\proves{M}{\givenA{A}{\encaps{e_2}}}
		}
		{
		\proves{M}{\givenA{A}{\encaps{e_1 = e_2}}}
		}
		\quad(\textsc{Enc-=})
		\and
\infer
		{
		\proves{M}{\givenA{A}{\encaps{e_1}}} \\
		\proves{M}{\givenA{A}{\encaps{e_2}}}
		}
		{
		\proves{M}{\givenA{A}{\encaps{e_1 + e_2}}}
		}
		\quad(\textsc{Enc-+})
		\and
\infer
		{
		\proves{M}{\givenA{A}{\encaps{e_1}}} \\
		\proves{M}{\givenA{A}{\encaps{e_2}}}
		}
		{
		\proves{M}{\givenA{A}{\encaps{e_1 < e_2}}}
		}
		\quad(\textsc{Enc-<})
		\and
\infer
		{
		\proves{M}{\givenA{A}{\encaps{e}}} \\
		\proves{M}{\givenA{A}{\encaps{e_1}}} \\
		\proves{M}{\givenA{A}{\encaps{e_2}}}
		}
		{
		\proves{M}{\givenA{A}{\encaps{\prg{if}\ e\ \prg{then}\ e_1\ \prg{else}\ e_2}}}
		}
		\quad(\textsc{Enc-If})
		\and
\infer
		{\proves{M}{A\ \longrightarrow\ \internal{x}}}
		{\proves{M}{\givenA{A}{\encaps{\access{x}{y}}}}}
		\quad(\textsc{Enc-IntAccess})
		\and
\infer
		{}
		{\proves{M}{\givenA{A}{\encaps{\wrapped{x}}}}}
		\quad(\textsc{Enc-Inside}_1)
		\and
\infer
		{\proves{M}{A\ \longrightarrow\ \wrapped{x}}}
		{\proves{M}{\givenA{A}{\encaps{\neg \access{x}{y}}}}}
		\quad(\textsc{Enc-Inside}_2)
		\and
\infer
		{
		\proves{M}{A_1\ \longrightarrow\ A_2} \\
		\proves{M}{A\ \longrightarrow\ A'} \\
		\proves{M}{\givenA{A_2}{\encaps{A}}}
		}
		{\proves{M}{\givenA{A_1}{\encaps{A'}}}}
		\quad(\textsc{Enc-Conseq})
\end{mathpar}
\caption{Assertion Encapsulation Proof Rules}
\label{f:asrt-encap}
\end{figure}

%%%\clearpage
%%\section{More about the Expressiveness of \Nec Specifications}
\label{s:expressiveness:appendix}

 We continue the comparison of expresiveness between \emph{Chainmail} and \Nec, by 
 considering the examples studied in \cite{FASE}.
 
\subsubsection{ERC20}
The ERC20 \cite{ERC20} is a widely used token standard describing the basic functionality of any Ethereum-based token 
contract. This functionality includes issuing tokens, keeping track of tokens belonging to participants, and the 
transfer of tokens between participants. Tokens may only be transferred if there are sufficient tokens in the 
participant's account, and if either they (using the \prg{transfer} method) or someone authorized by the participant (using the \prg{transferFrom} method) initiated the transfer. 

We specify these necessary conditions here using \Nec. Firstly, \prg{ERC20Spec1} 
says that if the balance of a participant's account is ever reduced by some amount $m$, then
that must have occurred as a result of a call to the \prg{transfer} method with amount $m$ by the participant,
or the \prg{transferFrom} method with the amount $m$ by some other participant.
\begin{lstlisting}[language = Chainmail, mathescape=true, frame=lines]
ERC20Spec1 $\triangleq$ from e : ERC20 $\wedge$ e.balance(p) = m + m' $\wedge$ m > 0
              next e.balance(p) = m'
              onlyIf $\exists$ p' p''.[$\calls{\prg{p'}}{\prg{e}}{\prg{transfer}}{\prg{p, m}}$ $\vee$ 
                     e.allowed(p, p'') $\geq$ m $\wedge$ $\calls{\prg{p''}}{\prg{e}}{\prg{transferFrom}}{\prg{p', m}}$]
\end{lstlisting}
Secondly, \prg{ERC20Spec2} specifies under what circumstances some participant \prg{p'} is authorized to 
spend \prg{m} tokens on behalf of \prg{p}: either \prg{p} approved \prg{p'}, \prg{p'} was previously authorized,
or \prg{p'} was authorized for some amount \prg{m + m'}, and spent \prg{m'}.
\begin{lstlisting}[language = Chainmail, mathescape=true, frame=lines]
ERC20Spec2 $\triangleq$ from e : ERC20 $\wedge$ p : Object $\wedge$ p' : Object $\wedge$ m : Nat
              next e.allowed(p, p') = m
              onlyIf $\calls{\prg{p}}{\prg{e}}{\prg{approve}}{\prg{p', m}}$ $\vee$ 
                     (e.allowed(p, p') = m $\wedge$ 
                      $\neg$ ($\calls{\prg{p'}}{\prg{e}}{\prg{transferFrom}}{\prg{p, \_}}$ $\vee$ 
                              $\calls{\prg{p}}{\prg{e}}{\prg{allowed}}{\prg{p, \_}}$)) $\vee$
                     $\exists$ p''. [e.allowed(p, p') = m + m' $\wedge$ $\calls{\prg{p'}}{\prg{e}}{\prg{transferFrom}}{\prg{p'', m'}}$]
\end{lstlisting}

\subsubsection{DAO}
The Decentralized Autonomous Organization (DAO)~\cite{Dao}  is a well-known Ethereum contract allowing 
participants to invest funds. The DAO famously was exploited with a re-entrancy bug in 2016, 
and lost \$50M. Here we provide specifications that would have secured the DAO against such a 
bug. \prg{DAOSpec1} says that no participant's balance may ever exceed the ether remaining 
in DAO.
\begin{lstlisting}[language = Chainmail, mathescape=true, frame=lines]
DAOSpec1 $\triangleq$ from d : DAO $\wedge$ p : Object
            to d.balance(p) > d.ether
            onlyIf false
\end{lstlisting}
Note that \prg{DAOSpec1} enforces a class invariant of \prg{DAO}, something that could be enforced
by traditional specifications using class invariants.
The second specification \prg{DAOSpec2} states that if after some single step of execution, a participant's balance is \prg{m}, then 
either 
\begin{description}
\item[(a)] this occurred as a result of joining the DAO with an initial investment of \prg{m}, 
\item[(b)] the balance is \prg{0} and they've just withdrawn their funds, or 
\item[(c) ]the balance was \prg{m} to begin with
\end{description}
\begin{lstlisting}[language = Chainmail, mathescape=true, frame=lines]
DAOSpec2 $\triangleq$ from d : DAO $\wedge$ p : Object
            next d.balance(p) = m
            onlyIf $\calls{\prg{p}}{\prg{d}}{\prg{repay}}{\prg{\_}}$ $\wedge$ m = 0 $\vee$ $\calls{\prg{p}}{\prg{d}}{\prg{join}}{\prg{m}}$ $\vee$ d.balance(p) = m
\end{lstlisting}

\sophiaPonder[small changes over Julian's]{\subsubsection{Safe}
\cite{FASE} used as a running example   a Safe, where a treasure 
was secured within a \texttt{Safe} object, and access to the treasure was only granted by 
providing the correct password. }
\ Using \Nec, we express \texttt{SafeSpec}, that requires that the treasure cannot be 
removed from the safe without knowledge of the secret.
\begin{lstlisting}[language = Chainmail, mathescape=true, frame=lines]
SafeSpec $\triangleq$ from s : Safe $\wedge$ s.treasure != null
            to s.treasure == null
            onlyIf $\neg$ inside(s.secret)
\end{lstlisting}

The module  \prg{SafeModule} described  below satisfies  \prg{SafeSpec}.

\begin{lstlisting}[frame=lines]
module SafeModule
     class Secret{}
     class Treasure{}
     class Safe{
         field treasure : Treasure
         field secret : Secret
         method take(scr : Secret){
              if (this.secret==scr) then {
                   t=treasure
                   this.treasure = null
                   return t } 
          }
 }
\end{lstlisting}

 

%%%\clearpage
%%\section{More \Nec Logic rules}
%%\label{a:necSpec}
%%Here we give the complete version of the rules in Fig. \ref{f:only-through} and 
Fig. . \ref{f:only-through} which appeared in the main paper.

\begin{figure}[t]
\footnotesize
\begin{mathpar}
\infer
	{\proves{M}{\onlyIfSingle{A}{\neg A}{A'}}}
	{
	\proves{M}{\onlyThrough{A}{\neg A}{A'}}
	}
	\quad(\textsc{Changes})
	\and
\infer
	{
	\proves{M}{A_1\ \longrightarrow\ A_1'}\\
	\proves{M}{A_2\ \longrightarrow\ A_2'}\\
	\proves{M}{A_3'\ \longrightarrow\ A_3}\\
	\proves{M}{\onlyThrough{A_1'}{A_2'}{A_3'}}
	}
	{\proves{M}{\onlyThrough{A_1}{A_2}{A_3}}}
	\quad(\textsc{$\longrightarrow$})
	\and
\infer
	{
	\proves{M}{\onlyThrough{A_1}{A_2}{A}} \\\\
	\proves{M}{\onlyThrough{A_1'}{A_2}{A'}}
	}
	{\proves{M}{\onlyThrough{A_1\ \vee\ A_1'}{A_2}{A\ \vee\ A'}}}
	\quad(\textsc{$\vee$I$_1$})
	\and
\infer
	{
	\proves{M}{\onlyThrough{A_1}{A_2}{A}} \\\\
	\proves{M}{\onlyThrough{A_1}{A_2'}{A'}}
	}
	{\proves{M}{\onlyThrough{A_1}{A_2\ \vee\ A_2'}{A\ \vee\ A'}}}
	\quad(\textsc{$\vee$I$_2$})
	\and
\infer
	{
	\proves{M}{\onlyThrough{A_1}{A'}{\prg{false}}} \\\\
	\proves{M}{\onlyThrough{A_1}{A_2}{A\ \vee\ A'}}
	}
	{\proves{M}{\onlyThrough{A_1}{A_2}{A}}}
	\quad(\textsc{$\vee$E$_1$})
	\and
\infer
	{
	\proves{M}{\onlyThrough{A'}{A_2}{\prg{false}}} \\\\
	\proves{M}{\onlyThrough{A_1}{A_2}{A\ \vee\ A'}}
	}
	{\proves{M}{\onlyThrough{A_1}{A_2}{A}}}
	\quad(\textsc{$\vee$E$_2$})
	\and
\infer
	{
	\proves{M}{\onlyThrough{A_1}{A_2}{A_3}} \\\\
	\proves{M}{\onlyThrough{A_1}{A_3}{A}}
	}
	{\proves{M}{\onlyThrough{A_1}{A_2}{A}}}
	\quad(\textsc{Trans$_1$})
	\and
\infer
	{
	\proves{M}{\onlyThrough{A_1}{A_2}{A_3}} \\\\
	\proves{M}{\onlyThrough{A_3}{A_2}{A}}
	}
	{\proves{M}{\onlyThrough{A_1}{A_2}{A}}}
	\quad(\textsc{Trans$_2$})
	\and
\infer
	{
	\proves{M}{\onlyIf{A_1}{A_2}{A}}
	}
	{\proves{M}{\onlyThrough{A_1}{A_2}{A}}}
	\quad(\textsc{If})
	\and
\infer
	{}
	{\proves{M}{\onlyThrough{A_1}{A_2}{A_2}}}
	\quad(\textsc{End})
	\and
\infer
	{
	\forall y,\; \proves{M}{\onlyThrough{([y / x]A_1)}{A_2}{A}}
	}
	{\proves{M}{\onlyThrough{\exists x. [A_1]}{A_2}{A}}}
	\quad(\textsc{$\exists_1$})
	\and
\infer
	{
	\forall y,\; \proves{M}{\onlyThrough{A_1}{([y / x]A_2)}{A}}
	}
	{\proves{M}{\onlyThrough{A_1}{A_2}{A}}}
	\quad(\textsc{$\exists_2$})
\end{mathpar}
\caption{\emph{Only Through}}
\label{app:f:only-through}
\end{figure}
\begin{figure}[t]
\footnotesize
\begin{mathpar}
\infer
	{
	\proves{M}{A_1\ \longrightarrow\ A_1'}\\
	\proves{M}{A_2\ \longrightarrow\ A_2'}\\
	\proves{M}{A_3'\ \longrightarrow\ A_3}\\
	\proves{M}{\onlyIf{A_1'}{A_2'}{A_3'}}
	}
	{\proves{M}{\onlyIf{A_1}{A_2}{A_3}}}
	\quad(\textsc{If-$\longrightarrow$})
	\and
\infer
	{
	\proves{M}{\onlyIf{A_1}{A_2}{A}} \\\\
	\proves{M}{\onlyIf{A_1'}{A_2}{A'}}
	}
	{\proves{M}{\onlyIf{A_1\ \vee\ A_1'}{A_2}{A\ \vee\ A'}}}
	\quad(\textsc{If-$\vee$I$_1$})
	\and
\infer
	{
	\proves{M}{\onlyIf{A_1}{A_2}{A}} \\\\
	\proves{M}{\onlyIf{A_1}{A_2'}{A'}}
	}
	{\proves{M}{\onlyIf{A_1}{A_2\ \vee\ A_2'}{A\ \vee\ A'}}}
	\quad(\textsc{If-$\vee$I$_2$})
	\and
\infer
	{
	\proves{M}{\onlyIf{A_1}{A_2}{A\ \vee\ A'}} \\\\
	\proves{M}{\onlyThrough{A'}{A_2}{\prg{false}}}
	}
	{\proves{M}{\onlyIf{A_1}{A_2}{A}}}
	\quad(\textsc{If-$\vee$E})
	\and
\infer
	{
	\proves{M}{\onlyIf{A_1}{A_2}{A}} \\\\
	\proves{M}{\onlyThrough{A_1}{A_2}{A'}}
	}
	{\proves{M}{\onlyIf{A_1}{A_2}{A\ \wedge\ A'}}}
	\quad(\textsc{If-$\wedge$I})
	\and
\infer
	{
	\proves{M}{\onlyThrough{A_1}{A_2}{A_3}} \\\\
	\proves{M}{\onlyIf{A_1}{A_3}{A}}
	}
	{\proves{M}{\onlyIf{A_1}{A_2}{A}}}
	\quad(\textsc{If-Trans)}
	\and
\infer
	{}
	{\proves{M}{\onlyIf{A_1}{A_2}{A_1}}}
	\quad(\textsc{If-Start})
	\and
\infer
	{
	\forall y,\; \proves{M}{\onlyIf{([y / x]A_1)}{A_2}{A}}
	}
	{\proves{M}{\onlyIf{\exists x. [A_1]}{A_2}{A}}}
	\quad(\textsc{If-$\exists_1$})
	\and
\infer
	{
	\forall y,\; \proves{M}{\onlyIf{A_1}{([y / x]A_2)}{A}}
	}
	{\proves{M}{\onlyIf{A_1}{A_2}{A}}}
	\quad(\textsc{If-$\exists_2$})
\end{mathpar}
\caption{\emph{Only If}}
\label{app:f:only-if}
\end{figure}%%\clearpage
%%\newpage
\section{\SpecO Logic}
\label{app:assert_logic}


{
\begin{figure}[ht]
\footnotesize
\begin{mathpar}
\infer
		{}
		{\proves{M}{\calls{x}{y}{m}{\overline{z}}\ \longrightarrow\ \external{x}}}
		\quad(\textsc{Caller-Ext})
		\and
\infer
		{}
		{\proves{M}{\calls{x}{y}{m}{\overline{z}}\ \longrightarrow\ \access{x}{y}}}
		\quad(\textsc{Caller-Recv})
		\and
\infer
		{}
		{\proves{M}{\calls{x}{y}{m}{\ldots, z_i, \ldots}\ \longrightarrow\ \access{x}{z_i}}}
		\quad(\textsc{Caller-Args})
		\and
\infer
		{C \in M}
		{\proves{M}{x\ :\ C\ \longrightarrow\ \internal{x}}}
		\quad(\textsc{Class-Int})
		\and
\infer
		{(\prg{field}\ \_\ f\ :\ D)\ \in\ M(C).(\prg{flds})}
		{\proves{M}{e : C\ \longrightarrow\ e.f : D}}
		\quad(\textsc{Fld-Class})
		\and
\infer
		{(\prg{class}\ \enclosed\ C \{\_; \_\})\ \in\ M}
		{\proves{M}{x : C\ \longrightarrow\ \wrapped{x}}}
		\quad(\textsc{Inside-Int})
		\and
\infer
		{}
		{\proves{M}{\false\ \longrightarrow\ A}}
		\quad(\textsc{Absurd})
		\and
\infer
		{}
		{\proves{M}{A\ \vee\ \neg A}}
		\quad(\textsc{Excluded Middle})
\end{mathpar}
\normalsize
\caption{Assumed Properties of the \SpecO proof system.}
\label{f:assertProperties}
\end{figure}}

In Fig. \ref{f:assertProperties} we present some assumed rules of the 
\SpecO proof system, of the form $\proves{M}{A}$. These rules
are relatively simple, with none assuming any surprising results.
They are straightforward, and would be easy to prove sound. 
\textsc{Caller-Ext}, \textsc{Caller-Recv}, \textsc{Caller-Args},
and \textsc{Class-Int} are simple properties that arise from 
the semantics of \SpecO.
\textsc{Fld-Class} and \textsc{Inside-Int} are directly drawn from 
the simple type system of \Loo.
\textsc{Absurd} and \textsc{Excluded Middle} are common logical properties.
%%%\clearpage
%%\newpage
\section{Bank Account Example}
\label{app:BankAccount}


\begin{figure}[ht]
\begin{lstlisting}[mathescape=true, frame=lines]
module Mod4
  class Account
    field password : Object
    method authenticate(pwd : Object) : bool
      (PRE:  a : Account $\wedge$ b : Bank
       POST: b.balance(a)$_\prg{old}$ == b.balance(a)$_\prg{new}$)
      (PRE:  a : Account
       POST: res != a.password)
      (PRE:  a : Account
       POST: a.password$_\prg{old}$ == a.password$_\prg{new}$)
      {return pwd == this.password}
    method changePassword(pwd : Object, newPwd : Object) : void
      (PRE:  a : Account
       POST: res != a.password)
      (PRE:  a : Account $\wedge$ b : Bank
       POST: b.balance(a)$_\prg{old}$ == b.balance(a)$_\prg{new}$)
      (PRE:  a : Account $\wedge$ pwd != this.password
       POST: a.password$_\prg{old}$ = a.password$_\prg{new}$)
      {if pwd == this.password
        this.password := newPwd}

  class confined Ledger
    field acc1 : Account
    field bal1 : int
    field acc2 : Account
    field bal2 : int
    ghost intrnl balance(acc) : int = 
      if acc == acc1
        bal1
      else if acc == acc2
        bal2
      else -1
    method transfer(amt : int, from : Account, to : Account) : void
      (PRE:  a : Account $\wedge$ b : Bank $\wedge$ (a != acc1 $\wedge$ a != acc2)
       POST: b.balance(a)$_\prg{old}$ == b.balance(a)$_\prg{new}$)
      (PRE:  a : Account
       POST: res != a.password)
      (PRE:  a : Account
       POST: a.password$_\prg{old}$ == a.password$_\prg{new}$)
      {if from == acc1 && to == acc2
         bal1 := bal1 - amt
         bal2 := bal2 + amt
       else if from == acc2 && to == acc1
         bal1 := bal1 + amt
         bal2 := bal2 - amt}
      

  class Bank
    field book : Ledger
    ghost intrnrl balance(acc) : int = book.balance(acc)
    method transfer(pwd : Object, amt : int, from : Account, to : Account) : void
      (PRE:  a : Account $\wedge$ b : Bank $\wedge$ $\neg$ (a == acc1 $\wedge$ a == acc2)
       POST: b.balance(a)$_\prg{old}$ a= b.balance(a)$_\prg{new}$)
      (PRE:  a : Account
       POST: res != a.password)
      (PRE:  a : Account
       POST: a.password$_\prg{old}$ == a.password$_\prg{new}$)
      {if (from.authenticate(pwd))
         book.transfer(amt, from, to)}
\end{lstlisting}
\caption{Bank Account Module}
\label{f:ex-bank}
\end{figure}

%%%\clearpage
%%\section{Proof of Guarantee of Safety in \S\ref{sec:how}}
\label{app:safety}
\jm[]{
In this section we provide a proof sketch that \SrobustB ensures our balance
does not decrease in contexts with no access to our password. This 
property is expressed in \S\ref{sec:how}, and the example is repeated below.
}
\begin{lstlisting}[mathescape=true, language=chainmail, frame=lines]
module $\ModParam{1}$
     ...
    method cautious(untrusted:Object)
        a = new Account
        p = new Password
        a.set(null,p)
        ...
        untrusted.make_payment(a)
        ...
\end{lstlisting}
\jm[]{
Informally, the property we wish to express of the above code snippet is 
that as long as there is no \prg{untrusted} does not have external access (whether transitive or direct)
to \prg{a.password} before the call on line 7, then \prg{a.balance} will not decrease during the 
execution of line 8. This property is expressed and proven in Theorem \ref{thm:safety}.
}
%\begin{itemize}
%\item
%suppose we rewrite \prg{Mod$\_1$} such that we duplicate all methods used in line 7. In the duplicated version of the methods
%we add a tuple as an argument that contains the account, and the current password. In every duplicate method, whenever the
%method \prg{setPassword} is called, we check if the receiver is \prg{a}, and if so we update the password in the tuple to the 
%new password. In this manner by the end of line 7, we still know what the password of the account is.
%\item
%We then insert the following code at the end of line 7:\\
%\prg{a.setPassword(t.snd(), new Password())}
%\item
%We are now in an arising program state where there is no external knowledge of the password, and we are able to apply \SrobustB
%\end{itemize}

\begin{lemma}[]

\end{lemma}
\begin{proof}
\end{proof}

\begin{theorem}[\SrobustB Guarantees Account Safety]
\label{thm:safety}
Let \prg{BankMdl} be some module that satisfies \SrobustB, 
$M$ be some external module, and $\sigma_1 = (\chi_1, \phi_1 : \psi_1)$ be some \textit{Arising} program state. If the continuation of $\phi_1$, $c_1$, is
\begin{verbatim}
a = new Account; 
p = new Password; 
a.set(null,p); 
s; 
untrusted.make_payment(a, z1, ..., zn); 
s'
\end{verbatim}
let 
\begin{itemize}
\item
$\sigma_2 = (\chi_2, \phi_2 : \psi_2)$ be the program state immediately preceding the execution of \prg{s}
\item
$\sigma_3 = (\chi_3, \phi_3 : \psi_3)$ be the program state immediately following the execution of \prg{s} where $\satisfies{\prg{BankMdl};\sigma_3}{\prg{a.password} \neq \prg{zk}}$ for all $i \leq \prg{k} \leq n$
\item
$\sigma_4 = (\chi_4, \phi_4 : \psi_4)$ be the program state immediately following the execution of\\ \prg{untrusted}\prg{.make\_payment}\prg{(a, z1, ..., zn)} 
\end{itemize}
if 
\begin{itemize}
\item
for all objects $o \in \chi_3$ such that $\satisfies{\prg{BankMdl}; \sigma_3}{\access{o}{\prg{a.password}}}$ and $o$ is transitively accessible (i.e. the transitive closure of $\access{\_}{\_}$) from \prg{untrusted}
then $\satisfies{\prg{BankMdl}; \sigma_3}{\internal{o}}$, and
\item
$\satisfies{\prg{BankMdl}; \sigma_3}{\prg{a.balance}\geq b}$
\end{itemize}
then 
\begin{itemize}
\item
$\satisfies{\prg{BankMdl};\sigma_4}{\prg{a.balance} \geq b}$.
\end{itemize}
\end{theorem}
\begin{proof}
The challenge in constructing a proof is that \SrobustB is not directly applicable to $\sigma$ since there may be an external 
object that has access to \prg{a.password}, depending on what code exists in \prg{s}. For example, if \prg{s} 
is blank then $\sigma_1.(\prg{this})$ has access to \prg{a}. The second difficulty is that any applicable program state must be 
\textit{Arising}, and thus we are not able to artificially modify the heap to attain such a program state. The solution is 
to create a \prg{M'} that results in an effectively equivalent heap (from the perspective of \prg{a.balance}) after execution of \prg{untrusted.make\_payment(a, z1, ..., zn)}. To attain such a module we
\begin{enumerate}
\item
introduce a class \prg{Tuple} to $M'$ that has two fields, \prg{fst : Account} and \prg{snd : Password}, and three methods \prg{fst()}, \prg{snd()}, and \prg{setSnd(p)} that 
return the first, and second elements of the \prg{Tuple}, and set the second element of the \prg{Tuple}.
\item
duplicate every method called in \prg{s}, and rewrite the code to call the duplicate methods
\item
the duplicate methods are identical to the original methods, except they take an extra argument that is a tuple \prg{t}, and
\item
whenever a method called in \prg{s} calls \prg{a'.setPassword(p1, p2)} for some 
\prg{a'}, \prg{p1}, and \prg{p2}, the following code is subsequently inserted:
\begin{lstlisting}[mathescape=true, language=chainmail, frame=lines]
if (a' == t.fst() && p1 == t.snd())
  t.setSnd(p2)
\end{lstlisting}
\item
the following statement is inserted at the beginning of \prg{s} (in $\sigma_2$)
\begin{lstlisting}[mathescape=true, language=chainmail, frame=lines]
t = new Tuple(a, p)
\end{lstlisting}
\item
the following statement is inserted at the end of \prg{s} (in $\sigma_3$)
\begin{lstlisting}[mathescape=true, language=chainmail, frame=lines]
a.setPassword(t.snd(), new Password())
\end{lstlisting}
\end{enumerate}
Clearly the resulting heap at the end of \prg{s}, when executed in the context of \prg{M'} ($\chi_3'$)
will be equivalent to $\chi_3$, except in the case of \prg{a.password}, which is now a new object.
This will not alter the execution of \prg{untrusted.make\_payment(a, z1, ..., zn)}, as we know that all other objects with knowledge of the old value of \prg{a.password} are not reachable from \prg{untrusted},
and thus from the perspective of \prg{untrusted} the heaps are equivalent.
Thus, it follows that \prg{a.balance} in $\chi_3'$ is equal to \prg{a.balance} in $\chi_3$.
Since \SrobustB is applicable to $\sigma_3'$, we can conclude that \prg{a.balance} in $\chi_4'$ (and thus $\chi_4$)
remains unchanged.
\end{proof}





%%%\clearpage
%%\section{Crowdsale}
\jm[]{It is notable that \Nec is able to encode the motivating example of \citeauthor{VerX}: 
an escrow smart contract that ensures that the contract may not be maliciously coerced.
The motivating \prg{Crowdsale} example consists of a \prg{Crowdsale} contract 
for crowd sourcing funding. A \prg{Crowdsale} object consists of an \prg{Escrow} object,
an amount raised, a funding goal, and a closing time in which the goal must be met for 
the fund to be successful. An \prg{Escrow} consists of a ledger of investors and how much
they have invested. There are several properties that \citeauthor{VerX} sought to encode,
and we have provided the encoding of those specifications in Fig. \ref{f:verx:encoding}.
\prg{R0} states that if an investor claims a refund from an escrow, then the balance of 
the escrow decreases by the amount the investor had deposited in the escrow. 
\prg{R1} states that if at anytime the escrow has not yet succeeded, then the deposits must
be less than the balance of the escrow. 
\prg{R2\_1} and \prg{R2\_2} combine to express a single property: no one may ever withdraw and 
then subsequently claim a refund or visa versa.
\prg{R3} states that if the funding goal is ever met, then no one may subsequently claim a refund.}

\begin{figure}[htb]
\begin{lstlisting}[language=chainmail]
contract Crowdsale {
Escrow escrow;
  uint256 closeTime;
  uint256 raised = 0;
  uint256 goal = 10000 * 10**18;
  function constructor() {
    escrow = new Escrow(0x1234);
    closeTime = now + 30 days;
  }
  function invest() payable {
    require(raised < goal);
    // fix: uncomment pre-condition below:
    // require(now<=closeTime);
    escrow.deposit.value(msg.value)(msg.sender);
    raised += msg.value;
  }
  function close() {
    require(now > closeTime || raised >= goal);
    if (raised >= goal) {
      escrow.close();
    } else {
      escrow.refund();
    }
  }
}
\end{lstlisting}
\caption{Crowdsale Contract}
\label{f:verx:crowdsale}
\end{figure}

\begin{figure}[htb]
\begin{lstlisting}[language=chainmail]
contract Escrow {
  address owner, beneficiary;
  mapping(address => uint256) deposits;
  enum State {OPEN, SUCCESS, REFUND}
  State state = OPEN;
  constructor(address b) {
    owner = msg.sender;
    beneficiary = b;
  }
  modifier onlyOwner {
    require(msg.sender == owner);
  }
  function close() onlyOwner {state = SUCCESS;}
  function refund() onlyOwner {state = REFUND;}
  function deposit(address p) onlyOwner payable {
    deposits[p] = deposits[p] + msg.value;
  }
  function withdraw() {
    require(state == SUCCESS);
    beneficiary.transfer(this.balance);
  }
  function claimRefund(address p) {
    require(state == REFUND);
    uint256 amount = deposits[p];
    deposits[p] = 0;
    p.call.value(amount)();
  }
}
\end{lstlisting}
\caption{Escrow Contract}
\label{f:verx:escrow}
\end{figure}

\begin{figure}[htb]
\begin{lstlisting}[mathescape=true, language=chainmail]
(R0) $\triangleq$ e : Escrow $\wedge$ $\calls{\_}{\prg{e}}{\prg{claimRefund}}{\prg{p}}$
          next e.balance = nextBal onlyIf nextBal = e.balance - e.deposits(p)
(R1) $\triangleq$ e : Escrow $\wedge$ e.state $\neq$ SUCCESS $\longrightarrow$ sum(deposits) $\leq$ e.balance
(R2_1) $\triangleq$ e : Escrow $\wedge$ $\calls{\_}{\prg{e}}{\prg{withdraw}}{\prg{\_}}$
           to $\calls{\_}{\prg{e}}{\prg{claimRefund}}{\prg{\_}}$ onlyIf false
(R2_2) $\triangleq$ e : Escrow $\wedge$ $\calls{\_}{\prg{e}}{\prg{claimRefund}}{\prg{\_}}$
           to $\calls{\_}{\prg{e}}{\prg{withdraw}}{\prg{\_}}$ onlyIf false
(R3) $\triangleq$ c : Crowdsale $\wedge$ sum(deposits) $\geq$ c.escrow.goal
         to $\calls{\_}{\prg{c.escrow}}{\prg{claimRefund}}{\prg{\_}}$ onlyIf false
\end{lstlisting}
\caption{Encoding VerX Crowdsale Example in Necessity}
\label{f:verx:encoding}
\end{figure}
%
%\clearpage
%\section*{Push Symbols} 
%
%pushSymbol             ---                      $\pushSymbol$                \\    
%Pushsymboll            ---                      $\pushSymbolL$               \\ 
%pushSymbolA            ---                      $\pushSymbolA$              \\ 
%pushSymbolAA           ---                      $\pushSymbolAA$              \\
%Pushes{a}{b}{c}        ---                      \Pushes{a}{b}{c}           \\
%PushA{a}{b}{c}{d}      ---                      \PushA{a}{b}{c}{d}         \\
%PushALong{a}{b}{c}{d}  ---                      \PushALong{a}{b}{c}{d}     \\
%PushAS{a}{b}           ---                      \PushAS{a}{b}              \\ 
%PushASLong{a}{b}       ---                      \PushASLong{a}{b}          \\
%PushASLongG{a}{b}      ---                      \PushASLongG{a}{b}         \\
%PushS{a}{b}            ---                      \PushS{a}{b}               \\ 
%PushSF{a}{b}           ---                      \PushSF{a}{b}              \\ 
%PushSLong{a}{b}        ---                      \PushSLong{a}{b}           \\
%FIXSymbolA             ---                      $\FIXSymbolA$ \\
%FIXSymbolA nomaths     ---                      \FIXSymbolA  \\

\end{document}
