%% For double-blind review submission, w/o CCS and ACM Reference (max submission space)
%\documentclass[acmsmall,review]{acmart}\settopmatter{printfolios=true,printccs=false,printacmref=false}
%% For double-blind review submission, w/ CCS and ACM Reference
\documentclass[acmsmall,review,anonymous,screen]{acmart}\settopmatter{printfolios=true,printacmref=false}
%% For single-blind review submission, w/o CCS and ACM Reference (max submission space)
%\documentclass[acmsmall,review]{acmart}\settopmatter{printfolios=true,printccs=false,printacmref=false}
%% For single-blind review submission, w/ CCS and ACM Reference
%\documentclass[acmsmall,review]{acmart}\settopmatter{printfolios=true}
%% For final camera-ready submission, w/ required CCS and ACM Reference
%\documentclass[acmsmall]{acmart}\settopmatter{}
\usepackage[shortlabels]{enumitem}
\usepackage{mathtools}
\usepackage{wrapfig}
\usepackage{stackrel}

\DeclareSymbolFont{arrows}{U}{FdSymbolC}{m}{n}

% \usepackage{ amssymb }
\DeclareTextFontCommand{\texttt}{\ttfamily}

%% Journal information
%% Supplied to authors by publisher for camera-ready submission;
%% use defaults for review submission.
\acmJournal{PACMPL}
\acmVolume{}
\acmNumber{POPL} % CONF = POPL or ICFP or OOPSLA
\acmArticle{}
\acmYear{2025}
\acmMonth{1}
\acmDOI{} % \acmDOI{10.1145/nnnnnnn.nnnnnnn}
\startPage{1}

%% Copyright information
%% Supplied to authors (based on authors' rights management selection;
%% see authors.acm.org) by publisher for camera-ready submission;
%% use 'none' for review submission.
\setcopyright{none}
%\setcopyright{acmcopyright}
%\setcopyright{acmlicensed}
%\setcopyright{rightsretained}
%\copyrightyear{2018}           %% If different from \acmYear

%% Bibliography style
\bibliographystyle{ACM-Reference-Format}
%% Citation style
%% Note: author/year citations are required for papers published as an
%% issue of PACMPL.
%\citestyle{acmauthoryear}   %% For author/year citations
\citestyle{acmnumeric}

%%%%%%%%%%%%%%%%%%%%%%%%%%%%%%%%%%%%%%%%%%%%%%%%%%%%%%%%%%%%%%%%%%%%%%
%% Note: Authors migrating a paper from PACMPL format to traditional
%% SIGPLAN proceedings format must update the '\documentclass' and
%% topmatter commands above; see 'acmart-sigplanproc-template.tex'.
%%%%%%%%%%%%%%%%%%%%%%%%%%%%%%%%%%%%%%%%%%%%%%%%%%%%%%%%%%%%%%%%%%%%%%


%% Some recommended packages.
\usepackage{booktabs}   %% For formal tables:
                        %% http://ctan.org/pkg/booktabs
\usepackage{subcaption} %% For complex figures with subfigures/subcaptions
                        %% http://ctan.org/pkg/subcaption
 \usepackage{ stmaryrd }                       

\usepackage{relsize}
\usepackage{mathpartir}
\usepackage{amsmath}
\usepackage{amsthm}
\usepackage{listings}
\usepackage{xspace}
\usepackage{definitions}
\usepackage{multirow,bigdelim}
\usepackage{pbox}
\usepackage{courier}
\usepackage{soul}
\usepackage{centernot}
 

\newcommand\multibrace[3]{\rdelim\}{#1}{3mm}[\pbox{#2}{#3}]}


%%these COLOUR MACROS ARE ACTIVELY FUCKING EVIL
%%DO NOT DO THIS. EVER
%%AT LEast have some ovious way to TURN THEM OFF

\definecolor{ferngreen}{rgb}{0.31, 0.47, 0.26}

\newcommand{\sue}[1] {{\color{brown}{#1}}}
\definecolor{amber}{rgb}{1.0, 0.75, 0.0}
\definecolor{amethyst}{rgb}{0.6, 0.4, 0.8}

\newcommand{\ponders}[3]{\marginpar{\tiny\itshape\raggedright\textcolor{#2}{\textbf{#1:} #3}}\ignorespaces}
\renewcommand{\ponders}[3]{{#3\ignorespaces}}  %TURNING IT OFF. NOT THAT IT FUCKING HEPLS

\marginparwidth=1.6cm \marginparsep=0cm



\begin{document}

%% Title information
%\title[Specification and Proof of Necessary Conditions]{Specification
%and Proof of Necessary Conditions}         %% [Short Title] is
%optional;
%\title{Necessity Specifications are Necessary
%\title{Reasoning about External Calls with Limited Effects}
\title{What is Protection?}


%% Author with two affiliations and emails.
\author{Julian Mackay}
%\authornote{with author2 note}          %% \authornote is optional;
                                        %% can be repeated if necessary
\orcid{0000-0003-3098-3901}             %% \orcid is optional
\affiliation{
  %\position{Position2a}
  %\department{Engineering and Computer Science}             %% \department is recommended
  \institution{Victoria University of Wellington}           %% \institution is required
  %\streetaddress{Street2a Address2a}
  %\city{City2a}
  %\state{State2a}
  %\postcode{Post-Code2a}
  \country{New Zealand}                   %% \country is recommended
}
\email{julian.mackay@ecs.vuw.ac.nz}         %% \email is recommended

 

%% Abstract
%% Note: \begin{abstract}...\end{abstract} environment must come
%% before \maketitle command


  

%% 2012 ACM Computing Classification System (CSS) concepts
%% Generate at 'http://dl.acm.org/ccs/ccs.cfm'.
\begin{CCSXML}
<ccs2012>
   <concept>
       <concept_id>10011007.10010940.10010992.10010993.10011683</concept_id>
       <concept_desc>Software and its engineering~Access protection</concept_desc>
       <concept_significance>300</concept_significance>
       </concept>
   <concept>
       <concept_id>10011007.10011074.10011099.10011692</concept_id>
       <concept_desc>Software and its engineering~Formal software verification</concept_desc>
       <concept_significance>500</concept_significance>
       </concept>
   <concept>
       <concept_id>10003752.10003790.10011741</concept_id>
       <concept_desc>Theory of computation~Hoare logic</concept_desc>
       <concept_significance>500</concept_significance>
       </concept>
   <concept>
       <concept_id>10011007.10011006.10011008.10011009.10011011</concept_id>
       <concept_desc>Software and its engineering~Object oriented languages</concept_desc>
       <concept_significance>300</concept_significance>
       </concept>
 </ccs2012>
\end{CCSXML}

\ccsdesc[300]{Software and its engineering~Access protection}
\ccsdesc[500]{Software and its engineering~Formal software verification}
\ccsdesc[500]{Theory of computation~Hoare logic}
\ccsdesc[300]{Object oriented programming~Object capabilities}

%\newcommand{\callsOn}[2]{{#1}{\stackrel{calls}{\longrightarrow}}{#2}}

\newcommand{\ExtCallsOn}[1]{ {\mathcal E}\!\mathit{xt}{\mathcal C}\!\mathit{ll}{\mathcal O\!}\mathit{n}(#1)}
\newcommand{\callsOn}[1] {{\ensuremath{\mathcal C}\!{\mathit{lld}}(#1)}}
\newcommand{\enrich}[1] {{\ensuremath{#1}^{+\!\!+}}} %{\prg{+}\!\prg{+}}}}
\newcommand{\as}[2]{#1^{#2}}
\newcommand{\preserve}[1]{\oopenBr\, #1\ccloseBr}
% {{\ensuremath{[#1[#2]}}} % _{#2}}}}

\newcommand{\Guards}[2]{{\mathcal G}\!\mathit{uards}( {#1}, {#2})}


\begin{abstract}

the aim is to systematically derive the concept of protection
\end{abstract}

\maketitle 

\section{Capabilities guarding effects}

\subsection{Preliminaries}

We define the operations $\as {\_} {\sigma}$ which turns expressions and assertions to variable-free expressions and assertions by 
replacing  the free variables in the first argument by their values in $\sigma$.
Thus,
% $\as{\sigma'}{\sigma}$ which expands the variable map from $\sigma'$ to be that of $\sigma$,
 $\as {\re} {\sigma}$  replaces all the free variables in $\re$ by addresses (their values in $\sigma$; similarly,
  $\as {A} {\sigma}$  replaces all the free variables in $A$ by addresses (their values in $\sigma$).
 \footnote{Should we call this an adaptation of the states?}\footnote{Perhaps we should adopt this notation in the paper too?}

\begin{definition} for a state  $\sigma$, an assertion $A$, and a term $\re$, we define:
$~ \ $

\begin{itemize}
%\item
% I THINK THAT ONE IS WRONG!
%$\as {\sigma'} {\sigma}\ \ \triangleq \ \ {\sigma'}[\overline {x\! \mapsto\! \interpret {\sigma} {x}}] $ \ \ \  where $\overline x = dom (\sigma)$
\item
$\as {\re} {\sigma}\ \  \triangleq \ \ {\re}[\overline {x\! \mapsto\! \interpret {\sigma} {x}}] $ \ \ \  where $\overline x = \fv(\re)$
\item
$\as {A} {\sigma}\ \  \triangleq \ \ {A}[\overline {x\! \mapsto\! \interpret {\sigma} {x}}] $ \ \ \  where $\overline x = \fv(A)$
\end{itemize}
\end{definition}

Note that  $\as {\re} {\sigma}$  is a variable-free expression, while $\interpret \sigma \re$ is a value. 
Moreover,  t $\interpret {\sigma} {\as {\re} {\sigma}}$= $\interpret \sigma \re$ always holds, but when $\sigma \neq \sigma'$ then
$\interpret {\sigma} {\as {\re} {\sigma'}}$ needs not be equal to  $\interpret \sigma \re$, nor need it be equal to $\interpret {\sigma'} {\as {\re} {\sigma}}$. Similar comments apply to $\as {A} {\sigma}$ and $\interpret {\sigma} {\as {A} {\sigma}}$, and $\interpret {\sigma} {\as {A} {\sigma'}}$.
 
\subsection{Capability Guarding an Effect}{
We say that a capability $\re$ guards an effect %$\Eff$, if $\Eff$ 
if the effect cannot take place -- or be observed--  from an  external state, unless some external objet makes a call on $\re$.  To make this concept precise, we need to make precise 1) the effect   being observed, and 2) making a call on $\re$. Wrt 1): the effect   is observed iff we go from a state satisfying $A$ to a state which does not satisfy $A$. And wrt 2): an external object makes a call on $\re$ iff $\re$ is the receiver or one of the arguments in a call that will be executed next. This leads to the following definitions:



\begin{definition}
\label{def:guards}
$~ \ $

\begin{itemize}

\item
$M, \sigma \models \,     \callsOn {\re}\ \ \ \ \ \triangleq \ \  \exists x,y_0,\overline y,m.[\ \ \  \sigma.\prg{cont}\txteq x:=y_0.m(\overline y); \_ \ \ \wedge\  \ \interpret {\sigma} {\re}\! \in\!  \{\, \interpret {\sigma} {y_0}, \overline{ \interpret {\sigma} {y}} \, \} \  \ ] $
  $~ \\ $

\item

$M  \models \,  \Guards {\re} {A}   \ \triangleq \ \ 
   \forall \Mtwo, \sigma, \sigma'.$\\
   $\strut \hspace{2cm} [\ \ M, \sigma \models \,  (\extThis \wedge A) \ \wedge \ 
   % \ \ \wedge\ \ \sigma.\prg{cont}\neq \_; \_ \ \  \ \wedge $\\ $\strut \hspace{5.05cm}    
   \leadstoBoundedStar   {\Mtwo\cdot M} {\sigma} {\sigma'}\ \ \wedge \ \ M, \sigma' \models   \,  (\, \extThis \wedge \neg( \as {A}{\sigma})\, )    $\\
$\strut \hspace{5cm}  
   \Longrightarrow \exists \sigma''. [ \   \leadstoBoundedStar   {\Mtwo\cdot M} {\sigma} {\sigma'} \ \wedge\ \ M,  {\sigma''} \models \,     \callsOn {\as {\re} {\sigma}} \ ]\  \ ] $
  \end{itemize}


\end{definition}

\noindent
In the definition above, when we check that $A$ no longer holds (i.e. $... \sigma' ... \models ... \neg A$) we turn $A$ to a variable-free assertion $\neg (\as {A}{\sigma})$, so that its validity is not affected by difference in the variable-map of $\sigma$ and $\sigma'$.
Thus, the definition above is \emph{not} equivalent with \\
$\strut \ \ \ \ \ \ \ \ \ M  \models \,  \Guards {\re} {A}   \ \triangleq \ \ 
   \forall \Mtwo, \sigma, \sigma'. [\ \ ... \ \wedge \    ... \ \wedge \ \ M, {\sigma'}   \models   \,  (\extThis \wedge \neg A)    \  \Longrightarrow ... \ ]\  \ ] $.\\
   Nor is it equivalent with\\
$\strut \ \ \  \ \ \ \ \models \,  \Guards {\re} {A}   \ \triangleq \ \ 
   \forall \Mtwo, \sigma, \sigma'.  [\ \ ...   \Longrightarrow \exists \sigma''. [ \   ... \ \wedge\ \ M, {\sigma''}  \models \,     \callsOn {\interpret  {\sigma} \re}\ ]\  \ ] $   
\footnote{The definition of  $\callsOn {\re}$ has changed slightly, It used to be $M, \sigma \models \,     \callsOn {\re}\ \triangleq \  \exists x,y_0,\overline y,m.[\   \sigma.\prg{cont}\txteq x:=y_0.m(\overline y)  \ \ \wedge\  ... \  \  \ ]$ i.e. it did not allow for 
other statements after the method call}
\footnote{On 27th Febr. the definition was as per below, and that was a wrong definition: \\
  $M  \models \,  \Guards {\re} {A}   \ \triangleq \ \ 
   \forall \Mtwo, \sigma, \sigma'.[\ \ M, \sigma \models \,  (\extThis \wedge A) \ \ \wedge\ \ \sigma.\prg{cont}\neq \_; \_ \ \  \ \wedge $\\
$\strut \hspace{5.05cm}   \leadstoBoundedStarFin  {\Mtwo\cdot M} {\sigma} {\sigma'}\ \ \wedge \ \ M, \as {\sigma'} {\sigma} \models   \,  (\extThis \wedge \neg A)    $\\
$\strut \hspace{5cm}  
   \Longrightarrow \  M, \sigma \models \,     \callsOn {\re} \ \ ] $}

\vspace{.3cm}
   
The definition of  $\Guards {\re} {A}$ implies that if we are interested in the preservation of $A$, it is useful to have a guarantee that  $\re$ will never by called on from an external state.  
Such a guarantee is described in Def \ref{def:insidef}.

\begin{definition}
\label{def:insidef} $~ \ $

\begin{itemize}
\item
$M, \sigma \models \,  \insidef{\re}  \   \ \triangleq \ \    \forall \Mtwo, \sigma'.
% $\\ $\strut \hspace{2.4cm}
 [ 
 \ \  \leadstoBoundedStar  {\Mtwo\cdot M} {\sigma} {\sigma'}\ \ \wedge\ \ M, \sigma' \models \,  \extThis
  \ \  \Longrightarrow\ \ M, {\sigma'}  \not \models \,   \callsOn {\as \re \sigma}\ \ ] $
\end{itemize}
\end{definition}

In the above, in  $ \callsOn {\as \re \sigma}$ we replace all free variables in $\re$ by thei values as in $\sigma$.
%update the variable map of $\sigma'$ to be that from $\sigma$, because the free variables from $\re$ might have a different meaning in $\sigma$ and $\sigma'$.
Thus, that the requirement in Def. \ref{def:insidef} is different from
$ \forall \Mtwo, \sigma'. [ \  \leadstoBoundedStar  {...} {...} {...}  \wedge\  ...
\   \Longrightarrow\   M, \sigma' \not \models \,   \callsOn {\interpret {\sigma} {\re}}\ \ ] $

\subsection{Protection}

Lemma \ref{lemma:insidef} says that if a capability $\re$ guards some assertion $A$, and $\re$ is never called in any external  state reachable from $\sigma$, then $A$ will never get invalidated:

\begin{lemma}
$~ $ \\
\label{lemma:insidef}
\begin{itemize}

\item
$ \Guards {\re} {A} \ \ \wedge\ \ M, \sigma \models \  \insidef{\re} \ \wedge \  M, \sigma \models \  A \wedge \extThis \ \ \Longrightarrow $\\
$\strut \hspace{3.6cm}\forall \Mtwo,  \sigma'.[\ \   \leadstoBoundedStar  {\Mtwo\cdot M} {\sigma} {\sigma'}\ \ \wedge\ \ M, \sigma' \models \,  \extThis  \ \  \Longrightarrow\ \ M,  \sigma'  \models\,  \as A \sigma \ \ ]$.
\end{itemize}
\end{lemma}

Proof by unfolding the definitions. 
 
 \vspace{.3cm}
 Lemma \ref{lemma:insidef} motivates that we want to find some criteria that guarantee that $M, \sigma \models \  \insidef{\re}$.
 An exact characterization of $ \insidef{\re}$ would be undecidable; we will, instead, over-approximate it through the concept of protection, defined in Def. \ref{def:protect}.
 
  
 
 
 
 
 
\begin{definition}
$~ \ $
\label{def:protect}

\begin{itemize}

\item
{{$\Relevant {\alpha} {\sigma}  \ \ \ \ \ \triangleq \ \  \  \{ \ \alpha' \, \mid \ \exists n\!\in\!\mathbb{N}.\exists f_1,...f_n.. [\ \interpret {\sigma} {\alpha.f_1...f_n} = \alpha'\ \}$}}.

\item
$ \LRelevantO   \sigma  \ \  \ \triangleq \ \  \  \{ \ \alpha \ \mid \ { \exists x\in dom(\sigma) \wedge \alpha \in \Relevant {\interpret  {\sigma} x}
{\sigma} \ \} } $.

\item
\label{sect:semantics:assert:prt}
$\satisfiesA{M}{\sigma} {\inside {\re}}  \ \ \ \  \ \triangleq \ \ \   $
% \begin{enumerate}
%\item
 $(1)\ \ \satisfiesA{M}{\sigma}{\extThis}\ \ \Longrightarrow\ \ \forall x\!\in\! \sigma.\ \interpret \sigma x  \neq \interpret \sigma \re $,\ \ \ \ and 
% \item
\\
$~\strut  \hspace{2.7cm} (2) \ \ \forall \alpha.\forall f.[\ \alpha\!\in\!{\LRelevantO {\sigma}} \wedge\   \satisfiesA{M}{\sigma}{\external {\alpha}} 
  \ \Longrightarrow \  
  \interpret {\sigma} {\alpha.f}\! \neq \! \interpret {\sigma} {\re}     \ ] $.
% \end{enumerate}

\end{itemize}

\end{definition}

\subsection{Protection as a sufficient condition}

We define preservation of protection\footnote{We have a little problem with free variables in Def  \ref{def:protect:preserve}}

\begin{definition}
$~ \ $
\label{def:protect:preserve}

\begin{itemize}

\item
$M, \sigma \models  \preserve { \inside {\re} }\ \ \triangleq \ \ \forall \Mtwo, \sigma'.$\\
$\strut ~ \hspace{1.5cm} [\ M, \sigma \models\, ( \extThis \wedge \inside {\re} )\ \wedge \ \ \leadstoBoundedStar  {\Mtwo\cdot M} {\sigma} {\sigma'}\ \ \wedge\ \ M, \sigma' \models\,   \extThis\ \ \Longrightarrow \ \ M, \sigma' \models\,  \inside {\as \re \sigma}\ \ ]$


\item
$M \models  \preserve { \inside {\re} }\ \ \triangleq \ \ \forall \sigma.[\  M, \sigma \models  \preserve { \inside {\re} }\  ] $\\
 
\end{itemize}

\end{definition}

 
Lemma \ref{lemma:inside:implies:noCalls} says that $\satisfiesA{M}{\sigma} {\inside {\re}} $ and its preservation are sufficient conditions for $M, \sigma \models \  \insidef{\re}$. 
 
\begin{lemma}
$~ $ \\
\label{lemma:inside:implies:noCalls}
\begin{itemize}

\item
$\satisfiesA{M}{\sigma} {\inside {\re} }   \ \ \wedge\ \ M \models \,  \preserve { \inside {\re} }\ \ \ 
% $  \\  $\strut \hspace{1cm}
 \Longrightarrow \ \ \  M, \sigma \models \  \insidef{\re}$

\end{itemize}
\end{lemma}

Proof by unfolding the definitions. 

\subsection{Protection as an almost necessary  condition}

To obtain some intuition why lemma \ref{lemma:inside:implies:noCalls} holds, we first discuss that $\satisfiesA{M}{\sigma} {\inside {\re}} $ is, 
``almost'',  a necessary condition for $\satisfiesA{M}{\sigma} {\insidef {\re}} $. 

For this, we define the concept of extension of a state, and a module

\begin{definition}
For a class \prg{C}, method indetnifier \prg{m}, state $\sigma$, and module $M$, we define:
% $~ $ \\

\begin{itemize}

\item
$\enrich{\prg{C}}$ is a sequence of calls of form $\prg{this.f}_1.\prg{m}_1\prg{(\_)}$, ... $\prg{this.f}_n.\prg{m}_n\prg{(\_)}$  where $\prg{f}_1$, ... $\prg{f}_n$  are all the fields in class $C$ which point to external objects.

\item 
$\enrich{\prg{C.m}}$ is a sequence of calls of form $\prg{x.m(\_)}$  to all  formal parameters of external types in the method $\prg{m}$ of class  $\prg{C}$, followed by a sequence from $\enrich{\prg{C}}$.

 \item
  $\enrich{\sigma}$   is  the set of  states identical to $\sigma$, except for the continuation in the top frame, which is enriched by 
 $\enrich{\prg{C.m}}$ , where \prg{C} and \prg{m} are the class  of the receiver on the top frame, and \prg{m} is the method being executed in the top frame. 
\item
$\enrich{M}$ is an enrichment of $M$: 
it is the set of modules, that are essentially as in $M$, but where 
the body of each of the methods \prg{m} of a class \prg{C} is preceded by $\enrich{\prg{C}}$  followed by $\enrich{\prg{C.m}}$.
\end{itemize}
\end{definition}

%We first consider (1) from the definition of $ {\inside {\re}}$:
%Assume that we had a state $\sigma_o$ such that $\satisfiesA{M}{\sigma_o}{\extThis}$ and with  a variable $x$ so that  $\interpret {\sigma_o} x  = \interpret {\sigma_o} \re $.Then, we could chose modules $\Mtwo$ with method body for the current, external, receiver\footnote{But the frame in $\sigma$ contains the code of the current execution ... :-( }, so that it calls a method on $x$. This gives that we obtain a $\sigma_e'$ so that
%$\leadstoBoundedStar  {\Mtwo\cdot M} {\sigma_o} {\sigma_e}\ \ \wedge\ \ M, \sigma_e\models \,  \extThis
%\wedge\ \ M, \sigma_e \models \,   \callsOn {\re}$. And this means that $M, \sigma_o \not\models \  \insidef{\re}$.
%
%We now consider (2) from the definition of $ {\inside {\re}}$:
%Assume that we had a state $\sigma_o$ and an address $\alpha$ and a field $f$, so that $\alpha\!\in\!{\LRelevantO {\sigma_o}}$, and $\satisfiesA{M}{\sigma_o}{\external {\alpha}}$ and  
%$\interpret {\sigma_o} {\alpha.f}\! \neq \! \interpret {\sigma} {\re}$. 
%Assume that there is a further state $\sigma_e$ such that   $\leadstoBoundedStar  {\Mtwo\cdot M} {\sigma_o} {\sigma_e}$ so that $\interpret {\sigma} {\alpha}$, and that the heap of $\sigma_e$ is identical to that of $\sigma_o$. Then, we can chose the body of the current method in $\sigma_e$ to read the field $f$, and then use its value in a further call. Therefore, we have $M, \sigma_e \models \,  \extThis
%\wedge\ \ M, \sigma_e \models \,   \callsOn {\re}$. And this means that $M, \sigma_o \not\models \  \insidef{\re}$.
%
%On the other hand, $\satisfiesA{M}{\sigma} {\inside {\re}} $ is not a sufficient condition for $\satisfiesA{M}{\sigma} {\insidef {\re}} $.
%Namely, assume that $M$ had a pubic method which returned $\re$; then any external object could call that public method, obtain $\re$ and then make a call on it.
%This means that we need to also  require that $\satisfiesA{M}{\sigma} {\inside {\re}} $ is preserved by execution.

 One can show that assertion satisfaction in the context of $M$ is equivalent with satisfaction in the context of $\enrich{M}$.
 
 \begin{lemma}
 For all $M$, $\sigma$ and $A$: 
$~ $ \\
\label{lemma:equiv:}

\begin{itemize}

\item
$\satisfiesA{M}{\sigma} {A} \ \ \Longleftrightarrow \ \ \  \satisfiesA{\enrich M}{\sigma} {A} $
\item
$\satisfiesA{M}{\sigma} {A} \ \ \Longleftrightarrow \ \ \  \satisfiesA{M}{\enrich \sigma} {A} $
\end{itemize}
\end{lemma} 

Moreover, we show below that $\inside {\re} $ and its preservation are ``almost'' necessary conditions. 


\begin{lemma}
Given $M$, $\sigma$ and $\re$: ~ 

\begin{itemize}
\item
$M, \sigma \models   \preserve { \inside {\re} } \ \ \wedge \ \   M, {\sigma} \not\models \inside{\re}  \ \ \Longrightarrow  \ \ \ 
 \exists  M\!\in\! \enrich M, \sigma'\!\in\! \enrich \sigma.  [ \ \ M', \sigma' \not \models \insidef {\as \re \sigma}  \  ] $\\
 %$ \exists \Mtwo, M', \sigma', \sigma''.$\\
%$\strut \hspace{3cm} [ \  M'\!\in\! \enrich M\ \wedge \  \sigma'\!\in\! \enrich \sigma\ \wedge\   \leadstoBoundedStar  {\Mtwo\cdot M'} {\sigma'} {\sigma''} \ \wedge\ 
%M, \sigma'' \models \callsOn {\re}\ ] $
 
\item
$M, \sigma \not\models   \preserve { \inside {\re} } \ \ \wedge \ \   M, {\sigma} \models\  \inside{\re}  \ \ \Longrightarrow \ \ \ 
 \exists  M\!\in\! \enrich M, \sigma'\!\in\! \enrich \sigma.  [ \ \ M', \sigma' \not \models \insidef {\as \re \sigma}  \  ] $\\
%\exists \Mtwo, M', \sigma', \sigma''$\\
%$\strut \hspace{3cm} [ \  M'\!\in\! \enrich M\ \wedge \  \sigma'\!\in\! \enrich \sigma\ \wedge\  \leadstoBoundedStar  {\Mtwo\cdot M'} {\sigma'} {\sigma''} \ \wedge\ 
%M, \sigma'' \models \callsOn {\re}\ ] $
\end{itemize}
\end{lemma} 
 
 aa
\end{document}
