\section{Proving Adherence to \SpecLang Specifications}
\label{s:inference}

In this Section we provide a proof system for constructing 
proofs of the \SpecLang specifications defined in \S \ref{s:holistic-guarantees}.
%As discussed in \S \ref{s:approach},
Such proofs consist of 
 three parts: % proof of a \SpecLang specification:
\begin{description} 
\item[(Part 1)]
Proving Assertion Encapsulation (\S \ref{s:encaps-proof})
\item[(Part 2)]
Proving that   method bodies adhere to specifications written in \AssertLang (\S \ref{s:classical-proof})
\item[(Part 3)]
Proving that modules adhere to \SpecLang specifications (\S \ref{s:module-proof})
\end{description}

Part 1 is, to a certain extent, orthogonal to the main aims of our work;
in this paper we propose a simple approach based on the type system, while also acknowledging that 
better solutions are possible.
For Parts 2-3, we 
TODO
% came up with the key ideas outlined in  \S \ref{s:approach}, which we
% develop in more detail in \S \ref{s:classical-proof}-\S \ref{s:emergent-proof}.}


\subsection {Assertion Encapsulation}
\label{s:encaps-proof}

{
%A key component of constructing 
{\SpecLang proofs  hinge on the fact that some assertions cannot be invalidated unless some 
} internal (and thus known)
computation took place. 
{We refer to this property as \emph{Assertion Encapsulation}.}
}
and define the property $M\ \models  \encaps{A}$, which states that 
 assertion $A$ is encapsulated by module $M$.
We  do not mandate how this property should be derived -- instead, we rely on a judgment 
$M\ \vdash  \encaps{A}$ provided by some external system. \footnote{This is simpler than the oopsla-33 setting}
Thus, \SpecLang is parametric over the derivation of the encapsulation
     judgment; in fact, several ways to do that are possible \cite{TAME2003,ownEncaps,objInvars}. For example,
 the appendices of
    \cite{necessityFull} present a 
	%Appendix~\ref{s:encap-proof} and
    %Figure~\ref{f:asrt-encap}  we present a 
    rudimentary system that is sufficient to support our example
    proof.  

% \subsubsection{Assertion Encapsulation Semantics}


{As we said earlier,  an assertion $A$  is  encapsulated by a module $M$ under condition $A'$,
if in all possible states which arise from execution of module $M$ with any other external module $M_{ext}$, and which satisfy $A'$, 
the validity of $A$} 
{ can only be changed via computations internal to that module} -- \emph{i.e.},  via a call to
a method from $M$, i.e.,
calls to objects defined in $M$ but accessible from the
outside.


\begin{definition}[Assertion Encapsulation]
\label{def:encapsulation}
An assertion $A$ is \emph{encapsulated} by module $M$ and assertion $A'$, written as\\
\strut \hspace{1cm}   $M\ \vDash \encaps{A}$, \\
if and only if   % \\ \strut \hspace{1cm} 
for all external modules $M_{ext}$, and all states $\sigma$, $\sigma'$
such that $\arising{M_{ext}}{M}{\sigma}$:

\begin{tabular}{lr}
$\;\;\;\;$- $\reduction{M_{ext}}{M}{\sigma}{\sigma'}$  & \rdelim\}{3}{4mm}[ $\;\;\;\Rightarrow\;\;\;$  $\exists x,\ m,\ \overline{z}. (\ 
  \sigma.\prg{cont}= x.m(\overline{z})\ \wedge\  \satisfiesA{M}{\sigma}{\internal{x}}
\ )$] \\
$\;\;\;\;$- $\satisfiesA{M}{\sigma}{A}$ \\
$\;\;\;\;$- $\satisfiesA{M}{\sigma' }{\neg A}$ \\
\end{tabular} 
\end{definition}


\noindent
\footnote{We might not need view adaptation, but if we need, we have to also say Note that this definition   uses adaptation, 
${\sigma' \triangleleft \sigma}$. The application of the adaptation operator is necessary
because we  interpret the assertion $A$ in the current state, $\sigma$, while we interpret the assertion $\neg A$ in 
the future state, $\sigma' \triangleleft \sigma$. 
Here the original def

\begin{tabular}{lr}
$\;\;\;\;$- $\reduction{M_{ext}}{M}{\sigma}{\sigma'}$  & \rdelim\}{3}{4mm}[$\;\;\;\Rightarrow\;\;\;$  $\exists x,\ m,\ \overline{z}. (\ \satisfiesA{M}{\sigma}{\calls{\_}{x}{m}{\overline{z}} \wedge\ \internal{x}}\ )$] \\
$\;\;\;\;$- $\satisfiesA{M}{\sigma}{A \wedge  A'}$ \\
$\;\;\;\;$- $\satisfiesA{M}{\sigma' \triangleleft \sigma}{\neg A}$ \\
\end{tabular} 
}

Revisiting the examples from \S~\ref{s:outline}, % we can see
both \ModB and \ModC encapsulate   the  {equality of the \prg{balance} of an account to some value \prg{bal}: }
 {This equality can only be invalidated} through calling  methods on internal objects.\footnote{Here 
 the point would be better made if we used the module with the ledger, ie that \prg{a.bal} is a ghost field}
%
\\
\strut \hspace{1cm}
$\ModB\ \vDash\ \encaps{ \prg{a}:\prg{Account}\ \wedge \prg{a.balance}=\prg{bal}}$
\\
\strut \hspace{1cm}
$\ModC\ \vDash\encaps{ \prg{a}:\prg{Account}\ \wedge \prg{a.balance}=\prg{bal}}$


The property that a variable is protected from another one is not encapsulated, but  the property that a variable is protected \emph{is}encapsulated, regardless of the module. Note also, that  encapsulation of an assertion does not imply encapsulation of its negation; 
 for example,  ${\inside{x}}$ is encapsulated (as per below), but $\neg  {\inside{x}}$ is not. \footnote{Susan wanted to omit this for oospla, but depends on target conference, we whould keep it.}



\begin{lemma}
For any module $M$, and variables $x$ and $y$:
\begin{itemize}
\item $M \models \encaps{\inside{y}}$
\item $M \models \neg{\encaps{\inside{y}}}$
\item $M \not\models \protectedFrom{y}{x}$
\end{itemize}
\end{lemma}

TODO outline the proofs? 
\noindent 
%The key consequence of soundness is that -- SD dropped; it is   not a consequence of soundness!


{In general},  code that does not contain 
calls to a {given} module is guaranteed not to invalidate any assertions encapsulated by that module.
 Assertion encapsulation has been used in proof systems to {address}   the  {frame} problem
 \cite{objInvars,encaps}. 

\subsubsection{\sdN{Deriving} Assertion Encapsulation}

%As we have already stated at the beginning of this section,
%encapsulation is a deep topic that is well studied in the literature, 
%and is not the focus of this paper. For now, we simply assume the existence 
%of a proof system for encapsulation as it is secondary to the central topic 
%of this paper. We need only assert that such an algorithmic proof system 
%must be sound (Definition \ref{lem:encap-soundness}).
%% \susan[I commented out what was there as I thought it was repetious]
%% {We are assuming the existence of a proof system for encapsulation and only need to assert that such an algorithmic proof system nust be sound.}
%% The construction of the algorithmic system is not central to our work,
%% because, as we shall see in later sections, our logic 
%% does not rely on the specifics of an encapsulation algorithm, only its soundness.

Like OOPSLA 22 Our logic does not {deal with, nor} rely on, the specifics of  how   encapsulation
{is derived}.
  % model, 
{Instead, it relies} on an encapsulation judgment and expects it to be sound:

\begin{definition}[Encapsulation Soundness]
\label{lem:encap-soundness}
A judgement of the form $\proves{M}{\encaps{A}}$  is\  \emph{sound}, \ if 
for all modules $M$, and assertions $A$:\\

$\strut \hspace{1.5cm} \proves{M}{\encaps{A}} \ \ \ \ $ implies $\ \ \ \ \satisfies{M}{\encaps{A}}$.
\end{definition}




\paragraph{Types for Assertion Encapsulation}
\label{types}
TODO: somethibg simple ere 
%\sdNr[I have unified the two separate descriptions of the types system.]{}
%%To allow for an easy way to judge encapsulation of
%\sdN{Even though the derivation of assertion encapsulation  is not the focus of this paper, 
%for illustrative purposes, we will outline %here how it can be derived with the use of 
%now a  very simple type system which supports such derivations:}
%We assume that 
%%assertions, we assume a very simple type system, where 
%field declarations, method arguments
%and method results are annotated with class names, and that classes may  
%be annotated as \enclosed. A  \enclosed object  
%\sdN{is not} accessed by external objects; that is, it is always \inside. 
%
%The type system then checks 
%that field assignments, method calls, and method returns adhere to these expectations,
%and in particular, that objects of \enclosed type
%are never returned from method bodies 
%\sdN{-- this is a simplified version of the type system described in \cite{confined}.}
%Because the type system is so simple, we do not include its formalization in the paper.
%Note however, that the type system has one further implication: modules are typed 
%in isolation, thereby implicitly prohibiting
%method calls from internal objects to external objects. 
%
%Based on this type system, we define a predicate $\intrnl{e}$, in 
%\jm[]{the appendices %of the full paper 
%\cite{necessityFull},} %Appendix \ref{s:encap-proof},
%which asserts that any \sdN{objects read} during the evaluation of $e$ are internal.
%Thus, any assertion that only involves $\intrnl{\_}$ expressions is encapsulated -- more can be found in 
%\jm[]{the appendices %of the full paper 
%\cite{necessityFull}.} %in Appendix \ref{s:encap-proof}.

\subsection{Method bodies adhere to \AssertLang specifications}
\label{s:classical-proof}
 
We now develop a Hoare logic, which can prove assertions of the from \\
\strut \hspace{1cm} $\hproves{M}{A}{\prg{s}}{A'}$.\\
where \prg{s} is a statement in \Loo, and $A$ and $A'$ are assertions in \AssertLang.

The challenges here are 1) that \AssertLang assertions support, on top of the classical features, also ??what-shall-we-call-them? protection features, and 2) we need to reason about calls to external modules.


We assume that there exists some
proof system  that   allows us to prove 
 specifications of the form  $\hproves{M}{A}{\prg{s}}{A'}$.
{We further assume that such a proof system is sound, i.e. that 
if xxx TODO 
% if $\hproves{M}{\hoare{P}{\prg{res = x.m($\overline{z}$)}}{Q}}$, then 
% for every program state $\sigma$ that satisfies $P$, the execution of the method call \prg{x.m($\overline{z}$)}
 % esults in a program state satisfying $Q$.}
 We then expand the proof rules as follows ....

\subsubsection{Reasoning about protection}
We expand that logic with rules about protection, as in Fig. 

\begin{figure}[t]
\footnotesize
\begin{mathpar}
\infer
	{ }
	{\hproves{M} 
						{x\neq u \wedge \protectedFrom{y}{x}}
						{u:=v}
						{ \protectedFrom{y}{x}}
	}
	\quad(\textsc{prot-1})
	\and
\infer
	{zzz	}
	{
	yyy
	}
	\quad(\textsc{prot-2})
\end{mathpar}
\caption{Protection Logic}
\label{f:protection}
\end{figure}

 

 \textsc{xxxl} states that   yyy
  
 
\subsubsection{reasoning about calls to unknown code}

xxx TODO add the rule we have xxxx

\subsection{Proving \SpecLang Specifications}
\label{s:module-proof}

\begin{figure}[thb]
\footnotesize
\begin{mathpar}
\infer
	{
	M \vdash \encaps{A}
	\\\\
	\textit{for all}\ \ C \in dom(M)\ \ \textit{and}\ \  m \in M(C).\prg{exprt\_mths},   \textit{and} \prg{body}=M(C,m)}\\\\
				 \hproves{M}{\prg{this}:\prg{C} \wedge At} {\prg{mBody}} {A}
						.
	}
	{
	M\ \vdash\ \TwoStates{A}{A}
	}
	\quad(\textsc{module-invariant})
	\end{mathpar}
\caption{Module invariants}
\label{f:module:invariats}
\end{figure}

TODO explain. Also, we only look at the methods exported from the module.


\subsection{Soundness of the \SpecLang Logic}

\label{s:soundness}

\begin{theorem}[Soundness]
\label{thm:soundness}
Assuming a sound \SpecO proof system, $\proves{M}{A}$, and
a sound encapsulation inference system, $\proves{M}{\givenA{A}{\encaps{A'}}}$,
 and  that on top of these systems we built
 the \SpecLang logic according to the rules in Figures \ref{f:classical->singlestep},  \ref{f:only-if-single}, 
 \ref{f:only-through},  and \ref{f:only-if},   then, for    all modules $M$, and all \SpecLang specifications  $S$:
 
 $$\proves{M}{S}\ \ \ \ \ \ \ \mbox{implies}\ \ \ \ \ \  \ \ \ \satisfies{M}{S}$$
\end{theorem}

\begin{proof}
by induction on the derivation of $\proves{M}{S}$.
\end{proof}
%\jm[]{The proof of soundness (Theorem. \ref{thm:soundness}) proves
%that our proof system for \SpecLang adheres to the semantics of \SpecLang specifications.
%We make two assumptions for soundness: (1) a sound proof system for assertion encapsulation, 
%and (2) a sound proof system for \SpecO. It is notable that \SpecLang is parametric with both 
%of these judgments.}

Theorem. \ref{thm:soundness} demonstrates 
 that the   \SpecLang logic is sound with respect to the semantics of \SpecLang specifications.
 The \SpecLang logic parametric wrt to the algorithms for proving validity of assertions
 $\proves{M}{A}$, and 
 assertion encapsulation ($\proves{M}{\givenA{A}{\encaps{A'}}}$), and is sound
 provided that these two proof systems are sound.


 xxxxx