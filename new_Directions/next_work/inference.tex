We will now develop an inference system to prove that a module % is well-formed $\vdash M$, \ie that a module 
satisfies its specification. This is done in three phases.
%\TODO{\red{TODO: refer to what we said in approach section}}

In the first phase we develop a logic of triples ${\hproves{M}  {A} {\ s\ }{A'} }$, which have the expected meaning, \ie 
(*) execution of statement $s$ in a state satisfying {the \emph{precondition}}  $A$ will lead to a state satisfying  {the \emph{postcondition}}  $A'$.
These triples only apply to statements $s$ that  do not contain method calls  (even internal calls) -- this is so, because method calls may contain calls to external methods, and therefore can only be described through quadruples.
Our triples extend an underlying Hoare logic  (${M \vdash_{ul}  \{A\} {\ s\ } \{A'\} }$) and  introduce new judgements  which talk about protection.

In the second phase we develop a logic of quadruples ${\hprovesN{M}  {A} {\ s\ }{A'} {A''}}$. These promise, as usual, that (*). 
In addition, they promise that (**) any intermediate external states reachable during execution of that statement will satisfy the invariant  $A'$.  
% {We call $A''$ the \midcond.}
 We incorporate all triples from the first phase,       
introducing invariants, give the usual substructural rules, and deal with method calls. 
For method calls we use the methods' specs, and in addition, for public methods we use the fact that they preserve the module's invariants. 
For calls to external  methods, we   use %can only use  the fact that they preserve
 the module's invariants. 
 
In the third phase, we prove that a modules adherences to its specifications. 
For method specifications we prove that the body maps the precondition to the postcondition and preserves the method's invariant. 
For module invariants we prove that they  are preserved by the public methods of the module.



\subsection{Preliminaries: Specification Lookup,  Renamings, Underlying Hoare Logic}

% Before describing the three phases, we  start with some preliminaries.
First some preliminaries: % shorter 
The judgement    $\promises M S$ expresses that $S$ is part of $N$'s specification.  
In particular, it allows   \emph{safe  renamings}. 
These renamings are   a convenience, akin to the Barendregt convention, and  allow simpler Hoare rules  -- \cf Sect. \ref{sect:wf},
Def. \ref{d:promises}, and Ex. \ref{e:rename}. We also require an underlying Hoare logic with judgements $M \vdash_{ul} \{ A \} stmt \{ A' \}$, 
% with  $\Stable{A}$,  $\Stable{A'}$ 
-- \cf Ax. \ref{ax:ul}.



\subsection{First Phase: Triples}

In  Fig. \ref{f:underly}  we introduce our triples, of the form ${   \hproves{M}  {A} {stmt}  {A'}}$. 
These promise, as expected, that any execution of $stmt$ in a state   satisfying $A$ leads to a state satisfying $A'$.
% chopped; we say it later in the lemma
%These triples only apply to statements that do not contain method calls.

{\small{
\begin{figure}[tht]
$
\begin{array}{c}
\inferruleSD{\hspace{3.5cm} [\sc{embed\_u}]}
	{\Stable{A}  \ \  \Stable{A'}  \ \ \ \ \ \ \ \ \ \ { M \vdash_{ul} \{ \ A\ \} {\ stmt\ }\{\ A'\ \}} \ \ \ \ \ \ \ \ \ \ 
	 stmt  \ \mbox{contains no method call}   
	 }
	{\hproves{M}  {A} {\ stmt\ }{A'} } 
\\	
\begin{array}{lcl}
\inferruleSD{\hspace{0.5cm} [\sc{prot-new}]}
	{
	\sdN{u \txtneq   x	 }
	}
	{	 
 	\hproves{M} 
 						{ \sdN{true} } %{\  \internal {u}   \ }
 					{\  u = \prg{new}\ C \ }
 						  {  \inside{u}\  \wedge  \protectedFrom{u}{x}} 
 	}
&
&
	{\inferruleSD{\hspace{0.5cm} [\sc{prot-1}]}
	{   stmt \ \mbox{ is free of  method cals, or assignment to $z$}
	\\
	{\hproves{M}  {\re = z} {\ stmt\ }{ \re=z} }
	}
	{\hproves{M} 
						{\  \inside{\re}  \ }
						{\  stmt \ }
						{\  \inside{\re}\ }
	}
}
% \end{array}
%\\
% this onw ia wronf!!!
%\\
%	{{\inferruleSD{\hspace{2.3cm} [\sc{prot-2}]}
%	{   stmt \ \mbox{contains no method call, and no assignment to $z$ or $z'$}
%	\\
%	{\hproves{M}  {\re = z \wedge \re' = z'} {\ stmt\ }{\re = z \wedge \re' = z'} }
%	}
%	{\hproves{M} 
%						{\  \protectedFrom{\re}{\re'}  \ }
%						{\  stmt \ }
%						{\  \ \protectedFrom{\re}{\re'}\ }
%	}
%	}}
%\\
%\\
%\mbox{\sdN{We had this rule, but I believe  that it is not sound when $y$=$z$, or $x$=$y$}}
\\ \\
      {\inferruleSD{\hspace{0.5cm} [\sc{prot-3}]}
	{ }
	{\hproves{M} 
						{\ \protectedFrom{y.f}{z}\ }
						{\ x =y.f\ }
						{\ \protectedFrom{x}{z}\ }
	}
}
%
%\\
%\\
%\sdN{\mbox{Is his rule is subsumed by {\sc{Prot-1}}}}
%\\
%       {\inferruleSD{\hspace{5.3cm} [\sc{prot-???}]}
%	{ }
%	{\hproves{M} 
%						{\  \inside{x}\ \wedge \   \protectedFrom{x}{y'} \ }
%						{\ y.f=y'\ }
%						{\  \inside{x}\ }
%	}
%}	
%\\
& &
        {\inferruleSD{\hspace{0.5cm} [\sc{prot-4}]}
	{ }
	{\hproves{M} 
						{\ \protectedFrom{x}{z} \ \wedge \   \protectedFrom{x}{y'} }
						{\ y.f=y'\ }
						{\ \protectedFrom{x}{z} \ }
	}
}	
\\
\\
{\inferrule
	{C \in M}
	{M \vdash x : C \rightarrow x : \prg{intl}}
	\quad[\textsc{Intl-M}]
	}
& &
{\inferrule
	{M \vdash x \neq y \wedge y : \prg{intl} \rightarrow \protectedFrom{x}{y}}
	{}
	\quad[\textsc{Prot-Intl}]
	}
\end{array}
\end{array}
 $
\caption{Embedding the Underlying Hoare Logic, and Protection}
\label{f:protection}
\label{f:underly}
\end{figure}
}}


With rule {\sc{embed\_u}} in Fig. \ref{f:underly},  any assertion $M \vdash_{ul} \{ A \} s \{ A' \}$  whose statement does not contain a method call, and which 
can be proven in the underlying Hoare logic, can also be proven in our logic. %More in \ref{s:types}.
In \textsc{Prot-1}, we see that  protection of an object $o$ is preserved by internal code which does not call any methods: namely any heap modifications will
ony affect internal objects, and this will not expose new, unmitigated external access to $o$.
%In \textsc{Prot-2}, the assignment $y=y'.f$ does not create a new route from $z$ to $x$, and therefore  $\protectedFrom{x}{z}$ is preserved.
\jm{In \textsc{Prot-3}, if $y.f$ is protected from some $z$, then the field access $x=y.f$ retains that protection, and therefore $\protectedFrom{x}{z}$.}
In \textsc{Prot-4}, if $x$ is protected from $y'$, then the assignment $y.f=y'$ does not create a new unprotected route  to $x$, and therefore   $\protectedFrom{x}{z}$ is preserved.
}
Moreover, ``protection'' of an object can decrease is if we call an external method, and pass it an internal object as argument. This is then covered by the rule in Fig. \ref{f:external:calls}.
\jm{Moreover, we introduce some consequence rules in Fig. \ref{f:protection:conseq} to reason about protection apart from a Hoare logic. These are primarily useful when applying \textsc{Consquence} style Hoare rules.} 



\begin{lemma}
\label{l:no:meth:calls}
If ${\hproves{M}  {A} {\ stmt\ }{A'} }$, then $stmt$ contains no method calls.
\end{lemma}
  
%\jm{Fig. \ref{f:protection} extends an assumed underlying Hoare logic with rules for reasoning about protection. To supplement this, we introduce some consequence rules in Fig. \ref{f:protection:conseq} to reason about protection apart from a Hoare logic. These are primarily useful when applying \textsc{Consquence} style Hoare rules.}


%\begin{figure}[htb]
%\begin{mathpar}
%%\inferrule
%%	{M \vdash x : \prg{int} \rightarrow \protectedFrom{y}{x} }
%%	{}
%%	\quad[\textsc{Prot-Int}_1]
%%	\and
%%\inferrule
%%	{M \vdash x : \prg{int} \rightarrow \neg \protectedFrom{x}{y}}
%%	{}
%%	\quad[\textsc{Prot-Int}_2]
%%	\and
%%\inferrule
%%	{M \vdash x : \prg{bool} \rightarrow \protectedFrom{y}{x} }
%%	{}
%%	\quad[\textsc{Prot-Bool}_1]
%%	\and
%%\inferrule
%%	{M \vdash x : \prg{bool} \rightarrow \neg \protectedFrom{x}{y}}
%%	{}
%%	\quad[\textsc{Prot-Bool}_2]
%%	\and
%%\inferrule
%%	{M \vdash x : \prg{str} \rightarrow \protectedFrom{y}{x} }
%%	{}
%%	\quad[\textsc{Prot-Str}_1]
%%	\and
%\inferrule
%	{C \in M}
%	{M \vdash x : C \rightarrow x : \prg{intl}}
%	\quad[\textsc{Intl-M}]
%	\and
%\inferrule
%	{M \vdash x \neq y \wedge y : \prg{intl} \rightarrow \protectedFrom{x}{y}}
%	{}
%	\quad[\textsc{Prot-Intl}]
%\end{mathpar}
%\caption{Consequence Rules for Protection}
%\label{f:protection:conseq}
%\end{figure}

%\jm{Julian: I added the above, but these are not sensible with the current definition of protection because protection is only defined between objects, and not ints, bools, and strings. They are necessary for the example proofs (or alternatively we can use object encodings of these types, but I think that might be more difficult). The rules for strings are potentially not necessary. They only help make the Seciton 2 example work when we send back a message as a string. Similarly, we could probably drop the bool rules.}

\subsection{Second Phase: Quadruples}

\subsubsection{Introducing invariants, and substructural rules}
% Fig. \ref{f:substructural} 
We now introduce  quadruple rules. %  of the from ${\hprovesN{M}  {A} {\ s\ }{A'} {A''} }$. % which do not deal with method calls.
Rule {\sc{mid}} embeds  triples  ${\hproves{M}  {A} {\ s\ }{A'} }$  into quadruples ${\hprovesN{M}  {A} {\ s\ }{A'} {A''} }$; this is sound, because $s$ is guaranteed not to contain method calls (by lemma \ref{l:no:meth:calls})




%\begin{figure}[htb]
% \vspace{-.4cm}
\begin{center}
$
\begin{array}{c}
% \begin{array}{lcl}
\inferruleSD{[\sc{Mid}]}
	{\hproves{M}  {A} {\ s\ }{A'} }
	{\hprovesN{M}  {A} {\ s\ }{A'} {A''} }
% MOVED
% THIS RULE IS NOT SOUND -- we need to adapt it
%{
%\inferrule[\sc{types-2}]
%	{ \hprovesN{M}  {A} {\ s\ }  {A'} {A''}  }
%	{\hprovesN{M}  {x:C \wedge A} {\ s\ }  {x:C\wedge A'} {A''}}
%}
% \end{array}
% \\
% \\
% *** MOVED **
%\begin{array}{lcl}
%\inferruleSD{[\sc{combine}]}
%	{  \begin{array}{l}
%	\hprovesN{M}  {A_1} {\ s\ } {A_2}  {A} \\ % \hspace{1.4cm}  
%	\hprovesN{M}  {A_3} {\ s\ } {A_4} {A}
%	\end{array}
%	}
%	{ \hprovesN{M}  {A_1 \wedge A_3 }{\ s\ } {A_2 \wedge A_4} {A} }
%& &
%\inferruleSD{[\sc{sequ}]}
%	{  \begin{array}{l} 
%	\hprovesN{M}  {A_1} {\ s_1\ } {A_2}  {A}  \\ % \hspace{1.4cm} 
%	\hproves{M}  {A_2} {\ s_2\ } {A_3} {A}
%	\end{array}
%	}
%	{   \hprovesN{M}  {A_1   }{\ s_1; \, s_2\ } {  A_3} {A} }
%\end{array}
%\\ \\
%\inferruleSD{ \hspace{3cm} [\sc{consequ}]}
%	{
%%	\begin{array}{l}
%	 { \hprovesN  {M}  {A_4} {\, s\, } { A_5} {A_6}  }
%	 \hspace{1.4cm} 
%	 M \vdash A_1 \rightarrow A_4 
%	 \hspace{1.4cm} 
%	{ M \vdash A_5   \rightarrow  A_2  }
%	 \hspace{1.4cm}   
%	{  M \vdash A_6 \rightarrow A_3 }
%%	\end{array}
%	}
%	{   \hprovesN{M}  {A_1 }{\ s\ } {A_2} {A_3} }
  \end{array}
 $
%  \vspace{-.4cm}
%\caption{Hoare Quadruples - incorporating Hoare triples }
%\label{f:substructural}
%\end{figure}
%  \vspace{-.3cm}
 \end{center}
 
The remaining substructural quadruple rules appear in  Fig. \ref{f:substructural:app}, and are as expected: 
Rules   {\sc{sequ}} and {\sc{consequ}} are  the usual rules for statement sequences and consequence, adapted to quadruples.
Rule {\sc{combine}} combines two quadruples for the same statement into one.
The last three  rules  apply to \emph{any} statements -- even those containing method calls.



\subsubsection{Reasoning about   calls}
\label{s:calls}
is described in Fig. \ref{f:internal:calls}. {\sc{Call\_Int}}  and {\sc{Call\_Int\_Adapt}}  for internal methods, whether public or private; \ % \  {\sc{Call\_Ext}} 
and {\sc{Call\_Ext\_Adapt}} for  external methods.




\begin{figure}[htb]
{\small{
$\begin{array}{c}
 \inferruleSD{\hspace{4.7cm} [\sc{Call\_Int}]}
	{
	   	\begin{array}{c}
		\promises  M {\mprepostN{A_1}{p\ C}{m}{x}{C}{A_2} {A_3}}  \\
		{A_1' = A_1[y_0,\overline y/\prg{this}, \overline x]  \ \ \ \ \ \ \ \ \  A_2' = A_2[y_0,\overline y,u/\prg{this}, \overline x,\prg{res}]}  
		          	\end{array}
		}
	{  \hprovesN {M} 
						{ \  y_0:C,\overline {y:C} \wedge  {A_1'}\ }  
						 { \ u:=y_0.m(y_1,.. y_n)\    }
					         { \ {A_2'} \ } 
						{  \ A_3 \ }  
}
 \\
  \\
{ \inferruleSD{\hspace{4.7cm} [\sc{Call\_Int\_Adapt}]}
	{
	   	\begin{array}{c}
		\promises  M {\mprepostN{A_1}{p\ C}{m}{x}{C}{A_2} {A_3}}   \\
				{A_1' = A_1[y_0,\overline y/\prg{this}, \overline x]  \ \ \ \ \ \ \ \ \  A_2' = A_2[y_0,\overline y,u/\prg{this}, \overline x,\prg{res}] }  

          	\end{array}
		}
	{  \hprovesN {M} 
						{ \,  y_0\!:\!C, \overline {y\!:\!C} \wedge {\PushASLong {{(y_0,\overline {y})}} {A_1'}}\ }
												{ \, u\!:=\!y_0.m(y_1,.. y_n)\,    }
						{ \  { \PushASLong  {{(y_0,\overline {y})}} {A_2'}  } \ }  
						{  \  A_3\  }  
}
}
\\
 \\ 
 \inferruleSD{\hspace{4.7cm} [\sc{Call\_Ext}\_Adapt]}
 	{ 
	 \promises M   {\TwoStatesN {\overline {x:C}} {A}} 
        }
	{   \hprovesN{M} 
						{ \    { \external{y_0}} \,     \wedge \,  \overline{x:C}\  \wedge\ {\PushASLong {{(y_0,\overline {y})}}  {A}}  \ } 
						{ \ u:=y_0.m(y_1,.. y_n)\    }
						{ \   {\PushASLong {{(y_0,\overline {y})}}  A}  \ }
						{\  A \   }
	}	
\\
 \\ 
{
 \inferruleSD{\hspace{4.7cm} [\sc{Call\_Ext}\_Adapt\_Strong]}
 	{ 
	 \promises M   {\TwoStatesN {\overline {x:C}} {A}} 
        }
	{   \hprovesN{M} 
						{ \    { \external{y_0}} \,     \wedge \,  \overline{x:C}\ \wedge  A   \wedge\ {\PushASLong {{(y_0,\overline {y})}}  {A} }\  }   
						{ \ u:=y_0.m(y_1,.. y_n)\    }
						{ \   A \wedge {\PushASLong {{(y_0,\overline {y})}}  {A} } \  }  
						{\  A \   }	
}
}

\end{array}
$
}}
\caption{Hoare Quadruples for Internal and External Calls -- here $\overline y$ stands for $y_1, ... y_n$}
\label{f:internal:calls}
\label{f:external:calls}
\label{f:calls}
\end{figure}

For the internal calls, we  start as usual by looking up the method's specification, and 
naming the formal   parameters in the method's pre- and post-condition.
  {\sc{Call\_Int}} is as expected:   we  require the precondition, and guarantee the postcondition and invariant.
For {\sc{Call\_Int\_Adapt}} we require the adapted pre-condition ($  \PushASLong {(y_0, \overline y)}{A_1'}$  rather than $A_1'$) and also ensure the adapted post-condition ($ \PushASLong {(y_0, \overline y)}{A_2'}$  rather than $A_2'$).
Remember that$  \PushASLong {(y_0, \overline y)}{A_1}$ at the caller's side guarantees that $A_1$ holds at the start of the call  (after pushing the frame with   $y_0, \overline y$), while  
$A_2$ at the end of the cal  guarantees that  $ \PushASLong {(y_0, \overline y)}{A_2}$  holds when returning to the caller's   (after popping the called frame)
-- cf.  lemma \ref{lemma:push:ass:state}.
 {\sc{Call\_Int}} and {\sc{Call\_Int\_Adapt}}  are applicable whether the method is public or private.


For external methods,  {\sc{Call\_Ext\_Adapt}}, we consider the module's invariants. If the module promises to preserve $A$, \ie if  $\promises M   {\TwoStatesN {\overline {x:D}} {A}}$, and   $ \PushASLong {(y_0, \overline y)}{A}$  holds before the call, then it also holds after  the call.
% SHORTER
%, and $A$ is also a \midcond of the call.
In  {\sc{Call\_Ext\_adapt}},  we require that the adapted version,  \ie that   $ \PushASLong {(y_0, \overline y)}{A}$ holds before the call. Then, the 
adapted version version also holds after the call. 
% SHORTER
% Moreover, $A$ is also a \midcond of the call.


\vspace{.1cm}

Notice that %at the point of the call, 
 for internal calls, in {\sc{Call\_Int}} we require the the \emph{un-adapted} %version of the the 
 method  precondition (\ie $A_1'$), while for external calls, both {\sc{Call\_Ext\_Adapt}} and {\sc{Call\_Ext\_Adapt\_Strong}}, we require the 
 \emph{adapted} % version of the 
 invariant (\ie $ \PushASLong {(y_0, \overline y)}{A}$). 
This is so, because when the called is internal, then  %variable-free, 
 $\Pos{\_}$-assertions are preserved against pushing of frames -- \cf Lemma \ref{l:preserve:asrt}. 
On the other hand, when the called is external, then $\Pos{\_}$-assertions are not necessarily preserved against pushing of frames -- \cf Ex. \ref{push:does:not:preserve} 
Therefore, in order to guarantee that $A$ holds upon entry to the called, we need to know that $ \PushASLong {(y_0, \overline y)}{A}$ held at the caller site -- \cf Lemma \ref{lemma:push:ass:state}.

Remember also, that $A$ does not imply $ \PushASLong {(y_0, \overline y)} {A}$, nor does $ \PushASLong {(y_0, \overline y)}{A}$  imply $A$ -- \cf example \ref{push:does:not:imply}.


Finally notice, that while $\Pos{\_}$,-assertions are preserved against pushing of internal frames, they are not necessarily preserved against popping of such frames \cf Ex. \ref{ex:motivate:scoped}.
Nevertheless, {\sc{Call\_Int}} guarantees the unadapted version, $A$, 
upon return from the method call. 
This is sound, because of our % use  of the concept of
 \emph{\scoped satisfaction} of assertions -- more in Sect.  \ref{s:scoped:valid}.
 



\TODO{TODO: revisit that section}

%{Rules  {\sc{CallAndAlias}}  and  {\sc{CallNonAlias}} say that calls preserve aliasing, resp. non-aliasing, between variables, ie they preserve $x=x$ resp. $x\neq x$. These two rules apply to internal as well as external calls. When the callee's receiver is external, they make the extra requirement that  $\PushAS {y}{\extract{M}}$ -- we  will discuss this requirement together with the discussion of Fig. \ref{f:external:calls}.     Note that $x=x'$ expresses that $x$ and $x'$ are aliases, while  $u\txteq x$ expresses that $u$ and $x$ are textually the same --
%the latter is stronger, i.e.   $x\txteq u$ implies $x=u$. 
%% It is possible that variables are aliases, without being textually the same, i.e. it is possible to have $x=x'$ while $u\not\equiv x'$. 
% As $...\equiv ...$ is a textual assertion, and thus  state-independent,  it is a side-condition of the rules and is  not part of the Hoare triple's precondition.
%}
%
%
%\small{
%\begin{figure}[hbt]
%$\begin{array}{c}
%\inferruleSD{\hspace{4.7cm}  [{\sc{CallAndAlias}}}
%	{ 
%	{   x \txtneq u\txtneq x'  }
%	}
%	{   \hproves{M}  { \ x=x'\   \wedge \ ({\external{y_0}}  \rightarrow \  \PushAS {y}{\extract{M}})\  }	{ \ u:=y_0.m(y_1,.. y_n)\  } { \  x=x'\ }	 }
%
%\\ \\ 
%\inferruleSD{\hspace{4.7cm} [{\sc{CallNonAlias}}]}	
%{ 
%		{ \ x \txtneq u \txtneq x'\   }
%	}
%	{   \hproves{M}   { \ x\neq x'\   \wedge \ ({\external{y_0}}  \rightarrow \  \PushAS {y}{\extract{M}})\  } { \ u:=y_0.m(y_1,.. y_n))\  } { \  x\neq x'\ }	 }
%\\
%\\
%\end{array}
%$
%\caption{Logic for Aliasing around Calls }
%\label{f:internal:alias:calls}
%\end{figure}
%}




\subsection{Third phase: Proving adherence to Module Specifications}
\label{sect:wf}

In Fig. \ref{f:wf} we  define the judgment $\vdash M$, which says that  
$M$ has been proven to be well formed. 
 

\begin{figure}[thb]
$
\begin{array}{l}
\begin{array}{lcl}
\inferruleSDNarrow 
{~ \strut  {\sc{WellFrm\_Mod}}}
{  \vdash \SpecOf {M}
  \hspace{1.2cm}  M \vdash \SpecOf {M}
}
{
\vdash M  
}
& \hspace{0.7cm} &
\inferruleSDNarrow 
{~ \strut   {\sc{Comb\_Spec}}}
{  
M \vdash S_1 \hspace{1.2cm}  M \vdash S_2
}
{
M \vdash S_1 \wedge S_2
}
\end{array}
\\
\inferruleSD 
{~ \strut \hspace{6.5cm} {\sc{method}}}
{  
 \prg{mBody}(m,D,M)=p \ (\overline{y:D})\{\  stmt \ \}       
    \\
  {\hprovesN{M} { \ \prg{this}:\prg{D}, \overline{y:D}\, \wedge\, A_1\  } {\ stmt\ } {\ A_2\, \sdN{\wedge\, \PushASLong {\prg{res}} {A_2}} \ }   {A_3} } 
}
{
M \vdash {\mprepostN {A_1}{p\ D}{m}{y}{D}{A_2} {A_3} }
}
\\
\inferruleSD 
{~ \strut \hspace{6.5cm} {\sc{invariant}}}
{
\begin{array}{l}
\forall  D,  m:\ \ \  \ \  \prg{mBody}(m,D,M)=\prg{public} \ (\overline{y:D})\{\  stmt \ \}      \ \ \Longrightarrow  
  \\
   ~ \strut \hspace{0.3cm}  \ \ \ 
 \begin{array}{l}
   \hprovesN {M}  
{ \ \prg{this}:\prg{D}, \overline{y:D},\,   \overline{x:C} \, \wedge\,  A\ }   
  	{\ stmt\ }   
	 {\  A\, \wedge\, \PushASLong {\prg{res}} {A} \ }  
{\ A \ }
 \end{array}
 \end{array}
}
{
M \vdash \TwoStatesN{ \overline{x:C}} {A}
}
\end{array}
$
\caption{Methods' and Modules' Adherence to Specification}
\label{f:wf}
\end{figure}

\footnoteSD{Julian's reflections on the invariant and method call rules:

 I think it somewhat makes sense for the Invariant rule to be asymmetric since
calls to internal code are asymmetric from the perspective of protection. There is no change in specific protection when entering 
internal code, but there may be when returning to external code from internal code. 

I'd have to think about it more, but I don't believe there is any decrease in general protection when entering internal code,
but there may be when exiting internal code. The adaptation in the post-condition is there to address the potential
for protection to be lost on return.

Magic wand adaptation (i.e. A -* x) expresses what must hold before crossing the boundary from internal to external code
(i.e. before an internal method return or call to an external method) for the adapted assertion to still be satisfied after crossing 
that boundary. I think there are a few things that are useful to consider here:
it is good to note that if you remove protection from the assertion language, the magic wand would be the identity
(i.e. A -* x = A) since only protection is modified when passing values between objects via method calls
more importantly, the way we define protection is based around fields and external/internal objects, and passing a value
to an internal object (via a method call) modifies neither a field nor grants any external object access to any other object
According to my understanding of the intended semantics of the magic wand, entering internal code from frame should
not affect any sort of protection.
My impression is that part of the intent of the magic wand is to be able to express "what needs to hold before a call
so that <x>  holds within the new frame (whether that frame is internal or external)?". I still need to think about how that
fits in with my above description of magic wand adaptation.
}


 {{\sc{Method}} says that  a module satisfies a method specification if the method body satisfies the corresponding pre-, post- and \midcond.
 }
 {{\sc{Invariant}} says that  a module satisfies an invariant specification $\TwoStatesN{ \overline{x:C}} {A}$,  if the method body of each public method
 has $A$ as its  pre-, post- and \midcond. Moreover, since the method is public, \prg{this} and the parameters are not protected upon method entry ($\overline{ \neg\inside{y}}\, \wedge\, \overline{x:C}$)),
 and the result is not protected upon method exit ($ \neg\inside{res}$)
}
 
 \vspace{.1cm}

\noindent
\textbf{Barendregt} In  {\sc{method}} we \sdN{implicitly} require   the free variables  in a method's precondition  not to overlap with variables in its body, unless they are the receiver or one of the parameters ($\sdN{\vs(stmt) \cap \fv(A_1) \subseteq   \{ \prg{this}, y_1, ... y_n \} }$).  And in {\sc{invariant}} we require   the free variables in $A$ (which are a subset of  $\overline x$) not to overlap with the variable  in $stmt$ ($ \sdN{ \vs(stmt)\,  \cap\, \overline x\, =\, \emptyset}$).
This can easily be achieved through renamings, \cf Def. \ref{d:promises}.





 
