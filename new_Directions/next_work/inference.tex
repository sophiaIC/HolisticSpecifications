We will now develop an inference system to prove that a module % is well-formed $\vdash M$, \ie that a module 
satisfies its specification. This is done in three phases.
%\TODO{\red{TODO: refer to what we said in approach section}}

In the first phase we develop a logic of triples ${\hproves{M}  {A} {\ s\ }{A'} }$, which have the expected meaning, \ie 
(*) execution of statement $s$ in a state satisfying {the \emph{precondition}}  $A$ will lead to a state satisfying  {the \emph{postcondition}}  $A'$.
These tuples only apply to statements $s$ that  do not contain method calls  (even internal calls) -- this is so, because method calls may contain calls to external methods, and therefore can only be described through quadruples.
Our triples extend an underlying Hoare logic  (${M \vdash_{ul}  \{A\} {\ s\ } \{A'\} }$) and  introduce new judgements  which talk about protection.

In the second phase we develop a logic of quadruples ${\hprovesN{M}  {A} {\ s\ }{A'} {A''}}$. These promise, as usual, that (*) 
In addition, they promise that (**) any intermediate external states reachable during execution of that statement will satisfy the invariant  $A'''$.  
{We call $A''$ the \midcond.}
 We incorporate all triples from the first phase,       
introducing invariants, give the usual substructural rules, and deal with method calls. 
For method calls we use the methods' specs, and in addition, for public methods we use the fact that they preserve the module's invariants. 
For call to external  methods, we can only use the fact that they preserve the module's invariants. 
 
In the third phase, we prove adherence to each of the specifications. 
For method specifications we prove that the body maps the precondition to the postcondition and preserves the method's invariant. 
For module invariants we prove that they  are preserved by all public methods of the module.

\vspace{.1cm}
Before describing these three phases, we introduce some preliminaries.


\subsection{Preliminaries: Specification Lookup and Renamings}

The judgment    $\promises M S$ expresses that $S$ is guaranteed by the module. 
In particular, it allows   \emph{safe  renamings}. 
These renamings are   a convenience, akin to the Barendregt convention, and  allow simpler Hoare rules for  method calls -- \cf Sect. \ref{s:calls}.
\Cf Definition \ref{d:promises}.

\subsection{First Phase: Triples}


\begin{figure}[tht]
$
\begin{array}{c}
% \begin{array}{lcl}
\inferrule[\sc{extend}]
	{ M \vdash_{ul} \{ \ A\ \} {\ stmt\ }\{\ A'\ \} \\
	 stmt  \ \mbox{contains no method call} }
	{\hproves{M}  {A} {\ stmt\ }{A'} } 
% MOVED
%	& &
%\inferrule[\sc{types-1}]
%	{  stmt \ \mbox{contains no method call} \\
%	stmt  \ \mbox{contains   no assignment to $x$}}
%	{\hproves{M}  {x:C} {\ stmt\ }{x:C} }
% \end{array}
% \end{array}
\end{array}
 $
\caption{Embedding underlying Hoare Loagic}
\label{f:underly}
\end{figure}
 
\subsubsection{Embedding an underlying Hoare Logic}

\begin{axiom}
\label{ax:ul}
{Assume   Hoare logic with judgements % of the form 
\ $M \vdash_{ul} \{ A \} s \{ A' \}$, \ 
so that $\Stable{A}$ and $\Stable{A'}$. }
\end{axiom}


In  Fig. \ref{f:underly} and \ref{f:protection} we introduce our triples, of the form ${   \hproves{M}  {A} s {A'}}$. 
These promise, as expected, that any execution of $stmt$ in a state that satisfies $A$ leads to a state that satisfies $A'$.
These tuples only apply to statements that do not contain method calls.

With rule {{\sc{extend}} in Fig. \ref{f:underly},  any assertion $M \vdash_{ul} \{ A \} s \{ A' \}$  whose statement does not contain a method call, and which 
can be proven in the underlying Hoare logic, to also be proven in our logic. More in \ref{s:types}.
 

\begin{figure}[tht]
$
\begin{array}{c}
	{{\inferruleSD{\hspace{2.3cm} [\sc{prot-1}]}
	{   stmt \ \mbox{contains no method call, and no assignment to $z$}
	\\
	{\hproves{M}  {\re = z} {\ stmt\ }{ \re=z} }
	}
	{\hproves{M} 
						{\  \inside{\re}  \ }
						{\  stmt \ }
						{\  \inside{\re}\ }
	}
}}
\\
% this onw ia wronf!!!
%\\
%	{{\inferruleSD{\hspace{2.3cm} [\sc{prot-2}]}
%	{   stmt \ \mbox{contains no method call, and no assignment to $z$ or $z'$}
%	\\
%	{\hproves{M}  {\re = z \wedge \re' = z'} {\ stmt\ }{\re = z \wedge \re' = z'} }
%	}
%	{\hproves{M} 
%						{\  \protectedFrom{\re}{\re'}  \ }
%						{\  stmt \ }
%						{\  \ \protectedFrom{\re}{\re'}\ }
%	}
%	}}
\\
%\\
%\mbox{\sdN{We had this rule, but I believe  that it is not sound when $y$=$z$, or $x$=$y$}}
%\\
%      {\inferruleSD{\hspace{5.3cm} [\sc{prot-????}]}
%	{ }
%	{\hproves{M} 
%						{\ \protectedFrom{x}{z} \ }
%						{\ y =y'.f\ }
%						{\ \protectedFrom{x}{z}\ }
%	}
%}
%
%\\
%\\
%\sdN{\mbox{Is his rule is subsumed by {\sc{Prot-1}}}}
%\\
%       {\inferruleSD{\hspace{5.3cm} [\sc{prot-???}]}
%	{ }
%	{\hproves{M} 
%						{\  \inside{x}\ \wedge \   \protectedFrom{x}{y'} \ }
%						{\ y.f=y'\ }
%						{\  \inside{x}\ }
%	}
%}	
%\\
%\\
\\
\\
        {\inferruleSD{\hspace{5.3cm} [\sc{prot-4}]}
	{ }
	{\hproves{M} 
						{\ \protectedFrom{x}{z} \ \wedge \   \protectedFrom{x}{y'} }
						{\ y.f=y'\ }
						{\ \protectedFrom{x}{z} \ }
	}
}	 
\end{array}
 $
\caption{Triples -- protection}
\label{f:protection}
\end{figure}







 

 \subsubsection{Reasoning about protection}

\red{ NEED CHECKING:
In Fig. \ref{f:protection} we  give rules about the preservation of protection.
In \textsc{Prot-1}, if $y$ is internal, then no new external object becomes locally reachable, and therefore $ \inside{x}$ is preserved;
similarly, if the rhs of the assignment ($y'.f$) is not an alias of $x$, then no new unprotected route to $x$ is introduced, and therefore  $ \inside{x}$ is preserved.
In \textsc{Prot-2}, the assignment $y=y'.f$ does not create a new route from $z$ to $x$, and therefore  $\protectedFrom{x}{z}$ is preserved.
In \textsc{Prot-3}, if $x$ is protected from $y'$, then the assignment $y.f=y'$ does not create a new unprotected route   to $x$, and therefore  $\inside {x}$ is preserved.
Similarly, in \textsc{Prot-4}, if $x$ is protected from $y'$, then the assignment $y.f=y'$ does not create a new unprotected route  to $x$, and therefore   $\protectedFrom{x}{z}$ is preserved.
}

Moreover, "protection" of an object can decrease is if we call an eternal method, and pass it an internal object as argument. This is then covered by the rule in Fig. \ref{f:external:calls}.

\begin{lemma}
\label{l:no:meth:calls}
If ${\hproves{M}  {A} {\ s\ }{A'} }$, then $s$ contains no method calls.
\end{lemma}
  


\begin{figure}[htb]
$
\begin{array}{c}
% \begin{array}{lcl}
\inferrule[\sc{Mid}]
	{\hproves{M}  {A} {\ s\ }{A'} }
	{\hprovesN{M}  {A} {\ s\ }{A'} {A''} }
\\
\\
% MOVED
% THIS RULE IS NOT SOUND -- we need to adapt it
%{
%\inferrule[\sc{types-2}]
%	{ \hprovesN{M}  {A} {\ s\ }  {A'} {A''}  }
%	{\hprovesN{M}  {x:C \wedge A} {\ s\ }  {x:C\wedge A'} {A''}}
%}
% \end{array}
% \\
% \\
 \\
 \\
\inferrule[\sc{combine}]
	{  \hprovesN{M}  {A_1} {\ s\ } {A_2}  {A} \hspace{1.4cm}  \hprovesN{M}  {A_3} {\ s\ } {A_4} {A}}
	{ \hprovesN{M}  {A_1 \wedge A_3 }{\ s\ } {A_2 \wedge A_4} {A} }
\\
\\
\inferrule[\sc{sequ}]
	{  \hprovesN{M}  {A_1} {\ s_1\ } {A_2}  {A}  \hspace{1.4cm} \hproves{M}  {A_2} {\ s_2\ } {A_3} {A}}
	{   \hprovesN{M}  {A_1   }{\ s_1; \, s_2\ } {  A_3} {A} }
\\ \\
\inferrule[\sc{consequ}]
	{
%	\begin{array}{l}
	 { \hprovesN  {M}  {A_4} {\, s\, } { A_5} {A_6}  }
	 \\
	 M \vdash A_1 \rightarrow A_4 
	 \hspace{1.4cm} 
	{ M \vdash A_5   \rightarrow  A_2  }
	 \hspace{1.4cm}   
	{  M \vdash A_6 \rightarrow A_3 }
%	\end{array}
	}
	{   \hprovesN{M}  {A_1 }{\ s\ } {A_2} {A_3} }
  \end{array}
 $
\caption{Hoare Quadruples - incorporating Hoare triples and substructural rules }
\label{f:substructural}
\end{figure}

\subsection{Second Phase: Quadruples}

\subsubsection{Introducing mid-conditions, and substructural rules}
Fig. \ref{f:substructural} introduces quadruple rules of the from ${\hprovesN{M}  {A} {\ s\ }{A'} {A''} }$ which do not deal with method calls.

Rule {\sc{mid}} embeds  triples  ${\hproves{M}  {A} {\ s\ }{A'} }$  into quadruples ${\hprovesN{M}  {A} {\ s\ }{A'} {A''} }$; this is sound, because $s$ is guaranteed not to contain method calls (by lemma \ref{l:no:meth:calls})\footnote{The restriction to statements which do not contain method calls in rule {\sc{types-1}} was imposed so as to make this lemma valid}, and therefore its execution is guaranteed not to reach any external states.

 
The remaining rules in Fig. \ref{f:substructural} apply to \emph{any} statements -- even those containing method calls.
 Rule {\sc{types-2}} generalizes {\sc{types-1}} to any statement, provided that  there already exists a triple for that statement.
Rule {\sc{combine}} combines two quadruples for the same statement into one.
Rule   {\sc{sequ}}is the usual rule  for statement sequences  adapted to quadruples.
% SD: at the moment the below is not needed
%{{Rule   {\sc{consequ}} is more interesting, because it employs a new inference, {{$\inferd$}}, rather than  the usual inference, $\rightarrow$.
%For stable assertions, $\rightarrow$ and $\inferd$ behave the same way -- more  in section 
%} }



\subsubsection{Reasoning about   calls}
\label{s:calls}
is described in Fig. \ref{f:internal:calls}. {\sc{Call\_Int}}  and {\sc{Call\_Int\_Adapt}}  or internal methods, whether public or private;  \  {\sc{Call\_Ext}} 
and {\sc{Call\_Ext\_Adapt}} for  external methods.


% {We now move to the discussion of  external calls. For all external calls we require that the module's invariants hold before the external call.
%This is enforced through the requirement ${\external{y_0}}  \rightarrow \  \PushAS {y}{\extract{M}}$ in rules
%{\sc{CallAndAlias}} and {\sc{CallNonAlias}} in  Fig. \ref{f:internal:alias:calls}, and the precondition part  
% $\PushAS {y}{\extract{M}}$ in all rules in Fig.  \ref{f:external:calls}. }
%
%
%{Looking more closely, we notice  that we do not require that invariants hold (ie ${\extract{M}}$), but require that they will hold after the method call has been pushed on the stack, (ie $  \PushAS {y}{\extract{M}}$). We use the  assertion push-function $\pushSymbol$   defined earlier in Figure \ref{f:Push}.}
%Remember that the assertion $\PushAS y A$ is \emph{hypothetical}: if a state satisfies $\PushAS y A$, then after pushing
%onto that state a frame which contained the values  of $\overline y$, assertion $A$ will hold, and conversely,   if a state satisfies  $A$ with a top frame containing the  values of $\overline y, \overline z$, then  after popping that frame, the state  satisfies assertion  $\PushAS y A$ (cf. llemma \ref{lemma:push:ass:state}).
% 
% {In rule {\sc{ExtCall}} we   ensure that the module's invariants are preserved by the external call. } {Notice that while  {\sc{CallAndAlias}}, {\sc{CallNonAlias}}, {\sc{ExtCall\_WithSpec\_Weak}} and {\sc{ExtCall\_WithSpec\_Weak}}    include 
% $\PushAS y {A_1}$ in their precondition,  they do not include it in the postcondition. A variation where was  $\PushAS y {A_1}$ in the postcondition is admissible, as it would be the outcome of the combination of these rules with  {\sc{ExtCall}} through the application of {\sc{Consequ}}.}
% 
%{The precondition of {\sc{ExtCall\_WithSpec\_Weak}} is stronger than that of {\sc{ExtCall}}. It makes use of one of the  module's  promises: Here $  {\TwoStatesQ {\overline {x:C}} {A_1}{A_2}}$, i.e. that  $\overline {x:C} \ \wedge\ A_1$ can only lead to $A_2$. 
%% Moreover, the call's precondition requires  $\overline {x:C} \wedge \PushAS y {A_1}$. 
% Therefore, by lemma \ref{lemma:push:ass:state}.\ref{lemma:push:ass:state:one}, right after pushing a frame with $\overline y$, ie right after  entering the external call, $\overline {x:C}$ and  $A_1$ hold. `Therefore, by the semantics of $  {\TwoStatesQ {\overline {x:C}} {A_1}{A_2}}$, at the end of the external call,  $\overline {x:C} \wedge A_2$ will hold. This, together with  lemma \ref{lemma:push:ass:state}.\ref{lemma:push:ass:state:two} gives that after popping the frame, ie after  exiting the external call, $\PushAS y {A_1}$ will hold.}
%
%
%{The precondition of {\sc{ExtCall\_WithSpec\_Strong}} is stronger than that of {\sc{ExtCall\_WithSpec\_Weak}}: it also requires $A_1$. Therefore, the module's promise, $  {\TwoStatesQ {\overline {x:C}} {A_1}{A_2}}$, guarantees that after exiting the external call, $A_2$ will hold.}
 





\begin{figure}[htb]
$\begin{array}{c}
 \inferruleSD{\hspace{4.7cm} [\sc{Call\_Int}]}
	{
	   	\begin{array}{l}
		\promises  M {\mprepostN{A_1}{p\ C}{m}{y}{C}{A_2} {A_3}}  
          	\end{array}
		}
	{  \hprovesN {M} 
						{ \  y_0:C,\overline {y:C} \wedge  A_1[y_0/\prg{this}] \ }
						 { \ u:=y_0.m(y_1,.. y_n)\    }
					         { A_2[ u/res,y_0/\prg{this}] } 
						{   A_3  }	
}
 \\
\\
{ \inferruleSD{\hspace{4.7cm} [\sc{Call\_Int\_Adapt}]}
	{
	   	\begin{array}{l}
		\promises  M {\mprepostN{A_1}{p\ C}{m}{y}{C}{A_2} {A_3}}  
          	\end{array}
		}
	{  \hprovesN {M} 
						{ \  y_0:C, \overline {y:C} \wedge {\PushAS {y}{A_1[y_0/\prg{this}]}}  \ }  
						{ \ u:=y_0.m(y_1,.. y_n)\    }
						{  { \PushAS {y}{A_2[ u/res,y_0/\prg{this}]}}   }
						{   A_3  }	
}
}
\\
 \\ 
 \inferruleSD{\hspace{4.7cm} [\sc{Call\_Ext}]}
 	{ 
   	 \promises M   {\TwoStatesN {\overline {x:C}} {A}} 
        }
	{   \hprovesN{M} 
						{ \    { \external{y_0}} \,     \wedge \,  \overline{x:C}\  \wedge\ {{A}}\ }  
						{ \ u:=y_0.m(y_1,.. y_n)\    }
						{ \   {{A}}  \ }
						{\  A \   }	
}
\\
 \\ 
 \inferruleSD{\hspace{4.7cm} [\sc{Call\_Ext}\_Adapt]}
 	{ 
   	 \promises M   {\TwoStatesN {\overline {x:C}} {A}} 
        }
	{   \hprovesN{M} 
						{ \    { \external{y_0}} \,     \wedge \,  \overline{x:C}\  \wedge\ {\PushAS {y}{A}}\ }  
						{ \ u:=y_0.m(y_1,.. y_n)\    }
						{ \   {\PushAS {y}{A}}  \ }
						{\  A \   }	
}

\end{array}
$
\caption{Hoare Quadruples for Internal and External Calls}
\label{f:internal:calls}
\label{f:external:calls}
\label{f:calls}
\end{figure}
% }}

  {\sc{Call\_Int}} is as expected:  
We look up the method's specification,   require the precondition, and guarantee the postcondition and midcondiition after the relevant renamings.
%: $ {\mprepostN{A_1}{\prg{private} C}{m}{y}{C}{A_2} {A_3}}  $  promises that if $m$ starts in a state that satisfies $A_1$, then  its execution will lead to a state that satisfies $A_2$, and
% that during its execution, $A_3$  any reachable scoped external state will satisfy $A_3$. Notice that in the post condition ($A_2$) we replace the $result$ by the left hand side of the assignment ($u$).
For {\sc{Call\_Int\_Adapt}} we require a weaker pre-condition ($  \PushAS {y}{A_1}$  rather than $A_1$) and also ensure a weaker post-condition ($\PushAS {y}{A_2}$, rather than $A_2$).
Remember that $  \PushAS {y}{A_1}$ guarantees that $A_1$ held before popping the callee's frame (ie $  \PushAS {y}{A_1}$);
%Looking more closely, we notice  that {\sc{IntCall}} does not require the method's precondition ($A_1$) to hold before the call, but instead requires that it will hold after the method call has been pushed on the stack, (ie $  \PushAS {y}{A_1}$).
%Similarly, after the call it does not guarantee the method's post condition ($A_2$), but instead guarantees that it held before popping the calllee's frame (ie $  \PushAS {y}{A_1}$).
%Remember that the assertion $\PushAS y A$ is \emph{hypothetical}: if a state satisfies $\PushAS y A$, then after pushing
%onto that state a frame which contained the values  of $\overline y$, assertion $A$ will hold, and conversely,   if a state satisfies  $A$ with a top frame containing the  values of $\overline y, \overline z$, then  after popping that frame, the state  satisfies assertion  $\PushAS y A$ 
cf.  lemma \ref{lemma:push:ass:state}).
Moreover, the call has the same \midcond,   here $A_3$, as the specification of the method.
 {\sc{Call\_Int}} is applicable whether the method is public or private.
 \footnote{\se{Appendix~\ref{app:proof} shows stronger alternative rules.}}
\move{\footnote{
The rules {\sc{Call\_Int}} and {\sc{Call\_Int\_Adapt}}   are weaker forms of    rule {\sc{Call\_Int\_Combine}} given here. 
Even though  {\sc{Call\_Int\_Combine}} is sound, we did not include in the presented system, for the sake of simplicity and separation of concerns.
A similar stronger rule can be expressed for {\sc{Call\_Ext}} and {\sc{Call\_Ext\_Adapt}}.
 \\
$
{\begin{array}{c}
  \inferruleSD  {\hspace{4.7cm} [\sc{Call\_Int\_Combine}]}
 	{
 	  	\begin{array}{c}
 		  \promises  M { \mprepostN  {A_{1a} \wedge A_{1r} }  {p\ C} {m} {y} {C} { A_{2a} \wedge A_{2r}}   {A_3}  }
		\\
		A_1'\txteq    A_{1a}[y_0/\prg{this}]  \wedge {\PushAS {y}{A_{1r}[y_0/\prg{this}]}}
		\\
		A_2'\txteq    A_{2a}[u/res,y_0/\prg{this}]  \wedge {\PushAS {y}{A_{2r}[u/res,y_0/\prg{this}]}}\
       	\end{array}
 		}
 	{  \hprovesN {M} 
						{ \  y_0:C, {\overline {y:C}} \wedge A_1' \ }  
						{ \ u:=y_0.m(y_1,.. y_n)\    }
						{ \ A_2'\ }
						{  \  A_3 \ }	
      }
\end{array}
}
$
}}

%%%%%%%%%%%%%
{For external methods,  {\sc{Call\_Ext}}, we consider the module's invariants. If the module promises to preserve $A$, \ie if  $\promises M   {\TwoStatesN {\overline {x:D}} {A}}$, and   $A$  holds before the call, then it also holds after and during the call. }
In  {\sc{Call\_Ext\_adapt}}, we require a weaker version,  \ie that   $  \PushAS {y}{A}$ holds before the call. Then the 
weaker version also holds after the call. Moreover, $A$ is also a \midcond of the call.
}


\vspace{.1cm}
 {{
\noindent
\textbf{Implicit renamings} In all rules above  the actual and formal parameter names were identical. We achieved this through the implicit renamings from Def. \ref{d:promises}.  
 }}
 
\subsubsection{Aliasing and Calls}

\TODO{TODO: revisit that section}

%{Rules  {\sc{CallAndAlias}}  and  {\sc{CallNonAlias}} say that calls preserve aliasing, resp. non-aliasing, between variables, ie they preserve $x=x$ resp. $x\neq x$. These two rules apply to internal as well as external calls. When the callee's receiver is external, they make the extra requirement that  $\PushAS {y}{\extract{M}}$ -- we  will discuss this requirement together with the discussion of Fig. \ref{f:external:calls}.     Note that $x=x'$ expresses that $x$ and $x'$ are aliases, while  $u\txteq x$ expresses that $u$ and $x$ are textually the same --
%the latter is stronger, i.e.   $x\txteq u$ implies $x=u$. 
%% It is possible that variables are aliases, without being textually the same, i.e. it is possible to have $x=x'$ while $u\not\equiv x'$. 
% As $...\equiv ...$ is a textual assertion, and thus  state-independent,  it is a side-condition of the rules and is  not part of the Hoare triple's precondition.
%}
%
%
%\small{
%\begin{figure}[hbt]
%$\begin{array}{c}
%\inferruleSD{\hspace{4.7cm}  [{\sc{CallAndAlias}}}
%	{ 
%	{   x \txtneq u\txtneq x'  }
%	}
%	{   \hproves{M}  { \ x=x'\   \wedge \ ({\external{y_0}}  \rightarrow \  \PushAS {y}{\extract{M}})\  }	{ \ u:=y_0.m(y_1,.. y_n)\  } { \  x=x'\ }	 }
%
%\\ \\ 
%\inferruleSD{\hspace{4.7cm} [{\sc{CallNonAlias}}]}	
%{ 
%		{ \ x \txtneq u \txtneq x'\   }
%	}
%	{   \hproves{M}   { \ x\neq x'\   \wedge \ ({\external{y_0}}  \rightarrow \  \PushAS {y}{\extract{M}})\  } { \ u:=y_0.m(y_1,.. y_n))\  } { \  x\neq x'\ }	 }
%\\
%\\
%\end{array}
%$
%\caption{Logic for Aliasing around Calls }
%\label{f:internal:alias:calls}
%\end{figure}
%}




\subsection{Third phase: Proving adherence to Module Specifications}
\label{sect:wf}

In Fig. \ref{f:wf} we  define the judgment $\vdash M$, which says that % module $M$ 
$M$ has been proven to be well formed. 
%has been proven to satisfy its specification.



\begin{figure}[thb]
$
\begin{array}{l}
\begin{array}{lcl}
\inferruleSDNarrow 
{~ \strut  {\sc{WellFrm\_Mod}}}
{  M \vdash \SpecOf {M}
}
{
\vdash M  
}
& \hspace{0.7cm} &
\inferruleSDNarrow 
{~ \strut   {\sc{Comb\_Spec}}}
{  
M \vdash S_1 \hspace{1cm}  M \vdash S_2
}
{
M \vdash S_1 \wedge S_2
}
\end{array}
\\
\\
\inferruleSD 
{~ \strut \hspace{6.5cm} {\sc{method}}}
{  
 \prg{mBody}(m,D,M)=p \ (\overline{y:D})\{\  stmt \ \}  %\ \ \vee \ \   \prg{mBody}(m,D,M)=\prg{public} \ \overline{x:C}\{\  stmt \ \}       
    \\
  {\hprovesN{M} { \ \prg{this}:\prg{D}, \overline{y:D}\, \wedge\, A_1\  } {\ stmt\ } {\ A_2 \ }   {A_3} } 
}
{
M \vdash {\mprepostN {A_1}{p\ D}{m}{y}{D}{A_2} {A_3} }
}
\\
\\
\inferruleSD 
{~ \strut \hspace{6.5cm} {\sc{invariant}}}
{  
\begin{array}{l}
\forall  D,  m:\ \ \  \ \  \prg{mBody}(m,D,M)=\prg{public} \ (\overline{y:C})\{\  stmt \ \}      \ \ \Longrightarrow  
    \\
   ~ \strut \hspace{0.7cm}  \ \ \ \  \  %SD chop: do not know whay I had added \ A_{inv} \triangleq \overline{x:C}\, \wedge\, A
%  \\
%  ~ \strut \hspace{1.7cm} \ \ \ \ \  \ \ \ 
 {\hprovesN{M} { \ \prg{this}:\prg{D}, \overline{y:D}\, \wedge\, \neg\inside{this}\wedge\overline{ \neg\inside{y}}\, \wedge\, \overline{x:C}\, \wedge\,  A\  } {\ stmt\ } {\ A\ } {\ A \ }  }\ \ \ \ \  \ \  
\end{array}
}
{
M \vdash \TwoStatesN{ \overline{x:C}} {A}
}
\end{array}
$
\caption{Methods' and Modules' Adherence to Specification}
\label{f:wf}
\end{figure}

%$\begin{array}{lclcl}
%PRE(A_1 \wedge A_2} \trangleq PRE(A_1) \wedge PRE(A_2} & PRE{\protectedFrom \re \re
%
%{{\sc{WellFrm\_Mod}} and 
% {\sc{Comb\_Spec}} together say that a module is well formed if it satisfies each conjunct of its specification.
% }
% 
 {{\sc{Method}} says that  a module satisfies a method specification if the method body satisfies the corresponding pre-, post- and \midcond.
 }
 {{\sc{Invariant}} says that  a module satisfies an invariant specification $\TwoStatesN{ \overline{x:C}} {A}$,  if the method body of each public method
 has $A$ as its  pre-, post- and \midcond.}
 
 \vspace{.1cm}

\noindent
\textbf{Implicit renamings} We follow a form of Barendregt convention: In  {\sc{Method}}  we assume, wlog, that no free variable  in $A_1$   clashes with a variable  in $stmt$.
In {\sc{Invariant}} we assume, wlog, that none of
 $\overline x$ clash with  $\overline y$ or a variable  in $stmt$. 
% : Any of $\overline x$  which clashes with $\ovelrine y$ or a variable  in $stmt$,   gets implicitly renamed  in the proof that $ {\hprovesN{M} { \ \prg{this}:\prg{D}\, \wedge\, \overline{y:D}\,  \overline{x:C}\, \wedge\,  \wedge\,  A\  } {\ stmt\ } {\ A\ } {\ A \ }  } $


 
\footnoteSD{{QUESTIONS: }

\begin{enumerate}
\item 
Can we express what Fig \ref{f:calls} does, but more streamlined? -- SD has no ideas here  :-(
\item
In the rule for $\models M$,   I wonder whether we should be using ${\extract{M}}$ rather than $\PushAS {y} {\extract{M}}$. -- SD thinks this is only a matter of more thinking
\item
Do we want to forbid private methods to make external calls -- see Sect \ref{s:privateMs}? If we do that, then we need to extend the oper semantics and the inference rules in trivial, and boring ways...
\item
In  Fig \ref{f:calls} I wonder whether there are also stronger versions, where we require ${\extract{M}}$ rather than $\PushAS {y} {\extract{M}}$, pretty much as in the difference between {\sc{ExtCall\_WithSpec\_Weak}}  and the other rule, {\sc{ExtCall\_WithSpec\_Strong}}. But this would blow up the rules even more! 
\item
 Does the consequence rule require that the assertions are encapsulated? And if an assertion is encapsulated, is its consequence also encapsulated? TODO - EASY
\item
All rules require that the variables in the quantifiers do not appear in the meathod bodies, and are disjoint from the parameters.
TODO explain -EASY. 
\item  We need to add some dynamic type checking to the language, ie the public method call crashes if the actual params do not fit the formal types. Thisi is only for convenience; we cloud type them all as \prg{Object}, and crash by hand. TODO - EASY
\item
Do we need "modifies" or "does not modify" clauses too? Or can we assume that the underlying HL does this implicitly? For example, does\\
$\strut \ \ \  \hproves{M}  {a:Account \wedge p:Password \wedge a.passwd=p} {stmts}  {a.passwd=p}$\\ (with $a$ and $p$  free in $stmts$),  implicitly promise that no account's password has been modified? Shall we require the underlying logic to support such judgments -- say that if it has modifies clauses, or SL, it could do that?\\
Such an approach might solve the "late binding" issues mentioned below.
\end{enumerate}
 }




 
