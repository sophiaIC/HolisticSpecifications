Using our Hoare logic, we have developed a mechanised proof in Coq, that, indeed, $M_{good} \vdash S_2 \wedge S_3$.
%We  expand our  semantics and logic to deal with scalars and conditionals, % and  rewrite the example in the syntax of \Loo.
%and then highlight the most interesting proof steps of that proof.
This proof is \sdnew{part of the accompanying artifact.}
% the current submission (in a \prg{*.zip} file), and will be submitted as an artifact with the final version.

%The  mechanised proof models
Our proof models  \LangOO, the assertion language, the specification language, and the Hoare logic from \S \ref{s:hoare:first},  \S   \ref{s:hoare:second},  \S  \ref{sect:wf},  \S \ref{app:hoare} and Def. \ref{def:push}.
In keeping with   the start of  \S \ref{sect:proofSystem}, our proof assumes the existence of an underlying Hoare logic,  
and several, standard, properties of that underlying logic, the assertions logic (\eg equality of objects implies equality of field accesses) and of type systems
(\eg  fields of objects of different types cannot be aliases of one another).
All assumptions  are clearly indicated in the associated artifact.
%
%It is primarily concerned with demonstrating the application of our approach to our running example.
% and uses
% but does use 
%
%In particular, as in we assume the existence of an underlying Hoare logic,  (in the same manner as \S \ref{sect:proofSystem}), 
%and make several assumptions of properties that underlying logic, all of which are standard in Hoare logics. 
%We further assume several unsurprising rules of consequence for the underlying assertion language such as 
%transitivity of consequence, the law of the excluded middle, or equality of field accesses from equality of the object.
%
Appendix \ref{s:app:example} from \cite{externalCallsFull} %included in the auxiliary material, 
outlines % the main ingredients of
 that proof. 

% Finally, we discuss why  $M_{good} \not\vdash S_2 $
