\newcommand{\STwo}{\ensuremath{S_2}}
\newcommand{\STwoStrong}{\ensuremath{S_{2,strong}}}
\newcommand{\SPT}{~ \strut \hspace{.9cm}}
%\newcommand{\Aparams}{\prg{A}_{prms}}
\newcommand{\Alocals}{\prg{A}_{buy}}
\newcommand{\Alocalsb}{\prg{A}_{1}}
\newcommand{\Ids}{\prg{Ids}_{buy}}
%\newcommand{\Aparamstr}{\prg{A}_{prmsT}}
\newcommand{\Alocalstr}{\prg{A}_{trns}}
\newcommand{\Idstr}{\prg{Ids}_{trns}}
%\newcommand{\Aparamsset}{\prg{A}_{prmsS}}
\newcommand{\Alocalsset}{\prg{A}_{set}}
\newcommand{\Alocalssets}{\prg{A}_{2}}
\newcommand{\Idsset}{\prg{Ids}_{set}}
\newcommand{\stmtsP}{\prg{stmts}_{10,11,12}}
\newcommand{\step}[1]{ \vspace{.1cm} \noindent {\textbf{#1}}}

\section{  {Proving Limited Effects for the Shop/Account Example}}

\label{s:app:example}

In Section \ref{s:outline} we introduced a \verb|Shop| that allows clients to make purchases through the
\verb|buy| method.
The body if this method  includes a method call to an unknown external object (\verb|buyer.pay(...)|).

In this section  we use our Hoare logic from Section \ref{sect:proofSystem} to {outline the proof} that the \verb|buy| method
does not expose the \verb|Shop|'s  \verb|Account|, its \verb|Key|, or allow the \verb|Account|'s balance to be illicitly modified. 

%We need to extebd $\STwo$
% More generally,   given  the following scoped invariants, \\  
%$\strut  \SPSP  \STwo\ \  \triangleq \ \ \TwoStatesN  {\prg{a}:\prg{Account}}  {\inside{\prg{a.key}}} $ 
% \\
%$\strut  \SPSP  S_3\ \  \triangleq \ \ \TwoStatesN{ \prg{a}:\prg{Account},\prg{b}:\prg{int} } {\inside{\prg{a.key}} \wedge \prg{a.\balance} \geq \prg{b} } $ 
%\\
We {outline the proof} that $M_{good} \vdash \STwo$, and that $M_{fine} \vdash \STwo$.
{We  also show why $M_{bad} \not\vdash \STwo$.}

{We   rewrite the code of $M_{good}$ and so $M_{fine}$
so that it adheres to the syntax as defined in Fig. \ref{f:loo-syntax} (\S \ref{s:app:syntax:transform}). 
We  extend the specification $\STwo$, so that is also makes a specification for the private method \prg{set} (\S \ref{s:extend:spec}). 
After that, we outline the proofs  that $M_{good} \vdash \STwo$, and that $M_{fine} \vdash \STwo$ (in \S \ref{s:app:example:proofs}),
and that  $M_{good} \vdash \SThree$, and that $M_{fine} \vdash \SThree$ (\S \ref{s:app:example:proofs:S3}).
These proofs have been mechanized in Coq, and the source code will be
submitted as an artefact. %{xxxx}.
We also discuss why $M_{bad} \not\vdash \STwo$ (\S \ref{s:bad:not:S2}).}






\subsection{Expressing the \prg{Shop} example in the syntax from Fig. \ref{f:loo-syntax}}
\label{s:app:syntax:transform}

{
We now express our example in the syntax of Fig. \ref{f:loo-syntax}. 
For this, we  add a return type to each of the methods; 
We turn all local variables to parameter; We add an explicit assignment to the variable \prg{res}: and We   add a temporary variable \prg{tmp} to which we assign the result of our \prg{void} methods.
For simplicity, we allow %sequences of field reads, \eg \prg{this.accnt.blnce} rather than using   temporaty variables, as 
the shorthands \prg{+=} and \prg{-=}.
And we also allow definition of local variables, \eg  \prg{int price := ..} }

\begin{lstlisting}[mathescape=true, language=Chainmail, frame=lines]
module M$_{good}$
  ...   
  class Shop
    field accnt : Account, 
    field invntry : Inventory, 
    field clients: ..
  
    public method buy(buyer:external, anItem:Item, price: int, 
            myAccnt: Account, oldBalance: int, newBalance: int, tmp:int) : int
      price := anItem.price;
      myAccnt := this.accnt;
      oldBalance := myAccnt.blnce;
      tmp := buyer.pay(myAccnt, price)     // $\red{\mbox{external\ call}}!$
      newBalance := myAccnt.blnce;
      if (newBalance == oldBalance+price) then
          tmp := this.send(buyer,anItem)
      else
         tmp := buyer.tell("you have not paid me") ; 
      res := 0
     
      private method send(buyer:external, anItem:Item) : int
       ... 
  class Account
    field blnce : int 
    field key : Key
    
    public method transfer(dest:Account, key':Key, amt:nat) :int
      if (this.key==key') then
        this.blnce-=amt;
        dest.blnce+=amt
      else
        res := 0
      res := 0
	  
     public method set(key':Key) : int
      if (this.key==null)  then
      		this.key:=key'
      else 
        res := 0
      res := 0
\end{lstlisting}
% I removed class Key
% it does not have to belong to c=the module

\noindent
Remember that $M_{fine}$ is identical to $M_{good}$, except for the method \prg{set}. We describe the module below.

\begin{lstlisting}[mathescape=true, language=Chainmail, frame=lines]
module M$_{fine}$
  ...   
  class Shop
     ...  $ as\  in\  M_{good}$
  class Account
    field blnce : int 
    field key : Key
    
    public method transfer(dest:Account, key':Key, amt:nat) :int
       ...  $as\ in\ M_{good}$
	  
     public method set(key':Key, k'':Key) : int
      if (this.key==key')  then
      		this.key:=key''
      else 
        res := 0
      res := 0
\end{lstlisting}

\subsection{Proving that $M_{good}$ and $M_{fine}$ satisfy  $\STwo$}
% \subsubsection{Extending the specification $\STwo$}
\label{s:extend:spec}

We redefine $\STwo$ so that it also describes the behaviour of method \prg{send}. and have:
\\
{\sprepost
		{\strut \ \ \ \ \ \ \ \ \ S_{2a}} 
		{ \prg{a}:\prg{Account} \wedge  \prg{e}:\prg{external}  \wedge \protectedFrom{\prg{a.key}} {\prg{e} } } % \wedge \prg{a.blnce}=\prg{b} }
		 {\prg{private Shop}}
		 {\prg{send}}
		 {\prg{buyer}:\prg{external},\prg{anItem}:\prg{Item} }
		 { \protectedFrom{\prg{a.key}} {e} }
		 { \protectedFrom{\prg{a.key}} {e} }
}
\\
{\sprepost
		{\strut \ \ \ \ \ \ \ \ \ S_{2b}} 
		{ \prg{a}:\prg{Account} \wedge \prg{a.blnce} =\prg{b} }
		 {\prg{private Shop}}
		 {\prg{send}}
		 {\prg{buyer}:\prg{external}, \prg{anItem}:\prg{Item} }
		 { \prg{a.blnce} =\prg{b} }
		{   \prg{a.blnce} =\prg{b} }
}
\\
$\strut  \SPSP  \STwoStrong \ \  \triangleq \ \ \ \STwo \ \wedge \ S_{2a} \ \wedge \ S_{2b} $

%$\strut  \SPSP  \STwo\ \  \triangleq \ \ \TwoStatesN   \ \wedge\  
%{ \mprepostN {\prg{a:Account}\wedge \protectedFrom{\prg{a.key}} {\prg{buyer}}}
%                     {A_2} {A_3} }\ $ 



%\subsection{Demonstrating that $M_{good} \vdash \STwoStrong$, and that $M_{fine} \vdash \STwoStrong$}
 \label{s:app:example:proofs}
 

 
For brevity we only show that \verb|buy| satisfies our scoped invariants, as the all other methods of 
the \verb|M|$_{good}$ interface are relatively simple, and do not make any external calls. 
%Our approach follows the 3 phases outlined in Section \ref{sect:proofSystem}. That is, in
%phase 1 we use more an assumed underlying Hoare logic and more traditional Hoare triples to prove the adherence of internal code to
%the specification. In phase 2 we use Hoare quadruples to prove external calls adhere
%to the specification, and finally in phase 3 we use raise the results from phase 1 and 2 to proved
%the entire module satisfies the specification.

{ To write our proofs more succinctly, we will use \prg{ClassId}::\prg{methId}.\prg{body} as a shorthand for the method body of \prg{methId} defined in \prg{ClassId}.}

  

\begin{lemma}[$M_{good}$ satisfies $\STwoStrong$]
\label{lemma:exampleKeyProtect}
\label{l:Mgood:S2}
$M_{good} \vdash \STwoStrong$
\end{lemma}
\begin{proofO}
%We construct our proof tree using a top down approach.  
In order to prove that 
$$M_{good} \vdash \TwoStatesN {\prg{a}:\prg{Account}}  {\inside{\prg{\prg{a.key}}}}$$
we have to apply  \textsc{Invariant} from Fig. \ref{f:wf}.
 That is, for each  class $C$  defined in $M_{good}$, and for each public method $m$ in $C$, with parameters $\overline{y:D}$, we have to prove that
 \small
\begin{align*}
M_{good}\ \vdash \ \ &   \{ \ \prg{this}:\prg{C},\, \overline{y:D},\, \prg{a}:\prg{Account}\, \wedge\,
		             {\inside{\prg{a.key}}}\ \wedge\       \protectedFrom {\prg{a.key}} {(\prg{this},\overline y)} \  \} \\
		& \SPT  \prg{C}::\prg{m}.\prg{body}\  \\
		&
                   \{\ {\inside{\prg{a.key}}}\ \wedge\ \ \protectedFrom {\prg{a.key}} {\prg{res}}   \ \}\ ||\ \{\ {\inside{\prg{a.key} } } \ \} \\
\end{align*}


\normalsize
Thus, we need to prove  three Hoare quadruples: one for \prg{Shop::buy}, one for  \prg{Account::transfer}, and one for  \prg{Account::set}.  That is, we have to prove that
 \small
\begin{align*}
\text{(1?)}  \ \ \ \ M_{good}\ \vdash  \  \ 
		&	\{  \ \Alocals, \, \prg{a}:\prg{Account} \, \wedge\, {\inside{\prg{a.key}}} \, \wedge \, \protectedFrom {\prg{a.key}} {\Ids}  \  \} \\
		& \SPT \prg{Shop}::\prg{buy}.\prg{body}\ \\  
		& \{ {\inside{\prg{a.key}}}\ \wedge\ {\PushASLong {\prg{res}} {\inside{\prg{a.key}}}}  \} \ \ \  || \ \ \ 
		   \{ {\inside{\prg{a.key}}} \}
\\
\text{(2?)}  \ \ \ \ M_{good} \vdash \ 
		&	\{  \ \Alocalstr, \, \prg{a}:\prg{Account}\, \wedge\,  {\inside{\prg{a.key}}} \, \wedge \, \protectedFrom {\prg{a.key}} {\Idstr}  \  \} \\
		& \SPT \prg{Account}::\prg{transfer}.\prg{body}\ \\  
		& \{ {\inside{\prg{a.key}}}\ \wedge\ {\PushASLong {\prg{res}} {\inside{\prg{a.key}}}}  \} \ \ \  || \ \ \ 
		   \{ {\inside{\prg{a.key}}} \}
\\
\text{(3?)}  \ \ \ \ M_{good} \vdash \ 
		&	\{  \ \Alocalsset, \, \prg{a}:\prg{Account}\, \wedge\,  {\inside{\prg{a.key}}} \, \wedge \, \protectedFrom {\prg{a.key}} {\Idsset}  \  \} \\
		& \SPT \prg{Account}::\prg{set}.\prg{body}\ \\  
		& \{ {\inside{\prg{a.key}}}\ \wedge\ {\PushASLong {\prg{res}} {\inside{\prg{a.key}}}}  \} \ \ \  || \ \ \ 
		   \{ {\inside{\prg{a.key}}} \}
\end{align*}
\normalsize
where we are using ? to indicate that this needs to be proven, and 
where we are using the shorthands\\
\small
$
\begin{array}{c}
\begin{array}{lcl}
 \Alocals\ &  \triangleq   \  &   \prg{this}:\prg{Shop}, \prg{buyer} : \prg{external}, \prg{anItem} : \prg{Item},\, \prg{price} : \prg{int},  \\
&  & \prg{myAccnt} : \prg{Account},\, \prg{oldBalance}:  \prg{int},  \prg{newBalance}:  \prg{int},  \prg{tmp}:  \prg{int}.\\
  \Ids\ &   \triangleq   & \prg{this},  \prg{buyer}, \prg{anItem}, \prg{price}, \prg{myAccnt},\, \prg{oldBalance},\  \prg{newBalance},  \prg{tmp}.\\ 
\Alocalstr\  & \triangleq \  &   \prg{this}:\prg{Account}, \prg{dest} : \prg{Account}, \prg{key'} : \prg{Key},\prg{amt}:\prg{nat} \\
  \Idstr\  &  \triangleq  & \prg{this},\, \prg{dest} ,\, \prg{key'},\, \prg{amt} \\
  \Alocalsset\  & \triangleq \  &   \prg{this}:\prg{Account}, \, \prg{key'} : \prg{Key},\, \prg{key''} : \prg{Key}.\\
    \Idsset\  &  \triangleq  & \prg{this},\, \prg{key'} ,\, \prg{key''}. \\
\end{array}
\end{array}
$
\normalsize

We will also need to prove that \prg{Send} satisfies specifications $S_{2a}$ and $S_{2b}$. 


We outline the proof of (1?) in Lemma \ref{l:buy:sat},% in \S \ref{s:buy:sat},
and the proof of (2) in Lemma \ref{l:transfer:S2}.
We do not prove (3), but will prove that \prg{set} from $M_{fine}$
satisfies $S_2$; shown in  Lemma \ref{l:set:sat} -- ie for module $M_{fine}$.
%, and the proof for \prg{Account::set} from module $M_{fine}$  in Lemma \ref{l:set:sat} in \S{s:set:sat}.

\end{proofO}

We also want to prove that $M_{fine}$ satisfies the specification $\STwoStrong$.

\begin{lemma} [$M_{fine}$ satisfies $\STwoStrong$]
\label{l:Mfine:S2}

$M_{fine} \vdash \STwoStrong$
\end{lemma}
\begin{proofO}
%We construct our proof tree using a top down approach.  
The proof of
$$M_{fine} \vdash \TwoStatesN {\prg{a}:\prg{Account}}  {\inside{\prg{\prg{a.key}}}}$$
goes along similar lines to the proof of lemma \ref{l:Mgood:S2}.
Thus, we need to prove the following  three Hoare quadruples: 
 \small
\begin{align*}
\text{(4?)}  \ \ \ \ M_{fine}\ \vdash  \  \ 
		&	\{  \ \Alocals, \, \prg{a}:\prg{Account} \, \wedge\, {\inside{\prg{a.key}}} \, \wedge \, \protectedFrom {\prg{a.key}} {\Ids}  \  \} \\
		& \SPT \prg{Shop}::\prg{buy}.\prg{body}\ \\  
		& \{ {\inside{\prg{a.key}}}\ \wedge\ {\PushASLong {\prg{res}} {\inside{\prg{a.key}}}}  \} \ \ \  || \ \ \ 
		   \{ {\inside{\prg{a.key}}} \}
\\
\text{(5?)}  \ \ \ \ M_{fine} \vdash \ 
		&	\{  \ \Alocalstr, \, \prg{a}:\prg{Account}\, \wedge\,  {\inside{\prg{a.key}}} \, \wedge \, \protectedFrom {\prg{a.key}} {\Idstr}  \  \} \\
		& \SPT \prg{Account}::\prg{transfer}.\prg{body}\ \\  
		& \{ {\inside{\prg{a.key}}}\ \wedge\ {\PushASLong {\prg{res}} {\inside{\prg{a.key}}}}  \} \ \ \  || \ \ \ 
		   \{ {\inside{\prg{a.key}}} \}
\\
\text{(6?)}  \ \ \ \ M_{fine} \vdash \ 
		&	\{  \ \Alocalsset, \, \prg{a}:\prg{Account}\, \wedge\,  {\inside{\prg{a.key}}} \, \wedge \, \protectedFrom {\prg{a.key}} {\Idsset}  \  \} \\
		& \SPT \prg{Account}::\prg{set}.\prg{body}\ \\  
		& \{ {\inside{\prg{a.key}}}\ \wedge\ {\PushASLong {\prg{res}} {\inside{\prg{a.key}}}}  \} \ \ \  || \ \ \ 
		   \{ {\inside{\prg{a.key}}} \}
\end{align*}

 
\normalsize

The proof of (4?) is identical to that of (1?); the proof of (5?) is identical to that of (2?). 
We outline the proof (6?)    in Lemma \ref{l:set:sat} in \S \ref{s:set:sat}.

\end{proofO}



% \subsubsection{Proving that \prg{Shop::buy} from $M_{good}$ satisfies $\STwoStrong$}
\label{s:buy:sat}

\begin{lemma}[\prg{Shop::buy} satisfies $S_2$]
\label{l:buy:sat}
 
\begin{align*}
\text{(1)}  \ \ \ \ M_{good} \vdash 
		&	\{  \ \Alocals\,\prg{a}:\prg{Account}\ \wedge\, {\inside{\prg{a.key}}} \, \wedge \, \protectedFrom {\prg{a.key}} {\Ids}  \  \} \\
		& \SPT \prg{Shop}::\prg{buy}.\prg{body}\ \\  
		& \{ {\inside{\prg{a.key}}}\ \wedge\ {\PushASLong {\prg{res}} {\inside{\prg{a.key}}}}  \} \ \ \  || \ \ \ 
		   \{ {\inside{\prg{a.key}}} \}
\end{align*}

\end{lemma}

\begin{proofO}
We will use the shorthand $\Alocalsb \triangleq \Alocals, \,\prg{a}:\prg{Account}$.
We will split the proof into 1) proving that statements 10, 11, 12 preserve the protection of \prg{a.key} from the \prg{buyer}, 2) proving that the external call 

\step{1st Step: proving statements 10, 11, 12}

We apply the underlying Hoare logic and prove that the statements on lines 10, 11, 12 do not affect the value of \prg{a.key}, ie that for a $z\notin \{ \prg{price}, \prg{myAccnt}, \prg{oldBalance} \}$, we have 

\begin{align*}
\text{(10)}  \ \ \ \ {M_{good} \vdash_{ul}} 
		&	\{  \ \Alocalsb\  \wedge\ z=\prg{a.key} \} \\
		&   \SPT \prg{price:=anItem.price}; \\  
		&   \SPT \prg{myAccnt:=this.accnt}; \\  
                 &   \SPT \prg{oldBalance := myAccnt.blnce};\\
		& \{ z=\prg{a.key} \}
\end{align*}

We then apply {\sc{Embed\_UL}}, {\sc{Prot-1}} and {\sc{Prot-2}} and {\sc{Combine}} and and {\sc{Types-2}} on (10) and use the shorthand $\stmtsP$ for the statements on lines 10, 11 and 12, and obtain: 
\\
\begin{align*}
\text{(11)}  \ \ \ \ M_{good} \vdash 
		&	\{  \ \Alocalsb\  \wedge\ {\inside{\prg{a.key}}} \, \wedge\, \protectedFrom{\prg{buyer}} {\prg{a.key}}  \} \\
		& \SPT \stmtsP\ \\  
		& \{ \ {\inside{\prg{a.key}}}  \, \wedge\, \protectedFrom{\prg{buyer}} {\prg{a.key}}   \}
\end{align*}



We apply  {\sc{Mid}}  on (11) and obtain 
\begin{align*}
\text{(12)}  \ \ \ \ M_{good} \vdash 
		&	\{  \ \Alocalsb\, \wedge\, \protectedFrom {\prg{a.key}} {\prg{buyer}}\  \} \\
		& \SPT \stmtsP\ \\  
		& \{ \ \Alocalsb\, \wedge \  {\inside{\prg{a.key}}} \, \wedge\, \protectedFrom{\prg{buyer}} {\prg{a.key}}  \ \} \ \ || \\
		& \{ \ {\inside{\prg{a.key}}}\  \}
\end{align*}

\step{2nd Step: Proving the External Call}

 

We now need to prove that the external method call \prg{buyer.pay(this.accnt, price)} protects the \prg{key}. i.e.
%\small
\begin{align*}
\text{(13?)} \ \ \ M_{good} \vdash & \{ \ \Alocalsb \   \wedge\    {\inside{\prg{a.key}}},\, \wedge\, \protectedFrom {\prg{a.key}} {\prg{buyer}}  \} \\
		  		& \SPT  \prg{tmp := buyer.pay(myAccnt, price)}\ \\  
		  		& \{ \ \ \ \Alocalsb \ \wedge\ {\inside{\prg{a.key}}} \, \wedge\, \protectedFrom{\prg{buyer}} {\prg{a.key}} \} \ \ \  || \ \ \  \\
		  		&   \{ \   {\inside{\prg{a.key}}}\  \}
\end{align*}
\normalsize

We use that $M \vdash \TwoStatesN  {\prg{a}:\prg{Account}}  {\inside{\prg{a.key}}}$
 and  obtain
 \\
 \small
\begin{align*}
\text{(14)} \ \ \ M_{good} \vdash & \{ \ \prg{buyer}:\prg{external},\,  {\inside{\prg{a.key}}} \, \wedge\, 
\protectedFrom {\prg{a.key}} {(\prg{buyer},\prg{myAccnt},\prg{price})} \  \} \\
		  		& \SPT  \prg{tmp := buyer.pay(myAccnt, price)}\ \\  
		  		& \{ \ \inside{\prg{a.key}} \, \wedge\, 
\protectedFrom {\prg{a.key}} {(\prg{buyer},\prg{myAccnt},\prg{price})}\ \} \ \ \  || \ \ \  \\
		  		&   \{ \   {\inside{\prg{a.key}}}\  \}
\end{align*}
\normalsize 
 
 
 Moreover by the type declarations and the type rules, we obtain that all objects directly or indirectly accessible accessible from \prg{myAccnt} are internal or scalar.
 This, together with  \textsc{Prot-Intl}, gives that
\\
$
\begin{array}{llll}
& (15) & & M_{good} \vdash \Alocalsb   \longrightarrow \protectedFrom {\prg{a.key}} {\prg{myAccnt}} 
\end{array}
$
\\
Similarly, because \prg{price} is a \prg{nat}, and because of \textsc{Prot-Int}$_1$, we obtain 
\\
$
\begin{array}{llll}
& (16) & & M_{good} \vdash \Alocalsb \   \longrightarrow \protectedFrom {\prg{a.key}} {\prg{price}} 
\end{array}
$
 

We apply {\textsc{Consequ}} on (15), (16) and (14) and obtain (13)!

\normalsize


\end{proofO}

\begin{lemma} [ \prg{transfer} satisfies $S_2$]
\label{l:transfer:S2}
\small
\begin{align*}
\text{(2)}  \ \ \ \ M_{good} \vdash \ 
		&	\{  \ \Alocalstr, \, \prg{a}:\prg{Account}\, \wedge\,  {\inside{\prg{a.key}}} \, \wedge \, \protectedFrom {\prg{a.key}} {\Idstr}  \  \} \\
		& \SPT \prg{Account}::\prg{transfer}.\prg{body}\ \\  
		& \{ {\inside{\prg{a.key}}}\ \wedge\ {\PushASLong {\prg{res}} {\inside{\prg{a.key}}}}  \} \ \ \  || \ \ \ 
		   \{ {\inside{\prg{a.key}}} \}
\end{align*}
\normalsize

\end{lemma}

\begin{proofO}

To prove (2), we will need to prove that

\small
\begin{align*}
\text{(21?)}  \ \ \ \ M_{good} \vdash \ 
		&	\{  \ \Alocalstr, \, \prg{a}:\prg{Account}\, \wedge\,  {\inside{\prg{a.key}}} \, \wedge \, \protectedFrom {\prg{a.key}} {\Idstr}  \  \} \\
		&  \SPT   \prg{if (this.key==key') then }\\
		& \SPT \SPT   \SPT\SPT  \prg{this.\balance:=this.\balance-amt} \\
	        & \SPT \SPT   \SPT\SPT  \prg{dest.\balance:=dest.\balance+amt} \\
		& \SPT   \prg{else }\\
		& \SPT\SPT   \SPT\SPT  \prg{res:=0} \\
		& \SPT \prg{res:=0} \\
		& \{ {\inside{\prg{a.key}}}\ \wedge\ {\PushASLong {\prg{res}} {\inside{\prg{a.key}}}}  \} \ \ \  || \ \ \ 
		   \{ {\inside{\prg{a.key}}} \}
\end{align*}
\normalsize

Using the underlying Hoare logic we can prove that no account's \prg{key} gets modified, namely

\small
\begin{align*}
\text{(22)}  \ \ \ \ M_{good}\ \vdash_{ul} \ 
		&	\{  \ \Alocalstr, \, \prg{a}:\prg{Account}\, \wedge\,   {\inside{\prg{a.key}}} \\
		&  \SPT   \prg{if (this.key==key') then }\\
		& \SPT \SPT   \SPT\SPT  \prg{this.\balance:=this.\balance-amt} \\
	        & \SPT \SPT   \SPT\SPT  \prg{dest.\balance:=dest.\balance+amt} \\
		& \SPT   \prg{else }\\
		& \SPT\SPT   \SPT\SPT  \prg{res:=0} \\
		& \SPT \prg{res:=0} \\
		& \{    \inside{\prg{a.key}}   \} \ \ \ 
\end{align*}
\normalsize

Using (22) and {\sc{[Prot-1]}},  we obtain

\small
\begin{align*}
\text{(23)}  \ \ \ \ M_{good}\ \vdash  \ 
		&	\{  \ \Alocalstr, \, \prg{a}:\prg{Account}\, \wedge\,  z=\prg{a.key} \} \\
		&   \SPT   \prg{if (this.key==key') then }\\
		& \SPT \SPT   \SPT\SPT  \prg{this.\balance:=this.\balance-amt} \\
	        & \SPT \SPT   \SPT\SPT  \prg{dest.\balance:=dest.\balance+amt} \\
		& \SPT   \prg{else }\\
		& \SPT\SPT   \SPT\SPT  \prg{res:=0} \\
		& \SPT \prg{res:=0} \\
		& \{  z=\prg{a.key} \} \ \ \ 
\end{align*}
\normalsize


Using (23) and   {\sc{[Embed-UL]}}, we obtain 

\small
\begin{align*}
\text{(24)}  \ \ \ \ M_{good}\ \vdash \ 
		&	\{  \ \Alocalstr, \, \prg{a}:\prg{Account}\, \wedge\,  z=\prg{a.key} \} \\
		&  \SPT   \prg{if (this.key==key') then }\\
		& \SPT \SPT   \SPT\SPT  \prg{this.\balance:=this.\balance-amt} \\
	        & \SPT \SPT   \SPT\SPT  \prg{dest.\balance:=dest.\balance+amt} \\
		& \SPT   \prg{else }\\
		& \SPT\SPT   \SPT\SPT  \prg{res:=0} \\
		& \SPT \prg{res:=0} \\
		& \{  z=\prg{a.key} \}  \ \ \  || \ \ \  \{  z=\prg{a.key} \} 
\end{align*}
\normalsize

 {\sc{[Prot\_Int]}} and the fact that $z$ is an \prg{int}  gives us that ${\PushASLong {\prg{res}} {\inside{\prg{a.key}}}}$.
Using  {\sc{[Types]}},  and  {\sc{[Prot\_Int]}} and   {\sc{[Consequ]}}   on (24) we obtain (21?).



\end{proofO}

% \subsubsection{Proving method \prg{set} from $M_{fine}$}
\label{s:set:sat}
 

We want to prove that this public method satisfies the specification  $\STwoStrong$, namely

\begin{lemma}[$\prg{set}$ satisfies $\STwo$]
\label{l:set:sat}
\label{l:satisfies:Mfine:pec2}
 
\begin{align*}
\text{(6)}  \ \ \ \ M_{fine} \vdash 
		&	\{  \ \Alocalsset\ \wedge\  {\inside{\prg{a.key}}} \, \wedge \, \protectedFrom {\prg{a.key}} {\Idsset}  \  \} \\
%	Latex experiment; does not work 
%	&
%		\mbox{\begin{lstlisting}[mathescape=true, language=Chainmail, frame=lines]
%      if (this.key==key') then
%        this.blnce-=amt;
%        dest.blnce+=amt
%      else
%        res := 0
%      res := 0
%\end{lstlisting}
%}
%\\
		& \SPT   \prg{if (this.key==key') then }\\
		& \SPT \SPT   \SPT\SPT  \prg{this.key:=key''} \\
	        & \SPT   \prg{else }\\
		& \SPT\SPT   \SPT\SPT  \prg{res:=0} \\
		& \SPT \prg{res:=0} \\
& \{ {\inside{\prg{a.key}}}\ \wedge\ {\PushASLong {\prg{res}} {\inside{\prg{a.key}}}}  \} \ \ \  || \ \ \ 
	   \{ {\inside{\prg{a.key}}} \}
\end{align*}

\end{lemma}

\begin{proofO}
We will be  using the shorthand 
 $\SPT  \Alocalssets\ \triangleq \  \prg{a}:\prg{Account},\  \Alocalsset$.\\


To prove (6), we will use the  {\sc{Sequence}} rule, and we want to prove
\\
\begin{align*}
\text{(61?)}  \ \ \ \ M_{fine} \vdash 
		&	\{  \ \Alocalssets\ \wedge\  {\inside{\prg{a.key}}} \, \wedge \, \protectedFrom {\prg{a.key}} {\Idsset} \  \} \\
		& \SPT   \prg{if (this.key==key') then }\\
		& \SPT \SPT   \SPT\SPT  \prg{this.key:=key''} \\
	        & \SPT   \prg{else }\\
		& \SPT\SPT   \SPT\SPT  \prg{res:=0} \\
		& \{ \ \Alocalssets\,\wedge\ {\inside{\prg{a.key}}}\    \} \ \ \  || \ \ \ 
		   \{ {\inside{\prg{a.key}}} \}
\end{align*}
and that
\begin{align*}
\text{(62?)}  \ \ \ \ M_{fine} \vdash
          &  \{ \ \Alocalssets\,\wedge \inside{\prg{a.key}} \  \}  \\
		& \SPT\SPT   \SPT\SPT  \prg{res:=0} \\
		& \{ {\inside{\prg{a.key}}}\ \wedge\ {\PushASLong {\prg{res}} {\inside{\prg{a.key}}}}  \} \ \ \  || \ \ \ 
		   \{ {\inside{\prg{a.key}}} \}
\end{align*}

(62?) follows   from the types, and {\sc{Prot-Int}}$_1$, the fact that \prg{a.key} did not change, and  \sdN{ {\sc{Prot-1}}}.

\vspace{.5cm}
We now  want to  prove (61?). For this, will apply the {\sc{If-Rule}}. That is, we need to prove that

\begin{align*}
\text{(63?)}  \ \ \ \ M_{fine} \vdash 
		&	\{  \ \Alocalssets\,\wedge\, {\inside{\prg{a.key}}} \, \wedge \, \protectedFrom {\prg{a.key}} {\Idsset} \, \wedge  \,  \prg{this.key}=\prg{key'}\  \} \\
		& \SPT \SPT   \SPT\SPT  \prg{this.key:=key''} \\
		& \{ {\inside{\prg{a.key}}}  \} \ \ \  || \ \ \ 
		   \{ {\inside{\prg{a.key}}} \}
\end{align*}
 
and that
 
\begin{align*}
\text{(64?)}  \ \ \ \ M_{fine} \vdash 
		&	\{  \ \Alocalssets\,\wedge\, {\inside{\prg{a.key}}} \, \wedge \, \protectedFrom {\prg{a.key}} {\Idsset} \, \wedge  \,  \prg{this.key}\neq\prg{key'}\  \} \\
		& \SPT\SPT   \SPT\SPT  \prg{res:=0} \\
		& \{ {\inside{\prg{a.key}}}\   \} \ \ \  || \ \ \ 
		   \{ {\inside{\prg{a.key}}} \}
\end{align*}

(64?) follows easily from  the fact that \prg{a.key} did not change, and  {\sc{Prot-1}}.

\vspace{.5cm}
We look at the proof of (63?).  We will apply the {\sc{Cases}} rule, and distinguish on whether \prg{a.key}=\prg{this.key}. That is, we want to prove that\\
\small{
\begin{align*}
\text{(65?)}  \ \ \ \ M_{fine} \vdash 
		&	\{  \ \Alocalssets\,\wedge\, {\inside{\prg{a.key}}} \, \wedge \, \protectedFrom {\prg{a.key}} {\Idsset} \, \wedge  \,  \prg{this.key}=\prg{key'}\ \wedge\ \prg{this.key}=\prg{a.key}  \} \\
			& \SPT \SPT   \SPT\SPT  \prg{this.key:=key''} \\
	       	& \{ {\inside{\prg{a.key}}}\   \} \ \ \  || \ \ \ 
		   \{ {\inside{\prg{a.key}}} \}
\end{align*}
}
\\
and that
\\
\small{
\begin{align*}
\text{(66?)}  \ \ \ \ M_{fine} \vdash 
		&	\{  \ \Alocalssets\,\wedge\, {\inside{\prg{a.key}}} \, \wedge \, \protectedFrom {\prg{a.key}} {\Idsset} \, \wedge  \,   \,  \prg{this.key}=\prg{key'}\  \wedge \prg{this.key}\neq\prg{a.key'}\  \} \\
		& \SPT \SPT   \SPT\SPT  \prg{this.key:=key''} \\
		& \{ {\inside{\prg{a.key}}}\   \ \ \  || \ \ \ 
		   \{ {\inside{\prg{a.key}}} \}
\end{align*}
}
\vspace{.2cm}
\normalsize
We can prove (65?) through application of {\sc{Absurd}}, {\sc{ProtNeq}}, and {\sc{Consequ}}, as follows

\begin{align*}
\text{(61c)}  \ \ \ \ M_{fine} \vdash 
		&	\{  \ false  \} \\
		& \SPT \SPT   \SPT\SPT  \prg{this.key:=key''} \\
		& \{ {\inside{\prg{a.key}}}\   \} \ \ \  || \ \ \ 
		   \{ {\inside{\prg{a.key}}} \}
\end{align*}

By  {\sc{ProtNeq}}, we have $M_{fine} \vdash  \protectedFrom {\prg{a.key}} {\prg{key'}} \, \longrightarrow\, {\prg{a.key}}\neq {\prg{key'}}$, and therefore obtain

\begin{align*}
\text{(61d)}  \ \ \ \ M_{fine} \vdash  ... \wedge \, \protectedFrom {\prg{a.key}} {\Idsset} \, \wedge  \, \prg{this.key}=\prg{a.key}\, \wedge\,  \prg{this.key}=\prg{key'}\ \longrightarrow \ false 
\end{align*}

We apply  {\sc{Consequ}} on (61c) and (61d) and obtain (61aa?).

\vspace{.5cm}
We can prove (66?) by proving that \prg{this.key}$\neq$\prg{a.key} implies that $\prg{this}\neq \prg{a}$ (by the underlying Hoare logic), which again implies that the assignment \prg{this.key := ... } leaves the value of \prg{a.key} unmodified. We apply {\sc{Prot-1}}, and are done.

\end{proofO} 



\subsection{Showing that $M_{bad}$ does not satisfy $S_2$ nor $S_3$}

\subsubsection{$M_{bad}$ does not satisfy $S_2$}
$M_{bad}$ does not satisfy $S_2$. We can argue this semantically (as in \S \ref{s:bad:not:S2}), and also in terms of the proof system (as in \ref{s:bad:not:S2:proof}).


\subsubsection{$M_{bad}\nvDash S_2$}
\label{s:bad:not:S2}
 The reason is that $M_{bad}$ exports the public method \prg{set}, which updates the key without any checks. 
So, it could start in a state where the key of the account was protected, and then update it to something not protected.


In more detail: Take any state $\sigma$, where $M_{bad},\sigma \models a_0:\prg{Account}, k_0:\prg{Key} \wedge \inside{a_0.\prg{key}}$. 
Assume also that $M_{bad},\sigma \models \extThis$.  
Finally, assume that the continuation in $\sigma$ consists of $a_0.\prg{set}(k_0)$.
Then we obtain that $M_{bad}, \sigma \leadsto^* \sigma'$, where $\sigma'=\sigma[a_0.\prg{key}\mapsto k_0]$.
We also  have that $M_{bad},\sigma' \models \extThis$, and because $k_0$ is a local variable, we also have that $M_{bad},\sigma' \nvDash \inside{k_0}$.
Moreover, $M_{bad}, \sigma' \models a_0.\prg{krey}=k_0 $.
Therefore, $M_{bad},\sigma' \nvDash \inside{a_0.\prg{key}}$.

\subsubsection{$M_{bad}  \nvdash S_2$}
\label{s:bad:not:S2:proof}

In order to prove that $M_{bad}  \vdash S_2$, we would have needed to prove, among other things,  that \prg{set} satisfied $S_2$, which means proving that

\small{
\begin{align*}
\text{(ERR\_1?)}  \ \ \ \ M_{bad}\ \vdash \ 
		&	\{  \ \prg{this}:\prg{Account}, \prg{k'}:\prg{Key}, a:\prg{Account}\, \wedge\, \inside{a.\prg{key}}  \, \wedge \, \protectedFrom {\prg{a.key}} { \{\prg{this},\prg{k'}\} }\   \} \\
			& \SPT \SPT   \SPT\SPT  \prg{this.key:=k'}; \\
			& \SPT \SPT   \SPT\SPT \prg{res}:=0 \\ 
	       	& \{ \  \inside{a.\prg{key}}  \, \wedge \, \protectedFrom {a.\prg{key}} {\prg{res}} \   \} \ \ \  || \ \ \ 
		   \{ ... \}
\end{align*}
}

However, we cannot  establish $\text{(ERR\_1?)}$.
Namely, when we  take the case where $\prg{this}=a$,  we would need to establish, that

\small{
\begin{align*}
\text{(ERR\_2?)}  \ \ \ \ M_{bad}\ \vdash \ 
		&	\{  \ \prg{this}:\prg{Account}, \prg{k'}:\prg{Key}\, \wedge\, \inside{\prg{this}.\prg{key}}  \, \wedge \, \protectedFrom {\prg{this.key}} { \{\prg{this},\prg{k'}\} }\   \} \\
			& \SPT \SPT   \SPT\SPT  \prg{this.key:=k'}  \\
	       	& \{ \ \inside{\prg{this}.\prg{key}}  \   \} \ \ \  || \ \ \ 
		   \{ ... \}
\end{align*}
} 

And there is no way to prove $\text{(ERR\_2?)}$. In fact, $\text{(ERR\_2?)}$  is not sound, for the reasons outlined in \S \ref{s:bad:not:S2}.

\subsubsection{$M_{bad}$ does not satisfy $S_3$}

We have already argued in Examples \ref{e:versions} and \ref{e:public} that $M_{bad}$ does not satisfy $S_3$, by giving a semantic argument -- ie we are in state where $ \inside{a_0.\prg{key}}$, and execute $\prg{a}_0.\prg{set(k1)}; \prg{a}_0.\prg{transfer}(...\prg{k1})$. 
%We can also see that if a module satisfies $S_3$ is also satisfies $S_2$.

Moreover, if we attempted to prove that \prg{set} satisfies $S_3$, we would have to show that

\small{
\begin{align*}
\text{(ERR\_3?)}  \ \ \ \ M_{bad}\ \vdash \ 
		&	\{  \ \prg{this}:\prg{Account},\ \prg{k'}:\prg{Key}, a:\prg{Account},\, b:\prg{int}\ \wedge \\
  	 &	  \ \ \     \inside{a.\prg{key}}  \, \wedge \,    \protectedFrom {\prg{a.key}} { \{\prg{this},\prg{k'} \} }   
 	  \ \wedge\ a.\prg{\balance}\geq b \ \}  \\
 					& \SPT \SPT   \SPT\SPT  \prg{this.key:=k'}; \\ 
 			& \SPT \SPT   \SPT\SPT \prg{res}:=0 \\  
 	       	& \{ \  \inside{a.\prg{key}}  \, \wedge \, \protectedFrom {a.\prg{key}} {\prg{res}} \ \wedge\  a.\prg{\balance}\geq b \ \}  \ \ \  || \ \ \ 
		   \{ ... \}
\end{align*}
}

which, in the case of $a=\prg{this}$ would imply that

\small{
\begin{align*}
\text{(ERR\_4?)}  \ \ \ \ M_{bad}\ \vdash \ 
		&	\{  \ \prg{this}:\prg{Account},\ \prg{k'}:\prg{Key}, \, b:\prg{int}\ \wedge \\
  	 &	  \ \ \     \inside{\prg{this}.\prg{key}}  \, \wedge \,    \protectedFrom {\prg{\prg{this}.key}} { \{\prg{this},\prg{k'} \} }   
 	  \ \wedge\ \prg{this}.\prg{\balance}\geq b \ \}  \\
 					& \SPT \SPT   \SPT\SPT  \prg{this.key:=k'} \\
 	& \{ \  \inside{\prg{this}.\prg{key}}     \ \}  \ \ \  || \ \ \ 
		   \{ ... \}
\end{align*}
}

And  $\text{(ERR\_4?)}$ cannot be proven and does not hold.

\newcommand{\SThree}{S_3}
\newcommand{\SThreeStrong}{S_{3,strong}}

\subsection{Demonstrating that $M_{good} \vdash \SThree$, and that $M_{fine} \vdash \SThree$}
 \label{s:app:example:proofs}


 \subsection{Extending the specification $\SThree$}
\label{s:extend:spec:three}

As in \S \ref{s:extend:spec}, we redefine $\SThree$ so that it also describes the behaviour of method \prg{send}. and have:
\\
%{\sprepost
%		{\strut \ \ \ \ \ \ \ \ \ S_{3a}} 
%		{ \prg{a}:\prg{Account}, \prg{e}:\prg{external}, \prg{b}:\prg{int} \ \wedge \protectedFrom{\prg{a.key}} {\prg{e} }\ \wedge\ \prg{a}.\balance \geq\prg{b}\  } % \wedge \prg{a.blnce}=\prg{b} }
%		 {\prg{private Shop}}
%		 {\prg{send}}
%		 {\prg{buyer}:\prg{external},\prg{anItem}:\prg{Item} }
%		 { \protectedFrom{\prg{a.key}} {e} \wedge   \prg{a}.\balance \geq\prg{b}\ } 
%		 { \protectedFrom{\prg{a.key}} {e}\wedge   \prg{a}.\balance \geq\prg{b}\ } 
%}
%\\
%{\sprepost
%		{\strut \ \ \ \ \ \ \ \ \ S_{2b}} 
%		{ \prg{a}:\prg{Account} \wedge \prg{a.blnce} =\prg{b} }
%		 {\prg{private Shop}}
%		 {\prg{send}}
%		 {\prg{buyer}:\prg{external}, \prg{anItem}:\prg{Item} }
%		 { \prg{a.blnce} =\prg{b} }
%		{   \prg{a.blnce} =\prg{b} }
%}
\\
$\strut  \SPSP  \SThreeStrong \ \  \triangleq \ \ \ \SThree \ \wedge \ S_{2a} \ \wedge \ S_{2b} $

%$\strut  \SPSP  \STwo\ \  \triangleq \ \ \TwoStatesN   \ \wedge\  
%{ \mprepostN {\prg{a:Account}\wedge \protectedFrom{\prg{a.key}} {\prg{buyer}}}
%                     {A_2} {A_3} }\ $ 


  

\begin{lemma}[module $M_{good}$  satisfies $\SThreeStrong$]
\label{lemma:S3}
\label{l:Mgood:S3}
$M_{good} \vdash \SThreeStrong$
\end{lemma}
\begin{proofO}
%We construct our proof tree using a top down approach.  
In order to prove that 
$$M_{good} \vdash \TwoStatesN {\prg{a}:\prg{Account}, b:\prg{int} }{\  \inside{\prg{\prg{a.key}}} \wedge \prg{a.\balance}\geq b\ }$$
we have to apply  \textsc{Invariant} from Fig. \ref{f:wf}.
 That is, for each  class $C$  defined in $M_{good}$, and for each public method $m$ in $C$, with parameters $\overline{y:D}$, we have to prove that they satisfy the corresponding quadruples.
 

% \small
%\begin{align*}
%M_{good}\ \vdash \ \ &   \{ \ \prg{this}:\prg{C},\, \overline{y:D},\, \prg{a}:\prg{Account}\, \wedge\,
%		             {\inside{\prg{a.key}}}\ \wedge\       \protectedFrom {\prg{a.key}} {(\prg{this},\overline y)} \  \} \\
%		& \SPT  \prg{C}::\prg{m}.\prg{body}\  \\
%		&
%                   \{\ {\inside{\prg{a.key}}}\ \wedge\ \ \protectedFrom {\prg{a.key}} {\prg{res}}   \ \}\ ||\ \{\ {\inside{\prg{a.key} } } \ \} \\
%\end{align*}


\normalsize
Thus, we need to prove  three Hoare quadruples: one for \prg{Shop::buy}, one for  \prg{Account::transfer}, and one for  \prg{Account::set}.  That is, we have to prove that
 \small
\begin{align*}
\text{(31?)}  \ \ \ \ M_{good}\ \vdash  \  \ 
		&	\{  \ \Alocals, \, \prg{a}:\prg{Account}, b:\prg{int} \, \wedge\, {\inside{\prg{a.key}}} \, \wedge \, \protectedFrom {\prg{a.key}} {\Ids} \wedge \prg{a.\balance}\geq b \  \} \\
		& \SPT \prg{Shop}::\prg{buy}.\prg{body}\ \\  
		& \{ {\inside{\prg{a.key}}}\ \wedge\ {\PushASLong {\prg{res}} {\inside{\prg{a.key}}}}  \wedge \prg{a.\balance}\geq b \} \ \ \  || \ \ \ 
		   \{ {\inside{\prg{a.key}}}  \wedge \prg{a.\balance}\geq b \}
\\
\text{(32?)}  \ \ \ \ M_{good} \vdash \ 
		&	\{  \ \Alocalstr, \, \prg{a}:\prg{Account}\, , b:\prg{int} \, \wedge\,  {\inside{\prg{a.key}}} \, \wedge \, \protectedFrom {\prg{a.key}} {\Idstr}  \ \wedge \prg{a.\balance}\geq b \  \} \\
		& \SPT \prg{Account}::\prg{transfer}.\prg{body}\ \\  
		& \{ {\inside{\prg{a.key}}}\ \wedge\ {\PushASLong {\prg{res}} {\inside{\prg{a.key}}} } \wedge \prg{a.\balance}\geq b \  \} \ \ \  || \ \ \ 
		   \{ {\inside{\prg{a.key}}} \wedge \prg{a.\balance}\geq b \  \}
\\
\text{(33?)}  \ \ \ \ M_{good} \vdash \ 
		&	\{  \ \Alocalsset, \, \prg{a}:\prg{Account}\, , b:\prg{int} \, \wedge\,  {\inside{\prg{a.key}}} \, \wedge \, \protectedFrom {\prg{a.key}} {\Idsset}  \ \wedge \prg{a.\balance}\geq b \  \} \\
		& \SPT \prg{Account}::\prg{set}.\prg{body}\ \\  
 		& \{ {\inside{\prg{a.key}}}  \wedge\ {\PushASLong {\prg{res}} {\inside{\prg{a.key}}}}   \wedge \prg{a.\balance}\geq b \  \}\ \ \  || \ \ \ 
		   \{ {\inside{\prg{a.key}}} \wedge \prg{a.\balance}\geq b \  \} 
\end{align*}
\normalsize
where we are using ? to indicate that this needs to be proven, and 
where we are using the shorthands $\Alocals$,   $\Ids$, $\Alocalstr$, $\Idstr$, $\Alocalsset$ as defined earlier.

 \end{proofO}
 
The proofs for $M_{fine}$ are similar.

%
 We outline the proof of (31?) in Lemma \ref{l:buy:sat:S3}. %  in \S \ref{s:buy:sat:S3}.
 We outline the proof of (32?) in  Lemma \ref{l:transfer:sat:S3}.

\subsubsection{Proving that \prg{Shop::buy} from $M_{good}$ satisfies $\SThreeStrong$ and also $S_4$}
\label{s:buy:sat:S3}

\begin{lemma}[function $M_{good}::\prg{Shop}::\prg{buy}$  satisfies $\SThreeStrong$ and also $S_4$]
\label{l:buy:sat:S3}
 
\begin{align*}
\text{(31)}  \ \ \ \ M_{good}\ \vdash  \  \ 
		&	\{  \ \Alocals, \, \prg{a}:\prg{Account}, b:\prg{int}, \, \wedge\, {\inside{\prg{a.key}}} \, \wedge \, \protectedFrom {\prg{a.key}} {\Ids} \wedge \prg{a.\balance}\geq b \  \} \\
		& \SPT \prg{Shop}::\prg{buy}.\prg{body}\ \\  
		& \{ {\inside{\prg{a.key}}}\ \wedge\ {\PushASLong {\prg{res}} {\inside{\prg{a.key}}}}  \wedge \prg{a.\balance}\geq b \} \ \ \  || \ \ \ 
		   \{ {\inside{\prg{a.key}}}  \wedge \prg{a.\balance}\geq b \}
\end{align*}

\end{lemma}

\begin{proofO}
Note that (31) is a proof that $M_{good}::\prg{Shop}::\prg{buy}$  satisfies $\SThreeStrong$ and also  hat $M_{good}::\prg{Shop}::\prg{buy}$  satisfies $S_4$. This is so, because application of {\sc{[Method]}} on $S_4$ gives us exactly the proof obligation from (31).

This proof is similar to the proof of lemma \ref{l:buy:sat}, with the extra requirement here that we need to argue about balances not decreasing.
To do this, we will leverage the assertion about balances given in $S_3$.

We will use the shorthand $\Alocalsb \triangleq \Alocals, \,\prg{a}:\prg{Account}, b:\prg{int}$.
We will split the proof into 1) proving that statements 10, 11, 12 preserve the protection of \prg{a.key} from the \prg{buyer}, 2) proving that the external call 

\step{1st Step: proving statements 10, 11, 12}

We apply the underlying Hoare logic and prove that the statements on lines 10, 11, 12 do not affect the value of \prg{a.key} nor that of \prg{a.\balance}.  Therefore, for a $z,z'\notin \{ \prg{price}, \prg{myAccnt}, \prg{oldBalance} \}$, we have 

\begin{align*}
\text{(40)}  \ \ \ \ {M_{good} \vdash_{ul}} 
		&	\{  \ \Alocalsb\  \wedge\ z=\prg{a.key} \wedge\ z'=\prg{a.\balance}  \} \\
		&   \SPT \prg{price:=anItem.price}; \\  
		&   \SPT \prg{myAccnt:=this.accnt}; \\  
                 &   \SPT \prg{oldBalance := myAccnt.blnce};\\
		& \{ z=\prg{a.key} \wedge\ z'=\prg{a.\balance} \}
\end{align*}

We then apply {\sc{Embed\_UL}}, {\sc{Prot-1}} and {\sc{Prot-2}} and {\sc{Combine}} and and {\sc{Types-2}} on (10) and use the shorthand $\stmtsP$ for the statements on lines 10, 11 and 12, and obtain: 
\\
\begin{align*}
\text{(41)}  \ \ \ \ M_{good} \vdash 
		&	\{  \ \Alocalsb\  \wedge\ {\inside{\prg{a.key}}} \, \wedge\, \protectedFrom{\prg{buyer}} {\prg{a.key}} \wedge\ z'=\prg{a.\balance}  \} \\
		& \SPT \stmtsP\ \\  
		& \{ \ {\inside{\prg{a.key}}}  \, \wedge\, \protectedFrom{\prg{buyer}} {\prg{a.key}} \wedge\ z'=\prg{a.\balance}   \}
\end{align*}



We apply  {\sc{Mid}}  on (11) and obtain 
\begin{align*}
\text{(42)}  \ \ \ \ M_{good} \vdash 
		&	\{  \ \Alocalsb\, \wedge\, \protectedFrom {\prg{a.key}} {\prg{buyer}}\ \wedge\ z'=\prg{a.\balance}  \} \\
		& \SPT \stmtsP\ \\  
		& \{ \ \Alocalsb\, \wedge \  {\inside{\prg{a.key}}} \, \wedge\, \protectedFrom{\prg{buyer}} {\prg{a.key}}  \ \wedge\ z'=\prg{a.\balance}  \} \ \ || \\
		& \{ \ {\inside{\prg{a.key}}}\  \wedge\ z'=\prg{a.\balance} \}
\end{align*}

\step{2nd Step: Proving the External Call}

 

We now need to prove that the external method call \prg{buyer.pay(this.accnt, price)} protects the \prg{key}, and does nit decrease the balance, i.e.
\small
\begin{align*}
\text{(43?)} \ \ \ M_{good} \vdash & \{ \ \Alocalsb \   \wedge\    {\inside{\prg{a.key}}} \, \wedge\, \protectedFrom {\prg{a.key}} {\prg{buyer}} \wedge\ z'= \prg{a.\balance}  \} \\
		  		& \SPT  \prg{tmp := buyer.pay(myAccnt, price)}\ \\  
		  		& \{ \ \ \ \Alocalsb \ \wedge\ {\inside{\prg{a.key}}} \, \wedge\, \protectedFrom{\prg{buyer}} {\prg{a.key}} \wedge \ \prg{a.\balance}\geq z'\  \} \ \ \  || \ \ \  \\
		  		&   \{ \   {\inside{\prg{a.key}}}\ \wedge \  \prg{a.\balance}\geq z'  \}
\end{align*}
\normalsize

We use that $M \vdash \TwoStatesN  {\prg{a}:\prg{Account},\prg{b}:\prg{int}, }  {\inside{\prg{a.key}} \wedge \prg{a.\balance}\geq z'}   $
 and  obtain
 \\
 \small
\begin{align*}
\text{(44)} \ \ \ M_{good} \vdash & \{ \ \prg{buyer}:\prg{external},\,  {\inside{\prg{a.key}}} \, \wedge\, 
\protectedFrom {\prg{a.key}} {(\prg{buyer},\prg{myAccnt},\prg{price})} \  \wedge\ z'\geq \prg{a.\balance}  \} \\
		  		& \SPT  \prg{tmp := buyer.pay(myAccnt, price)}\ \\  
		  		& \{ \ \inside{\prg{a.key}} \, \wedge\, 
\protectedFrom {\prg{a.key}} {(\prg{buyer},\prg{myAccnt},\prg{price})}\ \wedge\ z'\geq \prg{a.\balance}  \} \ \ \  || \ \ \  \\
		  		&   \{ \   {\inside{\prg{a.key}}}\  \wedge\ z'\geq \prg{a.\balance}  \}
\end{align*}
\normalsize 
 
In order to obtain (43?) out of (44), we apply \textsc{Prot-Intl} and \textsc{Prot-Int}$_1$,   which gives us\\
$
\begin{array}{llll}
& (45) & & M_{good} \vdash \Alocalsb \wedge  {\inside{\prg{a.key}}}\  \longrightarrow\ \protectedFrom {\prg{a.key}} {\prg{myAccnt}} 
\\
& (46) & & M_{good} \vdash \Alocalsb \wedge  {\inside{\prg{a.key}}}\  \longrightarrow\ \protectedFrom {\prg{a.key}} {\prg{price}} 
\\
& (47) & & M_{good} \vdash \Alocalsb \wedge  z'= \prg{a.\balance}\   \longrightarrow\  z'\geq \prg{a.\balance} 
\end{array}
$

We apply {\textsc{Consequ}} on (44), (45), (46) and (47) and obtain (43)!

\normalsize

\step{3nd Step: Proving the Remainder of the Body}

 We now need to prove that lines 15-19 of the method preserve the protection of \prg{a.key}, and do not decrease \prg{a.balance}.
 We outline the  remaining proof in less detail.
 
 We prove the internal call on line 16, using the method specification for \prg{send}, using $S_{2a}$ and $S_{2b}$, and applying rule {\sc{[Call\_Int]}}, and obtain

 \small
\begin{align*}
\text{(48)} \ \ \ M_{good}\  \vdash \ & \{ \ \prg{buyer}:\prg{external},\, \prg{item}:\prg{Intem} \wedge {\inside{\prg{a.key}}} \, \wedge\, 
\protectedFrom {\prg{a.key}} {(\prg{buyer}}  \  \wedge\ z'= \prg{a.\balance}  \} \\
		  		& \SPT  \prg{tmp := this.send(buyer,Item)}\ \\  
		  		& \{ \ \inside{\prg{a.key}} \, \wedge\, 
\protectedFrom {\prg{a.key}} {\prg{buyer}} \ \wedge\ z'= \prg{a.\balance}  \} \ \ \  || \ \ \  \\
		  		&   \{ \   {\inside{\prg{a.key}}}\  \wedge\ z'= \prg{a.\balance}  \}
\end{align*}
\normalsize  
 

We now need to prove that the external method call \prg{buyer.tell("You have not paid me")} also protects the \prg{key}, and does nit decrease the balance. We can do this by applying the rule about protection from strings  {\sc{[Pror\_Str]}}, the fact that $M_{good} \vdash S_{3}$, and rule  {\sc{[Call\_Extl\_Adapt]}} and obtain:


 \small
\begin{align*}
\text{(49)} \ \ \ M_{good}\  \vdash \ & \{ \ \prg{buyer}:\prg{external},\, \prg{item}:\prg{Intem} \wedge {\inside{\prg{a.key}}} \, \wedge\, 
\protectedFrom {\prg{a.key}} {(\prg{buyer}}  \  \wedge\ z'/geq  \prg{a.\balance}  \} \\
		  		& \SPT  \prg{tmp:=buyer.tell("You have not paid me")}\ \\  
		  		& \{ \ \inside{\prg{a.key}} \, \wedge\, 
\protectedFrom {\prg{a.key}} {\prg{buyer}} \ \wedge\ z'\geq \prg{a.\balance}  \} \ \ \  || \ \ \  \\
		  		&   \{ \   {\inside{\prg{a.key}}}\  \wedge\ z'\geq  \prg{a.\balance}  \}
\end{align*}
\normalsize 

We can now apply  {\sc{[If\_Rule}}, and {\sc{[Conseq}} on (49) and (50),  and obtain

 \small
\begin{align*}
\text{(50)} \ \ \ M_{good}\  \vdash \ & \{ \ \prg{buyer}:\prg{external},\, \prg{item}:\prg{Intem} \wedge {\inside{\prg{a.key}}} \, \wedge\, 
\protectedFrom {\prg{a.key}} {(\prg{buyer}}  \  \wedge\ z'\geq  \prg{a.\balance}  \} \\
		  		& \SPT  \prg{if} ... \prg{then}\\
				& \SPT \SPT  \prg{tmp:=this.send(buyer,anItem)}\ \\  
				& \SPT  \prg{else}\\
				& \SPT \SPT  \prg{tmp:=buyer.tell("You have not paid me")}\ \\  
		  		& \{ \ \inside{\prg{a.key}} \, \wedge\, 
\protectedFrom {\prg{a.key}} {\prg{buyer}} \ \wedge\ z'\geq \prg{a.\balance}  \} \ \ \  || \ \ \  \\
		  		&   \{ \   {\inside{\prg{a.key}}}\  \wedge\ z'\geq  \prg{a.\balance}  \}
\end{align*}
\normalsize 

The rest follows through application of {\sc{[Prot\_Int}}, and {\sc{[Seq]}}.



\end{proofO}

\begin{lemma}[function $M_{good}::\prg{Account}::\prg{transfer}$ satisfies $S_3$]

 \small
 \begin{align*}
\text{(32)}  \ \ \ \ M_{good} \vdash \ 
		&	\{  \ \Alocalstr, \, \prg{a}:\prg{Account}\, , b:\prg{int} \, \wedge\,  {\inside{\prg{a.key}}} \, \wedge \, \protectedFrom {\prg{a.key}} {\Idstr}  \ \wedge \prg{a.\balance}\geq b \  \} \\
		& \SPT \prg{Account}::\prg{transfer}.\prg{body}\ \\  
		& \{ {\inside{\prg{a.key}}}\ \wedge\ {\PushASLong {\prg{res}} {\inside{\prg{a.key}}} } \wedge \prg{a.\balance}\geq b \  \} \ \ \  || \ \ \ 
		   \{ {\inside{\prg{a.key}}} \wedge \prg{a.\balance}\geq b \  \}
\end{align*}
\normalsize


\end{lemma}

\begin{proofO}
We will use   the shorthand $stmts_{28-33}$ for the statements in the body of \prg{transfer}. 
We will prove   the preservation of protection, separately from the balance not decreasing when the key is protcted.

For the former, applying the steps in the proof of Lemma \ref{l:transfer:S2},  we obtain

\small
 \begin{align*}
\text{(21)}  \ \ \ \ M_{good} \vdash \ 
		&	\{  \ \Alocalstr, \, \prg{a}:\prg{Account}\, \wedge\,  {\inside{\prg{a.key}}} \, \wedge \, \protectedFrom {\prg{a.key}} {\Idstr}  \  \} \\
		&  \SPT   stmts_{28-33} \\
		& \{ {\inside{\prg{a.key}}}\ \wedge\ {\PushASLong {\prg{res}} {\inside{\prg{a.key}}}}  \} \ \ \  || \ \ \ 
		   \{ {\inside{\prg{a.key}}} \}
\end{align*}
\normalsize

For the latter, we  rely on the underlying Hoare logic to ensure that no balance decreases, except perhaps that of the receiver, in which case its key was not protected.
Namely, we have that

\small
 \begin{align*}
 \text{(71)}  \ \ \ \ M_{good} \vdash_ul \ 
		&	\{  \ \Alocalstr, \, \prg{a}:\prg{Account}\, \wedge\,  \prg{a.\balance}=b\ \wedge \ (\prg{this}\neq \prg{a} \vee prg{this}.\prg{key}\neq \prg{key}' ) \  \} \\
		&  \SPT   stmts_{28-33} \\
		& \{ \prg{a.\balance}\geq b \}
\end{align*}
\normalsize

We apply rules {\sc{Embed\_UL}} and   {\sc{Mid}} on (71), and obtain

\small
 \begin{align*}
 \text{(72)}  \ \ \ \ M_{good} \vdash  \ 
		&	\{  \ \Alocalstr, \, \prg{a}:\prg{Account}\, \wedge\,  \prg{a.\balance}=b\ \wedge \ (\prg{this}\neq \prg{a} \vee prg{this}.\prg{key}\neq \prg{key}' ) \  \} \\
		&  \SPT   stmts_{28-33} \\
		& \{ \prg{a.\balance}\geq b \} \ \ \  || \ \ \  \{ \prg{a.\balance}\geq b \}
\end{align*}
\normalsize

Moreover, we have 

\small
$\begin{array}{llll}
 \text{(73)} \ \ \  & M_{good}\  & \vdash  \ & \protectedFrom {\prg{a.key}} {\Idstr}\ \  \rightarrow \ \ \protectedFrom {\prg{a.key}} {\prg{key'}} \\
 \text{(74)} &  M_{good}  & \vdash  & \protectedFrom {\prg{a.key}} {\prg{key'}} \ \  \rightarrow\ \  \prg{a.key} \neq \prg{key'}  \\
  \text{(75)} &  M_{good}  & \vdash  &    \prg{a.key} \neq  \prg{key'}  \ \ \rightarrow \ \ \prg{a}\neq \prg{this} \vee \prg{this.key}\neq \prg{key}'
\end{array}
$

normalsize

Applying (73), (74), (75) and {\sc{Conseq}} on (72) we obtain:

\small
 \begin{align*}
 \text{(76)}  \ \ \ \ M_{good} \vdash  \ 
		&	\{  \ \Alocalstr, \, \prg{a}:\prg{Account}\, \wedge\,  \prg{a.\balance}=b\ \wedge \ \protectedFrom {\prg{a.key}} {\Idstr}\ \  \} \\
		&  \SPT   stmts_{28-33} \\
		& \{ \prg{a.\balance}\geq b \} \ \ \  || \ \ \  \{ \prg{a.\balance}\geq b \}
\end{align*}
\normalsize

We combine  (72) and (76) through {\sc{Combine}} and obtain (32).

\end{proofO}

\subsection{Dealing with polymorphic function calls}

The case split rules together with the rule of consequence allow our Hoare logic to formally reason about polymorphic calls, where the receiver may be internal or external.

We demonstrate this through an example where we may have an external receiver, or a receiver from a class $C$. Assume we had a module $M$ with a scoped invariant (as in A), and an internal method specification as in (B). 

$\begin{array}{lclcl}
& (A)& M & \vdash & \TwoStatesN{y_1:D}{A} \\
& (B) & M & \vdash & \mprepostLong {A_1}{ \prg{private}\ C} {m}   {y_1: D} {A_2} { \parallel \{A_3\}}  \\
\\
  \end{array}
$

Assume also implications as in (C)-(H)

$\begin{array}{lclcl}
&  (C) & M & \vdash & A_0 \ \rightarrow \ \PushASLong {(y_0,y_1)}{A}\\
&  (D) & M & \vdash &  \PushASLong {(y_0,y_1)}{A} \rightarrow A_4\\
 & (E) &  M & \vdash & A \rightarrow A_5\\
  & (F) &  M & \vdash & A_0 \rightarrow A_1[y_0/\prg{this}]\\
  & (G) &  M & \vdash & A_2[y_0,u/\prg{this},res] \rightarrow A_4 \\
    & (H) &  M & \vdash & A_3  \rightarrow A_5 
\\
  \end{array}
$

Then, by application of {\sc{Call\_Ext\_Adapt}}  on (A) we obtain (I)

$
\SPT (I)\ \ {  \hprovesN {M} 
						{ \  y_0:external, y_1: D \wedge  {\PushASLong {(y_0,y_1)} {A}}\ }  
						 { \ u:=y_0.m(y_1)\    }
					         { \   \PushASLong {(y_0,y_1)}{A} \ } 
						{  \ A \ }  }
						\\
						$
						
By application of the rule of consequence on (I) and (C), (D), and (E), we obtain

$
\SPT (J)\ \ {  \hprovesN {M} 
						{ \  y_0:external, y_1: D \wedge  A_0\ }  
						 { \ u:=y_0.m(y_1)\    }
					         { \   A_4 \ } 
						{  \ A_5\ }  }
						\\
						$

Then, by application of {\sc{[Call\_Intl]}}  on (B) we obtain (K)

$
\SPT (K)\ \ {  \hprovesN {M} 
						{ \  y_0:C, y_1: D \wedge  A_1[y_0/\prg{this}]\ }  
						 { \ u:=y_0.m(y_1)\    }
					         { \  A_2[y_0,u/\prg{this},res] \ } 
						{  \ A_3 \ }  }
						\\
						$
						
By application of the rule of consequence on (K) and (F), (G), and (H), we obtain

$
\SPT (L)\ \ {  \hprovesN {M} 
						{ \  y_0:C, y_1: D \wedge  A_0\ }  
						 { \ u:=y_0.m(y_1)\    }
					         { \  A_4 \ } 
						{  \ A_5 \ }  }
						\\
						$

By case split, {\sc{[Cases]}}, on (J) and (L), we obtain

$
\SPT (polymoprhic)\ \ {  \hprovesN {M} 
						{ \  (y_0: external \vee y_0:C), y_1: D \wedge  A_0\ }  
						 { \ u:=y_0.m(y_1)\    }
					         { \  A_4 \ } 
						{  \ A_5 \ }  }
						\\
						$


