% We  now discuss   syntax and semantics of  our specifications, and illustrate them through examples.
 
\subsection{\textbf{Syntax, Semantics, and Examples of Specifications}}
% chopped below, for space
% Our specification language   supports scoped invariants,   method specifications, and {conjunctions}. 

\begin{definition} [Specifications Syntax]     We define the syntax  of  specifications, $S$:
\label{f:holistic-syntax}
\[
\begin{syntax}
\syntaxElement{S}{ }
		  {\syntaxline
				{\  \TwoStatesN {\overline {x:C}} {A}\  }
 				{\ \mprepostN{A}{p\ C}{m}{y}{C}{A} {A}\ } 
				{\ S\, \wedge \, S\ }
		 \endsyntaxline
 		}
\endSyntaxElement\ 
\syntaxElement{p}{ } 
 	 {\syntaxline
                                  {\    \prg{private} \ } 	
				 {\   \prg{public} \ } 	
		 \endsyntaxline
 		}
\endSyntaxElement 
\end{syntax}
\]


\end{definition}

In Def. \ref{f:holistic-wff}  later on we describe  well-formedness of $S$, but  we first discuss  semantics and some examples.
%To motivate this definition, we first define the semantics of these specifications, and show some examples.
%
\label{ssect:sem}
 % 
%The semantics of specifications uses  
%For the semantics we 
We use quadruples involving states: % rather than statements}:
${\satAssertQuadruple  \Mtwo  M     {A} \sigma {A'} {A''} }$ 
  says that   if $\sigma$ satisfies $A$, then any terminating scoped execution of its continuation (${\leadstoBoundedStarFin { \Mtwo\madd M}{\sigma}  {\sigma'} }$) will satisfy $A'$, and any intermediate reachable external state 
  (${\leadstoBoundedStar  {\Mtwo\madd M}{\sigma}  {\sigma''}}$) will satisfy the ``mid-condition'', $A''$. 
  
 
\begin{definition} \label{def:hoare:sem}
\label{def:shallow:spec:sat:state}
For modules $\Mtwo$, $M$, state $\sigma$, and assertions $A$, $A'$ and  $A''$, we define:
\begin{itemize}
\item
$ {\satAssertQuadruple  \Mtwo  M     {A} \sigma {A'} {A''} } \ \ \triangleq \ \ \forall \sigma',\sigma''.[
$  \\
$\strut \hspace{.2cm} M,  \sigma \models  {A}   
  \  \ \Longrightarrow \ \   [ \ \  {\leadstoBoundedStarFin { \Mtwo\madd M}{\sigma}  {\sigma'} }\ \ \Longrightarrow\ \   M,  \sigma' \models  {A'}  
 \ \ \ \  ] \ \ \ \wedge$\\ 
 $\strut   \   \hspace{2.5cm}  [ \ \   {\leadstoBoundedStar  {\Mtwo\madd M}{\sigma}  {\sigma''} }\ \  \ \Longrightarrow\   \   M,  \sigma'' \models  {(\extThis \rightarrow {\as \sigma {A''}} )}\ \ \  ] \ \ \ \ \ ]$ 
\end{itemize} 
\end{definition}

{\begin{example}
Consider ${\satAssertQuadruple  {...}  {...}     {A_1} {\sigma_{4}} {A_2} {A_3} }$   
for Fig. \ref{fig:illusrPreserve}.
It  means  that  if  $\sigma_4$ satisfies $A_1$, 
then $\sigma_{23}$ will satisfy $A_2$, while $\sigma_6$-$\sigma_9$,\ $\sigma_{13}$-$\sigma_{17}$, and $\sigma_{20}$-$\sigma_{21}$ will satisfy $A_3$.
It does not imply anything about $\sigma_{24}$ because $\notLeadstoBoundedStar {...} {\sigma_4} {\sigma_{24}}$.
Similarly, if $\sigma_8$ satisfies $A_1$ 
then $\sigma_{14}$ will satisfy $A_2$, and  $\sigma_{8}$, $\sigma_{9}$, $\sigma_{13}$, $\sigma_{14}$  will satisfy $A_3$, while making no claims about  $\sigma_{10}$, $\sigma_{11}$, $\sigma_{12}$, nor  about $\sigma_{15}$ onwards.
 \end{example}}

{Now  we} % should have been "to the" but this way we avoid bad line break in 3 lines
 define    $\satisfies{M}{\TwoStatesN {\overline {x:C}} {A}}$ 
to mean that  if an external state $\sigma$ satisfies $A$, then all future external states reachable from $\sigma$—including nested  
 calls and returns but  {\emph{stopping} before}   returning from the active call in $\sigma$— also satisfy $A$. 
 And  $\satisfies{M} { \mprepostN {A_1}{p\ D}{m}{y}{D}{A_2} {A_3} }$ means that scoped execution of a call to $m$ from $D$   in  states satisfying $A_1$ leads to final states satisfying $A_2$ (if it terminates),
 and to intermediate external states satisfying $A_3$.

\begin{definition}  [Semantics of  Specifications]
We define $\satisfies{M}{{S}}$ by cases over $S$:  

\label{def:necessity-semantics}

\begin{enumerate}
 \item
\label{def:necessity-semantics-first}
 $\satisfies{M}{\TwoStatesN {\overline {x:C}} {A}} \ \  \ \triangleq   \ \ \ {\forall   \Mtwo,  \sigma.[\ {\satAssertQuadruple  \Mtwo  M    {\extThis \wedge \overline {x:C} \wedge A} \sigma {A} {A} }\ ].}$
  \item
   \label{def:necessity-semantics-second}
 $\satisfies{M} { \mprepostN {A_1}{p\ D}{m}{y}{D}{A_2} {A_3} }\  \ \ \   \triangleq  $ \\ %  \ \ \ \forall   \Mtwo,  \sigma, y_0,\overline y.[\ $    \\
$\strut \ \   \forall   \Mtwo,  \sigma, y_0,\overline y.[\ 
% $\\ $\strut  \ \ \   \ \ \ \ \ \ \ \ \   \  \ \ \
 \ \sigma.\prg{cont}\txteq {u:=y_0.m(y_1,..y_n)} \ \ \Longrightarrow \ \ $\\
$\strut  \ \ \   \ \ \ \ \ \ \ \ \   \ \ \  \ \ 
\ \ \ \ \ \ \ \ \ {\satAssertQuadruple  { \Mtwo} {M} { y_0\!:\!D, \overline {y\!:\!D}   \wedge   A[y_0/\prg{this}]}  {\ \sigma\ }   {A_2[u/res,y_0/\prg{this}] }{A_3 } } \  \ \  ]  $   
%$\strut  \ \ \   \ \ \ \ \ \ \ \ \   \  \mbox{where}$\\
%$\strut  \ \ \   \ \ \ \ \ \ \ \ \   \   A_1' \triangleq   y_0:D,{\overline {y:D}}   \wedge   A[y_0/\prg{this}],\  \  A_2' \triangleq A_2[u/res,y_0/\prg{this}],\ \ A_3' \triangleq A_3  \  ]$  
 \item
 $\satisfies{M}{S\, \wedge\, S'}$\ \ \  \ \ \  $\triangleq$  \  \ \  \   $\satisfies{M}{S}\ \wedge \ \satisfies{M}{S'}$
\end{enumerate}
\end{definition}

Fig. \ref{fig:illusrPreserve} in  \S \ref{sect:approach:scoped}  illustrated  the meaning of ${\TwoStatesN {\overline {x:C}} {A_0}}$. 
Moreover, $M_{good} \models S_2 \wedge S_3 \wedge S_4$, and  $M_{fine} \models S_2 \wedge S_3 \wedge S_4$,
 while $M_{bad} \not\models S_2$.
We continue with some examples -- more in % can be found in 
\A ~\aref{E.1.1}{\ref{app:spec}}.

{
 \begin{example}[Scoped Invariants and Method Specs]
 \label{example:twostate}
 \label{example:mprepostl}
  % 
 $S_5$  says % shorter than guarantees  
  that   non-null keys are immutable:
 \\
 \begin{tabular}{lcll}
$\strut \ \ \ \ \ \ \ \ S_5$ & $\triangleq$   & ${\TwoStatesN {\prg{a}:\prg{Account},\prg{k}:\prg{Key}}  {\prg{null}\neq \prg{k}=\prg{a.\password}}} $  \end{tabular}
\\
$S_9$    guarantees that \prg{set} preserves the protectedness of any account, and any key.  \\
   {\sprepost
		{\strut \ \ \ \ \ \ \ \ \ S_9} 
		{  a:\prg{Account}, a':\prg{Account}\wedge  \inside{a}\wedge  \inside{a'.\prg{key}} }
		{\prg{public Account}} {\prg{set}} {\prg{key'}:\prg{Key}}
		{   \inside{a}\wedge  \inside{a'.\prg{key}}  }
		{   \inside{a}\wedge  \inside{a'.\prg{key}} }
}

\noindent
Note that  $a$, $a'$ are disjoint from \prg{this} and the formal parameters of \prg{set}. 
In that sense, $a$ and $a'$ are universally quantified; a call of \prg{set} will preserve protectedness for \emph{ all} accounts and their keys. 

\end{example}

\subsection{\textbf{Well-formedness of Specifications}} We now define what it means for a specification to be well-formed:

\begin{definition}% [Specifications Well-formed]    
 {\emph{Well-formedness}} of specifications,  $\vdash S$,  is   defined by cases on $S$:
\label{f:holistic-wff}

\begin{itemize}
\item
  $\strut \  \vdash {\TwoStatesN {\overline {x:C}} {A}} \ \ \ \triangleq\  \ \ \fv(A)\subseteq\{  \overline x \}\,\wedge\, {M \vdash \encaps {{\overline {x: C}} \wedge A}} $.
    \item
 $\strut \ \vdash {\mprepostN{\overline{x:C'} \wedge A}{p\ C}{m}{y}{C}{A'} {A''}}\ \ \ \triangleq$\\
$\strut \hspace{0.7cm}  [ \ \ {\small{   {\prg{res},\prg{this}\!\notin\! \overline{x}, \overline{y}}\ \wedge\  {\fv(A)\!\subseteq\! \overline x, \overline y, \prg{this}}\     \wedge\    \fv(A')\!\subseteq \! \overline{x}, \overline{y}, \prg{this}, \prg{res}\   \wedge\   \fv(A'')\!\subseteq\!  {\overline{x}} }} $\\
    $\strut \hspace{0.7cm} \ \ \  \wedge\  \Pos {A } \, \ \wedge \ \,  \Pos {A'} \, \ \wedge \  \,  M \vdash \encaps  {\overline {x: C'} \wedge A''}\ \ \  ]$ 
  \item  $\strut \   \vdash S\, \wedge \, S' \ \ \triangleq \ \  \vdash S\, \ \wedge\,  \ \vdash S'  $.  
\end{itemize}

\end{definition}

Def \ref{f:holistic-wff}'s  requirements about  free variables are relatively straightforward -- more  in. \S \aref{E.1.1}{\ref{wff:spec:free:more}}.

%Def \ref{f:holistic-wff}'s  requirements about encapsulation are more interesting:
Def \ref{f:holistic-wff}'s  requirements about encapsulation are motivated by  Def.   \ref{def:necessity-semantics}. If  $\overline {x:C}\wedge A$ in the scoped invariant  were not encapsulated,  then it could be invalidated by some external code, and it would be impossible to ever satisfy Def.   \ref{def:necessity-semantics}(\ref{def:necessity-semantics-first}). 
Similarly, if a method specification's mid-condition, $A''$, could be invalidated by some external code, then it would be impossible to ever satisfy Def.   \ref{def:necessity-semantics}(\ref{def:necessity-semantics-second}). 


Def \ref{f:holistic-wff}'s  requirements about stability are motivated by our Hoare logic rule for internal calls,   {\sc{[Call\_Int]}}, Fig \ref{f:internal:calls}. The requirement    $\Pos {A}$ for the method's precondition  gives that $A$ is preserved when an internal frame is pushed, \cf Lemma \ref{l:preserve:asrt}.
The requirement     $\Pos {A'}$ for the method's postcondition gives,  in the context of \strong satisfaction,  that $A'$ is preserved when an internal frame is popped, \cf Lemma \aref{G.42}{\ref{l:calls:return:deep}}. This is crucial for soundness of  {\sc{[Call\_Int]}}.

%\sdnew
{In  \S \A ~\aref{E.1.2}{\ref{wff:spec:encaps:more}} we discuss the encapsulation requirements, and why these do not restrict expressiveness.}
  %precondition's free variables all appear in the formal parameters or 
%{\emph{Well-formedness}},  $\vdash S$,  is   defined by cases on $S$:\\
%  $\strut \ \  \vdash {\TwoStatesN {\overline {x:C}} {A}} \ \ \ \triangleq\  \ \ \fv(A)\subseteq\{  \overline x \}\,\wedge\, {M \vdash \encaps {{\overline {x: C}} \wedge A}} $;\\
% $\strut \ \  \vdash {\mprepostN{A}{p\ C}{m}{y}{C}{A'} {A''}}\ \ \ \triangleq\  \ \  \exists \overline x, \overline {C'}.[ $\\
%  $\strut \hspace{1cm}  {\prg{res}\notin \overline{x}, \overline{y}}\,  \wedge\,  \ {\fv(A_0)\subseteq \overline{x},\overline y, \prg{this}}\   \wedge \fv(A')\subseteq  \fv(A),\prg{res}\   \wedge\  \fv(A'')\subseteq  {\overline{x}} $\\
%  $\strut \hspace{1cm}\wedge\  \Pos A\, \wedge\, \Pos {A'}\, \wedge \,\Pos {A''}\, \wedge  \,  M \vdash \encaps  {\overline {x: C'} \wedge A''}\ \ \  ]$ \\
%  $\strut \ \   \vdash S\, \wedge \, S' \ \ \triangleq \ \  \vdash S\, \wedge\, \vdash S'  $.  



\subsection{\textbf{Discussion}}  

\paragraphSD{Difference with Object and History Invariants.}  Our scoped invariants are similar to, but different from, history invariants  and object invariants.
% but neither of these provide what we need. 
% SD chopped the above, since we already said "different"
We compare through an example:

\vspace{-.35cm}
 
\noindent
\begin{flushleft}
\begin{tabular}{@{}lr@{}}
  \begin{minipage}{.85\textwidth}
Consider $\sigma_a$ making a call  transitioning to $\sigma_b$,    execution of $\sigma_b$'s continuation   eventually resulting in $\sigma_c$, and $\sigma_c$ returning  to $\sigma_d$. 
Suppose all four states are external, and the module guarantees $\TwoStatesN {\overline{x:Object}} {A}$, and $\sigma_a \not\models A$, but $\sigma_b \models A$. 
Scoped invariants %require   
ensure  $\sigma_c \models A$, but allow   $\sigma_d \not\models A$.\end{minipage}
& 
\begin{minipage}{.18\textwidth}
\resizebox{2cm}{!}{
\includegraphics[width=\linewidth]{diagrams/compare.png}
} 
\end{minipage}
\end{tabular}
\end{flushleft}


{\emph{History  invariants}} \cite{liskov94behavioral,usinghistory,Cohen10}, instead, consider {all  future states including any method returns}, and therefore {would  require that   $\sigma_d \models A$. Thus, they are,}  for our purposes,  both
 \emph{unenforceable} and overly \emph{restrictive}.\ \  \emph{Unenforceable}: \ Take $A \txteq \inside{\prg{acc.key}}$,  assume  in $\sigma_a$ a path to an external object which has access to $\prg{acc.key}$, assume that path is unknown in $\sigma_b$: then, the transition from $\sigma_b$ to $\sigma_c$ cannot eliminate that path—hence, $\sigma_d \not\models \inside{\prg{acc.key}}$.\ \  \emph{Restrictive}:\ Take $A \txteq \inside{\prg{acc.key}}\wedge a.\prg{blnce}\geq b$; then,  requiring  $A$   to hold in all states from $\sigma_a$ until termination would prevent all future withdrawals from $a$, rendering the account useless.

{\emph{Object invariants}}  \cite{Meyer92,MeyerDBC92,BarDelFahLeiSch04,objInvars,MuellerPoetzsch-HeffterLeavens06}, on the other hand, expect %require -- too many require in the para
invariants to hold in all (visible) states,
here would require,  \eg that $\sigma_a \models A$. Thus, they  are %equally 
\emph{inapplicable} for us: They would require, \eg, that for all % objects 
 $\prg{acc}$, in all (visible) states, $\inside{\prg{acc.key}}$, and thus prevent \emph{any} withdrawals from \emph{any} account in \emph{any} state.
 
 

\paragraphSD{Difference between Postconditions and Invariants.}
% The interested reader might have noticed that 
In all  method specification examples so far, the post-condition and   mid-condition were identical.
However, this need not be so. 
Assume a method \prg{tempLeak} defined in \prg{Account}, with an external argument \prg{extArg}, and  method body:
\\
$\strut \hspace{1cm} \prg{extArg.m(this.key); this.key:=new Key}$} 
\\
Then, the assertion   $ \inside{\prg{this.key}}$  is  invalidated by the external call \prg{extArg.m(this.key)}, but is  established by \prg{this.key:=new Key}.
Therefore, $ \inside{\prg{this.key}}$  is a valid post-condition but not a valid   mid-condition.
The specification of \prg{tempLeak} could be\\
%{\small{
$
{\sprepost
		{\strut \ \ \ \ \ \ \ \ \ S_{\prg{tempLeak}}} 
		{  \ \prg{true}\  }
		{\prg{public Account}} {\prg{tempLeak}} {\prg{extArg}:\prg{external}}
		{  \  \inside{\prg{this.key} }\  }
		{  \  \prg{true}\  }
}
$
%}}


 
\footnoteSD{First bullet: This means that we require all objects to satisfy even if not locally relevant. Second Bullet: notice that we are asking for globally relevant objects}  
\footnoteSD{{TODO: Make an example that demonstrates the difference if in the second bullet we had asked for locally relevant objects ${\overline o}$.}}
\footnoteSD{TODO: explain why we did not require the stronger $\leadstoFin{M_{ext}\!\circ \!M}{\sigma}{\sigma'}$ rather than $\leadstoBoundedStar {M_{ext}\!\circ \!M}{\sigma}  {\sigma'}$.}

\newcommand{\paragraphSDD}[1]{\vspace{.02cm}{\textit{#1}}}
 
 \paragraphSD{Expressiveness} 
%\subsection{Expressiveness} 
In  \S \ref{app:expressivity} we argue the expresseness of our approach  through a sequence of capability patterns studied in related approaches from the literature  
 \cite{OOPSLA22,dd,VerX,irisWasm23,ddd} and written in our specification language.
These approaches %in  \cite{OOPSLA22,dd,VerX,irisWasm23}  
 are based on temporal logics \cite{VerX,OOPSLA22}, or on extensions of Coq/Iris \cite{dd,irisWasm23,ddd}, and
none offer a Hoare logic    for external calls.
%Other approaches in the literature are either unable to handle external method calls \cite{OOPSLA22}, or use  bespoke proofs \cite{dd,irisWasm23}, or model checking \cite{VerX}.
%Their specification languages are based on temporal logics \cite{VerX,OOPSLA22}, or on extensions of Coq and Iris \cite{dd,IrisWasm23}.
%More in  \S \ref{app:expressivity}. 
% we argue our approach is able to prove comparable specifications to those proposed in  \cite{OOPSLA22,dd,VerX,irisWasm23}, in the presence of external method calls, using a Hoare logic.=≠≠≠q±±≠≠≠qq≠q≠≠www≠≠
%We   summarize here.

 %% We continue the comparison of expresiveness between \emph{Chainmail} and \Nec, by 
 %% considering the examples studied in \cite{FASE}.
 
%\begin{example}[ERC20]

%\paragraphSDD{DOM} % is the motivating example  in \cite{dd}:
%Access to any DOM node
%gives read/write  permissions to  all its \prg{parent} and \prg{children} nodes. 
%These permissions are attenuated   through a \prg{Proxy} class, %which has a field \prg{node} pointing to a \prg{Node}, and a field \prg{height}, 
% which restricts the range of \prg{Node}s which may be modified through the use of the particular \prg{Proxy}. 
%We  express such  attenuation   through two scoped invariants.
%% The corresponding specification in \ref{OOPSLA22} is comparable, but not able to prove external calls. % not specific as to the frame from which any modification originated.
%
%\paragraphSDD{DAO} %Decentralized Autonomous Organization\
% ~\cite{Dao}  is a well-known Ethereum contract   which was exploited with a re-entrancy bug in 2016, 
%and lost \$50M. 
%Our scoped invariants  would have secured %the DAO 
% against that bug. % such a  bug. 
%But note  that  they are about precluded effects, and 
%%. They are, essentially, simple object invariants and 
%thus expressible % could have been expressed 
% with techniques proposed in the 90's \cite{MeyerDBC92}.
%% \cite{OOPSLA22}  gives one  further specification, which says  that any reduction of funds can only be caused through a call to a specific method -- such specifications are beoind our scope.
% 
% \paragraphSDD{ERC20} is a widely used % token 
% standard describing  basic functionality of Ethereum-based token 
%contracts. 
%The Solidity security model is not based on access to  capabilities but on who the caller  is. 
%We  adapted our approach correspondingly, and 
%express 1) that  the owner of an account is always authorized on that account,  2) any execution which does not contain calls from a participant  authorized on some account will not affect the balance nor  who is authorized on  that account. 
%% The specifications from \cite{OOPSLA22} are more API-specific, in that they pinpoint which method calls caused an effect, and less specific in that they do not pinpoint the frame from which the effect occurred. 
%
%\paragraphSDD{Stack} is a Wasm module exporting separate functions to read or modify its contents \cite{irisWasm23}. We specify that in the absence of external access to the latter capability, the contents will not change.  
%
%\paragraphSDD{Sealer/Unsealer} was proposed by \citet{JamesMorris} and formalized in \cite{ddd}. It proposes an unsealer for  safe passing of sealed high-integrity values. Our specification says that without access to the unsealer, the value does not get read, and that the unsealer does not get leaked.



 

%% KEEP ALL BELOW, but currently not needed 
%\subsection{\SpecLang Entailments}
%
%{We define entailment of specifications wrt a module in the expected way.} %The usual definition of entailment applies to our specifications as well}
%
%\begin{definition}[Satisfaction of Assertions by a module] 
%\label{def:assertion-inference-semantics}
%We define satisfaction of an assertion $A$ by a  module $M$ as:
%\begin{itemize}
%\item
%{
%$M \models A$   \ \ \ iff \ \ \  $\forall \overline{M}. \forall \sigma
%[\ \    \arising{\sigma}{M\madd\overline{M}}\   \  \wedge\ \  \satisfiesA {M}   {\sigma} {\external{\prg{this}}} 
%\   \ \Longrightarrow \ \ \satisfiesA{M}{\sigma}{A}\ \ ]$
%}\footnote{Not sure about the need for external and arising.}
%\end{itemize}
%\end{definition}
%
%%TODO: Here we will say that assertions are classical, as proven in FASE
%
%\begin{definition}[Stronger Specifications] 
%\label{def:specification-implication-semantics}
%Specification $S$ is stronger than another specification $S'$  in the context of a  module: 
% \begin{itemize}[itemsep=5pt]
%\item 
%$\stronger M  S  {S'}$   \ \ \ iff \ \ \  $M\models S$ implies $M \models S'$
%\item
%$\strongerEq M  S  {S'}$   \ \ \ iff \ \ \ $\stronger M  S  {S'}$  \ and \  $\stronger M   {S'} S$    
%\end{itemize}
%\end{definition}
%
%\noindent
%{Interestingly, entailment can deduce some method specifications out of two-state invariants.}
%
%{
% \begin{example}
% \label{example:entail}
% Any module $M$ whose code does not call  method \prg{buy} gives   $\stronger M {S_2 \wedge S_S4} {S_9}$
%\end{example}
%
%
%% Remember $S_1$, ... $S_4$ as defined in Sect. \ref{s:bankSpecEx}, and consider the specifications $S_6$ and $S_7$ from Example \ref{example:mprepostl}.
%% Then, for any module $M$ %which has a public method \prg{set}, 
%% we have that
%\begin{example}
% \label{example:entail}
%For any module $M$,  we have  $\strongerEq M {S_2 \wedge S_4} {S_2 \wedge S_{4a}}$, where $S_{4a}$ defined as 
%\\
%\begin{tabular}{lcll}
%  $S_{4a}$   & $\triangleq$   &  
% $ \TwoStatesQ{\prg{a}:\prg{Account},\prg{b}:\prg{int}}  {\inside{\prg{a.\password}} \wedge \prg{a.\balance}\geq\prg{b}} 
% {\inside{\prg{a.password}} \wedge \prg{a.\balance}\geq\prg{b}} $
% \end{tabular}
%\ \end{example}
%}
% 
%%Some properties of $M \models \_  \subseteq \_ $ are given below:
%%
%%\begin{lemma}
%%For assertions $A$, $A'$, variables $\overline y$, and $\overline x$, specifications $S$, $S'$, $S''$, and module $M$:
%%\begin{itemize} [topsep=6pt,itemsep=5pt,parsep=0pt,partopsep=0pt]
%%%\item
%%% $\stronger M {\OneStateQ {\overline {x:C}}  {A}}  {\TwoStatesQ {\overline {x:C}} {A}{A}} $ 
%%%    \item
%%%  $\strongerEq  M  {\OneStateQ    {y:\prg{Object}}   {\forall \overline {x:C}[ A ] } } 
%%%    {\OneStateQ {\overline {x:C}}  {A}} $.
%%\item
%%$\strongerEq M    {\TwoStatesQ {\overline {x:C}} {A}{A'}}    {\TwoStatesQ {\overline {y:C}} {A[y/x]}{A'[y/x]}}$
%%\item
%%$  M  \models    \overline {x:C} \wedge A_1'  \rightarrow A_1$ \ \ \  and \ \ \
%%$  M  \models  \overline {x:C} \wedge A_2'  \rightarrow A_2$  \ \ \ \ 
%%implies\\
%% $\strut \hspace{5cm} \stronger M  {\TwoStatesQ {\overline {x:C}} {A_1}{A_2}}     {\TwoStatesQ {\overline {x:C}} {A_1'}{A_2'}}$
%%
%%\item
%%$\stronger M  S {S''}$ and $\stronger M {S''} {S'}$\ \  \ implies\  \ \ $\stronger M S  {S'}$.
%%
%%\end{itemize}
%%
%%\end{lemma}


%%%%%%%%%%%%%%%%%%%%%%%%%%%%%%%%%%%%%%%%%%%%%%%%%%%%%%%%%%%%%%%%%%%%%%


