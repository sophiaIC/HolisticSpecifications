
\newcommand{\paragraphSD}[1]{\vspace{.1cm}\noindent{\emph{#1}}}

 % \subsection{Summary: Our Motivation  and Our Work}

\paragraphSD{Our motivation}
%Our work is motivated by 
comes from the OCAP approach to security, whereby object capabilities guard against un-sanctioned effects.
Miller \cite{miller-esop2013,MillerPhD} advocates
 \emph{defensive consistency}: whereby %\textit{``
 {``An object is defensively
  consistent when it can defend its own invariants and provide correct
  service to its well behaved clients, despite arbitrary or malicious
  misbehaviour by its other clients.''}  Defensively consistent
modules  are  hard to design % write,  understand,  -- Shorter, and let us not be too negtive and
 and  verify, but
make it much
easier to make guarantees about systems composed of multiple components
\cite{Murray10dphil}.

 \paragraphSD{Our Work} aims to elucidate such guarantees. We want to formalize and prove  that 
%  the guarantee that 
\cite{permissionAuthority}:
\begin{quote}
\emph{Lack of eventual access implies that certain properties will be preserved, even in the presence of external calls.}
\end{quote}
For this, we had  to  model the concept of  lack of eventual access,  determine the temporal scope of the preservation, and  develop a Hoare logic framework to formally prove such guarantees.

% To model
For lack of eventual access,  we introduced protection, which is a property of all  the paths of all external objects accessible from the current stack frame.
%  To determine the 
For the  temporal scope of preservation, we developed scoped invariants, which ensure that a given property holds as long as we have not returned from the current method (top of current stack has not been popped yet). 
 For our Hoare logic, we introduced an adaptation operator, which translates assertions between the caller’s and callee’s frames. 
 Finally, to prove the soundness of our approach, we developed the notion of scoped satisfaction,  which mandates that an assertion must be satisfied from a particular stack frame onward. 
 Thus, most concepts in this work are  \emph{scope-aware}, as they depend  on the current stack frame.
 
 With these concepts, we have developed a specification language for modules \taming effects, a Hoare Logic for proving external calls, protection, and adherence to specifications, and have proven it sound. % and have applied it to our running Example. 

% \subsection{Related Work}
\paragraphSD{Lack of Eventual Access}   Efforts to restrict ``eventual access'' have been extensively explored, with
 Ownership Types  being a prominent example \cite{simpleOwnership,existOwn}.
These types enforce encapsulation boundaries to safeguard internal implementations, thereby ensuring representation independence and defensive consistency
\cite{ownalias,NobPotVitECOOP98,Banerjee:2005}.
%and thus 
% By ensuring confinement, ownership systems  
%enforce representation independence, \sdN{and  % (a property close to
%defensive consistency} % )
%\cite{Banerjee:2005}.
Ownership is fundamental to key systems like Rust’s memory safety  
\cite{RustPL2,RustBelt18},
Scala's Concurrency \cite{ScalaCapabilities,ScalaLightweightAffine},
Java heap analyses \cite{PotterNC98,HillNP02,MitECOOP06}, 
and   plays a critical role in program verification
\cite{BoyLisShrPOPL03,hypervisor} including Spec$\#$
\cite{BarLeiSch05,BarDelFahLeiSch04} and universes
\cite{DieDroMue07,DietlMueller05,LuPotPOPL06},
Borrowable Fractional Ownership \cite{borrow-fract-vmcai2024},
and recently integrated into languages like OCAML \cite{ocaml-ownership-icfp2024,funk-ownership-oopsla2024}.

\sdN{Ownership types are closely related to the notion of protection: both are scoped relative to a frame. However, ownership requires an object to control some part of the path, while protection demands that module objects control the endpoints of paths. }

\sdN{In future work we want to   explore how to express protection within Ownership Types, with the primary challenge being how to accommodate for capabilities accessible to some external objects while still   inaccessible to others. }
Moreover,  tightening some
rules in our current Hoare logic (e.g.\ Def. \ref{def:chainmail-protection-from})
may lead to a native  Hoare logic of ownership.
Also, recent approaches like
%
%% Whereas ownership types typically impose a geometric structure 
%% on references over the whole heap, a group of recent approaches have 
%% focussed on a local view of the sharing or accessibility of objects.
%
the Alias
Calculus \cite{meyer-alias-calculus-scp2015,meyer-auto-alias-sncs2020},
Reachability
Types \cite{romf-reachability-types-oopsla2021,rompf-poly-reachability-oopsla2024}
and Capturing
Types \cite{odersky-capturing-types-toplas2023,scoped-effects-oopsla2022,odersky-reach-prog2024}
abstract fine-grained method-level descriptions of 
references and aliases flowing into and out of methods and fields,
and likely accumulate enough information to express 
protection. Effect exclusion
\cite{fx-exclusion-icfp2023} directly prohibits nominated
effects, but within a closed, fully-typed world.



 
\paragraphSD{Temporal scope of the guarantee} Starting with loop invariants\cite{Hoare69,Floyd67}, property preservation at various granularities and durations has been widely and successfully adapted and adopted \cite{Hoare74,liskov94behavioral,usinghistory,Cohen10,Meyer92,MeyerDBC92,BarDelFahLeiSch04,objInvars,MuellerPoetzsch-HeffterLeavens06,DrossoFrancaMuellerSummers08}.
In our work, the temporal scope of the preservation guarantee includes all nested calls, until termination of the currently executing method, but not beyond. 
We compare with object and history invariants in \S \ref{sect:bounded}.

Such guarantees are maintained by the module as a whole.
\citet{FASE}  proposed ``holistic specifications'' which take an external
perspective across the interface of a module. % one or more modules.
%% (compared with pre- and post- conditions that supply internal
%% specifications for individual method implementations, and class,
%% monitor, or module invariants that make internal specifications for all
%% the methods in a single class, monitor, or module.
\citet{OOPSLA22} builds upon this work, offering a specification
language based on \emph{necessary} conditions and temporal operators.
 %  rather than sufficient
% conditions as in most specification languages, and a % Hoare NOT A HOARE logicc to 
% that can prove modules  adhere to such specifications. 
Neither of these systems support any kind of external calls.
\sdN{Like \cite{FASE,OOPSLA22} we propose ``holistic specifications'', albeit without temporal logics, and with sufficient conditions.
In addition, we introduce protection, and develop a Hoare logic for protection and external calls.} 


\paragraphSD{Hoare Logics} were first developed in Hoare's seminal 1969 paper \cite{Hoare69}, and have inspired a plethora of influential further developments and tools. We shall discuss a few only.

\sdN{Separation logics  \cite{IshtiaqOHearn01,Reynolds02}  % facilitate 
 reason  about disjoint memory regions. %, enabling more modular and scalable verification. 
% The separating conjunction focuses on the disjointness of assertions' footprints rather than eventual access. 
Incorporating Separation Logic's powerful framing mechanisms  will pose several challenges: 
We have no specifications and no footprint for external calls. 
Because protection is ``scope-aware'', expressing it as a predicate would require quantification over all possible paths and variables within the current stack frame.
We may also require a  new separating conjunction \susan{operator}.
%We want to explore these in further work.  % that would apply to protection.  
}
%
Hyper-Hoare Logics \cite{hyper-hoare-pldi2024,compositional-hypersafety-oopsla2022}  reason about the execution of several programs, and  \sdN{could thus be applied to our problem, if extended to model all possible  
sequences of calls of internal public methods.}

 Incorrectness Logic
\cite{IncorrectnessLogic}
% draws on Reverse Hoare Logic \cite{reverseHoare}. 
under-approximates  postconditions, and thus
reasons about the presence of bugs, rather than their absence.
%Incorrectness Logic ensures all states that satisfy a
%postcondition are reachable from the precondition,
%excluding false negatives.
%
Our work, like classical Hoare Logic, over-approximates postconditions,
 and differs from Hoare and Incorrectness Logics
by tolerating interactions between verified code and unverified components.
% code and
% untrusted, unknown, potentially antagonistic components.
\sdN{Interestingly, even though earlier  work in the space \cite{FASE,OOPSLA22} employ \emph{necessary} conditions for effects (\ie under-approximate pre-conditions), we can, instead, employ \emph{sufficient} conditions for the lack of effects (over-approximate postconditions).
Incorporating our work into  Incorrectness Logic might require  under-approximating  eventual access, while protection over-approximates it.}

Rely-Guarantee
\cite{relyGuarantee-HayesJones-setss2017,relyGuarantee-vanStaden-mpc2015}
and Deny-Guarantee \cite{DenyGuarantee} %reasoning techniques
distinguish between assertions guaranteed by a thread, and those a
thread can reply upon. 
% Assume-guarantee reasoning, especially circula
% assume-guarantee  \cite{circular-assume-guarantee-fm2015} evinces
% the main reason callbacks are hard to handle: circular control flow can lead
% all too easily towards circular reasoning.
Our Hoare quadruples are (roughly) Hoare triples plus 
the ``guarantee'' portion of rely-guarantee.
When a
specification includes a guarantee, that guarantee must be maintained
by every ``atomic step'' in an execution
\cite{relyGuarantee-HayesJones-setss2017}, rather than just at method
boundaries as in visible states semantics
\cite{MuellerPoetzsch-HeffterLeavens06,DrossoFrancaMuellerSummers08,considerate}.
In concurrent reasoning,  
this is because shared state may be accessed
by another co{\"o}perating thread at any time:
while in our case, it is because unprotected
state may be accessed by an untrusted component within the same
thread.  
%SD chopped below. I think it was already clear
%Guarantees correspond to our
%properties that must be preserved by all code linked to the current
%module. Deny-guarantee assumes
%co{\"o}peration:
%%composition is legal only if threads adhere to their guarantees,
%while we use these ``guarantees'' precisely to
%ensure our code can interoperate safely with external untrusted code.
%%irrespective of whatever the untrusted code does.

\paragraphSD{Models and Hoare Logics for the interaction with the the external world}
\citet{Murray10dphil} made the first attempt to formalise defensive
consistency, % SD chop, is that crucial? space! based on counterfactual causation~\cite{Lewis_73}
to tolerate interacting with any untrustworthy object,
although without a specification language for describing effects
(i.e.\ when an object is correct).

 \citet{CassezFQ24} propose one approach to reason about external calls.
Given that external callbacks are necessarily restricted to the module's public interface,
external callsites are replaced  with a
generated \texttt{externalcall()} method that  nondeterministically invokes any method in that interface.
\citet{iris-wasm-pldi2023}'s Iris-Wasm is similar.
WASM's
modules are very loosely coupled: a module
has its own byte memory
and object table.
Iris-Wasm ensures models 
can only be
modified via their explicitly exported interfaces. 
 
 

\citet{ddd}  designed OCPL, a logic
that separates internal implementations (``high values'')
%which must not be exposed to attacking code)
from interface objects
(``low values''). %
%which may be exposed).
OCPL supports defensive
consistency % (Swasey et al.\ use 
%(\sd{they} use the term
(called ``robust safety'' after the
security literature \cite{Bengtson})
by ensuring
%via a proof system that ensures
low values can never leak high values, a % to external attackers. 
 %\susan{How does this imply that high values can be exposed?}
%\james{typo fixed: it's LOW values that can be exposed}
%% This means that low values \textit{can} be exposed to external code,
%% and the behaviour of the system is described by considering attacks only
%% on low values.  %OCPL is a program logic, and Swasey
%% \sd{They}
and 
%use that logic to
prove %a number of
object-capability patterns, such as
sealer/unsealer, caretaker, and membrane.
%
%This work was then developed to prove the memory properties of Rust in
%the
%
RustBelt \cite{RustBelt18}
developed this approach to prove Rust memory safety using Iris \cite{iris-jfp2018},
and combined with RustHorn \cite{RustHorn-toplas2021} for the safe subset,
%(which verifies the safe subset of Rust \cite{RustHorn-toplas2021}),
produced RustHornBelt \cite{RustHornBelt-pldi2022} that verifies
both safe and unsafe Rust programs. % in classical first-order logic.
Similar techniques were extended to C \cite{RefinedC-pldi2021}.
%note that while C is not ``memory-safe'', neither is unsafe Rust.
%Compared to the work we present in this paper,
While these projects 
verify ``safe'' and ``unsafe'' code, 
%
%both projects have
%rather different interpretations of ``safety''.
the distinction is about memory safety:% only:
%both kinds of code are veridied.
%
%their ``unsafe'' code does 
%
%code that is not known to the verification system,
%is trusted, and indeed will be verified, but for which the language
%does not enforce its characteristic memory constraints,
%
whereas all our code is ''memory safe''
but unsafe / untrusted code is unknown to the verifier.

% devrise birkedal S&P 2016; 
\citet{dd} deploy step-indexing, Kripke worlds, and representing objects
as public/private state machines to model problems including the 
DOM wrapper and a mashup application.
Their distinction between public and private transitions %being related 
is similar  to our
distinction between internal and external objects.
This stream of work has culminated in VMSL, an Iris-based separation logic for
virtual machines to assure defensive consistency
\cite{vmsl-pldi2023}
% SD removed the joke
%(sadly not a logic for VAX/VMS systems),
%
%``robust safety'' (more or less defensive consistency),
%
and Cerise, which uses Iris invariants to support proofs of programs with outgoing calls and callbacks,
on capability-safe CPUs \cite{cerise-jacm2024},
via problem-specific proofs in Iris's logic.
Our work differs from Swasey, Schaefer's, and Devriese's work in that
they are primarily concerned   with  ensuring defensive consistency, 
while we focus on \sdN{module specifications}.

\paragraphSD{Smart Contracts} also pose the problem of external calls.
Rich-Ethereum \cite{rich-specs-smart-contracts-oopsla2021}
relies on Ethereum contracts' fields being instance-private
and unaliased. % , while a per-contract ``finished'' flag manages callbacks and termination.
Scilla \cite{sergey-scilla-oopsla2019}
is a minimalistic functional alternative to Ethereum,
% based on restricted actor-style communication, restricting recursion, and ensuring termination,
which has demonstrated that popular Ethereum
contracts avoid common contract errors \susan{when using Scilla}.
%
%SD do not knoiw where this was supposed to be.
% All these appproaches depend on
%environments which support encapsulation:
%we show how unrestricted references to mutable objects can be protected
%%by the emergent
%%behaviour of a module's code, i.e.\
%by programmer discipline,
%and that disciplline trusted, then verified.


The VerX tool can verify
specifications for Solidity contracts automatically \cite{VerX}.
VerX's specification language is based on temporal logic.
% ,  but only within a past modality, while e.g.\ \citet{OOPSLA22} has both past and future modalities.
It % Unlike Cerise, which handles callbacks, VerX
is restricted to ``effectively call-back free'' programs
\cite{Grossman,relaxed-callbacks-ToDES},
delaying any callbacks until the
incoming call to the internal object has finished.
% , turning a callback into a more straightforward  incoming call. 
 
\textsc{ConSol} \cite{consolidating-pldi2024}
provides a specification langauge for smart contracts,
checked at runtime \cite{FinFel01}.
% ,  with case studies  showing it could prevent some famous smart contract errors.
SCIO* \cite{secure-io-fstar-popl2024}, implemented in
F*, supports both
verified and unverified code.
% : unverified code  corresponds to our external modules. 
%Like \textsc{ConSol}, SCIO* uses runtime checks
%(higher-order contracts and reference monitors)
%to ensure unverified
%code cannot break verified code.
Both \textsc{Consol} and SCIO* are 
similar to gradual verification techniques 
\cite{gradual-verification-popl2024,Cok2022} that
insert dynamic checks between verified and unverified code,
and contracts for general access control 
\cite{DPCC14,AuthContract,cedar-oopsla2024}.




\paragraphSD{Programming languages with object capabilities}
Google's Caja \cite{Caja} applies (object-)capabilities \cite{Dennis66,JamesMorris,MillerPhD}, 
sandboxes, proxies, and wrappers to limit components'
access to \textit{ambient} authority.
Sandboxing has been validated
formally \cite{mmt-oakland10};
Many recent languages % and web systems
\cite{CapJavaHayesAPLAS17,CapNetSocc17Eide,DOCaT14} including Newspeak
\cite{newspeak17}, Dart \cite{dart15}, Grace \cite{grace,graceClasses}
and Wyvern \cite{wyverncapabilities} have adopted object capabilities.
\citet{schaeferCbC} has also adopted
an information-flow approach to ensure confidentially by construction.

\citet{stack-safety-csf2023} extend memory safety arguments to ``stack
safety'': ensuring method calls and returns are well bracketed (aka
``structured''), and that the integrity and confidentially of both
caller and callee are ensured, by assigning objects to security
classes. 
%While in our paper we aim to specify and prove higher-level
%properties: some of the concepts are parallel: their ``sealed''
%objects are close to our objects accessible from earlier stack fames
%(but no longer accessible from the top fame)e, and their ``public''
%objects are close to our external objects.
\citet{schaeferCbC} has also adopted
an information-flow approach to ensure confidentially by construction.


  \forget{
\paragraph{Future work} THERE AREN"T REALLY QUESTIONS ABOVE includes the questions mentioned above, and also
\sdN{the investigation of more ways to approximate eventual access, footprint and framing operators}, and   
the application of these techniques to
languages that rely on lexical nesting for access
control such as Javascript \cite{ooToSecurity},
rather than public/private annotations;
languages that support ownership types that can be leveraged for
verification
\cite{leveragingRust-oopsla2019,RustHornBelt-pldi2022,verus-oopsla2023};
and languages from the
functional tradition such as OCAML, which is gaining imperative
features such as ownership and uniqueness \cite{funk-ownership-oopsla2024,ocaml-ownership-icfp2024}. 
%
%Similarly there is potential in applying this approach within
%proof systems based e.g.\ on separation logic \cite{cerise-jacm2024},
%and post-Hoare logics like incorrectness logic \cite{IncorrectnessLogic} or
%hyperlogics \cite{compositional-hypersafety-oopsla2022,hyper-hoare-pldi2024},
%and to verifying (and ``disentangling'') concurrent programs \cite{seplog-disentanglelment-popl2024}.
%
We expect our techniques can be incorporated into existing program
verification tools \cite{Cok2022}, especially those attempting
gradual verification \cite{gradual-verification-popl2024},
thus paving the way towards practical verification for
the open world.
}
\paragraph{Future work} 

\susan{
We are interested in  looking at  the application of our techniques to
languages that rely on lexical nesting for access
control such as Javascript \cite{ooToSecurity},
rather than public/private annotations,
languages that support ownership types such as Rust, that can be leveraged for
verification
\cite{leveragingRust-oopsla2019,RustHornBelt-pldi2022,verus-oopsla2023},
and languages from the
functional tradition such as OCAML, which are gaining imperative
features such as ownership and uniqueness\cite{funk-ownership-oopsla2024,ocaml-ownership-icfp2024}. 
These different language paradigms may lead us to refine our ideas for eventual access, footprints and framing operators.
}

We expect our techniques can be incorporated into existing program
verification tools \cite{Cok2022}, especially those attempting
gradual verification \cite{gradual-verification-popl2024},
thus paving the way towards practical verification for
the open world.


%%%% %%%% %%%% %%%% %%%% %%%% %%%% %%%% %%%% %%%% %%%% %%%% %%%% %%%% 
%%%% %%%% %%%% %%%% %%%% %%%% %%%% %%%% %%%% %%%% %%%% %%%% %%%% %%%% 
