\section{Appendix to Section \ref{s:preserve} -- Preservation of Satisfaction of Assertions}
\label{app:preserve}

\sdN{
\noindent
\textbf{Proof of Lemma 
\ref{lemma:addr:expr}}

Take any $M$ $A$, $\sigma$

\begin{enumerate}
\item
To show that  $ \satisfiesA{M}{\sigma}{A}\ \ \ \ \Longleftrightarrow \ \ \ \ \ \satisfiesA{M}{\sigma}{A[{\interpret \sigma x}/x]} $ 

The proof goes by induction on the structure of $A$,   application of  Defs.  \ref{def:chainmail-semantics}, \ref{def:chainmail-protection-from}, and  \ref{def:chainmail-protection}, and auxiliary lemma \ref{aux:lemma:vars:eval}.

%\item Assume that   $\satisfiesA{M}{\sigma}{A[{\interpret \sigma x}/x]} $. To show that  $ \satisfiesA{M}{\sigma}{A}$. 

\item
To show that $\eval{M}{\sigma}{{\re}}{\alpha}  \ \ \Longrightarrow\ \  [ \ \satisfiesA{M}{\sigma}{A} \  \Longleftrightarrow\   \  \satisfiesA{M}{\sigma}{A[\alpha/\re]} \  \  ]$

The proof goes by induction on the structure of $A$,   application of  Defs.  \ref{def:chainmail-semantics}, \ref{def:chainmail-protection-from}, and  \ref{def:chainmail-protection}.

\item
To show that $ \satisfiesA{M}{\sigma}{A}   \ \ \ \Longleftrightarrow\ \ \ \satisfiesA{M}{\sigma[\prg{cont}\mapsto stmt]}{A}$ 

The proof goes by induction on the structure of $A$,   application of  Defs.  \ref{def:chainmail-semantics}, \ref{def:chainmail-protection-from}, and  \ref{def:chainmail-protection}.
\end{enumerate}

\noindent
\textbf{End of Proof}
}
\subsection{Preservation across method call and return}

%The proof of lemma \ref{lemma:addr:expr} is by induction on the structure of $A$.

We first give complete definitions for the concepts of $  \Stable {\_]}$ and $\Pos {\_}$

\vspace{.2cm}

% An assertion is {\emph {stable}}, written as $\Stable A$ if it does not contain $\inside {\re}$ assertions:
\begin{definition}
\label{def:Basic}
%\begin{figure}[hbt]
[$\Stable{\_}$] assertions: %  defined below

$
\begin{array}{l}
 \begin{array}{lll}
  \Stable {\inside{\re}}  \triangleq  false & \lclSPACE &   \Stable {\protectedFrom \re {\overline {u}}} =  
  \Stable  {\internal \re} =  %   \PushAS y  {(\external \re)} & \triangleq &   {\external \re}
    \Stable {\re}=   
     \Stable {\re:C}\   \triangleq \    true
 \end{array}
  \\
 \begin{array}{lcl}
 \Stable  {A_1  \wedge  A_2}\  \triangleq\     \Stable  { A_1}  \wedge    \Stable  {A_2}    &
\lclSPACE  &  
 \Stable  {\forall x:C.A} =\Stable  {\neg A} \   \triangleq\   \Stable A
 \end{array}
 \end{array}
$
\label{f:Basic}
 \end{definition}


 \begin{definition}
% \begin{figure}[hbt]
[$\Pos{\_}$] assertions: %  defined below

$
\begin{array}{l}
 \begin{array}{l}
  \Pos {\inside{\re}} =  \Pos {\protectedFrom \re {\overline {u}}} =  
  \Pos  {\internal \re} =   
    \Pos {\re}=   
     \Pos {\re:C}\   \triangleq \    true
 \end{array}
  \\
 \begin{array}{lll}
 \Pos  {A_1  \wedge  A_2}\  \triangleq\     \Pos  { A_1}  \wedge    \Pos  {A_2}   \  &    
 \Pos  {\forall x:C.A}   \triangleq\   \Pos A
\   &
  \Pos {\neg A}  \triangleq \Stable A % \Neg A
 \end{array}
 \end{array}
 $
 \label{def:Pos}
\end{definition}

%\begin{definition}
%[$\Neg{\_}$] assertions:
%
%$
%\begin{array}{l}
% \begin{array}{lll}
%  \Neg {\inside{\re}}\  \triangleq \    false &    \Neg {\protectedFrom \re {\overline {u}}} =  
%  \Neg  {\internal \re} =   
%    \Neg {\re}=   
%     \Neg {\re:C}\   \triangleq \    true
%\\ 
% \Neg  {A_1  \wedge  A_2}\  \triangleq\     \Neg  { A_1}  \wedge    \Neg  {A_2}   \ \   
% &    
% \Neg  {\forall x:C.A}   \triangleq\   \Neg A \ \ 
%&
%  \Neg {\neg A}  \triangleq \Pos A
% \end{array}
% \end{array}
%$
%\label{def:Neg}
%\end{definition}

The definition  of $\Pos{\_}$
%  and $\Neg {\_}$ are
is  less  general than would be   possible. \Eg $(\inside {x} \rightarrow  x.f=4) \rightarrow xf.3=7$  does not satisfy our definition of $\Pos {\_}$.
We have given these less general definitions in order to simplify our proofs.

\sdN{
\noindent
\textbf{Proof of  lemma \ref{l:preserve:asrt}}
Take any  state  $\sigma$, frame  $\phi$,  assertion  $A$, such that $ fv(A)=\emptyset$:
 
 
\begin{itemize}
\item 
To show\ \ \ \ \ \ \   $\Stable{A}\ \ \ \  \Longrightarrow \ \  \ \  [\ \ M, \sigma \models A \ \ \Longleftrightarrow \ \  M,{\PushSLong \phi \sigma} \models A\ \ ]$

By induction on the structure of the definition of $\Stable{A}$.

\item 
$\Pos{A}\ \ \   \wedge \ \ \    Rng(\phi) \subseteq  \LRelevantO   \sigma \ \ \ \ \Longrightarrow $
\\
$\strut \hspace{2cm} \ \  \ \  [\ \ M, \sigma \models A \  \wedge \  M, {\PushSLong \phi \sigma} \models  \intThis \ \ \Longrightarrow \ \  M,{\PushSLong \phi \sigma} \models A\ \ ]$

By induction on the structure of the definition of $\Pos{A}$.
The only interesting case is when $A$ has the form $\inside {\re}$. Because 
$ fv(A)=\emptyset$, we know that $\interpret {\sigma} {\re}$=$\interpret {\PushSLong \phi \sigma}  {\re}$. Therefore, we assume that 
 $\interpret {\sigma} {\re} = \alpha$ for some $\alpha$, assume that $ M,  \sigma  \models \inside \alpha$, and want to show that  $ M,{\PushSLong \phi \sigma} \models \inside \alpha$. 
 Because  $Rng(\phi) \subseteq  \LRelevantO   \sigma$, we also obtain that
  $ {\PushSLong \phi \sigma}\ \Rightarrow \subseteq \LRelevantO  {\sigma}$.
  The rest follows by unfolding and folding Def. \ref{def:chainmail-protection}.
  
  \end{itemize}
 
 
 
\noindent
\textbf{End Proof}
}

\subsection{Viewpoints and Protection}

\begin{lemma}
\label{l:adapt:stable}

For all assertions $A$, and variables $\overline y$

\begin{itemize}
\item
$\Stable{ \PushAS y A}$
\end{itemize}

\end{lemma}

Proof by induction over the structure of $A$. 


\beginProof{lemma:push:ass:state}
Take  any state  $\sigma$, assertion $A$,  variables   $\overline y, \overline z$, disjoint with one another.
Assume that  $\fv(A)=\emptyset$
\begin{enumerate}
 \item
 
 The proof is by induction on the structure of $A$. 
 
 The interesting case is when $A\txteq \inside {\alpha}$ for some $\alpha$, and we outline it here:
Assume that $M, \sigma \models \PushAS {y} { \inside {\alpha}}$. 
Therefore, $\forall \alpha'\in \LRelevantO \sigma. M, \sigma \models \protectedFrom  \alpha {\alpha'}$.

Take $\sigma'=  \PushAS {y} {A}$. Then, $\forall \alpha$. The assumption 
\item
$M, { \PushSLong {(\overline y, \overline z)} {\sigma} } \models\  A \  \ \ \ \Longrightarrow  \ \ \ \ M,  \sigma \models  \PushAS  {y} {A}$
\end{enumerate}
\completeProof

\subsection{Encapsulation}

{
Proofs of adherence to {\SpecLang specifications  hinge on the expectation that some, 
specific, assertions cannot be invalidated unless some 
} internal (and thus known) computation took place. 
{We call such assertions   \emph{encapsulated}.}
}
We define the  judgement,  $M\ \vdash  \encaps{A}$, in terms of the judgment  $M; \Gamma \vdash \encaps A  ; \Gamma'$
which checks that any objets read  in the validation of $A$ are internaml.
We assume a judgment $M; \Gamma \vdash e :  \prg{intl}$ which says that in the context of $\Gamma$, the expression $e$ belongs to a class from $M$.
We also assume that the judgement $M; \Gamma \vdash e :  \prg{intl}$ can deal with ghistfields -- eg through appropriate annotations of the ghost methods.
Note that it is possible for $M; \Gamma \vdash \encaps {e}$ to hold and 
$M; \Gamma \vdash  e : \prg{intl}$ not to hold.


\begin{figure}[thb]
$
\begin{array}{l}
\begin{array}{lclcl}
\inferruleSDNarrow 
{~ \strut  {\sc{Enc\_1}}}
{  
\begin{array}{l}
M; \Gamma \vdash \re : \prg{intl} \\
M; \Gamma \vdash \encaps{\re}; \Gamma
\end{array}
}
{
M; \Gamma \vdash \encaps{\re.f}; \Gamma
}
& &
\inferruleSDNarrow 
{~ \strut  {\sc{Enc\_2}}}
{  
\\
M; \Gamma \vdash \encaps{\re}; \Gamma 
}
{
M; \Gamma \vdash \encaps{\inside{\re}}; \Gamma
}
& &
\inferruleSDNarrow 
{~ \strut  {\sc{Enc\_3}}}
{  
\\
M; \Gamma \vdash \encaps{\re}; \Gamma 
}
{
M; \Gamma \vdash \encaps{\re:C}; \Gamma
}
\\ \\ 
\inferruleSDNarrow 
{~ \strut  {\sc{Enc\_4}}}
{  
\\
M; \Gamma, {x:C} \vdash \encaps {A}; \Gamma' 
}
{
M; \Gamma \vdash \encaps {\forall {x:C}. A}; \Gamma
}
&   &
\inferruleSDNarrow 
{~ \strut  {\sc{Enc\_5}}}
{   
\begin{array}{l}
M; \Gamma \vdash \encaps{A}; \Gamma'\\  \Stable A
\end{array}
}
{
M; \Gamma \vdash \encaps{ \neg A}; \Gamma'  
}
&    &
\inferruleSDNarrow 
{~ \strut  {\sc{Enc\_6}}}
{  
\begin{array}{l}
M; \Gamma \vdash \encaps{A_1}; \Gamma''   \\
  M; \Gamma'' \vdash \encaps{ A_2}; \Gamma' 
  \end{array} 
}
{
M; \Gamma \vdash \encaps{A_1 \wedge A_2}; \Gamma'
}
\end{array}
\end{array}
$
\caption{The judgment $M; \Gamma \vdash \encaps  {A}; \Gamma'$}
\label{f:encaps}
\end{figure}

{An assertion $A$  is  encapsulated by a module $M$  if in all possible states which arise from execution of module $M$ with any other  module $\Mtwo$, the validity of $A$}  {can only be changed via computations internal to that module}.
 

\begin{definition}[An assertion $A$ is \emph{encapsulated} by module $M$] $~$ \\
\label{d:encaps:sytactic}
\begin{itemize}
\item 
$M \vdash \encaps{A}  \ \   \triangleq  \ \  \exists \Gamma.[\ M; \emptyset \vdash \encaps{A}; \Gamma\ ]$ \ \  as defined in Fig. \ref{f:encaps}.
 \end{itemize}
  \end{definition}
  


\noindent
\textbf{More on Def. \ref{d:encaps}} {If the definition \ref{d:encaps} used the more general execution, $\leadstoOrig  {M\madd\Mtwo}  {\sigma}{\sigma'}$, rather than the scoped execution,  $\leadstoBounded {M\madd\Mtwo}  {\sigma}{\sigma'}$, then fewer assertions would have been encapsulated.}
Namely, assertions like    $\inside {x.f}$ would not be encapsulated.
Consider, \eg, a heap $\chi$, with objects $1$, $2$, $3$ and $4$, where  $1$, $2$ are external, and $3$, $4$ are internal, and  $1$ has fields pointing to $2$ and $4$, and $2$ has a field pointing to $3$, and $3$ has a field $f$ pointing to $4$. Take  state $\sigma$=$(\phi_1\!\cdot\!\phi_2,\chi)$, where $\phi_1$'s receiver is $1$,  $\phi_2$'s receiver is $2$,   and there are no local variables. 
We have  $...\sigma\models \externalexec \wedge \inside {3.f}$. 
We  return from the most recent all, 
getting  $\leadstoOrig  {...}  {\sigma}{\sigma'}$ where $\sigma'=(\phi_1,\chi)$; and have   $...,\sigma'\not\models  \inside {3.f}$.

\begin{example}
\label{ex:not:encaps}
For an assertion $A_{bal}\  \triangleq\ a:\prg{Account}\wedge a.\prg{balance}=b$, % can only be invalidated through internal  methods.  %on internal objects.
and modules \ModB and  \ModC  from \S~\ref{s:outline}, we have  \ \ \ $\ModB\ \models\ \encaps{ A_{bal} }$, \ \ \ and \ \ \ $\ModB\ \models\ \encaps{ A_{bal} }$.
\end{example}


\begin{example} Assume   further modules, $\ModD$ and $\ModE$,  which  use ledgers mapping  accounts to their balances, and export functions that update this map. In  $\ModD$ the ledger is  part of the {internal} module, %\emph{not} protected, while 
while in $\ModE$ it is part of the  {external} module.
Then  \ \ $\ModD \ \not\models\encaps{ A_{bal}} $, \ \  and \ \ $\ModE  \models \encaps{ A_{bal}} $.
Note that in both $\ModD$ and $\ModE$, the term \prg{a.balance} is a ghost field. 
\end{example}

\begin{note} Relative protection % (a variable is protected from another one) 
is not encapsulated, (\eg $M \not\models {\encaps{\protectedFrom{x}{y}}}$), even though    absolute protection is
(\eg $M \models \encaps{\inside{x}}$).
Encapsulation of an assertion does not imply encapsulation of its negation; 
 for example,  $M \not\models {\encaps{\neg\inside{x}}}$.
\end{note}

\TODO{{\red{TODO}: Say that there are more ways to define the encaps judgement. ALSO: show the rule for the ghostfields. }}

