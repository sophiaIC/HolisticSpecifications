\subsection{Crowdsale}
\jm[]{\Nec is able to encode the motivating example of \citeasnoun{VerX}: 
an escrow smart contract that ensures that the contract may not be coerced to 
pay out or refund more money than has been raised.
The motivating \prg{Crowdsale} example consists of a \prg{Crowdsale} contract 
for crowd sourcing funding. A \prg{Crowdsale} object consists of an \prg{Escrow} object,
an amount raised, a funding goal, and a closing time in which the goal must be met for 
the fund to be successful. An \prg{Escrow} consists of a ledger of investors and how much
they have invested. There are several properties that \citeasnoun{VerX} sought to encode,
and we have provided the encoding of those specifications in Fig. \ref{f:verx:encoding}.
\prg{R0} states that if an investor claims a refund from an escrow, then the balance of 
the escrow decreases by the amount the investor had deposited in the escrow. 
\prg{R1} states that if at anytime the escrow has not yet succeeded, then the deposits must
be less than the balance of the escrow. 
\prg{R2\_1} and \prg{R2\_2} combine to express a single property: no one may ever withdraw and 
then subsequently claim a refund or visa versa.
\prg{R3} states that if the funding goal is ever met, then no one may subsequently claim a refund.}

\begin{figure}[htb]
\begin{lstlisting}[language=chainmail]
class Crowdsale {
Escrow escrow;
  closeTime, raised, goal : int;
  method init() {
    if escrow == null
      escrow := new Escrow(new Object);
    	  closeTime := now + 30 days;
    	  raised := 0;
    	  goal := 10000 * 10**18;
  }
  method invest(investor : Object, value : int) {
    if raised < goal
      escrow.deposit(investor, value);
      raised += value;
  }
  method close() {
    if now > closeTime || raised >= goal
      if raised >= goal
        escrow.close();
      else
        escrow.refund();
  }
}
\end{lstlisting}
\caption{Crowdsale Contract}
\label{f:verx:crowdsale}
\end{figure}

\begin{figure}[htb]
\begin{lstlisting}[language=chainmail]
confined class Escrow {
  owner, beneficiary : Object;
  mapping(Object => uint256) deposits;
  OPEN, SUCCESS, REFUND : Object;
  state : Object;
  method init(o : Object, b : Object) {
    if owner == null || beneficiary == null
      owner := o;
      beneficiary := b;
      OPEN := new Object; SUCCESS := new Object; REFUND := new Object;
      state := OPEN;
      
  method close() {state = SUCCESS;}
  method refund() {state = REFUND;}
  method deposit(value : int, p : Object) {
    deposits[p] := deposits[p] + value;
  }
  method withdraw() {
    if state == SUCCESS
      return this.balance;
  }
  method claimRefund(p : Object) {
    if state == REFUND
      int amount := deposits[p];
      deposits[p] := 0;
      return amount;
  }
}
\end{lstlisting}
\caption{Escrow Contract}
\label{f:verx:escrow}
\end{figure}

\begin{figure}[htb]
\begin{lstlisting}[mathescape=true, language=chainmail]
(R0) $\triangleq$ e : Escrow $\wedge$ $\calls{\_}{\prg{e}}{\prg{claimRefund}}{\prg{p}}$
          next e.balance = nextBal onlyIf nextBal = e.balance - e.deposits(p)
(R1) $\triangleq$ e : Escrow $\wedge$ e.state $\neq$ e.SUCCESS $\longrightarrow$ sum(deposits) $\leq$ e.balance
(R2_1) $\triangleq$ e : Escrow $\wedge$ $\calls{\_}{\prg{e}}{\prg{withdraw}}{\prg{\_}}$
           to $\calls{\_}{\prg{e}}{\prg{claimRefund}}{\prg{\_}}$ onlyIf false
(R2_2) $\triangleq$ e : Escrow $\wedge$ $\calls{\_}{\prg{e}}{\prg{claimRefund}}{\prg{\_}}$
           to $\calls{\_}{\prg{e}}{\prg{withdraw}}{\prg{\_}}$ onlyIf false
(R3) $\triangleq$ c : Crowdsale $\wedge$ sum(deposits) $\geq$ c.escrow.goal
         to $\calls{\_}{\prg{c.escrow}}{\prg{claimRefund}}{\prg{\_}}$ onlyIf false
\end{lstlisting}
\caption{Encoding VerX Crowdsale Example in Necessity}
\label{f:verx:encoding}
\end{figure}
