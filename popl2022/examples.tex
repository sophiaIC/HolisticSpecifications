\section{Proof of Adherence to {{\prg{NecessityBankSpec}}}}
\label{s:examples}

In this section we return to the Bank Account example, 
providing a full proof and
the accompanying Coq formalism includes a mechanized version.


As we stated in Section \ref{sub:SpecO}, 
we assume the existence of a proof system for judgments
of the form $\proves{M}{A}$, denoting that in 
any arising program state, with internal module $M$, $A$ is satisfied. 
In this section, we make use of several rules that under such a logic should be sound.
We provide a description of these rules in Appendix \ref{app:assert_logic}. 
%\jm[]{Further, recall that as per Def. \ref{def:necessity-semantics},
%$\satisfies{M}{A}$ is defined for arising program states, and thus by Theorem \ref{thm:soundness},
%if $\proves{M}{A}$, it follows that for all arising program states in the context fo internal 
%module $M$, $A$ is satisfied.}

\begin{figure}[t]
\begin{lstlisting}[mathescape=true, frame=lines]
module Mod3
  class Password{}
  class Account
    field balance:int
    field pwd:Password
    method setPassword(pwd':Password, pwd'':Password):void
      {if(this.pwd==pwd') 
         this.pwd := pwd'}
	method transfer(pwd':Password, destAcc:Account)
	  {if(this.pwd==pwd' && this.balance > 100)
	     this.balance := this.balance - 100
	     destAcc.balance := destAcc.balance + 100}
\end{lstlisting}
\caption{Bank Account Module}
\label{f:ex-bank-short}
\end{figure}
We devote the rest of this section to the \jm[]{proof}  
\sophiaPonder[]{expressed in    \Nec logic
of a module's adherence to the Bank Account specification.}

\begin{lstlisting}[language=Chainmail, mathescape=true, frame=lines]
NecessityBankSpec $\triangleq$ from a:Account $\wedge$ a.Account=bal 
                     to a.balance < bal   onlyIf $\neg\wrapped{\prg{a.password}}$
\end{lstlisting}
That is, if the balance of an account ever decreases, it must be true that some object external to
\prg{Mod3} has access to the password of that account. 

A \prg{Bank} consists of a \prg{Ledger}, a method for transferring 
funds between accounts (\prg{transfer}), and a ghost field, \prg{balance}
for looking up the balance of an account at a bank.
%\footnote{A \Nec specification is independent of the implementation details of the code. It would need to hold for an account whose implementation did not use a ledger or ghost variables to hold balances.}
A \prg{Ledger} is
a mapping from \prg{Account}s to their balances. For brevity
our implementation only includes two accounts (\prg{acc1} and \prg{acc2}),
but it is easy to see how this could extend to a \prg{Ledger}
of arbitrary size. \prg{Ledger} is annotated as \enclosed, and as 
such the type system ensures our required encapsulation properties.
Finally, an \prg{Account} has some \prg{password} object, and 
methods to authenticate a provided password (\prg{authenticate}), 
change the password (\prg{changePass}).


\jm[]{Note, Figure \ref{f:ex-bank-short} does not provide the classical specifications of \prg{Mod4}, 
which can be found in full in Appendix \ref{app:BankAccount}. Informally, we introduce classical specifications
that state that 
\begin{description}
\item[(1)] no method returns the password, 
\item[(2)] the \prg{transfer} method in \prg{Ledger} results in a decreased balance to the \prg{from} \prg{Account},
\item[(3)] and the \prg{transfer} method in \prg{Bank} results in a decreased balance to the \prg{from} \prg{Account} \emph{only if} the correct password is supplied, and
\item[(4)] every other method in \prg{Mod4} never modifies any balance in any \prg{Bank}.
\end{description}}

\jm[]{While both the implementation and the specification being proven have changed from that of
\ref{s:outline}, the structure of the proofs do retain broad similarities. In particular the proof in this section 
follows the outline of our reasoning given in Sec. \ref{s:approach}}: we prove \jm[]{(1) encapsulation of \jm[]{the account balance and password}, 
(2) \emph{per-method} \Nec specifications on all \prg{Mod4} methods, (3) \emph{per-step} \Nec specifications for changing the balance and password,
and finally (4) the \emph{emergent} \Nec specification \prg{NecessityBankSpec}.}
%\begin{description}
%\item[Part 1:]
%\jm[]{We prove assertion \emph{encapsulation} for key assertions in the proof.}
%We prove that both \prg{b.getBal(a)=bal} and \prg{a.password=pwd} are encapsulated using the 
%encapsulation system laid out in Apdx. \ref{s:encap-proof}.
%\item[Part 2:]
%\jm[]{We prove \emph{per-method} \Nec specifications for \prg{Mod4}.}
%We use classical specifications to prove that only a call to \prg{Bank::transfer} 
%with the correct password may be used to decrease the balance of an account. Similarly, we
%use classical specifications to prove that no method can leak or illegally overwrite the password of an account.
%Note, it is this proof step that fails for \prg{Mod2}, as the password may be overwritten. Further, in the 
%proof, there is no distinction made between leaking and overwriting of the password, as both properties allow for external 
%objects to have no access to the password, but in the next moment have access to that password.
%\item[Part 3:]
%\jm[]{We prove \emph{per-step} \Nec specifications for \prg{Mod4}.}
%We combine the per-method necessary preconditions along with the encapsulation of \prg{b.getBal(a)=bal} to arrive at a per-step
%necessary precondition for reducing the balance using \emph{any} method in \jm[]{\prg{Mod4}}. Similarly, 
%we show that \emph{no} step may leak the password of an account.
%\item[Part 4:]
%\jm[]{Finally, we prove \emph{emergent} \Nec specifications.} We use our \Nec logic and the results of A, B, and C, to prove the emergent behavior specified in \prg{NecessityBankSpec}.
%\end{description}

%\clearpage
\subsection{Part 1: Assertion Encapsulation}
\label{s:BA-encap}
We base the soundness of our encapsulation of the type system of \Loo, and use the proof rules given in Figures \ref{f:intrnl} and \ref{f:asrt-encap}.
\jm[]{Informally, $\intrnl{e}$ indicates that any objects inspected during the evaluation of expression $e$ are internal. $\encaps{A}$ (see Section \ref{s:inference}) indicates 
that internal computation is necessary for a change in satisfaction of $A$. Rudimentary algorithms for proving $\intrnl{}$ and $\encaps{}$ are given in 
Appendix \ref{s:encap-proof}, and used here.}
We provide the proof for the encapsulation of \prg{b.balance(a)} below\\
%\begin{figure}[h]
\begin{proofexample}
\proofsteps{\prg{BalanceEncaps}}
	{\begin{proofexample}
		\proofsteps{\prg{aEnc}}
			{\proofstepwithrule
			{$\proves{\prg{Mod4}}{\givenA{\prg{b, b$^\prime$:Bank $\wedge$ a:Account $\wedge$ b.balance(a)=bal}}{\intrnl{\prg{a}}}}$}
				{by \textsc{Enc$_e$-Obj}}
		}
		\endproofsteps
	\end{proofexample}
		}
	{\begin{proofexample}
		\proofsteps{\prg{bEnc}}
			{\proofstepwithrule
			{$\proves{\prg{Mod4}}{\givenA{\prg{b, b$^\prime$:Bank $\wedge$ a:Account $\wedge$ b.balance(a)=bal}}{\intrnl{\prg{b}}}}$}
				{by \textsc{Enc$_e$-Obj}}
		}
		\endproofsteps
	\end{proofexample}
		}
	{\begin{proofexample}
		\proofsteps{\prg{getBalEnc}}
			{\proofstepwithrule
			{$\proves{\prg{Mod4}}{\givenA{\prg{b, b$^\prime$:Bank $\wedge$ a:Account $\wedge$ b.balance(a)=bal}}{\intrnl{\prg{b.balance(a)}}}}$}
				{by \prg{aEnc}, \prg{bEnc}, and \textsc{Enc$_e$-Ghost}}
		}
		\endproofsteps
	\end{proofexample}
		}
	{\begin{proofexample}
		\proofsteps{\prg{balEnc}}
			{\proofstepwithrule
			{$\proves{\prg{Mod4}}{\givenA{\prg{b, b$^\prime$:Bank $\wedge$ a:Account $\wedge$ b.balance(a)=bal}}{\intrnl{\prg{bal}}}}$}
				{by \textsc{Enc$_e$-Int}}
		}
		\endproofsteps
	\end{proofexample}
		}
		{\proofstepwithrule
			{
			$\proves{\prg{Mod4}}{\givenA{\prg{b, b$^\prime$:Bank $\wedge$ a:Account $\wedge$ b.balance(a)=bal}}{\encaps{\prg{b.balance(a)=bal}}}}$
			}{by \prg{getBalEnc}, \prg{balEnc}, \textsc{Enc-Exp}}}
\endproofsteps
\end{proofexample}\\
We omit the proof of $\encaps{\prg{a.password=pwd}}$, as its construction is very similar to that of $\encaps{\prg{b.balance(a)=bal}}$.
%\caption{Proof of encapsulation of \prg{b.getBal(a)=bal}}
%\end{figure}

\subsection{Part 2: Per-Method \Nec Specifications}
\label{s:BA-classical}
We now provide proofs for necessary preconditions on a per-method basis, leveraging 
classical specifications.
\jm[]{These proof steps are quite verbose, and for this reason, we only focus on proofs
of \prg{authenticate} from the \prg{Account} class.}

\jm[]{There are two \emph{per-method} \Nec specifications that we need
to prove of \prg{authenticate}: 
\begin{description}
\item[\textbf{\prg{AuthBalChange}}:] any change to the balance of an account may only occur if call to \prg{transfer} on the \prg{Bank} with the correct password is made. 
This may seem counter-intuitive as it is not possible to make two method calls (\prg{authenticate} and \prg{transfer}) at the same time, however we are able to prove this by first proving the 
absurdity that \prg{authenticate} is able to modify any balance.
\item[\textbf{\prg{AuthPwdLeak}}:] any call to \prg{authenticate} may only invalidate \wrapped{\prg{a.password}} (for any account \prg{a}) if \prg{false} is first satisfied -- clearly an absurdity.
\end{description}}

\paragraph{\emph{\textbf{\prg{AuthBalChange}}}}First we use the classical specification of the \prg{authenticate} method in \prg{Account} to prove that a call to \prg{authenticate} can only result in 
a decrease in balance in a single step if there were in fact a call to \prg{transfer} to the \prg{Bank}. This may seem 
odd at first, and impossible to prove, however we leverage the fact that we are first able to prove that \prg{false}
is a necessary condition to decreasing the balance, or in other words, it is not possible to decrease the balance by a
call to \prg{authenticate}. We then use the proof rule \textsc{Absurd} to prove our desired necessary condition.
This proof is presented as \prg{AuthBalChange} below.
\\
\noindent
{
	\begin{proofexample}
		\proofsteps{AuthBalChange}
			{\proofstepwithrule
				{\hoareEx
						{a, a$^\prime$:Account $\wedge$ b:Bank $\wedge$ b.balance(a$^\prime$)=bal}
						{a.authenticate(pwd)}
						{b.balance(a$^\prime$) == bal}
						}
					{by classical spec.}
			}
			{\proofstepwithrule
				{\hoareEx
						{a, a$^\prime$:Account $\wedge$ b:Bank $\wedge$ b.balance(a$^\prime$)=bal $\wedge$ $\neg$ false}
						{a.authenticate(pwd)}
						{$\neg$ b.balance(a$^\prime$) < bal}
						}
					{by classical Hoare logic}
			}
			{\proofstepwithrule
				{\onlyIfSingleExAlt
						{a, a$^\prime$:Account $\wedge$ b:Bank $\wedge$ b.balance(a$^\prime$)=bal $\wedge$ $\calls{\_}{\prg{a}}{\prg{authenticate}}{\prg{pwd}}$}
						{b.balance(a$^\prime$) < bal}
						{false}
						}
					{by \textsc{If1-Classical}}
			}
			{\proofstepwithrule
				{\onlyIfSingleExAlt
						{a:Account $\wedge$ a$^\prime$:Account $\wedge$ b:Bank $\wedge$ b.balance(a$^\prime$)=bal $\wedge$ $\calls{\_}{\prg{a}}{\prg{authenticate}}{\prg{pwd}}$}
						{b.balance(a$^\prime$) < bal}
						{$\calls{\_}{\prg{b}}{\prg{transfer}}{\prg{a$^\prime$.password, amt, a$^\prime$, to}}$}
						}
					{by \textsc{Absurd} and \textsc{If1-}$\longrightarrow$}
			}
		\endproofsteps
	\end{proofexample}
}

\paragraph{\emph{\textbf{\prg{AuthPwdLeak}}}} The proof of \prg{AuthPwdLeak} is given below, and is proven by application of classical Hoare logic rules and \textsc{If1-Inside}.

%We provide the statements of the specifications for the remaining methods in the module below,
%%(\textbf{changePassBalChange}, \textbf{Ledger::TransferBalChange}, and \textbf{Bank::TransferBalChange}), 
%but we elide the proof steps as they do not differ much from that of \textbf{AuthBalChange}.\\
%{
%	\begin{proofexample}
%		\proofsteps{changePassBalChange}
%			{\proofstepwithrule
%				{\onlyIfSingleEx
%						{a, a$^\prime$:Account $\wedge$ b:Bank $\wedge$ b.getBal(a$^\prime$)=bal $\wedge$ $\calls{\_}{\prg{a}}{\prg{changePass}}{\prg{pwd}}$}
%						{b.getBal(a$^\prime$) < bal}
%						{$\calls{\_}{\prg{b}}{\prg{transfer}}{\prg{a$^\prime$.password, amt, a$^\prime$, to}}$}
%						}
%					{by similar reasoning to \textbf{AuthBalChange}}
%			}
%		\endproofsteps
%	\end{proofexample}
%}
%\begin{minipage}{\textwidth}
%{
%	\begin{proofexample}
%		\proofsteps{Ledger::TransferBalChange}
%			{\proofstepwithrule
%				{\onlyIfSingleEx
%						{l:Ledger $\wedge$ a:Account $\wedge$ b:Bank $\wedge$ b.getBal(a)=bal $\wedge$ $\calls{\_}{\prg{l}}{\prg{transfer}}{\prg{amt, from, to}}$}
%						{b.getBal(a) < bal}
%						{$\calls{\_}{\prg{l}}{\prg{transfer}}{\prg{amt, from, to}}$}
%						}
%					{by \textsc{If-Start} and \textsc{If1-If}}
%			}
%			{\proofstepwithrule
%				{\onlyIfSingleEx
%						{l:Ledger $\wedge$ a:Account $\wedge$ b:Bank $\wedge$ b.getBal(a)=bal $\wedge$ $\calls{\_}{\prg{l}}{\prg{transfer}}{\prg{amt, from, to}}$}
%						{b.getBal(a) < bal}
%						{$\neg\wrapped{\prg{l}}$}
%						}
%					{by \textsc{If1-}$\longrightarrow$, \textsc{Caller-Ext}, and \textsc{Caller-Recv}}
%			}
%			{\proofstepwithrule
%				{\onlyIfSingleEx
%						{l:Ledger $\wedge$ a:Account $\wedge$ b:Bank $\wedge$ b.getBal(a)=bal $\wedge$ $\calls{\_}{\prg{l}}{\prg{transfer}}{\prg{amt, from, to}}$}
%						{b.getBal(a) < bal}
%						{\false}
%						}
%					{by \textsc{Intrnl-Wrapped} and \textsc{If1-}$\longrightarrow$}
%			}
%			{\proofstepwithrule
%				{\onlyIfSingleEx
%						{l:Ledger $\wedge$ a:Account $\wedge$ b:Bank $\wedge$ b.getBal(a)=bal $\wedge$ $\calls{\_}{\prg{l}}{\prg{transfer}}{\prg{amt, from, to}}$}
%						{b.getBal(a) < bal}
%						{$\calls{\_}{\prg{b}}{\prg{transfer}}{\prg{a.password, amt, a, to}}$}
%						}
%					{by \textsc{Absurd} and \textsc{If1-}$\longrightarrow$}
%			}
%		\endproofsteps
%	\end{proofexample}
%}
%\end{minipage}
%{
%	\begin{proofexample}
%		\proofsteps{Ledger::TransferBalChange}
%			{\proofstepwithrule
%				{\onlyIfSingleEx
%						{l:Ledger $\wedge$ a:Account $\wedge$ b:Bank $\wedge$ b.getBal(a)=bal $\wedge$ $\calls{\_}{\prg{l}}{\prg{transfer}}{\prg{amt, from, to}}$}
%						{b.getBal(a) < bal}
%						{$\calls{\_}{\prg{b}}{\prg{transfer}}{\prg{a.password, amt, a, to}}$}
%						}
%					{by similar reasoning to \textbf{\prg{AuthBalChange}}}
%			}
%		\endproofsteps
%	\end{proofexample}
%}
%{
%	\begin{proofexample}
%		\proofsteps{Bank::TransferBalChange}
%			{\proofstepwithrule
%				{\onlyIfSingleEx
%						{a:Account $\wedge$ b, b$^\prime$:Bank $\wedge$ b.getBal(a)=bal $\wedge$ $\calls{\_}{\prg{b$^\prime$}}{\prg{transfer}}{\prg{pwd, amt, from, to}}$}
%						{b.getBal(a) < bal}
%						{a == from $\wedge$ pwd == a.password $\wedge$ b$^\prime$ == b}
%						}
%					{by similar reasoning to \textbf{AuthBalChange}}
%			}
%			{\proofstepwithrule
%				{\onlyIfSingleEx
%						{a:Account $\wedge$ b:Bank $\wedge$ b.getBal(a)=bal $\wedge$ $\calls{\_}{\prg{b$^\prime$}}{\prg{transfer}}{\prg{pwd, amt, from, to}}$}
%						{b.getBal(a) < bal}
%						{$\calls{\_}{\prg{b}}{\prg{transfer}}{\prg{a.password, amt, a, to}}$}
%						}
%					{by \textsc{If1-}$\longrightarrow$}
%			}
%		\endproofsteps
%	\end{proofexample}
%}
%Below we provide the proofs for each method in \jm[]{\prg{Mod4}} that they cannot
%be used to leak the password of an account. \\
{
	\begin{proofexample}
		\proofsteps{AuthPwdLeak}
			{\proofstepwithrule
				{\hoareEx
						{a:Account $\wedge$ a$^\prime$:Account $\wedge$ a.password == pwd}
						{\prg{res}=a$^\prime$.authenticate(\_)}
						{res != pwd}
						}
					{by classical spec.}
			}
			{\proofstepwithrule
				{\hoareEx
						{a:Account $\wedge$ a$^\prime$:Account $\wedge$ a.password == pwd $\wedge$ $\neg$ false}
						{\prg{res}=a$^\prime$.authenticate(\_)}
						{res != pwd}
						}
					{by classical Hoare logic}
			}
			{\proofstepwithrule
				{\onlyIfSingleExAlt
						{$\wrapped{\prg{pwd}}$ $\wedge$ a, a$^\prime$:Account $\wedge$ a.password=pwd $\wedge$ $\calls{\_}{\prg{a}^\prime}{\prg{authenticate}}{\_}$}
						{$\neg \wrapped{\_}$}
						{false}
						}
					{by \textsc{If1-Inside}}
			}
		\endproofsteps
	\end{proofexample}
	}
%	{
%	\begin{proofexample}
%		\proofsteps{changePassLeak}
%			{\proofstepwithrule
%				{\onlyIfSingleEx
%						{$\wrapped{\prg{pwd}}$ $\wedge$ a, a$^\prime$:Account $\wedge$ a.password=pwd $\wedge$ $\calls{\_}{\prg{a}^\prime}{\prg{changePass}}{\_, \_}$}
%						{$\neg \wrapped{\prg{pwd}}$}
%						{false}
%						}
%					{by similar reasoning to \textbf{AuthPwdLeak}}
%			}
%		\endproofsteps
%	\end{proofexample}
%	}
%	{
%	\begin{proofexample}
%		\proofsteps{Ledger::TransferPwdLeak}
%			{\proofstepwithrule
%				{\onlyIfSingleExAlt
%						{$\wrapped{\prg{pwd}}$ $\wedge$ a:Account $\wedge$ l:Ledger $\wedge$ a.password=pwd $\wedge$ $\calls{\_}{\prg{l}}{\prg{transfer}}{\_, \_, \_}$}
%						{$\neg \wrapped{\prg{pwd}}$}
%						{false}
%						}
%					{by similar reasoning to \textbf{AuthPwdLeak}}
%			}
%		\endproofsteps
%	\end{proofexample}
%	}
%	{
%	\begin{proofexample}
%		\proofsteps{Bank::TransferPwdLeak}
%			{\proofstepwithrule
%				{\onlyIfSingleExAlt
%						{$\wrapped{\prg{pwd}}$ $\wedge$ a:Account $\wedge$ b:Bank $\wedge$ a.password=pwd $\wedge$ $\calls{\_}{\prg{l}}{\prg{transfer}}{\_, \_, \_, \_}$}
%						{$\neg \wrapped{\prg{pwd}}$}
%						{false}
%						}
%					{by similar reasoning to \textbf{AuthPwdLeak}}
%			}
%		\endproofsteps
%	\end{proofexample}
%	}

\subsection{Part 3: Per-Step \Nec Specifications}
The next step is to construct proofs of necessary conditions for
\emph{any} possible step in our external state semantics.
In order to prove the final result in the next section,
we need to prove three per-step \Nec specifications: \prg{BalanceChange}, \prg{PasswordChange}, and \prg{PasswordLeak}.
\begin{lstlisting}[language=Chainmail, mathescape=true, frame=lines]
BalanceChange $\triangleq$ from  a:Account $\wedge$ b:Bank $\wedge$ b.balance(a)=bal
                 next b.balance(a) < bal   onlyIf $\calls{\_}{\prg{b}}{\prg{transfer}}{\prg{a.password}, \_, \prg{a}, \_}$
                 
PasswordChange $\triangleq$ from a:Account $\wedge$ a.password=p
                  next $\neg$ a.password != p   onlyIf $\calls{\_}{\prg{a}}{\prg{changePass}}{\prg{a.password}, \_}$
                  
PasswordLeak $\triangleq$ from a:Account $\wedge$ a.password=p $\wedge$ inside<p>
                  next $\neg$ inside<p>   onlyIf false
\end{lstlisting}
\jm[]{We provide the proofs of these in Appendix \ref{app:BankAccount}, but describe the construction of the proof of \prg{BalanceChange} here:
by application of the rules/results
 \prg{AuthBalChange}, \prg{changePassBalChange}, \prg{Ledger::TransferBalChange}, \prg{Bank::TransferBalChange}, \prg{BalanceEncaps}, and \textsc{If1-Internal}.}
%
%by combining the results from \ref{s:BA-encap} and \ref{s:BA-classical} using \textsc{If1-Internal}. 
%Again, we elide the details of the proof of \prg{PasswordChange} and \prg{PasswordLeak} as they are similar to that
%of \prg{BalanceChange}. \\
%\noindent
%\begin{proofexample}
%\proofsteps{\prg{BalanceChange}}
%	{\proofstepwithrule
%			{\onlyIfSingleEx
%					{a:Account $\wedge$ b:Bank $\wedge$ b.getBal(a)=bal}
%					{b.getBal(a) < bal}
%					{$\calls{\_}{\prg{b}}{\prg{transfer}}{\prg{a.password, amt, a, to}}$}
%					}
%				{by \textbf{AuthBalChange}, \textbf{changePassBalChange}, \textbf{Ledger::TransferBalChange}, \textbf{Bank::TransferBalChange}, \textbf{BalanceEncaps}, and \textsc{If1-Internal}}
%		}
%\endproofsteps
%\end{proofexample}
%\begin{proofexample}
%\proofsteps{\prg{PasswordChange}}
%	{\proofstepwithrule{\onlyIfSingleEx
%				{a:Account $\wedge$ a.password=p}
%				{a.password $\neq$ p}
%				{$\calls{\_}{\prg{a}}{\prg{changePass}}{\prg{p}, \_}$}
%				}
%			{by similar reasoning to \textbf{BalanceChange}}
%	}
%\endproofsteps
%\end{proofexample}
%\begin{proofexample}
%\proofsteps{\prg{PasswordLeak}}
%	{\proofstepwithrule{\onlyIfSingleEx
%				{a:Account $\wedge$ a.password=p $\wedge$ $\wrapped{\prg{p}}$}
%				{$\neg \wrapped{\prg{p}}$}
%				{false}
%				}
%			{by similar reasoning to \prg{Balancechange}}
%	}
%\endproofsteps
%\end{proofexample}
\subsection{Part 4: Emergent \Nec Specifications}
Finally, we combine our module-wide single-step \Nec specifications to 
prove emergent behavior of the entire system. Informally the
reasoning used in the construction of the proof of \prg{NecessityBankSpec} can be stated as
\begin{description}
\item [(1)]
If the balance of an account decreases, then
by \prg{BalanceChange} there must have been a call
to \prg{transfer} in \prg{Bank} with the correct password.
\item [(2)]
If there was a call where the \prg{Account}'s password 
was used, then there must have been an intermediate program state
when some external object had access to the password.
\item [(3)]
Either that password was the same password as in the \jm[]{starting} 
program state, or it was different:
\begin{description}
\item [(Case A)]
If it is the same as the initial password, then since by \prg{PasswordLeak}
it is impossible to leak the password, it follows that some external object 
must have had access to the password initially.
\item [(Case B)]
If the password is different from the initial password, 
then there must have been an \jm[]{intermediate} program state when it 
changed. By \prg{PasswordChange} we know that this must have occurred
by a call to \prg{changePassword} with the correct password. Thus,
there must be a some \jm[]{intermediate} program state where the initial
password is known. From here we proceed by the same reasoning 
as \textbf{(Case A)}.
\end{description}
\end{description}
\begin{proofexample}
\proofsteps{\prg{NecessityBankSpec}}
	{\proofstepwithrule{\onlyThroughExAlt
				{a:Account $\wedge$ b:Bank $\wedge$ b.balance(a)=bal}
				{b.balance(a) < bal}
				{$\calls{\_}{\prg{b}}{\prg{transfer}}{\prg{a.password}, \_, \prg{a}, \_}$}
				}
			{by \textsc{Changes} and \prg{BalanceChange}}}
	{\proofstepwithrule{\onlyThroughExAlt
				{a:Account $\wedge$ b:Bank $\wedge$ b.balance(a)=bal}
				{b.balance(a) < bal}
				{$\exists$ o.[$\external{\prg{o}}$ $\wedge$ $\access{\prg{o}}{\prg{a.password}}$]}
				}
			{by $\longrightarrow$, \textsc{Caller-Ext}, and \textsc{Calls-Args}}}
	{\proofstepwithrule{\onlyThroughExAlt
				{a:Account $\wedge$ b:Bank $\wedge$ b.balance(a)=bal $\wedge$ a.password=pwd}
				{b.balance(a) < bal}
				{$\neg$$\wrapped{\prg{a.password}}$}
				}
			{by $\longrightarrow$}}
	{\proofstepwithrule{\onlyThroughEx
				{a:Account $\wedge$ b:Bank $\wedge$ b.balance(a)=bal $\wedge$ a.password=pwd}
				{b.balance(a) < bal}
				{$\neg$$\wrapped{\prg{a.password}}$ $\wedge$ (a.password=pwd $\vee$ a.password != pwd)}
				}
			{by $\longrightarrow$ and \textsc{Excluded Middle}}}
	{\proofstepwithrule{\onlyThroughEx
				{a:Account $\wedge$ b:Bank $\wedge$ b.balance(a)=bal $\wedge$ a.password=pwd}
				{b.balance(a) < bal}
				{($\neg$$\wrapped{\prg{a.password}}$ $\wedge$ a.password=pwd) $\vee$\\
				($\neg$$\wrapped{\prg{a.password}}$ $\wedge$ a.password != pwd)}
				}
			{by $\longrightarrow$}}
	{\proofstepwithrule{\onlyThroughExAlt
				{a:Account $\wedge$ b:Bank $\wedge$ b.balance(a)=bal $\wedge$ a.password=pwd}
				{b.balance(a) < bal}
				{$\neg$$\wrapped{\prg{pwd}}$ $\vee$
				a.password != pwd}
				}
			{by $\longrightarrow$}}
	{
	\begin{proofexample}
	\proofsteps{Case A ($\neg\wrapped{\prg{pwd}}$)}
			{\proofstepwithrule
				{\onlyIfExAlt
					{a:Account $\wedge$ b:Bank $\wedge$ b.balance(a)=bal $\wedge$ a.password=pwd}
					{$\neg$$\wrapped{\prg{pwd}}$}
					{$\wrapped{\prg{pwd}}\ \vee \neg\wrapped{\prg{pwd}}$}
					}
				{by \textsc{If-}$\longrightarrow$ and \textsc{Excluded Middle}}}
			{\proofstepwithrule{\onlyIfExAlt
					{a:Account $\wedge$ b:Bank $\wedge$ b.balance(a)=bal $\wedge$ a.password=pwd}
					{$\neg$$\wrapped{\prg{pwd}}$}
					{$\neg\wrapped{\prg{pwd}}$}
					}
				{by $\vee$E and \prg{PasswordLeak}}}
	\endproofsteps
	\end{proofexample}
	}
	{
	\begin{proofexample}
	\proofsteps{Case B (\prg{a.password != pwd})}
		{\proofstepwithrule{\onlyThroughExAlt
					{a:Account $\wedge$ b:Bank $\wedge$ b.balance(a)=bal $\wedge$ a.password=pwd}
					{a.password != pwd}
					{$\calls{\_}{\prg{a}}{\prg{changePass}}{\prg{pwd}, \_}$}
					}
				{by \textsc{Changes} and \textsc{PasswordChange}}}
		{\proofstepwithrule{\onlyThroughExAlt
					{a:Account $\wedge$ b:Bank $\wedge$ b.balance(a)=bal $\wedge$ a.password=pwd}
					{a.password != pwd}
					{$\neg\wrapped{\prg{pwd}}$}
					}
				{by $\vee$E and \prg{PasswordLeak}}}
		{\proofstepwithrule{\onlyIfExAlt
					{a:Account $\wedge$ b:Bank $\wedge$ b.balance(a)=bal $\wedge$ a.password=pwd}
					{a.password != pwd}
					{$\neg\wrapped{\prg{pwd}}$}
					}
				{by \textbf{Case A} and \textsc{Trans}}}
	\endproofsteps
	\end{proofexample}
	}
	{\proofstepwithrule{\onlyIfExAlt
				{a:Account $\wedge$ b:Bank $\wedge$ b.balance(a)=bal $\wedge$ a.password=pwd}
				{b.balance(a) < bal}
				{$\neg\wrapped{\prg{pwd}}$}
				}
			{by \textbf{Case A}, \textbf{Case B}, \textsc{If-}$\vee$I$_2$, and \textsc{If-}$\longrightarrow$}}
\endproofsteps
\end{proofexample}


%\subsection{ERC20}
%The ERC20 is a widely used token standard describing the basic functionality of any Ethereum-based token 
%contract. This functionality includes issuing tokens, keeping track of tokens belonging to participants, and the 
%transfer of tokens between participants. Tokens may only be transferred if there is sufficient tokens in the 
%participant's account, and if either they or someone authorized the participant initiated the transfer. We 
%specify these necessary conditions here using \SpecO and the Logic of Necessity. Firstly, \prg{ERC20Spec1} 
%says that if the balance of a participant's account is ever reduced by some amount $m$, then
%that must have occurred as a result of a call to the \prg{transfer} method with amount $m$ by the participant,
%or the \prg{transferFrom} method with the amount $m$ by some other participant.
%\begin{lstlisting}[language=Chainmail, mathescape=true, frame=lines]
%ERC20Spec1 $\triangleq$ from e:ERC20 $\wedge$ e.balance(p)=m + m' $\wedge$ m > 0
%              to1 e.balance(p)=m'
%              onlyIf $\exists$ p', p''.[$\calls{\prg{p}}{\prg{e}}{\prg{transfer}}{\prg{p', m}}$ $\vee$ $\calls{\prg{p''}}{\prg{e}}{\prg{transferFrom}}{\prg{p', m}}$]
%\end{lstlisting}
%Secondly, \prg{ERC20Spec2} specifies under what circumstances some participant \prg{p'} is authorized to 
%spend \prg{m} tokens on behalf of \prg{p}: either \prg{p} approved \prg{p'}, \prg{p'} was previously authorized,
%or \prg{p'} was authorized for some amount \prg{m + m'}, and spent \prg{m'}.
%\begin{lstlisting}[language=Chainmail, mathescape=true, frame=lines]
%ERC20Spec2 $\triangleq$ from e:ERC20 $\wedge$ p:Object $\wedge$ p':Object $\wedge$ m:Nat
%              to1 e.allowed(p, p')=m
%              onlyIf $\calls{\prg{p}}{\prg{e}}{\prg{approve}}{\prg{p', m}}$ $\vee$ 
%                     (e.allowed(p, p')=m $\wedge$ 
%                      $\neg$ ($\calls{\prg{p'}}{\prg{e}}{\prg{transferFrom}}{\prg{p, \_}}$ $\vee$ 
%                              $\calls{\prg{p}}{\prg{e}}{\prg{allowed}}{\prg{p, \_}}$)) $\vee$
%                     $\exists$ p''. [e.allowed(p, p')=m + m' $\wedge$ $\calls{\prg{p'}}{\prg{e}}{\prg{transferFrom}}{\prg{p'', m'}}$]
%\end{lstlisting}
%
%\subsection{DAO}
%The Decentralized Autonomous Organization (DAO) is a well-known Ethereum contract allowing 
%participants to invest funds. The DAO famously was exploited with a re-entrancy bug in 2016, 
%and lost \$50M. Here we provide specifications that would have secured the DAO against such a 
%bug. \prg{DAOSpec1} says that no participant's balance may ever exceed the ether remaining 
%in DAO.
%\begin{lstlisting}[language=Chainmail, mathescape=true, frame=lines]
%DAOSpec1 $\triangleq$ from d:DAO
%            to d.balance(p) > d.ether
%            onlyIf false
%\end{lstlisting}
%The second specification \prg{DAOSpec2} states that if a participant's balance is \prg{m}, then 
%either this occurred as a result of joining the DAO with an initial investment of \prg{m}, or the
%balance is \prg{0}, and they've just withdrawn their funds.
%\begin{lstlisting}[language=Chainmail, mathescape=true, frame=lines]
%DAOSpec2 $\triangleq$ from d:DAO
%            to1 d.balance(p)=m
%            onlyIf $\calls{\prg{p}}{\prg{d}}{\prg{repay}}{\prg{\_}}$ $\wedge$ m=0 $\vee$ $\calls{\prg{p}}{\prg{d}}{\prg{join}}{\prg{m}}$ $\vee$ d.balance(p)=m
%\end{lstlisting}
%
%\subsection{DOM}
%The Domain Object Model (DOM) is the representation of the objects comprising a web document.
%The DOM has a recursive tree structure.
%
%\prg{DOMSpec} states that if the property of a node in a DOM tree changes,
%it follows that either some non-node, non-wrapper object presently has 
%access to a node of the DOM tree, or to some wrapper with access to some 
%ancestor of the node that was modified.
%\begin{lstlisting}[language=Chainmail, mathescape=true, frame=lines]
%DOMSpec $\triangleq$ from nd:Node $\wedge$ n.property=p
%            to nd.property != p
%            onlyIf $\exists$ o.[$\neg$ o:Node $\wedge$ $\neg$ o:Wrapper $\wedge$ 
%                        ($\exists$ nd':Node.[$\access{\prg{o}}{\prg{nd'}}$] $\vee$ 
%                         $\exists$ w:Wrapper, k:$\mathbb{N}$.[$\access{\prg{o}}{\prg{w}}$ $\wedge$ nd.parnt$^{\prg{k}}$=w.node.parnt$^{\prg{w.height}}$] )]
%\end{lstlisting}







%\subsection{Bank Account 1 - Simple}
%\label{ex:bank1}
%\begin{lstlisting}[language=Chainmail, mathescape=true, frame=lines]
%BankSpec $\triangleq$  a:Account $\wedge$ a.balance=b to $\langle\ \neg$ a.balance $\neq$ b $\rangle$ only if $\neg\ \encapsulated{\prg{a}}$
%\end{lstlisting}
%We assume that the method \prg{deposit} conforms to the following 
%specification
%\begin{lstlisting}[language=Chainmail, mathescape=true, frame=lines]
%DepositSpec $\triangleq$  {a'.balance=b $\wedge$ a' $\neq$ a $\wedge$ a' $\neq$ from}
%					a.deposit(from, amt) 
%			   {a'.balance=b}
%\end{lstlisting}
%And all other methods \prg{m} in \prg{Account} conform to the 
%following spec
%\begin{lstlisting}[language=Chainmail, mathescape=true, frame=lines]
%AccountMethSpec $\triangleq$  {a:Account $\wedge$ a.balance=b}
%					  a.m(...) 
%			       {a.balance=b}
%\end{lstlisting}
%From \prg{DepositSpec} and \prg{AccountMethSpec} we can derive \prg{BASafety}
%\begin{lstlisting}[language=Chainmail, mathescape=true]
%BASafety $\triangleq$  a:Account $\wedge$ a.balance=b 
%            to a.balance != b
%            through Bank
%            onlyIf $\exists$ a'. [a':Account $\wedge$ $\calls{\_}{\prg{a'}}{\prg{deposit}}{\prg{a, \_}}$ $\vee$ $\calls{\_}{\prg{a}}{\prg{deposit}}{\prg{a'},\_}$
%\end{lstlisting}
%Then we assume that all methods $m$ in \prg{Account} observe the following specification:
%\begin{lstlisting}[language=Chainmail, mathescape=true, frame=lines]
%AccountEncapsulation $\triangleq$ {a:Account $\wedge$ $\encapsulated{\prg{a}}$}
%				          a.m(...)
%				       {$\encapsulated{\prg{a}}$}
%\end{lstlisting}
%And thus from \prg{AccoutnEncapsulation} and Definition \ref{def:module_necessary} we can derive \prg{BAEncapsulation}
%\begin{lstlisting}[language=Chainmail, mathescape=true, frame=lines]
%BAEncapsulation $\triangleq$  a:Account $\wedge$ $\changes{\I}{\encapsulated{\prg{a}}}$
%                 onlyIf false
%\end{lstlisting}
%%\begin{figure}[p]
%\begin{minipage}{\linewidth}
%\begin{proofBox}{white}{black}
%\footnotesize
%\begin{minipage}{0.75\textwidth}
%\begin{lstlisting}[language=Chainmail, mathescape=true, frame=single]
%a:Account $\wedge$ a.balance=b 
%  to $\langle\ \neg$ a.balance $\neq$ b $\rangle$ 
%  onlyThrough $\changes{\prg{\_}}{\prg{a.balance=b}}$
%\end{lstlisting}
%\end{minipage}
%\begin{minipage}{0.24\textwidth}
%\scriptsize
%\hfill by \textsc{Changes}
%\end{minipage}
%\begin{minipage}{0.75\textwidth}
%\begin{lstlisting}[language=Chainmail, mathescape=true]
%a:Account $\wedge$ a.balance=b $\wedge$ $\encapsulated{\prg{a.balance}}$ 
%  to $\langle\ \neg$ a.balance $\neq$ b $\rangle$ 
%  onlyThrough $\changes{\prg{\_}}{\prg{a.balance=b}}$
%\end{lstlisting}
%\end{minipage}
%\begin{minipage}{0.24\textwidth}
%\scriptsize
%\hfill by $\longrightarrow$ and \textsc{Encap-f}$_1s$
%\end{minipage}
%\begin{minipage}{0.75\textwidth}
%\begin{lstlisting}[language=Chainmail, mathescape=true]
%a:Account $\wedge$ a.balance=b $\wedge$ $\encapsulated{\prg{a.balance=b}}$ 
%  to $\langle\ \neg$ a.balance $\neq$ b $\rangle$ 
%  onlyThrough $\changes{\I}{\prg{a.balance=b}}$ $\vee$ 
%              $\changes{\I}{\encapsulated{\text{a.balance=b}}}$
%\end{lstlisting}
%\end{minipage}
%\begin{minipage}{0.24\textwidth}
%\scriptsize
%\hfill by \textsc{Encap-Write} and \textsc{Trans}$_1$
%\end{minipage}
%\begin{minipage}{0.75\textwidth}
%\begin{lstlisting}[language=Chainmail, mathescape=true]
%a:Account $\wedge$ a.balance=b $\wedge$ $\encapsulated{\prg{a.balance=b}}$ 
%  to $\langle\ \neg$ a.balance $\neq$ b $\rangle$ 
%  onlyThrough $\changes{I}{\prg{a.balance=b}}$ $\vee$ false
%\end{lstlisting}
%\end{minipage}
%\begin{minipage}{0.24\textwidth}
%\scriptsize
%\hfill by \textsc{Encap-Int-f} and \textsc{Trans}$_3$
%\end{minipage}
%\begin{minipage}{0.75\textwidth}
%\scriptsize
%\begin{lstlisting}[language=Chainmail, mathescape=true]
%a:Account $\wedge$ a.balance=b $\wedge$ $\encapsulated{\prg{a.balance=b}}$ 
%  to $\langle\ \neg$ a.balance $\neq$ b $\rangle$ 
%  onlyThrough $\changes{I}{\prg{a.balance=b}}$
%\end{lstlisting}
%\end{minipage}
%\begin{minipage}{0.24\textwidth}
%\scriptsize
%\hfill by $\vee$\textsc{E}$_1$
%\end{minipage}
%\begin{minipage}{0.75\textwidth}
%\begin{lstlisting}[language=Chainmail, mathescape=true]
%a:Account $\wedge$ a.balance=b $\wedge$ $\encapsulated{\prg{a.balance=b}}$ 
%  to $\langle\ \neg$ a.balance $\neq$ b $\rangle$ 
%  onlyThrough $\exists$ o.[$\access{\prg{o}}{\prg{a}}$ $\wedge$ $\external{\prg{o}}$]
%\end{lstlisting}
%\end{minipage}
%\begin{minipage}{0.24\textwidth}
%\scriptsize
%\hfill by \textsc{Int-Changes}, \textsc{BASafety}, and $\longrightarrow$
%\end{minipage}
%\begin{minipage}{0.75\textwidth}
%\begin{lstlisting}[language=Chainmail, mathescape=true, frame=none]
%ResultA $\triangleq$ a:Account $\wedge$ a.balance=b  
%            to $\langle\ \neg$ a.balance $\neq$ b $\rangle$ 
%            onlyThrough $\neg\encapsulated{\prg{a}}$
%\end{lstlisting}
%\end{minipage}
%\begin{minipage}{0.24\textwidth}
%\scriptsize
%\hfill by $\longrightarrow$
%\end{minipage}
%\end{proofBox}
%
%\begin{proofBox}{white}{black}
%\footnotesize
%\begin{minipage}{0.75\textwidth}
%\begin{lstlisting}[language=Chainmail, mathescape=true, frame=single]
%a:Account $\wedge$ $\encapsulated{\prg{a}}$ 
%  to $\neg\encapsulated{\prg{a}}$
%  onlyThrough $\changes{\_}{\encapsulated{\prg{a}}}$
%\end{lstlisting}
%\end{minipage}
%\begin{minipage}{0.24\textwidth}
%\scriptsize
%\hfill by \textsc{Changes}
%\end{minipage}
%\begin{minipage}{0.75\textwidth}
%\begin{lstlisting}[language=Chainmail, mathescape=true]
%a:Account $\wedge$ $\encapsulated{\prg{a}}$ 
%  to $\neg\encapsulated{\prg{a}}$
%  onlyThrough $\changes{I}{\encapsulated{\prg{a}}}$
%\end{lstlisting}
%\end{minipage}
%\begin{minipage}{0.24\textwidth}
%\scriptsize
%\hfill by \textsc{Encap}
%\end{minipage}
%\begin{minipage}{0.75\textwidth}
%\begin{lstlisting}[language=Chainmail, mathescape=true, frame=none]
%ResultB $\triangleq$ a:Account $\wedge$ $\encapsulated{\prg{a}}$ 
%  to $\neg\encapsulated{\prg{a}}$ 
%  onlyThrough false
%\end{lstlisting}
%\end{minipage}
%\begin{minipage}{0.24\textwidth}
%\scriptsize
%\hfill by \textsc{Int-Changes}, \textsc{BAEncapsulation}, and $\longrightarrow$
%\end{minipage}
%\end{proofBox}
%
%\begin{proofBox}{white}{black}
%\footnotesize
%\begin{minipage}{0.75\textwidth}
%\begin{lstlisting}[language=Chainmail, mathescape=true, frame=single]
%a:Account $\wedge$ a.balance=b $\wedge$ 
%  to $\neg\encapsulated{\prg{a}}$ 
%  onlyIf $\encapsulated{\prg{a}}$ $\vee$ $\neg\encapsulated{\prg{a}}$
%\end{lstlisting}
%\end{minipage}
%\begin{minipage}{0.24\textwidth}
%\scriptsize
%\hfill by \textsc{If-True}, \textsc{Excluded Middle}, and \textsc{If}-$\longrightarrow$
%\end{minipage}
%\begin{minipage}{0.75\textwidth}
%\begin{lstlisting}[language=Chainmail, mathescape=true]
%a:Account $\wedge$ a.balance=b $\wedge$ 
%  to $\neg\encapsulated{\prg{a}}$  
%  onlyIf $\neg \encapsulated{\prg{a}}$
%\end{lstlisting}
%\end{minipage}
%\begin{minipage}{0.24\textwidth}
%\scriptsize
%\hfill by \textsc{If}-$\vee$\textsc{E}$_1$ and \prg{ResultB}
%\end{minipage}
%\begin{minipage}{0.75\textwidth}
%\begin{lstlisting}[language=Chainmail, mathescape=true, frame=none]
%Result $\triangleq$ a:Account $\wedge$ a.balance=b $\wedge$ 
%           to $\langle\ \neg$ a.balance $\neq$ b $\rangle$ 
%           onlyIf $\neg \encapsulated{\prg{a}}$
%\end{lstlisting}
%\end{minipage}
%\begin{minipage}{0.24\textwidth}
%\scriptsize
%\hfill by \prg{ResultA} and \textsc{If-Trans}
%\end{minipage}
%%\begin{minipage}{0.75\textwidth}
%%\begin{lstlisting}[mathescape=true, frame=none]
%%Result $\triangleq$ a:Account $\wedge$ a.balance=b $\wedge$ 
%%           to $\langle\ \neg$ a.balance $\neq$ b $\rangle$ 
%%           onlyIf $\neg \bencapsulated{\prg{a}}$
%%\end{lstlisting}
%%\end{minipage}
%%\begin{minipage}{0.24\textwidth}
%%\scriptsize
%%\hfill by Lemma \ref{lem:behaveToTop}(\textsc{Read}) and $\longrightarrow$
%%\end{minipage}
%\end{proofBox}
%\end{minipage}
%%\end{figure} 
%
%\newpage
%
%\subsection{Bank Account 2 - No Leaking} 
%\label{ex:bank2}
%In this version of the Bank Account, we show that we 
%are able to reason about fairly complex forms of encapsulation.
%Specifically if some \prg{Bank} \prg{b} encapsulates
%some \prg{Account} a, then it must follow that 
%the \prg{Ledger} \prg{l} that contains \prg{a}
%is also encapsulated by \prg{b}, and subsequently that any change to
%the balance of \prg{a} requires a call to \prg{transfer}
%on \prg{b}. This is interesting because, while other examples
%do not provide a way to leak access to an account, this example 
%does, but only through the \prg{Ledger} interface, and thus
%encapsulation of the \prg{Ledger} is sufficient to ensure
%encapsulation of the account. Note, this example only proves 
%that leaking of an account is not possible, it does not prove 
%that encapsulation of an account is always assured under all 
%program states. Encapsulation of the account is shown in the
%example in Section \ref{ex:bank3}.
%\begin{lstlisting}[mathescape=true, frame=lines]
%module BankMdl
%  class Account
%    field bal:int
%    method transfer(to, amt)
%      this.bal := this.bal - amt
%      to.bal := to.bal + amt
%
%  class Ledger
%    field acc1:Account
%    field acc2:Account
%    method getAcc1()
%      return this.acc1
%    method getAcc2()
%      return this.acc2
%
%  class Bank
%    field accs:Ledger
%    method transfer(amt)
%      this.accs.getAcc1().transfer(this.accs.getAcc2(), amt)
%\end{lstlisting}
%\begin{lstlisting}[language=Chainmail, mathescape=true, frame=lines]
%BankSpec $\triangleq$ a:Account $\wedge$ a.bal=x $\wedge$ b:Bank $\wedge$ $\encapsulates{\prg{b}}{\prg{a}}$
%             to a.bal $\neq$ x
%             onlyThrough $\calls{\_}{\prg{b}}{\prg{transfer}}{\_}$
%\end{lstlisting}
%As with the example in Section \ref{ex:bank1}, we require 
%the \verb|Bank| module to prove several invariants:
%\begin{lstlisting}[language=Chainmail, mathescape=true, frame=lines]
%AccAccess $\triangleq$ a:Account $\wedge$ $\encapsulates{\prg{B}}{\prg{a}}$
%              to $\neg\encapsulates{\prg{B}}{\prg{a}}$
%              through BankMdl
%              onlyIf $\exists$ o, l. [ l:Ledger $\wedge$ ($\calls{\prg{o}}{\prg{l}}{\prg{getAcc1}}{}$ $\wedge$ l.acc1=a $\vee$ 
%                                                 $\calls{\prg{o}}{\prg{l}}{\prg{getAcc2}}{}$ $\wedge$ l.acc2=a)]
%\end{lstlisting}
%\begin{lstlisting}[language=Chainmail, mathescape=true, frame=lines]
%BalChange $\triangleq$ a:Account $\wedge$ a.bal=x
%              to a.bal $\neq$ x
%              through BankMdl
%              onlyIf $\exists$ a' o. [a':Account $\wedge$ $\external{\prg{o}}$ $\wedge$ 
%                            ($\calls{\_}{\prg{a}}{\prg{transfer}}{\prg{a'}, \_}$ $\vee$ $\calls{\_}{\prg{a'}}{\prg{transfer}}{\prg{a}, \_}$)] $\vee$
%                     $\exists$ b. [b:Bank $\wedge$ $\calls{\_}{\prg{b}}{\prg{transfer}}{\_}$ $\wedge$ 
%                            (b.accs.acc1=a $\vee$ b.accs.acc2=a)])]
%\end{lstlisting}
%%We also need to prove a more general property of the encapsulation of account objects by bank objects.
%%\begin{lstlisting}[language=Chainmail, mathescape=true, frame=lines]
%%AccEncap1 $\triangleq$ b:Bank $\wedge$ ($\encapsulates{\prg{b}}{\prg{b.accs.acc1}}$ $\vee$ $\encapsulates{\prg{b}}{\prg{b.accs.acc2}}$)
%%              $\longrightarrow$ $\encapsulates{\prg{b}}{\prg{b.accs}}$
%%\end{lstlisting}
%Finally we need to prove the following property:
%\begin{lstlisting}[language=Chainmail, mathescape=true, frame=lines]
%BankEncapUnique $\triangleq$ a:Account $\wedge$ b b':Bank $\wedge$ $\encapsulates{\prg{b}}{\prg{a}}$ $\wedge$
%                    (b'.accs.acc1=a || b'.accs.acc2=a)
%                    $\longrightarrow$ b'=b
%\end{lstlisting}
%
%
%
%\begin{proofBox}{white}{black}
%\footnotesize
%\begin{minipage}{0.75\textwidth}
%\begin{lstlisting}[language=Chainmail, mathescape=true, frame=single]
%a:Account $\wedge$ b:Bank $\wedge$ $\encapsulates{\prg{b}}{\prg{a}}$ $\wedge$ 
%  $\changes{\_}{\encapsulates{\prg{b}}{\prg{a}}}$
%\end{lstlisting}
%\end{minipage}
%\begin{minipage}{0.24\textwidth}
%\end{minipage}
%\begin{minipage}{0.75\textwidth}
%\begin{lstlisting}[language=Chainmail, mathescape=true]
%a:Account $\wedge$ b:Bank $\wedge$ $\encapsulates{\prg{b}}{\prg{a}}$ $\wedge$ 
%  $\changes{X_\prg{\{b\}}}{\encapsulates{\prg{b}}{\prg{a}}}$
%\end{lstlisting}
%\end{minipage}
%\begin{minipage}{0.24\textwidth}
%\scriptsize
%\hfill by \textsc{Changes}-$\longrightarrow$ and \textsc{Changes-Encap}
%\end{minipage}
%\begin{minipage}{0.75\textwidth}
%\begin{lstlisting}[language=Chainmail, mathescape=true]
%a:Account $\wedge$ b:Bank $\wedge$ $\encapsulates{\prg{b}}{\prg{a}}$ $\wedge$ l:Ledger $\wedge$ 
%  ($\calls{\prg{x}}{\prg{l}}{\prg{getAcc1}}{}$ $\wedge$ l.acc1=a $\vee$ 
%   $\calls{\prg{x}}{\prg{l}}{\prg{getAcc2}}{}$ $\wedge$ l.acc2=a)
%\end{lstlisting}
%\end{minipage}
%\begin{minipage}{0.24\textwidth}
%\scriptsize
%\hfill by \prg{AccAccess}, \textsc{Changes-Int}, and \textsc{Exists}
%\end{minipage}
%\begin{minipage}{0.75\textwidth}
%\begin{lstlisting}[language=Chainmail, mathescape=true]
%a:Account $\wedge$ b:Bank $\wedge$ $\encapsulates{\prg{b}}{\prg{a}}$ $\wedge$ l:Ledger $\wedge$ 
%  ($\calls{\prg{x}}{\prg{l}}{\prg{getAcc1}}{}$ $\wedge$ l.acc1=a $\vee$ 
%   $\calls{\prg{x}}{\prg{l}}{\prg{getAcc2}}{}$ $\wedge$ l.acc2=a) $\wedge$ $\external{\prg{x}}$
%\end{lstlisting}
%\end{minipage}
%\begin{minipage}{0.24\textwidth}
%\scriptsize
%\hfill by \textsc{Calls-Ext/Int}
%\end{minipage}
%\begin{minipage}{0.75\textwidth}
%\begin{lstlisting}[language=Chainmail, mathescape=true]
%a:Account $\wedge$ b:Bank $\wedge$ $\encapsulates{\prg{b}}{\prg{a}}$ $\wedge$ l:Ledger $\wedge$ 
%  x=this $\wedge$ $\access{\prg{x}}{\prg{l}}$ $\wedge$ $\access{\prg{l}}{\prg{a}}$ $\wedge$ $\external{\prg{x}}$
%\end{lstlisting}
%\end{minipage}
%\begin{minipage}{0.24\textwidth}
%\scriptsize
%\hfill by \textsc{Calls-}\prg{this} and \textsc{Calls-Recv}
%\end{minipage}
%\begin{minipage}{0.75\textwidth}
%\begin{lstlisting}[language=Chainmail, mathescape=true]
%a:Account $\wedge$ b:Bank $\wedge$ $\encapsulated{\prg{a}}$ $\wedge$ l:Ledger $\wedge$
%  $\external{\prg{this}}$ $\wedge$ $\neg \encapsulated{\prg{this}}$ $\wedge$ $\access{\prg{this}}{\prg{l}}$ $\wedge$ 
%  ($\encapsulated{\prg{l}}$ $\vee$ l $\in$ {b})
%\end{lstlisting}
%\end{minipage}
%\begin{minipage}{0.24\textwidth}
%\scriptsize
%\hfill by \textsc{Encap-}\prg{this}, \textsc{Encap-Access}, and \textsc{Encap-}$\cup$
%\end{minipage}
%\begin{minipage}{0.75\textwidth}
%\begin{lstlisting}[language=Chainmail, mathescape=true]
%a:Account $\wedge$ b:Bank $\wedge$ $\encapsulated{\prg{a}}$ $\wedge$ l:Ledger $\wedge$ 
%  $\external{\prg{this}}$ $\wedge$ $\neg \encapsulated{\prg{this}}$ $\wedge$ $\access{\prg{this}}{\prg{l}}$ $\wedge$ $\encapsulated{\prg{l}}$
%\end{lstlisting}
%\end{minipage}
%\begin{minipage}{0.24\textwidth}
%\scriptsize
%\hfill by $\vee$\textsc{E}
%\end{minipage}
%\begin{minipage}{0.75\textwidth}
%\begin{lstlisting}[language=Chainmail, mathescape=true]
%a:Account $\wedge$ b:Bank $\wedge$ $\encapsulated{\prg{a}}$ $\wedge$ l:Ledger $\wedge$
%  $\external{\prg{this}}$ $\wedge$ $\neg \encapsulated{\prg{this}}$ $\wedge$ $\encapsulated{\prg{l}}$ $\wedge$
%  ($\encapsulated{\prg{this}}$ $\vee$ this $\in$ $\I$)
%\end{lstlisting}
%\end{minipage}
%\begin{minipage}{0.24\textwidth}
%\scriptsize
%\hfill by \textsc{Encap-Access}
%\end{minipage}
%\begin{minipage}{0.75\textwidth}
%\begin{lstlisting}[language=Chainmail, mathescape=true]
%a:Account $\wedge$ b:Bank $\wedge$ $\encapsulated{\prg{a}}$ $\wedge$ l:Ledger $\wedge$
%  $\external{\prg{this}}$ $\wedge$ $\neg \encapsulated{\prg{this}}$ $\wedge$ $\encapsulated{\prg{l}}$ $\wedge$
%  false
%\end{lstlisting}
%\end{minipage}
%\begin{minipage}{0.24\textwidth}
%\scriptsize
%\hfill by $\longrightarrow$
%\end{minipage}
%\begin{minipage}{0.75\textwidth}
%\begin{lstlisting}[language=Chainmail, mathescape=true, frame=none]
%ResultA $\triangleq$ a:Account $\wedge$ b:Bank $\wedge$ $\encapsulates{\prg{b}}{\prg{a}}$ $\wedge$ 
%            $\changes{\_}{\encapsulates{\prg{b}}{\prg{a}}}$ $\longrightarrow$ false
%\end{lstlisting}
%\end{minipage}
%\begin{minipage}{0.24\textwidth}
%\end{minipage}
%\end{proofBox}
%
%
%%\begin{proofBox}{white}{black}
%%\footnotesize
%%\begin{minipage}{0.75\textwidth}
%%\begin{lstlisting}[language=Chainmail, mathescape=true, frame=single]
%%a:Account $\wedge$ b:Bank $\wedge$ $\encapsulates{\prg{b}}{\prg{a}}$
%%  to $\neg \encapsulates{\prg{b}}{\prg{a}}$
%%  onlyThrough $\changes{\_}{\encapsulates{\prg{b}}{\prg{a}}}$
%%\end{lstlisting}
%%\end{minipage}
%%\begin{minipage}{0.24\textwidth}
%%\scriptsize
%%\hfill by \textsc{Changes}
%%\end{minipage}
%%\begin{minipage}{0.75\textwidth}
%%\begin{lstlisting}[language=Chainmail, mathescape=true]
%%a:Account $\wedge$ b:Bank $\wedge$ $\encapsulates{\prg{b}}{\prg{a}}$
%%  to $\neg\encapsulates{\prg{b}}{\prg{a}}$
%%  onlyThrough $\encapsulates{\prg{b}}{\prg{a}}$ $\wedge$ $\changes{X_\prg{\{b\}}}{\encapsulates{\prg{b}}{\prg{a}}}$
%%\end{lstlisting}
%%\end{minipage}
%%\begin{minipage}{0.24\textwidth}
%%\scriptsize
%%\hfill by \textsc{Changes}-$\longrightarrow$, \textsc{Changes-Encap} and $\longrightarrow$
%%\end{minipage}
%%\begin{minipage}{0.75\textwidth}
%%\begin{lstlisting}[language=Chainmail, mathescape=true]
%%a:Account $\wedge$ b:Bank $\wedge$ $\encapsulates{\prg{b}}{\prg{a}}$
%%  to $\neg\encapsulates{\prg{b}}{\prg{a}}$
%%  onlyThrough $\encapsulates{\prg{b}}{\prg{a}}$ $\wedge$ l:Ledger $\wedge$ 
%%                  ($\calls{\prg{x}}{\prg{l}}{\prg{getAcc1}}{}$ $\wedge$ l.acc1=a $\vee$ 
%%                   $\calls{\prg{x}}{\prg{l}}{\prg{getAcc2}}{}$ $\wedge$ l.acc2=a)
%%\end{lstlisting}
%%\end{minipage}
%%\begin{minipage}{0.24\textwidth}
%%\scriptsize
%%\hfill by \prg{AccAccess}, \textsc{Changes-Int}, \textsc{Exists}, and $\longrightarrow$
%%\end{minipage}
%%\begin{minipage}{0.75\textwidth}
%%\begin{lstlisting}[language=Chainmail, mathescape=true]
%%a:Account $\wedge$ b:Bank $\wedge$ $\encapsulates{\prg{b}}{\prg{a}}$
%%  to $\neg\encapsulates{\prg{b}}{\prg{a}}$
%%  onlyThrough $\encapsulates{\prg{b}}{\prg{a}}$ $\wedge$ l:Ledger $\wedge$ 
%%                  ($\calls{\prg{x}}{\prg{l}}{\prg{getAcc1}}{}$ $\wedge$ l.acc1=a $\vee$ 
%%                   $\calls{\prg{x}}{\prg{l}}{\prg{getAcc2}}{}$ $\wedge$ l.acc2=a) $\wedge$ $\external{\prg{x}}$
%%\end{lstlisting}
%%\end{minipage}
%%\begin{minipage}{0.24\textwidth}
%%\scriptsize
%%\hfill by \textsc{Calls-Ext/Int} and $\longrightarrow$
%%\end{minipage}
%%\begin{minipage}{0.75\textwidth}
%%\begin{lstlisting}[language=Chainmail, mathescape=true]
%%a:Account $\wedge$ b:Bank $\wedge$ $\encapsulates{\prg{b}}{\prg{a}}$
%%  to $\neg\encapsulates{\prg{b}}{\prg{a}}$
%%  onlyThrough $\encapsulates{\prg{b}}{\prg{a}}$ $\wedge$ l:Ledger $\wedge$ x=this $\wedge$
%%                  $\access{\prg{x}}{\prg{l}}$ $\wedge$ $\access{\prg{l}}{\prg{a}}$ $\wedge$ $\external{\prg{x}}$
%%\end{lstlisting}
%%\end{minipage}
%%\begin{minipage}{0.24\textwidth}
%%\scriptsize
%%\hfill by \textsc{Calls-}\prg{this}, \textsc{Calls-Recv} and $\longrightarrow$
%%\end{minipage}
%%\begin{minipage}{0.75\textwidth}
%%\begin{lstlisting}[language=Chainmail, mathescape=true]
%%a:Account $\wedge$ b:Bank $\wedge$ $\encapsulates{\prg{b}}{\prg{a}}$
%%  to $\neg\encapsulates{\prg{b}}{\prg{a}}$
%%  onlyThrough $\encapsulated{\prg{a}}$ $\wedge$ l:Ledger $\wedge$ $\external{\prg{this}}$ $\wedge$
%%                  $\neg \encapsulated{\prg{this}}$ $\wedge$ $\access{\prg{this}}{\prg{l}}$ $\wedge$ 
%%                  ($\encapsulated{\prg{l}}$ $\vee$ l $\in$ {b})
%%\end{lstlisting}
%%\end{minipage}
%%\begin{minipage}{0.24\textwidth}
%%\scriptsize
%%\hfill by \textsc{Encap-}\prg{this}, \textsc{Encap-Access}, \textsc{Encap-}$\cup$, and $\longrightarrow$
%%\end{minipage}
%%\begin{minipage}{0.75\textwidth}
%%\begin{lstlisting}[language=Chainmail, mathescape=true]
%%a:Account $\wedge$ b:Bank $\wedge$ $\encapsulates{\prg{b}}{\prg{a}}$
%%  to $\neg\encapsulates{\prg{b}}{\prg{a}}$
%%  onlyThrough $\encapsulated{\prg{a}}$ $\wedge$ l:Ledger $\wedge$ $\external{\prg{this}}$ $\wedge$
%%                  $\neg \encapsulated{\prg{this}}$ $\wedge$ $\access{\prg{this}}{\prg{l}}$ $\wedge$ 
%%                  $\encapsulated{\prg{l}}$
%%\end{lstlisting}
%%\end{minipage}
%%\begin{minipage}{0.24\textwidth}
%%\scriptsize
%%\hfill by $\longrightarrow$
%%\end{minipage}
%%\begin{minipage}{0.75\textwidth}
%%\begin{lstlisting}[language=Chainmail, mathescape=true]
%%a:Account $\wedge$ b:Bank $\wedge$ $\encapsulates{\prg{b}}{\prg{a}}$
%%  to $\neg\encapsulates{\prg{b}}{\prg{a}}$
%%  onlyThrough $\encapsulated{\prg{a}}$ $\wedge$ l:Ledger $\wedge$ $\external{\prg{this}}$ $\wedge$
%%                  $\neg \encapsulated{\prg{this}}$ $\wedge$ $\encapsulated{\prg{l}}$ $\wedge$
%%                  ($\encapsulated{\prg{this}}$ $\vee$ this $\in$ $\I$)
%%\end{lstlisting}
%%\end{minipage}
%%\begin{minipage}{0.24\textwidth}
%%\scriptsize
%%\hfill by \textsc{Encap-Access} and $\longrightarrow$
%%\end{minipage}
%%\begin{minipage}{0.75\textwidth}
%%\begin{lstlisting}[language=Chainmail, mathescape=true, frame=none]
%%ResultA $\triangleq$ a:Account $\wedge$ b:Bank $\wedge$ $\encapsulates{\prg{b}}{\prg{a}}$
%%  to $\neg\encapsulates{\prg{b}}{\prg{a}}$
%%  onlyThrough false
%%\end{lstlisting}
%%\end{minipage}
%%\begin{minipage}{0.24\textwidth}
%%\scriptsize
%%\hfill by $\longrightarrow$
%%\end{minipage}
%%\end{proofBox}
%
%
%\begin{proofBox}{white}{black}
%\footnotesize
%\begin{minipage}{0.75\textwidth}
%\begin{lstlisting}[language=Chainmail, mathescape=true, frame=single]
%a:Account $\wedge$ a.bal=x $\wedge$ b:Bank $\wedge$ $\encapsulates{\prg{b}}{\prg{a}}$
%  to a.bal $\neq$ x
%  onlyThrough $\changes{\_}{\prg{a.bal=x}}$
%\end{lstlisting}
%\end{minipage}
%\begin{minipage}{0.24\textwidth}
%\scriptsize
%\hfill by \textsc{Changes} and $\longrightarrow$
%\end{minipage}
%\begin{minipage}{0.75\textwidth}
%\begin{lstlisting}[language=Chainmail, mathescape=true]
%a:Account $\wedge$ a.bal=x $\wedge$ b:Bank $\wedge$ $\encapsulates{\prg{b}}{\prg{a}}$
%  to a.bal $\neq$ x
%  onlyThrough $\encapsulates{\prg{b}}{\prg{a}}$ $\wedge$ 
%     ($\changes{X_{\{\prg{b}\}}}{\prg{a.bal=x}}$ $\vee$ $\changes{X_{\{\prg{b}\}}}{\encapsulates{\prg{b}}{\prg{a}}}$)
%\end{lstlisting}
%\end{minipage}
%\begin{minipage}{0.24\textwidth}
%\scriptsize
%\hfill by \textsc{Encap-Write} and $\longrightarrow$
%\end{minipage}
%\begin{minipage}{0.75\textwidth}
%\begin{lstlisting}[language=Chainmail, mathescape=true]
%a:Account $\wedge$ a.bal=x $\wedge$ b:Bank $\wedge$ $\encapsulates{\prg{b}}{\prg{a}}$
%  to a.bal $\neq$ x
%  onlyThrough $\encapsulates{\prg{b}}{\prg{a}}$ $\wedge$ $\changes{\I}{\prg{a.bal=x}}$ $\vee$ false
%\end{lstlisting}
%\end{minipage}
%\begin{minipage}{0.24\textwidth}
%\scriptsize
%\hfill by \prg{ResultA}, \textsc{Encap-}$\cup$, and $\longrightarrow$
%\end{minipage}
%\begin{minipage}{0.75\textwidth}
%\begin{lstlisting}[language=Chainmail, mathescape=true]
%a:Account $\wedge$ a.bal=x $\wedge$ b:Bank $\wedge$ $\encapsulates{\prg{b}}{\prg{a}}$
%  to a.bal $\neq$ x
%  onlyThrough $\encapsulates{\prg{b}}{\prg{a}}$ $\wedge$ 
%              $\exists$ a' o. [a':Account $\wedge$ $\external{\prg{o}}$ $\wedge$ 
%                          ($\calls{\prg{o}}{\prg{a}}{\prg{transfer}}{\prg{a'}, \_}$ $\vee$ 
%                           $\calls{\prg{o}}{\prg{a'}}{\prg{transfer}}{\prg{a}, \_}$)] $\vee$
%              $\exists$ b'. [b':Bank $\wedge$ $\calls{\_}{\prg{b'}}{\prg{transfer}}{\_}$ $\wedge$ 
%                      (b'.accs.acc1=a $\vee$ b'.accs.acc2=a)])]
%\end{lstlisting}
%\end{minipage}
%\begin{minipage}{0.24\textwidth}
%\scriptsize
%\hfill by \prg{BalChange}, \textsc{Changes-Int}, and $\longrightarrow$
%\end{minipage}
%\begin{minipage}{0.75\textwidth}
%\begin{lstlisting}[language=Chainmail, mathescape=true]
%a:Account $\wedge$ a.bal=x $\wedge$ b:Bank $\wedge$ $\encapsulates{\prg{b}}{\prg{a}}$
%  to a.bal $\neq$ x
%  onlyThrough $\encapsulates{\prg{b}}{\prg{a}}$ $\wedge$ 
%              ($\external{\prg{this}}$ $\wedge$ $\access{\prg{this}}{\prg{a}}$) $\vee$
%              b':Bank $\wedge$ $\calls{\_}{\prg{b'}}{\prg{transfer}}{\_}$ $\wedge$ 
%              (b'.accs.acc1=a $\vee$ b'.accs.acc2=a)])
%\end{lstlisting}
%\end{minipage}
%\begin{minipage}{0.24\textwidth}
%\scriptsize
%\hfill by \textsc{Exists}, \textsc{Calls-Recv}, \textsc{Calls-Args}, \textsc{Calls-}\prg{this}, and $\longrightarrow$
%\end{minipage}
%\begin{minipage}{0.75\textwidth}
%\begin{lstlisting}[language=Chainmail, mathescape=true]
%a:Account $\wedge$ a.bal=x $\wedge$ b:Bank $\wedge$ $\encapsulates{\prg{b}}{\prg{a}}$
%  to a.bal $\neq$ x
%  onlyThrough $\encapsulates{\prg{b}}{\prg{a}}$ $\wedge$ 
%              ($\encapsulates{\prg{b}}{\prg{this}}$ $\vee$ this $\in$ {b}) $\vee$
%              b':Bank $\wedge$ $\calls{\_}{\prg{b'}}{\prg{transfer}}{\_}$ $\wedge$ 
%              (b'.accs.acc1=a $\vee$ b'.accs.acc2=a)])
%\end{lstlisting}
%\end{minipage}
%\begin{minipage}{0.24\textwidth}
%\scriptsize
%\hfill by \textsc{Encap-Access}
%\end{minipage}
%\begin{minipage}{0.75\textwidth}
%\begin{lstlisting}[language=Chainmail, mathescape=true]
%a:Account $\wedge$ a.bal=x $\wedge$ b:Bank $\wedge$ $\encapsulates{\prg{b}}{\prg{a}}$
%  to a.bal $\neq$ x
%  onlyThrough $\encapsulates{\prg{b}}{\prg{a}}$ $\wedge$
%              b':Bank $\wedge$ $\calls{\_}{\prg{b'}}{\prg{transfer}}{\_}$ $\wedge$ 
%              (b'.accs.acc1=a $\vee$ b'.accs.acc2=a)])
%\end{lstlisting}
%\end{minipage}
%\begin{minipage}{0.24\textwidth}
%\scriptsize
%\hfill by \textsc{Encap-}\prg{this} and $\longrightarrow$
%\end{minipage}
%\begin{minipage}{0.75\textwidth}
%\begin{lstlisting}[language=Chainmail, mathescape=true]
%a:Account $\wedge$ a.bal=x $\wedge$ b:Bank $\wedge$ $\encapsulates{\prg{b}}{\prg{a}}$
%  to a.bal $\neq$ x
%  onlyThrough $\encapsulates{\prg{b}}{\prg{a}}$ $\wedge$
%              b':Bank $\wedge$ $\calls{\_}{\prg{b'}}{\prg{transfer}}{\_}$ $\wedge$ 
%              b'=b
%\end{lstlisting}
%\end{minipage}
%\begin{minipage}{0.24\textwidth}
%\scriptsize
%\hfill by \prg{BankEncapUnique} and $\longrightarrow$
%\end{minipage}
%\begin{minipage}{0.75\textwidth}
%\begin{lstlisting}[language=Chainmail, mathescape=true]
%Result $\triangleq$ a:Account $\wedge$ a.bal=x $\wedge$ b:Bank $\wedge$ $\encapsulates{\prg{b}}{\prg{a}}$
%           to a.bal $\neq$ x
%           onlyThrough $\calls{\_}{\prg{b}}{\prg{transfer}}{\_}$
%\end{lstlisting}
%\end{minipage}
%\begin{minipage}{0.24\textwidth}
%\scriptsize
%\hfill by $\longrightarrow$
%\end{minipage}
%\end{proofBox}
%
%\newpage
%
%\subsection{Bank Account 3 - General Encapsulation  of Accounts}
%\label{ex:bank3}
%Ideally we would like the encapsulation of accounts to arise from the definition of \prg{Bank},
%and not have to provide it as a precondition to our robustness property. A preferable holistic 
%specification to \prg{Bank} would be:
%\begin{lstlisting}[language=Chainmail, mathescape=true, frame=lines]
%BankSpec $\triangleq$ b:Bank $\wedge$ b.accs.acc1.bal=x $\wedge$ b:Bank
%             to b.accs.acc1.bal $\neq$ x
%             onlyThrough $\calls{\_}{\prg{b}}{\prg{transfer}}{\_}$
%\end{lstlisting}
%That is, under all possible program configurations, the balance of an 
%account can only be modified using the \prg{Bank} interface, i.e.
%encapsulation of the \prg{Account} by the \prg{Bank} is implicit.
%Unfortunately this proof is not possible using the \prg{Bank} as defined
%in Section \ref{ex:bank2} because there is no way to ensure encapsulation
%of the \prg{Account} by the \prg{Bank} when the \prg{Bank} is 
%created. For this we need to specify how \prg{Bank}s and \prg{Account}s are created.
%\begin{lstlisting}[mathescape=true, frame=lines]
%class Account {
%  field bal:int
%  constr (amt)
%    this.bal := amt
%  method transfer(to, amt)
%    this.bal := this.bal - amt
%    to.bal := to.bal + amt
%}
%class Ledger {
%  field acc1:Account
%  field acc2:Account
%  constr (amt1, amt2)
%    this.acc1 := new Account(amt1)
%    this.acc2 := new Account(amt2)
%  method getAcc1()
%    return this.acc1
%  method getAcc2()
%    return this.acc2
%}
%class Bank {
%  field accs:Ledger
%  constr (amt1, amt2)
%    this.accs := new Ledger(amt1, amt2)
%  method transfer(amt){
%    this.accs.getAcc1().transfer(this.accs.getAcc2(), amt)
%  }
%}
%\end{lstlisting}
%We also need to a proof rule that allows us to prove program invariants.
%\begin{mathpar}
%\infer
%		{
%		M;\ M',\ \sigma\ \vdash\ \exists o.[o:\prg{Object}\ \wedge\ \forall o'.[o'=o]]\ \longrightarrow\ A\\
%		M;\ M',\ \sigma\ \vdash\ \onlyThrough{A}{\neg A}{\prg{false}}
%		}
%		{
%		M;\ M',\ \sigma\ \vdash\ A
%		}
%		\quad(\textsc{Invariant})
%\end{mathpar}
%It is quite simple to prove the following property:
%\begin{lstlisting}[language=Chainmail, mathescape=true, frame=single]
%$\exists$ o.[o:Object $\wedge$ $\forall$ o'.[o'=o]] $\longrightarrow$ $\forall$ b.[b:Bank $\longrightarrow$ $\encapsulates{\prg{b}}{\prg{b.accs.acc1}}$]
%\end{lstlisting}
%Now we can use the above rule to demonstrate that given the appropriate
%traditional specifications on the methods on \prg{BankMdl} (including the constructors), 
%$\encapsulates{\prg{b}}{\prg{b.accs.acc1}}$ and $\encapsulates{\prg{b}}{\prg{b.accs.acc2}}$ 
%are program invariants. i.e. \prg{BankMdl} must use traditional specifications to 
%prove the following about all methods in \prg{Bank}, \prg{Ledger}, and \prg{Account}:
%\begin{lstlisting}[language=Chainmail, mathescape=true, frame=lines]
%{$\forall$ b.[b:Bank $\longrightarrow$ $\encapsulates{\prg{b}}{\prg{b.accs.acc1}}$ $\wedge$ $\encapsulates{\prg{b}}{\prg{b.accs.acc2}}$]}
%  _.m(...)
%{$\forall$ b.[b:Bank $\longrightarrow$ $\encapsulates{\prg{b}}{\prg{b.accs.acc1}}$ $\wedge$ $\encapsulates{\prg{b}}{\prg{b.accs.acc2}}$]}
%\end{lstlisting}
%With this we can prove the following internal module invariant for \prg{BankMdl}
%\begin{lstlisting}[language=Chainmail, mathescape=true, frame=lines]
%BankEncap $\triangleq$ $\forall$ b.[b:Bank $\longrightarrow$ $\encapsulates{\prg{b}}{\prg{b.accs.acc1}}$ $\wedge$ $\encapsulates{\prg{b}}{\prg{b.accs.acc2}}$
%              to $\neg\forall$ b.[b:Bank $\longrightarrow$ $\encapsulates{\prg{b}}{\prg{b.accs.acc1}}$ $\wedge$ $\encapsulates{\prg{b}}{\prg{b.accs.acc2}}$
%              through BankMdl
%              onlyIf false
%\end{lstlisting}
%
%Now we can construct a proof of \prg{BankSpec}. We will re-use the results from Section \ref{ex:bank2} 
%as the proofs are much the same.
%
%\begin{proofBox}{white}{black}
%\footnotesize
%\begin{minipage}{0.75\textwidth}
%\begin{lstlisting}[language=Chainmail, mathescape=true, frame=single]
%b:Bank $\wedge$ $\encapsulates{\prg{b}}{\prg{b.accs.acc1}}$
%  to $\neg$ $\encapsulates{\prg{b}}{\prg{b.accs.acc1}}$
%  onlyThrough $\changes{\_}{\encapsulates{\prg{b}}{\prg{b.accs.acc1}}}$
%\end{lstlisting}
%\end{minipage}
%\begin{minipage}{0.24\textwidth}
%\end{minipage}
%\begin{minipage}{0.75\textwidth}
%\begin{lstlisting}[language=Chainmail, mathescape=true]
%b:Bank $\wedge$ $\encapsulates{\prg{b}}{\prg{b.accs.acc1}}$
%  to $\neg$ $\encapsulates{\prg{b}}{\prg{b.accs.acc1}}$
%  onlyThrough false
%\end{lstlisting}
%\end{minipage}
%\begin{minipage}{0.24\textwidth}
%\scriptsize
%\hfill by \prg{ResultA} (from Section \ref{ex:bank1}) and \textsc{Trans}$_1$
%\end{minipage}
%\begin{minipage}{0.75\textwidth}
%\begin{lstlisting}[language=Chainmail, mathescape=true]
%EncapInv $\triangleq$ $\forall$ b.[b:Bank $\longrightarrow$ $\encapsulates{\prg{b}}{\prg{b.accs.acc1}}$]
%             to $\neg$ $\forall$ b.[b:Bank $\longrightarrow$ $\encapsulates{\prg{b}}{\prg{b.accs.acc1}}$]
%             onlyThrough false
%\end{lstlisting}
%\end{minipage}
%\begin{minipage}{0.24\textwidth}
%\scriptsize
%\hfill by {\color{red}Julian: We need some rule to extend quantification across the whole statement. Shouldn't be too hard ...}
%\end{minipage}
%\end{proofBox}
%
%\begin{proofBox}{white}{black}
%\footnotesize
%\begin{minipage}{0.75\textwidth}
%\begin{lstlisting}[language=Chainmail, mathescape=true, frame=single]
%b:Bank $\wedge$ a:Account $\wedge$ b.acc.acc1=a $\wedge$ a.bal=x $\wedge$ 
%  $\encapsulates{\prg{b}}{\prg{a}}$
%  to a.bal $\neq$ x
%  onlyThrough $\calls{\_}{\prg{b}}{\prg{transfer}}{\_}$
%\end{lstlisting}
%\end{minipage}
%\begin{minipage}{0.24\textwidth}
%\scriptsize
%\hfill by \prg{Result}(Section \ref{ex:bank2}) and $\longrightarrow$
%\end{minipage}
%\begin{minipage}{0.75\textwidth}
%\begin{lstlisting}[language=Chainmail, mathescape=true]
%b:Bank $\wedge$ b.acc.acc1.bal=x $\wedge$ $\encapsulates{\prg{b}}{\prg{b.acc.acc1}}$
%  to b.acc.acc1.bal $\neq$ x
%  onlyThrough $\calls{\_}{\prg{b}}{\prg{transfer}}{\_}$
%\end{lstlisting}
%\end{minipage}
%\begin{minipage}{0.24\textwidth}
%\scriptsize
%\hfill by $\longrightarrow$
%\end{minipage}
%\begin{minipage}{0.75\textwidth}
%\begin{lstlisting}[language=Chainmail, mathescape=true]
%Bank3Result $\triangleq$ b:Bank $\wedge$ b.acc.acc1.bal=x
%                to b.acc.acc1.bal $\neq$ x
%                onlyThrough $\calls{\_}{\prg{b}}{\prg{transfer}}{\_}$
%\end{lstlisting}
%\end{minipage}
%\begin{minipage}{0.24\textwidth}
%\scriptsize
%\hfill by \textsc{Invariant}, \prg{EncapInv}, and $\longrightarrow$
%\end{minipage}
%\end{proofBox}
%
%\newpage
%
%
%
%\subsection{Bank Account 4 - Ghost Field Balance {\color{red}(Julian: still working on this)}} 
%\label{ex:bank4}
%In this version of the Bank Account, we show that we 
%are able to reason about fairly complex forms of encapsulation.
%Specifically if some \prg{Bank} \prg{b} encapsulates
%some \prg{Account} a, then it must follow that 
%the \prg{Ledger} \prg{l} that contains \prg{a}
%is also encapsulated by \prg{b}, and subsequently that any change to
%the balance of \prg{a} requires a call to \prg{transfer}
%on \prg{b}. This is interesting because, while other examples
%do not provide a way to leak access to an account, this example 
%does, but only through the \prg{Ledger} interface, and thus
%encapsulation of the \prg{Ledger} is sufficient to ensure
%encapsulation of the account. Note, this example only proves 
%that leaking of an account is not possible, it does not prove 
%that encapsulation of an account is always assured under all 
%program states. Encapsulation of the account is shown in the
%example in Section \ref{ex:bank3}.
%\begin{lstlisting}[mathescape=true, frame=lines]
%module BankMdl
%  class Account
%
%  class Ledger
%    field acc:Account
%    field balance:int
%    field tail:Ledger
%    method addToBalance(amt)
%      if a == this.acc
%        this.balance := this.balance + amt
%    method find(a)
%      if a == this.acc
%        return this
%      else if this.tail == null
%        return null
%      else
%        return this.tail.find(a)
%    ghost balance(a)
%      if this.acc == a
%        return balance
%      else if this.tail == null
%        return -1
%      else 
%        return tail.balance(a)
%
%  class Bank
%    field accs:Ledger
%    
%    method transfer(from, to, amt)
%      dest := this.accs.find(from);
%      src := this.accs.find(to);
%      if dest != null && src != null && amt >= 0
%        dest.addToBalance(amt)
%        src.addToBalance(-1 * amt)
%        
%    ghost balance(acc) 
%      this.accs.balance(acc)
%      
%\end{lstlisting}
%\begin{lstlisting}[language=Chainmail, mathescape=true, frame=lines]
%BankSpec $\triangleq$ a:Account $\wedge$ a.bal=x $\wedge$ b:Bank $\wedge$ $\encapsulates{\prg{b}}{\prg{a}}$
%             to a.bal $\neq$ x
%             onlyThrough $\calls{\_}{\prg{b}}{\prg{transfer}}{\_}$
%\end{lstlisting}
%As with the example in Section \ref{ex:bank1}, we require 
%the \verb|Bank| module to prove several invariants:
%\begin{lstlisting}[language=Chainmail, mathescape=true, frame=lines]
%AccAccess $\triangleq$ a:Account $\wedge$ $\encapsulates{\prg{B}}{\prg{a}}$
%              to $\neg\encapsulates{\prg{B}}{\prg{a}}$
%              through BankMdl
%              onlyIf $\exists$ o, l. [ l:Ledger $\wedge$ ($\calls{\prg{o}}{\prg{l}}{\prg{getAcc1}}{}$ $\wedge$ l.acc1=a $\vee$ 
%                                                 $\calls{\prg{o}}{\prg{l}}{\prg{getAcc2}}{}$ $\wedge$ l.acc2=a)]
%\end{lstlisting}
%\begin{lstlisting}[language=Chainmail, mathescape=true, frame=lines]
%BalChange $\triangleq$ a:Account $\wedge$ a.bal=x
%              to a.bal $\neq$ x
%              through BankMdl
%              onlyIf $\exists$ a' o. [a':Account $\wedge$ $\external{\prg{o}}$ $\wedge$ 
%                            ($\calls{\_}{\prg{a}}{\prg{transfer}}{\prg{a'}, \_}$ $\vee$ $\calls{\_}{\prg{a'}}{\prg{transfer}}{\prg{a}, \_}$)] $\vee$
%                     $\exists$ b. [b:Bank $\wedge$ $\calls{\_}{\prg{b}}{\prg{transfer}}{\_}$ $\wedge$ 
%                            (b.accs.acc1=a $\vee$ b.accs.acc2=a)])]
%\end{lstlisting}
%%We also need to prove a more general property of the encapsulation of account objects by bank objects.
%%\begin{lstlisting}[language=Chainmail, mathescape=true, frame=lines]
%%AccEncap1 $\triangleq$ b:Bank $\wedge$ ($\encapsulates{\prg{b}}{\prg{b.accs.acc1}}$ $\vee$ $\encapsulates{\prg{b}}{\prg{b.accs.acc2}}$)
%%              $\longrightarrow$ $\encapsulates{\prg{b}}{\prg{b.accs}}$
%%\end{lstlisting}
%Finally we need to prove the following property:
%\begin{lstlisting}[language=Chainmail, mathescape=true, frame=lines]
%BankEncapUnique $\triangleq$ a:Account $\wedge$ b b':Bank $\wedge$ $\encapsulates{\prg{b}}{\prg{a}}$ $\wedge$
%                    (b'.accs.acc1=a || b'.accs.acc2=a)
%                    $\longrightarrow$ b'=b
%\end{lstlisting}
%
%
%
%\begin{proofBox}{white}{black}
%\footnotesize
%\begin{minipage}{0.75\textwidth}
%\begin{lstlisting}[language=Chainmail, mathescape=true, frame=single]
%a:Account $\wedge$ b:Bank $\wedge$ $\encapsulates{\prg{b}}{\prg{a}}$ $\wedge$ 
%  $\changes{\_}{\encapsulates{\prg{b}}{\prg{a}}}$
%\end{lstlisting}
%\end{minipage}
%\begin{minipage}{0.24\textwidth}
%\end{minipage}
%\begin{minipage}{0.75\textwidth}
%\begin{lstlisting}[language=Chainmail, mathescape=true]
%a:Account $\wedge$ b:Bank $\wedge$ $\encapsulates{\prg{b}}{\prg{a}}$ $\wedge$ 
%  $\changes{X_\prg{\{b\}}}{\encapsulates{\prg{b}}{\prg{a}}}$
%\end{lstlisting}
%\end{minipage}
%\begin{minipage}{0.24\textwidth}
%\scriptsize
%\hfill by \textsc{Changes}-$\longrightarrow$ and \textsc{Changes-Encap}
%\end{minipage}
%\begin{minipage}{0.75\textwidth}
%\begin{lstlisting}[language=Chainmail, mathescape=true]
%a:Account $\wedge$ b:Bank $\wedge$ $\encapsulates{\prg{b}}{\prg{a}}$ $\wedge$ l:Ledger $\wedge$ 
%  ($\calls{\prg{x}}{\prg{l}}{\prg{getAcc1}}{}$ $\wedge$ l.acc1=a $\vee$ 
%   $\calls{\prg{x}}{\prg{l}}{\prg{getAcc2}}{}$ $\wedge$ l.acc2=a)
%\end{lstlisting}
%\end{minipage}
%\begin{minipage}{0.24\textwidth}
%\scriptsize
%\hfill by \prg{AccAccess}, \textsc{Changes-Int}, and \textsc{Exists}
%\end{minipage}
%\begin{minipage}{0.75\textwidth}
%\begin{lstlisting}[language=Chainmail, mathescape=true]
%a:Account $\wedge$ b:Bank $\wedge$ $\encapsulates{\prg{b}}{\prg{a}}$ $\wedge$ l:Ledger $\wedge$ 
%  ($\calls{\prg{x}}{\prg{l}}{\prg{getAcc1}}{}$ $\wedge$ l.acc1=a $\vee$ 
%   $\calls{\prg{x}}{\prg{l}}{\prg{getAcc2}}{}$ $\wedge$ l.acc2=a) $\wedge$ $\external{\prg{x}}$
%\end{lstlisting}
%\end{minipage}
%\begin{minipage}{0.24\textwidth}
%\scriptsize
%\hfill by \textsc{Calls-Ext/Int}
%\end{minipage}
%\begin{minipage}{0.75\textwidth}
%\begin{lstlisting}[language=Chainmail, mathescape=true]
%a:Account $\wedge$ b:Bank $\wedge$ $\encapsulates{\prg{b}}{\prg{a}}$ $\wedge$ l:Ledger $\wedge$ 
%  x=this $\wedge$ $\access{\prg{x}}{\prg{l}}$ $\wedge$ $\access{\prg{l}}{\prg{a}}$ $\wedge$ $\external{\prg{x}}$
%\end{lstlisting}
%\end{minipage}
%\begin{minipage}{0.24\textwidth}
%\scriptsize
%\hfill by \textsc{Calls-}\prg{this} and \textsc{Calls-Recv}
%\end{minipage}
%\begin{minipage}{0.75\textwidth}
%\begin{lstlisting}[language=Chainmail, mathescape=true]
%a:Account $\wedge$ b:Bank $\wedge$ $\encapsulated{\prg{a}}$ $\wedge$ l:Ledger $\wedge$
%  $\external{\prg{this}}$ $\wedge$ $\neg \encapsulated{\prg{this}}$ $\wedge$ $\access{\prg{this}}{\prg{l}}$ $\wedge$ 
%  ($\encapsulated{\prg{l}}$ $\vee$ l $\in$ {b})
%\end{lstlisting}
%\end{minipage}
%\begin{minipage}{0.24\textwidth}
%\scriptsize
%\hfill by \textsc{Encap-}\prg{this}, \textsc{Encap-Access}, and \textsc{Encap-}$\cup$
%\end{minipage}
%\begin{minipage}{0.75\textwidth}
%\begin{lstlisting}[language=Chainmail, mathescape=true]
%a:Account $\wedge$ b:Bank $\wedge$ $\encapsulated{\prg{a}}$ $\wedge$ l:Ledger $\wedge$ 
%  $\external{\prg{this}}$ $\wedge$ $\neg \encapsulated{\prg{this}}$ $\wedge$ $\access{\prg{this}}{\prg{l}}$ $\wedge$ $\encapsulated{\prg{l}}$
%\end{lstlisting}
%\end{minipage}
%\begin{minipage}{0.24\textwidth}
%\scriptsize
%\hfill by $\vee$\textsc{E}
%\end{minipage}
%\begin{minipage}{0.75\textwidth}
%\begin{lstlisting}[language=Chainmail, mathescape=true]
%a:Account $\wedge$ b:Bank $\wedge$ $\encapsulated{\prg{a}}$ $\wedge$ l:Ledger $\wedge$
%  $\external{\prg{this}}$ $\wedge$ $\neg \encapsulated{\prg{this}}$ $\wedge$ $\encapsulated{\prg{l}}$ $\wedge$
%  ($\encapsulated{\prg{this}}$ $\vee$ this $\in$ $\I$)
%\end{lstlisting}
%\end{minipage}
%\begin{minipage}{0.24\textwidth}
%\scriptsize
%\hfill by \textsc{Encap-Access}
%\end{minipage}
%\begin{minipage}{0.75\textwidth}
%\begin{lstlisting}[language=Chainmail, mathescape=true]
%a:Account $\wedge$ b:Bank $\wedge$ $\encapsulated{\prg{a}}$ $\wedge$ l:Ledger $\wedge$
%  $\external{\prg{this}}$ $\wedge$ $\neg \encapsulated{\prg{this}}$ $\wedge$ $\encapsulated{\prg{l}}$ $\wedge$
%  false
%\end{lstlisting}
%\end{minipage}
%\begin{minipage}{0.24\textwidth}
%\scriptsize
%\hfill by $\longrightarrow$
%\end{minipage}
%\begin{minipage}{0.75\textwidth}
%\begin{lstlisting}[language=Chainmail, mathescape=true, frame=none]
%ResultA $\triangleq$ a:Account $\wedge$ b:Bank $\wedge$ $\encapsulates{\prg{b}}{\prg{a}}$ $\wedge$ 
%            $\changes{\_}{\encapsulates{\prg{b}}{\prg{a}}}$ $\longrightarrow$ false
%\end{lstlisting}
%\end{minipage}
%\begin{minipage}{0.24\textwidth}
%\end{minipage}
%\end{proofBox}
%
%
%%\begin{proofBox}{white}{black}
%%\footnotesize
%%\begin{minipage}{0.75\textwidth}
%%\begin{lstlisting}[language=Chainmail, mathescape=true, frame=single]
%%a:Account $\wedge$ b:Bank $\wedge$ $\encapsulates{\prg{b}}{\prg{a}}$
%%  to $\neg \encapsulates{\prg{b}}{\prg{a}}$
%%  onlyThrough $\changes{\_}{\encapsulates{\prg{b}}{\prg{a}}}$
%%\end{lstlisting}
%%\end{minipage}
%%\begin{minipage}{0.24\textwidth}
%%\scriptsize
%%\hfill by \textsc{Changes}
%%\end{minipage}
%%\begin{minipage}{0.75\textwidth}
%%\begin{lstlisting}[language=Chainmail, mathescape=true]
%%a:Account $\wedge$ b:Bank $\wedge$ $\encapsulates{\prg{b}}{\prg{a}}$
%%  to $\neg\encapsulates{\prg{b}}{\prg{a}}$
%%  onlyThrough $\encapsulates{\prg{b}}{\prg{a}}$ $\wedge$ $\changes{X_\prg{\{b\}}}{\encapsulates{\prg{b}}{\prg{a}}}$
%%\end{lstlisting}
%%\end{minipage}
%%\begin{minipage}{0.24\textwidth}
%%\scriptsize
%%\hfill by \textsc{Changes}-$\longrightarrow$, \textsc{Changes-Encap} and $\longrightarrow$
%%\end{minipage}
%%\begin{minipage}{0.75\textwidth}
%%\begin{lstlisting}[language=Chainmail, mathescape=true]
%%a:Account $\wedge$ b:Bank $\wedge$ $\encapsulates{\prg{b}}{\prg{a}}$
%%  to $\neg\encapsulates{\prg{b}}{\prg{a}}$
%%  onlyThrough $\encapsulates{\prg{b}}{\prg{a}}$ $\wedge$ l:Ledger $\wedge$ 
%%                  ($\calls{\prg{x}}{\prg{l}}{\prg{getAcc1}}{}$ $\wedge$ l.acc1=a $\vee$ 
%%                   $\calls{\prg{x}}{\prg{l}}{\prg{getAcc2}}{}$ $\wedge$ l.acc2=a)
%%\end{lstlisting}
%%\end{minipage}
%%\begin{minipage}{0.24\textwidth}
%%\scriptsize
%%\hfill by \prg{AccAccess}, \textsc{Changes-Int}, \textsc{Exists}, and $\longrightarrow$
%%\end{minipage}
%%\begin{minipage}{0.75\textwidth}
%%\begin{lstlisting}[language=Chainmail, mathescape=true]
%%a:Account $\wedge$ b:Bank $\wedge$ $\encapsulates{\prg{b}}{\prg{a}}$
%%  to $\neg\encapsulates{\prg{b}}{\prg{a}}$
%%  onlyThrough $\encapsulates{\prg{b}}{\prg{a}}$ $\wedge$ l:Ledger $\wedge$ 
%%                  ($\calls{\prg{x}}{\prg{l}}{\prg{getAcc1}}{}$ $\wedge$ l.acc1=a $\vee$ 
%%                   $\calls{\prg{x}}{\prg{l}}{\prg{getAcc2}}{}$ $\wedge$ l.acc2=a) $\wedge$ $\external{\prg{x}}$
%%\end{lstlisting}
%%\end{minipage}
%%\begin{minipage}{0.24\textwidth}
%%\scriptsize
%%\hfill by \textsc{Calls-Ext/Int} and $\longrightarrow$
%%\end{minipage}
%%\begin{minipage}{0.75\textwidth}
%%\begin{lstlisting}[language=Chainmail, mathescape=true]
%%a:Account $\wedge$ b:Bank $\wedge$ $\encapsulates{\prg{b}}{\prg{a}}$
%%  to $\neg\encapsulates{\prg{b}}{\prg{a}}$
%%  onlyThrough $\encapsulates{\prg{b}}{\prg{a}}$ $\wedge$ l:Ledger $\wedge$ x=this $\wedge$
%%                  $\access{\prg{x}}{\prg{l}}$ $\wedge$ $\access{\prg{l}}{\prg{a}}$ $\wedge$ $\external{\prg{x}}$
%%\end{lstlisting}
%%\end{minipage}
%%\begin{minipage}{0.24\textwidth}
%%\scriptsize
%%\hfill by \textsc{Calls-}\prg{this}, \textsc{Calls-Recv} and $\longrightarrow$
%%\end{minipage}
%%\begin{minipage}{0.75\textwidth}
%%\begin{lstlisting}[language=Chainmail, mathescape=true]
%%a:Account $\wedge$ b:Bank $\wedge$ $\encapsulates{\prg{b}}{\prg{a}}$
%%  to $\neg\encapsulates{\prg{b}}{\prg{a}}$
%%  onlyThrough $\encapsulated{\prg{a}}$ $\wedge$ l:Ledger $\wedge$ $\external{\prg{this}}$ $\wedge$
%%                  $\neg \encapsulated{\prg{this}}$ $\wedge$ $\access{\prg{this}}{\prg{l}}$ $\wedge$ 
%%                  ($\encapsulated{\prg{l}}$ $\vee$ l $\in$ {b})
%%\end{lstlisting}
%%\end{minipage}
%%\begin{minipage}{0.24\textwidth}
%%\scriptsize
%%\hfill by \textsc{Encap-}\prg{this}, \textsc{Encap-Access}, \textsc{Encap-}$\cup$, and $\longrightarrow$
%%\end{minipage}
%%\begin{minipage}{0.75\textwidth}
%%\begin{lstlisting}[language=Chainmail, mathescape=true]
%%a:Account $\wedge$ b:Bank $\wedge$ $\encapsulates{\prg{b}}{\prg{a}}$
%%  to $\neg\encapsulates{\prg{b}}{\prg{a}}$
%%  onlyThrough $\encapsulated{\prg{a}}$ $\wedge$ l:Ledger $\wedge$ $\external{\prg{this}}$ $\wedge$
%%                  $\neg \encapsulated{\prg{this}}$ $\wedge$ $\access{\prg{this}}{\prg{l}}$ $\wedge$ 
%%                  $\encapsulated{\prg{l}}$
%%\end{lstlisting}
%%\end{minipage}
%%\begin{minipage}{0.24\textwidth}
%%\scriptsize
%%\hfill by $\longrightarrow$
%%\end{minipage}
%%\begin{minipage}{0.75\textwidth}
%%\begin{lstlisting}[language=Chainmail, mathescape=true]
%%a:Account $\wedge$ b:Bank $\wedge$ $\encapsulates{\prg{b}}{\prg{a}}$
%%  to $\neg\encapsulates{\prg{b}}{\prg{a}}$
%%  onlyThrough $\encapsulated{\prg{a}}$ $\wedge$ l:Ledger $\wedge$ $\external{\prg{this}}$ $\wedge$
%%                  $\neg \encapsulated{\prg{this}}$ $\wedge$ $\encapsulated{\prg{l}}$ $\wedge$
%%                  ($\encapsulated{\prg{this}}$ $\vee$ this $\in$ $\I$)
%%\end{lstlisting}
%%\end{minipage}
%%\begin{minipage}{0.24\textwidth}
%%\scriptsize
%%\hfill by \textsc{Encap-Access} and $\longrightarrow$
%%\end{minipage}
%%\begin{minipage}{0.75\textwidth}
%%\begin{lstlisting}[language=Chainmail, mathescape=true, frame=none]
%%ResultA $\triangleq$ a:Account $\wedge$ b:Bank $\wedge$ $\encapsulates{\prg{b}}{\prg{a}}$
%%  to $\neg\encapsulates{\prg{b}}{\prg{a}}$
%%  onlyThrough false
%%\end{lstlisting}
%%\end{minipage}
%%\begin{minipage}{0.24\textwidth}
%%\scriptsize
%%\hfill by $\longrightarrow$
%%\end{minipage}
%%\end{proofBox}
%
%
%\begin{proofBox}{white}{black}
%\footnotesize
%\begin{minipage}{0.75\textwidth}
%\begin{lstlisting}[language=Chainmail, mathescape=true, frame=single]
%a:Account $\wedge$ a.bal=x $\wedge$ b:Bank $\wedge$ $\encapsulates{\prg{b}}{\prg{a}}$
%  to a.bal $\neq$ x
%  onlyThrough $\changes{\_}{\prg{a.bal=x}}$
%\end{lstlisting}
%\end{minipage}
%\begin{minipage}{0.24\textwidth}
%\scriptsize
%\hfill by \textsc{Changes} and $\longrightarrow$
%\end{minipage}
%\begin{minipage}{0.75\textwidth}
%\begin{lstlisting}[language=Chainmail, mathescape=true]
%a:Account $\wedge$ a.bal=x $\wedge$ b:Bank $\wedge$ $\encapsulates{\prg{b}}{\prg{a}}$
%  to a.bal $\neq$ x
%  onlyThrough $\encapsulates{\prg{b}}{\prg{a}}$ $\wedge$ 
%     ($\changes{X_{\{\prg{b}\}}}{\prg{a.bal=x}}$ $\vee$ $\changes{X_{\{\prg{b}\}}}{\encapsulates{\prg{b}}{\prg{a}}}$)
%\end{lstlisting}
%\end{minipage}
%\begin{minipage}{0.24\textwidth}
%\scriptsize
%\hfill by \textsc{Encap-Write} and $\longrightarrow$
%\end{minipage}
%\begin{minipage}{0.75\textwidth}
%\begin{lstlisting}[language=Chainmail, mathescape=true]
%a:Account $\wedge$ a.bal=x $\wedge$ b:Bank $\wedge$ $\encapsulates{\prg{b}}{\prg{a}}$
%  to a.bal $\neq$ x
%  onlyThrough $\encapsulates{\prg{b}}{\prg{a}}$ $\wedge$ $\changes{\I}{\prg{a.bal=x}}$ $\vee$ false
%\end{lstlisting}
%\end{minipage}
%\begin{minipage}{0.24\textwidth}
%\scriptsize
%\hfill by \prg{ResultA}, \textsc{Encap-}$\cup$, and $\longrightarrow$
%\end{minipage}
%\begin{minipage}{0.75\textwidth}
%\begin{lstlisting}[language=Chainmail, mathescape=true]
%a:Account $\wedge$ a.bal=x $\wedge$ b:Bank $\wedge$ $\encapsulates{\prg{b}}{\prg{a}}$
%  to a.bal $\neq$ x
%  onlyThrough $\encapsulates{\prg{b}}{\prg{a}}$ $\wedge$ 
%              $\exists$ a' o. [a':Account $\wedge$ $\external{\prg{o}}$ $\wedge$ 
%                          ($\calls{\prg{o}}{\prg{a}}{\prg{transfer}}{\prg{a'}, \_}$ $\vee$ 
%                           $\calls{\prg{o}}{\prg{a'}}{\prg{transfer}}{\prg{a}, \_}$)] $\vee$
%              $\exists$ b'. [b':Bank $\wedge$ $\calls{\_}{\prg{b'}}{\prg{transfer}}{\_}$ $\wedge$ 
%                      (b'.accs.acc1=a $\vee$ b'.accs.acc2=a)])]
%\end{lstlisting}
%\end{minipage}
%\begin{minipage}{0.24\textwidth}
%\scriptsize
%\hfill by \prg{BalChange}, \textsc{Changes-Int}, and $\longrightarrow$
%\end{minipage}
%\begin{minipage}{0.75\textwidth}
%\begin{lstlisting}[language=Chainmail, mathescape=true]
%a:Account $\wedge$ a.bal=x $\wedge$ b:Bank $\wedge$ $\encapsulates{\prg{b}}{\prg{a}}$
%  to a.bal $\neq$ x
%  onlyThrough $\encapsulates{\prg{b}}{\prg{a}}$ $\wedge$ 
%              ($\external{\prg{this}}$ $\wedge$ $\access{\prg{this}}{\prg{a}}$) $\vee$
%              b':Bank $\wedge$ $\calls{\_}{\prg{b'}}{\prg{transfer}}{\_}$ $\wedge$ 
%              (b'.accs.acc1=a $\vee$ b'.accs.acc2=a)])
%\end{lstlisting}
%\end{minipage}
%\begin{minipage}{0.24\textwidth}
%\scriptsize
%\hfill by \textsc{Exists}, \textsc{Calls-Recv}, \textsc{Calls-Args}, \textsc{Calls-}\prg{this}, and $\longrightarrow$
%\end{minipage}
%\begin{minipage}{0.75\textwidth}
%\begin{lstlisting}[language=Chainmail, mathescape=true]
%a:Account $\wedge$ a.bal=x $\wedge$ b:Bank $\wedge$ $\encapsulates{\prg{b}}{\prg{a}}$
%  to a.bal $\neq$ x
%  onlyThrough $\encapsulates{\prg{b}}{\prg{a}}$ $\wedge$ 
%              ($\encapsulates{\prg{b}}{\prg{this}}$ $\vee$ this $\in$ {b}) $\vee$
%              b':Bank $\wedge$ $\calls{\_}{\prg{b'}}{\prg{transfer}}{\_}$ $\wedge$ 
%              (b'.accs.acc1=a $\vee$ b'.accs.acc2=a)])
%\end{lstlisting}
%\end{minipage}
%\begin{minipage}{0.24\textwidth}
%\scriptsize
%\hfill by \textsc{Encap-Access}
%\end{minipage}
%\begin{minipage}{0.75\textwidth}
%\begin{lstlisting}[language=Chainmail, mathescape=true]
%a:Account $\wedge$ a.bal=x $\wedge$ b:Bank $\wedge$ $\encapsulates{\prg{b}}{\prg{a}}$
%  to a.bal $\neq$ x
%  onlyThrough $\encapsulates{\prg{b}}{\prg{a}}$ $\wedge$
%              b':Bank $\wedge$ $\calls{\_}{\prg{b'}}{\prg{transfer}}{\_}$ $\wedge$ 
%              (b'.accs.acc1=a $\vee$ b'.accs.acc2=a)])
%\end{lstlisting}
%\end{minipage}
%\begin{minipage}{0.24\textwidth}
%\scriptsize
%\hfill by \textsc{Encap-}\prg{this} and $\longrightarrow$
%\end{minipage}
%\begin{minipage}{0.75\textwidth}
%\begin{lstlisting}[language=Chainmail, mathescape=true]
%a:Account $\wedge$ a.bal=x $\wedge$ b:Bank $\wedge$ $\encapsulates{\prg{b}}{\prg{a}}$
%  to a.bal $\neq$ x
%  onlyThrough $\encapsulates{\prg{b}}{\prg{a}}$ $\wedge$
%              b':Bank $\wedge$ $\calls{\_}{\prg{b'}}{\prg{transfer}}{\_}$ $\wedge$ 
%              b'=b
%\end{lstlisting}
%\end{minipage}
%\begin{minipage}{0.24\textwidth}
%\scriptsize
%\hfill by \prg{BankEncapUnique} and $\longrightarrow$
%\end{minipage}
%\begin{minipage}{0.75\textwidth}
%\begin{lstlisting}[language=Chainmail, mathescape=true]
%Result $\triangleq$ a:Account $\wedge$ a.bal=x $\wedge$ b:Bank $\wedge$ $\encapsulates{\prg{b}}{\prg{a}}$
%           to a.bal $\neq$ x
%           onlyThrough $\calls{\_}{\prg{b}}{\prg{transfer}}{\_}$
%\end{lstlisting}
%\end{minipage}
%\begin{minipage}{0.24\textwidth}
%\scriptsize
%\hfill by $\longrightarrow$
%\end{minipage}
%\end{proofBox}
