\documentclass[11pt]{amsart}
\usepackage{geometry}                % See geometry.pdf to learn the layout options. There are lots.
\geometry{letterpaper}                   % ... or a4paper or a5paper or ... 
%\geometry{landscape}                % Activate for for rotated page geometry
%\usepackage[parfill]{parskip}    % Activate to begin paragraphs with an empty line rather than an indent
\usepackage{graphicx}
\usepackage{amssymb}
\usepackage{epstopdf}
\usepackage{definitions}

\newcommand{\rev}[1]{\emph #1}
\newcommand{\us}[1]{\bf #1}
\DeclareGraphicsRule{.tif}{png}{.png}{`convert #1 `dirname #1`/`basename #1 .tif`.png}

\usepackage{filecontents}
\begin{filecontents}{Response1.bib}
@article{Grossman,
author = {Grossman, Shelly and Abraham, Ittai and Golan-Gueta, Guy and Michalevsky, Yan and Rinetzky, Noam and Sagiv, Mooly and Zohar, Yoni},
title = {Online Detection of Effectively Callback Free Objects with Applications to Smart Contracts},
year = {2017},
issue_date = {January 2018},
publisher = {Association for Computing Machinery},
address = {New York, NY, USA},
volume = {2},
number = {POPL},
url = {https://doi.org/10.1145/3158136},
doi = {10.1145/3158136},
journal = {Proc. ACM Program. Lang.},
month = {dec},
articleno = {48},
numpages = {28},
keywords = {Program analysis, Modular reasoning, Smart contracts}
}
@article{Albert,
author = {Albert, Elvira and Grossman, Shelly and Rinetzky, Noam and Rodr\'{\i}guez-N\'{u}\~{n}ez, Clara and Rubio, Albert and Sagiv, Mooly},
title = {Taming Callbacks for Smart Contract Modularity},
year = {2020},
issue_date = {November 2020},
publisher = {Association for Computing Machinery},
address = {New York, NY, USA},
volume = {4},
number = {OOPSLA},
url = {https://doi.org/10.1145/3428277},
doi = {10.1145/3428277},
journal = {Proc. ACM Program. Lang.},
month = {nov},
articleno = {209},
numpages = {30},
keywords = {blockchain, program verification, program analysis, logic and verification, smart contracts, invariants}
}

@article{Permenev,
  title={VerX: Safety Verification of Smart Contracts},
  author={Anton Permenev and Dimitar I. Dimitrov and Petar Tsankov and Dana Drachsler-Cohen and Martin T. Vechev},
  journal={2020 IEEE Symposium on Security and Privacy (SP)},
  year={2020},
  pages={1661-1677}
  }
  
  @inproceedings{Drossopoulou,
author="Drossopoulou, Sophia and Noble, James and Mackay, Julian and Eisenbach, Susan",
editor="Wehrheim, Heike and Cabot, Jordi",
title="Holistic Specifications for Robust Programs",
booktitle="Fundamental Approaches to Software Engineering",
year="2020",
publisher="Springer International Publishing",
address="Cham",
pages="420--440",
}

@misc{Chlipala,
  author = "Adam Chlipala",
  title = "Certified Programming with Dependent Types",
 url = "http://adam.chlipala.net/cpdt/",
 year = 2019,
 date = "2021-12-02",
}

%@misc{Chlipala,
%  author = "Adam Chlipala",
%  title = "frap: Formal Reasoning about Programs",
% url = "https://github.com/achlipala/frap",
% year = 2019,
% date = "2021-12-02",
%}

\end{filecontents}


\usepackage{natbib}

\title{OOPSLA 2022 Round 2 Resubmission Changes}
%\author{The Author}
%\date{3 December 2021}     

\newcommand\multibrace[3]{\rdelim\}{#1}{3mm}[\pbox{#2}{#3}]}

\newcommand{\kjx}[1]{{\color{orange}{#1}}}
\newcommand{\scd}[1]{{\color{blue}{#1}}}
%\newcommand{\sdN}[1]{{\color{dkgreen}{#1}}}
%\newcommand{\jm}[1]{{\color{magenta}{JM: #1}}}
\newcommand{\sdcomment}[1]{{\ensuremath{\blacksquare}}\footnote{\color{dkgreen}{SD: #1}}}
\newcommand{\secomment}[1]{{\ensuremath{\blacksquare}}\footnote{\se{#1}}}
\newcommand{\jncomment}[1]{{\ensuremath{\blacksquare}}\footnote{\kjx{#1}}}

\newcommand{\sd}[1]{{\color{blue}{#1}}}
 \newcommand{\tobyM}[1]{#1} %[1]{{\color{purple}{Toby: #1}}}
\newcommand{\se}[1]{{\color{green}{#1}}}


\newcommand{\ponders}[3]{\marginpar{\tiny\itshape\raggedright\textcolor{#2}{\textbf{#1:} #3}}\ignorespaces}
\marginparwidth=1.6cm \marginparsep=0cm
\newcommand{\TODO}[1]{} % {{\color{red}#1}}
\newcommand{\sophia}[1]{{\color{blue}#1}}
\newcommand{\toby}[1]{} % {\ponders{Toby}{purple}{#1}}
\newcommand{\susan}[2][]{\ponders{Susan}{brown}{#1} \textcolor{brown}{#2}\xspace}
\newcommand{\james}[1]{\ponders{James}{orange}{#1}}
\newcommand{\jm}[2][]{\ponders{Julian}{magenta}{#1} \textcolor{magenta}{#2}\xspace}
\newcommand{\mrr}[2][]{\ponders{Matthew Ross}{offblue}{{#1}} \textcolor{offblue}{{#2}}\xspace}
\newcommand{\mrrz}[1]{\textcolor{offblue}{{#1}}\xspace}
\newcommand{\Mrr}[2][]{\ponders{Matthew Ross}{teal}{{#1}} \textcolor{teal}{{#2}}\xspace}
\newcommand{\Mrrz}[1]{\textcolor{teal}{{#1}}\xspace}

\newcommand{\sophiaPonder}[2][]{\ponders{Sophia}{blue}{#1} \textcolor{blue}{#2}\xspace}
\renewcommand{\sophia}[2][]                                      % Activate to display a given date or no date

\begin{document}
\maketitle
 We detail here what changes we have made to our 
 round 1 submission after feedback and advice by 
 the reviewers. Firstly, the reviewers requested 
 the following changes be made:
 
 \section{Mandatory Changes}

 \begin{description}
 
 \item[1]
 \emph{Clarify novelty/increment over existing work. In particular (but not only) over Chainmail} 
 \begin{itemize}
 \item
 
  	We have rewritten the introduction to more explicitly 
  	address the novelty of our contribution, and how it differs from
  	existing systems such as Chainmail and VerX
  \end{itemize}
 
 \item[2]
 \emph{
 Vastly expand on explanation/intuition of the 
 $\triangleleft$-operator. The explanation from the author response is not at all 
 convincing nor intuitive. It is yet too terse. Both an example and/or a picture 
 would also help greatly. It would also help greatly if later when the operator is 
 used in other definitions, it was explained just exactly how this operator works 
 in those definitions (e.g. semantics of encapsulation).
 }
 \begin{itemize}
 \item
  	We have expanded on the description and intuition of the 
  	$\triangleleft$-operator, providing not only a better description, but
  	an example and a figure demonstrating how it is used for variable rewrites.
 
 \end{itemize}
 
 \item[3]
 \emph{
 Streamline Section 2.4.
 }
 \begin{itemize}
 \item
  	We have significantly restructured and rewrite Section 2.4 (now Section 2.5).
  	We initially followed advice from the reviewers to take a more
  	top down approach to the the description of the system, but we 
  	found that this did not suit the complexity of the system
  	and ultimately obfuscated detail rather than clarify. Instead 
  	we have attempted to capture the intent of the reviewers' 
  	advice, and present the system bottom up, however starting from 
  	a more abstract description that includes a figure demonstrating 
  	the structure Necessity.
  	We feel that Section 2.5 now presents a much clearer picture of our
  	system.
 
 \end{itemize}
 
 \item[4]
 \emph{
 Avoid sudden/unmotivated topic changes.
 }
 \begin{itemize}
 \item
 The reviewers correctly identified that certain
  	aspects of the original submission lacked flow and changed 
  	topic frequently. This mainly occured in the introduction 
  	(with sudden switches to related work), and in the description
  	of our case studies in Section 3. We have reworked the introduction
  	with the reviewers' advice in mind, and significantly reduced the 
  	amount of related work in the introduction, and introduced what 
  	related work remains in a way that does not interfere with the flow 
  	of the paper. We further moved the Expressiveness section (now Section 3.4) that 
  	contained the case studies to the end of Seciton 3, and placed 
  	much of the case studies in Appendix C.
 \end{itemize}
 
 \item[5]
 \emph{
 Comment on calls from internal to external modules.
 }
 \begin{itemize}
 \item
 We have included an explanation that most of the related work does not support external calls \jm[TODO!!]{(lines xxx, yyy)}, and outlined our thoughts on possible solutions in lines \jm[TODO!!]{zzz}.
 \end{itemize}
 
 \item[6]
 \emph{
 The Coq proof should be completed: currently it has so many admits.
 }
 \begin{itemize}
 \item
The reviewers requested that we make the following changes to the Coq proofs:

\begin{description}
\item[P1] 
In the original Coq proof there was an admitted proof in a file that was 
not used by the main Coq model.
We proposed to address this by deleting the file.
\begin{itemize}
\item we have deleted the file
\end{itemize}

\item[P2]  The original Coq proof included some admitted 
proofs of properties of an assumed specification language for
the purposes of proving examples. We agreed to rename these 
proofs to \prg{Hypothesis} per the reviewers' request.
\begin{itemize}
\item we have renamed the the proofs
\end{itemize}

\item[P3] The original Coq formalism included a 
proof involving some variable renaming that we 
had admitted as it was not central to the main results.
The reviewers advised us to complete the proof.
\begin{itemize}
\item We have completed the omitted proof
\end{itemize}

\item[P4] The reviewers advised us to include a description of
each Coq file, and why different properties have been admitted
\begin{itemize}
\item \jm[I still need to finish this]{We have included this description in the readme}

\end{itemize}
\end{description}
 
 \end{itemize}
 
 \item[7]
 \emph{
 The motivating password example is still not realistic. The example does not provide any interface to allow someone to get the correct password, or at least initialize the password. I think the example should be extended in such a way, which may require a more powerful logic. If this doesn't work, the paper should provide another example and clarify the weakness of the logic.
 }
 \begin{itemize}
 \item
 As we already discussed with the reviewers after the rebuttal period, the language Loo does support object initialization. To demonstrate this, we introduced a method init in lines \jm[TODO!!]{xxxx}
 \end{itemize}
 
 \end{description}
 
 \section{Strongly Recommended Changes}

 \begin{description}
 
 \item[8]
 \emph{
 Clarify what you mean by "emergent behavior".
 }
 \begin{itemize}
 \item
We have included a description of \emph{emergent behaviour}
in the new introduction
 
 \end{itemize}
 
 \item[9]
 \emph{
 More clearly separate introduction from related work. In general streamline flow of the paper.
 }
 \begin{itemize}
 \item
We have followed the reviewers' advice, and 
this has greatly improved the flow and conciseness of the introduction.
 
 \end{itemize}
 
 \end{description}
 
 \section{Recommended Changes}
 The reviewers also requested that we address all of the other minor 
 changes requested by each reviewer. Here we detail those changes.

\jm[I still need to do this]{}

\subsection{Reviewer A}

\begin{itemize}
\item
\emph{In l. 36-55, the introduction suddenly becomes/mixes with a part on related work. To my opinion, this completely distracts from the description of the problem that this paper is attacking. I'd rather that this part is moved to the dedicated related work section.}

\item
\emph{l. 68. At this point (and also many points later), it was completely unclear to me what "emergent behaviour" is supposed to mean. }

\item
\emph{
Section 2.4, I must say, I find extremely tedious and very difficult to follow. I firmly believe that this can be streamlined to that one must not mentally follow through 9 steps (a - i). Many notions are also only explained later. E.g. it is not clear (and also does not become apparent from the explanations in Sec. 2.4) why one needs to construct from per-method conditions the single-step conditions. And is this really important in order to get an overview of the approach that the paper is taking?
}

\item
\emph{
.102: I have two problems with the notion of being encapsulated:
   - Maybe a minor comment, actually, but again - I'm sorry - I do not understand the name "encapsulated". The fact that the concept >>only by executing a pice of code C one can invalidate a logical assertion A<< is called >>C encapsulates A<< does not make much sense to me because "encapsulating" - to me - suggests rather that A is somehow part of C or that A is somehow wrapped/surrounded by C.
   }\\
\emph{   
   - More severely, I still cannot get my head around the notion that only by executing C one can invalidate A. Imagine C'=C;x:=x. Obviously C'$\neq$C, but if C can invalidate A, so can C'. How can there be a piece of code C, so that only C but not C;x:=x can invalidate A?
}

\item
\emph{
Section 2.4, I must say, I find extremely tedious and very difficult to follow. I firmly believe that this can be streamlined to that one must not mentally follow through 9 steps (a - i). Many notions are also only explained later. E.g. it is not clear (and also does not become apparent from the explanations in Sec. 2.4) why one needs to construct from per-method conditions the single-step conditions. And is this really important in order to get an overview of the approach that the paper is taking?
}

\item
\emph{
says "Note that our proofs of necessity do not inspect method bodies" This makes absolutely no sense to me. How can I infer - or more: prove - anything about an object C (the code) without looking at C? This needs an explanation. In the explanation that follows you mention pre and postconditions of methods, but how can I prove pre- and postconditions of methods, if I cannot look at the methods?
}

\item
\emph{
l.263 following: At this point I was wondering, which of Mod1, Mod2, Mod3 is internal, what is external? It would be good to refer back to that example and point out to the reader what is supposed to be internal and what external.
}

\item
\emph{
Def. 3.2: Why would one write Arising(M, Y, sigma) iff ... Y; M, $sigma_0$ ... Why flip the order of M and Y? Does that not cause unnecessary confusion? Or is there a good reason to flip the order?
}

\item
\emph{
Def. 3.8: It would be good to ostentatiously clarify that necessity specifications cannot nest, i.e. the nonterminal S does not appear on the right-hand side of the grammar. Only nonterminals A and those come from the language Assert, I suppose.
}

\item
\emph{
l.442-456: It is totally unclear how/why the $\triangleleft$-operator does the trick for you necessity modalities. Part of the reason is that the definition $\triangleleft$ is described/explained in not enough detail (for me). But I also firmly believe that the definition of the semantics of the necessity modalities deserves to be provided some intuition. In particular an intuition of how $\triangleleft$ defines these semantics.
}

\item
\emph{
l.482: I believe that it deserves an explenation why no module satisfies \prg{NecessityBankSpec}$_\prg{c}$. It is not obvious to me.
}
\end{itemize}

\subsection{Reviewer B}

\subsection{Reviewer C}

\subsection{Reviewer D}



\newpage



\section{Copied from hotcrp discussion}


 
 
  We will make all the minor changes suggested by the reviewers.
 \subsection*{External calls}
 We cannot promise a full treatment of external calls by the end of February, but we can share out current thinking:  As a first approach, we will require that the arguments to external calls do not include internal objects, except for the receiver and parameters (thus ensuring that external accessibility of internal methods does not increase); we would rely on the classical pre- and post- conditions of the internal methods -- as we currently do. As a more advanced approach, we will develop extensions to classical Hoare Logics, which would allow us to reason about points in the code where external calls are being made. This would be the first time we could be inspecting the code in the bodies of the functions.
 \subsection*{Novelty}
 We will strengthen our statements about VerX and Chainmail in line with what we said above.
 
 \subsection*{Presentation}
 
 For adaption, access, and encapsulation we will amend the explanations as stated above. Susan: or do you want to discuss Julian's cleaner definition for adaption he sent yesterday???
 
 For emergent behaviour we will include the reviewer's statement and also say that ``(S2) does not take account of the module's \emph{emergent behaviour}. That is, (S2) does not consider the behavior that emerges from the interaction between the 
\texttt{transfer} method, and the other methods of the bank module. What if the module leaks the password?''
 
 We will replace the current Bank Account proof with a simpler Coq proof that matches the straightforward introductory example. We will put the current example in an appendix so that we can 
show reasoning about ghost fields and more complex data structures. 

We will move the clarifying examples to Section 2.

The largest piece of work is the proof and that shouldn't take more than a week so we believe that we can make substantial improvements in presentation before mid January.


\subsubsection*{Reviewer A}
\begin{itemize}
\item Streamline Introduction: we will move the related work to the related work section
\item Rework Section 2 to be clearer: we will make the outline of the proof structure in Section 2 clearer, at a higher level, and more concise.
\item We will change the order of $M$ and $M'$ in the definition of Arising.
\item We will make clear which state is the original state.
\item We will clarify Def. 3.9, provide a clearer description and definition.
\item Ensure consistent usage of Section vs. section.
\end{itemize}

\subsubsection*{Reviewer B}
\begin{itemize}
\item Fix the flow of the paper. Present a ``consistent high-level story``
\item Clarify the differences between Necessity and Chainmail and VerX.
\item Be more explicit about the reasons and justifications for restricting external method calls
\item Provide better names for Mod1, Mod2, Mod3, etc
\item Emphasize the separation of Necessity from the inspection of code.
\item Clarify why shallow access is necessary
\item Restate NecessityBankSpec in 3.4.1
\item Provide better justification, explanation, and intuition for the example specifications in 3.4
\item Include a brief description of the expressiveness earlier in the paper than 3.4.3
\item Explain why the restriction on return values is sufficient in If1-Inside
\end{itemize}

\subsubsection*{Reviewer C}
\begin{itemize}
\item Rephrase the liveness and safety verification in the Section 1.
\item Rewrite Section 2.4. (Julian: Reviewer B appreciated this, but neither C nor A did)
\item Replace Section 5 with the simpler, original bank account example, and move the current one to the appendix.
\end{itemize}


  
\section{Proposed Changes}

\subsection{Unrealistic motivating example}

> The motivating password example is still not realistic. The example does not 
provide any interface to allow someone to get the correct password, or at 
least initialize the password. ... example should be extended in such a way, 
which may require a more powerful logic..*

This is a good point, and we thank the reviewers for pointing out that we should
have expanded:

We are not clear what is meant by "an interface to allow someone to get the 
correct password". Such a function (called, say, `getPassword`) is expressible
in the language, and we could add it to our example. But if the module allowed 
a client to read an account's password without prior knowledge of the password, 
then the module would no longer be robust, because anybody with access to
the account would be able to take the money out of the account. 

On the other hand, \ModC does support initialization of the password, through
execution of the following code

	p1=new Object; a=new Account(); a.setPassword(null,p1)

Namely, object creation, (`new Account()`) initializes the account's password
to `null`, and thus the call `a.setPassword(null,p1)` will succesfully set the password
to `p1`. 

*Our proposal* 

`P5`: We will add explanations as per above, showing how object intialization works 
in our approach. 

------
**Appendix**

More details about point 2, for the interested reviewer.

If we assumed some "unknown" method `m\_unknown`, and "untrusted" 
object `o\_untrust`, and some account `a0` with some money in it, and with password
`p0`, then, the following code

         `a=new Account(); a0.transfer(p0,100); o\_untrust.m\_unknown(a)`

gives \_no\_ guarantee about the call `o\_untrust.m\_unknown(a)` not removing the money in the account `a`, while

        `p1=new Object; a=new Account();  a.setPassword(null,p1);`
        `a0.transfer(p0,100);  o\_untrust.m\_unknown(a)`

\_does\_ guarantee that no money will be removed during the call `o\_untrust.m\_unknown(a)`
-- because no external objects involved in that call have access to `p1`.  

One may now ask how we would initialize the account so that it contains some money, 
without adding methods to class `Account` which make it possible to just create money
out of thin air. We can do this (initialize the money) in the language we have
provided: add a function, which pays money into the account only while the password 
is still `null`. Or, even better, we could follow the approach  from [Mark Samuel Miller, 
Chip Morningstar, and Bill Frantz. 2000. Capability-based Financial Instruments: 
From Object to Capabilities] and have a `Bank` object, with `Account`s belonging 
to that `Bank`. The `Account`s may transfer moneys across 
each other only if they belong to the same `Bank`, and only `Bank`s may create `Account`s 
belonging to them. We could give the code for this, but it would require another page, 
discussing issues that are not central to the contribution of this work. Therefore,
we propose not to do that, but are happy to take the reviewers' guidance.

------
We are looking forward to your answer whether you agree with our proposals `P1`,`P2`,`P3`, `P4` and `P5`, and thank you, again, for your interest.
 



\bibliographystyle{plainnat}
\bibliography{Response1} 


\end{document}  
%
% For emergent behaviour we will include the reviewer's statement and also say that ``(S2) does not take account of the module's \emph{emergent behaviour}. That is, (S2) does not consider the behavior that emerges from the interaction between the 
%\texttt{transfer} method, and the other methods of the bank module. What if the module leaks the password?''
% 
% We will replace the current Bank Account proof with a simpler Coq proof that matches the straightforward introductory example. We will put the current example in an appendix so that we can 
%show reasoning about ghost fields and more complex data structures. 
%
%We will move the clarifying examples to Section 2.
%
%The largest piece of work is the proof and that shouldn't take more than a week so we believe that we can make substantial improvements in presentation before mid January.
%
%
%
% 
%
%
% 
% %We propose the following amended explanation to clarify both it's importance, and it's meaning:
%
% 
% A list of the changes that you plan to make in
%  response to the reviews and the timeline for those changes.
%  
% 
%  
%\section{Response} A reviewer-by-reviewer list of answers to questions
%  with context extracted from the reviews. Use markdown syntax.
%
%\bibliographystyle{plainnat}
%\bibliography{Response1} 
%
%
%\end{document}  