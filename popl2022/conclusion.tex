\section{Conclusion}
\label{s:conclusion}

Bad things can happen to good programs. In an open world, every
accessible API is an attack surface: and worse, every combination of
API calls is a potential attack.  Developers can no longer consider
components in splendid isolation, verifying the sufficient pre and
postconditions of each method, but must reckon with the emergent
behaviour of entire software systems. In practice, this means they
need to identify the necessary conditions under which anything could
happen\cite{anything} --- good things and bad things alike --- and
then ensure the necessary conditions for bad things happening never arise.


%% A module written to be used with other code needs to ensure that no
%% other code may use it and cause unintended effects.
%% Developers whose
%% modules are going to be used in the wild, need the language to
%% constrain the usage of their code, and the capability to prove that
%% their code will always meet these constraints.

%This paper defines \Nec, which takes an holistic view of modules. 
Unlike classical specifications, \Nec does not focus on individual pieces of code.  
Rather it can describe the emergent behaviour of all of a module's
classes and all their methods.
Using \Nec,
programmers can write specifications defining the necessary
conditions for effects --- things happening --- in programs.
\Nec treats program effects as actions:
that is, as transitions from states satisfying some assertion 
$A_1$ to other states satisfying assertions $A_2$. 
\Nec specifications then permit those transitions
only if some other assertion $A_3$ holds beforehand;
or only if $A_3$ holds and $A_2$ is reached in a single step;
or only if $A_3$ holds at some point on the path between
$A_1$ and $A_2$.
\susan[This called the assertion language A - should it be Basicspec or just skip the name as I have done?]{}
The assertion language supports the usual expressions about the
values of variables, the contents of the heap, and predicates
to capture provenance and permission.

%% doesn't need to be said here
%%
%% Modules can satisfy Necessity Specifications only if
%% the underlying programming language and the runtime system 
%% can give some basic guarantees:
%% memory safety (addresses cannot be forged), 
%% and some notion of privacy (here fields are module-private, i.e. can be read and written only
%% by methods of the same module). Different notions of privacy 
%% (e.g. fields to be class-private,
%% or object-private, and fields to be private only if explicitly defined to be so)  would
%% also be sufficient to develop code that satisfies such Necessity
%% Specifications.
 
We have developed a \susan[We use logic throughout the paper and here we say we don't have our own logic]{} \jm[]{proof} system to prove that modules meet their specifications.  Our \jm[]{proof} system exploits the
sufficient conditions of classical method specifications (pre and
postconditions) to infer per method \Nec specifications, and
then generalise those to cover any single execution step. By
combining single step specifications, programs can describe
components' emergent behaviour --- i.e. the necessary conditions for
program effects, irrespective of how that effect is caused.  Finally,
we have proved our \jm[]{proof} system is sound, and then used it to
prove the specification of a bank account example.

%
%% these describe the necessary conditions for given effects
%% which take place after the execution of a known method call. 
%% Per-method Necessity Specifications can be generalized to one-step
%%  Necessity Specifications; these describe necessary conditions for
%%  given effects  which take place after the execution of one, 
%%  unknown, single, step. 
%%  Lastly, one-step
%% Necessity Specifications are combined with other \NecessitySpecifications
%%  to describe the emergent behaviour, ie  necessary conditions for some effect which
%%  may have taken place over many, unknown, internal or external steps.
%
%
%

%%the following feels like discussions but I reworded a biut.
%
%
Deriving per method \Nec specifications from classical method
pre and postconditions has two advantages: First, it means that we
did not need to develop a special purpose logic for that task. Second,
it means that all modules that have the same classical per method
specifications can be proven to satisfy the same Necessity
Specifications using the \emph{same} proof.  This holds even when the
classical specifications are ``similar enough''.  For instance, the
modules \prg{Mod1} and \prg{Mod3} from the introduction, and also
module \jm[]{\prg{Mod4}} from Sec. \ref{s:examples} satisfy
\prg{NecessityBankSpec}.  The proofs for these three modules differ
slightly in the proof of encapsulation of the assertion
\prg{a.password=pwd} (step \textbf(A)} in Sec. \ref{s:examples}),
  and in the proof of the per-method \Nec specification for
  \prg{transfer}, but otherwise are identical.


%%the following feels like discussions but I reworded a bit but...

We considered developing a bespoke logic to infer per method \Nec
specifications.  Such a logic might be more powerful than the
\Nec \jm[]{proof} system; we aim to consider that in further work.
Moreover, the classical specifications we are using to infer the per
method \Nec specifications are very ``basic'', and thus they need
to explicitly state what has not been modified -- this makes proofs
 cumbersome. In future work, we plan to start from classical
specifications which have more information about the affected part of
the state (e.g. using \prg{modifies}-clauses, or implicit frames
\cite{Leavens-etal07x,IshtiaqOHearn01,IDF,MattAlex,dafny13}) as we expect this could
make the step from classical specifications to per method \Nec
specifications more convenient.

Our \jm[]{proof} system is parametric with an algorithmic judgment which
can prove that an assertion is encapsulated. In this paper we have
used a rudimentary, type-based \jm[]{proof} system, but we aim to develop
a logic for inferring such assertions. The concept of encapsulation in
this work is very coarse, and based on the classes of objects. In the
future, we plan to refine our handling of encapsulation, and support
reasoning about more flexible configurations of objects.  Similarly,
to facilitate the formal treatment, we currently forbid internal
objects to call into external objects: we hope a better model of
encapsulation will let us remove this restriction.


To conclude: bad things can always happen to good programs. \Nec
specifications are necessary to make sure good programs don't do bad
things in response.
