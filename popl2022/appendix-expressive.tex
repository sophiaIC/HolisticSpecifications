\section{More about the Expressiveness of \Nec Specifications}
\label{s:expressiveness:appendix}

 %% We continue the comparison of expresiveness between \emph{Chainmail} and \Nec, by 
 %% considering the examples studied in \cite{FASE}.
 
\subsection{ERC20}
The ERC20 \cite{ERC20} is a widely used token standard describing the basic functionality of any Ethereum-based token 
contract. This functionality includes issuing tokens, keeping track of tokens belonging to participants, and the 
transfer of tokens between participants. Tokens may only be transferred if there are sufficient tokens in the 
participant's account, and if either they (using the \prg{transfer} method) or someone authorized by the participant (using the \prg{transferFrom} method) initiated the transfer. 

We specify these necessary conditions here using \Nec. Firstly, \prg{ERC20Spec1} 
says that if the balance of a participant's account is ever reduced by some amount $m$, then
that must have occurred as a result of a call to the \prg{transfer} method with amount $m$ by the participant,
or the \prg{transferFrom} method with the amount $m$ by some other participant.
\begin{lstlisting}[language = Chainmail, mathescape=true, frame=lines]
ERC20Spec1 $\triangleq$ from e : ERC20 $\wedge$ e.balance(p) = m + m' $\wedge$ m > 0
              next e.balance(p) = m'
              onlyIf $\exists$ p' p''.[$\calls{\prg{p'}}{\prg{e}}{\prg{transfer}}{\prg{p, m}}$ $\vee$ 
                     e.allowed(p, p'') $\geq$ m $\wedge$ $\calls{\prg{p''}}{\prg{e}}{\prg{transferFrom}}{\prg{p', m}}$]
\end{lstlisting}
Secondly, \prg{ERC20Spec2} specifies under what circumstances some participant \prg{p'} is authorized to 
spend \prg{m} tokens on behalf of \prg{p}: either \prg{p} approved \prg{p'}, \prg{p'} was previously authorized,
or \prg{p'} was authorized for some amount \prg{m + m'}, and spent \prg{m'}.
\begin{lstlisting}[language = Chainmail, mathescape=true, frame=lines]
ERC20Spec2 $\triangleq$ from e : ERC20 $\wedge$ p : Object $\wedge$ p' : Object $\wedge$ m : Nat
              next e.allowed(p, p') = m
              onlyIf $\calls{\prg{p}}{\prg{e}}{\prg{approve}}{\prg{p', m}}$ $\vee$ 
                     (e.allowed(p, p') = m $\wedge$ 
                      $\neg$ ($\calls{\prg{p'}}{\prg{e}}{\prg{transferFrom}}{\prg{p, \_}}$ $\vee$ 
                              $\calls{\prg{p}}{\prg{e}}{\prg{allowed}}{\prg{p, \_}}$)) $\vee$
                     $\exists$ p''. [e.allowed(p, p') = m + m' $\wedge$ $\calls{\prg{p'}}{\prg{e}}{\prg{transferFrom}}{\prg{p'', m'}}$]
\end{lstlisting}

\subsection{DAO}
The Decentralized Autonomous Organization (DAO)~\cite{Dao}  is a well-known Ethereum contract allowing 
participants to invest funds. The DAO famously was exploited with a re-entrancy bug in 2016, 
and lost \$50M. Here we provide specifications that would have secured the DAO against such a 
bug. \prg{DAOSpec1} says that no participant's balance may ever exceed the ether remaining 
in DAO.
\begin{lstlisting}[language = Chainmail, mathescape=true, frame=lines]
DAOSpec1 $\triangleq$ from d : DAO $\wedge$ p : Object
            to d.balance(p) > d.ether
            onlyIf false
\end{lstlisting}
Note that \prg{DAOSpec1} enforces a class invariant of \prg{DAO}, something that could be enforced
by traditional specifications using class invariants.
The second specification \prg{DAOSpec2} states that if after some single step of execution, a participant's balance is \prg{m}, then 
either 
\begin{description}
\item[(a)] this occurred as a result of joining the DAO with an initial investment of \prg{m}, 
\item[(b)] the balance is \prg{0} and they've just withdrawn their funds, or 
\item[(c) ]the balance was \prg{m} to begin with
\end{description}
\begin{lstlisting}[language = Chainmail, mathescape=true, frame=lines]
DAOSpec2 $\triangleq$ from d : DAO $\wedge$ p : Object
            next d.balance(p) = m
            onlyIf $\calls{\prg{p}}{\prg{d}}{\prg{repay}}{\prg{\_}}$ $\wedge$ m = 0 $\vee$ $\calls{\prg{p}}{\prg{d}}{\prg{join}}{\prg{m}}$ $\vee$ d.balance(p) = m
\end{lstlisting}

\sophiaPonder[small changes over Julian's]{\subsection{Safe}
\cite{FASE} used as a running example   a Safe, where a treasure 
was secured within a \texttt{Safe} object, and access to the treasure was only granted by 
providing the correct password. }
\ Using \Nec, we express \texttt{SafeSpec}, that requires that the treasure cannot be 
removed from the safe without knowledge of the secret.
\begin{lstlisting}[language = Chainmail, mathescape=true, frame=lines]
SafeSpec $\triangleq$ from s : Safe $\wedge$ s.treasure != null
            to s.treasure == null
            onlyIf $\neg$ inside(s.secret)
\end{lstlisting}

The module  \prg{SafeModule} described  below satisfies  \prg{SafeSpec}.

\begin{lstlisting}[frame=lines]
module SafeModule
     class Secret{}
     class Treasure{}
     class Safe{
         field treasure : Treasure
         field secret : Secret
         method take(scr : Secret){
              if (this.secret==scr) then {
                   t=treasure
                   this.treasure = null
                   return t } 
          }
 }
\end{lstlisting}

\subsection{Crowdsale}
\jm[]{\Nec is able to encode the motivating example of \citeasnoun{VerX}: 
an escrow smart contract that ensures that the contract may not be coerced to 
pay out or refund more money than has been raised.
The motivating \prg{Crowdsale} example consists of a \prg{Crowdsale} contract 
for crowd sourcing funding. A \prg{Crowdsale} object consists of an \prg{Escrow} object,
an amount raised, a funding goal, and a closing time in which the goal must be met for 
the fund to be successful. An \prg{Escrow} consists of a ledger of investors and how much
they have invested. There are several properties that \citeasnoun{VerX} sought to encode,
and we have provided the encoding of those specifications in Fig. \ref{f:verx:encoding}.
\prg{R0} states that if an investor claims a refund from an escrow, then the balance of 
the escrow decreases by the amount the investor had deposited in the escrow. 
\prg{R1} states that if at anytime the escrow has not yet succeeded, then the deposits must
be less than the balance of the escrow. 
\prg{R2\_1} and \prg{R2\_2} combine to express a single property: no one may ever withdraw and 
then subsequently claim a refund or visa versa.
\prg{R3} states that if the funding goal is ever met, then no one may subsequently claim a refund.}

\begin{figure}[htb]
\begin{lstlisting}[language=chainmail]
class Crowdsale {
Escrow escrow;
  closeTime, raised, goal : int;
  method init() {
    if escrow == null
      escrow := new Escrow(new Object);
    	  closeTime := now + 30 days;
    	  raised := 0;
    	  goal := 10000 * 10**18;
  }
  method invest(investor : Object, value : int) {
    if raised < goal
      escrow.deposit(investor, value);
      raised += value;
  }
  method close() {
    if now > closeTime || raised >= goal
      if raised >= goal
        escrow.close();
      else
        escrow.refund();
  }
}
\end{lstlisting}
\caption{Crowdsale Contract}
\label{f:verx:crowdsale}
\end{figure}

\begin{figure}[htb]
\begin{lstlisting}[language=chainmail]
confined class Escrow {
  owner, beneficiary : Object;
  mapping(Object => uint256) deposits;
  OPEN, SUCCESS, REFUND : Object;
  state : Object;
  method init(o : Object, b : Object) {
    if owner == null || beneficiary == null
      owner := o;
      beneficiary := b;
      OPEN := new Object; SUCCESS := new Object; REFUND := new Object;
      state := OPEN;
      
  method close() {state = SUCCESS;}
  method refund() {state = REFUND;}
  method deposit(value : int, p : Object) {
    deposits[p] := deposits[p] + value;
  }
  method withdraw() {
    if state == SUCCESS
      return this.balance;
  }
  method claimRefund(p : Object) {
    if state == REFUND
      int amount := deposits[p];
      deposits[p] := 0;
      return amount;
  }
}
\end{lstlisting}
\caption{Escrow Contract}
\label{f:verx:escrow}
\end{figure}

\begin{figure}[htb]
\begin{lstlisting}[mathescape=true, language=chainmail]
(R0) $\triangleq$ e : Escrow $\wedge$ $\calls{\_}{\prg{e}}{\prg{claimRefund}}{\prg{p}}$
          next e.balance = nextBal onlyIf nextBal = e.balance - e.deposits(p)
(R1) $\triangleq$ e : Escrow $\wedge$ e.state $\neq$ e.SUCCESS $\longrightarrow$ sum(deposits) $\leq$ e.balance
(R2_1) $\triangleq$ e : Escrow $\wedge$ $\calls{\_}{\prg{e}}{\prg{withdraw}}{\prg{\_}}$
           to $\calls{\_}{\prg{e}}{\prg{claimRefund}}{\prg{\_}}$ onlyIf false
(R2_2) $\triangleq$ e : Escrow $\wedge$ $\calls{\_}{\prg{e}}{\prg{claimRefund}}{\prg{\_}}$
           to $\calls{\_}{\prg{e}}{\prg{withdraw}}{\prg{\_}}$ onlyIf false
(R3) $\triangleq$ c : Crowdsale $\wedge$ sum(deposits) $\geq$ c.escrow.goal
         to $\calls{\_}{\prg{c.escrow}}{\prg{claimRefund}}{\prg{\_}}$ onlyIf false
\end{lstlisting}
\caption{Encoding VerX Crowdsale Example in Necessity}
\label{f:verx:encoding}
\end{figure}


 
