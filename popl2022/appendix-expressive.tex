\section{More about the Expressiveness of \Nec Specifications}
\label{s:expressiveness:appendix}

 We continue the comparison of expresiveness between \emph{Chainmail} and \Nec, by 
 considering the examples studied in \cite{FASE}.
 
\subsubsection{ERC20}
The ERC20 \cite{ERC20} is a widely used token standard describing the basic functionality of any Ethereum-based token 
contract. This functionality includes issuing tokens, keeping track of tokens belonging to participants, and the 
transfer of tokens between participants. Tokens may only be transferred if there are sufficient tokens in the 
participant's account, and if either they (using the \prg{transfer} method) or someone authorized by the participant (using the \prg{transferFrom} method) initiated the transfer. 

We specify these necessary conditions here using \Nec. Firstly, \prg{ERC20Spec1} 
says that if the balance of a participant's account is ever reduced by some amount $m$, then
that must have occurred as a result of a call to the \prg{transfer} method with amount $m$ by the participant,
or the \prg{transferFrom} method with the amount $m$ by some other participant.
\begin{lstlisting}[language = Chainmail, mathescape=true, frame=lines]
ERC20Spec1 $\triangleq$ from e : ERC20 $\wedge$ e.balance(p) = m + m' $\wedge$ m > 0
              next e.balance(p) = m'
              onlyIf $\exists$ p' p''.[$\calls{\prg{p'}}{\prg{e}}{\prg{transfer}}{\prg{p, m}}$ $\vee$ 
                     e.allowed(p, p'') $\geq$ m $\wedge$ $\calls{\prg{p''}}{\prg{e}}{\prg{transferFrom}}{\prg{p', m}}$]
\end{lstlisting}
Secondly, \prg{ERC20Spec2} specifies under what circumstances some participant \prg{p'} is authorized to 
spend \prg{m} tokens on behalf of \prg{p}: either \prg{p} approved \prg{p'}, \prg{p'} was previously authorized,
or \prg{p'} was authorized for some amount \prg{m + m'}, and spent \prg{m'}.
\begin{lstlisting}[language = Chainmail, mathescape=true, frame=lines]
ERC20Spec2 $\triangleq$ from e : ERC20 $\wedge$ p : Object $\wedge$ p' : Object $\wedge$ m : Nat
              next e.allowed(p, p') = m
              onlyIf $\calls{\prg{p}}{\prg{e}}{\prg{approve}}{\prg{p', m}}$ $\vee$ 
                     (e.allowed(p, p') = m $\wedge$ 
                      $\neg$ ($\calls{\prg{p'}}{\prg{e}}{\prg{transferFrom}}{\prg{p, \_}}$ $\vee$ 
                              $\calls{\prg{p}}{\prg{e}}{\prg{allowed}}{\prg{p, \_}}$)) $\vee$
                     $\exists$ p''. [e.allowed(p, p') = m + m' $\wedge$ $\calls{\prg{p'}}{\prg{e}}{\prg{transferFrom}}{\prg{p'', m'}}$]
\end{lstlisting}

\subsubsection{DAO}
The Decentralized Autonomous Organization (DAO)~\cite{Dao}  is a well-known Ethereum contract allowing 
participants to invest funds. The DAO famously was exploited with a re-entrancy bug in 2016, 
and lost \$50M. Here we provide specifications that would have secured the DAO against such a 
bug. \prg{DAOSpec1} says that no participant's balance may ever exceed the ether remaining 
in DAO.
\begin{lstlisting}[language = Chainmail, mathescape=true, frame=lines]
DAOSpec1 $\triangleq$ from d : DAO $\wedge$ p : Object
            to d.balance(p) > d.ether
            onlyIf false
\end{lstlisting}
Note that \prg{DAOSpec1} enforces a class invariant of \prg{DAO}, something that could be enforced
by traditional specifications using class invariants.
The second specification \prg{DAOSpec2} states that if after some single step of execution, a participant's balance is \prg{m}, then 
either 
\begin{description}
\item[(a)] this occurred as a result of joining the DAO with an initial investment of \prg{m}, 
\item[(b)] the balance is \prg{0} and they've just withdrawn their funds, or 
\item[(c) ]the balance was \prg{m} to begin with
\end{description}
\begin{lstlisting}[language = Chainmail, mathescape=true, frame=lines]
DAOSpec2 $\triangleq$ from d : DAO $\wedge$ p : Object
            next d.balance(p) = m
            onlyIf $\calls{\prg{p}}{\prg{d}}{\prg{repay}}{\prg{\_}}$ $\wedge$ m = 0 $\vee$ $\calls{\prg{p}}{\prg{d}}{\prg{join}}{\prg{m}}$ $\vee$ d.balance(p) = m
\end{lstlisting}

\sophiaPonder[small changes over Julian's]{\subsubsection{Safe}
\cite{FASE} used as a running example   a Safe, where a treasure 
was secured within a \texttt{Safe} object, and access to the treasure was only granted by 
providing the correct password. }
\ Using \Nec, we express \texttt{SafeSpec}, that requires that the treasure cannot be 
removed from the safe without knowledge of the secret.
\begin{lstlisting}[language = Chainmail, mathescape=true, frame=lines]
SafeSpec $\triangleq$ from s : Safe $\wedge$ s.treasure != null
            to s.treasure == null
            onlyIf $\neg$ inside(s.secret)
\end{lstlisting}

The module  \prg{SafeModule} described  below satisfies  \prg{SafeSpec}.

\begin{lstlisting}[frame=lines]
module SafeModule
     class Secret{}
     class Treasure{}
     class Safe{
         field treasure : Treasure
         field secret : Secret
         method take(scr : Secret){
              if (this.secret==scr) then {
                   t=treasure
                   this.treasure = null
                   return t } 
          }
 }
\end{lstlisting}

 
