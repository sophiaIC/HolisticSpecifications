
%\subsection{Proving Encapsulation}
\label{s:encap-proof}

%We start by giving providing the syntax for type contexts in Fig. \ref{f:context-syntax}.
%\begin{figure}[t]
%\[
%\begin{syntax}
%\syntaxElement{\Gamma}{Type Context}
%		{
%		\syntaxline
%				{\emptyset}
%				{\alpha : C,\ \Gamma}
%		\endsyntaxline
%		}
%\endSyntaxElement\\
%\end{syntax}
%\]
%\caption{}
%\label{f:context-syntax}
%\end{figure}
%We construct type contexts out of assertions using the following rules:
%\begin{mathpar}
%\infer
%		{}
%		{\textit{Env}(\alpha : C) = \alpha : C,\ \emptyset}
%		\and
%\infer
%		{}
%		{\textit{Env}(A_1\ \wedge\ A_2) = \textit{Env}(A_1) \cup \textit{Env}(A_2)}
%\end{mathpar}
%\begin{definition}[Assertion Encapsulation]
%For all modules $M$, and assertions $A$, and $A'$ we say $M\ \vdash\ A\ \Rightarrow\ A'$ if and only if M
%\end{definition}

\kjx{
Assertion encapsulation (Definition \ref{def:encapsulation}) is
critical to our approach.  Assertion encapsulation ensures that a
change in satisfaction of an assertion can only depend on computation
\textit{internal} to the module in which the assertion is encapsulated
--- this is related to the footprint of an
assertion \cite{objInvars,encaps}.
If the footprint of an assertion is contained 
within a module, then that assertion is encapsulated,
however there are assertions that are encapsulated by a module 
whose footprint is not contained within the module. 
Specifically, the assertion $\inside{x}$ is not 
contained within an module $M$ since its due to the
universal quantification contained withing 
$\inside{x}$, the footprint consists of portions 
of the heap that are external to $M$. $\inside{x}$ is 
encapsulated by $M$ since if only objects that derive 
from $M$ have access to $x$, it follows that a method call
on $M$ is required to gain access to $x$.
Necessity Logic itself does not depend on the details
of the encapsulation scheme --- only that we can determine
whether an assertion is encapsulated within a particular
part of the program.  For reasons of simplicity, 
we have adopted an encapsulation model for \Loo
based on 
\citeauthor{confined}'s \textit{Confined Types} [\citeyear{confined}]
(and we rely on their proof).
%
% We see no reason why a different
% encapsulation mechanism could not be used instead.
%
% KJX move to related work?
%
Confined types partition the objects accessible to code within a
module, based on those objects' defining classes and modules:
\begin{itemize}
\item instances of non-\enclosed classes %defined in a module
constitute their defining module's encapsulation
boundary \cite{TAME2003},
and may be accessed anywhere.
\item instances of \enclosed classes %defined in the module
are encapsulated \inside their defining module.
\item instances of \enclosed classes defined in \emph{other} modules
are not accessible elsewhere
\item instances of non-\enclosed classes defined in \emph{other}
modules are visible, however methods may only be invoked on such 
objects when the confinement system guarantees the particular instance
is only accessible \inside \emph{this} module.
\end{itemize}
%
\noindent \Loo's Confined Types rely on three syntactic restrictions
to enforce this encapsulation model:
\begin{itemize}
\item \enclosed class declarations must be annotated.
\item \enclosed objects may not be returned by methods of non-\enclosed
classes.
\item Ghost fields 
may be annotated as \prg{intrnl}; if so, they must only refer to objects \inside
their defining module --- i.e.\ either defined directly in that module, or
instances of non-\enclosed classes defined in \emph{other} modules
where those particular instances are only ever accessed within the
defining module.
\end{itemize}
}
\jm[Yes, I think that's right. I'm not 100\% sure that assertion 
encapsulation is defined by the footprint, unless I misunderstand 
footprint. It is possible for an assertion to be encapsulated, 
but depend on external objects. For example $\neg\access{x}{y}$:
if $\inside{y}$ is true, then $\neg\access{x}{y}$ is encapsulated,
even if $x$ is external.]{}

%% Using Confined Types for \Loo means that modules needing to encapsulate
%% assertions must meet the following ownership restrictions: 

%% however we assume several properties enforced by the type system, including 
%% simple ownership properties:
%% \begin{itemize}
%% \item
%% Method calls may not be made to external, non-module methods.
%% \item
%% \jm[]{Classes may be optionally annotated as \enclosed: their instances (``\enclosed'' objects) are marked as \enclosed, and may not be returned by methods of non-\enclosed classes.}
%% \item
%% Ghost fields may be annotated as \prg{intrnl} and thus may only include and be passed references to objects belonging to module internal classes.
%% \end{itemize}
%% %
%% \kjx
%% These rules enforce a simple model 
%%   Instances of classes confined in a module are internal to that module;
%% however, non-confined 

%% %
%% These encapsulation properties are easily enforceable, and we
%% do not define the type system as ownership types have been 
%% well covered in the literature. 
%% We specifically use a simple ownership system to model 
%% encapsulation as the theory has been well established by others, 





\jm[]{
%To assist in the definition of our simple encapsulation system,
We define internally evaluated expressions ($\intrnl{\_}$) 
whose evaluation only inspects internal objects or primitvies (i.e. integers or booleans).}
\jm[]{\begin{definition}[Internally Evaluated Expressions]
For all modules $M$, assertions $A$, and expressions $e$, 
$\satisfies{M}{\givenA{A}{\intrnl{e}}}$ if and only if for all heaps $\chi$, stacks $\psi$, and frames $\phi$
such that $\satisfiesA{M}{(\chi, \phi : \psi)}{A}$, we have for all values $v$, such that $\eval{M}{(\chi, \phi : \psi)}{e}{v}$
then $\eval{M}{(\chi', \phi' : \psi)}{e}{v}$, where 
\begin{itemize}
\item $\chi'$ is the internal portion of $\chi$, i.e. \\
$\chi' = \{\alpha \mapsto o| \alpha \mapsto o \in \chi\ \wedge \ o.(\prg{cname}) \in M \}$ and
\item $\phi'.(\prg{local})$ is the internal portion of the $\phi.(\prg{local})$ i.e. \\
$\phi' = \{x \mapsto v| x \mapsto v \in \chi\ \wedge \ (v \in \IntSet\ \vee\ v = \true\ \vee\ v = false)\ \vee\ (\exists \alpha, \ v = \alpha \wedge \class{(\chi, \phi : \psi)}{\alpha} \in M\}$
\end{itemize}
\end{definition}}


The encapsulation proof system consists of two relations 
\begin{itemize}
\item
Purely internal expressions: $\proves{M}{\givenA{A}{\intrnl{e}}}$ and
\item
Assertion encapsulation: $\proves{M}{\givenA{A}{\encaps{A'}}}$
\end{itemize}

Fig. \ref{f:intrnl} gives proof rules for an expression comprising purely internal objects.
\jm[]{Primitives are $Enc_e$ (\textsc{Enc$_e$-Int}, \textsc{Enc$_e$-Null}, \textsc{Enc$_e$-True}, and \textsc{Enc$_e$-False}).
Addresses of internal objects are $Enc_e$ (\textsc{Enc$_e$-Obj}). Field accesses with internal types of $Enc_e$ expressions
are themselves $Enc_e$ (\textsc{Enc$_e$-Field}). Ghost field accesses annotated as $Enc_e$ on $Enc_e$ 
expressions are themselves $Enc_e$ (\textsc{Enc$_e$-Ghost}).}

\begin{figure}[h]
\footnotesize
\begin{mathpar}
\infer
		{}
		{\proves{M}{\givenA{A}{\intrnl{i}}}}
		\quad(\textsc{Enc$_e$-Int})
		\and
\infer
		{}
		{\proves{M}{\givenA{A}{\intrnl{\nul}}}}
		\quad(\textsc{Enc$_e$-Null})
		\and
\infer
		{}
		{\proves{M}{\givenA{A}{\intrnl{\true}}}}
		\quad(\textsc{Enc$_e$-True})
		\and
\infer
		{}
		{\proves{M}{\givenA{A}{\intrnl{\false}}}}
		\quad(\textsc{Enc$_e$-False})
		\and
\infer
		{
		\proves{M}{A\ \longrightarrow\ \alpha : C}\\
		C\ \in\ M
		}
		{
		\proves{M}{\givenA{A}{\intrnl{\alpha}}}
		}
		\quad(\textsc{Enc$_e$-Obj})
		\and
\infer
		{
		\proves{M}{\givenA{A}{\intrnl{e}}}\\
		\proves{M}{A\ \longrightarrow\ e : C}\\
		[\prg{field}\ \_\ f\ :\ D]\ \in\ M(C).(\prg{flds}) \\
		D\ \in\ M
		}
		{
		\proves{M}{\givenA{A}{\intrnl{e.f}}}
		}
		\quad(\textsc{Enc$_e$-Field})
		\and
\infer
		{
		\proves{M}{\givenA{A}{\intrnl{e_1}}}\\
		\proves{M}{\givenA{A}{\intrnl{e_2}}}\\
%		\proves{M}{\givenA{A}{\intrnl{e}}} \\
		\proves{M}{A\ \longrightarrow\ e_1 : C} \\
		\prg{ghost}\ \prg{intrnl}\ g(x : \_)\{e\} \in M(C).(\prg{gflds})
		}
		{
		\proves{M}{\givenA{A}{\intrnl{e_1.g(e_2)}}}
		}
		\quad(\textsc{Enc$_e$-Ghost})
\end{mathpar}
\caption{Internal Proof Rules}
\label{f:intrnl}
\end{figure}


\jm[]{Fig. \ref{f:asrt-encap} gives proof rules for whether an assertion is encapsulated, that is whether 
a change in satisfaction of an assertion requires interaction with the internal module.
An \prg{Intrl} expression is also an encapsulated assertion (\textsc{Enc-Exp}). A field
access on an encapsulated expression is an encapsulated expression. Binary and ternary operators
applied to encapsulated expressions are themselves encapsulated assertions (\textsc{Enc-=}, \textsc{Enc-+}, \textsc{Enc-<}, \textsc{Enc-If}).
An internal object may only lose access to another object via internal computation (\textsc{Enc-IntAccess}).
Only internal computation may grant external access to an $\wrapped{\_}$ object (\textsc{Enc-Inside}$_1$).
If an object is $\wrapped{\_}$, then nothing (not even internal objects) may gain access
to that object except by internal computation (\textsc{Enc-Inside}$_2$).
If an assertion $A_1$ implies assertion $A_2$, then $A_1$ implies the encapsulation of any assertion that
$A_2$ does. Further, if an assertion is encapsulated, then any assertion that is implied by it is also encapsulated.
These two rules combine into an encapsulation rule for consequence (\textsc{Enc-Conseq}).}

\begin{figure}[h]
\footnotesize
\begin{mathpar}
\infer
		{\proves{M}{\givenA{A}{\intrnl{e}}}}
		{\proves{M}{\givenA{A}{\encaps{e}}}}
		\quad(\textsc{Enc-Exp})
		\and
\infer
		{\proves{M}{\givenA{A}{\intrnl{e}}}}
		{\proves{M}{\givenA{A}{\encaps{e.f}}}}
		\quad(\textsc{Enc-Field})
		\and
\infer
		{
		\proves{M}{\givenA{A}{\encaps{e_1}}} \\
		\proves{M}{\givenA{A}{\encaps{e_2}}}
		}
		{
		\proves{M}{\givenA{A}{\encaps{e_1 = e_2}}}
		}
		\quad(\textsc{Enc-=})
		\and
\infer
		{
		\proves{M}{\givenA{A}{\encaps{e_1}}} \\
		\proves{M}{\givenA{A}{\encaps{e_2}}}
		}
		{
		\proves{M}{\givenA{A}{\encaps{e_1 + e_2}}}
		}
		\quad(\textsc{Enc-+})
		\and
\infer
		{
		\proves{M}{\givenA{A}{\encaps{e_1}}} \\
		\proves{M}{\givenA{A}{\encaps{e_2}}}
		}
		{
		\proves{M}{\givenA{A}{\encaps{e_1 < e_2}}}
		}
		\quad(\textsc{Enc-<})
		\and
\infer
		{
		\proves{M}{\givenA{A}{\encaps{e}}} \\
		\proves{M}{\givenA{A}{\encaps{e_1}}} \\
		\proves{M}{\givenA{A}{\encaps{e_2}}}
		}
		{
		\proves{M}{\givenA{A}{\encaps{\prg{if}\ e\ \prg{then}\ e_1\ \prg{else}\ e_2}}}
		}
		\quad(\textsc{Enc-If})
		\and
\infer
		{\proves{M}{A\ \longrightarrow\ \internal{x}}}
		{\proves{M}{\givenA{A}{\encaps{\access{x}{y}}}}}
		\quad(\textsc{Enc-IntAccess})
		\and
\infer
		{}
		{\proves{M}{\givenA{A}{\encaps{\wrapped{x}}}}}
		\quad(\textsc{Enc-Inside}_1)
		\and
\infer
		{\proves{M}{A\ \longrightarrow\ \wrapped{x}}}
		{\proves{M}{\givenA{A}{\encaps{\neg \access{x}{y}}}}}
		\quad(\textsc{Enc-Inside}_2)
		\and
\infer
		{
		\proves{M}{A_1\ \longrightarrow\ A_2} \\
		\proves{M}{A\ \longrightarrow\ A'} \\
		\proves{M}{\givenA{A_2}{\encaps{A}}}
		}
		{\proves{M}{\givenA{A_1}{\encaps{A'}}}}
		\quad(\textsc{Enc-Conseq})
\end{mathpar}
\caption{Assertion Encapsulation Proof Rules}
\label{f:asrt-encap}
\end{figure}
