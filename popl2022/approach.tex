
\section{Outline of our approach}
\label{s:outline}

 \subsection{Bank Account Example}
\label{s:bank}
  
Consider module \prg{Mod1} which consists   classes  \prg{Password} and  \prg{Account}. It
represents  bank accounts
with a balance and a password, where funds may be transferred
between accounts only when sending the account's password:
%
% (Note that we assume private fields are accessible ``class-wide''.)
%
% (methods may read and write fields of any instance of a class)
%
%and that passwords are unforgeable and not enumerable (again as
%in Java, albeit without reflection).
%
% 
\begin{lstlisting}[language=Chainmail, frame=lines]
module Mod1
  class Account
    field balance:int 
    field pwd: Password
    method transfer(dest:Account, pwd':Object) -> void
      if this.pwd==pwd'
        this.balance-=100
        dest.balance+=100
  class Password
\end{lstlisting}
%
\noindent 
We can capture the intended
semantics of the \prg{transfer} method by writing ``classical''
specifications in terms of pre- and post-conditions ---
\prg{Mod1}'s implementation of the \prg{transfer} method meets
this specification.



\begin{lstlisting}[mathescape=true, frame=lines, language=Chainmail]
ClassicBankSpec  $\triangleq$
   method transfer(dest:Account, pwd':Password) -> void {
       ( PRE:  this.balance=bal1 $\wedge$ this.pwd==pwd' $\wedge$ dest.balance=bal2 $\wedge$ dest=/=this 
         POST: this.balance == bal1-100 $\wedge$  dest.balance == bal2+100 )
       ( PRE: this.balance=bal1 $\wedge$ this.pwd=/=pwd' $\wedge$ dest.balance=bal2
         POST: this.balance == bal1 $\wedge$  dest.balance=bal2 )
       ( PRE: a : Account $\wedge$ a=/=this $\wedge$ a=/=dest  $\wedge$ a.balance=bal $\wedge$ a.pwd=pwd1
         POST:  a.balance=bal $\wedge$ a.pwd=pwd1)
       ( PRE: a : Account $\wedge$ a.pwd=pwd1  
         POST: a.pwd=pwd1)       
\end{lstlisting}
%\footnote{Perhaps omit some of the lines here, but we do need them all in the full discussion}
 
 

Now consider the following alternative implementations:
\prg{Mod2} allows any client to reset its password at any time, while
\prg{Mod3} first requires the existing password in order to change it.
The problem is that although the \prg{transfer} method is the same in
all three alternatives, and although each one satisfies \prg{(ClassicBankSpec)},
code sequences such as
%
\prg{account.set(42); account.transfer(yourAccount, 42)}
%
are enough to drain the account in \prg{Mod2} without supplying the password.

%% The method \prg{transfer} in all three versions of the class \prg{Account} satisfies the (ClassicSpec), 
%% however, while executing the first and third version of \prg{Account} won't exhibit unwanted behaviour, the second version doesn't preclude it.
%Namely version II allows any client to change the password of the
%account, and then to repeatedly withdraw money from it.

  
% On the other hand, we expect our software -- even if complex -- to provide some simple, high level
%guarantees, e.g. email addressed to me personally will not be read by a third party unless I explicitly 
%forwarded it to them.
%We except  our software to  behave correctly, even when used by a careless or malicious third party. 
%Such use of a software often consist of a sequence of actions performed on the module. 
%
%Software components respond to single actions, 
%or to sequences of such single actions. 
%When thinking about a software component we want think about the behaviour of each 
%action in isolation, but also about the \emph{emergent behaviour}, ie all
% the possible effects of the combinations of these actions. 
  
  

\begin{tabular}{lll}
\begin{minipage}[b]{0.35\textwidth}
\begin{lstlisting}[language=chainmail, frame=lines]
module Mod2
  class Account
    field balance:int 
    field pwd: Password 
    method transfer(..) 
      ... as earlier ...
    method set(pwd': Object)
      this.pwd=pwd'
      
  class Password
\end{lstlisting}
\end{minipage}
&\ \ \  \ \   &%
\begin{minipage}[b]{0.50\textwidth}
\begin{lstlisting}[language=chainmail, frame=lines]
module Mod3
  class Account
    field balance:int 
    field pwd: Password 
    method transfer(..) 
      ... as earlier ...
    method set(pwd',pwd'': Object)
      if (this.pwd==pwd') 
        this.pwd=pwd''
  class Password
\end{lstlisting}
\end{minipage} 
\end{tabular}




We need to rule out \prg{Mod2} while permitting \prg{Mod3} and
\prg{Mod1}. The catch is the vulnerability present in \prg{Mod2} is the result
of  \emph{emergent} behaviour from the interactions of the \prg{set}
and \prg{transfer} methods --- even though \prg{Mod3} also has a
\prg{set} method, it does not exhibit the unwanted interaction.
This is exactly where a necessary condition can help:
we want to avoid transferring money
(or more generally, reducing an account's balance)
\textit{without} the existing account password.  Phrasing the same condition
the other way around gives us a positive statement that still
rules out the theft: that money \textit{can only} be
transferred when the account's password is known.
In our necessity condition syntax:

%
%% The flaw in version {\sf{II}} arises from 
%% can be used to overwrite the
%% password, and then using the new password \prg{transfer}  can be called.
% If we want the \prg{Account} class to be robust, we must prohibit the password from being freely available.
%
%
%% In this paper, we show how \textit{necessity specifications} can solve
%% this progkem.  
%%  Therefore, we propose a %holistic 
%%  specification which says that
%%  the \prg{balance} of an \prg{Account} reduces only if an object which does not belong to the
%%  module has access to the password:
%
%
 %% Therefore, we propose a %holistic 
 %% specification which says that
 %% the \prg{balance} of an \prg{Account} reduces only if an object which does not belong to the
 %% module has access to the password:
%
\begin{lstlisting}[language = Chainmail, mathescape=true, frame=lines]
NecessityBankSpec  $\triangleq$  from a:Account $\wedge$ a.balance==bal
                      to a.balance < bal
                      onlyIf $\exists$ o.[$\external{\texttt{o}}$ $\wedge$ $\access{\texttt{o}}{\texttt{a.pwd}}$]
\end{lstlisting}
%
%
% 
%% In more detail, the specification from above says that if in the current
%% configuration \prg{a} is an \prg{Account},
%% and in some future configuration \prg{a} will have a balance less than the current one, then,in the \emph{current} configuration
%% there must exist some object \prg{o}, which is \emph{external} to our module (does not belong to module
 %% \prg{AccountMdl}), and which has access to \prg{a}'s password.
 %
%% T9hu having access to the password is a necessary condition for the balance to reduce.
The critical point of this necessity specification is that it is
expressed in terms of observable effects and the shape of the heap --- the account's balance is reduced 
and there is external access to the password ($\external{\prg{o}}\ \wedge\ \access{\prg{o}}{\prg{a.pwd}}$)
--- and not in terms of individual methods --- such as
\prg{set} or \prg{transfer}.
This gives necessity specifications the
vital advantage that they constrain
\textit{implementation} of a bank account with a balance and a
password, irrespective of the API it
offers, the services it exports, or the dependencies on other parts of
the system.
\jm[TODO: in the proof later we need to mention that there is a difference between the overall proof (no mention of the methods), and the intermediate ones (that do mention the methods)]{}


%\subsection{Internal and External Modules, Objects, and Calls}

\subsection{Internal and External objects and calls}
\label{s:concepts}

Our work concentrates on guarantees made for the \emph{open} setting; that is, a certain module
$M$ is programmed in such a robust manner
%\sophia[We had agreed no to say "robust", but I think it is OK here. Alternatively, say defensively robust?]{}
 that execution of $M$ together with \emph{any} \externalM 
module $M'$ will uphold these guarantees. In the tradition of
visible states semantics, we are  only interested in upholding the guarantees while 
$M'$, the  \externalM module,  is executing.   
 We therefore distinguish between \emph{\internalO} and
\emph{\externalO} objects: those that belong to classes defined in $M$, and the rest. Similarly, we 
distinguish between \emph{\internalC} calls, i.e. calls made % from \externalO objects
 to \internalO objects, and \emph{\externalC} calls, i.e. calls made % from \externalC objects
 to \externalC objects. % We are less
% interested in calls made from \internalO objects to \internalO objects, because we only need 
% establish the guarantees when the \externalM module is executing. And we 


Moreover, because we only require that the guarantees are upheld while 
  the  \externalM module  is executing, we develop an \emph{external states} semantics, \sophia{I like the term} where
 any internal calls are executed in one, large, step.
 Note that we do not -- yet -- support calls from
\internalO objects to \externalO objects.\sophia[we need to think where our system forbids that, ... ?]{} 
 With the external steps semantics,  the executing object (the \prg{this}) is always   external. 
 
%\subsection{Reasoning about Necessity}

%\footnote{Shall we skip this paragraph?} Our Logic of Necessity is based on the following ingredients: 
%\textbf{a:} A  % concept of a encapsulation, 
%way to  establish that a certain effect is \emph{encapsulated}, in the 
%sense that it can only take place if an \internalC call was made; 
%\textbf{b:} A way to obtain necessary conditions given an effect and \emph{one}
% specific \internalC call;
%\textbf{c:} A way to obtain necessary conditions given a certain encapsulated effect and \emph{any},
%\emph{single} \internalC call;
%\textbf{d:} A way to combine one-call   necessary conditions  so as to obtain necessary conditions 
%for effects over 
%\emph{any number} of \internalC or \externalC calls.

\subsection{Reasoning about Necessity}
\label{s:approach}

We  now outline the proof that \prg{Mod3} adheres to \prg{NecessityBankSpec}),
and  will then use that outline to 
 introduce the main ingredients of our Necessity Logic. The proof consists,
broadly, of the following parts:


\begin{description}
\item[P1:] We establish that the balance 
 may change only   through \prg{Mod3}-\internalC calls.  

\item[P2:] 
For each method of \prg{Mod3} (i.e. methods of \prg{Account} and \prg{Password}),  we establish that   
if the method were called and  caused the  balance to reduce, then, before the call,
the caller had access  to the password.

\item[P3:]  
From \textbf{P1} and \textbf{P2}, we obtain that if the balance were to 
reduce in \emph{any}  \emph{single} call, then some
external object would have to have had access to the password before the call.

\item[P4:] We establish % in the class \prg{Account} in Version\_III 
that   an external object has access to 
the password after \emph{one} internal call, only if it already had access before that call.
From that, we establish  that  an external object will have access to 
the password after \emph{any number} of external or internal 
steps, only if it already had access before these steps.

\item[P5:] Combining  the results from \textbf{P3} and \textbf{P4}, we obtain
that  \prg{Mod3} satisfies $(NecessityBankSpec)$

\end{description} 
 
We  now outline  the new formal concepts needed to accomplish the five parts of the proof from above:

\begin{description}
\item[Assertion encapsulation] % We introduce the concept of \emph{assertion-encapsulation}: 
%  \item[from effect to module]
An assertion $A$  is
\emph{encapsulated} by a module $M$, if  $A$ can be invalidated only through
    an $M$-\internalC call. 
  In short, an $M$-\internalC call is a \emph{necessary} condition for
a given effect to take place.
%
%\footnote{Notice that for the sake of simplicity, our example
%consists of a module with one class only, but
%    in general our approach accommodates modules with any number of classes.
%     Notice also, that while in this example  the balance is 
%represented by a field, our approach also works when the balance is a ghost method, defined 
%recursively over several objects of different classes.}
%
In \textbf{P1}, the assertion \prg{a:Account\! $\wedge$\! a.balance=bal} is encapsulated by \prg{Mod3}.

We assume that there exists some algorithm to judge assertion encapsulation.
The construction of such an algorithm is not the focus of our work;
  we  outline a rudimentary such algorithm, but more powerful
approaches are possible. 
 

%\item[from effect and call to necessary condition]
\item[Per-method necessity specification]
  We want to infer
a necessary condition given an effect and
a single, specified, method call. 

In \textbf{P2},   a necessary condition for the  reduction of \prg{a.balance}  after the call \prg{a.transfer(a',pwd)} is that the caller had access to \prg{a.password} before the call.

We addressed the challenge of the inference of necessary conditions 
 by leveraging the sufficient conditions from classical specifications:
 \sophia[
  Here we should 
  pat ourselves on the back.  
  James, need your  magic here.]{} 
 As we will see in Section \ref{s:classical-proof}, 
if the negation of a method's
 classical postcondition implies  the effect we are interested in, then the negation of the 
 classical precondition  is the necessary precondition for the effect and the method call. Thus, 
  a method's sufficient conditions are used to infer a method's and effect's necessary conditions.



  
%\item[from effect and single step to necessary condition]
\item[Single step necessity specification]
We want to infer
a necessary condition given an effect and
a single, unspecified step. This step could be a internal, or an external method call,
or any other external step, such as field assignment, or conditional, etc.

In \textbf{P3},   a necessary condition for the  reduction of \prg{a.balance}  after \emph{any}
step, is that the caller  had access to \prg{a.password} before the call.
And similarly in \textbf{P4},   a necessary condition for an external object's
access to \prg{a.password}  after \emph{any}
step, is that that object had access to \prg{a.password} before the call.

For effects that are encapsulated in a module $M$, we can infer such one-step
necessary conditions by combining the necessary conditions for that effect and 
all   methods in $M$.



%\item[from effect to necessary conditions]
\item[Necessity specifications of emergent behaviour]
  
We now need to consider the \emph{emergent} behaviour of the internal module
combined with any internal module. This step is crucial; namely, remember that while \prg{Mod2} adheres to
the guarantee from the \textbf{P3}, it does not adhere    
\sophiaPonder[Used to say $(NecessitySpec)$ -- when I still find such errors, I worry]{to  $(NecessityBankSpec)$.}
Our Logic of Necessity allows us to combine  several such one-step necessary conditions, and obtain a several-step necessity specification.
 
\end{description} 
