

\section{Outline of our approach}
\label{s:outline}
%We now give an outline of our approach %using a bank account example,   where funds may be transferred between accounts only if the account's password has been supplied.
%\sophiaPonder[more succinct]{expanding on the bank example from earlier.}


 \subsection{Bank Account -- three modules}
\label{s:bank}
  
Module \prg{Mod1} consists of the \prg{Account} class with a
\prg{transfer} method, and an empty 
\prg{Password} class where each instance models a unique password.
%
% (Note that we assume private fields are accessible ``class-wide''.)
%
% (methods may read and write fields of any instance of a class)
%
%and that passwords are unforgeable and not enumerable (again as
%in Java, albeit without reflection).
%
% 
\begin{lstlisting}[language=Chainmail, frame=lines]
module Mod1
  class Account
    field balance:int 
    field pwd: Password
    method transfer(dest:Account, pwd':Object) -> void
      if this.pwd==pwd'
        this.balance-=100
        dest.balance+=100
  class Password
\end{lstlisting}
%
\noindent 
We can capture the intended
semantics of % the \prg{transfer} method 
\prg{transfer} % by writing
through  a pre and postcondition 
JML-style \cite{Leavens-etal07}  ``classical''
specification
%\sophiaPonder[]{
-- note, all free variables are are universally quantified.
%}
\prg{Mod1}'s implementation of the \prg{transfer} method meets
this specification.



\begin{lstlisting}[mathescape=true, frame=lines, language=Chainmail]
ClassicBankSpec  $\triangleq$
   method transfer(dest:Account, pwd':Password) -> void {
       ( PRE:  this.balance=bal1 $\wedge$ this.pwd==pwd' $\wedge$ dest.balance=bal2 $\wedge$ dest=/=this 
         POST: this.balance == bal1-100 $\wedge$  dest.balance == bal2+100 )
       ( PRE: this.balance=bal1 $\wedge$ this.pwd=/=pwd' $\wedge$ dest.balance=bal2
         POST: this.balance == bal1 $\wedge$  dest.balance=bal2 )
       ( PRE: a : Account $\wedge$ a=/=this $\wedge$ a=/=dest  $\wedge$ a.balance=bal $\wedge$ a.pwd=pwd1
         POST:  a.balance=bal $\wedge$ a.pwd=pwd1)
       ( PRE: a : Account $\wedge$ a.pwd=pwd1  
         POST: a.pwd=pwd1)       
\end{lstlisting}
%\footnote{Perhaps omit some of the lines here, but we do need them all in the full discussion}
 
 

Now consider the following alternative implementations:
\prg{Mod2} allows any client to reset an account's password at any time, while
\prg{Mod3} requires the existing password in order to change it.

%% The method \prg{transfer} in all three versions of the class \prg{Account} satisfies the (ClassicSpec), 
%% however, while executing the first and third version of \prg{Account} won't exhibit unwanted behaviour, the second version doesn't preclude it.
%Namely version II allows any client to change the password of the
%account, and then to repeatedly withdraw money from it.

  
% On the other hand, we expect our software -- even if complex -- to provide some simple, high level
%guarantees, e.g. email addressed to me personally will not be read by a third party unless I explicitly 
%forwarded it to them.
%We except  our software to  behave correctly, even when used by a careless or malicious third party. 
%Such use of a software often consist of a sequence of actions performed on the module. 
%
%Software components respond to single actions, 
%or to sequences of such single actions. 
%When thinking about a software component we want think about the behaviour of each 
%action in isolation, but also about the \emph{emergent behaviour}, ie all
% the possible effects of the combinations of these actions. 
  
  

\begin{tabular}{lll}
\begin{minipage}[b]{0.42\textwidth}
\begin{lstlisting}[language=chainmail, frame=lines]
module Mod2
  class Account
    field balance:int 
    field pwd: Password 
    method transfer(..) 
      ... as earlier ...
    method set(pwd': Object)
      this.pwd=pwd'
      
  class Password
\end{lstlisting}
\end{minipage}
&\ \ \  \ \   &%
\begin{minipage}[b]{0.45\textwidth}
\begin{lstlisting}[language=chainmail, frame=lines]
module Mod3
  class Account
    field balance:int 
    field pwd: Password 
    method transfer(..) 
      ... as earlier ...
    method set(pwd',pwd'': Object)
      if (this.pwd==pwd') 
        this.pwd=pwd''
  class Password
\end{lstlisting}
\end{minipage} 
\end{tabular}

Although the \prg{transfer} method is the same in
all three alternatives, and each one satisfies \prg{(ClassicBankSpec)},
code such as
%
\prg{account.set(42); account.transfer(yours, 42)}
%
is enough to drain the account in \prg{Mod2} without knowing the password.


 \subsection{Bank Account -- the right specification}
\label{s:bankSpecEx}

We need to rule out \prg{Mod2} while permitting \prg{Mod1} and
\prg{Mod3}. The catch is that the vulnerability present in \prg{Mod2} is the result
of  \emph{emergent} behaviour from the interactions of the \prg{set}
and \prg{transfer} methods --- even though \prg{Mod3} also has a
\prg{set} method, it does not exhibit the unwanted interaction.
This is exactly where a necessary condition can help:
we want to avoid transferring money
(or more generally, reducing an account's balance)
\textit{without} the existing account password.  Phrasing the same condition
the other way around % gives us a positive statement that still
rules out the theft: that money \textit{can only} be
transferred when the account's password is known.
In \Nec  syntax:

%
%% The flaw in version {\sf{II}} arises from 
%% can be used to overwrite the
%% password, and then using the new password \prg{transfer}  can be called.
% If we want the \prg{Account} class to be robust, we must prohibit the password from being freely available.
%
%
%% In this paper, we show how \textit{necessity specifications} can solve
%% this progkem.  
%%  Therefore, we propose a %holistic 
%%  specification which says that
%%  the \prg{balance} of an \prg{Account} reduces only if an object which does not belong to the
%%  module has access to the password:
%
%
 %% Therefore, we propose a %holistic 
 %% specification which says that
 %% the \prg{balance} of an \prg{Account} reduces only if an object which does not belong to the
 %% module has access to the password:
%
\begin{lstlisting}[language = Chainmail, mathescape=true, frame=lines]
NecessityBankSpec  $\triangleq$   from a:Account $\wedge$ a.balance==bal
                        to a.balance < bal
                        onlyIf $\exists$ o.[$\external{\texttt{o}}$ $\wedge$ $\access{\texttt{o}}{\texttt{a.pwd}}$]
\end{lstlisting}
%
%
% 
%% In more detail, the specification from above says that if in the current
%% configuration \prg{a} is an \prg{Account},
%% and in some future configuration \prg{a} will have a balance less than the current one, then,in the \emph{current} configuration
%% there must exist some object \prg{o}, which is \emph{external} to our module (does not belong to module
 %% \prg{AccountMdl}), and which has access to \prg{a}'s password.
 %
%% T9hu having access to the password is a necessary condition for the balance to reduce.
The critical point of this \Nec specification is that it is
expressed in terms of observable effects (the account's balance is
reduced: \prg{a.balance < bal}) and the shape of the heap 
(external access to the password:
$\external{\prg{o}}\ \wedge\ \access{\prg{o}}{\prg{a.pwd}}$) 
rather than in terms of individual methods such as
\prg{set} and \prg{transfer}.
This gives our specifications the
vital advantage that they can be used to constrain
\textit{implementation} of a bank account with a balance and a
password, irrespective of the API it
offers, the services it exports, or the dependencies on other parts of
the system.

%\sophiaPonder[skip that part?]
{This example also demonstrates that 
%the specification need  refer  to  individual methods in a module.
%Moreover, 
adherence to   \Nec specifications is not monotonic:
adding a method to a module does not necessarily preserve adherence to
a specification, 
and while separate methods may adhere to a  specification, their combination does
not necessarily do so. This is why we say \Nec specifications capture a module's \emph{emergent behaviour}. 
}

%\jm[TODO: in the proof later we need to mention that there is a difference between the overall proof (no mention of the methods), and the intermediate ones (that do mention the methods)]{}


%\subsection{Internal and External Modules, Objects, and Calls}

\subsection{Internal and External objects and calls}
\label{s:concepts}

Our work concentrates on guarantees made in an \emph{open} setting; that is, a given module
$M$ must be programmed so that %in such a robust manner
%\sophia[We had agreed no to say "robust", but I think it is OK here. Alternatively, say defensively robust?]{}
execution of $M$ together with \emph{any} \externalM 
module $M'$ will uphold these guarantees. In the tradition of
visible states semantics, we are  only interested in upholding the guarantees while 
$M'$, the  \emph{\externalM} module, is executing. A module can
temporarily break its own invariants,
%%just as a class or monitor can
%%break their invariants,
so long as the broken invariants are never visible externally.
   
We therefore distinguish between \emph{\internalO}
objects --- instances of classes defined in $M$ ---
and \emph{\externalO} objects defined in any other module, and
between \emph{\internalC} calls  \susan[]{(from either an internal or an external object)}  made % from \externalO objects
 to \internalO objects and \emph{\externalC} calls made % from \externalC objects
 to \externalC objects. % We are less
% interested in calls made from \internalO objects to \internalO objects, because we only need 
% establish the guarantees when the \externalM module is executing. And we 
%
%
Because we only require guarantees while 
the  \externalM module  is executing,
we develop an \emph{external states} semantics, where
 any internal calls are executed in one, large, step.
With external steps semantics,  the executing object (\prg{this}) is always   external. 
Note that we do not support calls from
\internalO objects to \externalO objects.
 

\newcommand{\vertsp}{\vspace{.03in}} 
 
\subsection{Reasoning about Necessity}
\label{s:approach}

\scd{We now give a sketch of the   novel concepts we developed in  \Nec logic.
For illustration, we  outline  a proof} that  \prg{Mod3} adheres to \prg{NecessityBankSpec}.
% we developoed in this work

\begin{description}
\item[Part 1: Assertion Encapsulation.]  
% \scd{We  introduce the concept} of  \emph{assertion  encapulation}: 
% \scd{We define that an}
\scd{An} assertion $A$  is
\emph{encapsulated} by  module $M$, if  $A$ can be invalidated only through
 % $M$-\internalC calls. 
 \scd{calls to methods defined in $M$}.
  In other words, a  \sophiaPonder[said $M$-\internalC call]{call to  $M$}  is a \emph{necessary} condition for
invalidation of $A$.
\sophiaPonder[said: Our formalisation -- but I think what I say is stronger]{Our \Nec logic}  is parametric with respect to the 
particular encapsulation
mechanism: here we rely on rudimentary annotations inspired by confinement types
\cite{confined}.
% In \textbf{P1}, the assertion \prg{a:Account\! $\wedge$\! a.balance=bal} is encapsulated by \prg{Mod3}.
%%%
%%%Determining encapsulation is a challenge, but not central to this work.
%%%We therefore outline a rudimentary types-based algorithm, and relegate more
%%%approaches to further work.
%%%

\vertsp

For our proof, we establish that  \textbf{(a)} the balance is encapsulated 
by \prg{Mod3}, and  \textbf{(b)},   external accessibility to an account's password may 
only be obtained through  \sophiaPonder[said: \prg{Mod3}-internal] calls to \prg{Mod3} -- that is,  the property that
no external object has access to the password is encapsulated by \prg{Mod3}.

\vertsp
 
 
\item[Part 2: Per-method conditions]   % \emph{per-method-condition}, \ie., a 
are  necessary conditions for   given  effect and
  given single, specified, method call. 
To infer these, we  leverage   sufficient conditions 
from classical specifications: % Namely, if 
If the negation of a method's
 classical postcondition implies  \scd{a given effect}, %the effect we are interested in,
  then the negation of the 
 classical precondition  is a necessary precondition for the effect and the method call. 
More in \S\ref{s:classical-proof}. 

\vertsp

For our proof, we establish  for each method of \prg{Mod3} (\ie each method in \prg{Account} and \prg{Password}) that if the method is called, then 
 \textbf{(c)},  if it  causes the  balance to reduce, then the method called was
 \prg{transfer} and the correct password was provided, and 
  \textbf{(d)} it will not provide external accessibility to the password.
  
\vertsp

 

 \item[Part 3: Single-step conditions] are
 necessary conditions for a given  effect and
a single, \emph{unspecified} step. This step could be an internal call, or any kind of external step.
  % 
    For effects encapsulated by $M$, we can infer such single-step
 conditions by combining the per-method conditions for that effect from 
all   methods in $M$. 
% We also have  expected sub-structural rules,\eg  like the rule of consequence.
% Drop above, as we dor 
More in \S\ref{s:module-proof}.
 
\vertsp

For our proof, from \textbf{(a)}  and  \textbf{(c)} we obtain that \textbf{(e)}  if the balance were to 
reduce in \emph{any}  \emph{single} step -- whether an internal call, or any external step --
 the method called was
 \prg{transfer} and the correct password was provided. Similarly,   from
 \textbf{(b)}  and  \textbf{(d)} we obtain that \textbf{(f)}
 external accessibility to the password will not be provided by any single step.
 
\vertsp

 
\item[Part 4: Emergent behaviour] is the behaviour than can be observed by
any number of steps. 
We infer the \emph{emergent} behaviour out of the conditions of many possible 
single steps.  
This part is crucial;   remember that while \prg{Mod2} satisfies  
(S3) for one single step, it does not satisfy it for any number of steps. More in \S\ref{s:emergent-proof}.

\vertsp

For our proof, from \textbf{(e)} we obtain that  \textbf{(g)}   if the balance reduces in any 
number of steps, then at some step  \prg{transfer} was called with   the correct password.
From  \textbf{(g)}  we obtain that \textbf{(h)} if the balance reduces in any 
number of steps, then at some intermediate step some external object had access to the password.
From \textbf{(f)} we obtain that  \textbf{(i)}  external accessibility to the password will not be provided by any sequence of steps.
Using  \textbf{(h)} and  \textbf{(i)}  we establish that
 \prg{Mod3} indeed adheres to \prg{NecessityBankSpec}.
 
\end{description} 
  
\noindent


  % Thus, 
 %  a method's sufficient conditions are used to infer a method's and effect's necessary conditions.

%\begin{description}
%\item[Part 1] We establish that  \textbf{(a)} the balance may only change through  
%\prg{Mod3}-internal calls.
%Also,  \textbf{(b)},   external accessibility to an account's password may 
%only be obtained through  \prg{Mod3}-internal calls -- that is,  the property that
%no external object has access to the password can only be invalidated through an internal call.
% 
%\item[Part 2] We establish  for each method of \prg{Mod3} (\ie each method in \prg{Account} and \prg{Password}) that if the method is called, then 
% \textbf{(c)},  if it  causes the  balance to reduce, then the method called was
% \prg{transfer} and the correct password was provided, and 
%  \textbf{(d)} it will not provide external accessibility to the password.
%
% \item[Part 3]  From \textbf{(a)}  and  \textbf{(c)} we obtain that \textbf{(e)}  if the balance were to 
%reduce in \emph{any}  \emph{single} step -- whether an internal call, or any external step --
% the method called was
% \prg{transfer} and the correct password was provided. Similarly,   from
% \textbf{(b)}  and  \textbf{(d)} we obtain that \textbf{(f)}
% external accessibility to the password will not be provided by any single step.
%    
% 
%\item[Part 4] From \textbf{(e)} we obtain that  \textbf{(g)}   if the balance reduces in any 
%number of steps, then at some step  \prg{transfer} was called with   the correct password.
%From  \textbf{(g)}  we obtain that \textbf{(h)} if the balance reduces in any 
%number of steps, then at some intermediate step some external object had access to the password.
%From \textbf{(f)} we obtain that  \textbf{(i)}  external accessibility to the password will not be provided by any sequence of steps.
%Using  \textbf{(h)} and  \textbf{(i)}  we establish that
% \prg{Mod3} indeed adheres to \prg{NecessityBankSpec}.
%
%\end{description} 
% 
%\noindent
%
%
%\begin{description}
%\item[Part 1] 
%We  say assertion $A$  is
%\emph{encapsulated} by  module $M$, if  $A$ can be invalidated only through
%  $M$-\internalC calls. In short, an $M$-\internalC call is a \emph{necessary} condition for
%invalidation of $A$.
%% In \textbf{P1}, the assertion \prg{a:Account\! $\wedge$\! a.balance=bal} is encapsulated by \prg{Mod3}.
%%%%
%%%%Determining encapsulation is a challenge, but not central to this work.
%%%%We therefore outline a rudimentary types-based algorithm, and relegate more
%%%%approaches to further work.
%%%%
%Our formalisation is parametric with respect to the encapsulation
%mechanism: here we rely on rudimentary annotations inspired by confinement types
%\cite{confined}.
%  
%\item[Part 2]
% Here we infer a \emph{per-method-condition}, \ie., a 
% necessary condition given an effect and
%a single, specified, method call. 
%% 
%%In \textbf{P2},   a necessary condition for the  reduction of \prg{a.balance}  after the call \prg{a.transfer(a',pwd)} is that the caller had access to \prg{a.password} before the call.
%We address this  challenge % of the inference of necessary conditions 
% by leveraging the sufficient conditions from classical specifications:
%If the negation of a method's
% classical postcondition implies  the effect we are interested in, then the negation of the 
% classical precondition  is the necessary precondition for the effect and the method call.
%More in \S\ref{s:classical-proof}.  
% % Thus, 
% %  a method's sufficient conditions are used to infer a method's and effect's necessary conditions.



  
%\item[from effect and single step to necessary condition]

%In \textbf{P3},   a necessary condition for the  reduction of \prg{a.balance}  after \emph{any}
%step, is that the caller  had access to \prg{a.password} before the call.
%And similarly in \textbf{P4},   a necessary condition for an external object's
%access to \prg{a.password}  after \emph{any}
%step, is that that object had access to \prg{a.password} before the call.





%\item[from effect to necessary conditions]
 

\noindent
Note that our proofs of necessity do not inspect method
bodies: we rely on simple annotations to infer encapsulation, and on
classical pre and postconditions to infer per-method conditions. 
