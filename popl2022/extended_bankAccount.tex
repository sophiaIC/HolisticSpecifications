\clearpage
\section{Proof of \ModD's Adherence to \SrobustB}
\label{app:examples}

\sd{We now describe  the poof that \ModD's adheres to \SrobustB;
the accompanying Coq formalism includes a mechanized version.}


%% \sophiaPonder[should these issues not have already been said in section 4]{As we stated} in Section \ref{sub:SpecO}, 
%% we assume the existence of a proof system for judgments
%% of the form $\proves{M}{A}$, denoting that in 
%% any arising program state, with internal module $M$, $A$ is satisfied. 
%% We \sd{list}   these rules in Appendix \ref{app:assert_logic}. 
%% %\jm[]{Further, recall that as per Def. \ref{def:necessity-semantics},
%% %$\satisfies{M}{A}$ is defined for arising program states, and thus by Theorem \ref{thm:soundness},
%% %if $\proves{M}{A}$, it follows that for all arising program states in the context fo internal 
%% %module $M$, $A$ is satisfied.}

 
Even though both the implementation and the specification being proven differ from those in \S
\ref{s:outline}, the structure of the proofs do retain broad similarities. In particular the proof in this section 
follows the outline of our reasoning given in Sec. \ref{s:approach}.
Namely, we prove:\\
\begin{enumerate}
\item
encapsulation of the account's balance and   password;
\item \emph{per-method} \Nec specifications on all \ModD methods,  
\item \emph{per-step} \Nec specifications for changing the balance and password, 
and finally 
\item the \emph{emergent} \Nec specification \SrobustB. 
\end{enumerate}

\noindent
\sd{We now discuss each of these four parts of the proof.}

 \subsection{Part 1: Assertion Encapsulation}
\label{s:BA-encap}
%% \sophiaPonder[Do we need all that: We base the soundness of our encapsulation of the type system of \Loo, and use the proof rules given in Figures \ref{f:intrnl} and \ref{f:asrt-encap}.
%%  Informally, $\intrnl{e}$ indicates that any objects inspected during the evaluation of expression $e$ are internal. $\encaps{A}$ (see Section \ref{s:inference}) indicates 
%% that internal computation is necessary for a change in satisfaction of $A$. Rudimentary algorithms for proving $\intrnl{}$ and $\encaps{}$ are given in 
%% Appendix \ref{s:encap-proof}, and used here.]{}
% We provide the proof for the 
Using \sd{the rules for proving $\intrnl{}$ and $\encaps{}$ from
Appendix \ref{s:encap-proof}  we prove encapsulation of \prg{b.balance(a)}} as  below\\
%\begin{figure}[h]
\begin{proofexample}
\proofsteps{\prg{BalanceEncaps}}
	{\begin{proofexample}
		\proofsteps{\prg{aEnc}}
			{\proofstepwithrule
			{$\proves{\ModD}{\givenA{\prg{b, b$^\prime$:Bank $\wedge$ a:Account $\wedge$ b.balance(a)=bal}}{\intrnl{\prg{a}}}}$}
				{by \textsc{Enc$_e$-Obj}}
		}
		\endproofsteps
	\end{proofexample}
		}
	{\begin{proofexample}
		\proofsteps{\prg{bEnc}}
			{\proofstepwithrule
			{$\proves{\ModD}{\givenA{\prg{b, b$^\prime$:Bank $\wedge$ a:Account $\wedge$ b.balance(a)=bal}}{\intrnl{\prg{b}}}}$}
				{by \textsc{Enc$_e$-Obj}}
		}
		\endproofsteps
	\end{proofexample}
		}
	{\begin{proofexample}
		\proofsteps{\prg{getBalEnc}}
			{\proofstepwithrule
			{$\proves{\ModD}{\givenA{\prg{b, b$^\prime$:Bank $\wedge$ a:Account $\wedge$ b.balance(a)=bal}}{\intrnl{\prg{b.balance(a)}}}}$}
				{by \prg{aEnc}, \prg{bEnc}, and \textsc{Enc$_e$-Ghost}}
		}
		\endproofsteps
	\end{proofexample}
		}
	{\begin{proofexample}
		\proofsteps{\prg{balEnc}}
			{\proofstepwithrule
			{$\proves{\ModD}{\givenA{\prg{b, b$^\prime$:Bank $\wedge$ a:Account $\wedge$ b.balance(a)=bal}}{\intrnl{\prg{bal}}}}$}
				{by \textsc{Enc$_e$-Int}}
		}
		\endproofsteps
	\end{proofexample}
		}
		{\proofstepwithrule
			{
			$\proves{\ModD}{\givenA{\prg{b, b$^\prime$:Bank $\wedge$ a:Account $\wedge$ b.balance(a)=bal}}{\encaps{\prg{b.balance(a)=bal}}}}$
			}{by \prg{getBalEnc}, \prg{balEnc}, \textsc{Enc-Exp}}}
\endproofsteps
\end{proofexample}\\
We omit the proof of $\encaps{\prg{a.password=pwd}}$, as its construction is very similar to that of $\encaps{\prg{b.balance(a)=bal}}$.
%\caption{Proof of encapsulation of \prg{b.getBal(a)=bal}}
%\end{figure}

\subsection{Part 2: Per-Method \Nec Specifications}
\label{s:BA-classical}
We now provide proofs for per-method specifications, working from
method pre- and postconditions.
\funcSpecs.
\sophiaPonder[It said "These proof steps are quite verbose" ..." -- please do not say that, put it in a positive way]{}
\sd{Here we  focus on proofs
of \prg{authenticate} from the \prg{Account} class.}

\jm[]{There are two \emph{per-method} \Nec specifications that we need
to prove of \prg{authenticate}: 
\begin{description}
\item[\textbf{\prg{AuthBalChange}}:] any change to the balance of an account may only occur if call to \prg{transfer} on the \prg{Bank} with the correct password is made. 
This may seem counter-intuitive as it is not possible to make two method calls (\prg{authenticate} and \prg{transfer}) at the same time, however we are able to prove this by first proving the 
absurdity that \prg{authenticate} is able to modify any balance.
\item[\textbf{\prg{AuthPwdLeak}}:] any call to \prg{authenticate} may only invalidate \wrapped{\prg{a.password}} (for any account \prg{a}) if \prg{false} is first satisfied -- clearly an absurdity.
\end{description}}

\paragraph{\emph{\textbf{\prg{AuthBalChange}}}}
First we use the \funcSpec of the \prg{authenticate} method in \prg{Account} to prove that a call to \prg{authenticate} can only result in a decrease in balance in a single step if there were in fact a call to \prg{transfer} to the \prg{Bank}. 
This may seem 
odd at first, and impossible to prove, however we leverage the fact that we are first able to prove that \prg{false}
is a necessary condition to decreasing the balance, or in other words, it is not possible to decrease the balance by a
call to \prg{authenticate}. 
We then use the proof rule \textsc{Absurd} to prove our desired necessary condition.
This proof is presented as \prg{AuthBalChange} below.
\\
\noindent
{
	\begin{proofexample}
		\proofsteps{AuthBalChange}
			{\proofstepwithrule
				{\hoareEx
						{a, a$^\prime$:Account $\wedge$ b:Bank $\wedge$ b.balance(a$^\prime$)=bal}
						{a.authenticate(pwd)}
						{b.balance(a$^\prime$) == bal}
						}
					{by \funcSpecs}
			}
			{\proofstepwithrule
				{\hoareEx
						{a, a$^\prime$:Account $\wedge$ b:Bank $\wedge$ b.balance(a$^\prime$)=bal $\wedge$ $\neg$ false}
						{a.authenticate(pwd)}
						{$\neg$ b.balance(a$^\prime$) < bal}
						}
					{by Hoare logic}
			}
			{\proofstepwithrule
				{\onlyIfSingleExAlt
						{a, a$^\prime$:Account $\wedge$ b:Bank $\wedge$ b.balance(a$^\prime$)=bal $\wedge$ $\calls{\_}{\prg{a}}{\prg{authenticate}}{\prg{pwd}}$}
						{b.balance(a$^\prime$) < bal}
						{false}
						}
					{by \textsc{If1-Classical}}
			}
			{\proofstepwithrule
				{\onlyIfSingleExAlt
						{a:Account $\wedge$ a$^\prime$:Account $\wedge$ b:Bank $\wedge$ b.balance(a$^\prime$)=bal $\wedge$ $\calls{\_}{\prg{a}}{\prg{authenticate}}{\prg{pwd}}$}
						{b.balance(a$^\prime$) < bal}
						{$\calls{\_}{\prg{b}}{\prg{transfer}}{\prg{a$^\prime$.password, amt, a$^\prime$, to}}$}
						}
					{by \textsc{Absurd} and \textsc{If1-}$\longrightarrow$}
			}
		\endproofsteps
	\end{proofexample}
}

\paragraph{\emph{\textbf{\prg{AuthPwdLeak}}}} The proof of \prg{AuthPwdLeak} is given below, and is proven by application of Hoare logic rules and \textsc{If1-Inside}.

\sophiaPonder[Do we want to show the other proofs? Or at least list what else is proven?]{}

%We provide the statements of the specifications for the remaining methods in the module below,
%%(\textbf{changePassBalChange}, \textbf{Ledger::TransferBalChange}, and \textbf{Bank::TransferBalChange}), 
%but we elide the proof steps as they do not differ much from that of \textbf{AuthBalChange}.\\
%{
%	\begin{proofexample}
%		\proofsteps{changePassBalChange}
%			{\proofstepwithrule
%				{\onlyIfSingleEx
%						{a, a$^\prime$:Account $\wedge$ b:Bank $\wedge$ b.getBal(a$^\prime$)=bal $\wedge$ $\calls{\_}{\prg{a}}{\prg{changePass}}{\prg{pwd}}$}
%						{b.getBal(a$^\prime$) < bal}
%						{$\calls{\_}{\prg{b}}{\prg{transfer}}{\prg{a$^\prime$.password, amt, a$^\prime$, to}}$}
%						}
%					{by similar reasoning to \textbf{AuthBalChange}}
%			}
%		\endproofsteps
%	\end{proofexample}
%}
%\begin{minipage}{\textwidth}
%{
%	\begin{proofexample}
%		\proofsteps{Ledger::TransferBalChange}
%			{\proofstepwithrule
%				{\onlyIfSingleEx
%						{l:Ledger $\wedge$ a:Account $\wedge$ b:Bank $\wedge$ b.getBal(a)=bal $\wedge$ $\calls{\_}{\prg{l}}{\prg{transfer}}{\prg{amt, from, to}}$}
%						{b.getBal(a) < bal}
%						{$\calls{\_}{\prg{l}}{\prg{transfer}}{\prg{amt, from, to}}$}
%						}
%					{by \textsc{If-Start} and \textsc{If1-If}}
%			}
%			{\proofstepwithrule
%				{\onlyIfSingleEx
%						{l:Ledger $\wedge$ a:Account $\wedge$ b:Bank $\wedge$ b.getBal(a)=bal $\wedge$ $\calls{\_}{\prg{l}}{\prg{transfer}}{\prg{amt, from, to}}$}
%						{b.getBal(a) < bal}
%						{$\neg\wrapped{\prg{l}}$}
%						}
%					{by \textsc{If1-}$\longrightarrow$, \textsc{Caller-Ext}, and \textsc{Caller-Recv}}
%			}
%			{\proofstepwithrule
%				{\onlyIfSingleEx
%						{l:Ledger $\wedge$ a:Account $\wedge$ b:Bank $\wedge$ b.getBal(a)=bal $\wedge$ $\calls{\_}{\prg{l}}{\prg{transfer}}{\prg{amt, from, to}}$}
%						{b.getBal(a) < bal}
%						{\false}
%						}
%					{by \textsc{Intrnl-Wrapped} and \textsc{If1-}$\longrightarrow$}
%			}
%			{\proofstepwithrule
%				{\onlyIfSingleEx
%						{l:Ledger $\wedge$ a:Account $\wedge$ b:Bank $\wedge$ b.getBal(a)=bal $\wedge$ $\calls{\_}{\prg{l}}{\prg{transfer}}{\prg{amt, from, to}}$}
%						{b.getBal(a) < bal}
%						{$\calls{\_}{\prg{b}}{\prg{transfer}}{\prg{a.password, amt, a, to}}$}
%						}
%					{by \textsc{Absurd} and \textsc{If1-}$\longrightarrow$}
%			}
%		\endproofsteps
%	\end{proofexample}
%}
%\end{minipage}
%{
%	\begin{proofexample}
%		\proofsteps{Ledger::TransferBalChange}
%			{\proofstepwithrule
%				{\onlyIfSingleEx
%						{l:Ledger $\wedge$ a:Account $\wedge$ b:Bank $\wedge$ b.getBal(a)=bal $\wedge$ $\calls{\_}{\prg{l}}{\prg{transfer}}{\prg{amt, from, to}}$}
%						{b.getBal(a) < bal}
%						{$\calls{\_}{\prg{b}}{\prg{transfer}}{\prg{a.password, amt, a, to}}$}
%						}
%					{by similar reasoning to \textbf{\prg{AuthBalChange}}}
%			}
%		\endproofsteps
%	\end{proofexample}
%}
%{
%	\begin{proofexample}
%		\proofsteps{Bank::TransferBalChange}
%			{\proofstepwithrule
%				{\onlyIfSingleEx
%						{a:Account $\wedge$ b, b$^\prime$:Bank $\wedge$ b.getBal(a)=bal $\wedge$ $\calls{\_}{\prg{b$^\prime$}}{\prg{transfer}}{\prg{pwd, amt, from, to}}$}
%						{b.getBal(a) < bal}
%						{a == from $\wedge$ pwd == a.password $\wedge$ b$^\prime$ == b}
%						}
%					{by similar reasoning to \textbf{AuthBalChange}}
%			}
%			{\proofstepwithrule
%				{\onlyIfSingleEx
%						{a:Account $\wedge$ b:Bank $\wedge$ b.getBal(a)=bal $\wedge$ $\calls{\_}{\prg{b$^\prime$}}{\prg{transfer}}{\prg{pwd, amt, from, to}}$}
%						{b.getBal(a) < bal}
%						{$\calls{\_}{\prg{b}}{\prg{transfer}}{\prg{a.password, amt, a, to}}$}
%						}
%					{by \textsc{If1-}$\longrightarrow$}
%			}
%		\endproofsteps
%	\end{proofexample}
%}
%Below we provide the proofs for each method in \jm[]{\ModD} that they cannot
%be used to leak the password of an account. \\
{
	\begin{proofexample}
		\proofsteps{AuthPwdLeak}
			{\proofstepwithrule
				{\hoareEx
						{a:Account $\wedge$ a$^\prime$:Account $\wedge$ a.password == pwd}
						{\prg{res}=a$^\prime$.authenticate(\_)}
						{res != pwd}
						}
					{by \funcSpec}
			}
			{\proofstepwithrule
				{\hoareEx
						{a:Account $\wedge$ a$^\prime$:Account $\wedge$ a.password == pwd $\wedge$ $\neg$ false}
						{\prg{res}=a$^\prime$.authenticate(\_)}
						{res != pwd}
						}
					{by Hoare logic}
			}
			{\proofstepwithrule
				{\onlyIfSingleExAlt
						{$\wrapped{\prg{pwd}}$ $\wedge$ a, a$^\prime$:Account $\wedge$ a.password=pwd $\wedge$ $\calls{\_}{\prg{a}^\prime}{\prg{authenticate}}{\_}$}
						{$\neg \wrapped{\_}$}
						{false}
						}
					{by \textsc{If1-Inside}}
			}
		\endproofsteps
	\end{proofexample}
	}
%	{
%	\begin{proofexample}
%		\proofsteps{changePassLeak}
%			{\proofstepwithrule
%				{\onlyIfSingleEx
%						{$\wrapped{\prg{pwd}}$ $\wedge$ a, a$^\prime$:Account $\wedge$ a.password=pwd $\wedge$ $\calls{\_}{\prg{a}^\prime}{\prg{changePass}}{\_, \_}$}
%						{$\neg \wrapped{\prg{pwd}}$}
%						{false}
%						}
%					{by similar reasoning to \textbf{AuthPwdLeak}}
%			}
%		\endproofsteps
%	\end{proofexample}
%	}
%	{
%	\begin{proofexample}
%		\proofsteps{Ledger::TransferPwdLeak}
%			{\proofstepwithrule
%				{\onlyIfSingleExAlt
%						{$\wrapped{\prg{pwd}}$ $\wedge$ a:Account $\wedge$ l:Ledger $\wedge$ a.password=pwd $\wedge$ $\calls{\_}{\prg{l}}{\prg{transfer}}{\_, \_, \_}$}
%						{$\neg \wrapped{\prg{pwd}}$}
%						{false}
%						}
%					{by similar reasoning to \textbf{AuthPwdLeak}}
%			}
%		\endproofsteps
%	\end{proofexample}
%	}
%	{
%	\begin{proofexample}
%		\proofsteps{Bank::TransferPwdLeak}
%			{\proofstepwithrule
%				{\onlyIfSingleExAlt
%						{$\wrapped{\prg{pwd}}$ $\wedge$ a:Account $\wedge$ b:Bank $\wedge$ a.password=pwd $\wedge$ $\calls{\_}{\prg{l}}{\prg{transfer}}{\_, \_, \_, \_}$}
%						{$\neg \wrapped{\prg{pwd}}$}
%						{false}
%						}
%					{by similar reasoning to \textbf{AuthPwdLeak}}
%			}
%		\endproofsteps
%	\end{proofexample}
%	}

\subsection{Part 3: Per-Step \Nec Specifications}
The next step is to construct proofs of necessary conditions for
\emph{any} possible step in our external state semantics.
In order to prove the final result in the next section,
we need to prove three per-step \Nec specifications: \prg{BalanceChange}, \prg{PasswordChange}, and \prg{PasswordLeak}.
\begin{lstlisting}[language=Chainmail, mathescape=true, frame=lines]
BalanceChange $\triangleq$ from  a:Account $\wedge$ b:Bank $\wedge$ b.balance(a)=bal
                 next b.balance(a) < bal   onlyIf $\calls{\_}{\prg{b}}{\prg{transfer}}{\prg{a.password}, \_, \prg{a}, \_}$
                 
PasswordChange $\triangleq$ from a:Account $\wedge$ a.password=p
                  next $\neg$ a.password != p   onlyIf $\calls{\_}{\prg{a}}{\prg{changePass}}{\prg{a.password}, \_}$
                  
PasswordLeak $\triangleq$ from a:Account $\wedge$ a.password=p $\wedge$ inside<p>
                  next $\neg$ inside<p>   onlyIf false
\end{lstlisting}
\jm[]{We provide the proofs of these in Appendix \ref{app:BankAccount}, but describe the construction of the proof of \prg{BalanceChange} here:
by application of the rules/results
 \prg{AuthBalChange}, \prg{changePassBalChange}, \prg{Ledger::TransferBalChange}, \prg{Bank::TransferBalChange}, \prg{BalanceEncaps}, and \textsc{If1-Internal}.}
\sophiaPonder[somethif missing. Where is Appendix F?]{}
\james{\textbf{MUST BE FIXED NOW}}
%
%by combining the results from \ref{s:BA-encap} and \ref{s:BA-classical} using \textsc{If1-Internal}. 
%Again, we elide the details of the proof of \prg{PasswordChange} and \prg{PasswordLeak} as they are similar to that
%of \prg{BalanceChange}. \\
%\noindent
%\begin{proofexample}
%\proofsteps{\prg{BalanceChange}}
%	{\proofstepwithrule
%			{\onlyIfSingleEx
%					{a:Account $\wedge$ b:Bank $\wedge$ b.getBal(a)=bal}
%					{b.getBal(a) < bal}
%					{$\calls{\_}{\prg{b}}{\prg{transfer}}{\prg{a.password, amt, a, to}}$}
%					}
%				{by \textbf{AuthBalChange}, \textbf{changePassBalChange}, \textbf{Ledger::TransferBalChange}, \textbf{Bank::TransferBalChange}, \textbf{BalanceEncaps}, and \textsc{If1-Internal}}
%		}
%\endproofsteps
%\end{proofexample}
%\begin{proofexample}
%\proofsteps{\prg{PasswordChange}}
%	{\proofstepwithrule{\onlyIfSingleEx
%				{a:Account $\wedge$ a.password=p}
%				{a.password $\neq$ p}
%				{$\calls{\_}{\prg{a}}{\prg{changePass}}{\prg{p}, \_}$}
%				}
%			{by similar reasoning to \textbf{BalanceChange}}
%	}
%\endproofsteps
%\end{proofexample}
%\begin{proofexample}
%\proofsteps{\prg{PasswordLeak}}
%	{\proofstepwithrule{\onlyIfSingleEx
%				{a:Account $\wedge$ a.password=p $\wedge$ $\wrapped{\prg{p}}$}
%				{$\neg \wrapped{\prg{p}}$}
%				{false}
%				}
%			{by similar reasoning to \prg{Balancechange}}
%	}
%\endproofsteps
%\end{proofexample}
\subsection{Part 4: Emergent \Nec Specifications}
Finally, we combine our module-wide single-step \Nec specifications to 
prove emergent behaviour of the entire system. Informally the
reasoning used in the construction of the proof of \SrobustB can be stated as
\begin{description}
\item [(1)]
If the balance of an account decreases, then
by \prg{BalanceChange} there must have been a call
to \prg{transfer} in \prg{Bank} with the correct password.
\item [(2)]
If there was a call where the \prg{Account}'s password 
was used, then there must have been an intermediate program state
when some external object had access to the password.
\item [(3)]
Either that password was the same password as in the {starting} 
program state, or it was different:
\begin{description}
\item [(Case A)]
If it is the same as the initial password, then since by \prg{PasswordLeak}
it is impossible to leak the password, it follows that some external object 
must have had access to the password initially.
\item [(Case B)]
If the password is different from the initial password, 
then there must have been an  {intermediate} program state when it 
changed. By \prg{PasswordChange} we know that this must have occurred
by a call to \prg{changePassword} with the correct password. Thus,
there must be a some  {intermediate} program state where the initial
password is known. From here we proceed by the same reasoning 
as \textbf{(Case A)}.
\end{description}
\end{description}
\begin{proofexample}
\proofsteps{\SrobustB}
	{\proofstepwithrule{\onlyThroughExAlt
				{a:Account $\wedge$ b:Bank $\wedge$ b.balance(a)=bal}
				{b.balance(a) < bal}
				{$\calls{\_}{\prg{b}}{\prg{transfer}}{\prg{a.password}, \_, \prg{a}, \_}$}
				}
			{by \textsc{Changes} and \prg{BalanceChange}}}
	{\proofstepwithrule{\onlyThroughExAlt
				{a:Account $\wedge$ b:Bank $\wedge$ b.balance(a)=bal}
				{b.balance(a) < bal}
				{$\exists$ o.[$\external{\prg{o}}$ $\wedge$ $\access{\prg{o}}{\prg{a.password}}$]}
				}
			{by $\longrightarrow$, \textsc{Caller-Ext}, and \textsc{Calls-Args}}}
	{\proofstepwithrule{\onlyThroughExAlt
				{a:Account $\wedge$ b:Bank $\wedge$ b.balance(a)=bal $\wedge$ a.password=pwd}
				{b.balance(a) < bal}
				{$\neg$$\wrapped{\prg{a.password}}$}
				}
			{by $\longrightarrow$}}
	{\proofstepwithrule{\onlyThroughEx
				{a:Account $\wedge$ b:Bank $\wedge$ b.balance(a)=bal $\wedge$ a.password=pwd}
				{b.balance(a) < bal}
				{$\neg$$\wrapped{\prg{a.password}}$ $\wedge$ (a.password=pwd $\vee$ a.password != pwd)}
				}
			{by $\longrightarrow$ and \textsc{Excluded Middle}}}
	{\proofstepwithrule{\onlyThroughEx
				{a:Account $\wedge$ b:Bank $\wedge$ b.balance(a)=bal $\wedge$ a.password=pwd}
				{b.balance(a) < bal}
				{($\neg$$\wrapped{\prg{a.password}}$ $\wedge$ a.password=pwd) $\vee$\\
				($\neg$$\wrapped{\prg{a.password}}$ $\wedge$ a.password != pwd)}
				}
			{by $\longrightarrow$}}
	{\proofstepwithrule{\onlyThroughExAlt
				{a:Account $\wedge$ b:Bank $\wedge$ b.balance(a)=bal $\wedge$ a.password=pwd}
				{b.balance(a) < bal}
				{$\neg$$\wrapped{\prg{pwd}}$ $\vee$
				a.password != pwd}
				}
			{by $\longrightarrow$}}
	{
	\begin{proofexample}
	\proofsteps{Case A ($\neg\wrapped{\prg{pwd}}$)}
			{\proofstepwithrule
				{\onlyIfExAlt
					{a:Account $\wedge$ b:Bank $\wedge$ b.balance(a)=bal $\wedge$ a.password=pwd}
					{$\neg$$\wrapped{\prg{pwd}}$}
					{$\wrapped{\prg{pwd}}\ \vee \neg\wrapped{\prg{pwd}}$}
					}
				{by \textsc{If-}$\longrightarrow$ and \textsc{Excluded Middle}}}
			{\proofstepwithrule{\onlyIfExAlt
					{a:Account $\wedge$ b:Bank $\wedge$ b.balance(a)=bal $\wedge$ a.password=pwd}
					{$\neg$$\wrapped{\prg{pwd}}$}
					{$\neg\wrapped{\prg{pwd}}$}
					}
				{by $\vee$E and \prg{PasswordLeak}}}
	\endproofsteps
	\end{proofexample}
	}
	{
	\begin{proofexample}
	\proofsteps{Case B (\prg{a.password != pwd})}
		{\proofstepwithrule{\onlyThroughExAlt
					{a:Account $\wedge$ b:Bank $\wedge$ b.balance(a)=bal $\wedge$ a.password=pwd}
					{a.password != pwd}
					{$\calls{\_}{\prg{a}}{\prg{changePass}}{\prg{pwd}, \_}$}
					}
				{by \textsc{Changes} and \textsc{PasswordChange}}}
		{\proofstepwithrule{\onlyThroughExAlt
					{a:Account $\wedge$ b:Bank $\wedge$ b.balance(a)=bal $\wedge$ a.password=pwd}
					{a.password != pwd}
					{$\neg\wrapped{\prg{pwd}}$}
					}
				{by $\vee$E and \prg{PasswordLeak}}}
		{\proofstepwithrule{\onlyIfExAlt
					{a:Account $\wedge$ b:Bank $\wedge$ b.balance(a)=bal $\wedge$ a.password=pwd}
					{a.password != pwd}
					{$\neg\wrapped{\prg{pwd}}$}
					}
				{by \textbf{Case A} and \textsc{Trans}}}
	\endproofsteps
	\end{proofexample}
	}
	{\proofstepwithrule{\onlyIfExAlt
				{a:Account $\wedge$ b:Bank $\wedge$ b.balance(a)=bal $\wedge$ a.password=pwd}
				{b.balance(a) < bal}
				{$\neg\wrapped{\prg{pwd}}$}
				}
			{by \textbf{Case A}, \textbf{Case B}, \textsc{If-}$\vee$I$_2$, and \textsc{If-}$\longrightarrow$}}
\endproofsteps
\end{proofexample}
