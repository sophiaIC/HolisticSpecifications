\section{Necessary Conditions and Robustness}
\label{s:intro}

%\subsection{Necessary conditions and Robustness}

%% Today's   software has been built 
%% over decades by combining modules and components of
%% different provenance and 
%% %different degrees of 
%% trustworthiness, and
%% is \emph{open}, interacting with other programs, devices, and people.

% according to IEEE standard,
% robust = The degree to which a system or component can function correctly in the presence 
% of invalid inputs or stressful environmental conditions.  

{Software needs} to be both {\emph{correct}} ({programs do what they
  are supposed to}) and {\emph{robust}} ({programs only do what
  they're supposed to}). Robustness means that
programs don't do what they aren't supposed to do, even in the
presence of untrusted or malicious {clients} \cite{ieeeStandard}.
% robust = The degree to which a system or component can function correctly in the presence 
% of invalid inputs or stressful environmental conditions.  
{Correctness is} classically specified
%formally 
through \citeasnoun{Hoare69} triples: a  precondition, a code snippet, and a
 postcondition. 
 For example,  {part of the classical specification
   of a \prg{transfer} method for a bank module is that the source account's balance decreases:}
 \begin{quote}
   \Scorrect\ \ $\triangleq$  
 % SD I could not make the below work ...
 % {\scriptsize \lstinline*{pwd=src.pwd  $\wedge$ \lstinline* src.bal=b} src.transfer(dst,pwd) {src.bal=b-100 * $\wedge$ \ldots } } }\\
 {\footnotesize{ $\{\,$\prg{pwd=src.pwd} $\,\wedge\,$ \prg{src.bal=b}$\,\}$ \prg{src.transfer(dst,pwd)} $\{$ \prg{src.bal=b-100}$\,\wedge\,\dots \}$ }} Calling \prg{transfer} on  {an account with the correct password} will transfer the money.
\end{quote}
Assuming termination, the precondition is a \emph{sufficient} condition for the {code snippet to behave correctly}: 
%assuming termination, 
the precondition (\eg providing the right 
password) guarantees that
the code (\eg call the \prg{transfer} function)
will always achieve the postcondition (the money is transferred).
 
    \vspace{.05in}
    % I think we need the vspace
 
%While 
\Scorrect  describes  {the \emph{correct use} of the module, but is \emph{not} concerned with its \emph{robustness}.}
{For example, can I pass an account to foreign untrusted code, in the expectation of receiving a payment,
but without fear that a malicious client might use the account to steal my money \cite{ELang}?}
 A first approach to specify robustness could be:
 

\begin{quote}
\SrobustA\ \ $\triangleq$ \ \ An account's balance does not decrease unless \prg{transfer} was called 
with the correct password.
\end{quote}

Specification \SrobustA % gives   the guarantee 
{guarantees} that it is not possible to  take money out of the account through some method other than \prg{transfer}
{or without providing the password}.
  Calling \prg{transfer}   with the  correct password is 
a \emph{necessary condition} for reducing the account's  balance.

\SrobustA is  crucial, but  not   enough:
it does not take  account of the module's \emph{emergent behaviour},
{that is, does not cater for the potential interplay of several methods offered by the module.}
 What if the module provided further methods which leaked the password,  
 {or allowed for it to be arbitrarily changed}?
{ While no single procedure call is capable of breaking the intent of \SrobustA, a sequence of calls might.}
{What} we really need is
 \begin{quote}
\SrobustB\ \ $\triangleq$ \ \ The balance of an account does not {\emph{ever}} decrease in the future unless some external 
object  {\emph{now}} has access to the account's current password.
\end{quote}
With \SrobustB, I can confidently pass my account to some untrusted client who
 does not have
 knowledge of the password; they may or may not make the payment I was expecting, but I
 know they will not  {be able to} steal my money \cite{ooToSecurity,miller-esop2013}.
 Note that \SrobustB  does not mention
 the names of any functions in the module, and 
 thus can be expressed without reference to any particular API ---
 indeed \SrobustB can constrain \emph{any} API with an account, an account
 balance, and a password.

\vspace{.04in}
 
 % \subsection{Robustness} 
 \subsection{\sd{Earlier Works}} % SD I do not think this section is about robustness, since all the paper is about Robustness% Robustness is not a new idea:
 
 %Our paper is 
 We are not the first  to address robustness guarantees or verification.
%\sd{robustness}:
% In contrast to \jm{correctness}  which is usually specified with \sophiaPonder{why talk again about Hiare triples}% such 
% Hoare triples,  
 %{\sd{Many} 
%kinds of} guarantees have been proposed\sophiaPonder[dropped: ``proposed for  robustness'']{}, differing in the level 
%of granularity,   target  language or calculi, and intended use.  
Object capabilities  \cite{MillerPhD, dd, threoremsFreeSep}, 
information control flow \cite{Zdancewic:Myers:01,noninteferenceOS}, 
 correspondence assertions \cite{Maffeis:aiamb:thesis00},
 sandboxing  \cite{robustSafetyPatrignani,sandbox},
robust linear temporal logic   \cite{RLTL2022}, are some of the proposals to ensure some level of robust safety. 
\sd{However, the guarantees in most of the  works are generic, \emph{e.g.} no dependencies of low value on
high values, rather than problem-specific,  \emph{e.g.} no decrease in balance without access to password,
which are the subject of our work.}
%
\Nec is the first approach that is able to  both express and prove
robustness specifications such as  \SrobustB.

%% {However, to our knowledge, 
%% our work is the first that is able to 
%% express specifications of the form %(S$_{\text{robust}_2}$),
%%  \SrobustB, along with a system
%% for verifying these specifications.}


\Nec is related 
% {Our work is related}
 to system such as {\sc{VerX}} \cite{VerX} and  \emph{Chainmail} \cite{FASE}.
Both these approaches can express necessary conditions
  like \SrobustA using
  temporal logic operators and implication, for example,  \SrobustA
 could be written as:
\\
 $\strut ~  ~ \hspace{.1in} \strut  \prg{a:Account} \ \wedge\ \prg{a.balance==bal}  \ \wedge\ \textbf{next}(\prg{a.balance<bal}) $\\
 $\strut ~ \hspace{2in} \strut \strut \strut \longrightarrow\    \textbf{calls}\, \prg{a.transfer(\_,a.password)}$
 \\
 { However, to express \SrobustB, one also needs what we call \emph{capability operators}, which talk about 
 provenance (``external object'') and
  permission (``\prg{x} has access to \prg{y}''). 
   {\sc{VerX}}  does not support capability operators, and thus cannot express   \SrobustB, 
   while  \emph{Chainmail} does support capability operators, and can express  \SrobustB. 
}  
 {\sc{VerX}} comes with a symbolic 
  execution system which can demonstrate adherence to its specifications, but doesn't have a proof logic, % to prove adherence,
   whereas, \emph{Chainmail}   {has neither a symbolic execution system, nor a proof logic.}
  % \susan[I have separated the tools rather than separating the features as I think it is clearer]{}
  
 {Temporal operators in {\sc{VerX}}   and  \emph{Chainmail}  are first class, \ie may appear in any assertions 
and form new assertions. This makes {\sc{VerX}}   and  \emph{Chainmail} very expressive,
and allows specifications which talk about any number of points in time.
However, this expressivity comes at the cost of making it very difficult to develop a logic to
prove adherence to such specifications.}
  
\vspace{.04in}

\subsection{\Nec}
\label{intro:this:work}
\Nec is a language for specifying a module's robustness guarantees 
% such as \SrobustB, 
and a logic 
to prove adherence to such specifications.

For the specification language we adopted  
\emph{Chainmail}'s    capability operators.
{For the 
  temporal operators, we observed that while their
   unrestricted combination with  other logical connectives allows us to talk about any
   number of points in time, the examples found in the literature talk about two or at most three such points. }

  
  {This led to the  crucial insight that we could merge  temporal operators and the implication 
 logical connective into our three}
 %Therefore, \jm[]{in searching for a sweet spot between expressiveness and the ability to 
 %prove examples}, we replaced the temporal operators with  {our three novel}
  \emph{necessity} operators. 
 One such necessity operator is \\
$ 
\strut \hspace{1.7in} \onlyIf {A_{curr}} {A_{fut}} {A_{nec}}
$  
% \begin{lstlisting}[mathescape=true, language=chainmail, frame=lines]
%                                from ${A_{curr}}$ to ${A_{fut}}$ onlyIf ${A_{nec}}$ 
%\end{lstlisting}
%  %      $A$          from ${A_{curr}}$ to ${A_{fut}}$ onlyIf ${A_{nec}}$          from ${A_{curr}}$ to ${A_{fut}}$ onlyThrough ${A_{nec}}$
\\
This form says that  
a  {transition} from a current state satisfying assertion $A_{curr}$ to a future
state satisfying $A_{fut}$  is possible only if  the   necessary 
condition
$A_{nec}$ holds in the \emph{current} state.
Using this operator, we can formulate  \SrobustB  
as
\begin{lstlisting}[language = Chainmail, mathescape=true, frame=lines]
   $\text{\SrobustB}$  $\triangleq$   from   a:Account $\wedge$ a.balance==bal    to   a.balance < bal
               onlyIf  $\exists$ o.[$\external{\texttt{o}}$ $\wedge$ $\access{\texttt{o}}{\texttt{a.pwd}}$]
\end{lstlisting}
Namely, a transition from a  {current} state where an account's balance is \prg{bal}, to a  {future} state where 
it has decreased, may \emph{only} occur if  {in the current state} some unknown client object  
has access to that account's password. 
More discussion in \S\ref{s:bankSpecEx}. 


We also support two further \Nec operators:
\\
$ 
\strut \hspace{.4in} \onlyIfSingle {A_{curr}} {A_{fut}} {A_{nec}}. \strut \hspace{.4in} \onlyThrough {A_{curr}} {A_{fut}} {A_{intrm}}
$ 
\\
{The  first says    that 
a  \emph{one-step} {transition} from a current state satisfying assertion $A_{curr}$ to a future
state satisfying $A_{fut}$  
is possible only if % the   necessary condition
$A_{nec}$ holds in the \emph{current} state.   
The   second says that a change from %a current state satisfying 
 $A_{curr}$ to  $A_{fut}$  may happen only if % the necessary condition
 $A_{intrm}$ holds in some \emph{intermediate} state.}
 
  
  \vspace{.02in}
  
Unlike  \emph{Chainmail}'s temporal operators, 
 the necessity operators %  $\onlyIf {\_} {\_} {\_}$  and $\onlyThrough {\_} {\_} {\_}$
 are second class, and may not appear in the assertions  {(\eg  ${A_{curr}}$)}. 
 This simplification enabled us to develop our proof logic. 
 Thus, we {have reached} a  sweet spot between expressiveness and 
 provability.
 



%\jm[not sure if this should go here ...\sd{Actually, not sure what is said here is true} Here what it said, but I think it breaks the flow
%Unlike other work on specifying robust safety for software,  \Nec allows proofs of not just simple safety 
%specifications, but specifications that rely on reasoning about emergent behavior and complex interactions 
%across a module's interface (such as \SrobustB).
%Further, beyond unforgeability and privacy, \Nec is agnostic about much of the specifics
%of module implementation, only that module code can be verified using some Hoare logic. 
%This is a significant advantage, as it makes \Nec broadly applicable to any such languages that use Hoare logic to establish correctness.. \sd{it is not
%a matter of language}]{}


%\textit{Necessity Specifications}  are
%\Nec supports the usual assertions, $A$, (including permission, provenance,
%and control), as well as three \jm[]{novel} necessity 
%{operators. We now show one of these operators} 
% shown below.
% \sophian[Needs to be laid out differently to put the four forms in]{}
% Thus, we  have % one  of  the two -- omitting, as there are three!
% the following forms:
%\jm[are we missing \texttt{from $A_{curr}$} nxt $A_{fut}$ onlyIf $A_{nec}$?]{}
%\begin{lstlisting}[mathescape=true, language=chainmail, frame=lines]
%       $A$          from ${A_{curr}}$ to ${A_{fut}}$ onlyIf ${A_{nec}}$          from ${A_{curr}}$ to ${A_{fut}}$ onlyThrough ${A_{nec}}$
%\end{lstlisting}
 


 %\sophiaPonder[not sure we need to have the latter, 
% and do not have temp logic form of it]{}.
% For same later section?
%Unlike \citeauthor{VerX} or \citeauthor{FASE}
%the necessity operators $\onlyIf {\_} {\_} {\_}$  and $\onlyThrough {\_} {\_} {\_}$
%are second class, and may nor appear in the assertions $A$.
%%support necessity specifications with this explicit
%% ``$\onlyIf {A_1} {A_2} {A_3}$'' syntax
%%
%%and concomitant specialised inference system.
%%
%%
%Our assertions $A$ support the usual expressions about program state
%%(e.g. \prg{x.f > 3}),
% logical connectives and quantifiers, 
%%(e.g. $\wedge$, $\forall$), 
%and additional predicates
%to capture \textit{provenance} (whether an object $o$'s definition is
%$\internal{\texttt{o}}$ or $\external{\texttt{o}}$) to the current
%module, and \textit{permission} \cite{miller-esop2013} (whether an
%object $o$ has direct access to another object $o'$:
%$\access{\texttt{o}}{\texttt{o'}}$).

% \susan[Probably the encapsulation statement should be the last introduced as a 'pruning'? technique.\sd{SD: agrre}]{\Nec specifications are high-level, system-wide properties. 
We faced the challenge  how to develop a logic that would enable us to prove that code 
 adhered to  specifications {talking of system-wide properties.} The Eureka moment was the realisation that all the information we required was hiding  { in the individual methods' classical specifications.}
%%Of course it was not straightforward to extract what we needed, but we came up with a successful technique.}
%\sophiaPonder[I think we need to make the brown stuff fit better the next paragraph.]{}
%

 {Our logic %The \Nec logic
 is based on the crucial insight that the specification $\onlyIf{A_1} {A_2} {A_3}$ is 
logically equivalent
% with 
{to}
 $\forall \prg{stmts}. \{ A_1 \wedge \neg A_3\} \prg{stmts} \{ \neg A_2 \}$ -- that is,
 with an infinite conjunction  of Hoare triples.}
%We want to leverage prior work on program verification, however, we have the problem that Hoare logics only reason about given, known, statements.
 {This leaves the challenge that no Hoare logics support such infinite conjunctions.
 Three breakthroughs lead to our solution of that challenge:}
 \begin{description}
 \item
 [Per-method specs] 
  The Hoare triple 
$ \{ A_1 \wedge \neg A_3 \} \ \prg{x.m(ys)}\  \{ \neg A_2 \}$ is logically equivalent 
% with 
\susan[]{to}
the specification
$ \onlyIfSingle {(A_1 \wedge {\calls{\_}{\prg{x}}{\prg{m}}{\prg{ys}} } )} {A_2} {A_3}$. % is logically equivalent with the  
Thus, we can  leverage Hoare triples to reason about one particular method call.  
 
 \item 
 [Per-step specs] % Second, 
 If an assertion $A_2$  is \emph{encapsulated} by a module (\ie the only way we can go from a 
 state that satisfies $A_2$ to a state that does not, is through a call to a method in that module), then
{the
\emph{finite conjunction}
 $ \onlyIfSingle {(\, A_1 \wedge \neg A_2 \wedge {\calls{\_}{\prg{x}}{\prg{m}}{\prg{ys}}}\, )} {A_2} {A_3}$
 for all methods of that module is logically equivalent 
 % with 
 {to}
 $ \onlyIfSingle {A_1 \wedge \neg A_2} { A_2} {A_3}$. }
  \item [Proof logic  for emergent behaviour] %an inference system 
  combines several specifications to reason about the
  emergent behaviour, \emph{e.g.,} 
   $ \onlyThrough  {A_1} {A_2} {A_3}$  and $ \onlyIf  {A_1} {A_3} {A_4}$ implies 
   $ \onlyIf  {A_1} {A_2} {A_4}$.
 \end{description}
More in \S\ref{s:approach}.
 

%The \Nec logic is based on four main ideas: 
%First, some assertions are \emph{encapsulated}, \ie
%their validity may only be affected by internal calls. 
%% \susan[same question - is may the correct word? This said module-internal before I edited it.]{}
%Second, by leveraging the classical  specifications of methods 
%(\ie the sufficient conditions) we can obtain per-method-call
% \Nec specifications; that is, necessary preconditions
%  for a given effect and a given method call (at a very abstract level, this is
%similar to \citeasnoun{threoremsFreeSep}).
%Third we infer a per-step-condition, \ie, a necessary condition given an effect and a single, unspecified step. This step could be an internal call, or any kind of external step.
%Fourth,  a novel proof system allows us to combine 
%per-method \Nec specifications and encapsulated assertions 
% to obtain per-module   \Nec specifications; these describe a module's
% emergent behavior.
%More in \S\ref{s:approach}.


 
  The current work is based on a simple, imperative, typed, object oriented
language with unforgeable addresses and private fields. We believe
 that our approach is applicable to several programming paradigms, and 
 that   unforgeability and privacy
 can be replaced 
 by lower level mechanisms such as capability machines \cite{vanproving,davis2019cheriabi}.
 


%The below is good and perhaps we shpuld use it
% The challenge here is twofold: How do we specify the bad things we are
%concerned about, and how do we prove that the bad things we've
%specified do not happen?  These challenges are difficult because we
%cannot refer to just one component of a software system.  A sufficient
%specification can deal with a single component in isolation --- a
%single function for pre- and postconditions; a single class or data
%structure for invariants. A necessary specification, however, must
%provide guarantees which encompass the software system in its
%entirety, and constrain the emergent behaviour of all its components,
%for an open system, all possible sequences of API invocations.
% 
% 
%
% 
%The importance of distinguishing between sufficient and necessary
%specifications of various kinds has a long history in Computer
%Science.
% 
%  Type systems ensure entire classes of bad things can't
%happen, preserving execution even if memory structures are greatly
%corrupted \cite{Rinard03}.
%%
%More recently, \citeauthor{ddd} and \citeauthor{sandbox} with their robust safety and
%\citeauthor{FASE} with their holistic systems have tackled open world systems to prevent bad things from happening from untrusted code.
%\citeauthor{ddd} use techniques from security to ensure \jm[]{there} isn't undesirable leakage, \citeauthor{sandbox} build a sandbox and have a sophisticated type system to protect it and \citeauthor{FASE} have
%necessary conditions, which they expressed through temporal
%operators.

% In the next section, (\S\ref{s:outline}),  we outline our approach using the
% bank account as  a motivating example.%This example demonstrates, among other things, that the specification need  refer  to  individual methods in a module.
%Moreover, adherence to a Necessity Specification is not monotonic, in the following two senses:
%Adding a method to a module does not necessarily preserve adherence to a Necessity Specification,
%and while separate methods may adhere to a  Necessity Specification, their combination does
%not necessarily do so: Necessity Specifications capture a module's emergent behaviour.


\subsection{Paper Organization and Contributions}

%%In the next section, (\S\ref{s:outline}),  we outline our approach using a
%% bank account as  a motivating example.

%\jm[should this be ``a bank account''? This is the first time we mention it]{}
%
The contributions of this paper are:\begin{enumerate}
 \item
%\Chainspec, a 
A language to
express \Nec specifications (\S\ref{s:semantics}).
%\sophiaPonder[to revisit  that is  expressive  enough to encode the \emph{Chainmail} examples from \citeasnoun{FASE}.
%-- this bit may diminish our contribution!] {}

 \item
%\Chainlogic, a 
A logic for proving a module's adherence to 
 \Nec specifications (\S\ref{s:inference}), and a proof of soundness of the logic, (\S\ref{s:soundness}),
both mechanized in Coq. 
 \item
A proof in our logic % the bank account 
  that our bank module obeys its \Nec specification (\S\ref{s:examples}),  also  mechanized in Coq.
\end{enumerate}


%\jm[]{
%Our formalization of \Nec does have two limitations. 
%Firstly, it is specifically a logic for 
%necessity and robustness properties. \Nec 
%is parametric with respect assertion satisfaction, correctness,
%encapsulation, and the type system. Much of these are
%well-trod ground in the literature, and where needed we
%do introduce simplistic language mechanisms to deal with them (e.g. a simple type system or notion of encapsulation).
%Secondly, we do not allow for ``callbacks'' 
%to external objects. This is a common restriction in the literature
%as many either prohibit callbacks, or require
%``effectively callback free contracts'' \cite{VerX},
%or place significant restrictions on callbacks \cite{}
%}

Our formalization of \Nec % has a number of limitations: it
{  is 
parametric} with respect to assertion
satisfaction, encapsulation, and the type system.
{{Moreover, in line} with other work in the literature,} we forbid 
``callbacks'' out to external objects; and also
classical specifications require explicit framing.
We will discuss these  issues % limitations 
further as we place \Nec into the context of 
related work (\S\ref{s:related}) and consider our overall conclusions
(\S\ref{s:conclusion}). 
%
The Coq proofs of 
(2) and (3) above appear in the
supplementary material, along with \sd{a}ppendices containing expanded 
definitions and further examples.
