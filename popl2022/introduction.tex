 \newcommand{\prg}[1]{{\texttt{#1}}}
 \section{Introduction}
I DIDN"T CHANGE ANYTHING TECHINICAL (OTHER THAN CHANGING 10 to 100). INSTEAD I WROTE A QUERY AND I WOULD LIKE SOMEONE ELSE TO READ WHAT I WROTE AND MAKE THE CHANGE. - SUSAN
%Current systems are complex, and  built out of many different components of different provenance and different degree of trustworthiness.
  
{\se{Although complex software is built out of modules/components of different provenance and different degrees of trustworthiness, we want all of its behaviour to be unsurprising.}}
In order for the resulting complex system to offer the
desired services, the individual components 
need to  be able to make ``good things''   happen.
But in order for the resulting  complex system
 to offer these services \emph{robustly},  the individual components 
 need to be able to prevent ``bad things'' from happening. 

To demonstrate that  good things \emph{can} happen, we use witnesses;
eg  a precondition and a code snippet, whose execution will establish the
wished-for effect.  For example, if I send an email to a valid address it will
be delivered to its recipient. The precondition and the code snippet
are \emph{sufficient} conditions for the desired effect. 

Our work is about the specification and verification of robustness: that bad things \emph{will 
not} happen. For example, nobody gets to read an email sent only to me unless I forwarded it to them.   
Note that such an assertion is agnostic to the set of methods supported by the mail-module.

The challenge here is twofold: How do we specify the ``bad things'',
 and how do we prove that such bad things do not happen? 

To specify undesirable properties, we cannot refer to one particular piece of code, instead we need to provide guarantees which encompass the software in its entirety -- the \emph{holistic} system. We need to consider the emerging behaviour of all possible functions executable in all possible sequences, including interleavings. Rather than considering the sufficient conditions to achieve a given effect, we express necessary preconditions for the holistic system. For example, a third party will not get to read my email unless I forward it to them - the forward is a necessary precondition for the reading. To reason about necessary conditions we develop our Logic of Necessity.

There has been work in the past to expand specification of systems beyond the specification of good things. <<James we need a related work paragraph>>

%To specify the bad things, we cannot
%refer to one particular piece of code, % and cannot refer to one witness;
% instead, we need to make a guarantee which encompasses all possible
% functions executable by a module in all their possible sequences or even interleavings --
 %we need to consider their \emph{emerging} behaviour.
 %Rather than considering the sufficient conditions to achieve a certain effect,
 %we express the necessary precondition. For example,    a third party 
 %will not get to read my email unless I forward it to them -- the forwarding
% is a necessary precondition for the reading.
 
 %To reason about necessary conditions, we develop a special
 %logic with such necessity triples. The most basic such holistic assertions 
 %can be derived from classical assertions (sufficient \jm[old:truples]{triples})
 %under assumptions of encapsulation;
 %they can then be further combined using our logic of necessity.
 
 We illustrate our ideas in terms of the following example:
 \prg{Account}s have a balance and a 
password. One may transfer \prg{100} units from one \prg{Account} to another, but only provided
that  the caller provides the right password. Below we show the version \sf{I} of the code for 
a class \prg{Account}. 
We use a Java-like syntax,  and assume that fields are private in the sense of C++, \emph{i.e.,}
 only methods of that class may read or write these fields,
 and   that addresses are unforgeable -- so  there is no way to guess a password.

 

\begin{lstlisting}
class Account{
   field bal:int 
   field pwd: Password 
   method transfer(dest:Account, pwd':Password) -> void {
      if (this.pwd==pwd') {
            this.bal-=100;  dest.bal+=100; }  }
}
\end{lstlisting}

as well as a ``classical'' specification of the method \prg{transfer}:

 (ClassicSpec)$  \ \ $  $\triangleq$
\jm[changed: balance -> bal]{}  

\begin{lstlisting}[mathescape=true, frame=lines]
  method transfer(dest:Account, pwd':Password) -> void {
       ( PRE:  this.bal=bal1 $\wedge$ this.pwd==pwd' $\wedge$ dest.bal=bal2 $\wedge$ dest=/=this 
         POST: this.bal == bal1-100 $\wedge$  dest.bal=dest.bal+100 )
       ( PRE: this.bal=bal1 $\wedge$ this.pwd=/=pwd' $\wedge$ dest.bal=bal2
         POST: this.bal == bal1 $\wedge$  dest.bal=dest.balance )
       ( PRE: a : Account $\wedge$ a=/=this $\wedge$ a=/=dest  $\wedge$ a.bal=bal  
         POST:  a.bal=bal $\wedge$ a.pwd=pwd1)
       ( PRE: a : Account $\wedge$ a.pwd=pwd1  
         POST: a.pwd=pwd1)       
\end{lstlisting}\footnote{Perhaps omit some of the lines here, but we do need them all in the full discussion}
 
  
 
 Now consider two further versions of the class account, given in Figure \ref{fig:ExampleAccount}.
 In version {\sf{II}} and version {\sf{III}} the class has an additional method, \prg{set}, which enables the resetting of the password.
The method \prg{transfer} in all three versions of the class \prg{Account} satisfies the (ClassicSpec), 
however, while the first and third version of \prg{Account} are ``robust'', the second version is not.
Namely version II allows any client to change the password of the account, and then to repeatedly withdraw 
money from it.
  
% On the other hand, we expect our software -- even if complex -- to provide some simple, high level
%guarantees, e.g. email addressed to me personally will not be read by a third party unless I explicitly 
%forwarded it to them.
%We except  our software to  behave correctly, even when used by a careless or malicious third party. 
%Such use of a software often consist of a sequence of actions performed on the module. 
%
%Software components respond to single actions, 
%or to sequences of such single actions. 
%When thinking about a software component we want think about the behaviour of each 
%action in isolation, but also about the \emph{emergent behaviour}, ie all
% the possible effects of the combinations of these actions. 
  
  
 
 \begin{figure}[hbtp]
 \begin{tabular}{lll}
\begin{minipage}{0.35\textwidth}
\begin{lstlisting}
class Account{
   field bal:int 
   field pwd: Password 
   method transfer(..) 
        ... as earlier ...
   method set(pwd':Password){
         this.pwd=pwd' }
}   
\end{lstlisting}
\end{minipage}
  &\ \ \  \ \   &
\begin{minipage}{0.50\textwidth}
\begin{lstlisting}
class Account{
   field bal:int 
   field pwd: Password 
   method transfer(..) 
        ... as earlier ...
   method set(pwd',pwd'':Password){
         if (this.pwd==pwd'){this.pwd=pwd''} }
}
\end{lstlisting}
\end{minipage} 
 \end{tabular}
  \caption{class \prg{Account} version II, and \ \ \ \ \ \  class \prg{Account} version III}
 \label{fig:ExampleAccount}
 \end{figure}

 
 The remit of our work is to find ways to specify robustness. 
 The flaw in version {\sf{II}} arises from \emph{emergent} behaviour as \prg{set} 
 can be used to overwrite the
 password, and then using the new password \prg{transfer}  can be called.
 If we want the \prg{Account} class to be robust, we must prohibit the password from being freely available.
 Therefore, we propose a holistic specification which says that
 the \prg{balance} of an \prg{Account} reduces only if an object which does not belong to the
 class \prg{Account} has access to the password:
 
 \begin{lstlisting}[language = Chainmail, mathescape=true, frame=lines]
(HolisticSpec)  $\triangleq$  a:Account $\wedge$ a.balance==bal
                   to a.balance < bal
                   onlyIf $\exists$ o'.[$\external{\texttt{o'}}$ $\wedge$ $\access{\texttt{o'}}{\texttt{a.pwd}}$]
\end{lstlisting}
 
 WHY DOES o HAVE A PRIME?
 
 In more detail, the specification from above says that if in the current
 configuration \prg{a} is an \prg{Account},
 % with balance \prg{bal}, 
 and in some future configuration \prg{a} will have a balance less than the current one, then, in the \emph{current} configuration
 there must exist some object \prg{o}, which is \emph{external} to our module (does not belong to class
 \prg{Account}), and which has access to \prg{a}'s password.
 
 
 Thus, having access to the password is a necessary condition for the balance to reduce.
 Note, that the specification only talks about effects (here the balance reducing), and does not
 talk about individual methods (such as \prg{set} or \prg{transfer}).
 

 The holistic specification language, \Chainmail, extends traditional specifications with
notions of access (which object points to which), control (which methods are called)\footnote{Sophia -perhaps drop as confusing? Susan -I think both of the bracketed phrases should be omitted, as we have already used the word access when describing the passwords and we don't mention control anywhere}, 
 the distinction between external and internal objects, and necessary conditions, 
 whereby the assertion $A~to~A'~onlyIf~A''$ expresses that if in in the current configuration
 $A$ holds, and in some future configuration $A''$, <<SHOULD BE A SINGLE PRIME-SUSAN>> will hold, then $A''$ must also hold in the 
 current configuration.  
 
  %THIS NEEDS TO BE PART OF THE RELATED WORK EARLIER IN THE PAPER AND NOT HERE - SUSAN
 Earlier work by \cite{fase} was also concerned with 
 holistic specifications and necessary conditions, which they expressed through time operators.
 \Chainmail is less rich than Chainmail, however it comes with a proof system which Chainmail lacked.
 
  
 The contributions of this paper are as follows:
 
 \begin{enumerate}
 \item
 We propose \Chainmail, a specification language with which to
express holistic specifications. 
 \item
 %We propose a logic of triples of assertions that allows us to prove
 %a module's adherence to such a holistic assertion. 
 We propose a Logic of Necessity for writing holistic assertions and for proving a module's adherence to said assertions.
 This logic builds on top of classical pre- post- conditions.
 \item
 We prove soundness of our logic.  
 \item
 We use our logic to prove adherence to the holistic assertion of the example
outlined  in this section
 \end{enumerate}
 
 QUESTIONS:
 \begin{itemize}
 \item
 holistic assertion -- different name or say we adopt from FASE; holistic -> comprehensive, integrated, aggregate <<I THINK WE SHOULD KEEP THEM AS HOLISTIC  - SUSAN>>
 \item
 Name for our logic?  Holistic logic? What is we have a different take on holistic next year? Necessity triples? 
\item
If our \Chainmail is simpler than the FASE Chainmail, then we need to argue that we can express all FASE examples
 \end{itemize}
  
The rest of the paper is organised as follows: .... 

%Section
%~\ref{sect:motivate:Bank} 
%\sd{gives an example from the literature} which we will use 
%to elucidate key points of \Chainmail.
%%motivates our work via an example, and then section
%~\ref{sect:chainmail} presents the \Chainmail\ specification
%language.  Section~\ref{sect:formal} introduces the formal model
%underlying \Chainmail, and then section~\ref{sect:assertions} defines
%the 
%semantics of \Chainmail's assertions.
%% SD the below is NOT ture
%%full details are relegated toappendices.   
%Section~\ref{sect:example} shows how key points of 
%exemplar problems can be specified in \Chainmail,
%section~\ref{sect:discussion}
%discusses our design, \ref{sect:related} considers related
%work, and section~\ref{sect:conclusion} concludes.
%We relegate various details to appendices.
