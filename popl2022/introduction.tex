\newcommand{\sophiaPonder}[2][]{\ponders{Sophia}{blue}{#1} \textcolor{blue}{#2}\xspace}
\renewcommand{\sophia}[2][]


\renewcommand{\Chainmail}{\textit{Necessity}$^{spec}$\xspace}  
\newcommand{\Chainspec}{\textit{Necessity}$^{spec}$\xspace}
\newcommand{\Chainlogic}{\textit{Necessity}$^{logic}$\xspace}
\newcommand{\NecessitySpecifications}{Necessity Specifications\xspace}
\newcommand{\NecessitySpecification}{Necessity Specification\xspace}
\renewcommand{\SpecO}{\textit{Basic}$^{spec}$\xspace}  
\renewcommand{\Loo}{\textit{Lang}$^{oo}$\xspace}

\section{Introduction}

Today's   software has been built 
over decades by combining modules and components of
different provenance and different degrees of trustworthiness. It 
is open, interacting with other programs, devices, and people.
In this complex setting, modules need to be both useful ``good things
will eventually happen'' and robust 
``bad things will never happen''. % (safety), 
% (liveness)  
\cite{Lamport77}. 
 
  \vspace{.03in}
\emph{Good things happening} are typically specified formally 
 through triples consisting of a  precondition, a code snippet, and some
 postcondition \cite{Hoare69}.
 For example,
 \begin{quote}
(*)\  \ Calling the \prg{transfer} function  using the correct 
 password will transfer the money.
\end{quote}
  The precondition is a \emph{sufficient} condition for the code snippet to
make the good thing happen: given the precondition (eg providing the right 
password), executing a
code snippet (eg call  the \prg{transfer} function) 
is guaranteed to achieve the postcondition (the money is transferred).
%Proving adherence to such specifications  poses considerable challenges, especially when in the 
%context of aliasing, re-entrancy, multithreading, memory models, to name a just a few. 
%Such challenges  have been studied intensively over the last  two decades 
% \cite{BIMutable,objInvars,DenyGuarantee,RealxedC11} -- again just naming a few.

   \vspace{.03in}
\emph{Bad things not happening}
   irrespective of whether a module's clients are benign or malicious, is what makes a module  robust. 
%\sophiaPonder[I fear this sentence breaks the flow]
%{We go beyond  ``design by contract'' \cite{Meyer09},  and
%cannot rely on a module's client sticking to the module's contract.}
{For example}, for a bank module to 
be robust, it must 
guarantee  that
\begin{quote}
(**)\  \ The balance of an account does not decrease unless the \prg{transfer} function was called with the  correct password.
\end{quote}
%   in a context where no external agent knows the account's password, 
% the balance of 
%that account will not decrease -- such a guarantee allows one to confidently 
% pass their bank account into untrusted code, in the expectation of receiving some payment.}
The call of \prg{transfer}  with the  correct password is thus
a \emph{necessary condition} for any reduction of  balance.
 Condition   
(**) is crucial in order to confidently pass the bank account into untrusted code, 
in the expectation of receiving some payment, but without fear
that a malicious client might steal my money \cite{ELang,miller-esop2013}.
% Expressed as a \emph{necessary condition}, this guarantee 
%
% SD Perhaps this sentence should come later
% In this paper we propose how  to specify
%such necessary conditions formally, as well as ways to provide adherence to
%such specifications.
In contrast to ``good things   happening''  which are usually specified with % such 
Hoare triples,  a multitude of different guarantees have been proposed for ``bad things  not
happening'', differing in the level 
of granularity,   target  language or calculi, and intended use.
% , and the ways by which  the guarantees can be established.

 
%Type systems   % have been developed for various programming languages. They 
% give  relatively coarse-grained guarantees, \emph{e.g.,} 
% that ``message not understood'' errors 
%will not be thrown \cite{EiffelCook},
% private methods will not called by external modules \cite{JavaAccess},
%  no object or method will directly point inside the internal representation of some other object
%\cite{ownalias},
%or that race conditions will not arise  \cite{mindori}. % -- again, just naming a few.
%\sophiaPonder[]{Drop the types?}
 

\emph{Information-flow control}  systems have been developed for
various calculi and programming languages. Their 
 guarantees are coarse-grained:  the contents of high security variables cannot
be affected by the values of low security variables 
\cite{Zdancewic:Myers:01,noninteferenceOS}. 
 
\emph{Correspondence assertions} are more fine-grained. Proposed for 
process calculi, they %can guarantee
{express}  that if one principal ever reaches a certain point in the protocol, 
then some other principal has previously reached some other matching point in the protocol; 
the term \emph{robust safety}  expresses that correspondence assertions are true in the presence of any opponent %expressible in the calculus 
\cite{correspondence}. 
% SD thought the below was true, but nor think it is not
% Correspondence assertions have also
% been proposed for Javascript. 
A related approach,  \emph{authorisation policies}, 
%A related problem is \emph{authorisation policies}, which 
promises that certain actions will not be taken on certain objects
unless corresponding  rights had been granted  \cite{Maffeis:aiamb:thesis00}.

In   \emph{object capabilities} \cite{MillerPhD},
effects can only be produced by sending messages to objects,  there
is no ambient authority, and  objects 
have unforgeable identities.
In the context of object capabilities, \citeauthor{ddd} define  \emph{robust safety for Javascript}  
to mean that the untrusted environment of a program cannot violate 
its internal invariants, and developed  a verification methodology   to 
prove   that programs that  export only properly wrapped values  are robustly safe.
In the same context, \citeauthor{dd} develop Kripke models to reason about 
the result of execution of some given code snippet in parallel
with arbitrary, unknown code.  
 \cite{threoremsFreeSep} derive usage protocols out of a module's code.

{None} of the approaches outlined so far were developed with
 the aim of expressing guarantees such as (**):  
  Information flow control systems are too
 coarse grained. Correspondence assertions target process calculi and do not exactly match 
 the time of the effect to its cause. \citeauthor{ddd} are concerned with the preservation of
 internal invariants. \citeauthor{dd}   
 concerned with the \emph{effects} of known code,
 while \cite{threoremsFreeSep}  reverse-engineer usage protocols.
 
Recently,  {\sc{VerX}}  \cite{VerX}  and   \emph{Chainmail}  \cite{FASE} 
 added temporal operators to specification languages, and thus can express necessary conditions.  
 Both  {\sc{VerX}}  and   \emph{Chainmail} assertions support the usual 
   logical connectives and can refer to current state and function calls.
   %assertions about the current state, and in addition,
 %{\sc{VerX}}  can   talk about function calls,
 \emph{Chainmail} 
 %can talk about function calls, 
 can also talk about \emph{access} (whether an
 object has direct access to another object -- either through its fields or 
 through the arguments to a function call),  and about \emph{provenance} (whether an object
 belongs to the current module, or some client, unknown module).
Access and provenance are inspired by the object capabilities
 literature \cite{MillerPhD}. 
 
% is used to express safety properties of smart contracts;
% adherence to such specifications is proven through symbolic 
% execution.  \emph{Chainmail} is used to express general safety properties; it
% also includes predicates talking about provenance
% and access (more below), but lacks a proof system.
 
%\susan[font?]{}\paragraph{What we \emph{really} need} 
\vspace{.04in}

However, while (**) is a necessary condition  to confidently pass 
 an account into untrusted code, it is \emph{not} sufficient.
Guarantee (**) does not take  account of the module's \emph{emergent behaviour}.
 What if the module leaks the password or what if the module allows me to override the
 password without knowledge of the old one? 
 What we really need is
 \begin{quote}
(***)\  \ The balance of an account does not decrease ever in the future unless some module-external 
object has access to the account's password now.
\end{quote}
With (***), I can confidently pass my account to some untrusted client who
  does not have
 knowledge of the password; they may or may not make the payment I was expecting, but I
 know they will not steal my money \cite{ooToSecurity,miller-esop2013}.
 Note that (***)  does not mention
 the names of any functions in the module, and 
 thus can be expressed without reference to the particular API. 
 

  Both {\sc{VerX}} and  \emph{Chainmail} can express (**), but {\sc{VerX}}  cannot express (***). 
  Namely, (***)  requires the concepts of provenance and access.
 On the other hand, {\sc{VerX}} can prove adherence to  specifications through symbolic 
  execution, while  \emph{Chainmail}   lacks a proof system. 
  
\vspace{.04in}

This work is  the first to express necessity specifications, including those like (***), and 
also to prove adherence to such specifications.
We propose \Chainmail, a specification language % for   \emph{Necessity Specifications} 
 to describe a module's safety guarantees,
and 
\Chainlogic,   % \emph{Necessity Logic} 
a logic to prove a module's adherence to such specifications.
The current work is based on a simple, imperative, typed, object oriented
language with unforgeable addresses and private fields.\footnote{We believe
 that our approach is applicable to several programming paradigms, and 
 that   unforge-ability and privacy
 can be replaced 
 by lower level mechanisms such as capability machines \cite{vanproving,davis2019cheriabi}.
  }


 

%\textit{Necessity Specifications}  are
\Chainspec supports the usual assertions (incl. calls, provenance,
and access), as well as the application of one of two necessity operators shown below.
Thus, we  have % one  of  the two -- omitting, as there are three!
 following forms:
\begin{lstlisting}[mathescape=true, language=chainmail, frame=lines]
       $A$          from ${A_{curr}}$ to ${A_{fut}}$ onlyIf ${A_{nec}}$          from ${A_{curr}}$ to ${A_{fut}}$ onlyThrough ${A_{nec}}$
\end{lstlisting}
%
The first form says that $A$ always holds;  $A$ is an invariant.
The second form says that  a change from a current state satisfying $A_{curr}$ to a future
state satisfying $A_{fut}$ %(i.e.\ the transition  $A_{curr}$ to $A_{fut}$ being an effect) 
is possible only if the necessary condition
$A_{nec}$ holds in the \emph{current} state.
The third form says that a change from %a current state satisfying 
$A_{curr}$ to % a future state satisfying 
$A_{fut}$  may happen only if % the necessary condition
$A_{nec}$ holds in some \emph{intermediate} state.
Unlike  {\sc{VerX}} and  \emph{Chainmail} 
 the necessity operators %  $\onlyIf {\_} {\_} {\_}$  and $\onlyThrough {\_} {\_} {\_}$
 are second class, and may nor appear in the assertions $A$.
We formalize (***) % in \Chainspec  
in  \S\ref{s:bank}.


 %\sophiaPonder[not sure we need to have the latter, 
% and do not have temp logic form of it]{}.
% For same later section?
%Unlike \citeauthor{VerX} or \citeauthor{FASE}
%the necessity operators $\onlyIf {\_} {\_} {\_}$  and $\onlyThrough {\_} {\_} {\_}$
%are second class, and may nor appear in the assertions $A$.
%%support necessity specifications with this explicit
%% ``$\onlyIf {A_1} {A_2} {A_3}$'' syntax
%%
%%and concomitant specialised inference system.
%%
%%
%Our assertions $A$ support the usual expressions about program state
%%(e.g. \prg{x.f > 3}),
% logical connectives and quantifiers, 
%%(e.g. $\wedge$, $\forall$), 
%and additional predicates
%to capture \textit{provenance} (whether an object $o$'s definition is
%$\internal{\texttt{o}}$ or $\external{\texttt{o}}$) to the current
%module, and \textit{permission} \cite{miller-esop2013} (whether an
%object $o$ has direct access to another object $o'$:
%$\access{\texttt{o}}{\texttt{o'}}$).
 
\Chainlogic is based on four main ideas: 
First,  the fact that necessity operators are second class, allows us to 
 separate the proof about necessity from the rest.
 Second, assertions may be \emph{module-internal}, \textit{i.e.,}
their validity may only be affected by module-internal calls. 
Third, leveraging the classical  specifications of methods 
(\textit{i.e.,} the sufficient conditions) we can obtain per-method-call
 \NecessitySpecifications; that is, necessary preconditions
  for a given effect and a given method call;  at a very abstract level, this is
similar to  \cite{threoremsFreeSep}.
Fourth,  a special proof system allows us to combine 
per-method
 \NecessitySpecifications and module-internal assertions 
 to obtain per-module   \NecessitySpecifications; these describe a module's
 emergent behaviour.
We outline the proof for the bank example in  \S\ref{s:approach}.
 
%The below is good and perhaps we shpuld use it
% The challenge here is twofold: How do we specify the bad things we are
%concerned about, and how do we prove that the bad things we've
%specified do not happen?  These challenges are difficult because we
%cannot refer to just one component of a software system.  A sufficient
%specification can deal with a single component in isolation --- a
%single function for pre- and postconditions; a single class or data
%structure for invariants. A necessary specification, however, must
%provide guarantees which encompass the software system in its
%entirety, and constrain the emergent behaviour of all its components,
%for an open system, all possible sequences of API invocations.
% 
% 
%
% 
%The importance of distinguishing between sufficient and necessary
%specifications of various kinds has a long history in Computer
%Science.
% 
%  Type systems ensure entire classes of bad things can't
%happen, preserving execution even if memory structures are greatly
%corrupted \cite{Rinard03}.
%%
%More recently, \citeauthor{ddd} and \citeauthor{sandbox} with their robust safety and
%\citeauthor{FASE} with their holistic systems have tackled open world systems to prevent bad things from happening from untrusted code.
%\citeauthor{ddd} use techniques from security to ensure \jm[]{there} isn't undesirable leakage, \citeauthor{sandbox} build a sandbox and have a sophisticated type system to protect it and \citeauthor{FASE} have
%necessary conditions, which they expressed through temporal
%operators.

% In the next section, (\S\ref{s:outline}),  we outline our approach using the
% bank account as  a motivating example.%This example demonstrates, among other things, that the specification need  refer  to  individual methods in a module.
%Moreover, adherence to a Necessity Specification is not monotonic, in the following two senses:
%Adding a method to a module does not necessarily preserve adherence to a Necessity Specification,
%and while separate methods may adhere to a  Necessity Specification, their combination does
%not necessarily do so: Necessity Specifications capture a module's emergent behaviour.


\subsection{Paper Organization and Contributions}

In the next section, (\S\ref{s:outline}),  we outline our approach using the
bank account as  a motivating example. 
%
The contributions of this paper are\begin{enumerate}
 \item
\Chainspec, a language to
express Necessity Specifications (\S\ref{s:semantics}). Even though \Chainspec is
less expressive than  \emph{Chainmail}, we believe that it is expressive enough: we demonstrate this
by encoding all examples from \cite{FASE} in \Chainspec (\S\ref{s:expressiveness}). 
 \item
\Chainlogic, a logic for proving a module's adherence to its
 \NecessitySpecifications (\S\ref{s:inference}), and a proof of soundness of the logic, (\S\ref{s:soundness}),
 mechanized in Coq. 
 \item
A    proof that % the bank account 
  our bank module obeys its \NecessitySpecification, using   \Chainlogic  (\S\ref{s:examples}). This is also mechanized in Coq.
\end{enumerate}
 
 
 
% \begin{enumerate}
% \item
%\Chainmail, a language to
%express Necessity Specifications (\S\ref{s:semantics});
% \item
%%a logic for proving a module's adherence to its
%%necessity specifications (\S\ref{s:inference});
%%% This logic builds on top of classical pre- post- conditions.
%\begin{description}
%\item [(a)]
%{a logic for raising assertion encapsulation and classical Hoare specifications
%to Necessity Specifications for any single module interaction (\S\ref{s:module-proof});}
%\item [(b)]
% {a logic for raising single step Necessity Specifications for a module to an arbitrary number of steps
%and thus proving module adherence to Necessity Specifications for any execution, and any combination of module interactions, i.e. emergent behavior (\S\ref{s:emergent-proof});}
%%% This logic builds on top of classical pre- post- conditions.
%\end{description}
%\item
%  a mechanised {Coq} proof of soundness of the logic (\S\ref{s:soundness});
%\item
%  a  {mechanised Coq} proof that % the bank account 
%  our example obeys the Necessity
%  Specification (\S\ref{s:examples}).
%\end{enumerate}


\noindent Finally, we discuss % place Necessity Specifications into the context of
 related work (\S\ref{s:related}) and conclude (\S\ref{s:conclusion}).
The Coq proofs of 
(2) and (3) above % soundness of \Chainspec, and the adherence of the
% bank module to it \NecessitySpecification 
appear in the
supplementary material.
