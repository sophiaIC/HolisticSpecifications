\renewcommand{\sophia}[2][]{\ponders{Sophia}{blue}{#1} \textcolor{blue}{#2}\xspace}


\section{Introduction}
\label{s:intro}
%Current systems are complex, and  built out of many different components of different provenance and different degree of trustworthiness.

%\begin{flushright}
%  \textit{nice things are nicer than nasty ones}\\
%  Kingsley Amis, ``Lucky Jim.''
%\end{flushright}

\begin{flushright}
  \textit{Condition B is hard to formalize, \\since it requires saying precisely what a bad
plan is,\\ and we do not attempt to do so.}\\
 \cite{byzantine}
\end{flushright}

The days of single, monolithic programs are long gone.  Contemporary
software is built over decades, by combining modules and components of
different provenance and different degrees of trustworthiness, which can
interact with almost every other program, device, or person.
In order for the resulting complex system to be correct, we need to be
able to reason about individual components to ensure that they behave
correctly, i.e.\ that good things happen when our programs run.
For example: if I send an email to a valid address, it will be
delivered to its recipient, or if I provide the right password, I can
transfer money from one of my bank accounts to another. 
To prove that good things can
happen, program verification systems can use witnesses, e.g.\ a
precondition, a postcondition describing the good thing (the desired
effect), and a code snippet, whose execution will establish the
effect, given the precondition.  The critical point here is that the
precondition is a \emph{sufficient} condition for the code snippet to
make the good thing happen: given the precondition, executing a
correct code snippet is guaranteed to achieve the postcondition.

Unfortunately, in a system of any complexity, knowing that good things
will happen is not enough: we also need to be sure that bad things
cannot happen. For example: we  need to be sure that an email can
only be read by the intended recipient; or that someone can only transfer
money if they provide the account's password. To address this problem,
we need to consider the \emph{necessary} conditions without which some
postcondition (good or bad) cannot be achieved:
it is necessary that someone is sent an email before they can read it;
it is necessary that the correct password is provided before money can
be transferred.

%% If a necessary condition can never be achieved, then
%% bad things guarded by that condition cannot happen. \susan [ I have
%% a real problem with the last sentence. We don't have necessary
%% pre-conditions for bad things in the paper - for instance we don't
%% have a necessary condition ``Passwords can be changed by anyone.''
%% which we show cannot happen. I want to just remove the sentence.]{} 

The challenge here is twofold: How do we specify the bad things we are
concerned about, and how do we prove that the bad things we've
specified do not happen?  These challenges are difficult because we
cannot refer to just one component of a software system.  A sufficient
specification can deal with a single component in isolation --- a
single function for pre- and postconditions; a single class or data
structure for invariants. A necessary specification, however, must
provide guarantees which encompass the software system in its
entirety, and constrain the emergent behaviour of all its components,
for an open system, all possible sequences of API invocations.


%% \sophia[Is this paragraph important? Skip?]{} Rather than considering the sufficient conditions to
%% achieve a given effect, we express necessary preconditions for the
%% holistic system. For example, a third party will not get to read my
%% email unless I forward it to them - the forward is a necessary
%% precondition for the reading. To reason about necessary conditions we
%% develop our Logic of Necessity. 

%%%%%%%%%%%%%%%%%%%%%%%%%%%%%%%%%%%%%%%%%%%%%%%%%%%%%%%%%%%%%%%%%%%%%%
%%%%%%%%%%%%%%%%%%%%%%%%%%%%%%%%%%%%%%%%%%%%%%%%%%%%%%%%%%%%%%%%%%%%%%
%% related work bit 

The importance of distinguishing between sufficient and necessary
specifications of various kinds has a long history in Computer
Science. \citeauthor{Lamport77} distinguishes between ``good things
will eventually happen'' (liveness) and ``bad things will never
happen'' (safety) conditions in his seminal paper on ``Multiprocess
Programs''.  Type systems ensure entire classes of bad things can't
happen, preserving executing even if memory structures are greatly
corrupted \cite{Rinard03}.
%
More recently, \citeauthor{ddd} and \citeauthor{sandbox} with their robust safety and 
\citeauthor{FASE} with their holistic systems have tackled open world systems to prevent bad things from happening from untrusted code.
\citeauthor{ddd} use techniques from security to ensure their isn't undesirable leakage, \citeauthor{sandbox} build a sandbox and have a sophisticated type system to protect it and \citeauthor{FASE} have
necessary conditions, which they expressed through temporal
operators.


%%%%%%%%%%%%%%%%%%%%%%%%%%%%%%%%%%%%%%%%%%%%%%%%%%%%%%%%%%%%%%%%%%%%%%
%%%%%%%%%%%%%%%%%%%%%%%%%%%%%%%%%%%%%%%%%%%%%%%%%%%%%%%%%%%%%%%%%%%%%%

In this paper we introduce
%
 %% \textit{necessity specifications}:\\
 %% \strut \hspace{1.5in} $\onlyIf {A_1} {A_2} {A_3}$ \\
%
\textit{necessity specifications}:
%
%
%\hspace{1cm} $\onlyIf {A_1} {A_2} {A_3}$ \\ 
%
%% \begin{lstlisting}[mathescape=true, language=chainmail, frame=lines]
%% $\onlyIf {A_1} {A_2} {A_3}$ 
%% \end{lstlisting}
%
\begin{lstlisting}[mathescape=true, language=chainmail, frame=lines]
from ${A_1}$ to ${A_2}$ onlyIf ${A_3}$ 
\end{lstlisting}
%
to say a change from a current state satisfying $A_1$ to a future
state satisfying $A_2$ (i.e.\ an effect) is possible only if the necessary condition
$A_3$ holds in the current state
%
(i.e.\ $(A_1 \wedge \Diamond A_2) \longrightarrow A_3$).
%
Unlike \citeauthor{VerX} or \citeauthor{FASE}
which employ general temporal operators, 
we support necessity specifications with this explicit
%
``$\onlyIf {A_1} {A_2} {A_3}$'' syntax
%
and concomitant specialised inference system.
%
%
Our assertions $A$ support the usual expressions about program state
(e.g. \prg{x.f > 3}), logical connectives and quantifiers
(e.g. $\wedge$, $\forall$), and additional predicates
to capture \textit{provenance} (whether an object $o$'s definition is
$\internal{\texttt{o}}$ or $\external{\texttt{o}}$) to the current
module, and \textit{permission} \cite{miller-esop2013} (whether an
object $o$ has direct access to another object $o'$:
$\access{\texttt{o}}{\texttt{o'}}$).
 

%% , , and 
%%  % two assertions that are specific to reasoning about open programs: 
%%  the following two assertions specific to reasoning about open programs:\sophia[Have 
%%  we spoken about open earlier on?]{}
%%  \emph{Provenance:}\sophia[used to be called viewpoint, but I think it is provenance]{}
%%  an internal object belongs to the module under consideration ($\internal{\texttt{o}}$ ), all others
%%  belong to   unknown, external, module ($\external{\texttt{o}}$) .  \emph{Access:} 
%%  which object has access to which other object, $\access{\texttt{o}}{\texttt{o'}}$. 
%%  Provenance is relevant, because a
%%  module may make stronger guarantees about its own, internal objects. 
%%  And Access (or permission in \cite{miller-esop2013}'s work \sophia[need citation here]{}) is relevant because in general,
%%  access to objects is necessary to
%%  achieve effects (eg cannot print unless one had access to a printer).
  
 
%\footnote{I think this should appear somewhere later: Chainmail appears 
% to be less rich than Chainmail, however we have been able to specify 
 %all their examples, and in addition, we have a proof system for 
 %\SpecO has a proof system, something lacking 
 %ih Chainmail lacked. }
%To specify the bad things, we cannot
%refer to one particular piece of code, % and cannot refer to one witness;
% instead, we need to make a guarantee which encompasses all possible
% functions executable by a module in all their possible sequences or even interleavings --
 %we need to consider their \emph{emerging} behaviour.
 %Rather than considering the sufficient conditions to achieve a certain effect,
 %we express the necessary precondition. For example,    a third party 
 %will not get to read my email unless I forward it to them -- the forwarding
% is a necessary precondition for the reading.
 
 %To reason about necessary conditions, we develop a special
 %logic with such necessity triples. The most basic such holistic assertions 
 %can be derived from classical assertions (sufficient \jm[old:truples]{triples})
 %under assumptions of encapsulation;
 %they can then be further combined using our logic of necessity.

\subsection{Bank Account Example}
\label{s:bank}
  
Consider \prg{Account} class in module \prg{Mod1}
that represents a bank account
with a balance and a password, where funds may be transferred
between accounts only when sending account's password:
%
% (Note that we assume private fields are accessible ``class-wide''.)
%
% (methods may read and write fields of any instance of a class)
%
%and that passwords are unforgeable and not enumerable (again as
%in Java, albeit without reflection).
%
% 
\begin{lstlisting}[language=Chainmail, frame=lines]
module Mod1
  class Account
    field balance:int 
    field pwd: Password
    method transfer(dest:Account, pwd':Object) -> void
      if this.pwd==pwd'
        this.balance-=100
        dest.balance+=100
  class Password
\end{lstlisting}
%
\noindent 
We can capture the intended
semantics of the \prg{transfer} method by writing ``classical''
specifications in terms of pre- and post-conditions ---
\prg{Mod1}'s implementation of the \prg{transfer} method meets
this specification.



\begin{lstlisting}[mathescape=true, frame=lines, language=Chainmail]
ClassicBankSpec  $\triangleq$
   method transfer(dest:Account, pwd':Password) -> void {
       ( PRE:  this.balance=bal1 $\wedge$ this.pwd==pwd' $\wedge$ dest.balance=bal2 $\wedge$ dest=/=this 
         POST: this.balance == bal1-100 $\wedge$  dest.balance == bal2+100 )
       ( PRE: this.balance=bal1 $\wedge$ this.pwd=/=pwd' $\wedge$ dest.balance=bal2
         POST: this.balance == bal1 $\wedge$  dest.balance=bal2 )
       ( PRE: a : Account $\wedge$ a=/=this $\wedge$ a=/=dest  $\wedge$ a.balance=bal $\wedge$ a.pwd=pwd1
         POST:  a.balance=bal $\wedge$ a.pwd=pwd1)
       ( PRE: a : Account $\wedge$ a.pwd=pwd1  
         POST: a.pwd=pwd1)       
\end{lstlisting}
%\footnote{Perhaps omit some of the lines here, but we do need them all in the full discussion}
 
 

Now consider the following alternative implementations
\prg{Mod2} allows any client to reset its password at any time, while
\prg{Mod3} first requires the existing password in order to change it.
The problem is that although the \prg{transfer} method is the same in
all three alternatives, and although each one satisfies (ClassicSpec),
emergent code sequences such as
%
\prg{account.set(42); account.transfer(yourAccount, 42)}
%
are enough to drain the account \prg{Mod2} without supplying the password.

%% The method \prg{transfer} in all three versions of the class \prg{Account} satisfies the (ClassicSpec), 
%% however, while executing the first and third version of \prg{Account} won't exhibit unwanted behaviour, the second version doesn't preclude it.
%Namely version II allows any client to change the password of the
%account, and then to repeatedly withdraw money from it.

  
% On the other hand, we expect our software -- even if complex -- to provide some simple, high level
%guarantees, e.g. email addressed to me personally will not be read by a third party unless I explicitly 
%forwarded it to them.
%We except  our software to  behave correctly, even when used by a careless or malicious third party. 
%Such use of a software often consist of a sequence of actions performed on the module. 
%
%Software components respond to single actions, 
%or to sequences of such single actions. 
%When thinking about a software component we want think about the behaviour of each 
%action in isolation, but also about the \emph{emergent behaviour}, ie all
% the possible effects of the combinations of these actions. 
  
  
%% \susan[]{I removed the number on the module name because it isn't
%% referred to and doesn't need to be} 
 %% \begin{figure}[hbtp]
\begin{tabular}{lll}
\begin{minipage}[b]{0.35\textwidth}
\begin{lstlisting}[language=chainmail, frame=lines]
module Mod2
  class Account
    field balance:int 
    field pwd: Password 
    method transfer(..) 
      ... as earlier ...
    method set(pwd': Object)
      this.pwd=pwd'
      class Password
\end{lstlisting}
\end{minipage}
&\ \ \  \ \   &%
\begin{minipage}[b]{0.50\textwidth}
\begin{lstlisting}[language=chainmail, frame=lines]
module Mod3
  class Account
    field balance:int 
    field pwd: Password 
    method transfer(..) 
      ... as earlier ...
    method set(pwd',pwd'': Object)
      if (this.pwd==pwd') 
        this.pwd=pwd''
  class Password
\end{lstlisting}
\end{minipage} 
\end{tabular}
 %%  \caption{module \prg{AccountMdl} version II,  \ \ \ \ \ \hspace{.7in} \ \ \ \ \ \  module \prg{AccountMdl} version III}
 %% \label{fig:ExampleAccount}
 %% \end{figure}


We need to rule out \prg{Mod2} while permitting \prg{Mod3} and
\prg{Mod1}. The catch is the leak in \prg{Mod2} is the result
of  \emph{emergent} behaviour from the interactions of the \prg{set}
and \prg{transfer} methods --- even though \prg{Mod3} also has a
\prg{set} method, it does not exhibit the unwanted interaction.
This is exactly where a necessity condition can help:
we want to avoid transferring money
(or more generally, reducing an account's balance)
\textit{without} the account password.  Phrasing the same condition
the other way around gives us a positive statement that still
rules out the theft: that money \textit{can only} be
transferred when the account's password is supplied.
In our necessity condition syntax:
%
%% The flaw in version {\sf{II}} arises from 
%% can be used to overwrite the
%% password, and then using the new password \prg{transfer}  can be called.
% If we want the \prg{Account} class to be robust, we must prohibit the password from being freely available.
%
%
%% In this paper, we show how \textit{necessity specifications} can solve
%% this progkem.  
%%  Therefore, we propose a %holistic 
%%  specification which says that
%%  the \prg{balance} of an \prg{Account} reduces only if an object which does not belong to the
%%  module has access to the password:
%
%
 %% Therefore, we propose a %holistic 
 %% specification which says that
 %% the \prg{balance} of an \prg{Account} reduces only if an object which does not belong to the
 %% module has access to the password:
% 
 \begin{lstlisting}[language = Chainmail, mathescape=true, frame=lines]
NecessityBankSpec  $\triangleq$  from a:Account $\wedge$ a.balance==bal
                      to a.balance < bal
                      onlyIf $\exists$ o.[$\external{\texttt{o}}$ $\wedge$ $\access{\texttt{o}}{\texttt{a.pwd}}$]
\end{lstlisting}
%
%
% 
%% In more detail, the specification from above says that if in the current
%% configuration \prg{a} is an \prg{Account},
%% and in some future configuration \prg{a} will have a balance less than the current one, then,in the \emph{current} configuration
%% there must exist some object \prg{o}, which is \emph{external} to our module (does not belong to module
 %% \prg{AccountMdl}), and which has access to \prg{a}'s password.
 %
%% T9hu having access to the password is a necessary condition for the balance to reduce.
The critical point of this necessity specification is that it is
expressed in terms of observable effects --- reducing the account
balance --- and not in terms of individual methods --- such as
\prg{set} or \prg{transfer}.  This gives necessity specifications the
vital advantage that they constrain not only any
\textit{implementation} of a bank account with a balance and a
password (as do traditional sufficient specifications) but also any
\textit{design} of such a bank account, irrespective of the API it
offers, the services it exports, or the dependencies on other parts of
the system.
   
% QUESTIONS:
% \begin{itemize}
% \item
% holistic assertion -- different name or say we adopt from FASE; holistic -> comprehensive, integrated, aggregate \susan[I think we should keep holistic]{}
% \item
% Name for our logic?  Holistic logic? What is we have a different take on holistic next year? Necessity triples? 
%\item
%If our \SpecO is simpler than the FASE Chainmail, then we need to argue that we can express all FASE examples
% \end{itemize}


% The rest of the paper is organised as follows: .... 
%Section
%~\ref{sect:motivate:Bank} 
%\sd{gives an example from the literature} which we will use 
%to elucidate key points of \SpecO.
%%motivates our work via an example, and then section
%~\ref{sect:chainmail} presents the \SpecO\ specification
%language.  Section~\ref{sect:formal} introduces the formal model
%underlying \SpecO, and then section~\ref{sect:assertions} defines
%the 
%semantics of \SpecO's assertions.
%% SD the below is NOT ture
%%full details are relegated toappendices.   
%Section~\ref{sect:example} shows how key points of 
%exemplar problems can be specified in \SpecO,
%section~\ref{sect:discussion}
%discusses our design, \ref{sect:related} considers related
%work, and section~\ref{sect:conclusion} concludes.
%We relegate various details to appendices.








