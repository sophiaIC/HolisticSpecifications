\section{Necessary Conditions and Robustness}
\label{s:intro}

%\subsection{Necessary conditions and Robustness}

%% Today's   software has been built 
%% over decades by combining modules and components of
%% different provenance and 
%% %different degrees of 
%% trustworthiness, and
%% is \emph{open}, interacting with other programs, devices, and people.

% according to IEEE standard,
% robust = The degree to which a system or component can function correctly in the presence 
% of invalid inputs or stressful environmental conditions.  

{Software needs} to be both {\emph{correct}} ({programs do what they
  are supposed to}) and {\emph{robust}} ({programs only do what
  they're supposed to}). \jm[clarified robustness]{We use robustness to refer
  to \emph{robust safety}, a term 
  used in both the security~\cite{gordonJefferyRobustSafety, Bugliesi:resource-aware} and object capabilities literature~\cite{ddd} to mean that
programs don't do what they aren't supposed to do, even in the
presence of untrusted or malicious {clients}}.
% robust = The degree to which a system or component can function correctly in the presence 
% of invalid inputs or stressful environmental conditions.  
{Correctness is} \jm[]{traditionally} specified
%formally 
through \citeasnoun{Hoare69} triples: a  precondition, a code snippet, and a
 postcondition. 
 For example,  {part of the \funcSpec
   of a \prg{transfer} method for a bank module is that the source account's balance decreases:}
 \begin{quote}
   \Scorrect\ \ $\triangleq$  
 % SD I could not make the below work ...
 % {\scriptsize \lstinline*{pwd=src.pwd  $\wedge$ \lstinline* src.bal=b} src.transfer(dst,pwd) {src.bal=b-100 * $\wedge$ \ldots } } }\\
 {\footnotesize{ $\{\,$\prg{pwd=src.pwd} $\,\wedge\,$ \prg{src.bal=b}$\,\}$ \prg{src.transfer(dst,pwd)} $\{$ \prg{src.bal=b-100}$\,\wedge\,\dots \}$ }} Calling \prg{transfer} on  {an account with the correct password} will transfer the money.
\end{quote}
Assuming termination, the precondition is a \emph{sufficient} condition for the {code snippet to behave correctly}: 
%assuming termination, 
the precondition (\eg providing the right 
password) guarantees that
the code (\eg call the \prg{transfer} function)
will always achieve the postcondition (the money is transferred).
 
    \vspace{.05in}
    % I think we need the vspace
 
%While 
\Scorrect  describes  {the \emph{correct use} of the module, but is \emph{not} concerned with its \emph{robustness}.}
{For example, can I pass an account to foreign untrusted code, in the expectation of receiving a payment,
but without fear that a malicious client might use the account to steal my money \cite{ELang}?}
 A first \jm[]{attempt} to specify robustness could be:
 

\begin{quote}
\SrobustA\ \ $\triangleq$ \ \ An account's balance does not decrease unless \prg{transfer} was called 
with the correct password.
\end{quote}

Specification \SrobustA % gives   the guarantee 
{guarantees} that it is not possible to  take money out of the account through some method other than \prg{transfer}
{or without providing the password}.
  Calling \prg{transfer}   with the  correct password is 
a \emph{necessary condition} for reducing the account's  balance.

\SrobustA is  crucial, but  not   enough:
it does not take  account of the module's \emph{emergent behaviour},
{that is, does not cater for the potential interplay of several methods offered by the module.}
 What if the module provided further methods which leaked the password,  
 {or allowed for it to be arbitrarily changed}?
{ While no single procedure call is capable of breaking the intent of \SrobustA, a sequence of calls might.}
{What} we really need is
 \begin{quote}
\SrobustB\ \ $\triangleq$ \ \ The balance of an account does not {\emph{ever}} decrease in the future unless some external 
object  {\emph{now}} has access to the account's current password.
\end{quote}
With \SrobustB, I can confidently pass my account to \jm[]{any, potentially untrusted context,where my password is not known;} 
%some untrusted client who does not have knowledge of the password; 
\jm[]{the payment I was expecting may or may not be made,}
%they may or may not make the payment I was expecting, 
but I know 
%they will not  {be able to} steal my money 
\jm[]{that my money will not be stolen} \cite{ooToSecurity,miller-esop2013}. 
 Note that \SrobustB  does not mention
 the names of any functions in the module, and 
 thus can be expressed without reference to any particular API ---
 indeed \SrobustB can constrain \emph{any} API with an account, an account
 balance, and a password.

%\vspace{.04in}

% \subsection{{Earlier Work}} % SD I do not think this section is about robustness, since all the paper is about Robustness% Robustness is not a new idea, and anyway, this paper is also about robstness 
 

 %{\sd{Many} 
%kinds of} guarantees have been proposed\sophiaPonder[dropped: ``proposed for  robustness'']{}, differing in the level 
%of granularity,   target  language or calculi, and intended use.  
{Earlier work addressing robustness} includes object capabilities  \cite{MillerPhD, dd, threoremsFreeSep}, 
information control flow \cite{Zdancewic:Myers:01,noninteferenceOS}, 
 correspondence assertions \cite{Maffeis:aiamb:thesis00},
 sandboxing  \cite{robustSafetyPatrignani,sandbox},
robust linear temporal logic   \cite{RLTL2022} -- to name a few. %are some of the proposals to ensure some level of robust safety. 
{Most of these  %works 
propose \emph{generic} robustness guarantees (\emph{e.g.} no dependencies from high values to low values),
while we work with  \emph{problem-specific} guarantees  (\emph{e.g.} no decrease in balance without access to password).}
\jm[is VerX able to express \SrobustB?]{{{\sc{VerX}} \cite{VerX} and  \emph{Chainmail} \cite{FASE} also work on problem-specific guarantees.}
Both these approaches are able to express necessary conditions
  like \SrobustA using
  temporal logic operators and implication,
  and Chainmail is able to express \SrobustB,
  however neither have a proof logic
  able to prove adherence to such specifications.}
%

\jm[]{Developing a specification language with a proof logic that is able to prove properties such as \SrobustB is complex
and must tread a fine line: the language must be rich enough to express complex specifications; temporal operators are needed along with object capability style operators 
that describe \emph{permission} and \emph{provenance}, while also being simple enough that proof rules might be devised.}
\jm[This is the first mention of \Nec apart from the abstract]{In this paper we introduce \Nec,} the first approach that is able to  both express and prove
robustness specifications such as  \SrobustB.


%{{\sc{VerX}} \cite{VerX} and  \emph{Chainmail} \cite{FASE} also work on problem-specific guarantees.}
%Both these approaches can express necessary conditions
%  like \SrobustA using
%  temporal logic operators and implication. For example,  \SrobustA
% could be written as:
%\\
% $\strut ~  ~ \hspace{.35in} \strut  \prg{a:Account} \ \wedge\ \prg{a.balance==bal}  \ \wedge\ 
%\langle {\color{blue}\texttt{next}}\, \prg{a.balance<bal} \, \rangle $\\
% $\strut ~ \hspace{2.6in} \strut \strut \strut \longrightarrow\   \calls{\_}{\prg{a}}{\prg{transfer}}{\prg{\_,a.password}} $
% \\
%\sophiaPonder[expalined and used similar syntax to below, and explained. But  now it get too too long :-(]{That is, if in the next state the balance of \prg{a} decreases, then the current state is 
%a call to the method \prg{transfer} with the correct password passed. 
%Note that  the underscore indicates
%an existentially quantified variable used only once, for example,  $\calls{\_}{\prg{a}}{\prg{transfer}}{\prg{\_,a.password}}$
%is short for $\exists \prg{o}.\exists \prg{a'}. \calls{\prg{o}}{\prg{a}}{\prg{transfer}}{\prg{a',a.password}}$. 
%Here \prg{o} indicates the current receiver, i.e, the object whose method is currently being executed and making the call.
%}
%
% {However, to express \SrobustB, one also needs what we call \emph{capability operators}, which talk about 
% provenance (``external object'') and
%  permission (``\prg{x} has access to \prg{y}''). 
%   {\sc{VerX}}  does not support capability operators, and thus cannot express   \SrobustB, 
%   while  \emph{Chainmail} does support capability operators, and can express  \SrobustB. 
%}  
% {\sc{VerX}} comes with a symbolic 
%  execution system which can demonstrate adherence to its specifications, but doesn't have a proof logic, % to prove adherence,
%   whereas, \emph{Chainmail}   {has neither a symbolic execution system, nor a proof logic.}
%  % \susan[I have separated the tools rather than separating the features as I think it is clearer]{}
%  
% {Temporal operators in {\sc{VerX}}   and  \emph{Chainmail}  are first class, \ie may appear in any assertions 
%and form new assertions. This makes {\sc{VerX}}   and  \emph{Chainmail} very expressive,
%and allows specifications which talk about any number of points in time.
%However, this expressivity comes at the cost of making it very difficult to develop a logic to
%prove adherence to such specifications.}
  
\vspace{.04in}

\subsection{\Nec}
\label{intro:this:work}
\Nec is a language for specifying a module's robustness guarantees 
and a logic 
to prove adherence to such specifications.

\jm[this is a modification that Alex suggested, i.e. being more explicit about
the contribution. I have commented out the old paragraph below.]{
The core contribution of \Nec are three novel operators that merge
temporal operators and implication 
and most importantly are both expressive enough to capture the 
examples we have found in the literature and provable.
}
%
%%For the specification language we adopted  
%%\emph{Chainmail}'s    capability operators.
%%{For the 
%%  temporal operators, we observed that while their
%%   unrestricted combination with  other logical connectives allows us to talk about any
%%   number of points in time, the examples found in the literature talk about two or at most three such points. }
%
%  
%  {This led to the  crucial insight that we could merge  temporal operators and the implication 
% logical connective into our three}
%   \emph{necessity} operators. 
 One such necessity operator is \\
$ 
\strut \hspace{1.7in} \onlyIf {A_{curr}} {A_{fut}} {A_{nec}}
$  
% \begin{lstlisting}[mathescape=true, language=chainmail, frame=lines]
%                                from ${A_{curr}}$ to ${A_{fut}}$ onlyIf ${A_{nec}}$ 
%\end{lstlisting}
%  %      $A$          from ${A_{curr}}$ to ${A_{fut}}$ onlyIf ${A_{nec}}$          from ${A_{curr}}$ to ${A_{fut}}$ onlyThrough ${A_{nec}}$
\\
This form says that  
a  {transition} from a current state satisfying assertion $A_{curr}$ to a future
state satisfying $A_{fut}$  is possible only if  the   necessary 
condition
$A_{nec}$ holds in the \emph{current} state.
Using this operator 
\jm[]{and Chainmail's capability operators that we have adopted into 
\Nec}, 
we can formulate  \SrobustB  
as
\begin{lstlisting}[language = Chainmail, mathescape=true, frame=lines]
   $\text{\SrobustB}$  $\triangleq$   from   a:Account $\wedge$ a.balance==bal    to   a.balance < bal
               onlyIf  $\exists$ o.[$\external{\texttt{o}}$ $\wedge$ $\access{\texttt{o}}{\texttt{a.pwd}}$]
\end{lstlisting}
Namely, a transition from a  {current} state where an account's balance is \prg{bal}, to a  {future} state where 
it has decreased, may \emph{only} occur if  {in the current state} some unknown client object  
has access to that account's password. 
More discussion in \S\ref{s:bankSpecEx}. 


We also support two further \Nec operators:
\\
$ 
\strut \hspace{.4in} \onlyIfSingle {A_{curr}} {A_{fut}} {A_{nec}}. \strut \hspace{.4in} \onlyThrough {A_{curr}} {A_{fut}} {A_{intrm}}
$ 
\\
{The  first says    that 
a  \emph{one-step} {transition} from a current state satisfying assertion $A_{curr}$ to a future
state satisfying $A_{fut}$  
is possible only if % the   necessary condition
$A_{nec}$ holds in the \emph{current} state.   
The   second says that a change from %a current state satisfying 
 $A_{curr}$ to  $A_{fut}$  may happen only if % the necessary condition
 $A_{intrm}$ holds in some \emph{intermediate} state.}
 
  
  \vspace{.02in}
  
Unlike  \emph{Chainmail}'s temporal operators, 
 the necessity operators %  $\onlyIf {\_} {\_} {\_}$  and $\onlyThrough {\_} {\_} {\_}$
 are second class, and may not appear in the assertions  {(\eg  ${A_{curr}}$)}. 
 This simplification enabled us to develop our proof logic. 
 Thus, we {have reached} a  sweet spot between expressiveness and 
 provability.
 
We faced the challenge  how to develop a logic that would enable us to prove that code 
 adhered to  specifications {talking of system-wide properties.} 
 \jm[]{
 The solution arose from three observations: (1) proofs of \Nec specifications
 in the open world must be based around encapsulation and permission,
 (2) \Nec specifications of emergent behaviour can be built up from \Nec specifications on individual 
 functions, which (3) in turn can be encoded as traditional \funcSpecs.}
 
%%The Eureka moment was the realisation that all the information we required was hiding  { in the individual methods' classical specifications:}
%The Eureka moment \jm[four? is 2 an observation?]{arose from four crucial observations:} %{ in the individual methods' classical specifications:}
%\jm[]{\begin{enumerate}
%\item 
%If an assertion is invalidated over an arbitrary number of execution steps, then there must have
%been some single execution step that invalidated it.
%\item
%Proofs of necessary conditions in the open world must be based on restricted permission
%\item
%If all methods within a module share the same necessary precondition to invalidate an ``\emph{encapsulated}'' assertion,
%then that precondition is generally necessary to invalidate that assertion.
%\item
%Necessary preconditions to functions can be encoded using functional specifications.
%\end{enumerate}}
%%\jm[]{These four observations are the basis of \Nec.}
  

\vspace{.02in}
\sophiaPonder[moved to earlier-I think fits better]{ }
\jm[]{A strength of our} work % formalization of \Nec  
is \jm[]{that it is
parametric} with respect to assertion
satisfaction, encapsulation, \jm[]{and functional specifications, all of which are well covered in the literature}.
We require  
\funcSpecs to have explicit framing.
{{Moreover, in line} with other work in the literature,} we forbid 
``callbacks'' out to  { \color{blue}\texttt{external}} objects (ie unknown objects
whose code has not been checked).   While our work is based on 
  a simple, imperative, typed, object oriented
language with unforgeable addresses and private fields, we believe
 that % our approach
 it is applicable to several programming paradigms, and 
 that   unforgeability and privacy
 can be replaced 
 by lower level mechanisms such as capability machines \cite{vanproving,davis2019cheriabi}.
 


\subsection{Paper Organization and Contributions}

%%In the next section, (\S\ref{s:outline}),  we outline our approach using a
%% bank account as  a motivating example.

%\jm[should this be ``a bank account''? This is the first time we mention it]{}
%
The contributions of this paper are:
\begin{enumerate}
\item \susan[]{\jm[]{A language to express \Nec specifications (\S\ref{s:semantics}), including} three novel \Nec operators (\S\ref{s:holistic-guarantees}) that combine implication and temporal operators. }
% \item
%A language to
%express \Nec specifications (\S\ref{s:semantics}).

 \item
A logic for proving a module's adherence to \susan[]{its}
 \Nec specifications (\S\ref{s:inference}), and a proof of soundness of the logic, (\S\ref{s:soundness}),
both mechanised in Coq. 
 \item
A proof in our logic % the bank account 
  that our bank module obeys its \Nec specification (\S\ref{s:examples}),  also  mechanised in Coq.
\item \susan[]{Examples taken from the literature  (\S\ref{s:expressiveness} and \S\ref{s:expressiveness:appendix}) specified in \Nec  .}
\item \susan[]{A richer bank account example (\S\ref{app:BankAccount}) specified in \Nec and mechanised in Coq. }

\end{enumerate}


%\jm[]{
%Our formalization of \Nec does have two limitations. 
%Firstly, it is specifically a logic for 
%necessity and robustness properties. \Nec 
%is parametric with respect assertion satisfaction, correctness,
%encapsulation, and the type system. Much of these are
%well-trod ground in the literature, and where needed we
%do introduce simplistic language mechanisms to deal with them (e.g. a simple type system or notion of encapsulation).
%Secondly, we do not allow for ``callbacks'' 
%to external objects. This is a common restriction in the literature
%as many either prohibit callbacks, or require
%``effectively callback free contracts'' \cite{VerX},
%or place significant restrictions on callbacks \cite{}
%}


%We  discuss these  issues % limitations 
%further as 
\sophiaPonder[chopped "discuss issues"]{ } We place \Nec into the context of 
related work (\S\ref{s:related}) and consider our overall conclusions
(\S\ref{s:conclusion}). 
%
The Coq proofs of 
(2) and (3) above appear in the
supplementary material, along with \sd{a}ppendices containing expanded 
definitions and further examples.
