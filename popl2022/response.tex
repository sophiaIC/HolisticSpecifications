We sincerely thank the reviewers for detailed and thoughtful comments, and for the opportunity this gives us to explain our work better

***General Comments***

**G1: external calls**

Necessity does not --yet-- support calls of external methods from within internal modules. This is, indeed, a limitation, but it is not uncommon in the related literature. For example, VerX [Permenev et al] work on effectively call-back free contracts, while and [Grossman 2018] and  [Albert 2020] on drastically restricting the effect of a callback on a contract. Therefore, we argue that a treatment of  external calls in Necessity would bring some further complexity, and would  detract from the main focus of our paper.  

We cannot promise a full treatment of external calls by end February. But we can share out current thinking:  As a first approach, we will require that the arguments to external calls do not include internal objects, except for the receiver and parameters (thus ensuring that external accessibility of internal methods does not increase); we would rely on the classical pre- and post- conditions of the internal methods -- as we currently do. As a more advanced approach, we will develop extensions to classical Hoare Logics, which would allow us to reason about points in the code where external calls are being made. This would be the first time we could be inspecting the code in the bodies of the functions.

[Grossman 2018] Shelly Grossman, Ittai Abraham, Guy Golan-Gueta, Yan Michalevsky, Noam Rinetzky, Mooly Sagiv, and Yoni Zohar. Online detection of effectively callback free objects with applications to smart contracts. In Symposium on Principles of Programming Languages, POPL. ACM, 2018.

[Albert 2020] Elvira Albert, Shelly Grossman, Noam Rinetzky, Clara Rodríguez-Núñez, Albert Rubio, and Mooly Sagiv. 2020. Taming Callbacks for Smart Contract Modularity. Proc. ACM Program. Lang. 4, OOPSLA, Article 209, 2020


**G2:  overlap with Chainmail**

 It is true that some of the Necessity definitions, and their encodings, are inspired by the work of FASE2020, 
and as such there may be some similarities between the two formalisms, but these similarities do not extend to the contributions of our paper: the Necessity language and proof system, 
the soundness result, and the example of the proof system. Where there are similarities, these either have been explicitly stated in the paper (eg. permission, provenance, and control),
or are common coq idioms (eg. variable substitution).
SD: We both use the Chippala's library we do not share further code. 


** G3: The Definition of $\triangle$ -- adaptation **

Adaptation is indeed an important definition to the paper, but is not a central contribution. We thank the reviewers for picking up on this, as upon reflection we recognize that the precise meaning of adaptation was not clear in the text, and potentially detracts from a critical definition, Def. 3.10. We propose the following explanation:

``Adaptation is a variable renaming operator that ensures that variable names used at one point in the execution refer to the same object at a future point in the execution. This is necessary as, within assertions, objects are referenced using variable names, and not unique and immutable identifiers. This is in keeping with how many formal specification languages refer to objects (cite???? DO YOU KNOW SUCH A PLACE JULIAN ???). This presents a problem when the program to be executed is not necessarily known at verification time. During program execution variables may be either overwritten or lost from scope, and as such using the same variable name at different points in execution has little meaning unless there is a way to rewrite variables so that their intended meaning can be preserved. Adapting one program state with a second allows variables in the second state to point to the same object in the first.''


** G4: Advantage over VerX/Chainmail (rev4) **

** G5: Is access deep? **

Access is not deep, and only refers to objects that an object has direct access to via a field or within the context of the current scope. This does not pose a question of soundness for
Necessity, as Necessity is only concerned with the necessary preconditions required for program state to change, and in the case of access what are the necessary preconditions to 
gaining access. The underlying language upon which Necessity is presented restricts field reads and writes to objects of the containing class. This means that field reads (and writes) 
are necessarily encapsulated by the internal module, that is, internal module code is required to read internal module fields. Fundamentally this is a question of encapsulation and the 
semantics of the language, and not of Necessity. If the semantics of TooL where such that objects were generally able to gain access to the fields of their fields, then access to internally 
enclosed objects would not be said to be encapsulated, and If1-Inside would not be applicable when constructing a proof around such access.

 

*** Proposed Changes ***

We thank the reviewers for the proposed improvements, which we plan implement

C0: external

C1: Move the clarifying examples to section 2, and explain better
ie write the explanations

C2: Rename Mod1, Mod2, Mod3 to Mod_{basic}, ....

C3: move the definition of "inside" to ...

C4: make consistent use of Section/Section

C5: encapsulation -- better explanation

Assertions are not "encapsulated" by some arbitrary code, but rather by the internal module, i.e. known code. Assertion encapsulation captures a property that is essential to proofs of safety in the open world: certain operations may only occur within the "internal" module/code, thus, the satisfaction of properties that depend on such operations may only change as a result of internal module code. The simplest such operation in a Java-like language would be the mutation of a field of an object of an internal class. Satisfaction of assertions about the value of such a field may only change as a result of internal code being executed.
 
C6: adaptation -- better explanation

We propose to adopt the explanations given in G

C7: change the bank account example to be the same?
 
 We chose the more complex form of the bank account example in order to show case some of the features of Necessity, specifically:
1) The ability to reason about ghost fields, and not just simply values within fields
2) More complex data structures which allows for more potentially complex forms of emergent behavior

The reviewers make a good point that this example is potentially too complex for an initial proof example, as such we will 
replace this example with the simpler version, and mechanise that proof using Coq by 11 February. We will retain the existing proof as part 
of the appendix for the purposes of exhibiting the more sophisticated aspects of Necessity.
 
C8: make the section 2.4 top-down?

C9: emergent behaviour 

 The reviewer correctly identifies that we did not sufficiently indicate what we meant by ``emergent behavior''. We propose the follow change 
to Section 1, where ``emergent behavior'' is first mentioned:

``(S2) does not take accout of the module's \emph{emergent behavior}. That is, (S2) does not consider the behavior that emerges from the interaction between the 
\texttt{transfer} method, and the other methods of the bank module. What if the module leaks the password?''

 
C10: and we will implement
 


*** Detailed Response -- per reviewer ***

Reviewer A:

Reviewer B:

Reviewer C:

Reviewer D:

Bibliography



QUESTIONS for us