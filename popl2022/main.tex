%% For double-blind review submission, w/o CCS and ACM Reference (max submission space)
\documentclass[acmsmall,review,anonymous]{acmart}\settopmatter{printfolios=true,printccs=false,printacmref=false}
%% For double-blind review submission, w/ CCS and ACM Reference
%\documentclass[acmsmall,review,anonymous]{acmart}\settopmatter{printfolios=true}
%% For single-blind review submission, w/o CCS and ACM Reference (max submission space)
%\documentclass[acmsmall,review]{acmart}\settopmatter{printfolios=true,printccs=false,printacmref=false}
%% For single-blind review submission, w/ CCS and ACM Reference
%\documentclass[acmsmall,review]{acmart}\settopmatter{printfolios=true}
%% For final camera-ready submission, w/ required CCS and ACM Reference
%\documentclass[acmsmall]{acmart}\settopmatter{}


%% Journal information
%% Supplied to authors by publisher for camera-ready submission;
%% use defaults for review submission.
\acmJournal{PACMPL}
\acmVolume{1}
\acmNumber{CONF} % CONF = POPL or ICFP or OOPSLA
\acmArticle{1}
\acmYear{2022}
\acmMonth{1}
\acmDOI{} % \acmDOI{10.1145/nnnnnnn.nnnnnnn}
\startPage{1}

%% Copyright information
%% Supplied to authors (based on authors' rights management selection;
%% see authors.acm.org) by publisher for camera-ready submission;
%% use 'none' for review submission.
\setcopyright{none}
%\setcopyright{acmcopyright}
%\setcopyright{acmlicensed}
%\setcopyright{rightsretained}
%\copyrightyear{2018}           %% If different from \acmYear

%% Bibliography style
\bibliographystyle{ACM-Reference-Format}
%% Citation style
%% Note: author/year citations are required for papers published as an
%% issue of PACMPL.
\citestyle{acmauthoryear}   %% For author/year citations


%%%%%%%%%%%%%%%%%%%%%%%%%%%%%%%%%%%%%%%%%%%%%%%%%%%%%%%%%%%%%%%%%%%%%%
%% Note: Authors migrating a paper from PACMPL format to traditional
%% SIGPLAN proceedings format must update the '\documentclass' and
%% topmatter commands above; see 'acmart-sigplanproc-template.tex'.
%%%%%%%%%%%%%%%%%%%%%%%%%%%%%%%%%%%%%%%%%%%%%%%%%%%%%%%%%%%%%%%%%%%%%%


%% Some recommended packages.
\usepackage{booktabs}   %% For formal tables:
                        %% http://ctan.org/pkg/booktabs
\usepackage{subcaption} %% For complex figures with subfigures/subcaptions
                        %% http://ctan.org/pkg/subcaption
                        

\usepackage{relsize}
\usepackage{mathpartir}
\usepackage{amsmath}
\usepackage{amsthm}
\usepackage{listings}


%constrained reduction
\newcommand{\constrained}{\mathrel{\leadsto\ \!\!\!\!{\raisebox{1pt}{$\mathsmaller{\mathsmaller{\mathsmaller{\mathsmaller\rvert}}}$}}}}
\newcommand{\reduction}[4]{#1\ \fcmp\ #2\ \bullet\ #3\ \leadsto\ #4}
\newcommand{\reductions}[4]{#1\ \fcmp\ #2\ \bullet\ #3\ \leadsto^*\ #4}
\newcommand{\constrainedReduction}[4]{#1\ \fcmp\ #2\ \bullet\ #3\ \constrained\ #4}
\newcommand{\constrainedReductions}[4]{#1\ \fcmp\ #2\ \bullet\ #3\ \constrained^*\ #4}
\newcommand{\satisfies}[4]{#1\ \fcmp\ #2,\ #3 \vDash\ #4}

\newcommand\trans{\mathlarger{\mathlarger \leadsto}}
\newcommand\intstep{\hspace{1.5mm}{\raisebox{3pt}{$\bullet$}}\hspace{-1.5mm}{\hookrightarrow}}
\newcommand\en{\hspace{1.5mm}{\raisebox{0pt}{$\bullet$}}\hspace{-4mm}{\hookrightarrow}}
\newcommand\oi{\hspace{1mm}{\raisebox{1pt}{$\bullet$}}\hspace{-1mm}{\trans}}
\newcommand\ot{\hspace{2mm}{\raisebox{1pt}{$\bullet$}}\hspace{-3mm}{\trans}}
\newcommand\otAlt{\hspace{2mm}{\raisebox{0.5pt}{$\bullet$}}\hspace{-2.75mm}{\trans}}
\newcommand\mut[3]{\langle #1\ \texttt{mut}\ #2.#3 \rangle}
\newcommand\gives[3]{\langle #1\ \texttt{gives}\ #2\ \texttt{to}\ #3 \rangle}
\newcommand\exposes[2]{#1\ \texttt{exposes}\ #2}
\newcommand\univ{U}
\newcommand\onlyIf[3]{#1\ {\color{blue}\texttt{to}}\ #2\ {\color{blue}\texttt{only if}}\ #3}
\newcommand\oiInternal[3]{#1\ {\color{blue}\texttt{to}}\ #2\ \intstep\ #3}
\newcommand\ensures[3]{#1,\ #2\ \en\ #3}
\newcommand\onlyThrough[3]{#1\ {\color{blue}\texttt{to}}\ #2\ {\color{blue}\texttt{only though}}\ #3}
\newcommand\onlyIfProof[4]{#1\ \vdash\ #2,\ #3\ \texttt{only if}\ #4}
\newcommand\onlyThroughProof[4]{#1\ \vdash\ #2,\ #3\ \texttt{only if}\ #4}
\newcommand\hoare[3]{\{#1\}\ #2\ \{#3\}}
\newcommand\hoareIf[4]{#1,\ #2,\ \{#3\}\ \intstep\ #4}
\newcommand\rtrns[3]{\{#1\}\ #2\ \texttt{returns}\ #3}

\newcommand\encapsulated[1]{\langle \texttt{encapsulated}\ #1 \rangle}
\newcommand\encapsulates[2]{\langle #1\ \texttt{encapsulates}\ #2 \rangle}
\newcommand\encapsulatesStrong[2]{\langle #1\ \texttt{encapsulates}_\texttt{strong}\ #2 \rangle}
\newcommand\encapsulatesMdl[1]{\langle \texttt{encapsulating}_\texttt{int}\ #1 \rangle}
\newcommand\calls[4]{\langle #1\ \texttt{calls}\ #2.#3(#4) \rangle}
\newcommand\changes[2]{\langle #1\ \texttt{changes}\ #2 \rangle}
\newcommand\access[2]{\langle #1\ \texttt{access}\ #2 \rangle}
\newcommand\internal[1]{\langle #1\ \texttt{internal}\rangle}
\newcommand\external[1]{\langle #1\ \texttt{external}\rangle}
\newcommand\comprehension[2]{\{#1 | #2\}}
\newcommand\internalStep{\langle \texttt{internal step}\rangle}

\lstset{ % General setup for the package
	language=Java,
	basicstyle=\scriptsize\sffamily,
	numbers=left,
 	numberstyle=\tiny,
	frame=tb,
	tabsize=4,
	columns=fixed,
	showstringspaces=false,
	showtabs=false,
	keepspaces,
	morekeywords={field, method, module, calls, presumes, achieves, external, internal, access, requires, ensures, PRE, POST, only, through, if, to},
	commentstyle=\color{red},
	keywordstyle=\color{blue},
	deletekeywords={if}
}


\begin{document}

%% Title information
\title[An Inference System for Holisic Specifications]{An Inference System for Holisic Specifications}         %% [Short Title] is optional;
                                        %% when present, will be used in
                                        %% header instead of Full Title.
%\titlenote{with title note}             %% \titlenote is optional;
                                        %% can be repeated if necessary;
                                        %% contents suppressed with 'anonymous'
%\subtitle{Subtitle}                     %% \subtitle is optional
%\subtitlenote{with subtitle note}       %% \subtitlenote is optional;
                                        %% can be repeated if necessary;
                                        %% contents suppressed with 'anonymous'


%% Author information
%% Contents and number of authors suppressed with 'anonymous'.
%% Each author should be introduced by \author, followed by
%% \authornote (optional), \orcid (optional), \affiliation, and
%% \email.
%% An author may have multiple affiliations and/or emails; repeat the
%% appropriate command.
%% Many elements are not rendered, but should be provided for metadata
%% extraction tools.

%% Author with single affiliation.
\author{First1 Last1}
\authornote{with author1 note}          %% \authornote is optional;
                                        %% can be repeated if necessary
\orcid{nnnn-nnnn-nnnn-nnnn}             %% \orcid is optional
\affiliation{
  \position{Position1}
  \department{Department1}              %% \department is recommended
  \institution{Institution1}            %% \institution is required
  \streetaddress{Street1 Address1}
  \city{City1}
  \state{State1}
  \postcode{Post-Code1}
  \country{Country1}                    %% \country is recommended
}
\email{first1.last1@inst1.edu}          %% \email is recommended

%% Author with two affiliations and emails.
\author{First2 Last2}
\authornote{with author2 note}          %% \authornote is optional;
                                        %% can be repeated if necessary
\orcid{nnnn-nnnn-nnnn-nnnn}             %% \orcid is optional
\affiliation{
  \position{Position2a}
  \department{Department2a}             %% \department is recommended
  \institution{Institution2a}           %% \institution is required
  \streetaddress{Street2a Address2a}
  \city{City2a}
  \state{State2a}
  \postcode{Post-Code2a}
  \country{Country2a}                   %% \country is recommended
}
\email{first2.last2@inst2a.com}         %% \email is recommended
\affiliation{
  \position{Position2b}
  \department{Department2b}             %% \department is recommended
  \institution{Institution2b}           %% \institution is required
  \streetaddress{Street3b Address2b}
  \city{City2b}
  \state{State2b}
  \postcode{Post-Code2b}
  \country{Country2b}                   %% \country is recommended
}
\email{first2.last2@inst2b.org}         %% \email is recommended


%% Abstract
%% Note: \begin{abstract}...\end{abstract} environment must come
%% before \maketitle command
\begin{abstract}
Text of abstract \ldots.
\end{abstract}


%% 2012 ACM Computing Classification System (CSS) concepts
%% Generate at 'http://dl.acm.org/ccs/ccs.cfm'.
\begin{CCSXML}
<ccs2012>
<concept>
<concept_id>10011007.10011006.10011008</concept_id>
<concept_desc>Software and its engineering~General programming languages</concept_desc>
<concept_significance>500</concept_significance>
</concept>
<concept>
<concept_id>10003456.10003457.10003521.10003525</concept_id>
<concept_desc>Social and professional topics~History of programming languages</concept_desc>
<concept_significance>300</concept_significance>
</concept>
</ccs2012>
\end{CCSXML}

\ccsdesc[500]{Software and its engineering~General programming languages}
\ccsdesc[300]{Social and professional topics~History of programming languages}
%% End of generated code


%% Keywords
%% comma separated list
\keywords{keyword1, keyword2, keyword3}  %% \keywords are mandatory in final camera-ready submission


%% \maketitle
%% Note: \maketitle command must come after title commands, author
%% commands, abstract environment, Computing Classification System
%% environment and commands, and keywords command.
\maketitle


\section{Introduction}

\section{Robust Specifications}

\subsection{A Motivating Banking Example}
To illustrate robust specifications, we introduce an example 
of a bank account:
\begin{lstlisting}[mathescape=true]
module Bank
	class Account
		field pass, bal
	
		method transfer(pwd, from, amt)
			if (pwd == pass)
				from.bal := from.bal - amt
				this.bal := this.bal + amt

		method changePassword(pwd, newPwd)
			if pwd == this.pass
				this.pass := newPwd
\end{lstlisting}
Traditionally we are able to write the following specifications
\begin{lstlisting}[mathescape=true]
(TraditionalSpec) $\triangleq$
	method transfer(from, amt, pwd)
		PRE: pwd == pass
		POST: from.balance == from.bal$_{\textit{pre}}$ - amt && this.bal == this.bal$_{\textit{pre}}$ + amt
	method transfer(from, amt, pwd)
		PRE: pwd != pass
		POST: from.bal == from.bal$_{\textit{pre}}$ && this.bal == this.bal$_{\textit{pre}}$

	method changePassword(pwd, newPwd)
		PRE: pwd == pass
		POST: this.pass == pwd
	method changePassword(pwd, newPwd)
		PRE: pwd != pass
		POST: this.pass == this.pass$_{\textit{pre}}$
\end{lstlisting}
(Traditional Spec) states the expected behavior of each of the composite methods of the 
Account class. Firstly, if the \texttt{transfer} method is called with the correct password,
then the balances of both accounts will be correctly modified, and if the password is incorrect, 
then the balances will remain unchanged. Secondly, the password may only be changed if the 
existing password is known. 

Unfortunately, there is an bug in the \texttt{transfer} method, 
i.e. the password of the \texttt{from} account is not required. Any user may transfer any amount 
from any account to their account, and all they need is their password. This violates our notion 
of what a robust Account is. Ideally we would like the balance of our account to only decrease 
if we authorize it.
Part of the problem is that 
it is difficult to specify exactly what is expected behavior at the method level. Ideally 
we would like to be able to specify the behavior of the entire module. We can attempt to 
fix the bug in the \texttt{transfer} method:
\begin{lstlisting}[mathescape=true]
method transfer(pwd, to, amt)
	if (pwd == pass && amt >= 0 && this.balance >= amt)
		this.bal := this.bal - amt
		to.bal := to.bal + amt
\end{lstlisting}
Now \texttt{transfer} only decreases an account's balance if the password is known, and
we could update (TraditionalSpec) to reflect this. There is still a potential problem 
with our specification: we do not necessarily know if all other methods observe this
property. This uncertainty is even more problematic when we consider that \texttt{Account}'s
interface may be extended. We would not only like all existing methods to require 
the correct password to transfer money from our account, but also any future extensions
to \texttt{Account} to observe this property too.

\citeauthor{FASE} introduced \emph{holistic specifications}; i.e. specifications of
not just what is \emph{sufficient} to achieve an outcome (eg. \texttt{transfer} may 
decrease our \texttt{Account}'s balance), but what is \emph{necessary} (eg. \emph{only} 
\texttt{transfer} may decrease the balance of our account). Informally we might say 
that we want the \texttt{Account} interface to ensure the following behavior
\begin{enumerate}
\item
\label{prop:transfer}
The \emph{only} way to reduce the balance of account \texttt{a} is through a method call
to \texttt{transfer} on \texttt{a} with the correct password.
\item
\label{prop:noLeaking}
The balance of account \texttt{a} may be reduced \emph{only if} the password is presently
known.
\end{enumerate}
Property \ref{prop:transfer} and \ref{prop:noLeaking} above say that no matter the 
future modifications to \texttt{Account} nor the use by unknown parties, an account 
can not be coerced to modify it's balance improperly or leak it's secret. While 
\citeauthor{FASE} defined a model for such specifications, they did not design an
inference system for proof construction. In this paper we build on the work of 
\citeauthor{FASE}, presenting a more streamlined syntax and semantics for Chainmail, 
along with an inference system for constructing proofs of satisfaction. 

We define two triples that allow us to
capture the robustness properties \ref{prop:transfer} and \ref{prop:noLeaking}: \texttt{only if}
and \texttt{only through}.
\begin{lstlisting}[mathescape=true]
(HolisticSpec) $\triangleq$
	a : Account && a.bal == b to a.bal < b 
		only through $\exists$ o.[o calls a.transfer(a.pass, _, _)]
	a : Account to <o access a.pass> && <o external> 
		only if $\exists$ o'.[<o' access a.pass> && <o' external>]
\end{lstlisting}
(HolisticSpec) states that any execution path from a state where the balance of account \texttt{a}
is \texttt{b} to some state where the balance of \texttt{a} is less than \texttt{b} 
must pass \emph{through} some intermediate state where an object external to the \texttt{Bank} module 
called \texttt{transfer} on \texttt{a} with the correct password. Further, any execution
path from a state with some \texttt{Account} \texttt{a}, to another state where some object external 
to \texttt{Bank} has access to the password of \texttt{a} necessitates that some object external to 
\texttt{Bank} must know the password in the initial state.


\subsection{Only If/Only Through}
To capture necessary specifications, introduce two triples: \emph{Only If} and \emph{Only Through}.
Here we introduce the formal semantics of these triples.
\begin{definition}[Only If]
$M\ \vDash\ \onlyIf{A_1}{A_2}{A}$\ iff 
$\forall M,\ \sigma_1,\ \sigma_2$ such that 
$M; M', \sigma_1\ \constrained^*\ \sigma_2$
$M; M', \sigma_1\ \vDash\ A_1$ and 
$M; M', \sigma_2\ \vDash\ A_2$ then
$M; M', \sigma_1\ \vDash\ A$.
\end{definition}
\begin{definition}[Only Through]
$M\ \vDash\ \onlyThrough{A_1}{A_2}{A}$\ iff 
$\forall M,\ \sigma_1,\ \sigma_2$ such that 
$M; M', \sigma_1\ \constrained^*\ \sigma_2$, 
$M; M', \sigma_1\ \vDash\ A_1$, and 
$M; M', \sigma_2\ \vDash\ A_2$ then
$\exists\ \sigma$ such that 
\begin{itemize}
\item
$M; M,\ \sigma_1\ \constrained^*\ \sigma$ or $\sigma = \sigma_1$ and
\item
$M; M,\ \sigma\ \leadsto^*\ \sigma_2$ and 
\item
$M; M', \sigma\ \vDash\ A$.
\end{itemize}
\end{definition}
Informaly we have 
\begin{itemize}
\item
$M\ \vDash\ \onlyIf{A_1}{A_2}{A}$ requires that execution starting at $A_1$ may reach $A_2$ \emph{only if} $A$ was true at the start
\item
$M\ \vDash\ \onlyThrough{A_1}{A_2}{A}$ requires that execution starting at $A_1$ may reach $A_2$ \emph{only through} some intermediate state $A$
\end{itemize}

\section{Chainmail}

The Chainmail defined by \citeauthor{FASE} incorporated assertion forms for 
specifying several different aspects of a program: \emph{viewpoint}, \emph{space}, 
\emph{control}, \emph{permission}, and \emph{time}. In this work we use express
temporal properties using the triples \emph{only if} and \emph{only through}, 
and as a result Chainmail assertions are largely non-temporal 
(except in a single restricted case), and do not specify the program state
of the past, or the future.

The syntax of Chainmail assertions is given in Fig. \ref{f:chainmail_assertions}.
Assertions may be 

\begin{figure}[h]
\[
\begin{array}{lr}
A\;\;\;\; ::= & \textit{Assertions}\\  
|\;\;\;\; e                        \;\;\;\;  
|\;\;\;\; e\ :\ C                  \;\;\;\;
|\;\;\;\; e\ \in\ S                \;\;\;\;
|\;\;\;\; \neg A 					\;\;\;\;
|\;\;\;\; A\ \wedge\ A 			\;\;\;\;
|\;\;\;\; A\ \vee\ A \\
|\;\;\;\; A\ \longrightarrow\ A 	\;\;\;\;
|\;\;\;\; \forall\ x.\ [A] 		\;\;\;\;
|\;\;\;\; \exists\ x.\ [A] 		\;\;\;\;
|\;\;\;\; \forall\ S.\ [A] 		\;\;\;\;
|\;\;\;\; \exists\ S.\ [A] 		\\
|\;\;\;\; A\ \texttt{in}\ S        \;\;\;\; 
|\;\;\;\; \access{x}{y}            \;\;\;\;
|\;\;\;\; \internal{x}             \;\;\;\; 
|\;\;\;\; \external{x}             \\
|\;\;\;\; \calls{x}{y}{m}{args} 	\;\;\;\;
|\;\;\;\; \changes{S}{A} 			
\end{array}
\]
\caption{Chainmail Syntax}
\label{f:chainmail_assertions}
\end{figure}

\section{Inference System}

\subsection{Only If/Only Through}
\begin{figure}[t]
\begin{mathpar}
\infer
	{}
	{M\ \vdash\ \onlyThrough{A_1}{A_2}{\texttt{true}}}
	\quad(\textsc{True})
	\and
\infer
	{}
	{
	M\ \vdash\ \onlyThrough{A}{\neg A}{\exists S.[\changes{S}{A}]}
	}
	\quad(\textsc{Changes})
	\and
\infer
	{
	M\ \vdash\ A_1 \longrightarrow A_1'\\
	M\ \vdash\ A_2 \longrightarrow A_2'\\
	M\ \vdash\ A_3' \longrightarrow A_3\\
	M\ \vdash\ \onlyThrough{A_1'}{A_2'}{A_3'}
	}
	{M\ \vdash\ \onlyThrough{A_1}{A_2}{A_3}}
	\quad(\textsc{$\longrightarrow$})
	\and
\infer
	{
	M\ \vdash\ \onlyThrough{A_1}{A_2}{A} \\
	M\ \vdash\ \onlyThrough{A_1'}{A_2}{A'}
	}
	{M\ \vdash\ \onlyThrough{A_1\ \vee\ A_1'}{A_2}{A\ \vee\ A'}}
	\quad(\textsc{$\vee$I$_1$})
	\and
\infer
	{
	M\ \vdash\ \onlyThrough{A_1}{A_2}{A} \\
	M\ \vdash\ \onlyThrough{A_1}{A_2'}{A'}
	}
	{M\ \vdash\ \onlyThrough{A_1}{A_2\ \vee\ A_2'}{A\ \vee\ A'}}
	\quad(\textsc{$\vee$I$_2$})
	\and
\infer
	{
	M\ \vdash\ \onlyThrough{A_1}{A'}{\texttt{false}} \\
	M\ \vdash\ \onlyThrough{A_1}{A_2}{A\ \vee\ A'}
	}
	{M\ \vdash\ \onlyThrough{A_1}{A_2}{A}}
	\quad(\textsc{$\vee$E})
	\and
\infer
	{
	M\ \vdash\ \onlyThrough{A_1}{A_2}{A_3} \\
	M\ \vdash\ \onlyThrough{A_1}{A_3}{A}
	}
	{M\ \vdash\ \onlyThrough{A_1}{A_2}{A}}
	\quad(\textsc{Trans$_1$})
	\and
\infer
	{
	M\ \vdash\ \onlyThrough{A_1}{A_2}{A_3} \\
	M\ \vdash\ \onlyThrough{A_3}{A_2}{A}
	}
	{M\ \vdash\ \onlyThrough{A_1}{A_2}{A}}
	\quad(\textsc{Trans$_2$})
	\and
\infer
	{
	M\ \vdash\ \onlyIf{A_1}{A_2}{A}
	}
	{M\ \vdash\ \onlyThrough{A_1}{A_2}{A}}
	\quad(\textsc{If})
\end{mathpar}
\caption{\emph{Only Through}}
\label{f:only_through}
\end{figure}

\begin{figure}[p]
\begin{mathpar}
\infer
	{}
	{M\ \vdash\ \onlyIf{A_1}{A_2}{\texttt{true}}}
	\quad(\textsc{If-True})
	\and
\infer
	{
	M\ \vdash\ A_1 \longrightarrow A_1'\\
	M\ \vdash\ A_2 \longrightarrow A_2'\\
	M\ \vdash\ A_3' \longrightarrow A_3\\
	M\ \vdash\ \onlyIf{A_1'}{A_2'}{A_3'}
	}
	{M\ \vdash\ \onlyIf{A_1}{A_2}{A_3}}
	\quad(\textsc{If-$\longrightarrow$})
	\and
\infer
	{
	M\ \vdash\ \onlyIf{A_1}{A_2}{A} \\
	M\ \vdash\ \onlyIf{A_1'}{A_2}{A'}
	}
	{M\ \vdash\ \onlyIf{A_1\ \vee\ A_	1'}{A_2}{A\ \vee\ A'}}
	\quad(\textsc{If-$\vee$I$_1$})
	\and
\infer
	{
	M\ \vdash\ \onlyIf{A_1}{A_2}{A} \\
	M\ \vdash\ \onlyIf{A_1}{A_2'}{A'}
	}
	{M\ \vdash\ \onlyIf{A_1}{A_2\ \vee\ A_2'}{A\ \vee\ A'}}
	\quad(\textsc{If-$\vee$I$_2$})
	\and
\infer
	{
	M\ \vdash\ \onlyIf{A_1}{A_2}{A\ \vee\ A'} \\
	M\ \vdash\ \onlyThrough{A'}{A_2}{\texttt{false}}
	}
	{M\ \vdash\ \onlyIf{A_1}{A_2}{A}}
	\quad(\textsc{If-$\vee$E$$})
	\and
\infer
	{}
	{M\ \vdash\ \onlyIf{A_1}{A_2}{A_1}}
	\quad(\textsc{If-Start})
	\and
\infer
	{
	M\ \vdash\ \onlyThrough{A_1}{A_2}{A_3} \\
	M\ \vdash\ \onlyIf{A_1}{A_3}{A}
	}
	{M\ \vdash\ \onlyIf{A_1}{A_2}{A}}
	\quad(\textsc{If-Trans)}
	\and
\infer
	{
	M\ \vdash\ \onlyThrough{A_1}{A_2}{A_2}
	}
	{M\ \vdash\ \onlyIf{A_1}{A_2}{A_2}}
	\quad(\textsc{If-Ind)}
\end{mathpar}
\caption{\emph{Only if}}
\label{f:only_if}
\end{figure}

\subsection{Internal/External Boundary}

\section{Examples}

\subsection{Bank Account}

\subsection{Safe}

\subsection{DOM}

\subsection{DAO}

\section{Soundness}




%% Acknowledgments
\begin{acks}                            %% acks environment is optional
                                        %% contents suppressed with 'anonymous'
  %% Commands \grantsponsor{<sponsorID>}{<name>}{<url>} and
  %% \grantnum[<url>]{<sponsorID>}{<number>} should be used to
  %% acknowledge financial support and will be used by metadata
  %% extraction tools.
  This material is based upon work supported by the
  \grantsponsor{GS100000001}{National Science
    Foundation}{http://dx.doi.org/10.13039/100000001} under Grant
  No.~\grantnum{GS100000001}{nnnnnnn} and Grant
  No.~\grantnum{GS100000001}{mmmmmmm}.  Any opinions, findings, and
  conclusions or recommendations expressed in this material are those
  of the author and do not necessarily reflect the views of the
  National Science Foundation.
\end{acks}


%% Bibliography
\bibliography{Case}


%% Appendix
\appendix
\section{Appendix}

Text of appendix \ldots

\end{document}
