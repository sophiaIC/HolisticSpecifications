%% For double-blind review submission, w/o CCS and ACM Reference (max submission space)
\documentclass[acmsmall,review,anonymous]{acmart}\settopmatter{printfolios=true,printccs=false,printacmref=false}
%% For double-blind review submission, w/ CCS and ACM Reference
%\documentclass[acmsmall,review,anonymous]{acmart}\settopmatter{printfolios=true}
%% For single-blind review submission, w/o CCS and ACM Reference (max submission space)
%\documentclass[acmsmall,review]{acmart}\settopmatter{printfolios=true,printccs=false,printacmref=false}
%% For single-blind review submission, w/ CCS and ACM Reference
%\documentclass[acmsmall,review]{acmart}\settopmatter{printfolios=true}
%% For final camera-ready submission, w/ required CCS and ACM Reference
%\documentclass[acmsmall]{acmart}\settopmatter{}


%% Journal information
%% Supplied to authors by publisher for camera-ready submission;
%% use defaults for review submission.
\acmJournal{PACMPL}
\acmVolume{1}
\acmNumber{CONF} % CONF = POPL or ICFP or OOPSLA
\acmArticle{1}
\acmYear{2022}
\acmMonth{1}
\acmDOI{} % \acmDOI{10.1145/nnnnnnn.nnnnnnn}
\startPage{1}

%% Copyright information
%% Supplied to authors (based on authors' rights management selection;
%% see authors.acm.org) by publisher for camera-ready submission;
%% use 'none' for review submission.
\setcopyright{none}
%\setcopyright{acmcopyright}
%\setcopyright{acmlicensed}
%\setcopyright{rightsretained}
%\copyrightyear{2018}           %% If different from \acmYear

%% Bibliography style
\bibliographystyle{ACM-Reference-Format}
%% Citation style
%% Note: author/year citations are required for papers published as an
%% issue of PACMPL.
\citestyle{acmauthoryear}   %% For author/year citations


%%%%%%%%%%%%%%%%%%%%%%%%%%%%%%%%%%%%%%%%%%%%%%%%%%%%%%%%%%%%%%%%%%%%%%
%% Note: Authors migrating a paper from PACMPL format to traditional
%% SIGPLAN proceedings format must update the '\documentclass' and
%% topmatter commands above; see 'acmart-sigplanproc-template.tex'.
%%%%%%%%%%%%%%%%%%%%%%%%%%%%%%%%%%%%%%%%%%%%%%%%%%%%%%%%%%%%%%%%%%%%%%


%% Some recommended packages.
\usepackage{booktabs}   %% For formal tables:
                        %% http://ctan.org/pkg/booktabs
\usepackage{subcaption} %% For complex figures with subfigures/subcaptions
                        %% http://ctan.org/pkg/subcaption
                        

\usepackage{relsize}
\usepackage{mathpartir}
\usepackage{amsmath}
\usepackage{amsthm}
\usepackage{listings}
\usepackage{xspace}
\usepackage{definitions}
\usepackage{multirow,bigdelim}
\usepackage{pbox}
\usepackage{courier}

\newcommand\multibrace[3]{\rdelim\}{#1}{3mm}[\pbox{#2}{#3}]}

\newcommand{\kjx}[1]{{\color{orange}{KJX: #1}}}
\newcommand{\scd}[1]{{\color{blue}{#1}}}
%\newcommand{\sdN}[1]{{\color{dkgreen}{#1}}}
%\newcommand{\jm}[1]{{\color{magenta}{JM: #1}}}
\newcommand{\sdcomment}[1]{{\ensuremath{\blacksquare}}\footnote{\color{dkgreen}{SD: #1}}}
\newcommand{\secomment}[1]{{\ensuremath{\blacksquare}}\footnote{\se{#1}}}
\newcommand{\jncomment}[1]{{\ensuremath{\blacksquare}}\footnote{\kjx{#1}}}

\newcommand{\sd}[1]{{\color{blue}{#1}}}
 \newcommand{\tobyM}[1]{#1} %[1]{{\color{purple}{Toby: #1}}}
\newcommand{\se}[1]{{\color{green}{#1}}}


\newcommand{\ponders}[3]{\marginpar{\tiny\itshape\raggedright\textcolor{#2}{\textbf{#1:} #3}}\ignorespaces}
\marginparwidth=1.6cm \marginparsep=0cm
\newcommand{\TODO}[1]{} % {{\color{red}#1}}
\newcommand{\sophia}[1]{{\color{blue}#1}}
\newcommand{\toby}[1]{} % {\ponders{Toby}{purple}{#1}}
\newcommand{\susan}[2][]{\ponders{Susan}{brown}{#1} \textcolor{brown}{#2}\xspace}
\newcommand{\james}[1]{\ponders{James}{orange}{#1}}
\newcommand{\jm}[2][]{\ponders{Julian}{magenta}{#1} \textcolor{magenta}{#2}\xspace}
\newcommand{\mrr}[2][]{\ponders{Matthew Ross}{offblue}{{#1}} \textcolor{offblue}{{#2}}\xspace}
\newcommand{\mrrz}[1]{\textcolor{offblue}{{#1}}\xspace}
\newcommand{\Mrr}[2][]{\ponders{Matthew Ross}{teal}{{#1}} \textcolor{teal}{{#2}}\xspace}
\newcommand{\Mrrz}[1]{\textcolor{teal}{{#1}}\xspace}

\newcommand{\sophiaPonder}[2][]{\ponders{Sophia}{blue}{#1} \textcolor{blue}{#2}\xspace}
\renewcommand{\sophia}[2][]


\begin{document}

%% Title information
%\title[Specification and Proof of Necessary Conditions]{Specification
%and Proof of Necessary Conditions}         %% [Short Title] is
%optional;
%\title{Necessity Specifications are Necessary}
\title{\Nec Specifications are Necessary for Robustness}
                                        %% when present, will be used in
                                        %% header instead of Full Title.
%\titlenote{with title note}             %% \titlenote is optional;
                                        %% can be repeated if necessary;
                                        %% contents suppressed with 'anonymous'
%\subtitle{Subtitle}                     %% \subtitle is optional
%\subtitlenote{with subtitle note}       %% \subtitlenote is optional;
                                        %% can be repeated if necessary;
                                        %% contents suppressed with 'anonymous'


%% Author information
%% Contents and number of authors suppressed with 'anonymous'.
%% Each author should be introduced by \author, followed by
%% \authornote (optional), \orcid (optional), \affiliation, and
%% \email.
%% An author may have multiple affiliations and/or emails; repeat the
%% appropriate command.
%% Many elements are not rendered, but should be provided for metadata
%% extraction tools.

%% Author with single affiliation.
\author{First1 Last1}
\authornote{with author1 note}          %% \authornote is optional;
                                        %% can be repeated if necessary
\orcid{nnnn-nnnn-nnnn-nnnn}             %% \orcid is optional
\affiliation{
  \position{Position1}
  \department{Department1}              %% \department is recommended
  \institution{Institution1}            %% \institution is required
  \streetaddress{Street1 Address1}
  \city{City1}
  \state{State1}
  \postcode{Post-Code1}
  \country{Country1}                    %% \country is recommended
}
\email{first1.last1@inst1.edu}          %% \email is recommended

%% Author with two affiliations and emails.
\author{First2 Last2}
\authornote{with author2 note}          %% \authornote is optional;
                                        %% can be repeated if necessary
\orcid{nnnn-nnnn-nnnn-nnnn}             %% \orcid is optional
\affiliation{
  \position{Position2a}
  \department{Department2a}             %% \department is recommended
  \institution{Institution2a}           %% \institution is required
  \streetaddress{Street2a Address2a}
  \city{City2a}
  \state{State2a}
  \postcode{Post-Code2a}
  \country{Country2a}                   %% \country is recommended
}
\email{first2.last2@inst2a.com}         %% \email is recommended
\affiliation{
  \position{Position2b}
  \department{Department2b}             %% \department is recommended
  \institution{Institution2b}           %% \institution is required
  \streetaddress{Street3b Address2b}
  \city{City2b}
  \state{State2b}
  \postcode{Post-Code2b}
  \country{Country2b}                   %% \country is recommended
}
\email{first2.last2@inst2b.org}         %% \email is recommended


%% Abstract
%% Note: \begin{abstract}...\end{abstract} environment must come
%% before \maketitle command
\begin{abstract}
Traditional specifications provide sufficient conditions for
closed programs to be correct --- if a function is invoked with valid
preconditions, it will meet its postconditions upon return: calling a
function will make good things happen.
%
Unfortunately, sufficient
conditions are insufficient for reasoning about
%open
programs in an
open world --- frameworks that can be extended, systems that interact
with third parties, programs subject to unintentional or
%even
malicious attacks.
%
In an open world, \Nec specifications are
necessary: programmers need to reason about necessary conditions
to prove that bad things can never happen.
%
\Nec logic enables programmers to assert and to reason about the
necessary conditions for their programs to be correct, and to infer
the necessary conditions that their programs do (or do not) support. 
%
Using this logic, programmers can prove that bad things
do not happen, defending programs against an unfriendly open world.
\end{abstract}


%% 2012 ACM Computing Classification System (CSS) concepts
%% Generate at 'http://dl.acm.org/ccs/ccs.cfm'.
\begin{CCSXML}
<ccs2012>
<concept>
<concept_id>10011007.10011006.10011008</concept_id>
<concept_desc>Software and its engineering~General programming languages</concept_desc>
<concept_significance>500</concept_significance>
</concept>
<concept>
<concept_id>10003456.10003457.10003521.10003525</concept_id>
<concept_desc>Social and professional topics~History of programming languages</concept_desc>
<concept_significance>300</concept_significance>
</concept>
</ccs2012>
\end{CCSXML}

\ccsdesc[500]{Software and its engineering~General programming languages}
%\ccsdesc[300]{Social and professional topics~History of programming languages}
%% End of generated code


%% Keywords
%% comma separated list
%%%%%%%%%%%%%%%%%%%%%\keywords{keyword1, keyword2, keyword3}  %% \keywords are mandatory in final camera-ready submission


%% \maketitle
%% Note: \maketitle command must come after title commands, author
%% commands, abstract environment, Computing Classification System
%% environment and commands, and keywords command.
\maketitle

\section{Introduction}
%Current systems are complex, and  built out of many different components of different provenance and different degree of trustworthiness.

%\begin{flushright}
%  \textit{nice things are nicer than nasty ones}\\
%  Kingsley Amis, ``Lucky Jim.''
%\end{flushright}

\begin{flushright}
  \textit{Condition B is hard to formalize, \\since it requires saying precisely what a bad
plan is,\\ and we do not attempt to do so.}\\
  Leslie Lamport
\end{flushright}
\susan[that is the best quote I could find - from Byzantine Generals]{}
\sophia[better than Amis's quote, I think]{}

The days of single, monolithic programs are long gone.  Contemporary
software is built over decades, by combining modules and components of
different provenance and different degrees of trustworthiness, and can
interact with almost every other program, device, or person.
In order for the resulting complex system to be correct, we need to be
able to reason about individual components to ensure that they behave
correctly, i.e.\ that good things happen when our programs run.
For example: if I send an email to a valid address, it will be
delivered to its recipient, or if I provide the right password, I can
transfer money from one of my bank accounts to another. 
To prove that good things can
happen, program verification systems can use witnesses, e.g.\ a
precondition, a postcondition describing the good thing (the desired
effect), and a code snippet, whose execution will establish the
effect, given the precondition.  The critical point here is that the
precondition is a \emph{sufficient} condition for the code snippet to
make the good thing happen: given the precondition, executing a
correct code snippet is guaranteed to achieve the postcondition.

Unfortunately, in a system of any complexity, knowing that good things
will happen is not enough: we also need to be sure that bad things
cannot happen. For example: we also need to be sure that an email can
only be read by the intended recipient; or that I can only transfer
money if I provide the right password. To address this problem,
we need to consider the \emph{necessary} conditions under which some
postcondition can be achieved (otherwise\susan[replaced 'or' with otherwise as I thought or was slightly confusing]{} some bad thing can happen):
it is necessary that someone is sent an email before they can read it;
it is necessary that the correct password is provided before money can
be transferred.  If a necessary condition can never be achieved, then
bad things guarded by that condition cannot happen.

The challenge here is twofold: How do we specify the bad things we are
concerned about, and how do we prove that the bad things we've
specified do not happen?  These challenges are difficult because we
cannot refer to just one component of a software system.  A sufficient
specification can deal with a single component in isolation --- a
single function for pre- and postconditions; a single class or data
structure for invariants. A necessary specification, however, must
provide guarantees which encompass the software system in its
entirety, and constrain the emergent behaviour of all its components,
for an open system, all possible sequences of API invocations.


Rather than considering the sufficient conditions to
achieve a given effect, we express necessary preconditions for the
holistic system. For example, a third party will not get to read my
email unless I forward it to them - the forward is a necessary
precondition for the reading. To reason about necessary conditions we
develop our Logic of Necessity. 

There has been work in the past to expand specifications of systems beyond the specification of good things.  blah, blah, blah  \citeauthor{FASE} were also concerned with 
 holistic specifications and necessary conditions, which they expressed through time operators.
 \Chainmail appears to be less rich than Chainmail, however we have been able to specify all their examples. In addition, \Chainmail has a proof system, which Chainmail lacked.
%To specify the bad things, we cannot
%refer to one particular piece of code, % and cannot refer to one witness;
% instead, we need to make a guarantee which encompasses all possible
% functions executable by a module in all their possible sequences or even interleavings --
 %we need to consider their \emph{emerging} behaviour.
 %Rather than considering the sufficient conditions to achieve a certain effect,
 %we express the necessary precondition. For example,    a third party 
 %will not get to read my email unless I forward it to them -- the forwarding
% is a necessary precondition for the reading.
 
 %To reason about necessary conditions, we develop a special
 %logic with such necessity triples. The most basic such holistic assertions 
 %can be derived from classical assertions (sufficient \jm[old:truples]{triples})
 %under assumptions of encapsulation;
 %they can then be further combined using our logic of necessity.




We illustrate our ideas in terms of the following example:
\prg{Account}s have a balance and a password. One may transfer
\prg{100} units from one \prg{Account} to another, but only provided
that the caller provides the right password. Below we show version
\sf{I} of the code for a class \prg{Account}.  We use a Java-like
syntax, and assume that fields are ``class-wide'' private (again as in
Java, so methods may read and write fields of any instance of that
class) and that passwords are unforgeable and not enumerable (again as
in Java, albeit without reflection).

 
\begin{lstlisting}[language=Chainmail]
class Account{
   field balance:int 
   field pwd: Password 
   method transfer(dest:Account, pwd':Password) -> void {
      if (this.pwd==pwd') {
            this.balance-=100;  dest.balance+=100; }  }
}
\end{lstlisting}

as well as a ``classical'' specification of the method \prg{transfer}:


 (ClassicSpec)$  \ \ $  $\triangleq$
\begin{lstlisting}[mathescape=true, frame=lines, language=Chainmail]
  method transfer(dest:Account, pwd':Password) -> void {
       ( PRE:  this.balance=bal1 $\wedge$ this.pwd==pwd' $\wedge$ dest.balance=bal2 $\wedge$ dest=/=this 
         POST: this.balance == bal1-100 $\wedge$  dest.balance == bal2+100 )
       ( PRE: this.balance=bal1 $\wedge$ this.pwd=/=pwd' $\wedge$ dest.balance=bal2
         POST: this.balance == bal1 $\wedge$  dest.balance=bal2 )
       ( PRE: a : Account $\wedge$ a=/=this $\wedge$ a=/=dest  $\wedge$ a.balance=bal $\wedge$ a.pwd=pwd1
         POST:  a.balance=bal $\wedge$ a.pwd=pwd1)
       ( PRE: a : Account $\wedge$ a.pwd=pwd1  
         POST: a.pwd=pwd1)       
\end{lstlisting}\footnote{Perhaps omit some of the lines here, but we do need them all in the full discussion}
 
  
 
 Now consider two further versions of the class account, given in Fig. \ref{fig:ExampleAccount}.
 In version {\sf{II}} and version {\sf{III}} the class has an additional method, \prg{set}, which enables the resetting of the password.
The method \prg{transfer} in all three versions of the class \prg{Account} satisfies the (ClassicSpec), 
however, while executing the first and third version of \prg{Account} won't exhibit unwanted behaviour, the second version doesn't preclude it.
%Namely version II allows any client to change the password of the account, and then to repeatedly withdraw money from it.
  
% On the other hand, we expect our software -- even if complex -- to provide some simple, high level
%guarantees, e.g. email addressed to me personally will not be read by a third party unless I explicitly 
%forwarded it to them.
%We except  our software to  behave correctly, even when used by a careless or malicious third party. 
%Such use of a software often consist of a sequence of actions performed on the module. 
%
%Software components respond to single actions, 
%or to sequences of such single actions. 
%When thinking about a software component we want think about the behaviour of each 
%action in isolation, but also about the \emph{emergent behaviour}, ie all
% the possible effects of the combinations of these actions. 
  
  
 
 \begin{figure}[hbtp]
 \begin{tabular}{lll}
\begin{minipage}{0.35\textwidth}
\begin{lstlisting}[language=chainmail]
class Account{
   field balance:int 
   field pwd: Password 
   method transfer(..) 
        ... as earlier ...
   method set(pwd':Password){
         this.pwd=pwd' }
}   
\end{lstlisting}
\end{minipage}
  &\ \ \  \ \   &
\begin{minipage}{0.50\textwidth}
\begin{lstlisting}[language=chainmail]
class Account{
   field balance:int 
   field pwd: Password 
   method transfer(..) 
        ... as earlier ...
   method set(pwd',pwd'':Password){
         if (this.pwd==pwd'){this.pwd=pwd''} }
}
\end{lstlisting}
\end{minipage} 
 \end{tabular}
  \caption{class \prg{Account} version II, and \ \ \ \ \ \  class \prg{Account} version III}
 \label{fig:ExampleAccount}
 \end{figure}

 
 The flaw in version {\sf{II}} arises from \emph{emergent} behaviour as \prg{set} 
 can be used to overwrite the
 password, and then using the new password \prg{transfer}  can be called.
% If we want the \prg{Account} class to be robust, we must prohibit the password from being freely available.
 Therefore, we propose a holistic specification which says that
 the \prg{balance} of an \prg{Account} reduces only if an object which does not belong to the
 class \prg{Account} has access to the password:
 
 \begin{lstlisting}[language = Chainmail, mathescape=true, frame=lines]
(HolisticSpec)  $\triangleq$  a:Account $\wedge$ a.balance==bal
                   to a.balance < bal
                   onlyIf $\exists$ o.[$\external{\texttt{o}}$ $\wedge$ $\access{\texttt{o}}{\texttt{a.pwd}}$]
\end{lstlisting}
 
%\jm[good point. fixed]{WHY DOES o HAVE A PRIME?}
 
 In more detail, the specification from above says that if in the current
 configuration \prg{a} is an \prg{Account},
 % with balance \prg{bal}, 
 and in some future configuration \prg{a} will have a balance less than the current one, then, in the \emph{current} configuration
 there must exist some object \prg{o}, which is \emph{external} to our module (does not belong to class
 \prg{Account}), and which has access to \prg{a}'s password.
 
 
 Thus, having access to the password is a necessary condition for the balance to reduce.
 Note, that the specification only talks about effects (here the balance reducing), and does not
 talk about individual methods (such as \prg{set} or \prg{transfer}).
 

 The holistic specification language, \Chainmail, extends traditional specifications with
notions of access, %(which object points to which)
control 
%(which methods are called)
\footnote{Sophia -perhaps drop as confusing? Susan -I think both of the bracketed phrases should be omitted, as we have already used the word access when describing the passwords and we don't mention control anywhere}, 
 the distinction between external and internal objects, and necessary conditions, 
 whereby the assertion $A~to~A'~onlyIf~A''$ expresses that if in in the current configuration
 $A$ holds, and in some future configuration $A'$, 
\susan[{changed first " to '}]{}
 will hold, then $A''$ must also hold in the 
 current configuration.  
  
 The contributions of this paper are as follows:
 
 \begin{enumerate}
 \item
 We propose \Chainmail, a specification language with which to
express holistic specifications. 
 \item
  We propose a Logic of Necessity for writing holistic assertions and for proving a module's adherence to said assertions.
 This logic builds on top of classical pre- post- conditions.
 \item
 We prove soundness of our logic.  
 \item
 We use our logic to prove adherence to the holistic assertion of the example
outlined  in this section
 \end{enumerate}
 
 We have developed a Coq proof of soundness of our approach, and a Cow proof of the
 example in this paper. We make these proofs available as supplementary material.
% QUESTIONS:
% \begin{itemize}
% \item
% holistic assertion -- different name or say we adopt from FASE; holistic -> comprehensive, integrated, aggregate \susan[I think we should keep holistic]{}
% \item
% Name for our logic?  Holistic logic? What is we have a different take on holistic next year? Necessity triples? 
%\item
%If our \Chainmail is simpler than the FASE Chainmail, then we need to argue that we can express all FASE examples
% \end{itemize}


The rest of the paper is organised as follows: .... 
%Section
%~\ref{sect:motivate:Bank} 
%\sd{gives an example from the literature} which we will use 
%to elucidate key points of \Chainmail.
%%motivates our work via an example, and then section
%~\ref{sect:chainmail} presents the \Chainmail\ specification
%language.  Section~\ref{sect:formal} introduces the formal model
%underlying \Chainmail, and then section~\ref{sect:assertions} defines
%the 
%semantics of \Chainmail's assertions.
%% SD the below is NOT ture
%%full details are relegated toappendices.   
%Section~\ref{sect:example} shows how key points of 
%exemplar problems can be specified in \Chainmail,
%section~\ref{sect:discussion}
%discusses our design, \ref{sect:related} considers related
%work, and section~\ref{sect:conclusion} concludes.
%We relegate various details to appendices.



  
\section{The Meaning of Necessity}
\label{s:semantics}

 
In this section we define our \Nec specification language.
%In order to do that, 
\scd{We} first define 
an underlying programming language, \Loo  (\S \ref{sub:Loo}).
%modification to description of type system. Not sure if this is clear enough.]{
%We base \Loo on \LangOO, as defined by 
%\cite{FASE}. Whereas \LangOO was untyped, \Loo 
%%includes a simple type
%%system to allow for
%has type based restrictions on external access to private data.}
We then  define an assertion language, \SpecO,  which can talk about
 the contents of the state, as well \scd{as about} 
provenance, permission and control (\S \ref{sub:SpecO}).
Finally, we define the syntax and semantics of 
\Nec specifications  (\S \ref{s:holistic-guarantees}).
% our \Nec specification language  (\S \ref{s:holistic-guarantees}).


\subsection{\Loo}
\label{sub:Loo} 
%\jm[TODO: mention the type system and the restriction on external method calls]{}
%% We introduce a simple object-oriented language, \Loo, upon 
%% which our specification language sits.
 \Loo is a  simple  imperative, \scd{sequential}, 
class based, typed, object oriented language.
%\susan[Java fields are package private not class private - are fields so important that we need to say this?]{Fields are private to the class where they are defined.}
 Given its simplicity, %  the simplicity of \Loo, we do notdefine it here, instead, 
 we direct the reader to Appendix \ref{app:loo} for 
the full definitions, \scd{and} introduce here only % syntax and operational semantics.
 the concepts most relevant to the
treatment of the open world guarantees.
%\jm[]{\Loo fields are private in the way fields in Java are private,
%the privacy is class-wide, i.e. they may only be read or written to by 
%objects of the same class.}
\scd{\Loo is based} on \LangOO from
\cite{FASE}, \scd{with some small variations}, as well as 
the addition of a % while \LangOO is untyped, \Loo 
 a simple type system -- more in \sophiaPonder[TODO]{reference}.
%has type based restrictions on external access to private data.}


A \Loo state $\sigma$ consists of a 
heap $\chi$, and a  {stack $\psi$, which is a sequence of frames $\phi$}.
A frame $\phi$ consists of
local variable map, and a continuation, i.e. a sequence of statements to be executed.
 A statement may assign to variables, create new objects and push them to the heap, 
perform field reads and writes on objects,  or
 call methods on those objects. 

%Program 
 Modules are mappings
from class names to class definitions. 
Execution takes place in the context of  a module $M$ and   \scd{a state $\sigma$}.
It is % Execution
 \scd{defined} through a small-step, unsurprising, semantics, of the form \ \ 
   $M, \sigma \leadsto \sigma'$,\  c.f. Appendix \ref{app:loo}.
The   continuation of the top frame contains the statements \scd{that will be} % currently being 
executed next.
 % chopped, as generic 
 % There are several properties  of \Loo that are important to the central topic of this paper. 
 
As discussed in \S \ref{s:approach}, we are interested in guarantees which hold
\sophiaPonder[need to remind reader of internal/external]{during execution of an internal, 
known, trusted module $M$ when linked together with any
unknown, untrusted, module $M'$.} These guarantees need only hold 
when the external module is executing; \scd{we} are not concerned if they are
temporarily broken by the internal module. Therefore, we are only interested in states where the
executing object (the \prg{this}) is an external object. 
For this, we define the  \emph{external state semantics}, of the form 
$\reduction{M'}{M}{\sigma}{\sigma'}$, where $M'$ is the external
module, and $M$ is the internal module, and where we
collapse all internal steps into one single step.

 

\begin{definition}[External State Semantics]
\label{def:pair-reduce}
For  
% If we say "internal module", it is sounds as something makes the module be internal
  modules $M$,  $M'$, and program states $\sigma$, $\sigma'$, 
we say that $\ \ \ \ \ \ \ \ \reduction{M'}{M}{\sigma}{\sigma'}\ \ \ \ \ \ \ \ $ if and only if there exist 
$n\in\mathbb{N}$, \scd{and states $\sigma_0$,...$\sigma_n$}, such that
\begin{itemize}
% SD changed because the old version was slightly wrong
\item
$\sigma$=$\sigma_1$, and  $\sigma'$=$\sigma_n$,
\item
$M' \circ M, \sigma_i \leadsto \sigma_{i+1}$  \ \ \ for all $i\in [0..n)$,
\item
$\class{\sigma}{\sigma.\prg{this}}, \class{\sigma'}{\sigma'.\prg{this}}\in M'$,
\item
$\class{\sigma_i}{\sigma_i.\prg{this}} \in M$\ \ \ for all $i\in [1..n)$.
\end{itemize} 
\end{definition}
% In Definition \ref{def:pair-reduce}  % of external state semantics makes reference to a 
The function
$\class{\sigma}{x}$  looks up 
the class of   the object referred by variable $x$ in the heap of $\sigma$. 
  % for a specific variable in a specific program.
% SD not a variable, and no program.
The module linking operator $\circ$, applied to two modules, $M\circ M'$, % . The operator $\circ$
 %  results in the union of two disjoint modules.
combines the two modules into one module in the obvious way, provided their
domains are disjoint.
Full details in  Appendix \ref{app:loo}.
\begin{figure}[htb]
%  \vspace*{-2.5mm}
%  \begin{center}
%   \begin{minipage}{0.80\textwidth}
%     \begin{center}
%       \includegraphics[width=\linewidth]{diagrams/VisibleStates.pdf}
%     \end{center}
%   \end{minipage}
%   \end{center}
%   \vspace*{-2.5mm}
TODO new figure. \scd{should only have (A) and (B)}
   \caption{External States Semantics
     (Def. \ref{def:pair-reduce}). %
     % 
     (A) $\exec{{\color{blue}M'} \circ {\color{orange}M}}{\sigma_1}{\ldots}\leadsto \sigma_9$\ \ \and \ \ \ 
     (B) $\reduction{{\color{blue}M'}}{{\color{orange}M}}{\sigma_2}{\ldots}\leadsto \sigma_9$
    %  (c) $\reduction{{\color{orange}M'}}{{\color{blue}M}}{\sigma_1}{\ldots}\leadsto \sigma_8$
    }
   \label{fig:VisibleStates}
 \end{figure}
 
Fig. \ref{fig:VisibleStates} provides a simple graphical description of 
our external states semantics: \scd{(A) is the ``normal'' execution after 
linking two modules into one: \ $M' \circ M, ... \leadsto ...$. (B) is the
 external states execution when $M'$ is external,\   $\reduction{M'}{M}{...}{...}$  (B).}
\sophiaPonder[moved to earlier]{We  use} the notation\ \  $\reductions{M'}{M}{\sigma}{\sigma'}$ \ 
to denote
zero or more reduction steps starting at state $\sigma$ and ending at state $\sigma'$, in the context of internal module 
$M$ and external module $M'$.
 





\sophiaPonder[better connection, avoided repetition, tighten]{Not only are we unconcerned 
about internal states, 
%  (where \prg{this} belongs to a class from the inernal module),
 we are also unconcerned with  states which cannot ever arise from execution.
% 
% which are upheld by the internal 
%module. Moreover,    these guarantees  need to be satisfied only at `reachable' states,
%and need not be satisfied at states that are
%not reachable -- this is described formally in Definition \ref{def:arising}. 
\emph{Arising} states are those that  may arise by external states execution
% of  a given internal module linked with an external module.
% We describe the states of interest as the \emph{arising states}. 
starting at some initial configuration:}



\begin{definition}[Arising Program State]
\label{def:arising}
For   modules $M$ and  $M'$, a program state $\sigma$ is 
said to be an \emph{arising} state, formally \ \ \ $\arising{M}{M'}{\sigma}$,\ \ \ 
if and only if there exists some $\sigma_0$ such that $\initial{\sigma_0}$ and
$\reductions{M'}{M}{\sigma_0}{\sigma}$.
\end{definition}

% Definition \ref{def:arising} uses the definition for 
% In the definition above we used 
\scd{A state is \emph{Initial}} 
% he definition of which can be found in  Definition \ref{def:initial}. 
% which  characterize states at the start of program execution.
%The heap of an initial state should 
if its heap contains a single object of class \prg{Object}, and
its  stack   consists of a single frame, whose local variable map is a
mapping from \prg{this} to the single object, and whose continuation \scd{is}  any statement.
% to be executed.
More in Def. \ref{def:initial}and Def. \ref{def:arising}.
%%========================
%%old
%Finally, \Loo has a simple class based type system with the following properties:
%\begin{description}
%\item[(1)]
%Classes may be optionally annotated as \prg{enclosed}
%\item[(2)]
%Methods of non-\prg{enclosed} classes may not return objects of \prg{enclosed} classes
%\item[(3)]
%Modules are typed in isolation of other modules, thereby implicitly prohibiting
%method calls from internal objects to external objects.
%\end{description}
%The type system \jm[]{is included to allow a simple and convenient way to prove encapsulation of \SpecO assertions,
%however it is not fundamental to the definition of either the  \Nec specification language or logic, and could easily be substituted out 
%for any other encapsulation system.}
%%==================================

\jm[]{Finally, \Loo has a simple class based type system where 
classes may be optionally annotated as \enclosed. An object 
of an \enclosed type may not be accessed by module-external
objects. This helps simplify the demonstrative proof of the running 
Bank Account example (\S\ref{s:examples}). In order to enforce this 
restriction, methods of non-\enclosed classes are prohibited from
returning \enclosed objects.}

\jm[]{\Loo's type system has one further implication: modules are typed 
in isolation of other modules, thereby implicitly prohibiting
method calls from internal objects to external objects.}

\jm[]{The type system is included to allow a simple and convenient way to prove that access to certain 
objects is restricted. This restriction simplifies the proof of our running example to a manageable degree,
however it is not fundamental to the definition of either the  \Nec specification language or logic, 
and could easily be removed.}


\subsection{\SpecO}
\label{sub:SpecO}
\SpecO extends the expressiveness of standard specification languages
with assertion forms capturing key concepts of software security:
 \emph{permission}, \emph{provenance}, and \emph{control}.
 \emph{Permission} and \emph{Provenance} are inspired by the capabilities literature, while
\emph{Control} assists construction of proofs.
%That is, \SpecO specifications are able to specify which objects have
%access to which other object (\emph{permission}), whether an object's origin
%is internal or external to known code (\emph{viewpoint}), or which objects call which 
%methods (\emph{control}). 

\subsubsection{Syntax}

%\begin{figure}[t]
%\footnotesize
%\[
%\begin{syntax}
%\syntaxElement{A}{}
%		{
%		\syntaxline
%				{e}
%				{e : C}
%				{\neg A}
%				{A\ \wedge\ A}
%				{A\ \vee\ A}
%				{\all{x}{A}}
%				{\ex{x}{A}}
%		\endsyntaxline
%		}
%		{
%		\syntaxline
%				{\access{x}{y}}
%				{\internal{x}}
%				{\external{x}}
%		\endsyntaxline
%		}
%		{
%		\syntaxline
%				{\calls{x}{y}{m}{\overline{z}}}
%		\endsyntaxline
%		}
%\endSyntaxElement\\
%\end{syntax}
%\]
%\caption{\SpecO Assertions}
%\label{f:chainmail-syntax}
%\end{figure}
%

\begin{definition}
Assertions ($A$) in the basic specification language, \SpecO, are defined as follows

\label{f:chainmail-syntax}
%\footnotesize
\[
\begin{syntax}
\syntaxElement{A}{}
		{
		\syntaxline
				{e}
				{e : C}
				{\neg A}
				{A\ \wedge\ A}
				{A\ \vee\ A}
				{\all{x}{A}}
				{\ex{x}{A}}
		\endsyntaxline
		}
		{
		\syntaxline
				{\access{x}{y}}
				{\internal{x}}
				{\external{x}}
%		\endsyntaxline
%		}
%		{
%		\syntaxline
				{\calls{x}{y}{m}{\overline{z}}}
		\endsyntaxline
		}
\endSyntaxElement\\
\end{syntax}
\]


\end{definition}


Def. \ref{f:chainmail-syntax} gives the assertion syntax of \SpecO. % the \SpecO specification language.
An assertion may be an expression, a class assertion, the usual connectives and quantifiers, along 
with the following non-standard assertion forms:
\begin{itemize}
\item
\emph{Permission} ($\access{x}{y}$): % Which objects have access to which other objects (i.e.
  $x$ has access to $y$.
\item
{\emph{Provenance}} ($\internal{x}$ and $\external{y}$): %Which objects are internal or external to our component.
 $x$ is internal, and $y$ is external.
\item
\emph{Control} ($\calls{x}{y}{m}{\overline{z}}$): 
$x$ calls method $m$ on object $y$ with arguments $\overline{z}$.
\end{itemize}

\subsubsection{Semantics of \SpecO}
The semantics of \SpecO assertions is given in Definition \ref{def:chainmail-semantics}. 
\SpecO makes use of several language features of 
\Loo that can be found in Appendix \ref{app:loo}. Specifically, $\eval{M}{\sigma}{e}{v}$
is the evaluation relation for expressions, and is interpreted as expression $e$ evaluates
to value $v$ in the context of program state $\sigma$, with module $M$. The full
semantics of expression evaluation are given in Fig. \ref{f:evaluation}. It should 
be noted that expressions in \Loo may be recursively defined, and thus evaluation may not
necessarily terminate, however the logic remains classical because recursion is restricted
to expressions, and not generally to assertions.

Further, Definition \ref{def:chainmail-semantics} uses the interpretation of variables
within a specific frame or state: i.e. $\interpret{\phi}{x} = v$, meaning that $x$ maps to
value $v$ in the local variable map of frame $\phi$, and $\interpret{\sigma}{x} = v$ meaning $x$ 
maps to value $v$ in the top most frame of $\sigma$'s stack. And the term  $\interpret{\sigma}{x.f} = v$
has the obvious meaning.


\begin{definition}[Satisfaction of \SpecO Assertions] 
\label{def:chainmail-semantics}
We define satisfaction of an assertion $A$ by a program state $\sigma$ with internal module $M$ as:
\begin{itemize}
\item
$\satisfiesA{M}{\sigma}{e}$ \ \ \ iff \ \ \  $\eval{M}{\sigma}{e}{\true}$
\item
$\satisfiesA{M}{\sigma}{e : C}$ \ \ \ iff \ \ \  $\eval{M}{\sigma}{e}{\alpha}$ \textit{and} $\class{\sigma}{\alpha} = C$
\item
$\satisfiesA{M}{\sigma}{\neg A}$ \ \ \ iff \ \ \  ${M},{\sigma}\nvDash{A}$
\item
$\satisfiesA{M}{\sigma}{A_1\ \wedge\ A_2}$ \ \ \ iff \ \ \  $\satisfiesA{M}{\sigma}{A_1}$ and 
$\satisfiesA{M}{\sigma}{A_2}$
\item
$\satisfiesA{M}{\sigma}{A_1\ \vee\ A_2}$ \ \ \ iff \ \ \  $\satisfiesA{M}{\sigma}{A_1}$ or 
$\satisfiesA{M}{\sigma}{A_2}$
\item
$\satisfiesA{M}{\sigma}{\all{x}{A}}$ \ \ \ iff \ \ \  
$\satisfiesA{M}{\sigma[x \mapsto \alpha]}{A}$, \ 
\ \ for some $x$ fresh in $\sigma$, and for all $\alpha\!\in\!\sigma.\prg{heap}$.
\item
$\satisfiesA{M}{\sigma}{\ex{x}{A}}$ \ \ \ iff \ \ \  
$\satisfiesA{M}{\sigma[x \mapsto \alpha]}{A}$, \ 
\ \ for some $x$ fresh in $\sigma$, and for some $ \alpha\!\in\!\sigma.\prg{heap}$. 
\item
$\satisfiesA{M}{\sigma}{\access{x}{y}}$ \ \ \ iff \ \ \  
\begin{itemize}
\item
$\interpret{\sigma}{x.f}={\interpret{\sigma}{y}}$ for some $f$, \  or
\item
there exists some $z$, and some frame $\phi$ in the stack of $\sigma$ such that {$\interpret{\sigma}{x}=\interpret{\phi}{\prg{this}}$}, {and $\interpret{\sigma}{y}=\interpret{\phi}{z}$}
\end{itemize}
\item
$\satisfiesA{M}{\sigma}{\internal{x}}$ \ \ \ iff \ \ \  
$\textit{classOf}(\sigma,x) \in M$
\item
$\satisfiesA{M}{\sigma}{\external{x}}$ \ \ \ iff \ \ \  
$\textit{classOf}(\sigma,x) \not\in M$
\item
$\satisfiesA{M}{\sigma}{\calls{x}{y}{m}{z_1, \ldots, z_n}}$ \ \ \ iff \ \ \ 
\begin{itemize}
\item
$\sigma.\prg{contn} = (\_ := y'.m(z'_1,\ldots,z'_n))$, % and is superfluous, enums are ands, unless expltly stated   
\item
$\satisfiesA{M}{\sigma}{x = \prg{this}}$
%$\interpret{\sigma}{x} = \interpret{\sigma}{\prg{this}}$  % and
\item
$\satisfiesA{M}{\sigma}{y = y'}$
%$\sd{\interpret{\sigma}{y} = \interpret{\sigma}{y'}}$ % and
\item
$\satisfiesA{M}{\sigma}{z_i = z'_i}$\ \ \ for all $1\!\leq i\!\leq n$
%$\interpret{\sigma}{z_i} = \interpret{\sigma}{z'_i}$ \ \ \ for all $1\!\leq i\!\leq n$
\end{itemize}
\end{itemize}
\end{definition}

 
Finally, we define what it means for a module to satisfy an assertion:
a module $M$ satisfies an assertion $A$, if all states $\sigma$
arising from external steps execution of that
module with any other external module, satisfy $A$. 
 
\begin{definition} [Assertion Satisfaction by Modules]
\label{def:mdl-sat}
For a module $M$ and assertion $A$, we say that\ \  $\satisfies{M}{A}$ \ \ if and only if 
for all modules $M'$, and all $\sigma$, if $\arising{M'}{M}{\sigma}$, then $\satisfiesA{M}{\sigma}{A}$.
\end{definition}


% Thus, satisfaction by a module  allows us to talk 
% about what is true for a given module without introducing the 
% details of specific program configurations, a critical component 
% of constructing our Logic of Necessity in Section \ref{s:inference}. 

A proof system for such assertions is indicated by a judgment of the form $\proves{M}{A}$. 
We will not define such a judgment, and just rely on its existence (cf. Theorem 3.2).
We define soundness of such a judgment in the usual way:

\begin{definition}[Soundness of \SpecO Provability]
\label{ax:specW-prove-soundness}
A judgment of the form $M \vdash A$ is \emph{sound}, if for all
  all modules $M$ and assertions $A$, \ if $\proves{M}{A}$ then $\satisfies{M}{A}$.
\end{definition}

\subsubsection{Wrapping}

We define a useful shorthand: the $\wrapped{}$ predicate  states 
that only \internalO objects have access to some object.
That object may be either \internalO or \externalO.
\begin{definition}[Wrapped]
$\wrapped{o}\ \triangleq\ \all{x}{\neg \access{x}{o}\ \vee\ \internal{x}}$
\end{definition}
Wrapped is critical as it captures the conditions under \jm[]{an interaction
with the \internalM module is necessary}. 
\jm[]{As an example, consider the Bank Account example from Section \ref{s:outline}: if} only \internalO
objects have access to an account's password, then
it follows that access to the password may not 
be gained except by an interaction with the \internalM
module, and subsequently if the \internalM module
is secure we know that the password cannot be leaked.
\jm[]{This predicate forms a core aspect of Part 2 mentioned in Section \ref{s:outline},
and is elaborated on more in Section \ref{s:classical-proof}.}
 
 

%\begin{figure}[t]
%\begin{mathpar}
%\infer
%		{M;\ M',\ \sigma\ \vdash\ e : \prg{intrnl}}
%		{M;\ M',\ \sigma\ \vdash\ e : \prg{encap}}
%		\and
%\infer
%		{M;\ M',\ \sigma\ \vdash\ e : \prg{intrnl}}
%		{M;\ M',\ \sigma\ \vdash\ e.f : \prg{encap}}
%		\and
%\infer
%		{M;\ M',\ \sigma\ \vdash\ e : \prg{intrnl}}
%		{M;\ M',\ \sigma\ \vdash\ e.g(e') : \prg{encap}}
%\end{mathpar}
%\caption{Encapsulated Expressions}
%\label{f:intrnl}
%\end{figure}
	
%	\begin{figure}[h]
%	\[
%	\begin{array}{llr}
%	A & ::= & \textit{Assertions}\\  
%	| & e & \\
%	| & e\ :\ C & \\
%	| & e\ \in\ S & \\
%	| & A\ \prg{in}\ S & \\
%	| & \access{x}{y} \\
%	| & \internal{x} \\
%	| & \external{x} \\
%%	| & \mut x y f &\\
%%	| & \gives x y z &\\
%	| & \calls{x}{y}{m}{args} \\
%	| & \changes{S}{A} \\
%	| & \neg A & \\
%	| & A\ \wedge\ A & \\
%	| & A\ \vee\ A & \\
%	| & A\ \longrightarrow\ A & \\
%	| & \forall\ x.\ [A] & \\
%	| & \exists\ x.\ [A] & \\
%	| & \forall\ S.\ [A] & \\
%	| & \exists\ S.\ [A] &
%	\end{array}
%%	\begin{array}{llr}
%%	s & ::= & \textit{Source}\\
%%	| & \prg{int} & \\
%%	| & \prg{ext} & \\
%%	| & \_ &
%%	\end{array}
%	\]
%	\caption{Assertions}
%	\label{f:assertions_triple2}
%	\end{figure}





\subsection {\Nec Specifications}
\label{s:holistic-guarantees}

Our \Nec specification language extends \SpecO with novel 
%\emph{Necessity Specifications}.
\sophiaPonder[]{\emph{necessity operators}}.
In this Section we define its syntax (Def. \ref{f:holistic-syntax}) and semantics 
%(Def. \ref{d:necessOpers}) 
(Def. \ref{def:necessity-semantics})
%of
%\emph{\NecessitySpecifications}.
We have the following three operators:

% described below:





\paragraph{Only If}
[$\onlyIf{A_1}{A_2}{A}$]: If an arising program state satisfies $A_1$, and after some execution, a state program state satisfying $A_2$ is reached, 
then the original program state must have also satisfied $A$.
e.g. if the balance of a bank account changes over time, then there must be some external object in the current 
program state that has access to the account's password.

\paragraph{Single-Step Only If}
[$\onlyIfSingle{A_1}{A_2}{A}$]: If an arising program state satisfies $A_1$, and after a single step of execution, a state satisfying $A_2$ is reached, 
then the original program state must have also satisfied $A$.
e.g. if the balance of a bank account changes over a single execution step, then that execution step must be a method call to the bank \prg{transfer} method.

\paragraph{Only Through}
[$\onlyThrough{A_1}{A_2}{A}$]: If an arising program state satisfies $A_1$, and after some execution, a state satisfying $A_2$ is reached, then program execution must have passed through some state satisfying $A$.
e.g. if the balance of an account changes over time, then the bank's \prg{transfer} method must have been called 
in some intermediate state. Note 
that the intermediate state where $A$ is true might be the initial state,
or final state.

%\begin{figure}[t]
\begin{definition}
Assertions ($H$)
% in   \Chainmail \NecessitySpecifications  
are defined as follows:
% \footnotesize

\[
\begin{syntax}
\syntaxElement{H}{}
		{
		\syntaxline
				{A}
				{\onlyIf{A_1}{A_2}{A_3}}
				{\onlyThrough{A_1}{A_2}{A_3}}
		\endsyntaxline
		}
		{
		\syntaxline
				{\onlyIfSingle{A_1}{A_2}{A_3}}
		\endsyntaxline
		}
\endSyntaxElement\\
\end{syntax}
\]
%\caption{Syntax of \Chainmail Necessity Specifications}
\label{f:holistic-syntax}
\end{definition}
%\end{figure}


All three operators talk about assertions satisfied in the 
current state as well as of assertions satisfied in some future state. 
These assertions may contain variables, whose denotation might change during
program execution: the 
map may change, variables may be overwritten, or the entire local variable maps may be lost on a method return.
For this reason, before we provide the semantics of our \Nec specification language, we first introduce an adaptation operator
to account for variable renaming throughout the execution of a program.
\begin{definition}
\label{d:adapt}
$\adapt{\sigma'}{\sigma} \triangleq (\chi, \{\prg{local} := \beta[\overline{z} \mapsto \beta(\overline{z}')], \prg{contn}:= [\overline{z'}/\overline{z}]c\} : \psi)$
where 
\begin{itemize}
\item
$\sigma = (\_, \{\prg{local}:=\beta; \prg{contn}:=\_\} : \_)$, and
$\sigma' = (\chi, \{\prg{local}:=\beta', \prg{contn}:=c\} : \psi)$
\item
$dom(\beta') = \overline{z'}$, $dom(\beta) \cap \overline{z} = \emptyset$, and $|\overline{z'}| = |\overline{z}|$
\end{itemize}
\end{definition}
\jm[]{Def. \ref{d:adapt} allows satisfaction to take variable renaming during evaluation into account. 
As an example, consider the following code snippet.}
\begin{lstlisting}[frame=lines]
x.f := y
x := z
\end{lstlisting}
\jm[]{After the evaluation of line 1 , \texttt{x} has access to \texttt{y}, however when we overwrite \texttt{x} in line 2, 
this is no longer necessarily true, as \texttt{x} might now refer to a different object, even though the object previously referred to 
by \texttt{x} has not changed. A similar issue might occur when either calling a method, or returning from a method, as 
the entire variable map changes under such circumstances. Adaptation (Def. \ref{d:adapt}) provides a convenient way to refer to 
objects across time, while ignoring rewrites and new frames.}


We can now define our semantics,  $M \models H$, 
%the semantics of the Necessity Specifications,
%  in Definition \ref{def:necessity-semantics}.  The definition goes 
by cases over the \sophiaPonder[we were saying three forms]{four}possible syntactic forms of $H$: 


\noindent
\begin{definition}[\Nec Semantics]\susan[Can this be called Semantics or Necessity Specification Language]{}\jm[used to be called ``Necessity Specifications'']{}
\label{def:necessity-semantics}
For any assertions $A$, $A_1$, $A_2$, and $A$,  we define \\


$\bullet$ \ $\satisfies{M}{{A}}$ \ \ \ iff\ \ \ for all $M'$, $\sigma$,\ if $\arising{M}{M'}{\sigma}$, then $\satisfiesA{M}{\sigma}{A}$. (see Def. \ref{def:mdl-sat})\\

%$\bullet$ \ $\satisfies{M}{{A}}$ \ \ \ as defined in \ref{def:mdl-sat} \\

$\bullet$ \ $\satisfies{M}{\onlyIf {A_1}{A_2}{A}}$ \ \ iff\ \  for all $M'$, $\sigma$, $\sigma'$, such that $\arising{M}{M'}{\sigma}$; \\ % and\\

\begin{tabular}{lr}
$\;\;\;\;$- $\satisfiesA{M}{\sigma}{A_1}$  & \rdelim\}{3}{3mm}[$\;\;\;\Rightarrow\;\;\;$  $\satisfiesA{M}{\sigma}{A}$] \\
$\;\;\;\;$- $\satisfiesA{M}{\sigma' \triangleleft \sigma}{A_2}$   \\
$\;\;\;\;$- $\reductions{M}{M'}{\sigma}{\sigma'}$   \\
\end{tabular}\\ 

$\bullet$ \  $\satisfies{M}{\onlyIfSingle {A_1}{A_2}{A}}$\ \ iff\ \   for all $M'$, $\sigma$,   $\sigma'$, such that $\arising{M}{M'}{\sigma}$: \\

\begin{tabular}{lr}
$\;\;\;\;$- $\satisfiesA{M}{\sigma}{A_1}$  & \rdelim\}{3}{3mm}[$\;\;\;\Rightarrow\;\;\;$  $\satisfiesA{M}{\sigma}{A}$] \\
$\;\;\;\;$- $\satisfiesA{M}{\sigma' \triangleleft \sigma}{A_2}$   \\
$\;\;\;\;$- $\reduction{M}{M'}{\sigma}{\sigma'}$   \\
\end{tabular}\\ 

%% here as it was 
%$\bullet$ \  $\satisfies{M}{\onlyThrough {A_1}{A_2}{A}}$ \ \ iff\ \  for all $M'$, $\sigma$,   $\sigma'$, such that $\arising{M}{M'}{\sigma}$, and \\
%\begin{tabular}{lr}
%$\;\;\;\;$- $\satisfiesA{M}{\sigma}{A_1}$  & 
%\rdelim\}{3}{3mm}%[\makecell{Some really \\ longer text}]
%[$\;\;\;\Rightarrow\;\;\;$\pbox{9cm}{then for all $\sigma_1, \ldots, \sigma_n$ such that $\reduction{M}{M'}{\sigma}{\sigma_1}\leadsto \ldots \sigma_n \leadsto \sigma'$
%there exists some $\sigma_i$ such that $\satisfiesA{M}{\sigma_i \triangleleft \sigma}{A}$ where $0\leq i \leq n$, or $\satisfiesA{M}{\sigma}{A}$, or $\satisfiesA{M}{\sigma' \triangleleft \sigma}{A}$}] \\
%$\;\;\;\;$- $\satisfiesA{M}{\sigma' \triangleleft \sigma}{A_2}$   \\
%$\;\;\;\;$- $\reductions{M}{M'}{\sigma}{\sigma'}$   \\
%\end{tabular}\\ 
%$\bullet$ \  $\satisfies{M}{\onlyThrough {A_1}{A_2}{A}}$ \ \ iff\ \  for all $M'$, $\sigma_1$,   $\sigma_n$, such that $\arising{M}{M'}{\sigma}$: \\
  
$\bullet$ \  $\satisfies{M}{\onlyThrough {A_1}{A_2}{A}}$ \ \ iff\ \  for all $M'$, $\sigma_1$,   $\sigma_n$, such that $\arising{M}{M'}{\sigma_1}$: \\

\begin{tabular}{lr}
$\;\;\;\;$- $\satisfiesA{M}{\sigma_1}{A_1}$  & 
\rdelim\}{3}{3mm}%[\makecell{Some really \\ longer text}]
[$\;\;\;\Rightarrow\;\;\;$\pbox{9cm}{$\forall \sigma_2, \ldots, \sigma_{n-1}$.  \\ 
(\ \ $\forall i\!\in\![1..n).\ \reduction{M}{M'}{\sigma_i}{\sigma_{i+1}}$   \ $\Rightarrow$
$\exists i\!\in\![1..n]. \  \satisfiesA{M}{\sigma_i \triangleleft \sigma_1}{A}$ \ \ )   }] \\
$\;\;\;\;$- $\satisfiesA{M}{\sigma_n\triangleleft \sigma}{A_2}$   \\
$\;\;\;\;$- $\reductions{M}{M'}{\sigma}{\sigma_n}$   \\
\end{tabular} 
\end{definition} 



 
%With our language as defined in Definition \ref{def:necessity-semantics},
We are now able to state what the necessary preconditions to critical functions in 
software are, including safety properties of software in the open world. The semantics
of \emph{Single-Step Only If} allow for the statement of such necessary preconditions
for any execution step for any program to achieve a certain outcome. The semantics
of \emph{Only If} and \emph{Only Through} allow us to raise these necessary preconditions
to any arbitrary number of execution steps, and thus allow for reasoning about 
the execution of an entire program.
 
Looking back at the example from the Introduction,   it holds that
\\
\strut $\hspace{1in}$ \prg{Mod1} $\models$ \prg{NecessityBankSpec}
 \\
\strut $\hspace{1in}$ \prg{Mod2} $\not\models$ \prg{NecessityBankSpec}
 \\
\strut $\hspace{1in}$ \prg{Mod3} $\models$ \prg{NecessityBankSpec}
 

 
For more specification examples, consider the
bank account discussed in Section \ref{s:intro}. We have already shown
how we can specify knowledge of an account's password using \prg{NecessityBankSpec},
but we are also able to write other useful properties about the bank account. 
 
\begin{lstlisting}[language = Chainmail, mathescape=true, frame=lines]
NecessityBankSpec'  $\triangleq$  from a:Account $\wedge$ a.balance==bal
                       nxt a.balance < bal
                       onlyIf $\exists$ o.[$\external{\texttt{o}}$ $\wedge$ $\calls{\prg{o}}{\prg{a}}{\prg{transfer}}{\prg{\_, \_, \_}}$]
\end{lstlisting}
 
\prg{NecessityBankSpec$^\prime$} states that if over a single step the balance of an account decreases, then it must have occurred as 
a result of a call to \prg{transfer}.
 
\begin{lstlisting}[language = Chainmail, mathescape=true, frame=lines]
NecessityBankSpec''  $\triangleq$  from a:Account $\wedge$ a.password == pwd
                        to a.password != pwd
                        onlyThrough $\exists$ o.[$\external{\texttt{o}}$ $\wedge$ $\calls{\prg{o}}{\prg{a}}{\prg{set}}{\prg{pwd, \_}}$]
\end{lstlisting}
 
\prg{NecessityBankSpec$^{\prime\prime}$} states that if over an arbitrary number of execution steps, the password of an account changes,
then it follows that there must have been some intervening execution step that was a call to \prg{set} on the account 
with the correct password. Both of these specifications are important, and are both used as intermediate steps
when we present the full proof of \prg{NecessityBankSpec} later in Section \ref{s:examples}.
\Nec thus provides us with a rich language for talking about the necessary conditions
under which critical actions within of our software are allowed to occur. 
%\jm[]{It is worth discussing the semantics of \Nec specifications and their 
%relation to typical logical consequence and Hoare logic. A classical Hoare triple, 
%$\hoare{P}{C}{Q}$, denotes that any program state that satisfies $P$, after execution 
%of program $C$, will result in a program state that satisfies $Q$. Thus, $P$ represents 
%a subset of program states that after execution of $C$ results in a program state satisfying $Q$.
%Conversely, $Q$ represents a superset of program states resulting from the execution of $C$ in 
%a program state satisfying $P$. Thus, we can soundly strengthen the left hand side ($P$), and weaken
%the right hand side ($Q$). This intuition extends to all three specifications. 
%For example, from \prg{NecessityBankSpec'}, while it is somewhat contrived, we are able to
%strengthen the ``left hand side'' by adding information, and weaken the ``right hand side'', 
%and conclude that}
%\begin{lstlisting}[language = Chainmail, mathescape=true, frame=lines]
%NecessityBankSpec'''  $\triangleq$  from a:Account $\wedge$ a.balance == bal $\wedge$ bal == x + y $\wedge$ y > 0
%                       nxt a.balance == x
%                       onlyIf $\exists$ o.[$\external{\texttt{o}}$ $\wedge$ $\access{\prg{o}}{\prg{a}}$]
%\end{lstlisting}
%\jm[]{This follows because in the above specification, (a)\prg{x < bal}, and thus \prg{a.balance = x} implies and $a.balance < bal$,
%and (b) if an object calls a method on another object, it follows that it has access to that object.
%More generally, given a Single-Step Only-If specification, 
%$\onlyIfSingle{A_1}{A_2}{A}$, $A_1$ and $A_2$ represent a subset of single step execution paths starting from a program state 
%satisfying $A_1$ and reaching a program state satisfying $A_2$, that have $A$ as a necessary precondition. 
%In the same way the converse is true, i.e. $A$ represents a superset of initial program states
%for execution starting at a state satisfying $A_1$ and reaching a state satisfying $A_2$ after a single step of execution.
%As with Hoare logic, we are able to soundly strengthen the left hand side ($A_1$ and $A_2$)
%and strengthen the right hand side ($A$). This intuition also extends to Only-If and Only-Through specifications. 
%In some places later in this paper, we use the distinction ``\emph{left hand side}'' of a \Nec specification
%to denote the two left most assertions in the specification, and ``\emph{right hand side}'' to denote
%the necessity precondition.}
 




\subsection{Assertion Encapsulation}
\jm[lemmas? does A => enc(A') imply A => enc($\neg$A')?]{}
 %In order to reason about necessary requirements in an open world,
% and those assertions that may change due 
% to computation by external, unknown code.
In Section \ref{s:outline} we used the concept of encapsulation of \SpecO assertions 
 when proving adherence to \Nec specifications.
An assertion $A$ is encapsulated by a module $M$ if it cannot be invalidated unless an
internal method is called. 
Here we refine this concept, to allow for ``conditional'' encapsulation:
$M\ \vDash A\ \Rightarrow\ \encaps{A'}$ expresses that in states which satisfy $A$, the assertion 
$A'$ cannot be invalidated, unless a method from $M$ was called.

\begin{definition}[Assertion Encapsulation]
\label{def:encapsulation}
For % an internal module. -- SDL internal is nit an inherrent property
a module $M$ and assertion $A$, we define an assertion $A'$ as being 
encapsulated, written\ \  $M\ \vDash A\ \Rightarrow\ \encaps{A'}$, \ \ if and only if
%$M\ \vDash\ \onlyIfSingle{A}{\neg A}{\calls{x}{y}{m}{\overline{z}}\ \wedge\ \external{x}\ \wedge\ \internal{y}}$
for all external modules $M'$, and program states $\sigma$ and $\sigma'$
such that $\arising{M}{M'}{\sigma}$:

\begin{tabular}{lr}
$\;\;\;\;$- $\reduction{M}{M'}{\sigma}{\sigma'}$  & \rdelim\}{4}{3mm}[$\;\;\;\Rightarrow\;\;\;$  $\exists x,\ \overline{z}. (\satisfiesA{M}{\sigma}{\calls{\_}{x}{m}{\overline{\sd{z}}}\ \wedge\ \internal{x}})$] \\
$\;\;\;\;$- $\satisfiesA{M}{\sigma}{A}$   \\
$\;\;\;\;$- $\satisfiesA{M}{\sigma}{A'}$   \\
$\;\;\;\;$- $\satisfiesA{M}{\sigma' \triangleleft \sigma}{\neg A'}$   \\
\end{tabular}
\end{definition}


\section{The meaning of Necessity}\sophia[changed the title]{}
\label{s:semantics}

In the introduction we spoke of \emph{Necessity Specifications}, e.g., $\onlyIf {A_1} {A_2} {A_3}$. 
% There are three, related, forms of necessity specifications, i.e. $\onlyIf {A_1} {A_2} {A_3}$,
% $\onlyIf {A_1} {A_2} {A_3}$, and 
% $\onlyThrough {A_1} {A_2} {A_3}$, $\onlyIf {A_1} {A_2} {A_3}$.
In this section we will define the semantics of such specifications.
In order to do that, we first define 
an underlying programming language \Loo, (Section \ref{sub:Loo}).
We then  define an assertion language, \SpecO,  which can talk about
 the contents of the state, as well about 
  of provenance and permission(Section \ref{sub:SpecO}).
Finally, we define necessity specifications in \Chainmail (Section \ref{s:holistic-guarantees}).
\sophia[added]{} 



\subsection{\Loo}
\label{sub:Loo} 
% We introduce a simple object-oriented language, \Loo, upon 
% which our specification language sits.
 \Loo is a small, simple, imperative,
class based, object oriented language. 
Given the simplicity of \Loo, we do not
define it here, \sd{instead,} we direct the reader to Appendix \ref{app:loo} for 
the full \sd{definitions}. % syntax and operational semantics.
\sd{Here we outline the definitions, and  introduce the concepts most relevant to the
treatment of the open world guarantees.}

A \Loo configuration $\sigma$ consists of a 
heap $\chi$, and a frame stack $\psi$ --
\sd{the latter is a sequence of
frames}.    
A frame $\phi$ consists of
local variable map, and \sd{a continuation}, i.e. a sequence of statements to be executed.
 A statement may \sd{assign to variables}, create new objects and push them to the heap, 
perform field reads and writes on objects,  \sd{or}
 call methods on those objects. 

%Program 
\sd{Execution} is performed in the context of a module $M$,
which is a mapping
\sd{from} class names to class definitions. 
\sd{Execution has the format}  $M, \sigma \leadsto \sigma'$, and is
unsurprising,  c.f. Appendix \ref{app:loo}.
\sd{The statements being executed are those in the continuation of the top frame.}
 % chopped, as generic 
 % There are several properties  of \Loo that are important to the central topic of this paper. 
 
\sd{As we said in section \ref{s:approach}, we are interested in guarantees which hold
when the external module is executing, and are not concerned if the internal module xtemporarily
breaks them. Therefore, we are only interested in states where the
executing object (the \prg{this}) is an internal object. 
For this, we define the  \emph{external state semantics}, of the form 
$\reduction{M_1}{M_2}{\sigma}{\sigma'}$, where $M_1$ is the external
module, and $M_2$ is the internal module, and where we
collapse all internal steps into one single step.  }

 
% SD removed, as disagreed with some of what is said below
% which allow us to 
%we define two forms of the operational semantics for \Loo, one in Fig. \ref{f:loo-semantics}
%that is unsurprising and details the execution of specific 
%statements in the language, and a second called \emph{external state semantics} 
%that models that open world, where the components of a program can
%have both known (\internalO) and unknown (\externalO) provenance.

\begin{definition}[External State Semantics]
\label{def:pair-reduce}
For  
% If we say "internal module", it is sounds as something makes the module be internal
  modules $M_1$,  $M_2$, and program configurations $\sigma$, $\sigma'$, 
we say that $\ \ \ \ \ \ \ \ \reduction{M_1}{M_2}{\sigma}{\sigma'}\ \ \ \ \ \ \ \ $ if and only if there exists a 
$n\in\mathbb{N}$, such that
\begin{itemize}
% SD changed because the old version was slightly wrong
\item
\sd{$\sigma$=$\sigma_1$, and  $\sigma'$=$\sigma_n$},
\item
$M_1 \circ M_2, \sigma_i \leadsto \sigma_{i+1}$  \ \ \ for all $i\in [0..n)$,
\item
$\class{\sigma}{\sigma.\prg{this}}, \class{\sigma'}{\sigma'.\prg{this}}\in M_2$,
\item
$\class{\sigma_i}{\sigma_i.\prg{this}} \in M_1$\ \ \ for all $i\in [1..n)$.
\end{itemize} 
%$M_1 \circ M_2, \sigma \leadsto \sigma_1 \leadsto \ldots \sigma_n \leadsto \sigma'$ and $\class{\sigma_i}{\sigma_i.\prg{this}} \in M_1$ for all $0 \leq i \leq n$
%
%$\class{\sigma}{\sigma.\prg{this}}\ \in\ M_2$ and
%\item
%$\class{\sigma'}{\sigma'.\prg{this}}\ \in\ M_2$ and 
%\end{itemize} 
%%and
%%\begin{itemize}
%%\item
%%$\exec{M_1\ \circ\ M_2}{\sigma}{\sigma'}$ or
%%\item
%$M_1 \circ M_2, \sigma \leadsto \sigma_1 \leadsto \ldots \sigma_n \leadsto \sigma'$ and $\class{\sigma_i}{\sigma_i.\prg{this}} \in M_1$ for all $0 \leq i \leq n$
%\end{itemize}
\end{definition}\sophia[changed the def slightly, as it was slightly wrong]{} 

In the  definition above % of external state semantics makes reference to a 
we \sd{use  the function
$\class{\sigma}{\alpha}$. This function looks up 
the class of the address $\alpha$ in the heap of $\sigma$.} 
  % for a specific variable in a specific program.
% SD not a variable, and no program.
We also use module linking, $M_1\circ M_2$. The operator $\circ$
 %  results in the union of two disjoint modules.
\sd{combines the two modules into one module in the obvious way, provided that their 
domains are disjoint.}
Full details in  Appendix \ref{app:loo}.


\sd{In this work we are interested in guarantees which are upheld by the internal 
module. Therefore, these guarantees  need to be satisfied only at `reachable' states,
and need not be satisfied at states that are
not reachable -- this is described formally in Definition \ref{def:all-sat}. 
Reachable states are those that  may arise by external states execution.
We describe the states of interest as the \emph{arising configurations}. }
\sophia[Here we say why we need Arising. Can you guys say more eloquenlty? Perhaps take from FASE?]{}


% SD thinks that the below is "mechanics" bu we need "intention"
% and I tried to give intention. above
% An \emph{Arising} program configuration is defined as any program configuration
% that arises from some initial program configuration through external state semantics.
\begin{definition}[Arising Program Configuration]
\label{def:arising}
For   modules $M_1$ and  $M_2$, a program configuration $\sigma$ is 
said to be an \emph{arising} configuration, formally \ \ \ $\arising{M_1}{M_2}{\sigma}$,\ \ \ 
if and only if there exists some $\sigma_0$ such that $\initial{\sigma_0}$ and
$\reductions{M_1}{M_2}{\sigma_0}{\sigma}$.
\end{definition}

% Definition \ref{def:arising} uses the definition for 
In the definition above we used \emph{Initial} configurations, 
% he definition of which can be found in  Definition \ref{def:initial}. 
\sd{which are meant to characterise configurations at the start of program execution}.
The heap of an initial configuration should  contain a single object of class \prg{Object}, and
the  stack should consists of a single frame, whose local variable map contains only the 
mapping of \prg{this} to the single object, and whose continuation may be any statement.
% to be executed.
More in Definition \ref{def:initial}. 

%Finally, we assume that there is some type system defined for \Loo that enforces 
%two encapsulation properties:
%\begin{itemize}
%\item
%Classes may be optionally annotated as \enclosed, and objects of \enclosed classes
%may not be returned as method results from non-\enclosed objects
%\item
%Ghost fields may be optionally annotated as \prg{intrnl}, meaning only references to objects 
%internal to the defining module may be used in the definition of that ghost field.
%\end{itemize}
%We do not define this type system for two reasons: (1) such a type system is fairly straightforward
%in it's definition, and largely orthogonal to the central topic of this paper, and (2) while we
%specify a type system, we are only interested in the encapsulation properties that it affords,
%and there are several other equally appropriate mechanisms able to provide such encapsulation 
%properties.


\subsection{\SpecO}
\label{sub:SpecO}
\SpecO extends expressiveness of standard specification languages
with assertion forms capturing key concepts of software security:
 \emph{permission}, \sd{\emph{provenance}}, and \emph{control}.
%That is, \SpecO specifications are able to specify which objects have
%access to which other object (\emph{permission}), whether an object's origin
%is internal or external to known code (\emph{viewpoint}), or which objects call which 
%methods (\emph{control}). 

\subsubsection{Syntax}

\begin{figure}[t]
\footnotesize
\[
\begin{syntax}
\syntaxElement{A}{\SpecO Assertion}
		{
		\syntaxline
				{e}
				{e : C}
				{\neg A}
				{A\ \wedge\ A}
				{A\ \vee\ A}
				{\all{x}{A}}
				{\ex{x}{A}}
		\endsyntaxline
		}
		{
		\syntaxline
				{\access{x}{y}}
				{\internal{x}}
				{\external{x}}
		\endsyntaxline
		}
		{
		\syntaxline
				{\calls{x}{y}{m}{\overline{z}}}
		\endsyntaxline
		}
\endSyntaxElement\\
\end{syntax}
\]
\caption{\SpecO Assertions}
\label{f:chainmail-syntax}
\end{figure}

\sophia[I, fig. \ref{f:chainmail-syntax}, I think we can drop "SpecOAssertion" since it is in the caption]{}

Fig. \ref{f:chainmail-syntax} gives the assertion syntax of the \SpecO specification language.
An assertion may be an expression, a class assertion, the usual connectives and quantifiers, along 
with several non-standard assertion forms:
\begin{itemize}
\item
\emph{Permission} ($\access{x}{y}$): % Which objects have access to which other objects (i.e.
  \sd{$x$ has access to $y$}.
\item
\jm[updated to provenance]{\emph{Provenance}} ($\internal{x}$ and $\external{y}$): %Which objects are internal or external to our component.
 \sd{$x$ is internal, and $y$ is external}.
\item
\emph{Control} ($\calls{x}{y}{m}{\overline{z}}$): 
\sd{$x$ calls method $m$ on object $y$ with arguments $\overline{z}$}.
\end{itemize}

\subsubsection{Semantics of \SpecO}
The semantics of \SpecO assertions is given in Definition \ref{def:chainmail-semantics}. 
The definition of the semantics of \SpecO makes use of several language features of 
\Loo that can be found in Appendix \ref{app:loo}. Specifically, $\eval{M}{\sigma}{e}{v}$
is the evaluation relation for expressions, and is interpreted as expression $e$ evaluates
to value $v$ in the context of program configuration $\sigma$, with module $M$. The full
semantics of expression evaluation are given in Fig. \ref{f:evaluation}. It should 
be noted that expressions in \Loo may be recursively defined, and thus evaluation may not
necessarily terminate, however the logic remains classical because recursion is restricted
to expressions, and not generally to assertions.

Further, Definition \ref{def:chainmail-semantics} uses the interpretation of variables
within a specific frame or configuration: i.e. $\interpret{\phi}{x} = v$, meaning that $x$ maps to
value $v$ in the local variable map of frame $\phi$, and $\interpret{\sigma}{x} = v$ meaning $x$ 
maps to value $v$ in the top most frame of $\sigma$'s stack.

\sophia[some spurious $\circ M'$ below -- am removing them]{}
\begin{definition}[Satisfaction of \SpecO Assertions] 
\label{def:chainmail-semantics}
We define satisfaction of an assertion $A$ by a program configuration $\sigma$ with internal module $M$ as:
\begin{itemize}
\item
$\satisfiesA{M}{\sigma}{e}$ \ \ \ iff \ \ \  $\eval{M}{\sigma}{e}{\true}$
\item
$\satisfiesA{M}{\sigma}{e : C}$ \ \ \ iff \ \ \  $\eval{M}{\sigma}{e}{\alpha}$ \textit{and} $\sigma.\prg{heap}(\alpha).\prg{class} = C$
\item
$\satisfiesA{M}{\sigma}{\neg A}$ \ \ \ iff \ \ \  ${M},{\sigma}\nvDash{A}$
\item
$\satisfiesA{M}{\sigma}{A_1\ \wedge\ A_2}$ \ \ \ iff \ \ \  $\satisfiesA{M}{\sigma}{A_1}$ and 
$\satisfiesA{M}{\sigma}{A_2}$
\item
$\satisfiesA{M}{\sigma}{A_1\ \vee\ A_2}$ \ \ \ iff \ \ \  $\satisfiesA{M}{\sigma}{A_1}$ or 
$\satisfiesA{M}{\sigma}{A_2}$
\item
$\satisfiesA{M}{\sigma}{\all{x}{A}}$ \ \ \ iff \ \ \  
\sd{$\satisfiesA{M}{\sigma[x \mapsto \alpha]}{A}$, \ 
\ \ for some $x$ fresh in $\sigma$, and for all $\alpha\!\in\!\sigma.\prg{heap}$. } 
\footnote{used to say: $\satisfiesA{M}{\sigma'}{A}$, 
where $x$ is fresh in $\sigma$ and $\sigma' = \sigma[x \mapsto \alpha]$, 
for all $\alpha \in \sigma.\prg{heap}$
\\
but this feels wrong -- things introduced in wrong order}
\item
$\satisfiesA{M}{\sigma}{\ex{x}{A}}$ \ \ \ iff \ \ \  
\sd{$\satisfiesA{M}{\sigma[x \mapsto \alpha]}{A}$, \ 
\ \ for some $x$ fresh in $\sigma$, and for some $ \alpha\!\in\!\sigma.\prg{heap}$. } 
\footnote{used to say: $\satisfiesA{M}{\sigma}{A}$ \\
where $x$ is fresh in $\sigma$ and $\sigma' = \sigma[x \mapsto \alpha]$
for some $\alpha \in \sigma.\prg{heap}$}
\item
$\satisfiesA{M}{\sigma}{\access{x}{y}}$ \ \ \ iff \ \ \  
\begin{itemize}
\item
\footnote{used to say: $\exists\ f$ such that $\interpret{\sigma}{x.f}={\interpret{\sigma}{y}}$ ... :-(}
$\interpret{\sigma}{x.f}={\interpret{\sigma}{y}}$ \sd{for some $f$}, \  or
\item
there exists some $z$, and some frame $\phi$ in the stack of $\sigma$ such that $\interpret{\phi}{x}=\interpret{\phi}{\prg{this}}$ 
% and there exists such that $\interpret{\phi}{y}=\interpret{\phi}{z}$
\end{itemize}
\item
$\satisfiesA{M}{\sigma}{\internal{x}}$ \ \ \ iff \ \ \  
$\textit{classOf}(\sigma,x) \in M$
\item
$\satisfiesA{M}{\sigma}{\external{x}}$ \ \ \ iff \ \ \  
$\textit{classOf}(\sigma,x) \not\in M$
\item
$\satisfiesA{M}{\sigma}{\calls{x}{y}{m}{z_1, \ldots, z_n}}$ \ \ \ iff \ \ \ 
\begin{itemize}
\item
$\sigma.\prg{contn} = (\_ := y'.m(z'_1,\ldots,z'_n))$, % and is superfluous, enums are ands, unless expltly stated   
\item
$\interpret{\sigma}{x} = \interpret{\sigma}{\prg{this}}$  % and
\item
$\sd{\interpret{\sigma}{y} = \interpret{\sigma}{y'}}$ % and
\item
$\interpret{\sigma}{z_i} = \interpret{\sigma}{z'_i}$ \ \ \ for all $1\!\leq i\!\leq n$
\end{itemize}
\end{itemize}
\end{definition}

\jm[]{
We also define what it means for a module to satisfy an assertion in
Definition \ref{def:mdl-sat}. That is, assertions that are satisfied
by all arising program configurations for a given \internalM module.
\begin{definition} [Assertion Satisfaction by Modules]
\label{def:mdl-sat}
For a module $M$ and assertion $A$, we say that\ \  $\satisfies{M}{A}$ \ \ if and only if 
for all modules $M'$, and all $\sigma$, if $\arising{M'}{M}{\sigma}$, then $\satisfiesA{M}{\sigma}{A}$.
\footnote{used to say: \ \ $\satisfiesA{M}{\sigma}{A}$ for all program configurations $\sigma$. \ \  
but this is wrong }
\end{definition}
Satisfaction by a module is important as it allows us to talk 
about what is true for a given module without introducing the 
details of specific program configurations, a critical component 
of constructing our Logic of Necessity in Section \ref{s:inference}.}

\jm[I have put this here, but I'm not sure it makes sense here ...]{
In order to both define the Logic of Necessity, and construct many proofs of the examples presented in this paper,
we need the ability to reason about \Chainmail 
apart from Necessity Specifications. That is, we rely on the existence of
a relation of the form $\proves{M}{A}$. Again, as with encapsulation, this does not fall in the 
domain of our logic of necessity, but is a secondary relation upon which soundness relies. Such a logic is also useful in presenting proofs 
of examples in this paper. For this reason, we elide the definition 
of such a logic, but rely on the existence of one and its soundness. Lemma \ref{lem:specX-prove-soundness}
states the soundness of such a logic.
\begin{axiom}[Soundness of \Chainmail Provability]
\label{ax:specX-prove-soundness}
For all modules $M$ and assertions $A$, $\proves{M}{A}$ then $\satisfies{M}{A}$.
\end{axiom}
}

%\begin{figure}[t]
%\begin{mathpar}
%\infer
%		{M;\ M',\ \sigma\ \vdash\ e : \prg{intrnl}}
%		{M;\ M',\ \sigma\ \vdash\ e : \prg{encap}}
%		\and
%\infer
%		{M;\ M',\ \sigma\ \vdash\ e : \prg{intrnl}}
%		{M;\ M',\ \sigma\ \vdash\ e.f : \prg{encap}}
%		\and
%\infer
%		{M;\ M',\ \sigma\ \vdash\ e : \prg{intrnl}}
%		{M;\ M',\ \sigma\ \vdash\ e.g(e') : \prg{encap}}
%\end{mathpar}
%\caption{Encapsulated Expressions}
%\label{f:intrnl}
%\end{figure}
	
%	\begin{figure}[h]
%	\[
%	\begin{array}{llr}
%	A & ::= & \textit{Assertions}\\  
%	| & e & \\
%	| & e\ :\ C & \\
%	| & e\ \in\ S & \\
%	| & A\ \prg{in}\ S & \\
%	| & \access{x}{y} \\
%	| & \internal{x} \\
%	| & \external{x} \\
%%	| & \mut x y f &\\
%%	| & \gives x y z &\\
%	| & \calls{x}{y}{m}{args} \\
%	| & \changes{S}{A} \\
%	| & \neg A & \\
%	| & A\ \wedge\ A & \\
%	| & A\ \vee\ A & \\
%	| & A\ \longrightarrow\ A & \\
%	| & \forall\ x.\ [A] & \\
%	| & \exists\ x.\ [A] & \\
%	| & \forall\ S.\ [A] & \\
%	| & \exists\ S.\ [A] &
%	\end{array}
%%	\begin{array}{llr}
%%	s & ::= & \textit{Source}\\
%%	| & \prg{int} & \\
%%	| & \prg{ext} & \\
%%	| & \_ &
%%	\end{array}
%	\]
%	\caption{Assertions}
%	\label{f:assertions_triple2}
%	\end{figure}





\subsection{\Chainmail} % \subsection{Necessity Specifications}
\label{s:holistic-guarantees}

In this Section we define syntactic forms and semantics of
\emph{Necessity Specifications}. Fig. \ref{f:holistic-syntax} 
gives the syntax.
\footnote{
Here it used tio say: "We express satisfaction of Necessity Specifications as $\satisfies{M}{H}$.
That is, a module $M$ satisfies a necessity specification $H$. This allows 
the construction of proofs without considering either the details 
of the program configuration or the external client module." But the proofs are
a separate concern than the meaning of H}
We have three forms of Necessity Assertions, described below:

\begin{figure}[t]
\footnotesize
\[
\begin{syntax}
\syntaxElement{H}{Necessity Specification}
		{
		\syntaxline
				{\onlyIf{A_1}{A_2}{A_3}}
				{\onlyThrough{A_1}{A_2}{A_3}}
%		\endsyntaxline
%		}
%		{
%		\syntaxline
				{\onlyIfSingle{A_1}{A_2}{A_3}}
		\endsyntaxline
		}
\endSyntaxElement\\
\end{syntax}
\]
\caption{Syntax of \Chainmail}
\label{f:holistic-syntax}
\end{figure}

\sophia[In Fig. \ref{f:holistic-syntax}, I think we can drop "Necessity Specification" and just put \Chainmail assertions in the caption]{}
\jm[wrt. \Chainmail vs \SpecO: I have changed most references to \SpecO, as that was what I meant when I mentioned \Chainmail. I think there may be some mis-renamings though. I will check carefully tomorrow on my read through, but be aware there may be some weird renamings.]{}

\paragraph{Only If}
[$\onlyIf{A_1}{A_2}{A}$]: If an arising program configuration starts at some state $A_1$, and reaches some state $A_2$, 
then the original program state must have also satisfied $A$.
e.g. if the balance of a bank account changes over time, then there must be some external object in the current 
program state that has access to the account's password.

\paragraph{Single-Step Only If}
[$\onlyIfSingle{A_1}{A_2}{A}$]: If an arising program configuration starts at some state $A_1$, and reaches some state $A_2$ after a single execution step, 
then the original program state must have also satisfied $A$.
e.g. if the balance of a bank account changes over a single execution step, then that execution step must be a method call to the bank \prg{transfer} method.

\paragraph{Only Through}
[$\onlyThrough{A_1}{A_2}{A}$]: If an arising program configuration starts at some $A_1$ state, and reaches some $A_2$ state, then program execution must have passed through some $A$ state.
e.g. if the balance of an account changes over time, then the bank's \prg{transfer} method must have been called 
in some intermediate state. Note 
that the intermediate state where $A$ is true might be the initial state ($\sigma_1$),
or final state ($\sigma_2$). 

\sophia[We need the motivation, definition, and explanations for the viewpoint adaptation operator, ad in
FASE]{}

\jm[]{As programs execute, the local variable 
map may change, variables may be overwritten, or the entire local variable maps may be lost on a method return.
For this reason, before we provide the semantics of Necessity Specifications, we first introduce an adaptation operator
to account for variable renaming throughout the execution of a program.
\begin{definition}
$\adapt{\sigma}{\sigma'} \triangleq (\chi, \{\prg{local} := \beta'[\overline{z}' \mapsto \beta(\overline{z})], \prg{contn}:= [\overline{z}/\overline{z'}]c\} : \psi)$
where 
\begin{itemize}
\item
$\sigma = (\chi, \{\prg{local}:=\beta, \prg{contn}:=c\} : \psi)$ and
$\sigma' = (\_, \{\prg{local}:=\beta'; \prg{contn}:=\_\} : \_)$, and
\item
$dom(\beta) = \overline{z}$, $dom(\beta') \cap \overline{z}' = \emptyset$, and $|\overline{z}| = |\overline{z}'|$
\end{itemize}
\end{definition}}
We define $M \models H$ the semantics of the Necessity Specifications in Definition \ref{def:necessity-semantics}. \sd{The definition goes by cases over the three possible syntactic forms of $H$:}


\noindent
\begin{definition}[Necessity Specifications]
\label{def:necessity-semantics}
For any assertions $A_1$, $A_2$, and $A$,  we define \\

$\bullet$ \ $\satisfies{M}{\onlyIf {A_1}{A_2}{A}}$ \ \ iff\ \  for all $M'$, $\sigma$, $\sigma'$, such that $\arising{M}{M'}{\sigma}$; \\ % and\\

\begin{tabular}{lr}
$\;\;\;\;$- $\satisfiesA{M}{\sigma}{A_1}$  & \rdelim\}{3}{3mm}[$\;\;\;\Rightarrow\;\;\;$  $\satisfiesA{M}{\sigma}{A}$] \\
$\;\;\;\;$- $\satisfiesA{M}{\sigma' \triangleleft \sigma}{A_2}$   \\
$\;\;\;\;$- $\reductions{M}{M'}{\sigma}{\sigma'}$   \\
\end{tabular}\\ 

$\bullet$ \  $\satisfies{M}{\onlyIfSingle {A_1}{A_2}{A}}$\ \ iff\ \   for all $M'$, $\sigma$,   $\sigma'$, such that $\arising{M}{M'}{\sigma}$: \\

\begin{tabular}{lr}
$\;\;\;\;$- $\satisfiesA{M}{\sigma}{A_1}$  & \rdelim\}{3}{3mm}[$\;\;\;\Rightarrow\;\;\;$  $\satisfiesA{M}{\sigma}{A}$] \\
$\;\;\;\;$- $\satisfiesA{M}{\sigma' \triangleleft \sigma}{A_2}$   \\
$\;\;\;\;$- $\reduction{M}{M'}{\sigma}{\sigma'}$   \\
\end{tabular}\\ 

%% here as it was 
%$\bullet$ \  $\satisfies{M}{\onlyThrough {A_1}{A_2}{A}}$ \ \ iff\ \  for all $M'$, $\sigma$,   $\sigma'$, such that $\arising{M}{M'}{\sigma}$, and \\
%\begin{tabular}{lr}
%$\;\;\;\;$- $\satisfiesA{M}{\sigma}{A_1}$  & 
%\rdelim\}{3}{3mm}%[\makecell{Some really \\ longer text}]
%[$\;\;\;\Rightarrow\;\;\;$\pbox{9cm}{then for all $\sigma_1, \ldots, \sigma_n$ such that $\reduction{M}{M'}{\sigma}{\sigma_1}\leadsto \ldots \sigma_n \leadsto \sigma'$
%there exists some $\sigma_i$ such that $\satisfiesA{M}{\sigma_i \triangleleft \sigma}{A}$ where $0\leq i \leq n$, or $\satisfiesA{M}{\sigma}{A}$, or $\satisfiesA{M}{\sigma' \triangleleft \sigma}{A}$}] \\
%$\;\;\;\;$- $\satisfiesA{M}{\sigma' \triangleleft \sigma}{A_2}$   \\
%$\;\;\;\;$- $\reductions{M}{M'}{\sigma}{\sigma'}$   \\
%\end{tabular}\\ 
%$\bullet$ \  $\satisfies{M}{\onlyThrough {A_1}{A_2}{A}}$ \ \ iff\ \  for all $M'$, $\sigma_1$,   $\sigma_n$, such that $\arising{M}{M'}{\sigma}$: \\
  
$\bullet$ \  $\satisfies{M}{\onlyThrough {A_1}{A_2}{A}}$ \ \ iff\ \  for all $M'$, $\sigma_1$,   $\sigma_n$, such that $\arising{M}{M'}{\sigma_1}$: \\

\begin{tabular}{lr}
$\;\;\;\;$- $\satisfiesA{M}{\sigma_1}{A_1}$  & 
\rdelim\}{3}{3mm}%[\makecell{Some really \\ longer text}]
[$\;\;\;\Rightarrow\;\;\;$\pbox{9cm}{$\forall \sigma_2, \ldots, \sigma_{n-1}$.  \\ 
(\ \ $\forall i\!\in\![1..n).\ \reduction{M}{M'}{\sigma_i}{\sigma_{i+1}}$   \ $\Rightarrow$
$\exists i\!\in\![1..n]. \  \satisfiesA{M}{\sigma_n \triangleleft \sigma_1}{A}$ \ \ )   }] \\
$\;\;\;\;$- $\satisfiesA{M}{\sigma_n\triangleleft \sigma}{A_2}$   \\
$\;\;\;\;$- $\reductions{M}{M'}{\sigma}{\sigma_n}$   \\
\end{tabular} 
\end{definition} 

%\begin{definition}[\textsc{Only If Single-Step}]
%\label{def:oi-single}
%$\satisfies{M}{\onlyIfSingle {A_1}{A_2}{A}}$ if and only if for all
%$M'$, $\sigma_1$, and $\sigma_2$, such that 
%\begin{itemize}
%\item
%$\arising{M}{\sigma_1}$,
%\item
%$\satisfiesA{M}{\sigma_1}{A_1}$,
%\item
%$\satisfiesA{M}{\sigma_2}{A_2}$, and
%\item
%$\reduction{M}{M'}{\sigma_1}{\sigma_2}$
%\end{itemize}
%then $\satisfiesA{M}{\sigma_1}{A}$
%\end{definition}
%
%\begin{definition}[\textsc{Only Through}]
%\label{def:ot}
%$\satisfies{M}{\onlyThrough {A_1}{A_2}{A}}$ if and only if for all
%$M'$, $\sigma_1$, and $\sigma_2$, such that 
%\begin{itemize}
%\item
%$\arising{M}{\sigma_1}$,
%\item
%$\satisfiesA{M}{\sigma_1}{A_1}$,
%\item
%$\satisfiesA{M}{\sigma_2}{A_2}$, and
%\item
%$\reductions{M}{M'}{\sigma_1}{\sigma_2}$
%\end{itemize}
%then there exists $\sigma,$ such that
%\begin{itemize}
%\item
%$\reductions{M}{M'}{\sigma_1}{\sigma}$,
%\item
%$\reductions{M}{M'}{\sigma}{\sigma_2}$,
%\item
%$\satisfiesA{M}{M'}{\sigma}{A}$.
%\end{itemize}
%\end{definition}

\subsection{Encapsulation}
 %In order to reason about necessary requirements in an open world,
\footnote{used to say: "we differentiate between those assertions that require computation
by internal, known code, " I do not think that is correct}
% and those assertions that may change due 
% to computation by external, unknown code.
\susan[this is problematic]{}\sd{As we saw in section 2, when the concept of encapsulation of \SpecO assertions 
in useful when proving adherence to \SpecO specifications.}
An assertion $A$ is encapsulated by a module $M$ if it cannot be invalidated unless an
internal method is called. 
\sd{Here we refine this concept, to allow for ``conditional'' encapsulation:
$M\ \vDash A\ \Rightarrow\ \encaps{A'}$ expresses that in states which satisfy $A$, the assertion 
$A'$ cannit be invalidated, unless a method from $M$ was called.}

\begin{definition}[Assertion Encapsulation]
\label{def:encapsulation}
For % an internal module. -- SDL internal is nit an inherrent property
a module $M$ and assertion $A$, we define an assertion $A'$ as being 
encapsulated, written\ \  $M\ \vDash A\ \Rightarrow\ \encaps{A'}$, \ \ if and only if
%$M\ \vDash\ \onlyIfSingle{A}{\neg A}{\calls{x}{y}{m}{\overline{z}}\ \wedge\ \external{x}\ \wedge\ \internal{y}}$
for all external modules $M'$, and program configurations $\sigma$ and $\sigma'$
such that 

\begin{tabular}{lr}
$\;\;\;\;$- $\reduction{M}{M'}{\sigma}{\sigma'}$  & \rdelim\}{4}{3mm}[$\;\;\;\Rightarrow\;\;\;$  $\exists x,\ \overline{z}. (\satisfiesA{M}{\sigma}{\calls{\_}{x}{m}{\overline{\sd{z}}}\ \wedge\ \internal{x}})$] \\
$\;\;\;\;$- $\satisfiesA{M}{\sigma}{A}$   \\
$\;\;\;\;$- $\satisfiesA{M}{\sigma}{A'}$   \\
$\;\;\;\;$- $\satisfiesA{M}{\sigma' \triangleleft \sigma}{\neg A'}$   \\
\end{tabular}
\end{definition}
\sophia[Here we should talk about soundness of an inference system fir encapsulation, and
define soudness for it. Then say that am algorithmic rudimentary system is in the appendix.]{}


\jm[]{
As we have already stated, we assume the existence of such an assertion encapsulation system, 
along with an inference system for constructing proofs of assertion encapsulation, written $\proves{M}{\givenA{A_1}{\encaps{A_2}}}$.
For the purposes of the examples presented later in the paper, we introduce a rudimentary 
encapsulation system that relies on the type system of \Loo, but the Logic of Necessity 
does not rely on the specifics of any one encapsulation system, only its soundness.
\begin{axiom}[Encapsulation Soundness]
\label{lem:encap-soundness}
For all modules $M$, and assertions $A_1$ and $A_2$, if $\proves{M}{\givenA{A_1}{\encaps{A_2}}}$ then $\proves{M}{\givenA{A_1}{\encaps{A_2}}}$.
\end{axiom}
}

\jm[I'm not sure this is enough. example?]{
We also define the $\wrapped{}$ predicate that states 
that only \internalO objects have access to some object.
That object may be either \internalO or \externalO.
\begin{definition}[Wrapped]
$\wrapped{o}\ \triangleq\ \all{x}{\neg \access{x}{o}\ \vee\ \internal{x}}$
\end{definition}
Wrapped is critical as it captures the conditions under which 
reading or writing involving an object necessitates an interaction
with the \internalM module. If for example, only \internalO
objects have access to an account's password, then
it follows that access to the password may not 
be gained except by an interaction with the \internalM
module, and subsequently if the \internalM module
is secure we know that the password may not be leaked.
}


\subsection{Expressiveness of Necessity Specifications}

\susan[]{Chainmail \cite{FASE} was guided by a study of a sequence of exemplars from the object-capability literature and the smart contracts world. We show that \Chainmail is suitable for specifying them too.}

\subsubsection{ERC20}
The ERC20 is a widely used token standard describing the basic functionality of any Ethereum-based token 
contract. This functionality includes issuing tokens, keeping track of tokens belonging to participants, and the 
transfer of tokens between participants. Tokens may only be transferred if there are sufficient tokens in the 
participant's account, and if either they or someone authorized the participant initiated the transfer. We 
specify these necessary conditions here using \Chainmail and the Logic of Necessity. Firstly, \prg{ERC20Spec1} 
says that if the balance of a participant's account is ever reduced by some amount $m$, then
that must have occurred as a result of a call to the \prg{transfer} method with amount $m$ by the participant,
or the \prg{transferFrom} method with the amount $m$ by some other participant.
\begin{lstlisting}[language = Chainmail, mathescape=true, frame=lines]
ERC20Spec1 $\triangleq$ from e : ERC20 $\wedge$ e.balance(p) = m + m' $\wedge$ m > 0
              to1 e.balance(p) = m'
              onlyIf $\exists$ p', p''.[$\calls{\prg{p}}{\prg{e}}{\prg{transfer}}{\prg{p', m}}$ $\vee$ $\calls{\prg{p''}}{\prg{e}}{\prg{transferFrom}}{\prg{p', m}}$]
\end{lstlisting}
Secondly, \prg{ERC20Spec2} specifies under what circumstances some participant \prg{p'} is authorized to 
spend \prg{m} tokens on behalf of \prg{p}: either \prg{p} approved \prg{p'}, \prg{p'} was previously authorized,
or \prg{p'} was authorized for some amount \prg{m + m'}, and spent \prg{m'}.
\begin{lstlisting}[language = Chainmail, mathescape=true, frame=lines]
ERC20Spec2 $\triangleq$ from e : ERC20 $\wedge$ p : Object $\wedge$ p' : Object $\wedge$ m : Nat
              to1 e.allowed(p, p') = m
              onlyIf $\calls{\prg{p}}{\prg{e}}{\prg{approve}}{\prg{p', m}}$ $\vee$ 
                     (e.allowed(p, p') = m $\wedge$ 
                      $\neg$ ($\calls{\prg{p'}}{\prg{e}}{\prg{transferFrom}}{\prg{p, \_}}$ $\vee$ 
                              $\calls{\prg{p}}{\prg{e}}{\prg{allowed}}{\prg{p, \_}}$)) $\vee$
                     $\exists$ p''. [e.allowed(p, p') = m + m' $\wedge$ $\calls{\prg{p'}}{\prg{e}}{\prg{transferFrom}}{\prg{p'', m'}}$]
\end{lstlisting}

\subsubsection{DAO}
The Decentralized Autonomous Organization (DAO) is a well-known Ethereum contract allowing 
participants to invest funds. The DAO famously was exploited with a re-entrancy bug in 2016, 
and lost \$50M. Here we provide specifications that would have secured the DAO against such a 
bug. \prg{DAOSpec1} says that no participant's balance may ever exceed the ether remaining 
in DAO.
\begin{lstlisting}[language = Chainmail, mathescape=true, frame=lines]
DAOSpec1 $\triangleq$ from d : DAO
            to d.balance(p) > d.ether
            onlyIf false
\end{lstlisting}
The second specification \prg{DAOSpec2} states that if a participant's balance is \prg{m}, then 
either this occurred as a result of joining the DAO with an initial investment of \prg{m}, or the
balance is \prg{0}, and they've just withdrawn their funds.
\begin{lstlisting}[language = Chainmail, mathescape=true, frame=lines]
DAOSpec2 $\triangleq$ from d : DAO
            to1 d.balance(p) = m
            onlyIf $\calls{\prg{p}}{\prg{d}}{\prg{repay}}{\prg{\_}}$ $\wedge$ m = 0 $\vee$ $\calls{\prg{p}}{\prg{d}}{\prg{join}}{\prg{m}}$ $\vee$ d.balance(p) = m
\end{lstlisting}

\subsubsection{DOM}
The Domain Object Model (DOM) is the representation of the objects comprising a web document.
The DOM has a recursive tree structure.

\prg{DOMSpec} states that if the property of a node in a DOM tree changes,
it follows that either some non-node, non-wrapper object presently has 
access to a node of the DOM tree, or to some wrapper with access to some 
ancestor of the node that was modified.
\begin{lstlisting}[language = Chainmail, mathescape=true, frame=lines]
DOMSpec $\triangleq$ from nd : Node $\wedge$ n.property = p
            to nd.property != p
            onlyIf $\exists$ o.[$\neg$ o : Node $\wedge$ $\neg$ o : Wrapper $\wedge$ 
                        ($\exists$ nd' : Node.[$\access{\prg{o}}{\prg{nd'}}$] $\vee$ 
                         $\exists$ w : Wrapper, k : $\mathbb{N}$.[$\access{\prg{o}}{\prg{w}}$ $\wedge$ nd.parnt$^{\prg{k}}$ = w.node.parnt$^{\prg{w.height}}$] )]
\end{lstlisting}


%\subsection{Expressiveness}
%\label{s:expressiveness}
%
%We discuss expressiveness of \Nec operators, by comparing 
%them with one another, with temporal operators, and with other examples from the literature.
%
%\paragraph{Relationship between Necessity Operators}
%The three \Nec \ operators
%are related by generality. 
%%An 
% \emph{Only If} ($\onlyIf{A_1}{A_2}{A}$) implies
%  \emph{Single-Step Only If} ($\onlyIfSingle{A_1}{A_2}{A}$), since if $A$ is 
%a necessary precondition for multiple steps, then it must be a necessary 
%precondition for a single step. 
% \emph{Only If} also implies 
%an \emph{Only Through}, where the intermediate state is the starting state
%of the execution.  There is no further relationship between 
%\emph{Single-Step Only If} and \emph{Only Through}.
%
%
%\paragraph{Relationship with Temporal Logic}
%Two of the three \Nec operators can be expressed in traditional
%  temporal logic: 
%  ${\onlyIf{A_1}{A_2}{A}}$
%can be expressed  %%put in to get better line breaks
% as 
% $A_1\ \wedge\ \Diamond A_2\ \longrightarrow\ A$, and
% $\onlyIfSingle{A_1}{A_2}{A}$
%can be expressed  %%put in to get better line breaks
% as $\ A_1\ \wedge\ \bigcirc A_2\ \longrightarrow\ A$
% (where $\Diamond$ denotes any future state,  and
% $\bigcirc$ denotes the next state).
% Critically, 
%$\onlyThrough{A_1}{A_2}{A}$ cannot be encoded in temporal logics
%  without ``nominals'' (explicit state references), because the state where $A$ 
% holds must be between the state where $A_1$ holds, and the state
% where $A_2$ holds; and this must be so on \emph{every} execution path
% from $A_1$ to  $A_2$ \cite{hybridLogic2021,nominal-seplogic2020}.
% TLA+, for example, cannot describe ``only through'' conditions
% \cite{tlabook}, but we have found ``only through'' conditions critical
% to our proofs. 
%
%
%% \subsection{More Examples expressed in \Nec}
%% do not say \Nec Specifications
%% because it is language that is expressive, not the specification
%
% \sdfootnote{  SD chopped the below -- some of it had moved to earlier, rest perhaps not that illuminating
%
%In this section we introduce some further specification examples, and use them to elucidate finer points
%in the semantics of \Nec. % We also  discuss which modules    satisfy  which specifications.
%
% \subsubsection{More examples of the Bank}
%Looking back at the examples from  \S\ref{s:bankSpecEx},   it holds that
%%\\
%%\strut $\hspace{.2in}$  \ModA$\vDash$ \SrobustA    $\hspace{.6in}$ \ModB$\vDash$ \SrobustA
%%  $\hspace{.6in}$ \ModC$\vDash$ \SrobustA
%%  \\
%%\strut   $\hspace{.2in}$  \ModA$\vDash$ \SrobustB    $\hspace{.6in}$ \ModB$\nvDash$ \SrobustB
%%  $\hspace{.6in}$ \ModC$\vDash$ \SrobustB
%  \\
%  $\begin{array}{llll}
%  \ \  \ \ \ \ \ & \ModA \vDash  \SrobustA    \ \ \ \ \ \ & \ModB \vDash \SrobustA \ \ \ \ \ \
%  &  \ModC \vDash \SrobustA
%  \\
% &  \ModA \vDash  \SrobustB    \ \ \ \ \ & \ModB \nvDash \SrobustB \ \ \ \ \ 
%  &  \ModC \vDash \SrobustB
%  \end{array}$
% 
%
% 
%Consider now another four \Nec specifications:
% 
%\begin{lstlisting}[language = Chainmail, mathescape=true, frame=lines]
%     $\text{\SRobustNextAcc}$   $\triangleq$  from a:Account $\wedge$ a.balance==bal  next a.balance < bal
%                        onlyIf $\exists$ o.[$\external{\texttt{o}}$ $\wedge$ $\access{\prg{o}}{\prg{a.pwd}}$]                                           
%
%     $\text{\SRobustNextCall}$  $\triangleq$  from a:Account $\wedge$ a.balance==bal  next a.balance < bal
%                        onlyIf $\exists$ o.[$\external{\texttt{o}}$ $\wedge$ $\calls{\prg{o}}{\prg{a}}{\prg{transfer}}{\prg{\_, \_, \_}}$]
%                       
%     $\text{\SRobustToCall}$   $\triangleq$  from a:Account $\wedge$ a.balance==bal to a.balance < bal
%                        onlyIf $\exists$ o.[$\external{\texttt{o}}$ $\wedge$ $\calls{\prg{o}}{\prg{a}}{\prg{transfer}}{\prg{\_, \_, \_}}$]  
%                                          
%     $\text{\SRobustThroughCall}$  $\triangleq$  from a:Account $\wedge$ a.balance==bal to a.balance < bal
%                       onlyThrough $\exists$ o.[$\external{\texttt{o}}$ $\wedge$ $\calls{\prg{o}}{\prg{a}}{\prg{transfer}}{\prg{\_, \_, \_}}$]
%
%\end{lstlisting}
%
%
%{The specification \SRobustNextAcc  states that
%the balance of an account decreases \emph{in one step}, only if an external object has access to the 
%password. It a weaker specification than \SrobustB, because it applies when the 
%decrease   takes place in \emph{one} step, rather than in \emph{a number} of steps.}
%Even though \ModB does not satisfy \SrobustB, it does satisfy \SRobustNextAcc:
% 
%  $\begin{array}{llll}
%  \   & \ModA \vDash \SRobustNextAcc  \   & \ModB \vDash \SRobustNextAcc \  
%  & \ModC \vDash \SRobustNextAcc \\
%  
%  \end{array}$
%
%\vspace{.07in} % SD thinks some space is needed here
%
%The specifications \SRobustNextCall and   \SRobustToCall are similar:
%they both say that a decrease of the balance can only happen if the current statement is a call to \prg{transfer}.  
%The former considers   a \emph{single} step, while the latter allows for \emph{any number} of steps. 
% \SrobustB is slightly different, because it  says that such a decrease is only possible if some \emph{intermediate}
% step called \prg{transfer}.
% All three   modules satisfy  \SRobustNextCall.  
%On the other hand, the code \prg{a1=new Account; a2.transfer}$(...)$ decrements the balance of \prg{a2} and
%does call \prg{transfer} but not as a first step; therefore, none of the modules satisfy 
%\SRobustToCall. That is:
%
%
% $\begin{array}{llll}
%  & \ModA \vDash \SRobustNextCall     & \ModB \vDash \SRobustNextCall   
%  & \ModC \vDash \SRobustNextCall
%  \\
%  & \ModA \nvDash \SRobustToCall     & \ModB \nvDash \SRobustToCall  
%  & \ModC \nvDash \SRobustToCall
%  \end{array}$
%  
%  \vspace{.07in} % SD thinks some space is needed here
%
% Finally, \SRobustThroughCall is a weaker requirement than \SRobustToCall, because it only asks
%  that the \prg{transfer} method is called in \emph{some intermediate} step. 
%  All modules satisfy it:
% 
% 
%   $\begin{array}{llll}
%  & \ModA \vDash\SRobustThroughCall     & \ModB \vDash \SRobustThroughCall  
%  & \ModC \vDash \SRobustThroughCall
%   \end{array}$
%
%}




\subsection{The DOM} %\sophiaPonder[renamed Wrapper to Proxy]{  
\label{ss:DOM}
%This is the motivating example in \cite{dd},
%dealing with a tree of DOM nodes: Access to a DOM node
%gives access to all its \prg{parent} and \prg{children} nodes, with the ability to
%modify the node's \prg{property} -- where  \prg{parent}, \prg{children} and \prg{property}
%are fields in class \prg{Node}. Since the top nodes of the tree
%usually contain privileged information, while the lower nodes contain
%less crucial third-party information, we must be able to limit 
% access given to third parties to only the lower part of the DOM tree. We do this through a \prg{Proxy} class, which has a field \prg{node} pointing to a \prg{Node}, and a field \prg{height}, which restricts the range of \prg{Node}s which may be modified through the use of the particular \prg{Proxy}. Namely, when you hold a \prg{Proxy}  you can modify the \prg{property} of all the descendants of the    \prg{height}-th ancestors of the \prg{node} of that particular \prg{Proxy}.  We say that
%\prg{pr} has \emph{modification-capabilities} on \prg{nd}, where \prg{pr} is
%a  \prg{Proxy} and \prg{nd} is a \prg{Node}, if the \prg{pr.height}-th  \prg{parent}
%of the node at \prg{pr.node} is an ancestor of \prg{nd}.
%%}
%
%
%The specification \prg{DOMSpec} states that the \prg{property} of a node can only change if
%some external object presently has 
%access to a node of the DOM tree, or to some \prg{Proxy} with modification-capabilties
%to the node that was modified.
\begin{lstlisting}[language = Chainmail, mathescape=true,xleftmargin=2em,frame=lines,framexleftmargin=1.5em]
DOMNecSpec $\triangleq$ from nd : Node $\wedge$ nd.property = p  to nd.property != p
          onlyIf $\exists$ o.[ $\external {\prg{o}}$ $\wedge$ 
                       $( \  \exists$ nd':Node.[ $\access{\prg{o}}{\prg{nd'}}$ ]  $\vee$ 
                         $\exists$ pr:Proxy,k:$\mathbb{N}$.[$\, \access{\prg{o}}{\prg{pr}}$ $\wedge$ 
                         nd.parent$^{\prg{k}}$=pr.node.parent$^{\prg{pr.height}}$ ] $\,$ ) $\,$ ]
\end{lstlisting}
\begin{lstlisting}[language = Chainmail, mathescape=true,xleftmargin=2em,frame=lines,framexleftmargin=1.5em]
DOMSpec1 $\triangleq$ $\forall$ nd : Node
				$\openBr$$\inside{\texttt{nd}}$}{$\inside{\texttt{nd}}$$\closeBr$
DOMSpec2 $\triangleq$ $\forall$ nd : Node, p : Object
				$\openBr$$\inside{\texttt{nd}}$ $\wedge$ nd.property = p $\wedge$ 
				 $\forall$ pr : Proxy, [nd.parent$^\texttt{k}$ =  pr.node.parent$^\texttt{pr.height}$ $\longrightarrow$ $\inside{\texttt{pr}}$]$\closeBr$
				$\openBr$nd.property = p$\closeBr$
\end{lstlisting}

\subsection{The DAO}
\label{ss:DAO}

\begin{lstlisting}[language = Chainmail, mathescape=true, frame=lines]
DAONecSpec1 $\triangleq$ from d : DAO $\wedge$ p : Object
            to d.balance(p) > d.ether
            onlyIf false
\end{lstlisting}
\begin{lstlisting}[language = Chainmail, mathescape=true, frame=lines]
DAONecSpec2 $\triangleq$ from d : DAO $\wedge$ p : Object
            next d.balance(p) = m
            onlyIf $\calls{\prg{p}}{\prg{d}}{\prg{repay}}{\prg{\_}}$ $\wedge$ m = 0 $\vee$ $\calls{\prg{p}}{\prg{d}}{\prg{join}}{\prg{m}}$ $\vee$ d.balance(p) = m
\end{lstlisting}

\begin{lstlisting}[language = Chainmail, mathescape=true, frame=lines]
DAOSpec1 $\triangleq$ $\forall$ d : DAO, p : Object.
            $\openBr$d.balance(p) > d.ether$\closeBr$
\end{lstlisting}

\subsection{Safe}
\label{ss:Safe}


\begin{lstlisting}[language = Chainmail, mathescape=true, frame=lines]
SafeNecSpec $\triangleq$ from s : Safe $\wedge$ s.treasure != null
            to s.treasure == null
            onlyIf $\neg$ inside(s.secret)
\end{lstlisting}
\begin{lstlisting}[language = Chainmail, mathescape=true, frame=lines]
SafeSpec $\triangleq$ $\forall$ s : Safe, t : Object
		$\openBr$s.treasure = t $\wedge$ $\inside{\texttt{s.secret}}$$\closeBr$
		$\openBr$s.treasure = t$\closeBr$
\end{lstlisting}


\subsection{Crowdsale}
\label{ss:Crowdsale}
\begin{lstlisting}[mathescape=true, language=chainmail]
(NecR0) $\triangleq$ e : Escrow $\wedge$ $\calls{\_}{\prg{e}}{\prg{claimRefund}}{\prg{p}}$
          next e.balance = nextBal onlyIf nextBal = e.balance - e.deposits(p)
(NecR1) $\triangleq$ e : Escrow $\wedge$ e.state $\neq$ e.SUCCESS $\longrightarrow$ sum(deposits) $\leq$ e.balance
(NecR2_1) $\triangleq$ e : Escrow $\wedge$ $\calls{\_}{\prg{e}}{\prg{withdraw}}{\prg{\_}}$
           to $\calls{\_}{\prg{e}}{\prg{claimRefund}}{\prg{\_}}$ onlyIf false
(NecR2_2) $\triangleq$ e : Escrow $\wedge$ $\calls{\_}{\prg{e}}{\prg{claimRefund}}{\prg{\_}}$
           to $\calls{\_}{\prg{e}}{\prg{withdraw}}{\prg{\_}}$ onlyIf false
(NecR3) $\triangleq$ c : Crowdsale $\wedge$ sum(deposits) $\geq$ c.escrow.goal
         to $\calls{\_}{\prg{c.escrow}}{\prg{claimRefund}}{\prg{\_}}$ onlyIf false
\end{lstlisting}


%\paragraph{More examples}
%%or {\sc{VerX}}. 
%% Nevertheless, 
%%we believe that
%%it  is powerful enough for the purpose of straightforwardly
%%expressing robustness requirements. 
%In order to investigate \Nec's expressiveness,  
%we used it for
%examples provided in the literature. 
%% In this section we considered the DOM, %  example, proposed by  \citeasnoun{dd}. 
%In \jm[]{the appendices 
%%of the full paper 
%\cite{necessityFull}} % Appendix \ref{s:expressiveness:appendix},
%we compare with examples proposed by  \citeasnoun{FASE}, and \citeasnoun{VerX}.
 



\section{Proving Adherence to \SpecLang Specifications}

\subsection{Semantics of  a Hoare Triples -- Soundness of Hoare Logic}

We will develop a  Hoare logic with judgments of the form  $M\ \vdash\  \{\, A \,  \}\ e\  \{\, A' \, \}$ which promise that execution of the expression $e$ in a configuration with satisfies $A$ will lead to a configuration that satisfied $A'$. 
%{
%\begin{definition}[Semantics of Hoare triples and quadruples]
%
%For modules $M$, and assertions $A$, $A'$ and $A''$ we define the semantics of Hoare-triples, 
% $M\ \models\  \{\, A \,  \}\ e\  \{\, A' \, \}$, and Hoare quadruples, $M\ \models\  \{\, A \,  \}\ e\  \{\, A' \, \}\, \parallel\, A''$, as follows:
%\begin{itemize}
%\item
%$M\ \models\  \{\, A \,  \}\ e\  \{\, A' \, \}\, \parallel\, A''$ \\
% iff \\
% for a all $M_{ext}$, for all $\sigma$ such that $\sigma \in Arising ...$ \\
%$M,\sigma \ \models \ A  \ \wedge\  
%\sigma.cont$=$e \ \wedge\  M\circ M_{ext}, \sigma \leadsto^* \sigma'$  
%\\
%$\Longrightarrow$ \\
%$M_{int},\sigma' \ \models \ A'' \ \wedge\  (\ \sigma'.cont$ is a value $\ \Longrightarrow\ $$M_{int},\sigma' \ \models \ A'\ ). $
%\end{itemize}
%\end{definition}
%}
%
%\subsection{Hoare Rules}
%\label{s:inference}
%
%Note that $M_{int}\ \models\  \{\, A \,  \}\ e\  \{\, A' \, \}\, \parallel\, A''$  does {\emph not} imply that $M_{int}\ \models\  \{\, A \,  \}\ e\  \{\, A' \, \}$

%In this Section we provide a proof system for constructing 
%proofs of the \SpecLang specifications defined in \S \ref{s:holistic-guarantees}.
%Such proofs consist of 
% three parts:  
%\begin{description} 
%\item[(Part 1)]
%Proving Assertion Encapsulation (\S \ref{s:encaps-proof})
%\item[(Part 2)]
%Proving that   method bodies adhere to specifications written in \AssertLang (\S \ref{s:classical-proof})
%\item[(Part 3)]
%Proving that modules adhere to \SpecLang specifications (\S \ref{s:module-proof})
%\end{description}
%
%Part 1 is, to a certain extent, orthogonal to the main aims of our work;
%in this paper we propose a simple approach based on the type system, while also acknowledging that 
%better solutions are possible.
%For Parts 2-3, we 
%TODO
% came up with the key ideas outlined in  \S \ref{s:approach}, which we
% develop in more detail in \S \ref{s:classical-proof}-\S \ref{s:emergent-proof}.}

\begin{axiom}
We assume a judgment of the form $M \vdash A$, which had the property that\\
\strut \hspace{5cm} $M \vdash A $ \ \ \ \ implies \ \ \ \ $M \vDash A$
\end{axiom}


\subsection {Assertion Encapsulation}
\label{s:encaps-proof}

{
{\SpecLang proofs  hinge on the fact that some assertions cannot be invalidated unless some 
} internal (and thus known)
computation took place. 
{We refer to this property as \emph{Assertion Encapsulation},}
}
formally $M\ \vDash  \encaps{A}$, which states that 
 assertion $A$ is encapsulated by module $M$.


\subsubsection{Semantics of Assertion Encapsulation}

{An assertion $A$  is  encapsulated by a module $M$ under condition $A'$,
if in all possible states which arise from execution of module $M$ with any other external module $M_{ext}$, and which satisfy $A'$, 
the validity of $A$} 
{ can only be changed via computations internal to that module} -- \emph{i.e.},  via a call to
a method from $M$, i.e.,
calls to objects defined in $M$ but accessible from the
outside.


\begin{definition}[Assertion Encapsulation]
\label{def:encapsulation}
An assertion $A$ is \emph{encapsulated} by module $M$ and assertion $A'$, written as
\begin{itemize}
 \item     $M\ \vDash \encaps{A}$
 \end{itemize}
if, for all external modules $M_{ext}$, and all states $\sigma$, $\sigma'$
such that $\arising{M_{ext}}{M}{\sigma}$, {and variables $\overline{x}$ and objects $\overline{o}$}:

\begin{itemize}
 \item
  $\satisfiesA{M}{\sigma}{A}$,  \ \ \ \   $\overline{x}=Free(A)$, \  \  \ \ $\overline{o}=\sigma(\overline{x})$,\ \ \ \ $\reduction{M_{ext}}{M}{\sigma}{\sigma'}$, \ \ \ ${M},{\sigma'[\overline{y}\mapsto{\overline{o}}]}\not\vDash{A}$
%\item $\overline{x}=Free(A)$, $\overline{o}=\sigma(\overline{x})$
%\item $\reduction{M_{ext}}{M}{\sigma}{\sigma'}$   
%\item ${M},{\sigma'[\overline{y}\mapsto{\overline{o}}]}\not\vDash{A}$
 \end{itemize}

implies

 \begin{itemize}
 \item   $\exists y,\,m,\, \overline{y} .[\ \sigma.\prg{cont}= y.m(\overline{y})\ \wedge\  \satisfiesA{M}{\sigma}{\internal{y}} \ ]
$
 \end{itemize}
\end{definition}

Revisiting the examples from \S~\ref{s:outline}, % we can see
both \ModB and \ModC encapsulate   the  equality of the \prg{balance} of an account to some value \prg{bal}: 
Namely, this equality can only be invalidated through calling  methods on internal objects.
 
{On the other hand, assume two further modules, $Mod_{ul}$ and $Mod_{pl}$: both these modules use ledgers to keep a map between accounts and their balances, which export functions that allow the update of this map. In $Mod_{ul}$ the ledger is \emph{not} protected, while in $Mod_{pl}$ the ledger \emph{is} protected. Then, in the former, the and the balance of an account is \emph{not} encapsulated, and in the latter it  \emph{is} encapsulated. } 
%
\\
\strut \hspace{1cm}
$\ModB\ \vDash\ \encaps{ \prg{a}:\prg{Account}\ \wedge \prg{a.balance}=\prg{bal}}$
\\
\strut \hspace{1cm}
$\ModC\ \vDash \encaps{ \prg{a}:\prg{Account}\ \wedge \prg{a.balance}=\prg{bal}}$
\\
\strut \hspace{1cm} {$Mod_{ul}\ \not\vDash \encaps{ \prg{a}:\prg{Account}\ \wedge \prg{a.balance}=\prg{bal}}$}
\\
\strut \hspace{1cm} {$Mod_{pl}\  \vDash \encaps{ \prg{a}:\prg{Account}\ \wedge \prg{a.balance}=\prg{bal}}$}

\noindent
{Note that in the above, the term \prg{a.balance} is a ghost field.}



The property that a variable is protected from another one is not encapsulated, but  the property that a variable is protected \emph{is} encapsulated, regardless of the module. Note also, that  encapsulation of an assertion does not imply encapsulation of its negation; 
 for example,  ${\inside{x}}$ is encapsulated (as per below), but $\neg  {\inside{x}}$ is not.


\begin{lemma}
For any module $M$, and variables $x$ and $y$:
\begin{enumerate} 
\item $M \vDash \encaps{\inside{x}}$
\item $M \not\vDash {\encaps{\neg\inside{x}}}$
\item $M \not\vDash {\encaps{\protectedFrom{x}{z}}}$
\item $M \vDash A \rightarrow A' \ \ \wedge\ \  M \vDash  \encaps{A}$ \ \ implies \ \ $M \vDash  \encaps{A'}$.
\end{enumerate}
\end{lemma}

\begin{proof} Sketches:

(1) because if $y$ is protected, and since the fields are all private ... the only way .. 
\\
(2) Take a state where $\neg\inside{x}$ and that there is only external object that has access to $x$, and that this object becomes no longer accessible -- eg through field override. That means that we now have  $\inside{x}$.
\\
(3) it is always possible that another external object $z'$ has unprotected access to $x$; if $z'$ has access to $z$, then it can give it access to $x$ without invoking am internal method. (3) as a counterexample, 
\\
(4) Use that $M \vDash A \rightarrow A'$ implies $M \vDash \neg A'  \rightarrow \neg  A$. BUT CAREFUL, TODO! with the potential of non-termination on assertions!!
\end{proof}
%The key consequence of soundness is that -- SD dropped; it is   not a consequence of soundness!

\subsubsection{Deriving  Assertion Encapsulation}

{In general},  code that does not contain 
calls to a {given} module is guaranteed not to invalidate any assertions encapsulated by that module.
 Assertion encapsulation has been used in proof systems to {address}   the  {frame} problem
 \cite{objInvars,encaps}. 

We  do not mandate how this property should be derived -- instead, we rely on a judgment 
$M\ \vdash  \encaps{A}$ provided by some external system. \footnote{This is simpler than the oopsla-33 setting}
Thus, \SpecLang is parametric over the derivation of the encapsulation
     judgment; in fact, several ways to do that are possible \cite{TAME2003,ownEncaps,objInvars}. For example,
 the appendices of
    \cite{necessityFull} present a 
	%Appendix~\ref{s:encap-proof} and
    %Figure~\ref{f:asrt-encap}  we present a 
    rudimentary system that is sufficient to support our example
    proof.  


%As we have already stated at the beginning of this section,
%encapsulation is a deep topic that is well studied in the literature, 
%and is not the focus of this paper. For now, we simply assume the existence 
%of a proof system for encapsulation as it is secondary to the central topic 
%of this paper. We need only assert that such an algorithmic proof system 
%must be sound (Definition \ref{lem:encap-soundness}).
%% \susan[I commented out what was there as I thought it was repetious]
%% {We are assuming the existence of a proof system for encapsulation and only need to assert that such an algorithmic proof system nust be sound.}
%% The construction of the algorithmic system is not central to our work,
%% because, as we shall see in later sections, our logic 
%% does not rely on the specifics of an encapsulation algorithm, only its soundness.

Like OOPSLA 22 Our logic does not {deal with, nor} rely on, the specifics of  how   encapsulation
{is derived}.
  % model, 
{Instead, it relies} on an encapsulation judgment and expects it to be sound:

\begin{definition}[Encapsulation Soundness]
\label{lem:encap-soundness}
A judgement of the form $\proves{M}{\encaps{A}}$  is\  \emph{sound}, \ if 
for all modules $M$, and assertions $A$:\\

$\strut \hspace{1.5cm} \proves{M}{\encaps{A}} \ \ \ \ $ implies $\ \ \ \ \satisfies{M}{\encaps{A}}$.
\end{definition}




\subsubsection{Types for Assertion Encapsulation}
\label{types}
TODO: something simple ere 

\subsection{Assertion Inference}




\subsection{Weaker/Stronger Specifications}

We assume   the existence of a function $HS$ which allows us to look up the holistic specification of a module. 
In Figure \ref{fig:si} we   define a judgment $\stronger M S  {S'}$ which expresses that the specification $S$ is stronger than the specification $S'$ under module $M$.  

\begin{figure}[hbt]
$
\begin{array}{c}
\begin{array}{lcl}
\inferrule [HS-1]
	{ \\
	}
	{\strongerI M {S_1 \wedge S_2}  {S_1}
	}
&  & 
\inferrule [HS-2]
	{ \\
	}
	{\strongerI M {S_1 \wedge S_2}  {S_2}
	}
\\
\\
\inferrule [HS-rename-1]
	{ \\\\
	\strongerI M S {\OneStateQ {\overline {x:C}} {A} }
	\\
	\overline {y} \mbox{ free in } A
	}
	{\strongerI M S {\OneStateQ {\overline {y:C}} {A[\overline y/\overline x]} }
	}
&  & 
\inferrule [HS-rename-2]
	{ \\\\
	\strongerI M S {\TwoStatesQ {\overline {x:C}} {A} {A'} }
	\\
	\overline {y} \mbox{ free in } A, A'
	}
	{\strongerI M S {\TwoStatesQ {\overline {y:C}} {A[\overline y/\overline x]} {A'[\overline y/\overline x]} }
	}

\\
\\
\inferrule [HS-3]
	{ \\ 
	M \vdash (\overline {x:C} \wedge A) \rightarrow A' }
	{\strongerI M  {\OneStateQ {\overline {x:C}} {A} }  {\OneStateQ {\overline {x:C}} {A'} } }
	&  &
\inferrule [HS-4]
	{ \\ }
	{\strongerI M {\OneStateQ {\overline {x:C}} {A} } {\TwoStatesQ {\overline {x:C}} {A} {A} }
	}
\end{array}
\\
\\	
\inferrule [HS-5]
	{ \\ 
	M \vdash ({\overline {x:C}} \wedge A_1) \rightarrow A_1' \ \hspace{.5cm} M \vdash ({\overline {x:C}} \wedge A_2') \rightarrow A_2 }
	{\strongerI M   {\TwoStatesQ {\overline {x:C}} {A_1'}{A_2'} }   {\TwoStatesQ {\overline {x:C}} {A_1}{A_2} }
	}		
\end{array}
$
\label{fig:si}
\caption{Specification Implication}
\end{figure}

\begin{lemma}
For all modules $M$, and specifications $S$ and $S'$, we have that\\
\strut \hspace{2cm} $\strongerI M  S  {S'}    \ \ \ \ \Longrightarrow\ \ \ \ \stronger M S {S'}'$
\end{lemma}

We now define what it means for a module $M$ to promise a specification $S$:

\begin{definition}
Given module $M$ snd specification $S$:

\strut \hspace{2cm} $\promises M S$ \ \ \ \  iff \ \ \ \  $\strongerI M {HS(M)} S$
\end{definition}

Notice, that $\promises M S$ is only based on the spec of $M$, and does not guarantee that indeed $M$ satisfies $S$.

%\subsection{Proving method bodies while using  \AssertLang specifications}
%\label{s:classical-proof}
% 
%We now develop a Hoare logic, which can prove assertions of the from \\
%\strut \hspace{1cm} $\hproves{M}{A}{\prg{s}}{A'}$.\\
%where \prg{s} is a statement in \Loo, and $A$ and $A'$ are assertions in \AssertLang.
%
%The challenges here are 1) that \AssertLang assertions support, on top of the classical features, also ??what-shall-we-call-them? protection features, and 2) we need to reason about calls to external modules.
%
%
%We assume that there exists some
%proof system  that   allows us to prove 
% specifications of the form  $\hproves{M}{A}{\prg{s}}{A'}$.
%{We further assume that such a proof system is sound, i.e. that 
%if xxx TODO 
%% if $\hproves{M}{\hoare{P}{\prg{res = x.m($\overline{z}$)}}{Q}}$, then 
%% for every program state $\sigma$ that satisfies $P$, the execution of the method call \prg{x.m($\overline{z}$)}
% % esults in a program state satisfying $Q$.}
% We then expand the proof rules as follows ....
 


\subsection{Reasoning about protection}
We expand that logic with rules about protection, as in Fig. \ref{f:protection}. Essentially, the only what that the "protection" of an object can decrease is if we call an eternal method, and pass it an internal object as argument. This is then covered by the rule in Fig. \ref{f:external:calls}.

\begin{figure}[hbt]
$
\begin{array}{c}
\inferrule[\sc{prot-1}]
	{ }
	{\hproves{M} 
						{\ \protectedFrom{x}{z}\ \wedge \ \internal y }
						{\ y.f=y'\ }
						{\ \protectedFrom{x}{z}\ }
	}
	\\\\

%\inferrule[\sc{prot-1}]
%	{ }
%	{\hproves{M} 
%						{\ \protectedFrom{x}{z}\ \wedge \ \internal v }
%						{\ v=v'\ }
%						{\ \protectedFrom{x}{z}\ }
%	}
%	\\\\

	\inferrule[\sc{prot-2}]
	{ }
	{\hproves{M} 
						{\ \protectedFrom{x}{z}\  \wedge\ \internal {y'} \ \wedge\  x \neq y'}
						{\ y.f=y'\ }
						{\ \protectedFrom{x}{z}\ }
	}
	\\\\

	\inferrule[\sc{prot-3}]
	{ }
	{\hproves{M} 
						{\ \protectedFrom{x}{z}\ \wedge\  \protectedFrom{x}{y'} }
						{\ y.f=y'\ }
						{\ \protectedFrom{x}{z}\ }
	}
	\\\\

%	\inferrule[\sc{prot-2}]
%	{ }
%	{\hproves{M} 
%						{\ \protectedFrom{x}{z}\ \wedge\ \external v\ \wedge\  z\neq v }
%						{\ v=v'\ }
%						{\ \protectedFrom{x}{z}\ }
%	}
%	\\\\


%	\inferrule[\sc{prot-4}]
%	{ }
%	{\hproves{M} 
%						{\ \protectedFrom{x}{v}\ \wedge\ \external z\  }
%						{\ z=v\ }
%						{\ \protectedFrom{x}{z}\ }
%	}
%	\\\\

	\inferrule[\sc{prot-4}]
	{ }
	{\hproves{M} 
						{\ \protectedFrom{x}{z}\ \wedge\ z \neq \this}
						{\ y =y'.f\ }
						{\ \protectedFrom{x}{z}\ }
	}
	\\\\

%\inferrule[\textsc{prot-5}]
%	{}
%	{\hproves{M}
%			{}
%	}
\end{array}
$
\caption{Protection Logic}
\label{f:protection}
\end{figure}

TO-DISCUSS 1)Should f it be $\in HS(M)$ or something more general. ie can it be implied from .. ? Premise of  \textsc{prot-3} may be too strong!
2) What about the universally quantified $\overline x$? 3)  Note that $y'$ and $y$ talk of a different variable.
 

Explanations: \textsc{xxxl} states that   yyy
  
 
\subsection{Reasoning about calls}

We now show how to reason about external calls

\begin{figure}[hbt]
\begin{mathpar}
% the below are speacil cases of the last one
%\inferrule[\sc{ExtCall-1}]
%{ 
%	      \\\\
%		M \vdash A\ \rightarrow \  ( \  \external \{ z, z', z''\}\ \wedge\ \internal x \ \wedge\  \ y:C \ \wedge \ 
%  x\neq y \
%  \wedge\ \protectedFrom {y} {\{z,z',z''\} }\ )
%	\\\\ 
%	\promises M   {\TwoStatesQ {\overline {y:C}} {\inside y}{\inside y}} 
%		}
%	{\hproves{M} 
%						{ A }
%						{ \ z.m(z',x)\  }
%						{ \protectedFrom {y} {\{z,z',z''\} } }
%	}\\ \\
%\inferrule[\sc{ExtCall-2}]
%	{ 
%	      \\\\
%		M \vdash A\ \rightarrow \  ( \  \external \{ z, z', z''\}\ \wedge\ \internal x \ \wedge\  \ y:C \ \wedge \   x\neq y.f \
%  \wedge\ \protectedFrom {y.f} {\{z,z',z''\} }\ )
%		\\\\ 
%	\promises M   {\TwoStatesQ {\overline {y:C}} {\inside y.f}{\inside y.f}} 
%		}
%	{\hproves{M} 
%						{ A }
%						{ \ z.m(z',x)\  }
%						{ \protectedFrom {y.f} {\{z,z',z''\} } }
%	}
%\ \\
\inferruleSD{[\sc{ExtCall}]}
	{ 
		 M \vdash A\ \rightarrow \ \red{ {\external{z}}  }
		  \\
   	\promises M   {\TwoStatesQ {\overline {x:C}} {A_1}{A_2}}
           \\
		M \vdash {\lift A  {\{z,\overline {u}\}}  {\overline y}} \  \rightarrow \ (\  \red{ {\overline {x:C}}\ {\wedge\ A_1}} \  )	 
		}
	{   \hproves{M} 
						{ \ A\  }
						{ \ z.m(\overline u)\  }
						{ \  \llower {A_2}{(z,\overline y)} \ \wedge\ { \preserve  A  {(z,\overline y)} M }  \ }	
}
\end{mathpar}
\caption{Internal and External Calls Logic}
\label{f:external:calls}
\end{figure}

$\begin{array}{lcll}
\lift {v=v'} {\overline  y}  & = & v=v' 
\\
\lift {x.f=v} {\overline y} & = & x.f=v 
\\
\lift{ \inside x}  {\overline  y}  & = &   \inside x 
\\
\lift{\protectedFrom x {\overline {u}} }  {\overline  y}  & = &   \inside x & \mbox{if } { x\not\in \overline{y}}
\\
   & = &   \prg{true} & \mbox{otherwise}

\\
\lift {A_1 \wedge A_2} {\overline   y}  & = & \lift { A_1} {\overline y}    \ \wedge \lift {A_2} {\overline y}  
\\
\lift {\neg A} {\overline z} {\overline y}  & = & \neg (\lift {A} {\overline y}  )& \mbox{if  $A$ is protection-free}
\\
  & = &  \prg{true} & \mbox{otherwise}
\\
\lift {\forall \overline{x:C}.[ A ]} {\overline   y}  & = & \forall \overline{x:C}.[ \lift A  {\overline y}  ] & 
\\
\lift {\exists \overline{x:C}.[ A ]} {\overline  y}  & = & \exists \overline{x:C}.[ \lift A  {\overline y}   ] & 
\end{array}
$

$\begin{array}{lcll}
\\
\\
\llower {v=v'} {\overline   y}   & = & v=v' 
\\
\llower {x.f=v} {\overline   y}  & = & x.f=v 
\\
\llower{ \inside x}  {\overline   y}  & = &   \prg{true}
\\
\llower{ \protectedFrom x {\overline {u}} }  {\overline   y}  & = &     \protectedFrom x {\overline {u}}  
\\
\llower {A_1 \wedge A_2} {\overline   y}  & = & \llower { A_1} {\overline   y} \ \wedge\ \llower {A_2}   {\overline   y} 
\\
\llower {\neg A}  {\overline   y}  & = & \neg (\llower  {A} {\overline   y} ) & \mbox{if  $A$ is protection-free}
\\
  & = &  \prg{true} & \mbox{otherwise}
\\
\llower {\forall \overline{x:C}.[ A ]} {\overline   y}   & = & \forall \overline{x:C}.[ \llower A   {\overline   y} ] & 
\\
\llower {\exists \overline{x:C}.[ A ]} {\overline   y}  & = & \exists \overline{x:C}.[ \llower A  {\overline   y} ] & 
\end{array}
$

{\small{
$\begin{array}{lcll}
\\
\\
\preserve {v=v'} {\overline  y} M  & = &  v=v' 
\\
\preserve {x.f=v} {\overline  y} M & = & x.f=v  & \mbox{if} \ \ \promises M   {\TwoStatesQ {\overline {x':C},x:D} {x.f = v \wedge A_1}{x.f=v}} 
\\ 
& & & \mbox{and} \ \ M \vdash {\lift A  {\overline   y}} \  \rightarrow \ (\, {\overline {x:D}} \wedge A_1\, )
\\
   & = &   \prg{true} & \mbox{otherwise}
   \\
\preserve  { \inside x}  {\overline   y}  M & = &   \inside x 
\\
\preserve  { \protectedFrom x {\overline {u}} }  {\overline z} {\overline y} M  & = &   { \protectedFrom x {\overline {u}} }  & \mbox{if} \ \ \promises M   {\TwoStatesQ {\overline {x':C},x:D} {x.f = v \wedge A_1}{x.f=v}} 
\\ 
& & & \mbox{and} \ \ M \vdash {\lift A  {\overline  y}} \  \rightarrow \ (\, {\overline {x:D}} \wedge A_1\, )
\\
   & = &   \prg{true} & \mbox{otherwise}

\\
\preserve  {A_1 \wedge A_2} {\overline  y}  M & = & \preserve { A_1}  {\overline y}  M \ \ \wedge \\
& &  \preserve { A_2}  {\overline   y}  M
\\
\preserve  {\neg A}{\overline  y}  M   & = & \neg (\preserve {A} {\overline   y}  M)& \mbox{if  $A$ is protection-free}
\\
  & = &  \prg{true} & \mbox{otherwise}
\\
\preserve {\forall \overline{x:C}.[ A ]} {\overline   y}  M   & = & \forall \overline{x:C}.[ \preserve A  {\overline  y}  M ] & 
\\
\preserve {\exists \overline{x:C}.[ A ]} {\overline  y}  M   & = & \exists \overline{x:C}.[ \preserve A {\overline   y}  M  ] & 
\end{array}
$
}}

\subsection{Proving \SpecLang Specifications}

\subsubsection{Deriving sub-specifications}


\label{s:module-proof}

\begin{figure}[thb]
%\footnotesize
$
\begin{array}{c}
\inferrule [Two-State]
	{
	\\\\
	M \vdash \encaps{\overline {x:C}\, \wedge \, A}
	\\\\
	\textit{for all}\ \  \textit{public methods  from } D,\ \textit{with}\ \prg{mBody}(m,D,M)=\overline{y:D}\{\  s \ \}\\\\
				% \strut \hspace{3cm}
				\ \  {\hproves{M}{ \overline{x:C}\ \wedge \ A\ \wedge \ \prg{this}:\prg{D} \wedge\ \overline{y:D}  } {\ s\ } {\ A\ }} \ \parallel \  A 
	}
	{
	M\ \vdash\ {\TwoStatesQ {\overline {x:C}} {A} {A} }
	}
\\\\
\inferrule [External-Safe]
	{
		A_{ois}=\overline{\OneStateQ{\overline {y:C}}{A''}}\ \mbox{all object invariants in } HS(M)\ \ \ \ \ \ \ \  A_{strng}=A_{ois}\wedge A' 
		\\\\
 		 \hproves{M} {A_{strng} } {\ s\ } {\ A\ } \
 		\\\\ 	
				\forall  s', z, m.[\ \ 
				 (\  s = s'; z.m(\_); \_\ \wedge \ 
				  \hproves{M} {A_{strng}\ } {\ s'\ } {\  \external{z}\  }  \ \ \
				  \Longrightarrow\ \ \ \hproves{M}{A_{strng} } {\ s'\ } {\ A\ } \ \ ]
	}
	{
	{\hproves{M}{ A'}   {\ s\ } {\ A\ } }\  \parallel \  A   
	}
\\\\ 
\inferrule [One-State]
	{
 	M \vdash \encaps{\overline {x:C}\, \wedge \, A}
 	\\\\
 	\forall \mbox{ public } D. [\ \hproves{M}{\forall \overline{x:C}.[A]} {\ y=\prg{new}\ D\ } {\ \forall \overline{x:C}.[A]\ }	\ ]
   \\\\
 	\textit{for all}\ \  \textit{public methods  from } D,\ \textit{ with } \prg{mBody}(m,D,M)=\overline{y:D}\{\  s \ \}\\\\
%				% \strut \hspace{3cm}
 			\ \  {\hproves{M}{ \forall \overline{x:C}.[A]\ \wedge \ \prg{this}:\prg{D} \wedge\ \overline{y:D}  } {\ s\ } {\ \forall \overline{x:C}.[A]\ }} \ \parallel \  {\forall \overline{x:C}.[A]} 
	}
	{
	M\ \vdash\ \OneStateQ{\overline {x:C}}{A}
	}
\\\\
\begin{array}{lcl}
\inferrule[Weaken]
{
M \vdash S \\ \strongerI M S {S'}
}
{
M \vdash S'
}
&\ \ \  &
\inferrule[Multi]
	{
	M\ \vdash\ S 
	\\
	M\ \vdash\ S' 
	}
	{
	M\ \vdash\ S \wedge S'
	}
\end{array}

\end{array}
$
\caption{Inferring that module satisfies its specification}
\label{f:module:invariats}
\end{figure}

TODO: Does the consequence rule require that the assertions are encapsulated? And if an assertion is encapsulated, is its consequence also encapsulated?

The rules also require that the variables in the quatifiers do not appear in the bodies, and are disjoint from the parameters.
TODO explain. Also, we only look at the methods exported from the module.  Also, we ned to add some dynamic type checking to the language, ie the method call crashes if actual params do not fir the formal types. OR we type them all as \prg{Object}.

TODO: shall we drop one-state invariants? Do not know how to prove them here. But they are in the spirit of capabilities literature.

%\subsection{Soundness of the \SpecLang Logic}
%
% 
%\label{s:soundness}
%
%We will now prove soundness of the  \SpecLang Logic. For this, we will first prove soundness of our extended Hoare logic.
%
%
%\begin{lemma}
%Assuming a sound \SpecO proof system, $\proves{M}{A}$, and  and
%a sound encapsulation inference system, $\proves{M}{\encaps{A}}$. Then:
%\begin{itemize}
%\item
%The inference system  $M\ \vdash\  \{\, A \,  \}\ e\  \{\, A' \, \}$  defined in the previous section is sound.
%\end{itemize}
%\end{lemma}
%
%\begin{proof}
%Take arbitrary modules  $M$, $M'$, expression $e$,  assertions $A$, $A'$ and $A''$ and assume
%\begin{enumerate}
%\item
% $M\ \vdash\  \{\, A \,  \}\ e\  \{\, A' \, \}$ 
% \item
% $M,\sigma \ \models \ A$
% \item
%$ \sigma.cont$=$e$ 
%\item
%$M\circ M', \sigma \leadsto^* \sigma' \ \ \wedge\ \ \sigma'.cont$ is a value
%\end{enumerate}
%We want to show that
%\begin{enumerate}
%\item
%$M,\sigma' \ \models \ A' $
%\end{enumerate}
%The proof proceeds by induction over a lexicographic ordering over the tuples $(M, A, e, A', \sigma, \sigma')$ This ordering is the tuple of ($m_{cl}$, $m_{ext}$), where 
%$m_{cl}$ is the length of the maximal sequence of proof steps in "classical Hoare logic, ie excluding a step {\sc{ExtCall}} involved in proving that  ie excluding a step {\sc{ExtCall}}, and the $m_{ext}$ is the number of external calls that occurred ... {TODO: this needs to be refined!}
%
%\end{proof}

For the proof of soundness we will use the following two lemmas that give guarantees about preservation of properties when pushing new frames onto the stack, and when popping frames from the stack. We use the notation $\sigma \bullet \phi$ to indicate that the frame $\phi$ has been pushed on top of the ...\footnote{{these lemmas could also appear earlier ... or later}}

 
\begin{lemma}
For any module $M$, assertion $A$, variables $\overline x$, and variables $\overline x$,  % values $\overline v$, 
and any continuation, $cont$, and any states $\sigma$ and $\sigma'$ where 
$\sigma=(\psi,h)$, and $\sigma'=(((\overline {z \mapsto \sigma(x)}), cont)\cdot \psi, h)$,  
we have

\begin{itemize}
\item
$M,\sigma\ \models\  A$  \ \ implies \ \  $M,\sigma' \ \models\  \lift {A} {\overline x}   $
\item
$M,\sigma'\ \models\  A$  \ \ implies \ \  $M,\sigma  \ \models\  \llower  {A}  {\overline   x} $ 
\end{itemize}
\end{lemma}
\begin{proof}
in the next section
% by induction on the derivation of $\proves{M}{S}$.
\end{proof}


 


The specification of  a \prg{Purse} appears in Fig. \ref{fig:PurseSpec}. This specification is at level 1, and therefore pre-conditions for not mention $\obeys x {Purse}$ but postconditions do.
The specification of  an  \prg{Escrow} appears in Fig. \ref{fig:EscrowSpec}. This specifications is at level 2, and therefore both pre- and post-conditions may mention  $\obeys x {Purse}$ .

\newcommand{\sMT}{sellerM} 
\newcommand{\bMT}{buyerM}
\newcommand{\sGT}{sellerG} 
\newcommand{\bGT}{buyerG}
\newcommand{\sM}{\prg{\sMT}}
\newcommand{\bM}{\prg{\bMT}}
\newcommand{\sG}{\prg{\sGT}}
\newcommand{\bG}{\prg{\sGT}}


\subsection{Purse}

The \prg{Purse} has a ghost field (or abstract predicate), \prg{SameMint(\_)}, which guarantees that the receiver and argument belong to the same \prg{Mint}.  It also has another ghost field, which is the balance of the account.

\textbf{To discuss} \begin{enumerate}
\item
implicit obeys for the receiver
\item
\prg{SameMint} definition
\item
\prg{SameMint} role
\end{enumerate}
%   fields sellerMoney, sellerGoods, buyerMoney, buyerGoods //  Purse
%   fields price, amt   // Number
% had to drop this one

\begin{figure*}[t]
\begin{lstlisting}[mathescape=true, language=Chainmail, frame=lines]
$\textbf{specification}$ $Purse$ {
    
    $\textbf{ghost}$ balance:int
    $\textbf{ghost}$ SameMint(x):bool  
         
    $\textbf{scoped-invr}\  \forall b:nat.[\ \inside{\prg{this}}  \ \wedge \prg{this}.\prg{balance}\geq b \ ]$
    $\textbf{invr}\   \forall p:\prg{Object}[\ \ \prg{this.SameMint(p)}\ \ ]$ // enough?
    $\textbf{scoped-invr}\   \forall p:\prg{Object}.,\forall b:bool.[\ \ \neg \prg{this.SameMint(p)}\ \ ]$ // enough?
    $\textbf{invr}\  \prg{this}.\prg{balance}\geq 0$       
    $\textbf{invr}\  \forall p:\prg{Object}.[\ \ \prg{this.SameMint(p)} \ \longrightarrow \  \obeys {\prg{p}} {Purse}\ \ ]$
     
    true  // implicit $\obeys {\prg{this}} {Purse}$
         $\textbf{\{}$ this.transfer(p,amt) $\textbf\}$ : bool
    res $\wedge$  this.SameMint(p) // implicit $\obeys {\prg{p}} {Purse}$
    $\vee$
    $\neg$res $\wedge$ [ this==p $\vee$ this.balance$\leq$amt $\vee$ $\neg(\, \prg{this.SameMint(p)}\,)$ ]

    this$\neq$p $\wedge$ this.balance=bm$\geq$amt  $\wedge$   this.SameMint(p)  
     // implicit $\obeys {\prg{this}} {Purse} \ \wedge\ \    \obeys {\prg{p}} {Purse}$
         $\textbf{\{}$ this.transfer(p,amt) $\textbf\}$ : bool
    res $\wedge$ this.balance=balT-amt $\wedge$ p.balance=balP+amt 

}

\end{lstlisting}
\caption{Specification of  $Purse$ }
\label{fig:PurseSpec}
 \end{figure*}
 
 NOTE: Julian proposed to replace lines 7 and 8 above with   \\
 $\ \ \  $ $\textbf{scoped-invr}\   \forall p:\prg{Object}.,\forall b:bool.[\ \ \neg \prg{this.SameMint(p)}\ \ ]$ 
But this would require us to ecplain that $b$ is a value and not a variable. HMHHHHH Is this also a problem in the OOPSLA-25 paper!!!

A possible implementation of \prg{Purse} appears below. The Purses have a field storing therr Mint and their balance

\begin{lstlisting}[mathescape=true, language=Chainmail, frame=lines]
$\textbf{class}$ $PurseImpl\_a\ \textbf{implements} Purse$ {
    
    $\textbf{field}$  myMint: Mint
    $\textbf{field}$  myBalance: int
     
    $\textbf{ghost}$ SameMint(x) $\textbf{is}$ this.myMint = x.myMint
    $\textbf{ghost}$ balance $\textbf{is}$ this.myBalance
    
    method transfer(p:Object,amt:nat)  // external
    	if p.myMint == this.myMint and this.blance>= amt 
	then
		this.balance -= amt
		p.balance += amt
		// will throw exception if p is not from class \prg{Purse}
		return true
	else
		return false	
  
}
\end{lstlisting}

A anotherimplementation of \prg{Purse} appears below. The Purses have a field storing their Mint, but their balances are stored in a 
lookup table in the Mint

\begin{lstlisting}[mathescape=true, language=Chainmail, frame=lines]
 $\textbf{class}$ Mint{

	field myPurses ... a list of Purse
	field myBalances .., a map from Purse to int
	
	method inMint(p:Purse) : bool    //  internal
	...	
	method getBalance{p:Purse): int
	...
	method setBalance(p:Purse,amt:int): void
	...	
}
 $\textbf{class}$Purse{
    
    $\textbf{field}$  myMint: Mint
     
    $\textbf{ghost}$ SameMint(x) $\textbf{is}$ this.myMint.inMint(x)
    $\textbf{ghost}$ balance $\textbf{is}$ this.myMint = x.myMint
    
    method transfer(p:Object,amt:nat) // external 
    	if myMint.inMint(p) and myMint.getBalance(this)>= amt 
	then
		myMint.setBalance(this,...)
		myMint.setBalance(p,...)
		return true
	else
		return false	
  
}
\end{lstlisting}



\begin{figure*}[t]
\begin{lstlisting}[mathescape=true, language=Chainmail, frame=lines]
$\textbf{specification}$ $Escrow$ {
    
    //   1$^{st}$ case:
    $\obeys  {\{\sM,\sG\}} {Purse}$ $\wedge$ 
    $\sM$.SameMint($\bM$) $\wedge$  $\sG$.SameMint($\bG$)  $\wedge$ price, amt:$\mathbb{N}$  $\wedge$
    $\bM$.balance=bm$\geq$price  $\wedge$  $\sM$.balance=sG$\geq$amt $\wedge$ ...
    $\obeys  {p} {Purse}$   $\wedge$ p.balance = pM
           $\textbf{\{}$ this.deal($\sM,\, \bM,\, \sG,\, \bG$, price, amt) $\textbf\}$
    res $\wedge$
    $\bM$.balance=bM-price $\wedge$ $\sM$.balance=....  $\wedge$
    p.balance=pM

     //   2$^{nd}$ case:
    $\obeys  {\sM } {Purse}$ $\wedge$ $\neg(\obeys  { \bM} {Purse})$ $\wedge$ 
    ....
    $\obeys  {p} {Purse}$ $\wedge$ $\protectedFrom {p} {\{ \bM, \sG, \bG \}}$  $\wedge$ p.balance = pM
          $\textbf{\{}$ this.deal($\sM,\, \bM,\, \sG,\, \bG$, price, amt) $\textbf\}$
    $\neg$ res $\wedge$
    .... $\wedge$
    p.balance=pM
     
}

\end{lstlisting}
\caption{Specification of  $Escrow$.\prg{::deal} -- Incomplete}
\label{fig:EscrowSpec}
 \end{figure*}

%\section{Discussion}

\paragraph{Behavioural Specification Languages} 

Hatcliff et al.\ \cite{behavSurvey2012} provide an excellent survey of
contemporary specification approaches.  With a lineage back to Hoare
logic \cite{Hoare69}, Meyer's Design by Contract \cite{Meyer97} was the
first popular attempt to bring verification techniques to
object-oriented programs as a ``whole cloth'' language design in
Eiffel.  Several more recent specification languages are now making
their way into practical and educational use, including JML
\cite{Leavens-etal07}, Spec$\sharp$ \cite{BarLeiSch05}, Dafny
\cite{dafny} and Whiley \cite{whiley15}. Our approach builds upon
these fundamentals, particularly Leino \& Shulte's
%\kjx{and Naumann's} 
formulation of
two-state invariants \cite{usingHistory}, and Summers and
Drossopoulou's Considerate Reasoning \cite{Considerate}.
%
In general, these approaches assume a closed system, where modules
can be trusted to co{\"o}perate. In this paper we aim to
% illustrate the kinds of techniques required
work
in an open system where modules'
invariants must be protected irrespective of the behaviour of the rest
of the system.

%% \sd{\Chainmail assertions are} guarantees upheld throughout program execution. 
%% Other systems which give such ``permanent'' guarantees are  type systems, 
%% which ensure that well-formed programs  always produce well-formed runtime
%% configurations, or information flow control systems \cite{infoflow}, which ensure that values 
%% classified as high  will not be passed into contexts classified as low. 
%% Such  guarantees %made by types or information flow control
%%  are  practical to check, but   too coarse grained
%% for the purpose of fine-grained,  module-specific specifications. 


%% \Chainmail\ specifications can cross-cut the code they are
%% specifying; \sd{therefore,} they are related to
%% aspect-oriented specification
%% languages such as AspectJML \cite{AspectJML} and AspectLTL
%% \cite{AspectLTL}.
%% %
%% AspectJML is an aspect-oriented extension to JML;
%%  in much the same way that AspectJ is an aspect-oriented extension to
%% Java \cite{AspectJ}.  AspectJML offers AspectJ-style pointcuts 
%% that allow the definition of crosscutting specifications, such as 
%% shared pre- or post-conditions for a range of method calls. 
%% % SD removed the below, as I do not understand it.
%% % These crosscutting specifications can be checked dynamically along with
%% % traditional object-oriented JML assertions. In contrast, \Chainmail\
%% %specifications naturally cross-cut implementation and specification
%% %modules without any special notation, although, lacking wildcards,
%% %\Chainmail\ is not as flexible as AspectJML. 
%% % % SD removed the below, because I do not think it is important
%% %To our knowledge, the
%% %semantics of AspectJML have yet to be defined formally, although
%% %earlier work by Molderez and Janssens describes the formal core of a
%% %similar language \cite{DbCAspectJ}.

%% AspectLTL \cite{AspectLTL} is a specification language based on Linear
%% Temporal Logic (LTL). \sd{It} %AspectLTL 
%% adds cross-cutting aspects to more
%% traditional LTL module specifications: these aspects can further
%% constrain specifications in modules. In that sense, AspectLTL and
%% \Chainmail\ %both 
%% \sd{use} similar implicit join point models, rather than
%% importing AspectJ style explicit pointcuts as in AspectJML.
%% %% % SD removed the below, because I do not think it is important
%% %  AspectLTL
%% %has a formal definition, as does \Chainmail; unlike \Chainmail,
%% %AspectLTL has support for automated reasoning with an efficient
%% %synthesis algorithm.

%% % \paragraph{Concurrent Reasoning} Deny-Guarantee \cite{DenyGuarantee}
%% % distinguishes between assertions guaranteed by a thread, and actions
%% % denied to all other threads. Deny properties correspond to our
%% % requirements that certain properties be preserved by all code linked
%% % to the current module. Compared with our work, deny-guarantee assumes
%% % co{\"o}peration: composition is legal only if  threads adhere  to
%% % their deny properties. In our work, a module has to be robust  and
%% % ensure that these properties cannot be affected by  other code. 


%% %Finally, 
%% \sd{Our} work is also related to the causal obligations in Helm et
%% al.'s behavioural contracts \cite{helm90}. Causal obligations allow
%% programmers to specify e.g.\ that whenever one object receives a
%% message (such as a subject in the Observer pattern having its value
%% changed) that object must send particular messages off to other objects
%% (e.g.\ the subject must notify its observers). \Chainmail's control
%% %SD: not "control flow"
%%  operator % (`$\Calls{\_} {\_} {\_} {\_} $) 
%%  %allows  programmers to make
%%  \sd{supports}  similar specifications, (e.g. 
%%  ${\Calls{\_}  {\prg{setValue}} {\prg{s}} {\prg{v}}}  \rightarrow \Future{\Calls{\prg{s}}{\prg{notify}}{\prg{s.observer}}{\prg{v}}}$ --- when a subject receives a \prg{setValue} method,
%%   it must ``forward'' those messages to the observer.

\paragraph{Defensive Consistency}

%cute but wrong.
%To misparaphrase Tolstoy, secure systems are all alike;
%every insecure system is insecure in its own way
%\cite{WikipediaAnnaKareninaPrinciple}.

In an open world, we cannot rely on the kindness of strangers: rather
we have to ensure our code is correct regardless of whether it
interacts with friends or foes.
Attackers 
\textit{``only have to be lucky once''} while secure systems 
\textit{``have to be lucky always''} \cite{IRAThatcher}.
% SD 
Miller \cite{miller-esop2013,MillerPhD} defines the necessary approach
as \textbf{defensive consistency}: \textit{``An object is defensively
  consistent when it can defend its own invariants and provide correct
  service to its well behaved clients, despite arbitrary or malicious
  misbehaviour by its other clients.''}  Defensively consistent
modules are particularly hard to design, to write, to understand, and
to verify: but
% they have the great advantage that
they make it much
easier to make guarantees about systems composed of multiple components
\cite{Murray10dphil}.


\paragraph{Object Capabilities and Sandboxes.}
{{\em Capabilities} as a means to support the development of concurrent and distributed system  were developed in the 60's
by Dennis and Van Horn \cite{Dennis66}, and were adapted to the
programming languages setting in the 70's \cite{JamesMorris}. 
{\em Object capabilities} were first introduced~\cite{MillerPhD} in the early 2000s},
 and many recent % work attempts to manage
studies manage
to verify  safety or correctness of object capability programs.
Google's Caja \cite{Caja} applies   sandboxes, proxies, and wrappers
 to limit components'
access to \textit{ambient} authority.
% --- that is, capabilities that
%can be obtained from the wider environment, rather than being granted
%to a component explicitly.
Sandboxing has been validated
formally: Maffeis et al.\ \cite{mmt-oakland10} develop a model of
JavaScript, demonstrate that it obeys two principles of
object capability systems
%  (``connectivity begets connectivity'' and
%``no authority amplification''), and then % uses these principles to
and show  how untrusted applications can be prevented from interfering with
the rest of the system. 
Recent programming languages % and web systems
\cite{CapJavaHayesAPLAS17,CapNetSocc17Eide,DOCaT14} including Newspeak
\cite{newspeak17}, Dart \cite{dart15}, Grace \cite{grace,graceClasses}
and Wyvern \cite{wyverncapabilities} have adopted the object
capability model.

%% \paragraph{Verification of Dynamic Languages}
%% A few formal verification frameworks  address JavaScript's highly
%% dynamic, prototype-based semantics. Gardner et al.\ \cite{Gardner12}
%%  developed a formalisation of JavaScript based on separation logic
%% % that they have used
%% and verified   examples. Xiong and Qin et
%% al.\ \cite{XiongPhd,Qin11}  worked on similar lines.
%% % More substantially,
%% Swamy et al.\ \cite{JSDijkstraMonad}  recently
%% developed a mechanised verification technique for JavaScript based on
%% the Dijkstra Monad in the F* programming language.  Finally, Jang et
%% al.\ \cite{Quark} % have %  managed to provide
%% developed a machine-checked proof of
%% five important properties of a web browser --- again similar to our
%% % \prg{any\_code} 
%% invariants --- such as
%% % \textit{``no tab may interfere with
%% %  another tab''} and 
%% \textit{``cookies may not be shared across
%%   domains''} by writing the minimal kernel of the browser in Haskell.
  
%%   \paragraph{JavaScript analyses.}
%% More practically, 
%% Karim et al. apply static analysis on
%% Mozilla's JavaScript Jetpack extension framework \cite{adsafe}, including
%%  pointer analyses. % In a different direction,
%% Bhargavan et al.\ \cite{DefJS}
%% extend language-based sandboxing techniques to support defensive
%% components that can execute successfully  in otherwise untrusted
%% environments.   Politz et
%% al.\ \cite{ADsafety} use a JavaScript type checker to check
%% properties such as
%% % \textit{``widgets cannot obtain direct references
%%  % to DOM nodes''} and
%%  \textit{``multiple widgets on the same page
%%   cannot communicate.''}
%% % --- somewhat similar in spirit to our \textbf{Pol\_4}.
%% Lerner et al.\ extend this system to ensure browser
%% extensions observe \textit{``private mode''} browsing conventions,
%% such as that \textit{``no private browsing history retained''}
%% \cite{Lerner2013b}.  Dimoulas et al.\ \cite{DPCC14} generalise the
%% language and type checker based approach to enforce explicit policies,
%% % although the policies  are restricted to
%% that  describe  which components  may
%% access, or may influence the use of, particular capabilities.
%% Alternatively, Taly et al.\ \cite{secureJS}
%% model  JavaScript APIs in Datalog, and then
%% carry out a Datalog search for an ``attacker'' from the set of all
%% valid API calls. 



\paragraph{Verification of Object Capability Programs}
Murray made the first attempt to formalise defensive consistency and
correctness~\cite{Murray10dphil}.  Murray's model was rooted in
counterfactual causation~\cite{Lewis_73}: an object is defensively
consistent when the addition of untrustworthy clients cannot cause
well-behaved clients to be given incorrect service.  Murray formalised
defensive consistency very abstractly, over models of (concurrent)
object-capability systems in the process algebra CSP~\cite{Hoare:CSP},
without a specification language for describing effects, such as what
it means for an object to provide incorrect service.  Both Miller and
Murray's definitions are intensional, describing what it means for an
object to be defensively consistent.


Dro\-sso\-pou\-lou and Noble \cite{capeFTfJP,capeFTfJP14} have
analysed Miller's Mint and Purse example \cite{MillerPhD} 
% SD Chope details by
% expressing it in Joe-E 
% a Java subset without reflection and static
%fields, 
%and in Grace \cite{capeFTfJP14}, 
and discussed the six
capability policies 
% that characterise the correct behaviour of the
% program, 
as proposed in \cite{MillerPhD}.
%We argued that these policies require a novel
%approach to specification, and showed some first ideas on how to use
%temporal logic.
In %  an unpublished technical report
\cite{WAS-OOPSLA14-TR}, {they} % Drossopoulou and Noble
sketched a  specification language,  \sd{used}  it to  
specify the six policies from \cite{MillerPhD}, % however,
%{their} partial formalisation showed that % they allowed
\sd{showed} that several possible interpretations were possible, %.  They also 
\sd{and} uncovered
the need for another four further policies.
%  and formalised them as well, showing how different implementations of the underlying Mint and Purse
% systems coexist with different policies \cite{capeIFM14},
They also
  sketched how 
a trust-sensitive 
example (the escrow exchange) could be verified in an open world
\cite{swapsies}. 
% In contrast, our work focuses on the semantics of the  \Chainmail\ specification
% language and how it can be used to provide holistic specifications for
% robust programs.
\sd{Their work does not support the concepts of control, time, or space, as in \Chainmail,
but it offers a primitive expressing trust.}
 
Devriese et al.\ \cite{dd}  have deployed
   \sd{powerful} %rather more complex
  theoretical techniques to address similar problems:  % Devrise et al.\ 
  \sd{They} show how step-indexing, Kripke worlds, and representing objects
as state machines with public and private transitions can be used to
reason about % object-oriented programs in general.
\sd{object capabilities}.
Devriese have demonstrated solutions to a range of exemplar problems,
including the DOM wrapper (replicated in our
Section~\ref{sect:example:DOM}) and a mashup application.
% Although the formal techniques are much more sophisticated than we
%apply here, and consequently 
% not true can e.g.\ reason about recursion where we
%cannot, there are some similarities, e.g.\ with the 
\sd{Their} distinction
between public and private transitions %being related 
\sd{is similar} to the
distinction between internal and external objects.

More recently, Swasey et al.\ \cite{ddd}  designed OCPL, a logic
for object capability patterns, that supports specifications and
proofs for object-oriented systems in an open world.  
% The key idea here is to 
\sd{They} % say it simpler
draw on verification techniques for security and
information flow: separating internal implementations (``high values''
which must not be exposed to attacking code) from interface objects
(``low values'' which may be exposed).  OCPL supports defensive
consistency % (Swasey et al.\ use 
(\sd{they} use the term ``robust safety'' from the
security community \cite{Bengtson}) via a proof system that ensures
low values can never leak high values to external attackers. 
%\susan{How does this imply that high values can be exposed?}
%\james{typo fixed: it's LOW values that can be exposed}
This means that low values \textit{can} be exposed to external code,
and the behaviour of the system is described by considering attacks only
on low values.  %OCPL is a program logic, and Swasey
\sd{They} use that logic to
prove a number of object-capability patterns, including
sealer/unsealer pairs, the caretaker, and a general membrane.

Schaefer et al.\ \cite{schaeferCbC} have recently
% taken a similar approach to Swasey,
% adding support for
\sd{added}  support for information-flow security % in a setting 
\sd{using} refinement to ensure correctness (in this case confidentiality) by
construction. 
% Although designed to support
% confidentialty, it seems likely that the information-flow guarantees
% could also be used to ensure robustness.  
By enforcing encapsulation, \sd{all} % used to say both
these approaches share similarity with techniques such as
ownership types \cite{ownalias,NobPotVitECOOP98}, which also
protect internal implementation objects from accesses that cross
encapsulation boundaries.  Banerjee and Naumann demonstrated that by
ensuring confinement, ownership
systems can enforce representation independence (a property close to
``robust safety'') some time ago \cite{Banerjee:2005}.

 
\Chainmail\ differs from Swasey, Schaefer's, and Devriese's work in a number of ways:
% \citet{ddd} and \citet{schaeferCbC} 
\sd{They} are primarily concerned \sd{with} %about
mechanisms that ensure encapsulation (aka 
confinement) while we abstract away from any mechanism via the
$\External{}$ predicate. 
\sd{They use powerful mathematical techniques
% , such as Kripke worlds and step-indexing 
which  the users need  to understand in order to write their specifications,
while \Chainmail users only need  to understand  first order logic and 
the holistic operators presented in this paper.}
% While \Chainmail's $\Using{}{}$ is related to Banerjee
% and Naumann's region sets, the assertion languages here are mostly
% traditional (extensions of) Hoare logics --- Swasey et al.\ build on a
%concurrent separation logic. 
\sd{ Finally, none of these systems offer the kinds of
holistic assertions addressing control flow, change, or temporal
operations that are at the core of \Chainmail's approach.
}

Scilla \cite{scillaOOPSLA19} is a minimalistic typed functional
language for writing smart contracts that compiles to the Ethereum
bytecode. Scilla's semantic model is restricted, assuming actor based
communication and restricting recursion,  thus facilitating static
analysis of Scilla contracts and ensuring termination.
Scilla is able to demonstrate that a number of popular Ethereum
contracts avoid type errors, out-of-gas resource failures, and
preservation of virtual currency. 
Scilla's semantics are defined formally, but have not yet been represented in a
mechanised model.

%% \kjx{NPChecker \cite{NPcheckerOOPSLA19} analyses Ethereum smart
%% contracts to detect bugs related to nondeterministic
%% execution. NPChecker undertakes an information flow
%% analysis to detect potential read-write hazards
%% particularly reentrancy and ordering dependencies.
%% \textbf{We don't do concurrency. Do we need this one? I don't think so}
%% }


Finally, the recent VerX tool is able to verify a range of
specifications for solidity contracts automatically \cite{VerX}.
Similar to \Chainmail, VerX has a specification language based on
temporal logic.  VerX offers three temporal operators (always, once,
prev) but only within a past modality, while \Chainmail\ has two
temporal operators, both existential, but with both past and future
modalities.   VerX specifications can also include predicates th	at
model the current invocation on a contract (similar to \Chainmail's
``calls''), can access variables, and compute sums (only) over
collections. \Chainmail\ is strictly more expressive as a
specification language, including quantification over objects and sets
(so can compute arbitrary reductions on collections) and of course
specifications for permission (``access''), space (``in'') and
viewpoint (``external'') which have no analogues in VerX. 
Unlike \Chainmail, VerX includes a practical tool that has
been used to verify   a hundred properties across case studies of
twelve Solidity contracts.
%\textbf{(ideally also say something about proof status)}}

\jm{
\paragraph{Incorrectness Logic.} O'Hearn~\cite{IncorrectnessLogic} defined a Hoare
Logic for modelling program incorrectness. O'Hearn's Incorrectness Logic
is based on a Reverse Hoare Logic \cite{reverseHoare}, and empowers programmers to 
specify preconditions under which specific errors and program states may result. 
Incorrectness Logic provides a sound and compositional way to reason about 
the presence of bugs rather than the absence of bugs. 
As with Hoare logic, Incorrectness Logic provides a system
for reasoning about sufficent conditions for post-conditions to hold.
However, where Hoare logic specifies the shape of the result of execution 
of all program states that satisfy the precondition, Incorrectness Logic
specifies that all states that satisfy the postcondition are reachable
from those that satisfy the precondition. This suits the specification
of program errors, as it allows for the exclusion of false negatives.
In comparison, \Chainmail, as with Hoare Logic, is concerned with correctness
(as seen in the exemplars of Section \ref{sect:problemdriven}). 
Extending the comparison, \Chainmail differs from both Hoare Logic and Incorrectness,
in the ability to specify, not just sufficient conditions, but necessary conditions for 
reaching certain program states. Neither Incorrectness Logic, 
nor Hoare Logic allows for such specifications.
}

% SD chopped as did not like
%As with Swasey et al.\ this work does not provide a holistic
%assertion language like \Chainmail.
% SD Chopped, as it sounds as if their is not real code, which is debatable
% and what is an extensional framework? they would say that theirs is too.
%In contrast, \Chainmail\ is
%meant for describing and reasoning about real code, and we provide an
%expressive, extensional framework for evaluating defensive consistency
%in actual open systems.
%


%%%%%%%%%%%%%%%%%%%%%%%%%%%%%%%%%%%%%%%%%%%%%%%%%%%%%%%%%%%%
%%NOTES:
%% the other thing this section needs to do, particularly with Devrise, is to lay out precisely the way our work is more limited than theirs:
%% (Swasey, I'm more and more convinced, is just ownership-via-a-proof-system) 
%% we don't step-index, don't have logical relations, etc: what do we lose by NOT having those things
%% (or what do we gain by having those things...

%% The "deep" comparison with Swasey and with Devirese (and also
%% information flow control and temporal logics) needs to say why these
%% works are not as good (expressive? easy to understand?) as ours.
%% Currently the Related work just mentions them, but does not answer the
%% question as to why our work is important when theirs already has been
%% published.




%% *Difference between Spec Languages and Chainmail*  One way to tackle
%%  this would be to enumerate which elements of Chainmail appear at
%%  other works, which do not, and claim that Chainmail’s novelty is the
%%  good combination of these elements


%% Eg: reflection about contents of stack and heap (in classical Hoare
%% Logics), two state assertions (JML etc), invariants (Hoare and Meyer),
%% internal/external (Liskov?, Noble et al,modules in Neumann and also
%% O’Hearn), time (temporal logic, but they do not have the other stuff),
%% Control (none?), Space (in Sep. logic, and in effects, buyt the
%% meaning is different), Permissions (our earlier work, and less
%% flexible approaches such as owenrship types and perhaps also
%% oinformation flow control), Authority (effect systems and modifies
%% clauses, and perhaps also Bierman&Parkison abstract predicates, but
%% there it is tied to pre-post conditions.


%% Also, point out difference between our invariants and Hoare
%% triples. Subtle and needs thinking







%%%%%%%%%%%%%%%%%%%%%%%%%%%%%%%%%%%%%%%%%%%%%%%%%%%%%%%%%%%%

%% Neither effort addresses specification languages for security and
%% robustness, provides Hoare logics to reason about object-capability
%% programs.

%% , model protocols that dynamically ascribe trust
%% \cite{swapsies,lefthand} or quantify the damage an untrustworthy
%% object can do.






% \kjx{History-Based Specification and Verification of Scalable
%  Concurrent and Distributed Systems --- ICFEM15}


% \paragraph{Specifying Design Patterns}

% Techniques for specifying Design Patterns go back at least to 
% Helm's contracts \cite{Helm92}.

% more importantly: work on formalisation of design patterns.
% (again look at JC grant, even if refs are 5 years old)
% let's be shameless here...



% This search is similar to the quantification over
% potential code snippets in our model.
% The problem posed by the Escrow example is that it establishes a two-way
% dependency between trusted and untrusted systems --- precisely the
% kind of dependencies these techniques prevent.

% %These approaches are all based on static analyses.
%  The WebSand
% \cite{flowcaps11,sabelfeld-inlining2012} and Jeeves \cite{jeeves2012}
% projects use dynamic techniques to monitor safe execution of information flow policies.
%  Richards et al.\ \cite{FlacJS}   extended this approach by
% incorporating explicit dynamic ownership of objects (and thus of
% capabilities) and policies that may examine the history of objects'
% computations. While these dynamic techniques can restrict or terminate
% the execution of a component that breaches its security policies, they
% cannot guarantee in advance that such violations can never happen.
% While information flow policies are concerned with the flow of objects (and thus also capabilities)
% across the program code, our work is more concerned with the identification of the objects which protect
% the services.

%Compared with all these approaches, our work   focuses on
%\textit{general} techniques for specifying (and ultimately verifying)
%capability policies, whereas these systems are generally much more
%\textit{specific}: focusing on one (or a small number) of actual
%policies. % This seems to be because contemporary object capability
%programming is primarily carried out in JavaScript, but
% There are few

 
% \paragraph{Relational models of trust.}
% Artz and Gil \cite{artz-trust-survey-2007} survey various
% types of trust in computer science generally, although trust has also
% been studied in specific settings, 
% %
% ranging from peer-to-peer systems \cite{aberer-trust-p2p-2001} and
% cloud computing \cite{habib-trust-cloud-2011} 
% to mobile ad-hoc networks \cite{cho-trust-survey-adhocnets-2011}, 
% the internet of things \cite{lize-trust-IoT-2014}, 
% online dating \cite{norcie-trust-online-dating},
% and as a component of a wider socio-technical system
% \cite{cho-trust-sustainable-2013,walter-trust-cloud-ecis2013}. 
% %
% Considering trust (and risk) in systems design, Cahill et al.'s overview
% of the \textsc{Secure} project \cite{cahill-trust-pervasive-2003}
% gives a good introduction to both theoretical and practical issues of
% risk and trust, including a qualitative analysis of an e-purse
% example. This project builds on Carbone's trust model
% \cite{carbone-formal-trust-2003} which offers a core semantic model of
% trust based on intervals to capture both trust and uncertainty in that
% trust. Earlier Abdul-Rahman proposed using separate relations for
% trust and recommendation in distributed systems
% \cite{abdul-rahman-distributed-trust-1998}, more recently Huang and
% Nicol preset a first-order formalisation that makes the same
% distinction \cite{huang-formal-semantics-trust-calculus-2010}.
% Solhaug and St{\o}len \cite{solhaug-trust-uncertainty-2011} 
% consider how risk and trust are related to uncertainties over
% actual outcomes versus knowledge of outcomes.
% Compared with our work, these approaches produce models of trust
% relationships between high-level system components 
% (typically treating risk as uncertainty in trust) 
% but do not link those relations to the system's code. 



% \paragraph{Logical models of trust.}
% \sd{A detailed study of how web-users decide whether to trust appears in \cite{GilArtz}.}
% \sd{Starting with \cite{Lampson92},} various different logics have been used to measure trust in different
% kinds of systems.
% Murray and Lowe \cite{murray10-infoflow} model object capability
% programs in CSP, and use a model checker to ensure program executions
% do not leak authority.
% Carbone et al.\ \cite{carbone-formal-trust-2011}
% use linear temporal logic to model specific trust relationships in service
% oriented architectures.
% Ries et al.\ \cite{habib-trust-CertainLogic-2011} evaluate trust under
% uncertainty by evaluating Boolean expressions in terms of real values
% for average rating, certainty, and initial expectation.
% % Perhaps closer to our work, Aldini
% Aldini \cite{aldini-calculus-trust-IFIPTM2014} describes a temporal logic for
% trust that supports model checking to verify some trust properties.
% Primiero and Taddeo \cite{primiero-modal-theory-trust-2011} have
% developed a modal type theory that treats trust as a second-order
% relation over base relations between
% counterparties. Merro and Sibilio
% \cite{merro-calculus-trust-adhoc-facs2011} developed a trust model for
% a process calculus based on labelled transition systems.
% Compared with our proposal, these approaches use
% process calculi or other abstract logical models of systems, rather
% than engaging directly with the system's code.






%%%% %%%% %%%% %%%% %%%% %%%% %%%% %%%% %%%% %%%% %%%% %%%% %%%% %%%% 
%%%% %%%% %%%% %%%% %%%% %%%% %%%% %%%% %%%% %%%% %%%% %%%% %%%% %%%% 







%Susan:Please read first bit as I have just written it
%\se{When you write a module that is to be used with other code, the last thing you want to happen is that some other code uses it to cause effects that you never intended. Our specification language \Chainmail has been designed, so that developers whose modules are going to be used in the wild, have the language to constrain the usage of their code. In addition to classical function by function specification techniques, we have shown that a holistic or whole program approach is needed to make open world code robust. We have shown} 
% going to the old one, as running out of space.
% also, the new one brings new words, and I think all th words in concluson should have appeared earlier
In this paper we have motivated the need for holistic specifications,
presented the \Chainmail specification language for writing such
specifications, and shown 
how \Chainmail can be used to give holistic
specifications of key exemplar problems: the bank account,  the
wrapped DOM, the ERC20, and and the DAO.

To focus on the key attributes of a holistic specification language,
% we have tried to keep the
\sd{we have kept  \Chainmail simple, only requiring an understanding of first order logic.}
\sd{We believe that the holistic features (permission, control, time, space and viewpoint),
are intuitive concepts %for ptogrammers. 
when reasoning informally, and were pleased to have been able to provide their
formal semantics in what  we  argue is a simple manner.}
% below not true, we do have recusrions  
%do not even support recursive procedures to avoid circularities in the
%metatheory, let alone concurrency, exceptions, distribution,
%networking, etc. 

\sd{The development of the semantics of \Chainmail assertions posed several interesting 
challenges, \eg the treatment of the open world requires two-module execution
and the concept of external objects,
recursion is confined to ghostfields and assertions require termination of included expressions,
space required the concept of restricting runtime configurations,
and time required adaptation operators which apply bindings from one configuration to another.}  

\sd{\Chainmail is powerful enough to express many key examples from the
literature; nevertheless, it lacks several important features: It provides 
recursion  only in a restricted form, it has a rather inflexible notion of
module and does not support hierarchies of modules, and knows nothing about
concurrency or distribution.  We plan to remove these restrictions by applying
techniques such as step-indexing \cite{stepindex}, but hope to keep any mathematical 
sophitsication in the
model of \Chainmail without exposing it to the person who writes the specification.  We are also
 interested in extending \Chainmail\ to situations
where internal modules are typed, but the external modules are untyped.
%
We also plan to extend \Chainmail to support reasoning about
conditional trust in programs, and to quantify the risks involved in
interacting with untrustworthy software \cite{swapsies}.
}

\sd{To make these kinds of specifications
practically useful,  we plan to develop logics for proving adherence of module's code to holistic specs, as well
as logics for using holistic specs in the proof of open programs. We want to develop 
dynamic monitoring  and model checking techniques for our specifications. 
And finally, we plan to automate reasoning with these logics.}






%% Acknowledgments
\begin{acks}                            %% acks environment is optional
                                        %% contents suppressed with 'anonymous'
  %% Commands \grantsponsor{<sponsorID>}{<name>}{<url>} and
  %% \grantnum[<url>]{<sponsorID>}{<number>} should be used to
  %% acknowledge financial support and will be used by metadata
  %% extraction tools.
  This material is based upon work supported by the
  \grantsponsor{GS100000001}{National Science
    Foundation}{http://dx.doi.org/10.13039/100000001} under Grant
  No.~\grantnum{GS100000001}{nnnnnnn} and Grant
  No.~\grantnum{GS100000001}{mmmmmmm}.  Any opinions, findings, and
  conclusions or recommendations expressed in this material are those
  of the author and do not necessarily reflect the views of the
  National Science Foundation.
\end{acks}


%% Bibliography
\bibliography{Case,more}

\clearpage

%% Appendix
\appendix
\appendix

\section{\LangOO - full defintion}
\label{app:loo}


We introduce \LangOO, a simple, typed, class-based, object-oriented language.
To reduce the complexity of our formal models, \LangOO lacks many
common languages features, omitting static fields and methods, interfaces,
inheritance, subsumption, exceptions, and control flow.  
 \LangOO includes ghost fields,  that may only be used in the specification language.
The ghost fields may be defined recursively.
%
%\kjx{
%These features are
%well-understood: their presence (or absence) would not chanage the
%results we claim nor the structures of the proofs of those results.
%Similarly, while Loo is typed, we don't present or mechanise
%its type system. 
%Our results and proofs rely only upon type
%soundness --- in fact, we only need that an expression of
%type $T$ (where $T$ is a class $C$ declared in module $M$)
%will evaluate to an instance of some class from $M$,
%with the same confinement status as $C$.
%Featherweight Java extended with modules and assignment
%will more than suffice \cite{IgaPieWadTOPLAS01}.
%% well-understood that it is too boring to present here or to mechanise anew --- 
%%
%}


\subsection{Syntax}
The syntax of \LangOO is given in Fig. \ref{f:loo-syntax}.
\LangOO modules ($M$) map class names ($C$) to class definitions ($\textit{ClassDef}$).
A class definition consists of % \jm[]{an optional annotation \enclosed},
a list of field definitions, ghost field definitions, and method definitions.
{Fields, ghost fields, and methods all have types: {types are
    classes}. Ghost fields may be optionally 
annotated as \texttt{intrnl}, requiring the argument to have an internal type, and the 
body of the ghost field to only contain references to internal objects. This is enforced by the
limited type system of \LangOO.}
A program state ($\sigma$) is represented as a heap ($\chi$), stack ($\psi$) pair, 
where a heap is a map from addresses ($\alpha$) to objects ($o$), and a stack is a non-empty list of frames ($\phi$). A frame consists of a local variable
map and a continuation ($c$) that represents the statements that are yet to be executed ($s$).
% or a hole waiting to be filled by a method return in the frame above ($x := \bullet; s$).
A statement is either a field read ($x := y.f$), a field write ($x.f := y$), a method call
($u :=y_0.m(\overline{y})$), a constructor call ($\prg{new}\ C(\overline{x})$), a method return statement
($\prg{return}\ x$), or a sequence of statements ($s;\ s$).

\LangOO also includes syntax for expressions $e$ that may %only
be used in writing
specifications or the definition of ghost fields.


\begin{figure}[t]
\footnotesize
\[
\begin{syntax}
\syntaxID{x, y, z}{Variable}
\syntaxID{C, D}{Class Id.}
\syntaxID{f}{Field Id.}
\syntaxID{g}{Ghost Field Id.}
\syntaxID{m}{Method Id.}
\syntaxID{\alpha}{Address Id.}
\syntaxInSet{i}{\IntSet}{Integer}
\syntaxElement{v}{Value}
		{
		\syntaxline
				{\alpha}
				{i}
				{\true}
				{\false}
				{\nul}
		\endsyntaxline
		}
\endSyntaxElement\\
\\
\syntaxElement{Mdl}{Module Def.}
		{
		\syntaxline{\overline{C\ \mapsto\ CDef}}\endsyntaxline
		}
\endSyntaxElement\\
\syntaxElement{CDef}{Class Def.}
		{
		[An]\ \prg{class}\ C\ 
		\{\ \prg{cnstr}:= (\overline{x : T})\{s\};\ \prg{flds}:=\overline{fld};\ \prg{mths}:=\overline{mth};\ \prg{gflds}:=\overline{gfld};\  \}		
		}
\endSyntaxElement\\
\syntaxElement{An}{Class Annotation}
		{\enclosed}
\endSyntaxElement\\
\syntaxElement{T}{Type}
		{
		\syntaxline
%				{\_}
				{C}
		\endsyntaxline
		}
\endSyntaxElement\\
\syntaxElement{s}{Statement}
		{
		\syntaxline
				{\sdN{x:=y}}
				{\sdN{x:=v}}
				{x:=y.f}
				{x.f:=y}
				{x:=y_0.m(\overline{y})}
%		\endsyntaxline
%		}
%		{
%		\syntaxline
				{\new{C}{\overline{x}}}
				{\red{x}}
%				{\return{x}}
				{s;\ s}
		\endsyntaxline
		}
\endSyntaxElement\\
%\syntaxElement{c}{Continuation}
%		{
%		\syntaxline
%				{\sdN{s; \ x}}
%				{\sdN{x}}
%		\endsyntaxline
%		}
%\endSyntaxElement\\
\syntaxElement{mth}{Method Def.}
		{
		\prg{method}\ m\ (\overline{x : T})\sdN{:T}\{\ s\ \}
		}
\endSyntaxElement\\
\syntaxElement{fld}{Field Def.}
		{\syntaxline
			{\prg{field}\ f\ :\ T}
		\endsyntaxline}
\endSyntaxElement\\
\\
\syntaxElement{gfld}{Ghost Field Def.}
		{\syntaxline
			{\prg{ghost}\ g\ (\overline{x : T})\{\ gt\ \} : T}
			{\prg{ghost}\ \prg{intrnl}\ g\  (\overline{x : T})\{\ gt\ \} : T}
		\endsyntaxline}
\endSyntaxElement\\
\syntaxElement{\sdN{gt}}{\sdN{Ghost Term}}
		{
		\syntaxline
				{x}
				{v}
				{gt + gt}
				{gt = gt}
				{gt < gt}
	%	\endsyntaxline
	%	}
	%	{
	%	\syntaxline
				{\prg{if}\ gt\ \prg{then}\ gt\ \prg{else}\ gt}
				{gt.f}
				{t.g(gt)}
		\endsyntaxline
		}
\\
\endSyntaxElement\\
\syntaxElement{\sigma}{Program Config.}
		{(\prg{stack}:=\psi; \prg{heap}:=\chi)}
\endSyntaxElement\\
\syntaxElement{\psi}{Stack}
		{\syntaxline{\phi}{\phi \sdN{\cdot} \psi}\endsyntaxline}
\endSyntaxElement\\
\syntaxElement{\phi}{Frame}
		{\{\prg{local}:=\overline{x\mapsto v};\ \prg{contn}:=\sdN{s}\}}
\endSyntaxElement\\
\syntaxElement{\chi}{Heap}
		{\overline{\alpha \mapsto o}}
\endSyntaxElement\\
\syntaxElement{o}{Object}
		{\{\prg{class}:=C;\ \prg{flds}:=\overline{f \mapsto v} \}}
\endSyntaxElement\\
\end{syntax}
\]
\caption{\LangOO Syntax}
\label{f:loo-syntax}
\end{figure}

\footnoteSD{\red{JULIAN: Do we need the selectors? SD}}

\subsection{Semantics}
\LangOO is a simple object oriented language, and the operational semantics 
(given in Fig. \ref{f:loo-semantics} and discussed later)
do not introduce any novel or surprising features. The operational 
semantics make use of several helper definitions that we 
define here.

We provide a definition of reference interpretation in Definition \ref{def:interpret}
\begin{definition}
\label{def:interpret}
For a program state $\sigma = (\chi, \phi : \psi)$, we provide the following function definitions:
\begin{itemize}
\item
$\interpret{\sigma}{x}\ \triangleq\ \phi.(\prg{local})(x)$
\item
$\interpret{\sigma}{\alpha.f}\ \triangleq\ \chi(\alpha).(\prg{flds})(f)$
\item
$\interpret{\sigma}{x.f}\ \triangleq\ \interpret{\sigma}{\alpha.f}$ where $\interpret{\sigma}{x}=\alpha$
\end{itemize}
\end{definition}
That is, a variable $x$, or a field access on a variable $x.f$ 
has an interpretation within a program state of value $v$
if $x$ maps to $v$ in the local variable map, or the field
$f$ of the object identified by $x$ points to $v$.

Definition \ref{def:class-lookup} defines the class lookup function an object 
identified by variable $x$.
\begin{definition}[Class Lookup]
\label{def:class-lookup}
For program state $\sigma = (\chi, \phi \cdot\psi)$, class lookup is defined as 
$$\class{\sigma}{x}\ \triangleq\ \chi(\interpret{\sigma}{x}).(\prg{class})$$
\end{definition}

Definition \ref{def:meth-lookup} defines the method lookup function for a method
call $m$ on an object of class $C$.
\begin{definition}[Method Lookup]
\label{def:meth-lookup}
For module $\Mtwo$, class $C$, and method name $m$, method lookup is defined as 
$$\meth{\Mtwo}{C}{m}\ \triangleq\ M(C).\prg{mths}(m)\ \ \mbox{for some} M\in\Mtwo$$
\end{definition}

We borrow the definition  of module linking, given in Definition \ref{def:linking}.
\begin{definition}
\label{def:linking}
For all modules $\Mtwo$ and $M$, if the domains of$\Mtwo$ and $M$ are disjoint, 
we define the module linking function as $M\cdot \Mtwo\ \triangleq\ M\ \cup\ M'$.
\end{definition}
That is,  their linking is the union of the two if their domains are disjoint.


\newcommand{\Same}[4]{{SameModule(#1,#2,#3,#4)}}

Finally, we define what it means for two objects to come from the same module
\begin{definition}[Same Module]
\label{def:class-lookup}
For program state $\sigma$,  modules $\Mtwo$, and variables $x$ and $y$, we defone
$$\Same {x} {y} {\sigma}{\Mtwo}\ \triangleq\ \exists C, C', M[ \ M\in \Mtwo \wedge C, C'\in M \wedge  \class{\sigma}{x}=C \wedge \class{\sigma}{y} =C'\ ]$$
\end{definition}


Fig. \ref{f:loo-semantics} gives the operational semantics of \LangOO. 
Program state $\sigma_1$ reduces to $\sigma_2$ in the context of
modules$\Mtwo$ if $\exec{\Mtwo}{\sigma_1}{\sigma_2}$. The semantics in Fig. \ref{f:loo-semantics}
are unsurprising, but it is notable that reads (\textsc{Read}) and writes (\textsc{Write})
are restricted to the class that the field belongs to,
\sdN{and methods  may only be called if public, or from same module as current receiver.}
\begin{figure}[t]
\begin{minipage}{\textwidth}
\begin{minipage}{\textwidth}
\footnotesize
\begin{mathpar}
\infer
	{
	\sigma_1 = (\sdN{\overline{\phi}\cdot\phi},  \chi)\\
        \sdN{\phi}.(\prg{contn})  \txteq   u := y_0.m(\overline{y}); s \\
       % \phi_1' = \phi_1[\prg{contn} := (x := \bullet; s)]\\
	\meth{\Mtwo}{\class{\sigma_1}{y}}{m} = p \ C::m(\overline{x : T})\red{:T}\{s'\}\\
        	{\sdN{p=\prg{public} \ \vee \ \Same{\prg{this}} {y_0} {\sigma_1}{\Mtwo} }} \\
	\phi' = \{\prg{local}:= ([\prg{this}\ \mapsto\ \interpret{\sigma_1}{y_0}]\overline{[x\ \mapsto\ \interpret{\sigma_1}{y}]}), \prg{contn}:=s'\} \\
	  \sigma_2 = \sdN{(\overline{\phi}\cdot\phi\cdot\phi'},\chi)\\
	}
	{\exec{\Mtwo}{\sigma_1}{\sigma_2}}
	\quad(\textsc{Call})
	\and
\infer
	{
%	\sigma_1 = (\chi, \phi_1 \cdot\psi) \\
%	\sigma_2 = (\chi, \phi_2 \cdot\psi) \\
	\sdN{\sigma_1.(\prg{contn}) =  x := y.f; s} \\
	 \Same {\prg{this}}  {y}  {\sigma_1} {\Mtwo}\\
	%\phi_2 = \{\prg{local}:=\phi_1.(\prg{local})[x\ \mapsto\ v],\ \prg{contn}:=s\}
	\sigma_2=\sigma_1[x\mapsto  \interpret{\sigma_1}{y.f} \} ]
	}
	{\exec{\Mtwo}{\sigma_1}{\sigma_2}}
	\quad(\textsc{Read})
	\and
\infer
	{
 %	\sigma_1 = (\overline{\phi}\cdot\phi,\chi ) \\
	\sigma_1 .(\prg{contn}) =  x.f := y; s \\
	\Same {\prg{this}}  {x}  {\sigma_1} {\Mtwo}\\
%	\phi' = \phi [ \prg{contn}:=s ] \\ % \{\prg{local}:=\phi_1.(\prg{local}),\ \prg{contn}:=s\}\\
%	\chi' = \chi[\interpret{\sigma_1}{x}.f \mapsto\ \interpret{\sigma_1}{y} ]\\
	\sigma_2 = \sigma[\prg{contn}:=s][\interpret{\sigma_1}{x}.f \mapsto\ \interpret{\sigma_1}{y} ]	
	(\overline{\phi}\cdot\phi',\chi' )\\
	}
	{\exec{\Mtwo}{\sigma_1}{\sigma_2}}
	{}
	\quad(\textsc{Write})
	\and
\infer
	{
	\sigma_1.(\prg{contn})\ =\  x := \prg{new}\ C(\overline{z}); s \\
	fields(C)=\overline{f} \\
	\alpha \mbox{ fresh in } \sigma_1 \\
	% \phi' = \{\prg{local}:=[\prg{this} \mapsto \alpha],\overline{[p_i \mapsto \lfloor z_i \rfloor_{\sigma_1}}], \prg{contn} := s'\}\\
	\sigma_2 = \sigma_1[x\mapsto \alpha][\prg{cont}\mapsto s][\alpha  \mapsto  \{\prg{class}:=C, \prg{flds}:=\overline{f\ \mapsto\ \interpret {\sigma_1} {x}} ] 
%	\sigma_1 = (\chi, \phi \cdot\psi) \\
%	\phi.(\prg{contn}) = (x := \prg{new}\ C(\overline{z}); s)\\
%	M(C).(\prg{constr}) = (\overline{p : T})\{ s' \} \\
%	\phi' = \{\prg{local}:=[\prg{this} \mapsto \alpha],\overline{[p_i \mapsto \lfloor z_i \rfloor_{\sigma_1}}], \prg{contn} := s'\}\\
%	\sigma_2 = (\chi[\alpha\ \mapsto\ \{\prg{class}:=C, \prg{flds}:=\overline{f\ \mapsto\ \nul}], \phi' \cdot\phi[\prg{contn}\ :=\ (x := \bullet; s)] \cdot\psi)	
		}
	{\exec{\Mtwo}{\sigma_1}{\sigma_2}}
	\quad(\textsc{New})
	\and
\infer
	{
	\sigma_1 = (\sdN{\overline{\phi}\cdot}\phi_ \cdot\phi', \chi) \\
	\phi'.(\prg{contn}) \txtin  \red{z} \\ % )\ \textit{or}\ \phi_1.(\prg{contn}) = (\red{z})\\
	\phi.(\prg{contn})   \txteq  \red{x := y_0.m(\overline y)}; s \\
	\phi''= \phi[x \mapsto \interpret{\sigma_1}{z}][\prg{contn}:= \red{s} ]\\ 
	%(\prg{local} := \phi.(\prg{local})[x \mapsto \interpret{\sigma_1}{z}],  \prg{contn}:= \red{s} )   \\
	\sigma_2 = (\sdN{\overline{\phi}\cdot} \phi'', \chi)
	}
	{\exec{\Mtwo}{\sigma_1}{\sigma_2}}
	{}
	\quad(\textsc{Return})
\end{mathpar}
\caption{\LangOO operational Semantics}
\label{f:loo-semantics}
\end{minipage}
\begin{minipage}{\textwidth}
\footnotesize
\begin{mathpar}
\infer
		{}
		{\eval{M}{\sigma}{v}{v}}
		\quad(\textsc{E-Val})
		\and
\infer
		{}
		{\eval{M}{\sigma}{x}{\interpret{\sigma}{x}}}
		\quad(\textsc{E-Var})
		\and
\infer
		{
		\eval{M}{\sigma}{e_1}{i_1}\\
		\eval{M}{\sigma}{e_2}{i_2}\\
		i_1 + i_2 = i
		}
		{
		\eval{M}{\sigma}{e_1 + e_2}{i}
		}
		\quad(\textsc{E-Add})
		\and
\infer
		{
		\eval{M}{\sigma}{e_1}{v}\\
		\eval{M}{\sigma}{e_2}{v}
		}
		{
		\eval{M}{\sigma}{e_1 = e_2}{\true}
		}
		\quad(\textsc{E-Eq}_1)
		\and
\infer
		{
		\eval{M}{\sigma}{e_1}{v_1}\\
		\eval{M}{\sigma}{e_2}{v_2}\\
		v_1 \neq\ v_2
		}
		{
		\eval{M}{\sigma}{e_1 = e_2}{\false}
		}
		\quad(\textsc{E-Eq}_2)
		\and
\infer
		{
		\eval{M}{\sigma}{e}{\true}\\
		\eval{M}{\sigma}{e_1}{v}
		}
		{
%		\eval{M}{\sigma}{\ifthenelse{e}{e_1}{e_2}}{v}
		\eval{M}{\sigma}{e}{v}
		}
		\quad(\textsc{E-If}_1)
		\and
\infer
		{
		\eval{M}{\sigma}{e}{\false}\\
		\eval{M}{\sigma}{e_2}{v}
		}
		{
%		\eval{M}{\sigma}{\ifthenelse{e}{e_1}{e_2}}{v}
		\eval{M}{\sigma}{e}{v}
		}
		\quad(\textsc{E-If}_2)
		\and
\infer
		{
		\eval{M}{\sigma}{e}{\alpha}
		}
		{
		\eval{M}{\sigma}{e.f}{\interpret{\sigma}{\alpha.f}}
		}
		\quad(\textsc{E-Field})
		\and
\infer
		{
		\eval{M}{\sigma}{e_1}{\alpha}\\
		\eval{M}{\sigma}{e_2}{v'}\\
		\prg{ghost}\ g(x : T)\{e\} : T'\ \in\ M(\class{\sigma}{\alpha}).(\prg{gflds})\\
		\eval{M}{\sigma}{[v'/x]e}{v}
		}
		{
		\eval{M}{\sigma}{e_1.g(e_2)}{v}
		}
		\quad(\textsc{E-Ghost})
\end{mathpar}
\caption{\LangOO expression evaluation}
\label{f:evaluation}
\end{minipage}
\end{minipage}
\end{figure}

While the small-step operational semantics of \LangOO is given in Fig. \ref{f:loo-semantics},
specification satisfaction is defined over an abstracted notion of 
the operational semantics that models the open world. %, called \jm[]{\emph{external states semantics}}. 




An \emph{Initial} program state contains a single frame 
with a single local variable \prg{this} pointing to a single object 
in the heap of class \prg{Object}, and a continuation.
\begin{definition}[Initial Program State]
\label{def:initial}
A program state $\sigma$ is said to be an initial state ($\initial{\sigma}$)
if and only if
\begin{itemize}
\item
$\sigma.\prg{heap} = [\alpha\ \mapsto\ \{\prg{class}:=\prg{Object};\ \prg{flds}:=\emptyset\}]$ and
\item
$\sigma.\prg{stack} = \{\prg{local}:=[\prg{this}\ \mapsto\ \alpha];\ \prg{contn}:= s\}$
\end{itemize} 
for some address $\alpha$ and some statement $s$.
\end{definition}


%We give the semantics of module pair execution in Definition \ref{def:pair-reduce}
%\begin{definition}[External State Semantics]
%\label{def:pair-reduce-appendix}
%For all internal modules $M_1$, external modules $M_2$, and program configurations $\sigma$ and $\sigma'$, 
%we say that $\reduction{M_1}{M_2}{\sigma}{\sigma'}$ if and only if
%\begin{itemize}
%\item
%$\class{\sigma}{\sigma.(\prg{this})}\ \in\ M_2$ and
%\item
%$\class{\sigma'}{\sigma'.(\prg{this})}\ \in\ M_2$ and 
%\end{itemize} 
%and
%\begin{itemize}
%\item
%$\exec{M_1\ \circ\ M_2}{\sigma}{\sigma'}$ or
%\item
%$M_1 \circ M_2, \sigma \leadsto \sigma_1 \leadsto \ldots \sigma_n \leadsto \sigma'$ and $\class{\sigma_i}{\sigma_i.(\prg{this})} \in M_1$ for all $1 \leq i \leq\ n$
%\end{itemize}
%\end{definition}

Finally, we provide a semantics for expression evaluation is given in Fig. \ref{f:evaluation}. 
That is, given a module $M$ and a program state $\sigma$, expression $e$ evaluates to $v$
if $\eval{M}{\sigma}{e}{v}$. Note, the evaluation of expressions is separate from the operational
semantics of \LangOO, and thus there is no restriction on field access.



\newpage


\section{Encapsulation}

%\subsection{Proving Encapsulation}
\label{s:encap-proof}

%We start by giving providing the syntax for type contexts in Fig. \ref{f:context-syntax}.
%\begin{figure}[t]
%\[
%\begin{syntax}
%\syntaxElement{\Gamma}{Type Context}
%		{
%		\syntaxline
%				{\emptyset}
%				{\alpha : C,\ \Gamma}
%		\endsyntaxline
%		}
%\endSyntaxElement\\
%\end{syntax}
%\]
%\caption{}
%\label{f:context-syntax}
%\end{figure}
%We construct type contexts out of assertions using the following rules:
%\begin{mathpar}
%\infer
%		{}
%		{\textit{Env}(\alpha : C) = \alpha : C,\ \emptyset}
%		\and
%\infer
%		{}
%		{\textit{Env}(A_1\ \wedge\ A_2) = \textit{Env}(A_1) \cup \textit{Env}(A_2)}
%\end{mathpar}
%\begin{definition}[Assertion Encapsulation]
%For all modules $M$, and assertions $A$, and $A'$ we say $M\ \vdash\ A\ \Rightarrow\ A'$ if and only if M
%\end{definition}

\kjx{
Assertion encapsulation (Definition \ref{def:encapsulation}) is
critical to our approach.  Assertion encapsulation ensures that a
change in satisfaction of an assertion can only depend on computation
\textit{internal} to the module in which the assertion is encapsulated
--- this is related to the footprint of an
assertion \cite{objInvars,encaps}.
If the footprint of an assertion is contained 
within a module, then that assertion is encapsulated,
however there are assertions that are encapsulated by a module 
whose footprint is not contained within the module. 
Specifically, the assertion $\inside{x}$ is not 
contained within an module $M$ since its due to the
universal quantification contained withing 
$\inside{x}$, the footprint consists of portions 
of the heap that are external to $M$. $\inside{x}$ is 
encapsulated by $M$ since if only objects that derive 
from $M$ have access to $x$, it follows that a method call
on $M$ is required to gain access to $x$.
Necessity Logic itself does not depend on the details
of the encapsulation scheme --- only that we can determine
whether an assertion is encapsulated within a particular
part of the program.  For reasons of simplicity, 
we have adopted an encapsulation model for \Loo
based on 
\citeauthor{confined}'s \textit{Confined Types} [\citeyear{confined}]
(and we rely on their proof).
%
% We see no reason why a different
% encapsulation mechanism could not be used instead.
%
% KJX move to related work?
%
Confined types partition the objects accessible to code within a
module, based on those objects' defining classes and modules:
\begin{itemize}
\item instances of non-\enclosed classes %defined in a module
constitute their defining module's encapsulation
boundary \cite{TAME2003},
and may be accessed anywhere.
\item instances of \enclosed classes %defined in the module
are encapsulated \inside their defining module.
\item instances of \enclosed classes defined in \emph{other} modules
are not accessible elsewhere
\item instances of non-\enclosed classes defined in \emph{other}
modules are visible, however methods may only be invoked on such 
objects when the confinement system guarantees the particular instance
is only accessible \inside \emph{this} module.
\end{itemize}
%
\noindent \Loo's Confined Types rely on three syntactic restrictions
to enforce this encapsulation model:
\begin{itemize}
\item \enclosed class declarations must be annotated.
\item \enclosed objects may not be returned by methods of non-\enclosed
classes.
\item Ghost fields 
may be annotated as \prg{intrnl}; if so, they must only refer to objects \inside
their defining module --- i.e.\ either defined directly in that module, or
instances of non-\enclosed classes defined in \emph{other} modules
where those particular instances are only ever accessed within the
defining module.
\end{itemize}
}
\jm[Yes, I think that's right. I'm not 100\% sure that assertion 
encapsulation is defined by the footprint, unless I misunderstand 
footprint. It is possible for an assertion to be encapsulated, 
but depend on external objects. For example $\neg\access{x}{y}$:
if $\inside{y}$ is true, then $\neg\access{x}{y}$ is encapsulated,
even if $x$ is external.]{}

%% Using Confined Types for \Loo means that modules needing to encapsulate
%% assertions must meet the following ownership restrictions: 

%% however we assume several properties enforced by the type system, including 
%% simple ownership properties:
%% \begin{itemize}
%% \item
%% Method calls may not be made to external, non-module methods.
%% \item
%% \jm[]{Classes may be optionally annotated as \enclosed: their instances (``\enclosed'' objects) are marked as \enclosed, and may not be returned by methods of non-\enclosed classes.}
%% \item
%% Ghost fields may be annotated as \prg{intrnl} and thus may only include and be passed references to objects belonging to module internal classes.
%% \end{itemize}
%% %
%% \kjx
%% These rules enforce a simple model 
%%   Instances of classes confined in a module are internal to that module;
%% however, non-confined 

%% %
%% These encapsulation properties are easily enforceable, and we
%% do not define the type system as ownership types have been 
%% well covered in the literature. 
%% We specifically use a simple ownership system to model 
%% encapsulation as the theory has been well established by others, 





\jm[]{
%To assist in the definition of our simple encapsulation system,
We define internally evaluated expressions ($\intrnl{\_}$) 
whose evaluation only inspects internal objects or primitvies (i.e. integers or booleans).}
\jm[]{\begin{definition}[Internally Evaluated Expressions]
For all modules $M$, assertions $A$, and expressions $e$, 
$\satisfies{M}{\givenA{A}{\intrnl{e}}}$ if and only if for all heaps $\chi$, stacks $\psi$, and frames $\phi$
such that $\satisfiesA{M}{(\chi, \phi : \psi)}{A}$, we have for all values $v$, such that $\eval{M}{(\chi, \phi : \psi)}{e}{v}$
then $\eval{M}{(\chi', \phi' : \psi)}{e}{v}$, where 
\begin{itemize}
\item $\chi'$ is the internal portion of $\chi$, i.e. \\
$\chi' = \{\alpha \mapsto o| \alpha \mapsto o \in \chi\ \wedge \ o.(\prg{cname}) \in M \}$ and
\item $\phi'.(\prg{local})$ is the internal portion of the $\phi.(\prg{local})$ i.e. \\
$\phi' = \{x \mapsto v| x \mapsto v \in \chi\ \wedge \ (v \in \IntSet\ \vee\ v = \true\ \vee\ v = false)\ \vee\ (\exists \alpha, \ v = \alpha \wedge \class{(\chi, \phi : \psi)}{\alpha} \in M\}$
\end{itemize}
\end{definition}}


The encapsulation proof system consists of two relations 
\begin{itemize}
\item
Purely internal expressions: $\proves{M}{\givenA{A}{\intrnl{e}}}$ and
\item
Assertion encapsulation: $\proves{M}{\givenA{A}{\encaps{A'}}}$
\end{itemize}

Fig. \ref{f:intrnl} gives proof rules for an expression comprising purely internal objects.
\jm[]{Primitives are $Enc_e$ (\textsc{Enc$_e$-Int}, \textsc{Enc$_e$-Null}, \textsc{Enc$_e$-True}, and \textsc{Enc$_e$-False}).
Addresses of internal objects are $Enc_e$ (\textsc{Enc$_e$-Obj}). Field accesses with internal types of $Enc_e$ expressions
are themselves $Enc_e$ (\textsc{Enc$_e$-Field}). Ghost field accesses annotated as $Enc_e$ on $Enc_e$ 
expressions are themselves $Enc_e$ (\textsc{Enc$_e$-Ghost}).}

\begin{figure}[h]
\footnotesize
\begin{mathpar}
\infer
		{}
		{\proves{M}{\givenA{A}{\intrnl{i}}}}
		\quad(\textsc{Enc$_e$-Int})
		\and
\infer
		{}
		{\proves{M}{\givenA{A}{\intrnl{\nul}}}}
		\quad(\textsc{Enc$_e$-Null})
		\and
\infer
		{}
		{\proves{M}{\givenA{A}{\intrnl{\true}}}}
		\quad(\textsc{Enc$_e$-True})
		\and
\infer
		{}
		{\proves{M}{\givenA{A}{\intrnl{\false}}}}
		\quad(\textsc{Enc$_e$-False})
		\and
\infer
		{
		\proves{M}{A\ \longrightarrow\ \alpha : C}\\
		C\ \in\ M
		}
		{
		\proves{M}{\givenA{A}{\intrnl{\alpha}}}
		}
		\quad(\textsc{Enc$_e$-Obj})
		\and
\infer
		{
		\proves{M}{\givenA{A}{\intrnl{e}}}\\
		\proves{M}{A\ \longrightarrow\ e : C}\\
		[\prg{field}\ \_\ f\ :\ D]\ \in\ M(C).(\prg{flds}) \\
		D\ \in\ M
		}
		{
		\proves{M}{\givenA{A}{\intrnl{e.f}}}
		}
		\quad(\textsc{Enc$_e$-Field})
		\and
\infer
		{
		\proves{M}{\givenA{A}{\intrnl{e_1}}}\\
		\proves{M}{\givenA{A}{\intrnl{e_2}}}\\
%		\proves{M}{\givenA{A}{\intrnl{e}}} \\
		\proves{M}{A\ \longrightarrow\ e_1 : C} \\
		\prg{ghost}\ \prg{intrnl}\ g(x : \_)\{e\} \in M(C).(\prg{gflds})
		}
		{
		\proves{M}{\givenA{A}{\intrnl{e_1.g(e_2)}}}
		}
		\quad(\textsc{Enc$_e$-Ghost})
\end{mathpar}
\caption{Internal Proof Rules}
\label{f:intrnl}
\end{figure}


\jm[]{Fig. \ref{f:asrt-encap} gives proof rules for whether an assertion is encapsulated, that is whether 
a change in satisfaction of an assertion requires interaction with the internal module.
An \prg{Intrl} expression is also an encapsulated assertion (\textsc{Enc-Exp}). A field
access on an encapsulated expression is an encapsulated expression. Binary and ternary operators
applied to encapsulated expressions are themselves encapsulated assertions (\textsc{Enc-=}, \textsc{Enc-+}, \textsc{Enc-<}, \textsc{Enc-If}).
An internal object may only lose access to another object via internal computation (\textsc{Enc-IntAccess}).
Only internal computation may grant external access to an $\wrapped{\_}$ object (\textsc{Enc-Inside}$_1$).
If an object is $\wrapped{\_}$, then nothing (not even internal objects) may gain access
to that object except by internal computation (\textsc{Enc-Inside}$_2$).
If an assertion $A_1$ implies assertion $A_2$, then $A_1$ implies the encapsulation of any assertion that
$A_2$ does. Further, if an assertion is encapsulated, then any assertion that is implied by it is also encapsulated.
These two rules combine into an encapsulation rule for consequence (\textsc{Enc-Conseq}).}

\begin{figure}[h]
\footnotesize
\begin{mathpar}
\infer
		{\proves{M}{\givenA{A}{\intrnl{e}}}}
		{\proves{M}{\givenA{A}{\encaps{e}}}}
		\quad(\textsc{Enc-Exp})
		\and
\infer
		{\proves{M}{\givenA{A}{\intrnl{e}}}}
		{\proves{M}{\givenA{A}{\encaps{e.f}}}}
		\quad(\textsc{Enc-Field})
		\and
\infer
		{
		\proves{M}{\givenA{A}{\encaps{e_1}}} \\
		\proves{M}{\givenA{A}{\encaps{e_2}}}
		}
		{
		\proves{M}{\givenA{A}{\encaps{e_1 = e_2}}}
		}
		\quad(\textsc{Enc-=})
		\and
\infer
		{
		\proves{M}{\givenA{A}{\encaps{e_1}}} \\
		\proves{M}{\givenA{A}{\encaps{e_2}}}
		}
		{
		\proves{M}{\givenA{A}{\encaps{e_1 + e_2}}}
		}
		\quad(\textsc{Enc-+})
		\and
\infer
		{
		\proves{M}{\givenA{A}{\encaps{e_1}}} \\
		\proves{M}{\givenA{A}{\encaps{e_2}}}
		}
		{
		\proves{M}{\givenA{A}{\encaps{e_1 < e_2}}}
		}
		\quad(\textsc{Enc-<})
		\and
\infer
		{
		\proves{M}{\givenA{A}{\encaps{e}}} \\
		\proves{M}{\givenA{A}{\encaps{e_1}}} \\
		\proves{M}{\givenA{A}{\encaps{e_2}}}
		}
		{
		\proves{M}{\givenA{A}{\encaps{\prg{if}\ e\ \prg{then}\ e_1\ \prg{else}\ e_2}}}
		}
		\quad(\textsc{Enc-If})
		\and
\infer
		{\proves{M}{A\ \longrightarrow\ \internal{x}}}
		{\proves{M}{\givenA{A}{\encaps{\access{x}{y}}}}}
		\quad(\textsc{Enc-IntAccess})
		\and
\infer
		{}
		{\proves{M}{\givenA{A}{\encaps{\wrapped{x}}}}}
		\quad(\textsc{Enc-Inside}_1)
		\and
\infer
		{\proves{M}{A\ \longrightarrow\ \wrapped{x}}}
		{\proves{M}{\givenA{A}{\encaps{\neg \access{x}{y}}}}}
		\quad(\textsc{Enc-Inside}_2)
		\and
\infer
		{
		\proves{M}{A_1\ \longrightarrow\ A_2} \\
		\proves{M}{A\ \longrightarrow\ A'} \\
		\proves{M}{\givenA{A_2}{\encaps{A}}}
		}
		{\proves{M}{\givenA{A_1}{\encaps{A'}}}}
		\quad(\textsc{Enc-Conseq})
\end{mathpar}
\caption{Assertion Encapsulation Proof Rules}
\label{f:asrt-encap}
\end{figure}

\section{More about the Expressiveness of \Nec Specifications}
\label{s:expressiveness:appendix}

 We continue the comparison of expresiveness between \emph{Chainmail} and \Nec, by 
 considering the examples studied in \cite{FASE}.
 
\subsubsection{ERC20}
The ERC20 \cite{ERC20} is a widely used token standard describing the basic functionality of any Ethereum-based token 
contract. This functionality includes issuing tokens, keeping track of tokens belonging to participants, and the 
transfer of tokens between participants. Tokens may only be transferred if there are sufficient tokens in the 
participant's account, and if either they (using the \prg{transfer} method) or someone authorized by the participant (using the \prg{transferFrom} method) initiated the transfer. 

We specify these necessary conditions here using \Nec. Firstly, \prg{ERC20Spec1} 
says that if the balance of a participant's account is ever reduced by some amount $m$, then
that must have occurred as a result of a call to the \prg{transfer} method with amount $m$ by the participant,
or the \prg{transferFrom} method with the amount $m$ by some other participant.
\begin{lstlisting}[language = Chainmail, mathescape=true, frame=lines]
ERC20Spec1 $\triangleq$ from e : ERC20 $\wedge$ e.balance(p) = m + m' $\wedge$ m > 0
              next e.balance(p) = m'
              onlyIf $\exists$ p' p''.[$\calls{\prg{p'}}{\prg{e}}{\prg{transfer}}{\prg{p, m}}$ $\vee$ 
                     e.allowed(p, p'') $\geq$ m $\wedge$ $\calls{\prg{p''}}{\prg{e}}{\prg{transferFrom}}{\prg{p', m}}$]
\end{lstlisting}
Secondly, \prg{ERC20Spec2} specifies under what circumstances some participant \prg{p'} is authorized to 
spend \prg{m} tokens on behalf of \prg{p}: either \prg{p} approved \prg{p'}, \prg{p'} was previously authorized,
or \prg{p'} was authorized for some amount \prg{m + m'}, and spent \prg{m'}.
\begin{lstlisting}[language = Chainmail, mathescape=true, frame=lines]
ERC20Spec2 $\triangleq$ from e : ERC20 $\wedge$ p : Object $\wedge$ p' : Object $\wedge$ m : Nat
              next e.allowed(p, p') = m
              onlyIf $\calls{\prg{p}}{\prg{e}}{\prg{approve}}{\prg{p', m}}$ $\vee$ 
                     (e.allowed(p, p') = m $\wedge$ 
                      $\neg$ ($\calls{\prg{p'}}{\prg{e}}{\prg{transferFrom}}{\prg{p, \_}}$ $\vee$ 
                              $\calls{\prg{p}}{\prg{e}}{\prg{allowed}}{\prg{p, \_}}$)) $\vee$
                     $\exists$ p''. [e.allowed(p, p') = m + m' $\wedge$ $\calls{\prg{p'}}{\prg{e}}{\prg{transferFrom}}{\prg{p'', m'}}$]
\end{lstlisting}

\subsubsection{DAO}
The Decentralized Autonomous Organization (DAO)~\cite{Dao}  is a well-known Ethereum contract allowing 
participants to invest funds. The DAO famously was exploited with a re-entrancy bug in 2016, 
and lost \$50M. Here we provide specifications that would have secured the DAO against such a 
bug. \prg{DAOSpec1} says that no participant's balance may ever exceed the ether remaining 
in DAO.
\begin{lstlisting}[language = Chainmail, mathescape=true, frame=lines]
DAOSpec1 $\triangleq$ from d : DAO $\wedge$ p : Object
            to d.balance(p) > d.ether
            onlyIf false
\end{lstlisting}
Note that \prg{DAOSpec1} enforces a class invariant of \prg{DAO}, something that could be enforced
by traditional specifications using class invariants.
The second specification \prg{DAOSpec2} states that if after some single step of execution, a participant's balance is \prg{m}, then 
either 
\begin{description}
\item[(a)] this occurred as a result of joining the DAO with an initial investment of \prg{m}, 
\item[(b)] the balance is \prg{0} and they've just withdrawn their funds, or 
\item[(c) ]the balance was \prg{m} to begin with
\end{description}
\begin{lstlisting}[language = Chainmail, mathescape=true, frame=lines]
DAOSpec2 $\triangleq$ from d : DAO $\wedge$ p : Object
            next d.balance(p) = m
            onlyIf $\calls{\prg{p}}{\prg{d}}{\prg{repay}}{\prg{\_}}$ $\wedge$ m = 0 $\vee$ $\calls{\prg{p}}{\prg{d}}{\prg{join}}{\prg{m}}$ $\vee$ d.balance(p) = m
\end{lstlisting}

\sophiaPonder[small changes over Julian's]{\subsubsection{Safe}
\cite{FASE} used as a running example   a Safe, where a treasure 
was secured within a \texttt{Safe} object, and access to the treasure was only granted by 
providing the correct password. }
\ Using \Nec, we express \texttt{SafeSpec}, that requires that the treasure cannot be 
removed from the safe without knowledge of the secret.
\begin{lstlisting}[language = Chainmail, mathescape=true, frame=lines]
SafeSpec $\triangleq$ from s : Safe $\wedge$ s.treasure != null
            to s.treasure == null
            onlyIf $\neg$ inside(s.secret)
\end{lstlisting}

The module  \prg{SafeModule} described  below satisfies  \prg{SafeSpec}.

\begin{lstlisting}[frame=lines]
module SafeModule
     class Secret{}
     class Treasure{}
     class Safe{
         field treasure : Treasure
         field secret : Secret
         method take(scr : Secret){
              if (this.secret==scr) then {
                   t=treasure
                   this.treasure = null
                   return t } 
          }
 }
\end{lstlisting}

 

\section{More \Nec Logic rules}
\label{a:necSpec}
Here we give the complete version of the rules in Fig. \ref{f:only-through} and 
Fig. . \ref{f:only-through} which appeared in the main paper.

\begin{figure}[t]
\footnotesize
\begin{mathpar}
\infer
	{\proves{M}{\onlyIfSingle{A}{\neg A}{A'}}}
	{
	\proves{M}{\onlyThrough{A}{\neg A}{A'}}
	}
	\quad(\textsc{Changes})
	\and
\infer
	{
	\proves{M}{A_1\ \longrightarrow\ A_1'}\\
	\proves{M}{A_2\ \longrightarrow\ A_2'}\\
	\proves{M}{A_3'\ \longrightarrow\ A_3}\\
	\proves{M}{\onlyThrough{A_1'}{A_2'}{A_3'}}
	}
	{\proves{M}{\onlyThrough{A_1}{A_2}{A_3}}}
	\quad(\textsc{$\longrightarrow$})
	\and
\infer
	{
	\proves{M}{\onlyThrough{A_1}{A_2}{A}} \\\\
	\proves{M}{\onlyThrough{A_1'}{A_2}{A'}}
	}
	{\proves{M}{\onlyThrough{A_1\ \vee\ A_1'}{A_2}{A\ \vee\ A'}}}
	\quad(\textsc{$\vee$I$_1$})
	\and
\infer
	{
	\proves{M}{\onlyThrough{A_1}{A_2}{A}} \\\\
	\proves{M}{\onlyThrough{A_1}{A_2'}{A'}}
	}
	{\proves{M}{\onlyThrough{A_1}{A_2\ \vee\ A_2'}{A\ \vee\ A'}}}
	\quad(\textsc{$\vee$I$_2$})
	\and
\infer
	{
	\proves{M}{\onlyThrough{A_1}{A'}{\prg{false}}} \\\\
	\proves{M}{\onlyThrough{A_1}{A_2}{A\ \vee\ A'}}
	}
	{\proves{M}{\onlyThrough{A_1}{A_2}{A}}}
	\quad(\textsc{$\vee$E$_1$})
	\and
\infer
	{
	\proves{M}{\onlyThrough{A'}{A_2}{\prg{false}}} \\\\
	\proves{M}{\onlyThrough{A_1}{A_2}{A\ \vee\ A'}}
	}
	{\proves{M}{\onlyThrough{A_1}{A_2}{A}}}
	\quad(\textsc{$\vee$E$_2$})
	\and
\infer
	{
	\proves{M}{\onlyThrough{A_1}{A_2}{A_3}} \\\\
	\proves{M}{\onlyThrough{A_1}{A_3}{A}}
	}
	{\proves{M}{\onlyThrough{A_1}{A_2}{A}}}
	\quad(\textsc{Trans$_1$})
	\and
\infer
	{
	\proves{M}{\onlyThrough{A_1}{A_2}{A_3}} \\\\
	\proves{M}{\onlyThrough{A_3}{A_2}{A}}
	}
	{\proves{M}{\onlyThrough{A_1}{A_2}{A}}}
	\quad(\textsc{Trans$_2$})
	\and
\infer
	{
	\proves{M}{\onlyIf{A_1}{A_2}{A}}
	}
	{\proves{M}{\onlyThrough{A_1}{A_2}{A}}}
	\quad(\textsc{If})
	\and
\infer
	{}
	{\proves{M}{\onlyThrough{A_1}{A_2}{A_2}}}
	\quad(\textsc{End})
	\and
\infer
	{
	\forall y,\; \proves{M}{\onlyThrough{([y / x]A_1)}{A_2}{A}}
	}
	{\proves{M}{\onlyThrough{\exists x. [A_1]}{A_2}{A}}}
	\quad(\textsc{$\exists_1$})
	\and
\infer
	{
	\forall y,\; \proves{M}{\onlyThrough{A_1}{([y / x]A_2)}{A}}
	}
	{\proves{M}{\onlyThrough{A_1}{A_2}{A}}}
	\quad(\textsc{$\exists_2$})
\end{mathpar}
\caption{\emph{Only Through}}
\label{app:f:only-through}
\end{figure}
\begin{figure}[t]
\footnotesize
\begin{mathpar}
\infer
	{
	\proves{M}{A_1\ \longrightarrow\ A_1'}\\
	\proves{M}{A_2\ \longrightarrow\ A_2'}\\
	\proves{M}{A_3'\ \longrightarrow\ A_3}\\
	\proves{M}{\onlyIf{A_1'}{A_2'}{A_3'}}
	}
	{\proves{M}{\onlyIf{A_1}{A_2}{A_3}}}
	\quad(\textsc{If-$\longrightarrow$})
	\and
\infer
	{
	\proves{M}{\onlyIf{A_1}{A_2}{A}} \\\\
	\proves{M}{\onlyIf{A_1'}{A_2}{A'}}
	}
	{\proves{M}{\onlyIf{A_1\ \vee\ A_1'}{A_2}{A\ \vee\ A'}}}
	\quad(\textsc{If-$\vee$I$_1$})
	\and
\infer
	{
	\proves{M}{\onlyIf{A_1}{A_2}{A}} \\\\
	\proves{M}{\onlyIf{A_1}{A_2'}{A'}}
	}
	{\proves{M}{\onlyIf{A_1}{A_2\ \vee\ A_2'}{A\ \vee\ A'}}}
	\quad(\textsc{If-$\vee$I$_2$})
	\and
\infer
	{
	\proves{M}{\onlyIf{A_1}{A_2}{A\ \vee\ A'}} \\\\
	\proves{M}{\onlyThrough{A'}{A_2}{\prg{false}}}
	}
	{\proves{M}{\onlyIf{A_1}{A_2}{A}}}
	\quad(\textsc{If-$\vee$E})
	\and
\infer
	{
	\proves{M}{\onlyIf{A_1}{A_2}{A}} \\\\
	\proves{M}{\onlyThrough{A_1}{A_2}{A'}}
	}
	{\proves{M}{\onlyIf{A_1}{A_2}{A\ \wedge\ A'}}}
	\quad(\textsc{If-$\wedge$I})
	\and
\infer
	{
	\proves{M}{\onlyThrough{A_1}{A_2}{A_3}} \\\\
	\proves{M}{\onlyIf{A_1}{A_3}{A}}
	}
	{\proves{M}{\onlyIf{A_1}{A_2}{A}}}
	\quad(\textsc{If-Trans)}
	\and
\infer
	{}
	{\proves{M}{\onlyIf{A_1}{A_2}{A_1}}}
	\quad(\textsc{If-Start})
	\and
\infer
	{
	\forall y,\; \proves{M}{\onlyIf{([y / x]A_1)}{A_2}{A}}
	}
	{\proves{M}{\onlyIf{\exists x. [A_1]}{A_2}{A}}}
	\quad(\textsc{If-$\exists_1$})
	\and
\infer
	{
	\forall y,\; \proves{M}{\onlyIf{A_1}{([y / x]A_2)}{A}}
	}
	{\proves{M}{\onlyIf{A_1}{A_2}{A}}}
	\quad(\textsc{If-$\exists_2$})
\end{mathpar}
\caption{\emph{Only If}}
\label{app:f:only-if}
\end{figure}


\end{document}
