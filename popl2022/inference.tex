\section{Proving Necessity}
\label{s:inference}

In this Section we provide an inference system for constructing 
proofs of the \Nec specifications defined in Section \ref{s:holistic-guarantees}.
%%The inference system for proving a \NecessitySpecification consists of several steps:
%Proving a specification requires several steps, from the following four
%categories:
%\begin{description} 
%\item
%Proving Assertion Encapsulation (\S \ref{s:encaps-proof})
%\item
%Proving \Nec specifications from classical specifications for \jm[]{a single} internal method (\S \ref{s:classical-proof})
%\item
%Proving module-wide \jm[]{Single-Step} \Nec specifications by combining per-method \Nec specifications (\S \ref{s:module-proof})
%\item
%Raising necessary conditions to construct proofs of emergent behaviour (\S \ref{s:emergent-proof})
%\end{description}
\susan[I have commented out what was there]{Proving a specification requires four steps:
\begin{enumerate} 
\item
Proving Assertion Encapsulation (\S \ref{s:encaps-proof})
\item
Proving \Nec specifications from classical specifications for \jm[]{a single} internal method (\S \ref{s:classical-proof})
\item
Proving module-wide \jm[]{Single-Step} \Nec specifications by combining per-method \Nec specifications (\S \ref{s:module-proof})
\item
Raising necessary conditions to construct proofs of emergent behaviour (\S \ref{s:emergent-proof})
\end{enumerate}
}

\subsection {Assertion Encapsulation}
\label{s:encaps-proof}

\jm[]{In Section \ref{s:outline} we used proof of assertion encapsulation
when showing adherence to \Nec specifications. We now
define the semantics of assertion encapsulation and discuss the requirements 
for a proof system for it. We do not define a 
proof system as encapsulation is not the central focus of this paper, and \Nec logic 
is parametric to such a system. We do however informally discuss a rudimentary 
encapsulation proof system, which we use to prove a practical example adheres to its \Nec specification.
}


\subsubsection{Assertion Encapsulation Semantics}

In Section \ref{s:outline} we said that $A$ is encapsulated by  $M$ if it cannot be invalidated unless an
internal method was called. 
Here we refine this concept, to allow for ``conditional'' encapsulation:
$M\ \vDash A\ \Rightarrow\ \encaps{A'}$ expresses that in states which satisfy $A$, the assertion 
$A'$ cannot be invalidated, unless a method from $M$ was called.

\begin{definition}[Assertion Encapsulation]
\label{def:encapsulation}
\sd{An assertion $A'$ is \emph{encapsulated} by module $M$ and assertion $A$, written as}\ \  $M\ \vDash A\ \Rightarrow\ \encaps{A'}$, \ \ if and only if
for all external modules $M'$, and all states $\sigma$, $\sigma'$
such that $\arising{M}{M'}{\sigma}$:

\begin{tabular}{lr}
$\;\;\;\;$- $\reduction{M'}{M}{\sigma}{\sigma'}$  & \rdelim\}{3}{4mm}[$\;\;\;\Rightarrow\;\;\;$  $\exists x,\ \overline{z}. (\ \satisfiesA{M}{\sigma}{\calls{\_}{x}{m}{\overline{z}} \wedge\ \internal{x}}\ )$] \\
$\;\;\;\;$- $\satisfiesA{M}{\sigma}{A \wedge  A'}$ \\
$\;\;\;\;$- $\satisfiesA{M}{\sigma' \triangleleft \sigma}{\neg A'}$   
\end{tabular} 
\end{definition}

%\susan[I found this confusing. If you omit the highlighted sentence it is easier to understand.]{We use ${\sigma' \triangleleft \sigma}$ for the state as in $\sigma'$ but with the variable bindings as in the top frame from $\sigma$.} 
We will define 
${\sigma' \triangleleft \sigma}$ in  Definition \ref{d:adapt}, but 
we first look at some examples.
In both \prg{Mod2} and \prg{Mod3} the \prg{balance} of an account cannot change
unless an internal method was called:
\\
\strut \hspace{1cm}
$\prg{Mod2}\ \vDash \prg{a}:\prg{Account}\ \Rightarrow\ \encaps{\prg{a.balance}=\prg{bal}}$
\\
\strut \hspace{1cm}
$\prg{Mod3}\ \vDash \prg{a}:\prg{Account}\ \Rightarrow\ \encaps{\prg{a.balance}=\prg{bal}}$

%\susan[Does this help the story move on? I would omit it]{
%Note that encapsulation of an assertion does not imply encapsulation of its negation; 
%for example $\wrapped{o}$ is encapsulated, but $\neg  \wrapped{o}$ is not.}

\subsubsection{Proving Assertion Encapsulation}

%As we have already stated at the beginning of this section,
%encapsulation is a deep topic that is well studied in the literature, 
%and is not the focus of this paper. For now, we simply assume the existence 
%of a proof system for encapsulation as it is secondary to the central topic 
%of this paper. We need only assert that such an algorithmic proof system 
%must be sound (Definition \ref{lem:encap-soundness}).
\susan[I commented out what was there as I thought it was repetious]
{We are assuming the existence of a proof system for encapsulation and only need to assert that such an algorithmic proof system nust be sound.}
The construction of the algorithmic system is not central to our work,
because, as we shall see in later sections, our logic 
does not rely on the specifics of an encapsulation algorithm, only its soundness.

\begin{definition}[Encapsulation Soundness]
\label{lem:encap-soundness}
A judgment of the form $\proves{M}{\givenA{A}{\encaps{A}}}$  is\  \emph{sound}, \ if 
for all modules $M$, and assertions $A_1$ and $A_2$, if 
$\proves{M}{\givenA{A_1}{\encaps{A_2}}}$ then $\satisfies{M}{\givenA{A_1}{\encaps{A_2}}}$.
\end{definition}



% For the purposes of the examples presented later in the paper, 
\jm[]{For the purposes of proving examples, we define a rudimentary system for proving encapsulation in Appendix \ref{s:encap-proof}, Figure \ref{f:asrt-encap}.}
\jm[]{This encapsulation system is sound but not complete,
and is based, in part, on the footprint of expressions, and 
the fact that the evaluation of expressions containing no external objects
cannot change from state to state, except as a result of internal computation.
The encapsulation system also includes several rules for proving 
the encapsulation of assertions containing \susan[do you want permission?]{
access}, 
most importantly 
that $\wrapped{x}$ is always encapsulated. Thus, if $x$ is $\wrapped{}$, 
then internal computation is required to make $x$ not $\wrapped{}$. The full description of this system is given in
Appendix \ref{s:encap-proof}.} 
%\jm[]{We use that proof system for the proofs in Section \ref{s:examples}, but, as
%we said, the exact nature of that system is of little importance to this work. }
\susan[I commented out the last sentence as I don't think it provides new information.]{}


\subsubsection{Types}
\label{types}

\sophiaPonder[modified]{To allow for an easy way to judge encapsulation of
assertions, we assume a very simple type system, where field, method arguments
and method results are annotated with classes, and the type system checks 
that field assignments, method calls, and method returns adhere to these expectations.
Because the type system is so simple, we do not include its specification in the paper.
Note however, that the type system} has one further implication: modules are typed 
in isolation, thereby implicitly prohibiting
method calls from internal objects to external objects. 

\sophiaPonder[]{Based on this type system, we define a predicate $\intrnl{e}$, in Appendix \ref{s:encap-proof},
which asserts that any object reads during the evaluation of $e$ are internal.
Thus, any assertion that only involves $\intrnl{\_}$ expressions is encapsulated, more in Appendix \ref{s:encap-proof}.}

\sophiaPonder[]{Finally, a further small addition to the type system 
assists the knowledge that an object is \inside: Classes may
be annotated as \enclosed. A \enclosed object  
cannot be accessed by external objects; that is, it is always \inside. 
The type system needs to ensure that objects of \enclosed type
are never returned from method bodies, this is even simpler than in \cite{confined}. 
\susan[I would omit this]{Again, we omit the detailed description of this
simple type system.
}}

\subsection{Per-Method \Nec Specifications}
\label{s:classical-proof}
In this section we detail how we use classical specifications
to construct per-method \Nec specifications. 
That is, for some method $m$ in class $C$, we construct a specifications of the form:
$$\onlyIfSingle{A_1\ \wedge\ x : C\ \wedge\ \calls{\_}{x}{m}{\ldots}}{A_2}{A}$$
Thus, $A$ is a necessary precondition to reaching $A_2$ from $A_1$ via a method call $m$ to an object of class $C$.
%In order to do this
Note that if a precondition and a certain statement is \emph{sufficient}
to achieve a particular result, then the negation of that precondition
is \emph{necessary} to achieve the negation of the result after executing that statment.
Specifically, 
using classical Hoare logic, if $\hoare{P}{s}{Q}$ is true, then 
it follows that $\neg P$ is a \emph{necessary precondition} for $\neg Q$ to 
hold following the execution of $\prg{s}$.

We do not define a new assertion language and Hoare logic.
 Rather, we rely on prior work on such Hoare logics,
  and assume some underlying logic that can be used 
to prove  \emph{classical assertions}. 
Classical assertions are a subset of \SpecO, comprising only those 
assertions that are commonly present in other specification languages.
We provide this subset in Definition \ref{f:classical-syntax}. That is, classical assertions
are restricted to expressions, class assertions, the usual connectives, negation, 
implication, and the usual quantifiers.


\begin{definition}
%\begin{figure}[tbp]
% \footnotesize
Classical assertions, $P$, $Q$, are defined as follows 

\[
\begin{syntax}
\syntaxElement{P, Q} {} % {Classical Assertion}
		{
		\syntaxline
				{e}
				{e : C}
				{P\ \wedge\ P}
				{P\ \vee\ P}
				{P\ \longrightarrow\ P}
				{\neg P}
				{\forall x.[P]}
				{\exists x.[P]}
		\endsyntaxline
		}
\endSyntaxElement\\
\end{syntax}
\]
% \caption{Classical Assertion Syntax}
\label{f:classical-syntax}
% \end{figure}
\end{definition}


We assume that there exists some classical specification
inference system  that allows us to prove 
specifications of the form  $M\ \vdash\ \hoare{P}{\prg{s}}{Q}$. 
This implies that we can also have guarantees of  
$$M\ \vdash\ \hoare{P}{\prg{res} = x.m(\overline{z})}{Q}$$
That is,   % if we assume we are able to prove for some method $m$
% defined in module $M$,
 the execution of $x.m(\overline{z})$ 
with the precondition $P$ results in a program state that 
satisfies postcondition $Q$, where the returned value is represented
by \prg{res} in $Q$.

Figure \ref{f:classical->singlestep} introduces  %the first of the proof 
proof rules to infer per-method \Nec specifications.
These are rules whose conclusion have the  form Single-Step Only If. 

\begin{figure}[t]
\footnotesize
\begin{mathpar}
\infer
	{
	\proves{M}{\hoare
						{x : C \ \wedge\ P_1\ \wedge\ \neg P}
						{\prg{res} = x.m(\overline{z})}
						{\neg P_2}}
	}
	{
	\proves{M}{\onlyIfSingle
			{P_1\ \wedge\ x : C \wedge\ \calls{\_}{x}{m}{\overline{z}}}
			{P_2}
			{P}}
	}
	\quad(\textsc{If1-Classical})
	\and
\infer
	{
	\proves{M}{\hoare
						{x : C \ \wedge\ \neg P}
						{\prg{res} = x.m(\overline{z})}
						{\prg{res} \neq y}}
	}
	{
	\proves{M}{\onlyIfSingle{\wrapped{y}\ \wedge\ x : C \wedge\ \calls{\_}{x}{m}{\overline{z}}}{\neg \wrapped{y}}{P}}
	}
	\quad(\textsc{If1-Inside})
\end{mathpar}
\caption{Per-Method \Nec specifications}
\label{f:classical->singlestep}
\end{figure}

 

 \textsc{If1-Classical} states that  
if any state which satisfies $P_1$ and $\neg P$ and executes  the method $m$ on an obejct of class $C$, leads to a state that satisfies $\neg P_2$, then, any state which satisfies $P_1$ and calls $m$ on an object of class $C$ will lead to a state that satisfies $P_2$ only if the original state also satisfied $P$.
We can explain this also as follows: If the triple $.. \vdash \{R_1 \wedge R2\}\ s\ \{Q\}$ holds, then any state that satisfies $R_1$ and which upon execution of \prg{s} leads to a state that satisfies $\neg Q$, cannot satisfy $R_2$ -- because  if it did, then the ensuing state would have to satisfy $Q$,
 


 
\textsc{If1-Inside} states that a method which does not return an object $y$ 
preserves the ``insidedness'' of $y$. 
This rule is sound, as we do not support calls from internal to external code (see Section \ref{types})
-- in further work we plan to weaken this requirement, and will  
strengthen this rule.
In more detail,  \textsc{If1-Inside}   states that if $P$ is
a necessary precondition for returning an object $y$, then  
since we do not support calls from internal code to external code,
it follows that $P$ is a necessary precondition to leak $y$.
\textsc{If1-Inside}  is essentially  a specialized version of \textsc{If1-Classical}
for the $\wrapped{}$ predicate. Since $\wrapped{}$ is not a classical
assertion, we cannot use Hoare logic to reason about necessary conditions
for invalidating $\wrapped{}$.
 
 
 

\subsection{Per-Step \Nec Specifications}
\label{s:module-proof}

\begin{figure}[thb]
\footnotesize
\begin{mathpar}
\infer
	{
%	\textit{dom}(M) = \{C_1, \ldots, C_n\}\\
	\textit{for all}\ \ C \in dom(M)\ \ \textit{and}\ \  m \in M(C).\prg{mths}, \ \ \ \
				\proves{M}{\onlyIfSingle
								{A_1\ \wedge\ x : C\ \wedge\ \calls{\_}{x}{m}{\overline{z}}}
								{A_2}
								{A_3}}\\
	\proves{M}{A_1\ \longrightarrow\ \neg A_2}\\
	\proves{M}{\givenA{A_1}{\encaps{A_2}}}
	}
	{
	M\ \vdash\ \onlyIfSingle{A_1}{A_2}{A_3}
	}
	\quad(\textsc{If1-Internal})
%\end{mathpar}
%\caption{Combining per-method necessary conditions to achieve module-wide necessary conditions.}
%\label{f:singlestep->module}
%\end{figure}
%
%
%\begin{figure}[t]
%\footnotesize
%\begin{mathpar}

\infer
	{\proves{M}{\onlyIf{A_1}{A_2}{A}}}
	{\proves{M}{\onlyIfSingle{A_1}{A_2}{A}}}
	\quad(\textsc{If1-If})
	\and
\infer
	{
	\proves{M}{A_1 \longrightarrow A_1'}\\
	\proves{M}{A_2 \longrightarrow A_2'}\\
	\proves{M}{A_3' \longrightarrow A_3}\\
	\proves{M}{\onlyIfSingle{A_1'}{A_2'}{A_3'}}
	}
	{\proves{M}{\onlyIfSingle{A_1}{A_2}{A_3}}}
	\quad(\textsc{If1-$\longrightarrow$})
	\and
\infer
	{
	\proves{M}{\onlyIfSingle{A_1}{A_2}{A}} \\
	\proves{M}{\onlyIfSingle{A_1'}{A_2}{A'}}
	}
	{\proves{M}{\onlyIfSingle{A_1\ \vee\ A_1'}{A_2}{A\ \vee\ A'}}}
	\quad(\textsc{If1-$\vee$I$_1$})
	\and
\infer
	{
	\proves{M}{\onlyIfSingle{A_1}{A_2}{A}} \\
	\proves{M}{\onlyIfSingle{A_1}{A_2'}{A'}}
	}
	{\proves{M}{\onlyIfSingle{A_1}{A_2\ \vee\ A_2'}{A\ \vee\ A'}}}
	\quad(\textsc{If1-$\vee$I$_2$})
	\and
\infer
	{
	\proves{M}{\onlyIfSingle{A_1}{A_2}{A\ \vee\ A'}} \\
	\proves{M}{\onlyThrough{A'}{A_2}{\prg{false}}}
	}
	{\proves{M}{\onlyIfSingle{A_1}{A_2}{A}}}
	\quad(\textsc{If1-$\vee$E})
	\and
\infer
	{
	\proves{M}{\onlyIfSingle{A_1}{A_2}{A}} \\\\
	\proves{M}{\onlyIfSingle{A_1}{A_2}{A'}}
	}
	{\proves{M}{\onlyIf{A_1}{A_2}{A\ \wedge\ A'}}}
	\quad(\textsc{If1-$\wedge$I})
	\and
\infer
	{
	\forall y,\; \proves{M}{\onlyIfSingle{([y / x]A_1)}{A_2}{A}}
	}
	{\proves{M}{\onlyIfSingle{\exists x. [A_1]}{A_2}{A}}}
	\quad(\textsc{If1-$\exists_1$})
	\and
\infer
	{
	\forall y,\; \proves{M}{\onlyIfSingle{A_1}{([y / x]A_2)}{A}}
	}
	{\proves{M}{\onlyIfSingle{A_1}{\exists x. [A_2]}{A}}}
	\quad(\textsc{If1-$\exists_2$})
\end{mathpar}
\caption{Single-Step \Nec Specifications}
\label{f:only-if-single}
\end{figure}

We now raise per-method \Nec specifications 
to per-step \Nec specifications. 
The rules appear in Figure \ref{f:only-if-single}.

\textsc{If1-Internal} 
 lifts a per-method \Nec specification to a per-step \Nec specification.
Any \Nec specification which is satisfied for any method
calls sent to any object in a module, is satisfied for \emph{any step}, even
an external step, provided that the effect involved, \ie going from $A_1$ states to
$A_2$ states, is encapsulated.

 The remaining rules are more standard, and are reminiscent of the Hoare logic rule of consequence.
 We have five such rules:
 
The  rule for implication (\textsc{If1-$\longrightarrow$}) may strengthen
 properties of either the starting or ending state, or 
weaken the necessary precondition. 

There are two disjunction introduction rules: 
(a) \textsc{If1-$\vee$I1} states that any execution
starting from a state satisfying some disjunction that reaches some future state, 
must pass through either a necessary 
intermediate state for the first branch, or a necessary 
intermediate state for the second branch.
(b) \textsc{If1-$\vee$I2} states that any execution 
starting from some state and ending in a state satisfying a disjunction
must pass through either a necessary intermediate state for 
the first branch, or a necessary intermediate state for the second branch.


The disjunction
elimination rule (\textsc{IF1-$\vee$E}), 
is of note, as it mirrors typical disjunction elimination
rules, with a variation stating that if it is not possible  to reach 
the end state from one branch of the disjunction, then we can eliminate 
that branch. 

Two rules support existential elimination on the left hand side.
\textsc{If1-$\exists_1$} states that if any single step of execution starting
from a state satisfying $[y/x]A_1$ for all possible $y$, reaching some state satisfying
$A_2$ has $A$ as a necessary precondition, it follows that any single step execution
starting in a state where such a $y$ exists, and ending in a state satisfying $A_2$,
must have $A$ as a necessary precondition. \textsc{If1-$\exists_2$} is a similar rule
for an existential in the state resulting from the execution.

%Note that given the rule for implication, there is \jm[removed: no]{a} need 
%for conjunction introduction (\textsc{IF1-$\wedge$I}), but a rule 
%for conjunction elimination is derivable from the rule for implication. 

\begin{figure}[t]
\footnotesize
\begin{mathpar}
\infer
	{\proves{M}{\onlyIfSingle{A}{\neg A}{A'}}}
	{
	\proves{M}{\onlyThrough{A}{\neg A}{A'}}
	}
	\quad(\textsc{Changes})
	\and
\infer
	{
	\proves{M}{\onlyThrough{A_1}{A_2}{A_3}} \\\\
	\proves{M}{\onlyThrough{A_1}{A_3}{A}}
	}
	{\proves{M}{\onlyThrough{A_1}{A_2}{A}}}
	\quad(\textsc{Trans$_1$})
	\and
\infer
	{
	\proves{M}{\onlyThrough{A_1}{A_2}{A_3}} \\\\
	\proves{M}{\onlyThrough{A_3}{A_2}{A}}
	}
	{\proves{M}{\onlyThrough{A_1}{A_2}{A}}}
	\quad(\textsc{Trans$_2$})
	\and
\infer
	{
	\proves{M}{\onlyIf{A_1}{A_2}{A}}
	}
	{\proves{M}{\onlyThrough{A_1}{A_2}{A}}}
	\quad(\textsc{If})
	\and
\infer
	{}
	{\proves{M}{\onlyThrough{A_1}{A_2}{A_2}}}
	\quad(\textsc{End})
\end{mathpar}
\caption{\scd{Selected rules for} \emph{Only Through} -- rest in Figure \ref{app:f:only-through}}
\label{f:only-through}
\end{figure}
\begin{figure}[t]
\footnotesize
\begin{mathpar}
\infer
	{
	\proves{M}{\onlyThrough{A_1}{A_2}{A_3}} \\\\
	\proves{M}{\onlyIf{A_1}{A_3}{A}}
	}
	{\proves{M}{\onlyIf{A_1}{A_2}{A}}}
	\quad(\textsc{If-Trans)}
	\and
\infer
	{}
	{\proves{M}{\onlyIf{A_1}{A_2}{A_1}}}
	\quad(\textsc{If-Start})
\end{mathpar}
\caption{\scd{Selected rules for} \emph{Only If} -- the rest in Figure \ref{app:f:only-if}}
\label{f:only-if}
\end{figure}





%\subsection{Single-Step Necessary Conditions}
%\label{s:singleStep-proof}
%
\subsection{Emergent \Nec Specifications}
\label{s:emergent-proof}

\jm[]{We now show how per-step \Nec specifications are raised to 
multiple step \Nec specifications, allowing the specification
of emergent behavior. Figure \ref{f:only-through} present some of the rules for the 
construction of proofs for \emph{Only Through}, while Figure \ref{f:only-if}
provides some of the rules for the construction of proofs of \emph{Only If}. 
The full rules can be found in Appendix \ref{a:necSpec}, and are not presented here 
in full so as not to repeat rules from Figure \ref{f:only-if-single}.}

%The rules for both of these relations are fairly similar to each other, 
%and to those of the single step \Nec specification from 
%section \ref{s:module-proof}. 
%Both relations include 
%rules for implication along with disjunction introduction and elimination.
%%
%While Fig. \ref{f:only-if} includes a rule for conjunction introduction (\textsc{If-$\wedge$I}),
%such a rule is not possible for \emph{only through}, as unlike \emph{only if}, where
%the necessary condition must hold, specifically, in the starting state, 
%there is no such specific moment in time in which the necessary condition 
%for \emph{only through} must hold. 
%Both relations also include rules for existentials on the left hand side ($\exists_1$, $\exists_2$, \textsc{If-$\exists_1$}, and \textsc{If-$\exists_2$}) \jm[]{(Appendix \ref{a:necSpec})}.
%These rules follow the corresponding \textsc{If1-$\exists_1$} and \textsc{If1-$\exists_2$}.

\jm[]{\emph{Only Through} has several notable rules. \textsc{Changes}, in Figure \ref{f:only-through}, 
states that if the satisfaction of some assertion changes over time, 
then there must be some specific intermediate state where that change occurred.
 \textsc{Changes} is an important rule in the logic, and \susan[]{is an enabler for}
 % in allowing for 
 proofs of 
emergent properties. It is this rule that ultimately connects program 
execution to encapsulated properties. }

It may seem natural that \textsc{Changes} should take the more
general form:
$$\infer{\proves{M}{\onlyIfSingle{A_1}{A_2}{A_3}}}{\proves{M}{\onlyThrough{A_1}{A_2}{A_3}}}$$
This would not be sound as a \jm[used to say: ``transition from
one state to another'']{transition from a state satisfying one assertion to one satisfying another assertion}  is not required to occur in a single step;
however this is true for a change in satisfaction for a specific assertion (\ie $A$ to $\neg A$).


\emph{Only Through} also includes two transitivity rules (\textsc{Trans}$_1$ and \textsc{Trans}$_2$)
that say that necessary conditions to reach intermediate states or 
proceed from intermediate states are themselves necessary intermediate states. 

\jm[]{Finally, \emph{Only Through} includes \textsc{End}, stating that the ending condition is 
a necessary intermediate condition.}

\susan[I capitalised all the operators]{}
Moreover, any \emph{Only If} specification entails the corresponding
 \emph{Only Through} specification (\textsc{If}).
\emph{Only If} also includes a transitivity rule (\textsc{If-Trans}), but 
since the necessary condition must be true in the beginning state,
there is only a single rule. Finally, any starting condition is
itself a necessary precondition (\textsc{If-Start}). 



\subsection{Soundness of the \Nec Logic}

\label{s:soundness}

\begin{theorem}[Soundness]
\label{thm:soundness}
\sophiaPonder[renamed H to S]{}
Assuming a sound \SpecO proof system, $\proves{M}{A}$, and
a sound encapsulation inference system, $\proves{M}{\givenA{A}{\encaps{A'}}}$,
 and  and that on top of these systems we built
 the \Nec logic according to the rules in Figures \ref{f:classical->singlestep},  and \ref{f:only-if-single}, and \ref{f:only-if},  and \ref{f:only-through},   then, for    all modules $M$, and all \Nec specifications  $S$:
 
 $$\proves{M}{S}\ \ \ \ \ \ \ \mbox{implies}\ \ \ \ \ \  \ \ \ \satisfies{M}{S}$$
\end{theorem}

\begin{proof}
by induction on the derivation of $\proves{M}{S}$.
\end{proof}
\jm[]{The proof of soundness (Theorem. \ref{thm:soundness}) proves
that our proof system for \Nec adheres to the semantics of \Nec specifications.
We make two assumptions for soundness: (1) a sound proof system for assertion encapsulation, 
and (2) a sound proof system for \SpecO. It is notable that \Nec is parametric with both 
of these judgments.}

We have mechanized the proof of Theorem \ref{thm:soundness} in Coq, and it 
can be found in the associated artifact. 
The   Coq formalism deviates slightly from the system as
presented here,  mostly in the \scd{formalization} of the 
\SpecO language. The Coq version of \SpecO restricts variable usage to expressions, and allows only addresses to 
be used as part of non-expression syntax. 
\sophiaPonder[Julian: check!]{For example, in the Coq formalism
we can write assertions like $x.f==\prg{this}$ and
$x==\alpha_y$ and  $\access{\alpha_x}{\alpha_y}$, but we cannot write assertions 
like $\access{x}{y}$, where $x$ and $y$ are variables, and $\alpha_x$ and $\alpha_y$ are
addresses. }
\jm[checked! it all looks right]{}
The reason for this \scd{restriction in the Coq formalism} is to avoid spending % sizable 
\scd{significant} effort encoding variable
renaming and substitution, a well-known difficulty for languages such as Coq. 
% This is justifiable, as we are still 
\sophiaPonder[]{This restriction does not affect the expressiveness of %assertions in the
our  Coq formalism: we are
able} to express assertions such as $\access{x}{y}$, by using addresses and introducing equality \scd{expressions} % as part of expressions 
to connect variables to address, \ie
 $\access{\alpha_x}{\alpha_y} \wedge \alpha_x == x \wedge \alpha_y == y$.



