\section{Proving Necessity}
\label{s:inference}

In this Section we provide an inference system for constructing 
proofs of the \Nec specifications defined in Section \ref{s:holistic-guarantees}.
%The inference system for proving a \NecessitySpecification consists of several steps:
Proving a specification requires several steps, from the following four
categories:
\begin{description} 
\item
\jm[]{Part 1:} Proving Assertion Encapsulation (Section \ref{s:encaps-proof})
\item
\jm[]{Part 2:} Proving \Nec specifications from classical specifications for \jm[]{a single} internal method (Section \ref{s:classical-proof})
\item
\jm[]{Part 3:} Proving module-wide \jm[]{Single-Step} \Nec specifications by combining per-method \Nec specifications (Section \ref{s:module-proof})
\item
\jm[]{Part 4:} Raising necessary conditions to construct proofs of emergent behaviour (Section \ref{s:emergent-proof})
\end{description}


\begin{definition}
%\begin{figure}[tbp]
% \footnotesize
Classical assertions, $P$, $Q$, are defined as follows 

\[
\begin{syntax}
\syntaxElement{P, Q} {} % {Classical Assertion}
		{
		\syntaxline
				{e}
				{e : C}
				{P\ \wedge\ P}
				{P\ \vee\ P}
				{P\ \longrightarrow\ P}
				{\neg P}
				{\forall x.[P]}
				{\exists x.[P]}
		\endsyntaxline
		}
\endSyntaxElement\\
\end{syntax}
\]
% \caption{Classical Assertion Syntax}
\label{f:classical-syntax}
% \end{figure}
\end{definition}

\subsection{\jm[]{Part 1:} Assertion Encapsulation}
\label{s:encaps-proof}

As already stated in Section \ref{s:outline}, the first step in proving adherence to a \Nec specification
is to prove that an assertion is encapsulated, \jm[]{that is, an assertion may only be invalidated by 
a method call to the internal module.}
We assume the existence of  
an proof system for constructing proofs of assertion encapsulation, 
written $\proves{M}{\givenA{A_1}{\encaps{A_2}}}$.
Such an algorithmic system should be sound, in the sense  defined below:

\begin{definition}[Encapsulation Soundness]
\label{lem:encap-soundness}
A judgment of the form $\proves{M}{\givenA{A}{\encaps{A}}}$  is\  \emph{sound}, \ if 
for all modules $M$, and assertions $A_1$ and $A_2$, if 
$\proves{M}{\givenA{A_1}{\encaps{A_2}}}$ then $\satisfies{M}{\givenA{A_1}{\encaps{A_2}}}$.
\end{definition}


The construction of such an algorithmic system is not central to our work,
because, as we shall see in later sections, our logic 
does not rely on the specifics of an encapsulation algorithm, only its soundness.

% For the purposes of the examples presented later in the paper, 
\jm[]{We define such a system for proving encapsulation in Appendix \ref{s:encap-proof}
on top of a simple type system.}
\jm[]{This rudimentary encapsulation system employs annotations on classes to restrict
how objects may be returned by internal methods. Internal classes
may be annotated as \enclosed, and any path that starts at an internal object and
only navigates through \enclosed fields is encapsulated.} 
%\jm[]{We use that proof system for the proofs in Section \ref{s:examples}, but, as
%we said, the exact nature of that system is of little importance to this work. }
\susan[I commented out the last sentence as I don't think it provides new information.]{}

\subsection{\jm[]{Part 2:} Per-Method \Nec Specifications from Classical Specifications}
\label{s:classical-proof}
In this section we detail how we use classical specifications
to construct per-method \Nec specifications. 
That is, for some method $m$ in class $C$, we construct a specifications of the form:
$$\onlyIfSingle{A_1\ \wedge\ x : C\ \wedge\ \calls{\_}{x}{m}{\ldots}}{A_2}{A}$$
Thus, $A$ is a necessary precondition to reaching $A_2$ from $A_1$ via a method call $m$ to an object of class $C$.
In order to do this
note that if a precondition and a certain statement is \emph{sufficient}
to achieve a particular result, then the negation of that precondition
is \emph{necessary} to achieve the negation of the result after executing that statment.
Specifically, 
using classical Hoare logic, if $\hoare{P}{s}{Q}$ is true, then 
it follows that $\neg P$ is a \emph{necessary precondition} for $\neg Q$ to 
hold following the execution of $\prg{s}$.

We do not define a new assertion language and Hoare logic.
 Rather, we rely on prior work on such Hoare logics,
  and assume some underlying logic that can be used 
to prove  \emph{classical assertions}. 
Classical assertions are a subset of \SpecO, comprising only those 
assertions that are commonly present in other specification languages.
We provide this subset in Fig. \ref{f:classical-syntax}. That is, classical assertions
are restricted to expressions, class assertions, the usual connectives, negation, 
implication, and the usual quantifiers.


We assume that there exists some classical specification
inference system  that allows us to prove 
specifications of the form  $M\ \vdash\ \hoare{P}{\prg{s}}{Q}$. 
This implies that we can also have guarantees of  
$$M\ \vdash\ \hoare{P}{\prg{res} = x.m(\overline{z})}{Q}$$
That is,   % if we assume we are able to prove for some method $m$
% defined in module $M$,
 the execution of $x.m(\overline{z})$ 
with the pre-condition $P$ results in a program state that 
satisfies post-condition $Q$, where the returned value is represented
by \prg{res} in $Q$.

Fig. \ref{f:classical->singlestep} introduces  %the first of the proof 
proof rules to infer per-method \Nec specifications.
These are rules whose conclusion have the  form Single-Step Only If. 

\begin{figure}[t]
\footnotesize
\begin{mathpar}
\infer
	{
	\proves{M}{\hoare
						{x : C \ \wedge\ P_1\ \wedge\ \neg P}
						{\prg{res} = x.m(\overline{z})}
						{\neg P_2}}
	}
	{
	\proves{M}{\onlyIfSingle
			{P_1\ \wedge\ x : C \wedge\ \calls{\_}{x}{m}{\overline{z}}}
			{P_2}
			{P}}
	}
	\quad(\textsc{If1-Classical})
	\and
\infer
	{
	\proves{M}{\hoare
						{x : C \ \wedge\ \neg P}
						{\prg{res} = x.m(\overline{z})}
						{\prg{res} \neq y}}
	}
	{
	\proves{M}{\onlyIfSingle{\wrapped{y}\ \wedge\ x : C \wedge\ \calls{\_}{x}{m}{\overline{z}}}{\neg \wrapped{y}}{P}}
	}
	\quad(\textsc{If1-Wrapped})
\end{mathpar}
\caption{Per-Method \Nec specifications}
\label{f:classical->singlestep}
\end{figure}

 

 \textsc{If1-Classical} states that  
if any state which satisfies $P_1$ and $\neg P$ and executes  the method $m$ on an obejct of class $C$, leads to a state that satisfies $\neg P_2$, then, any state which satisfies $P_1$ and calls $m$ on an object of class $C$ will lead to a state that satisfies $P_2$ only if the original state also satisfied $P$.
We can explain this also as follows: If the triple $.. \vdash \{R_1 \wedge R2\}\ s\ \{Q\}$ holds, then any state that satisfies $R_1$ and which upon execution of \prg{s} leads to a state that satisfies $\neg Q$, cannot satisfy $R_2$ -- because  if it did, then the ensuing state would have to satisfy $Q$,
 


 
\textsc{If1-Wrapped} essentially states that a method which does not return an object $y$ 
preserves the ``wrappedness'' of that object $y$. 
This rule is sound, 
because we do not support calls from internal code to external code
-- in further work we plan to weaken this requirement, and will  
strengthen this rule.
In more detail,  \textsc{If1-Wrapped}   states that if $P$ is
a necessary precondition for returning an object $y$, then  
since we do not support calls from internal code to external code,
it follows that $P$ is a necessary precondition to leak $y$.
\textsc{If1-Wrapped}  is essentially  a specialized version of \textsc{If1-Classical}
for the $\wrapped{}$ predicate. Since $\wrapped{}$ is not a classical
assertion, we cannot use Hoare logic to reason about necessary conditions
for leaking of data.
 
 
 

\subsection{\jm[]{Part 3:} Per-Step \Nec Specifications}
\label{s:module-proof}

\begin{figure}[thb]
\footnotesize
\begin{mathpar}
\infer
	{
%	\textit{dom}(M) = \{C_1, \ldots, C_n\}\\
	\forall C \in dom(M).\forall m \in M(C).\prg{mths}, \ \ \
				\proves{M}{\onlyIfSingle
								{A_1\ \wedge\ x : C\ \wedge\ \calls{\_}{x}{m}{\overline{z}}}
								{A_2}
								{A_3}}\\
	\proves{M}{A_1\ \longrightarrow\ \neg A_2}\\
	\proves{M}{\givenA{A_1}{\encaps{A_2}}}
	}
	{
	M\ \vdash\ \onlyIfSingle{A_1}{A_2}{A_3}
	}
	\quad(\textsc{If1-Internal})
%\end{mathpar}
%\caption{Combining per-method necessary conditions to achieve module-wide necessary conditions.}
%\label{f:singlestep->module}
%\end{figure}
%
%
%\begin{figure}[t]
%\footnotesize
%\begin{mathpar}

\infer
	{\proves{M}{\onlyIf{A_1}{A_2}{A}}}
	{\proves{M}{\onlyIfSingle{A_1}{A_2}{A}}}
	\quad(\textsc{If1-If})
	\and
\infer
	{
	\proves{M}{A_1 \longrightarrow A_1'}\\
	\proves{M}{A_2 \longrightarrow A_2'}\\
	\proves{M}{A_3' \longrightarrow A_3}\\
	\proves{M}{\onlyIfSingle{A_1'}{A_2'}{A_3'}}
	}
	{\proves{M}{\onlyIfSingle{A_1}{A_2}{A_3}}}
	\quad(\textsc{If1-$\longrightarrow$})
	\and
\infer
	{
	\proves{M}{\onlyIfSingle{A_1}{A_2}{A}} \\
	\proves{M}{\onlyIfSingle{A_1'}{A_2}{A'}}
	}
	{\proves{M}{\onlyIfSingle{A_1\ \vee\ A_1'}{A_2}{A\ \vee\ A'}}}
	\quad(\textsc{If1-$\vee$I$_1$})
	\and
\infer
	{
	\proves{M}{\onlyIfSingle{A_1}{A_2}{A}} \\
	\proves{M}{\onlyIfSingle{A_1}{A_2'}{A'}}
	}
	{\proves{M}{\onlyIfSingle{A_1}{A_2\ \vee\ A_2'}{A\ \vee\ A'}}}
	\quad(\textsc{If1-$\vee$I$_2$})
	\and
\infer
	{
	\proves{M}{\onlyIfSingle{A_1}{A_2}{A\ \vee\ A'}} \\
	\proves{M}{\onlyThrough{A'}{A_2}{\prg{false}}}
	}
	{\proves{M}{\onlyIfSingle{A_1}{A_2}{A}}}
	\quad(\textsc{If1-$\vee$E})
	\and
\infer
	{
	\proves{M}{\onlyIfSingle{A_1}{A_2}{A}} \\\\
	\proves{M}{\onlyIfSingle{A_1}{A_2}{A'}}
	}
	{\proves{M}{\onlyIf{A_1}{A_2}{A\ \wedge\ A'}}}
	\quad(\textsc{If1-$\wedge$I})
	\and
\infer
	{
	\forall y,\; \proves{M}{\onlyIfSingle{([y / x]A_1)}{A_2}{A}}
	}
	{\proves{M}{\onlyIfSingle{\exists x. [A_1]}{A_2}{A}}}
	\quad(\textsc{If1-$\exists_1$})
	\and
\infer
	{
	\forall y,\; \proves{M}{\onlyIfSingle{A_1}{([y / x]A_2)}{A}}
	}
	{\proves{M}{\onlyIfSingle{A_1}{\exists x. [A_2]}{A}}}
	\quad(\textsc{If1-$\exists_2$})
\end{mathpar}
\caption{Single-Step \Nec Specifications}
\label{f:only-if-single}
\end{figure}

We now raise per-method \Nec specifications 
to per-step \Nec specifications. 
The rules appear in Fig. \ref{f:only-if-single}.

\textsc{If1-Internal} is central. It
 lifts a per-method \Nec specification to a per-step \Nec specification.
Essentially, any \Nec specification which is satisfied for any method
calls sent to any object in a module, is satisfied for \emph{any step}, even
an external step, provided that the effect involved, i.e. going from $A_1$ states to
$A_2$ states, is encapsulated.

 The remaining rules are more standard, and are reminiscent of the Hoare logic rule of consequence.
 We have five such rules:
 
The  rule for implication (\textsc{If1-$\longrightarrow$}) may strengthen
 properties of either the starting or ending state, or 
weaken the necessary pre-condition. 

There are two disjunction introduction rules: 
(a) \textsc{If1-$\vee$I1} states that any execution
starting from a state satisfying some disjunction and reaches some future state, 
must pass through the either a necessary 
intermediate state for the first branch, or a necessary 
intermediate state for the second branch.
(b) \textsc{If1-$\vee$I2} states that any execution 
starting from some state and ending in a state satisfying a disjunction
must pass through either a necessary intermediate state for 
the first branch, or a necessary intermediate state for the second branch.


The disjunction
elimination rule (\textsc{IF1-$\vee$E}), 
is of note, as it mirrors typical disjunction elimination
rules, with a variation stating that if it is not possible  to reach 
the end state from one branch of the disjunction, then we can eliminate 
that branch. 

There are two rules for existential elimination on the left hand side.
\textsc{If1-$\exists_1$} states that if any single step of execution starting
from a state satisfying $[y/x]A_1$ for all possible $y$, reaching some state satisfying
$A_2$ has $A$ as a necessary precondition, it follows that any single step execution
starting in a state where such a $y$ exists, and ending in a state satisfying $A_2$,
must have $A$ as a necessary precondition. \textsc{If1-$\exists_2$} is a similar rule
for an existential in the state resulting from the execution.

%Note that given the rule for implication, there is \jm[removed: no]{a} need 
%for conjunction introduction (\textsc{IF1-$\wedge$I}), but a rule 
%for conjunction elimination is derivable from the rule for implication. 

\begin{figure}[t]
\footnotesize
\begin{mathpar}
\infer
	{\proves{M}{\onlyIfSingle{A}{\neg A}{A'}}}
	{
	\proves{M}{\onlyThrough{A}{\neg A}{A'}}
	}
	\quad(\textsc{Changes})
	\and
\infer
	{
	\proves{M}{\onlyThrough{A_1}{A_2}{A_3}} \\\\
	\proves{M}{\onlyThrough{A_1}{A_3}{A}}
	}
	{\proves{M}{\onlyThrough{A_1}{A_2}{A}}}
	\quad(\textsc{Trans$_1$})
	\and
\infer
	{
	\proves{M}{\onlyThrough{A_1}{A_2}{A_3}} \\\\
	\proves{M}{\onlyThrough{A_3}{A_2}{A}}
	}
	{\proves{M}{\onlyThrough{A_1}{A_2}{A}}}
	\quad(\textsc{Trans$_2$})
	\and
\infer
	{
	\proves{M}{\onlyIf{A_1}{A_2}{A}}
	}
	{\proves{M}{\onlyThrough{A_1}{A_2}{A}}}
	\quad(\textsc{If})
	\and
\infer
	{}
	{\proves{M}{\onlyThrough{A_1}{A_2}{A_2}}}
	\quad(\textsc{End})
\end{mathpar}
\caption{\scd{Selected rules for} \emph{Only Through} -- rest in Fig. \ref{app:f:only-through}}
\label{f:only-through}
\end{figure}
\begin{figure}[t]
\footnotesize
\begin{mathpar}
\infer
	{
	\proves{M}{\onlyThrough{A_1}{A_2}{A_3}} \\\\
	\proves{M}{\onlyIf{A_1}{A_3}{A}}
	}
	{\proves{M}{\onlyIf{A_1}{A_2}{A}}}
	\quad(\textsc{If-Trans)}
	\and
\infer
	{}
	{\proves{M}{\onlyIf{A_1}{A_2}{A_1}}}
	\quad(\textsc{If-Start})
\end{mathpar}
\caption{\scd{Selected rules for} \emph{Only If} -- the rest in Fig. \ref{app:f:only-if}}
\label{f:only-if}
\end{figure}





%\subsection{Single-Step Necessary Conditions}
%\label{s:singleStep-proof}
%
\subsection{\jm[]{Part 4:} Emergent \Nec Specifications}
\label{s:emergent-proof}

\jm[]{The final step is to raise per-step \Nec specifications to 
multiple step \Nec specifications, allowing the specification
of emergent behavior. Fig. \ref{f:only-through} present some of the rules for the 
construction of proofs for \emph{Only Through}, while Fig. \ref{f:only-if}
provides some of the rules for the construction of proofs of \emph{Only If}. 
The full rules can be found in Appendix \ref{a:necSpec}, and are not presented here 
in full due to repetition from those rule in Fig. \ref{f:only-if-single}.}

%The rules for both of these relations are fairly similar to each other, 
%and to those of the single step \Nec specification from 
%section \ref{s:module-proof}. 
%Both relations include 
%rules for implication along with disjunction introduction and elimination.
%%
%While Fig. \ref{f:only-if} includes a rule for conjunction introduction (\textsc{If-$\wedge$I}),
%such a rule is not possible for \emph{only through}, as unlike \emph{only if}, where
%the necessary condition must hold, specifically, in the starting state, 
%there is no such specific moment in time in which the necessary condition 
%for \emph{only through} must hold. 
%Both relations also include rules for existentials on the left hand side ($\exists_1$, $\exists_2$, \textsc{If-$\exists_1$}, and \textsc{If-$\exists_2$}) \jm[]{(Appendix \ref{a:necSpec})}.
%These rules follow the corresponding \textsc{If1-$\exists_1$} and \textsc{If1-$\exists_2$}.

\jm[]{\emph{Only Through} has several notable rules. \textsc{Changes}, in Fig. \ref{f:only-through}, 
states that if the satisfaction of some assertion changes over time, 
then there must be some specific intermediate state where that change occurred.
 \textsc{Changes} is an important rule in the logic, and in allowing for proofs of 
emergent properties. It is this rule that ultimately connects program 
execution to encapsulated properties. }

It may seem natural that \textsc{Changes} should take the more
general form:
$$\infer{\proves{M}{\onlyIfSingle{A_1}{A_2}{A_3}}}{\proves{M}{\onlyThrough{A_1}{A_2}{A_3}}}$$
This however would not be sound as in general a transition from
one state to another is not required to occur in a single step,
however this is true for a change in satisfaction for a specific assertion (i.e. $A$ to $\neg A$).


\emph{Only Through} also includes two transitivity rules (\textsc{Trans}$_1$ and \textsc{Trans}$_2$)
that say that necessary conditions to reach intermediate states or 
proceed from intermediate states are themselves necessary intermediate states. 

\jm[]{Finally, \emph{Only Through} includes \textsc{End}, stating that the ending condition is 
a necessary intermediate condition.}

Moreover, any \emph{only if} specification entails the corresponding
 \emph{only through} specification (\textsc{If}).
\emph{Only if} also includes a transitivity rule (\textsc{If-Trans}), but 
since the necessary condition must be true in the beginning state,
there is only a single rule. Finally, any starting condition is
itself a necessary precondition (\textsc{If-Start}). 



\subsection{Soundness of the \Nec Logic}

\label{s:soundness}

\begin{theorem}[Soundness]
\label{thm:soundness}
\sophiaPonder[renamed H to S]{}
Assuming a sound \SpecO proof system, $\proves{M}{A}$, and
a sound encapsulation inference system, $\proves{M}{\givenA{A}{\encaps{A'}}}$,
 and  and that on top of these systems we built
 the \Nec logic according to the rules in Figures  \ref{f:classical->singlestep},  and \ref{f:only-if-single}, and \ref{f:only-if},  and \ref{f:only-through},   then, for    all modules $M$, and all \Nec specifications  $S$:
 
 $$\proves{M}{S}\ \ \ \ \ \ \ \mbox{implies}\ \ \ \ \ \  \ \ \ \satisfies{M}{S}$$
\end{theorem}

\begin{proof}
by induction on the derivation of $\proves{M}{S}$.
\end{proof}
We have mechanized the proof of Theorem \ref{thm:soundness} in Coq. This 
proof can be found in the associated artifact. 
The   Coq formalism deviates slightly from the system as
presented here,  mostly in the \scd{formalization} of the 
\SpecO language. The Coq version of \SpecO restricts variable usage to expressions, and allows only addresses to 
be used as part of non-expression syntax. 
\sophiaPonder[Julian: check!]{For example, in the Coq formalism
we can write assertions like $x.f==\prg{this}$ and
$x==\alpha_y$ and  $\access{\alpha_x}{\alpha_y}$, but we cannot write assertions 
like $\access{x}{y}$, where $x$ and $y$ are variables, and $\alpha_x$ and $\alpha_y$ are
addresses. }
The reason for this \scd{restriction in the Coq formalism} is to avoid spending % sizable 
\scd{significant} effort encoding variable
renaming and substitution, a well-known difficulty for languages such as Coq. 
% This is justifiable, as we are still 
\sophiaPonder[]{This restriction does not affect the expressiveness of %assertions in the
our  Coq formalism: we are
able} to express assertions such as $\access{x}{y}$, by using addresses and introducing equality \scd{expressions} % as part of expressions 
to connect variables to address, i.e.
 $\access{\alpha_x}{\alpha_y} \wedge \alpha_x == x \wedge \alpha_y == y$.



